\documentclass{article}
\usepackage[utf8]{inputenc}
\usepackage{amsmath}
\usepackage{amssymb}
\usepackage{parskip}
\usepackage{dsfont}
\usepackage{fullpage}

\title{Taylor Series}
\author{Paolo Bettelini}
\date{}

\begin{document}

\maketitle
\tableofcontents
\pagebreak

\section{Definition}
The Taylor series of \(f(x)\) around the point \(a\) is defined as
\begin{align*}
    \sum_{n=0}^{\infty}\frac{(x-a)^nf^{(n)}(a)}{n!}
\end{align*}

\pagebreak

\section{Divergence}

\pagebreak

\section{Euler's formula}

Euler's formula states that for every \(x\in\mathbb{R}\)
\begin{align*}
    e^{ix}=cos\,x+i\,sin\,x
\end{align*}

To understand this identity we must first look at the Taylor series of some functions.

\subsection{Sine function}
\subsection{Cosine function}
\subsection{Exponential function}

\pagebreak

\subsection{Proof}

Given the Taylor series for the exponential function
\begin{align*}
    e^x=\sum_{n=0}^{\infty}\frac{x^n}{n!}
\end{align*}
we plug in \(ix\) instead of \(x\):
\begin{align*}
    e^{ix}=\sum_{n=0}^{\infty}\frac{(xi)^n}{n!}
\end{align*}
The imaginary number \(i\) has some amazing property when it comes to exponentiation.
\begin{align*}
    \begin{cases}
        i^0=+1\\
        i^1=+i\\
        i^2=-1\\
        i^3=-i\\
    \end{cases}
    \quad
    \begin{cases}
        i^4=+1\\
        i^5=+i\\
        i^6=-1\\
        i^7=-i\\
    \end{cases}
    \quad
    \cdots
\end{align*}
We can use these properties to simplify the \(e^{ix}\) Taylor series
\begin{align*}
    e^{ix}
    &   =1
        +ix
        +\frac{(ix)^2}{2!}
        +\frac{(ix)^3}{3!}
        +\frac{(ix)^4}{4!}
        +\frac{(ix)^5}{5!}
        +\frac{(ix)^6}{6!}
        +\frac{(ix)^7}{7!}
        +\cdots
    \\
    \\
    &   =1
        +ix
        -\frac{x^2}{2!}
        -\frac{ix^3}{3!}
        +\frac{x^4}{4!}
        +\frac{ix^5}{5!}
        -\frac{x^6}{6!}
        -\frac{ix^7}{7!}
        +\cdots
    \\
    \\
    &=
    \left(
        1
        -\frac{x^2}{2!}
        +\frac{x^4}{4!}
        -\frac{x^6}{6!}
        +\cdots
    \right)
    +i
    \left(
        x
        -\frac{x^3}{3!}
        +\frac{x^5}{5!}
        -\frac{x^7}{7!}
        +\cdots
    \right)
\end{align*}
We notice that the two terms correspond to the sine and cosine Taylor series
\begin{align*}
    e^{ix}=cos\,x+i\,sin\,x
\end{align*}

\end{document}