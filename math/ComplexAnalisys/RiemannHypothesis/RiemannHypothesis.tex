\documentclass{article}
\usepackage[utf8]{inputenc}
\usepackage{amsmath}
\usepackage{parskip}
\usepackage{dsfont}
\usepackage{fullpage}

\title{Riemann Hypothesis}
\author{Paolo Bettelini}
\date{}

\newcommand{\exceptone}{
    \,\backslash\,\{1\}
}

\begin{document}

\maketitle
\tableofcontents
\pagebreak

\section{Zeta function}

\subsection{Definition}
The zeta function is defined as
\begin{align*}
    \zeta(s)=\sum_{n=1}^{\infty}\frac{1}{n^s},\quad Re(s)>0
\end{align*}

\subsection{Euler product}
The zeta function can be represented as an Euler product.
\\
We will start by using the first prime number: 2, and multiply both sides by $2^{-s}$.
\begin{align*}
    \zeta(s)\frac{1}{2^s}=\sum_{n=1}^{\infty}\frac{1}{(2n)^s}
\end{align*}
We then subtract the second definition from the first one, such that
\begin{align*}
    \zeta(s)-\zeta(s)\frac{1}{2^s}&=
    \sum_{n=1}^{\infty}\left[\frac{1}{n^s}\right]-
    \sum_{n=1}^{\infty}\left[\frac{1}{(2n)^s}\right]
    \\
    \zeta(s)\left(1-\frac{1}{2^s}\right)&=
    \sum_{n=1}^{\infty}\frac{1}{n^s},
    \quad n\neq 2k,k\in \mathds{Z}
\end{align*}
Here we are excluding the multiples of 2 from the series.

If we do the same with the next prime number, which is 3, we get
\begin{align*}
    \zeta(s)\left(1-\frac{1}{2^s}\right)\left(1-\frac{1}{3^s}\right)&=
    \sum_{n=1}^{\infty}\frac{1}{n^s},
    \quad n\neq 2k,n\neq 3k, k\in \mathds{Z}
\end{align*}

We can repeat this process with every prime number.
\\
Eventually, we will exclude every nth-term to sum as we use every prime number, except for n=1.
\begin{align*}
    \zeta(s)\prod_{p\in P}^{\infty}1-\frac{1}{p^s}=\frac{1}{1^s}=1
\end{align*}
Finally, we get the identity
\begin{align*}
    \zeta(s)=
    \prod_{p\in P}^{\infty}\frac{1}{1-p^{-s}}
\end{align*}

\pagebreak

\section{Analytic continuation}

\subsection{Zeta function for positive Re(s)}

We have seen that the classical zeta function definition only converges for $Re(s)>1$
\begin{align*}
    \zeta(s)=\sum_{n=1}^{\infty}\frac{1}{n^s},
    \quad Re(s)>1
\end{align*}

We can use the eta function $\eta(s)$, which is defined for $Re(s)>0\exceptone$, to analytically extend the zeta function domain to $Re(s)>0\exceptone$.
\\
The eta function is a Dirichlet series defined as
\begin{align*}
    \eta(s)=\sum_{n=1}^{\infty}\frac{(-1)^{n-1}}{n^s},
    \quad Re(s)>0\exceptone
\end{align*}

We start by splitting the zeta function into two distinct series, one for $n$ even and the other one for $n$ odd.
\\
The index for the even series will be $2n$, while the odd one will use $2n-1$ as the index.
\begin{align*}
    \zeta(s)=
    \sum_{n=1}^{\infty}\left[\frac{1}{(2n)^s}\right]+
    \sum_{n=1}^{\infty}\left[\frac{1}{(2n-1)^s}\right]
\end{align*}
We do the same thing with the eta function.
\\
Notice that $(-1)^n$ is 1 when $n$ is even and -1 when $n$ is odd.
\begin{align*}
    \eta(s)=
    \sum_{n=1}^{\infty}\left[\frac{1}{(2n)^s}\right]-
    \sum_{n=1}^{\infty}\left[\frac{1}{(2n-1)^s}\right]
\end{align*}
We subtract these two definition from eachother
\begin{align*}
    \zeta(s)-\eta(s)&=
    2\sum_{n=1}^{\infty}\frac{1}{(2n)^s}
    \\
    &=2^{1-s}\sum_{n=1}^{\infty}\frac{1}{k^s}
    \\
    \zeta(s)-\eta(s)&=2^{1-s}\zeta(s)
    \\
    \frac{1}{1-2^{1-s}}\eta(s)&=\zeta(s)
\end{align*}
We finally get
\begin{align*}
    \zeta(s)=\frac{1}{1-s^{1-s}}\sum_{n=1}^{\infty}\frac{(-1)^{n-1}}{n^s},
    \quad Re(s)>0\exceptone
\end{align*}

This series can be used to compute value of the zeta function along the critical strip $0<Re(s)<1$.

\pagebreak

\section{Prime-counting function}

\subsection{Properties of the prime-counting function}

The prime-counting function $\pi (x)$ is defined as the number of primes less or equals than $x$.

We can consider the difference between $\pi (x)$ of two consecutive integers
\begin{align*}
    \pi (x)-\pi (x-1)= 
    \begin{cases}
        1,& \text{if } x\in P
        \\
        0,& \text{otherwise}
    \end{cases}
\end{align*}

Given a series over all prime numbers, we can extend it to all integers and multiply each term by this difference.
\\
The terms whose index is not a prime number will be multiplied by 0.
\begin{align*}
    \sum_{p\in P}^{\infty}a_k=\sum_{n=2}^{\infty}\left[\pi (n) - \pi (n-1)\right]a_n
\end{align*}
Here we start at 2 since there are no prime numbers less than 2.

\subsection{Relationship with the zeta function}

We have seen that the zeta function can be written as an Euler Product

\begin{align*}
    \zeta (s)=\prod_{p\in P}^{\infty}\frac{1}{1-p^{-s}}
\end{align*}

However, we need convert this product into a series in order to apply the identity of the last paragraph.
\\
We can take the natural logarithm of both sides and use the multiplication property
\begin{align*}
    ln\left(\zeta (s)\right)&=ln\prod_{p\in P}^{\infty}\frac{1}{1-p^{-s}}
    \\
    &=\sum_{p\in P}^{\infty}ln\left(\frac{1}{1-p^{-s}}\right)
    \\
    &=\sum_{p\in P}^{\infty}-ln\left(1-p^{-s}\right)
\end{align*}
Now we can apply the identity
\begin{align*}
    ln\left(\zeta (s)\right)=\sum_{n=2}^{\infty}-ln\left(1-n^{-s}\right)\left[\pi (n) - \pi (n-1)\right]
\end{align*}

The next goal is to factor out $\pi (n)$
\begin{align*}
    ln\left(\zeta(s)\right)
    &=\sum_{n=2}^{\infty}\left[\pi (n-1)ln\left(1-n^{-s}\right)\right]
    -\sum_{n=2}^{\infty}\left[\pi (n)ln\left(1-n^{-s}\right)\right]
    \\&=\sum_{n=2}^{\infty}\left[\pi (n)ln\left(1-(n+1)^{-s}\right)\right]
    -\sum_{n=2}^{\infty}\left[\pi (n)ln\left(1-n^{-s}\right)\right]
    \\
    &=\sum_{n=2}^{\infty}\pi (n)\left[ln\left(1-(n+1)^{-s}\right)-ln\left(1-n^{-s}\right)\right]
\end{align*}
To simplify further more, we consider the derivative of the function $ln(1-x^{-s})$.
\\
Using the chain rule we get
\begin{align*}
    \frac{d}{dx}ln(1-x^{-s})=
    \frac{s}{x(x^s-1)}
\end{align*}
Therefore,
\begin{align*}
    ln(1-x^{-s})=
    \int \frac{s}{x(x^s-1)}\,dx+C
\end{align*}

Considering $f(x)=ln(1-x^{-s})$, our series can be expressed as
\begin{align*}
    ln\left(\zeta(s)\right)=
    \sum_{n=2}^{\infty}\pi(n)\left[f(n+1)-f(n)\right]
\end{align*}
which can be written as an integral from $n$ to $n+1$
\begin{align*}
    ln\left(\zeta(s)\right)&=
    \sum_{n=2}^{\infty}\pi(n)
    \int\limits_n^{n+1} f'(x)\,dx
    \\
    &=
    \sum_{n=2}^{\infty}\pi(n)
    \int\limits_n^{n+1}
    \frac{s}{x(x^s-1)}\,dx
    \\
    &=
    \sum_{n=2}^{\infty}
    \int\limits_n^{n+1}
    \frac{s\pi(x)}{x(x^s-1)}\,dx
\end{align*}
Instead of taking the sum of each of these integrals (2 to 3, 3 to 4, ...), we can make a single integral
\begin{align*}
    ln\left(\zeta(s)\right)=
    s\int\limits_2^\infty
    \frac{\pi(x)}{x(x^2-1)}\,dx
\end{align*}

\end{document}