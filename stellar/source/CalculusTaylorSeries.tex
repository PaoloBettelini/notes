\documentclass[preview]{standalone}

\usepackage{amsmath}
\usepackage{amssymb}
\usepackage{parskip}
\usepackage{fullpage}
\usepackage{hyperref}
\usepackage{stellar}

\hypersetup{
    colorlinks=true,
    linkcolor=black,
    urlcolor=blue,
    pdftitle={Taylor Series},
    pdfpagemode=FullScreen,
}

\begin{document}

\title{Taylor Series}
\id{taylor-series}
\genpage

\section{Definition}

A function \(f(x)\) can be approximated around a point \(a\) by a power series.
\\
The polynomial must be centered around \(a\), so the variable will be \(x-a\) such as
\begin{align*}
    \sum_{n=0}^{\infty}c_n{(x-a)}^n
    =c_0+c_1(x-a)+c_2{(x-a)}^2+c_3{(x-a)}^3+\cdots
\end{align*}
Our goal is to find the coefficients \(c_n\).
\\
We notice that \(f(a)=c_0\), which is the only coefficient that does not multiply a variable.
\\
If we take the derivative, the coefficient \(c_1\) will lose its variable
\begin{align*}
    c_1+2c_2(x-a)+3c_3{(x-a)}^2+\cdots
\end{align*}
Now we have \(f'(a)=c_1\).
\\
We take the derivative of the polynomial again
\begin{align*}
    2c_2+6c_3(x-a)+\cdots
\end{align*}
This time we have
\begin{align*}
    f''(a)&=2c_2\\
    c_2&=\frac{f''(a)}{2}
\end{align*}
And again
\begin{align*}
    f'''(a)&=6c_3\\
    c_3&=\frac{f'''(a)}{6}
\end{align*}

More generally, to extract the coefficient \(c_n\) we take the n-th derivative of the function.
\\
By the power rule, we know that
\begin{align*}
    \frac{d^k}{dx^k}\left(x^k\right)=k!
\end{align*}
For example
\begin{align*}
    \frac{d^4}{dx^4}\left(x^4\right)
    &=\frac{d^3}{dx^3}\left(4x^3\right)\\
    &=\frac{d^2}{dx^2}\left(4\cdot 3x^2\right)\\
    &=\frac{d}{dx}\left(4\cdot 3\cdot 2x\right)\\
    &=4\cdot 3\cdot 2\cdot 1\\
    &=4!
\end{align*}

This brings us to
\begin{align*}
    c_n=\frac{f^{(n)}(a)}{n!}
\end{align*}
\\
The Taylor series of \(f(x)\) around the point \(a\) is defined as
\begin{align*}
    \sum_{n=0}^{\infty}\frac{{(x-a)}^n f^{(n)}(a)}{n!}
\end{align*}

\pagebreak

\section{Divergence}

\end{document}