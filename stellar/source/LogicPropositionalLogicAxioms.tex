\documentclass[preview]{standalone}

\usepackage{amsmath}
\usepackage{amssymb}
\usepackage{stellar}

\hypersetup{
    colorlinks=true,
    linkcolor=black,
    urlcolor=blue,
    pdftitle={Logic},
    pdfpagemode=FullScreen,
}

\begin{document}

\title{Logical connectives}
\id{propositional-logic-axioms}
\genpage

\section{Axioms of propositional logic}

\begin{snippet}{logic-propositional-logic-axioms-introduction}
There are many ways to define propositional logic.
It is usual to define it with \snippetref[logic-modus-ponens][modus ponens] as its inference rule
and the following axioms using \(\lnot\) and \(\implies\):
\end{snippet}

\includesnpt{logic-propositional-logic-axioms}

\begin{snippet}{logic-propositional-logic-axioms-expl}
The axiom (1) states that if \(A\) is true, then no matter what \(B\) is, it will always be true.
\\
The axiom (2) states that if \(A\) implies that \(B\) implies \(C\), then if \(A\) implies \(B\),
\(A\) also implies \(C\).
\\
The axiom (3) states that if \(\lnot A\) implies \(\lnot B\), then \(A\) implies \(B\).
\end{snippet}

\end{document}
