\documentclass[preview]{standalone}

\usepackage{amsmath}
\usepackage{amssymb}
\usepackage{parskip}
\usepackage{fullpage}
\usepackage{hyperref}
\usepackage{bettelini}
\usepackage{stellar}

\hypersetup{
    colorlinks=true,
    linkcolor=black,
    urlcolor=blue,
    pdftitle={Measure Theory},
    pdfpagemode=FullScreen,
}

\begin{document}

\title{Measure Theory}
\id{measuretheory}
\genpage

\plain{<br>}

\begin{snippetdefinition}{sigma-algebra-definition}{\(\sigma\)-algebra}
    Let \(X\) be a set.
    A \(\sigma\)-\textit{algebra} is a subset \(\Sigma \subseteq \mathcal{P}(X)\)
    under the following conditions:
    \begin{enumerate}
        \item it contains the universal set: \(X \in \Sigma\);
        \item it is closed under complements: \(\forall A \in \Sigma, X \,\backslash\, A \in \Sigma\);
        \item it is closed under countable unions: \({\{A_i\}}_{i \in I} \subseteq \Sigma \implies \bigcup_{i \in I} A_i \in \Sigma \).
    \end{enumerate}
\end{snippetdefinition}

\begin{snippetcorollary}{empty-set-is-in-sigma-algebra}{Empty set is in \(\sigma\)-algebra}
    % by the complement
    Let \(\Sigma\) be a \(\sigma\)-algebra. Then, \(\emptyset \in \Sigma\).
\end{snippetcorollary}

\begin{snippetcorollary}{sigma-algebra-closed-under-intersection}{\(\sigma\)-algebra is closed under intersection}
    % by De Morgan's Law
    Let \(\Sigma\) be a \(\sigma\)-algebra. Then, \(\Sigma\) is closed under intersections
    \[
        {\{A_i\}}_{i \in I} \subseteq \Sigma \implies \bigcap_{i \in I} A_i \in \Sigma
    \]
\end{snippetcorollary}

% the elements of a sigma algebra are called measurable sets


\end{document}
