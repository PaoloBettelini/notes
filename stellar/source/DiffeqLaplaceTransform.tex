\documentclass[preview]{standalone}

\usepackage{amsmath}
\usepackage{amssymb}
\usepackage{parskip}
\usepackage{fullpage}
\usepackage{hyperref}
\usepackage{stellar}
\usepackage{bettelini}

\hypersetup{
    colorlinks=true,
    linkcolor=black,
    urlcolor=blue,
    pdftitle={Differential Equations},
    pdfpagemode=FullScreen,
}

\begin{document}

\title{Differential Equations Laplace Transform}
\id{diffeq-laplace-transform}
\genpage

\section{Laplace Transform}

\begin{snippetdefinition}{diffeq-laplace-transform}{Laplace Transform}{
    Given a piecewise continuous function \(f(t)\), the Laplace transform
    is defined as
    \[
        \mathcal{L}\{f(t)\}= \integral[0][\infty][e^{-st}f(t)][t]=F(s)
    \]
}
\end{snippetdefinition}

\subsection{Properties of the Laplace Transform}

\begin{snippetcorollary}{diffeq-laplace-transform-contant-property}{Laplace Transform constant property}{
    It is easy to see that given \(f(t)\) and \(g(t)\)
    \[
        \mathcal{L}\{af(t)+bg(t)\} = a\mathcal{L}\{f(t)\} + b\mathcal{L}\{g(t)\}
    \]
    for any constants \(a\) and \(b\).
}
\end{snippetcorollary}

\subsection{Inverse Laplace Transform}

\begin{snippetdefinition}{diffeq-inverse-laplace-transform}{Inverse Laplace Transform}{
    The Inverse Laplace Transform is defined as
    \[
        {\mathcal{L}}^{-1} \{F(s)\}= f(t)
    \]
}
\end{snippetdefinition}

\subsection{Properties of the Inverse Laplace Transform}

\begin{snippetcorollary}{diffeq-inverse-laplace-transform-contant-property}{Inverse Laplace Transform constant property}{
    Given the Laplace transforms \(F(s)\) and \(G(s)\)
    \[
        {\mathcal{L}}^{-1} \{aF(s)+bG(s)\} =
        a{\mathcal{L}}^{-1}\{F(s)\} +
        b{\mathcal{L}}^{-1}\{G(s)\}
    \]
    for any constants \(a\) and \(b\).
}
\end{snippetcorollary}

\subsection{Heaviside function}

%%%%%%%%%%%%%%%%%%%%%%%%% FROM HERE

A consider a function in the form \(H(t-c)f(t-c)\) where \(H\) is the
Heaviside step function, meaning \(f(t)\) is shifted by \(c\) and is \(0\) for \(t < c\).

\begin{align*}
    \mathcal{L}\{H(t-c)f(t-c)\} &=
    \integral[0][\infty][e^{-st}H(t-c)f(t-c)][t] \\
    &= \integral[c][\infty][e^{-st}f(t-c)][t]
\end{align*}
Now substitue for \(u=t-c\)
\begin{align*}
    \integral[0][\infty][e^{-s(u+c)}f(u)][u]
    &= \integral[0][\infty][e^{-su}e^{-cs}f(u)][u] \\
    &= e^{-cs} \integral[0][\infty][e^{-su}f(u)][u]
\end{align*}

Concluding that

\[
    \mathcal{L}\{H(t-c)f(t-c)\} =
    e^{-cs}F(s)
    \quad
    \text{and}
    \quad
    {\mathcal{L}}^{-1}\{e^{-cs}F(s)\} =
    H(t-c)f(t-c)
\]

\subsection{Laplace Transform of derivatives}

\begin{snippettheorem}{diffeq-laplace-transform-of-derivatives}{Laplace Transform of derivatives}{
    Let \(f'\), \(f''\), \(\cdots\), \(f^{(n-1)}\) be continuous functions
    and \(f^{(n)}\) a piecewise continuous functions. Then,
    \[
        \mathcal{L}\{f^{(n)}\} =
        s^n \mathcal{L}\{f\} -
        \sum_{k=1}^n s^{n-k}f^{(k-1)}(0)
    \]
}
\end{snippettheorem}

\subsection{Solving Initial value problems}

\begin{snippet}{diffeq-solving-iv-problems}
Given an initial value problem differential equation, we can apply the Laplace
transform on both sides of the equation. After applying the initial conditions
(\(y(0), y'(0), \cdots\)), the differential equation is transformed into an algebraic equation.

Note that if the initial conditions are not expressed as  \(y(0), y'(0), \cdots\),
we need to first make a change of variable.

We can then isolate \(\mathcal{L}\{y\}\) in the equation and get
\[
    \mathcal{L}\{y\} = M
\]
Then apply the inverse Laplace transform to solve the equation
\[
    y={\mathcal{L}}^{-1}\{M\}
\]
\end{snippet}

% from https://tutorial.math.lamar.edu/Classes/DE/IVPWithNonConstantCoefficient.aspx


\end{document}