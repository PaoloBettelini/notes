\documentclass[preview]{standalone}

\usepackage{amsmath}
\usepackage{amssymb}
\usepackage{parskip}
\usepackage{fullpage}
\usepackage{hyperref}
\usepackage{stellar}
\usepackage{bussproofs}

\hypersetup{
    colorlinks=true,
    linkcolor=black,
    urlcolor=blue,
    pdftitle={Assets},
    pdfpagemode=FullScreen,
}

\begin{document}

\title{Logic Assets}

\begin{snippet}{logic-axiom-definition}{Axiom}
    An axiom, postulate, or assumption is a statement that is taken to be true, to serve as a premise or starting point for further reasoning and arguments.
\end{snippet}

\begin{snippetdefinition}{logic-consistent-axiomatic-system}{Consistency}{
    An axiomatic system is said to be \textit{consistent} if it lacks contradiction.
    That is, it is impossible to derive both a statement and its negation from the system's axioms.
}
\end{snippetdefinition}

\begin{snippetdefinition}{logic-independent-proposition}{Independency}{
    A proposition is said to be \textit{independent} with respect to an axiomatic system
    if it cannot be proven or disproven from the axiomatic system.
}
\end{snippetdefinition}

\begin{snippetdefinition}{logic-completness}{Completness}{
    An axiomatic system is said to be \textit{complete} if for every statement,
    either the statement or its negation can be derived from the system.
}
\end{snippetdefinition}

\begin{snippetdefinition}{logic-sufficiency}{Sufficiency}{
    Given two statements \(P\) and \(Q\) where \(P \implies Q\) (\(P\) \textit{implies} \(Q\)),
    \(P\) suffices for \(Q\).
}
\end{snippetdefinition}

\begin{snippetdefinition}{logic-necessity}{Necessity}{
    Given two statements \(P\) and \(Q\) where \(P \impliedby Q\) (\(P\) \textit{is implied by} \(Q\)),
    \(Q\) suffices for \(P\).
}
\end{snippetdefinition}

\begin{snippetdefinition}{logic-biconditional-logical-connective}{Biconditional logical connective}{
    Given two statements \(P\) and \(Q\) where \(P \iff Q\)
    (\(P\) \textit{if and only if} \(Q\), or for short \(P\) \textit{iff} \(Q\)),
    \(P\) suffices for \(Q\) and \(Q\) suffices for \(P\).
}
\end{snippetdefinition}

\begin{snippetdefinition}{logic-logical-negation}{Logical Negation}{
    Given a statement \(P\), \(\lnot P\) is true if \(P\) is false
    and is false if \(P\) is true.
}
\end{snippetdefinition}

\begin{snippetdefinition}{logic-logical-conjunction}{Logical Conjunction}{
    Given two statements \(P\) and \(Q\), \(P \land Q\) is true if both \(P\) and \(Q\) are true.
}
\end{snippetdefinition}

\begin{snippetdefinition}{logic-logical-disjunction}{Logical Disjunction}{
    Given two statements \(P\) and \(Q\), \(P \lor Q\) is true if at least one in \(P\) and \(Q\) is true.
}
\end{snippetdefinition}

\begin{snippetdefinition}{logic-propositional-formula}{Propositional formula}{
    A \textit{propositional formula} is a formula constructed from propositions or
    propositional variables which has a unique truth value given all variables.
}
\end{snippetdefinition}

\begin{snippetexample}{logic-propositional-variable-example}{Propositional Variable}{
    ``It is snowy today''
}
\end{snippetexample}

\begin{snippetexample}{logic-sufficiency-example}{Sufficiency}{
    ``John is a king'' \(\implies\) ``John is a male''
}
\end{snippetexample}

\begin{snippetexample}{logic-necessity-example}{Necessity}{
    ``John is a male'' \(\impliedby\) ``John is a king''
}
\end{snippetexample}

\begin{snippetdefinition}{logic-modus-ponens}{Modus Ponens}
    \begin{prooftree}
        \AxiomC{\(P\implies Q\)}
        \AxiomC{\(P\)}
        \BinaryInfC{\(Q\)}
    \end{prooftree}
\end{snippetdefinition}

\begin{snippetdefinition}{logic-modus-tollens}{Modus Tollens}
    \begin{prooftree}
        \AxiomC{\(P\implies Q\)}
        \AxiomC{\(\lnot Q\)}
        \BinaryInfC{\(\lnot P\)}
    \end{prooftree}
\end{snippetdefinition}

\begin{snippetdefinition}{logic-fallacy-of-affirming-the-consequent}{Fallacy of affirming the consequent}
    \begin{prooftree}
        \AxiomC{\(P\implies Q\)}
        \AxiomC{\(Q\)}
        \BinaryInfC{\(P\)}
    \end{prooftree}
\end{snippetdefinition}

\begin{snippetdefinition}{logic-fallacy-of-denying-the-antecedent}{Fallacy of denying the antecedent}
    \begin{prooftree}
        \AxiomC{\(P\implies Q\)}
        \AxiomC{\(\lnot P\)}
        \BinaryInfC{\(\lnot Q\)}
    \end{prooftree}
\end{snippetdefinition}

\begin{snippetdefinition}{logic-propositional-logic-axioms}{Propositional Logic Axioms}{
    \begin{enumerate}
        \item \(A \implies (B \implies A)\)
        \item \((A \implies (B \implies C)) \implies ((A \implies B) \implies (A \implies C))\)
        \item \((\lnot A \implies \lnot B) \implies (B \implies A)\)
    \end{enumerate}
}
\end{snippetdefinition}

\end{document}
