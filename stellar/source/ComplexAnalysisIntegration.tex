\documentclass[preview]{standalone}

\usepackage{amsmath}
\usepackage{amssymb}
\usepackage{parskip}
\usepackage{fullpage}
\usepackage{hyperref}
\usepackage{wrapfig}
\usepackage{bettelini}
\usepackage{makecell}
\usepackage{stellar}

\hypersetup{
    colorlinks=true,
    linkcolor=black,
    urlcolor=blue,
    pdftitle={ComplexAnalysis},
    pdfpagemode=FullScreen,
}

\begin{document}
  
\title{Complex Analysis}
\id{complexanalysis-integration}
\genpage

\section{Complex integration}

\subsection{Complex integrals}

\begin{snippet}{complex-integral}
Let \(f(t)\) be a complex-valued function of a real parameter \(t\). Then
we can decompose \(f\) into its real and imaginary parts
\[
    \integral[a][b][f(t)][t]=\integral[a][b][u(t)][t]+i\integral[a][b][v(t)][t]
\]
\end{snippet}

\subsection{Contour integrals}

\begin{snippet}{contour-integral}
\sdefinition{Contour Integral}{
    Let \(f(z)\) be a complex-valued function.
    When computing a definite integral we need a way to go from \(z_0\) to \(z_1\).
    \[
        \integral[z_0][z_1][f(z)][z]
    \]
    In order to compute this we need a continuous parametrised curve \(z: [t_0;t_1] \to \mathbb{C}\) such that
    \(z(t_0)=z_0\) and \(z(t_1)=z_1\).
}
\end{snippet}

\begin{snippet}{contour-integral-parametrizes-curve}
\stheorem{Parametrized contour integral}{
    Let \(f(z)\) be a complex-valued function and
    let \(\Gamma\) be a smooth curve from \(z_0\) to \(z_1\), then
    \[
        \int_\Gamma f(z)\,dz = \integral[t_0][t_1][f(z(t))z'(t)][t]
    \]
}
\end{snippet}

% https://courses.maths.ox.ac.uk/pluginfile.php/94006/mod_resource/content/4/complex.pdf
% https://www.maths.ed.ac.uk/~jmf/Teaching/MT3/ComplexAnalysis.pdf
% https://dzackgarza.com/rawnotes/Class_Notes/2020/Spring/Complex%20Analysis/ComplexAnalysis.html

\end{document}