\documentclass[preview]{standalone}

\usepackage{amsmath}
\usepackage{amssymb}
\usepackage{parskip}
\usepackage{fullpage}
\usepackage{hyperref}
\usepackage{tikz}
\usepackage{stellar}
\usepackage{bettelini}

\hypersetup{
    colorlinks=true,
    linkcolor=black,
    urlcolor=blue,
    pdftitle={Assets},
    pdfpagemode=FullScreen,
}

\begin{document}

\title{Series}
\id{series}
\genpage

\section{Divergence and convergence}

\begin{snippet}{infinite-series-convergence}
An infinite series converges if the limit
of its partial sum sequence also converges,
otherwise it diverges.
\end{snippet}

\section{Properties}

\begin{snippet}{infinite-series-properties}
\stheorem{}{
    \[
        \left(
            \sum_{n=0}^\infty a_n
        \right)
        \left(
            \sum_{n=0}^\infty b_n
        \right)
        =
        \sum_{n=0}^\infty \sum_{k=0}^n a_k b_{n-k}
    \]
}
\end{snippet}

\section{Covergence theorem}

\begin{snippet}{convergence-theorem}
\stheorem{Convergence Theorem}{
    If \(\sum a_n\) converges then \(\lim_{n\to\infty}a_n=0\)
}
\end{snippet}

\begin{snippet}{convergence-theorem-proof}
\sproof{Convergence Theorem}{
    Consider the partial sum
    \[
        s_n = \sum_{k=1}^{n}a_k
    \]
    The sequence \(a_n\) can now be expressed as
    \[
        a_n = s_n - s_{n-1}
    \]
    Since \(\sum a_n\) converges, \(\lim_{n\to\infty}s_n=L\) for \(L\) finite. \\
    The limit \(\lim_{n\to\infty}s_{n-1}=L\) because \(n-1 \to \infty \text{ as } n \to \infty\).
    This implies the following
    \[
        \lim_{n \to \infty} a_n
        = \lim_{n \to \infty} s_n - s_{n-1} = L - L = 0
    \]
}
\end{snippet}

\section{Divergence test}

\begin{snippet}{divergence-test}
\stheorem{Divergence test}{
    If \(\lim_{n \to \infty} a_n \neq 0\) then \(\sum a_n\) diverges.
}
\end{snippet}
% TODO proof

\section{Absolute and conditional convergence}

\begin{snippet}{absolute-convergence}
\sdefinition{Absolute convergence}{
    A series \(\sum a_n\) is said to converge absolutely if
    \(\sum |a_n|\) converges.
}
\end{snippet}

\begin{snippet}{series-expl1}
This is a stronger type of convergence. Every absolutely convergent series is also convergent.

A series that is convergent but not absolutely convergent is called conditionally convergent.
\end{snippet}{series-expl2}

\section{Riemann rearrangement theorem}

\begin{snippet}{riemann-rearrangement-theorem}
\stheorem{Riemann rearranged theorem}{
    If a series is conditionally convergent, then its terms can be rearranged such that
    the series converges to any \(r\in \mathbb{R}\) or such that it diverges (to infinity or no finite value).
    If the series is absolutely convergent then any rearrangement of its terms will converge to the same value.
}
\end{snippet}

\section{Geometric series}

\begin{snippet}{geometric-series-definition}
\sdefinition{Geometric Series}{
    A geometric series is a series of the form
    \[
        \sum_{n=0}^\infty r^n
    \]
    meaning that the ratio between two adject terms is constant.
}
\end{snippet}

% theorem?
\begin{snippet}{geometric-series-convergence}
This type of series converges for \(|r| < 1\) and always converges absolutely.
\[
    \sum_{n=0}^\infty r^n = \frac{1}{1-r}
\]
\end{snippet}

\section{Telescoping series}

\begin{snippet}{geometric-series-definition}
\sdefinition{Geometric Series}{
    A telescoping series is a series where the terms in the partial sums cancel eachother,
    leaving a finite number of terms.
}
\end{snippet}

\begin{snippet}{geometric-series-example-1}
\sexample{Geometric Series}{
    \begin{align*}
        &\sum_{n=0}^\infty \frac{1}{n^2 + 3n + 2}
        = \sum_{n=0}^\infty \left[ \frac{1}{n+1} - \frac{1}{n+2} \right]
        = \lim_{N \to \infty} \sum_{n=0}^N \left[ \frac{1}{n+1} - \frac{1}{n+2} \right] \\
        &= \frac{1}{1} - \frac{1}{2} + \frac{1}{2} - \frac{1}{3}
        + \frac{1}{3} - \frac{1}{4} + \cdots + \frac{1}{n} - \frac{1}{n+1} +
        \frac{1}{n+1} - \frac{1}{n+2} \\
        &= \lim_{N \to \infty} 1 - \frac{1}{n+2} = 1
    \end{align*}
}
\end{snippet}

\section{Harmonic series}

\begin{snippet}{harmonic-series-definition}
\sdefinition{Harmonic Series}{
    The harmonic series is the following digergent series
    \[
        \sum_{n=1}^\infty \frac{1}{n}
    \]
}
\end{snippet}

\section{Integral Test}

\begin{snippet}{integral-test}
\sdefinition{Integral Test}{
    Let \(f(x)\) be a continuous function on \([k;\infty)\)
    such that it is decreasing and positive on the interval \([N; \infty)\)
    for some \(N\).
    \[
        \integral[k][\infty][f(x)][x] \text{ converges } \implies
        \sum_{n=k}^{\infty} f(n) \text{ converges}
    \]
    and
    \[
        \integral[k][\infty][f(x)][x] \text{ diverges } \implies
        \sum_{n=k}^{\infty} f(n) \text{ diverges}
    \]
}
\end{snippet}

\begin{snippet}{integral-test-proof}
\sproof{Integral Test}{
    TODO
}
\end{snippet}

\section{p-series}

\begin{snippet}{p-series-definition}
\sdefinition{p-series Test}{
    If \(k > 0\) then \[\sum_{n=k}^\infty \frac{1}{n^p}\]
    converges if \(p > 1\) and diverges if \(p \leq 1\).
}
\end{snippet}

\end{document}