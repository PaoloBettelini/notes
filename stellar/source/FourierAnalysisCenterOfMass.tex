\documentclass[preview]{standalone}

\usepackage{amsmath}
\usepackage{amssymb}
\usepackage{parskip}
\usepackage{fullpage}
\usepackage{hyperref}
\usepackage{bettelini}
\usepackage{stellar}

\hypersetup{
    colorlinks=true,
    linkcolor=black,
    urlcolor=blue,
    pdftitle={Cyclic groups},
    pdfpagemode=FullScreen,
}

\begin{document}

\title{Fourier Analysis}
\id{fourieranalysis-center-of-mass}
\genpage

\section{Center of mass}

\begin{snippet}{fourier-analysis-center-of-mass-expl}
    The next step is to create a new function. Instead of plotting our signal
    as time increases with a given frequency \(\xi\), we plot all of the
    signal at once but the frequency changes over time. Here we are moving from
    a time-dependant function to a frequency-dependant function. The argument is
    no longer the time \(f(t)\) but rather the frequency with which we want to plot our signal
    around the origin \(f(\xi)\). Now, what is the center of mass? The center of mass is
    basically the average point of the function, which is a complex number.
    You might notice that the center of mass (the blue dot) in the animation changes
    as the frequency changes. To find its value for a certain frequency, we can sample
    a bunch of points from the function and then divide it by the number of samples.
    This approach works when we are dealing with discrete functions, such as the animation
    (the signal you draw is a set of points), but we would need an infinite number of samples
    when the signal is continuous. An infinite amount of precision is achievable using an integral.
    To find the center of mass when the frequency is \(\xi\) we integrate our complex function over
    a certain period of time, and then divide it by the time length.

    \[
        \frac{1}{t_1-t_0}
        \integral[t_0][t_1][f(t)e^{-2\pi i t\xi}][t]
    \]
\end{snippet}

\includesnpt{fourier-lib}
\includesnpt{fourier-center-of-mass}

\end{document}
