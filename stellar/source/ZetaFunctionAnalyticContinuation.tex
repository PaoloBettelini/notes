\documentclass[preview]{standalone}

\usepackage{amsmath}
\usepackage{amssymb}
\usepackage{parskip}
\usepackage{fullpage}
\usepackage{hyperref}
\usepackage{bettelini}
\usepackage{stellar}

\hypersetup{
    colorlinks=true,
    linkcolor=black,
    urlcolor=blue,
    pdftitle={Zeta Function},
    pdfpagemode=FullScreen,
}

\newcommand{\exceptone}{
    \,\backslash\,\{1\}
}

\begin{document}

\title{Zeta Function}
\id{zetafunction-analytic-continuation}
\genpage

\plain{<br>}

\section{Zeta function for positive Re (s)}

\begin{snippetdefinition}{eta-function-definition}{Eta function}
    The \textit{eta function} is a Dirichlet series defined as
    \[
        \eta(s)=\sum_{n=1}^{\infty}\frac{{(-1)}^{n-1}}{n^s},
        \quad \Re(s)>0\exceptone
    \]
\end{snippetdefinition}

\begin{snippettheorem}{zeta-function-extension-for-positives}{Zeta fuction for \(\Re(s) > 0 \exceptone\)}
    The eta function can be used to analytically extend the zeta function
    \[
        \zeta(s)=\frac{1}{1-2^{1-s}}\sum_{n=1}^{\infty}\frac{{(-1)}^{n-1}}{n^s},
        \quad Re(s)>0\exceptone
    \]
\end{snippettheorem}

\begin{snippetproof}{zeta-function-extension-for-positives-proof}{Zeta fuction for \(\Re(s) > 0 \exceptone\)}
    We start by splitting the zeta function into two distinct series, one for \(n\) even and the other one for \(n\) odd.
    \\
    The index for the even series will be \(2n\), while the odd one will use \(2n-1\) as the index.
    \begin{align*}
        \zeta(s)=
        \sum_{n=1}^{\infty}\left[\frac{1}{{(2n)}^s}\right]+
        \sum_{n=1}^{\infty}\left[\frac{1}{{(2n-1)}^s}\right]
    \end{align*}
    We do the same thing with the eta function.
    \\
    Notice that \({(-1)}^n\) is 1 when \(n\) is even and -1 when \(n\) is odd.
    \begin{align*}
        \eta(s)=
        \sum_{n=1}^{\infty}\left[\frac{1}{{(2n)}^s}\right]-
        \sum_{n=1}^{\infty}\left[\frac{1}{{(2n-1)}^s}\right]
    \end{align*}
    We subtract these two definition from eachother
    \begin{align*}
        \zeta(s)-\eta(s)&=
        2\sum_{n=1}^{\infty}\frac{1}{{(2n)}^s}
        \\
        &=2^{1-s}\sum_{n=1}^{\infty}\frac{1}{k^s}
        \\
        &=2^{1-s}\zeta(s)
        \\
        \frac{1}{1-2^{1-s}}\eta(s)&=\zeta(s)
    \end{align*}
    We finally get
    \begin{align*}
        \zeta(s)=\frac{1}{1-2^{1-s}}\sum_{n=1}^{\infty}\frac{{(-1)}^{n-1}}{n^s},
        \quad Re(s)>0\exceptone
    \end{align*}
\end{snippetproof}

\section{Zeta function for negative Re (s)}

\section{Zeta function for s=0}

\section{Zeta function for s=1}

\begin{snippetproposition}{zeta-function-residue}{Zeta function residue at \(z=1\)}
    The zeta function is holomorphic everywhere except for a pole at \(Z=1\), which has a residue of \(1\).
\end{snippetproposition}

\begin{snippetproof}{zeta-function-residue-proof}{Zeta function residue at \(z=1\)}
    \[
        \lim_{s\to 1} \zeta(s) =
        \lim_{s\to 1} \frac{s-1}{1-s^{1-s}} \sum_{n=1}^{\infty} \frac{{(-1)}^{n+1}}{n^s}
    \]
    
    This is the alternating harmonic series
    
    \[
        \sum_{n=1}^{\infty} \frac{{(-1)}^{n+1}}{n}=\ln(2)
    \]
    and thus
    \[
        \lim_{s\to 1} \frac{s-1}{1-s^{1-s}} \ln(2)
        = \frac{\ln(2)}{\ln(2) \cdot 2^{1-s}}
        = \frac{1}{2^{1-s}} = 1
    \]
\end{snippetproof}

\plain{This is the only value of the complex plane that cannot be evaluated through analytic continuation.}

\end{document}
