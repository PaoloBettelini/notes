\documentclass[preview]{standalone}

\usepackage{amsmath}
\usepackage{amssymb}
\usepackage{parskip}
\usepackage{fullpage}
\usepackage{hyperref}
\usepackage{bettelini}
\usepackage{stellar}

\hypersetup{
    colorlinks=true,
    linkcolor=black,
    urlcolor=blue,
    pdftitle={Measure Theory},
    pdfpagemode=FullScreen,
}

\begin{document}

\title{Measure Theory}
\id{measuretheory-lebesgue}
\genpage

\plain{<br>}

\begin{snippet}{lebesgue-measure-meaning}
    The idea is to create a measure \(\mu\) which encapsulates
    and generalizes the classical meaning of a measure (i.e. length, area, volume, \(\cdots\)).
    However, such measure does not exist for the measurable space \((\mathbb{R}, \mathcal{P}(\mathbb{R}))\).
\end{snippet}

\begin{snippettheorem}{no-measure-on-r}{}
    Let \(M=(\mathbb{R}, \mathcal{P}(\mathbb{R}), \mu)\)
    be a measure space with
    \begin{enumerate}
        \item \(\mu([a;b]) = b-a, \quad b>1\);
        \item \textbf{traslation invariant:} \(\mu(A) = \mu(\{k + a \suchthat a \in A\}), \quad A \in \mathcal{P}(\mathbb{R}) \land x \in \mathbb{R}\).
    \end{enumerate}
    Then, \(M\) does not exist.
\end{snippettheorem}

\begin{snippetproposition}{condition-gives-only-zero-measure}{}
    Let \((\mathbb{R}, \mathcal{P}(\mathbb{R}), \mu)\) be a measure space where
    \begin{enumerate}
        \item \(\mu((0; 1]) < \infty\);
        \item \textbf{traslation invariant:} \(\mu(A) = \mu(\{k + a \suchthat a \in A\}), \quad A \in \mathcal{P}(\mathbb{R}) \land x \in \mathbb{R}\).
    \end{enumerate}
    Then, \(\mu = 0\).
\end{snippetproposition}

% proof: https://youtu.be/Ur3ofJ61bpk?list=PLBh2i93oe2qvMVqAzsX1Kuv6-4fjazZ8j

\section{Measurable maps}

\begin{snippetdefinition}{measurable-map-definition}{Measurable Map}
    Let \((\Omega_1, \Sigma_1)\) and \((\Omega_2, \Sigma_2)\) be measurable spaces. \\
    A map \(f \colon \Omega_1 \to \Omega_2\) is called \textit{measurable} (with respect to \(\Sigma_1\) and \(\Sigma_2\))
    if \(\forall \sigma_2 \in \Sigma_2, f^{-1}(\sigma_2) \in \Sigma_1\).
    That is, the preimage of any measurable set is still measurable.
\end{snippetdefinition}

\plain{if you consider a simple real function, we want the sets on the y axis to be measurable,}
\plain{as well as those on the x axis. Thus, the preimage of the x values must be measurable.}

% https://youtu.be/11heoNVavvM?list=PLBh2i93oe2qvMVqAzsX1Kuv6-4fjazZ8j&t=356

\end{document}
