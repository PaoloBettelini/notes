\documentclass[preview]{standalone}

\usepackage{amsmath}
\usepackage{amssymb}
\usepackage{parskip}
\usepackage{fullpage}
\usepackage{hyperref}
\usepackage{bettelini}
\usepackage{stellar}

\hypersetup{
    colorlinks=true,
    linkcolor=black,
    urlcolor=blue,
    pdftitle={Cyclic groups},
    pdfpagemode=FullScreen,
}

\begin{document}

\title{Fourier Analysis}
\id{fourieranalysis-epicycles}
\genpage

\section{Fourier Transform}

\begin{snippet}{fourier-analysis-epicycles-expl}
    So how does the cool epicycles animation work?
    First of all we need apply the Fourier transform operation, however
    the drawing is just a set of points, it's a discrete function rather
    than a continuous one, this means that we'll need to use the
    Discrete Fourier Transform operator.
    Each circle represents a discrete frequency, and each center revolves
    around the previous circle's circumference. The radius of the circle
    is the magnitude of the current frequency, which is the absolute value
    of the Fourier transform at that frequency. The revolution is based
    on the time passed and the phase of the Fourier transoform.
\end{snippet}

\includesnpt{fourier-lib}
\includesnpt{fourier-series-1d}

\end{document}
