\documentclass[a4paper]{article}

\usepackage[utf8]{inputenc}
\usepackage{amsmath}
\usepackage{amssymb}
\usepackage{parskip}
\usepackage{fullpage}
\usepackage{pgfplots}
\usepackage{hyperref}

\author{Paolo Bettelini}
\date{}

\pgfplotsset{compat=1.18}

\begin{document}

\maketitle
\tableofcontents
\pagebreak

\section{Elliptic Curves}

\subsection{Definition}

An elliptic curve \(E\) is a set of points such that
\[
	E=\{(x,y)\,|\,y^2=x^3+ax+b\}\cup\{O\},\quad 4a^3+27b^2\neq 0
\]

\begin{center}
	\begin{tikzpicture}
		\begin{axis}[
			xmin=-6,
			xmax=6,
			ymin=-7,
			ymax=7,
			xlabel={\(x\)},
			ylabel={\(y\)},
			scale only axis,
			axis lines=middle,
			domain=-1.912931:3,
			samples=100,
			smooth,
			clip=false,
			axis equal image=true,
			scale=0.9
		]
			\addplot [red] {sqrt(x^3+7)}
				node[right, color=black] {\(y^2=x^3+7\)};
			\addplot [red] {-sqrt(x^3+7)};
		\end{axis}
	\end{tikzpicture}
\end{center}
Where \(O\) is a point at infinity.

The elliptic curve is symmetrical about the x-axis.\\
The opposite of a point \(P\) is its reflection \(-P\).

The coefficients \(a,b\) can be part of
\begin{itemize}
	\item \(\realnumbers\) Real numbers
	\item \(\rationalnumbers\) Rational numbers
	\item \(\complexnumbers\) Complex numbers
	\item \(\naturalnumbers/p\naturalnumbers\) Finite field
\end{itemize}

\subsection{Addition}

Given two points \(P,Q\in E\) we can describe a unique third point.\\
We take the line that intersects \(P\) and \(Q\), the opposite of the third intersection with the curve is out point.
\[
	P+Q=-R
\]

If \(P=Q\), the intersection line will be given by the tangent at that point.\\
If \(P=-Q\), \(P+Q=O\).\\
If \(P=-P\) (inflection point, the concavity of the curve changes) \(R=P\), \(P+P=-P=P\).\\
We consider \(-O\) to be \(O\).

The intersection line \(mx+q\) is given by
\[
	m=\frac{P_y-Q_y}{P_y-Q_x}
\]
and
\[
	q=P_y-mP_x
\]

\begin{center}
	\begin{tikzpicture}
		\begin{axis}[
			xmin=-6,
			xmax=6,
			ymin=-6,
			ymax=6,
			xlabel={\(x\)},
			ylabel={\(y\)},
			scale only axis,
			axis lines=middle,
			domain=-1.912931:3,
			samples=200,
			smooth,
		]
			\addplot [draw=red] {sqrt(x^3+7)};
			\addplot [draw=red] {-sqrt(x^3+7)};
			\addplot [draw=blue][domain=-1.75:1.63] {0.617*x+2.361};

			\draw [color=black, fill=black] (-1.75,1.281) circle (0.1) node [above left] {\(P\)};
			\draw [color=black, fill=black] (0.5,2.669) circle (0.1) node [above] {\(Q\)};
			\draw [color=black, fill=black] (1.631,3.367) circle (0.1) node [below right] {\(R\)};
			\draw [dashed] (1.631,3.367) -- (1.631,-3.367);
			\draw [color=black, fill=black] (1.631,-3.367) circle (0.1) node [right] {\(-R\)};
		\end{axis}
	\end{tikzpicture}
\end{center}

\subsection{Scalar Multiplication}

Given a point \(P\in E\), multiplying \(kP\) where \(k\in\naturalnumbers\) is equivalent to adding \(P\) to itself \(k\) times.\\
Computing \(2P\) is the equivalent of \(P+P\) which can be calculated as \(P+Q=-R\).

\pagebreak

\section{Discrete logarithm problem}

\subsection{Definition}

Given an elliptic curve \(E\) and a point \(P\in E\), we consider
\[
	kP=Q,\quad
	k\in\naturalnumbers
\]
Given the value of \(P\) and \(Q\) it is a hard problem to find the value of \(k\).\\
We can use many ECC multiplication algorithms to compute \(kP\), but reversing this operation for big values of \(k\) is not an easy task.

Furthermore, given \(k_1\) and \(k_2\), we notice that
\begin{align*}
	k_1(k_2P)&=k_2(k_1P)\\
	&=(k_1+k_2)P
\end{align*}
It does not matter if we first multiply \(P\) by \(k_1\) and then \(k_2\) or viceversa, \(P\) will always be multiplied \(k_1+k_2\) times.

\subsection{Diffie–Hellman}

We can use the scalar multiplication function with the Diffie–Hellman method for a key-exchange.

We define an elliptic curve \(E\) over a finite field \(\mathbb{F}_p\).

The \textit{client} and the \textit{server} publicly establish the domain parameters
\begin{itemize}
	\item \(p\) The field that the curve is defined over \(\mathbb{F}_p\) (mod \(p\)).
	\item \(a\) The parameter \(a\) of the elliptic curve equation.
	\item \(b\) The parameter \(b\) of the elliptic curve equation.
	\item \(G\) The generator, a fixed point \(G\in E\).
	\item \(n\) The prime order of \(G\), the smallest prime such that \(kG=O\).\\In order words, the number of points that can be generated multiplying \(G\) with itself.
	\item \(h\) The cofactor, the ratio between the amount of points in \(E\) and the prime order of \(G\).\\Ideally we would want \(h=1\), which means that the points are well-distribuited.
\end{itemize}

We then proceed with the Diffie–Hellman key exchange method.
\begin{enumerate}
	\item The \textit{client} generates a private key \(k_c\in \naturalnumbers\) such that \(1\leq k_c\leq n-1\).
	\item The \textit{server} generates a private key \(k_s\in \naturalnumbers\) such that \(1\leq k_s\leq n-1\).
	\item The \textit{client} computes a public key \(P_c=k_c G\).
	\item The \textit{server} computes a public key \(P_s=k_s G\).
	\item The two parts publicly exchange the public keys.
	\item The \textit{client} computes \(R=k_c P_s\).
	\item The \textit{server} computes \(R=k_c P_s\).
\end{enumerate}

Now both parts share the same secret point \(R\in E\).

\end{document}

% key-exhcange
% signature
% encryption

    %ECIES (Integrated Encryption Scheme) is a hybrid public-key encryption scheme
    %Alice generates an ephemeral keypair, derives the shared DEK, and encrypts the data, and sends it with her ephemeral %publickey (edit) and destroys the ephemeral privatekey
    %Bob uses his privatekey to derive the DEK and decrypts the data
    %https://cryptobook.nakov.com/asymmetric-key-ciphers/ecies-public-key-encryption

    %Elliptic Curve Digital Signature Algorithm (ECDSA)
