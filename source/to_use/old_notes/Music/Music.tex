\documentclass[a4paper]{article}

\usepackage{amsmath}
\usepackage{amssymb}
\usepackage{parskip}
\usepackage{fullpage}
\usepackage{hyperref}

% lilypond is not an classic sty package.
% First you run lilypond-book <file.tex> or <file.lytex>
% lilypond-book --pdf file.tex --out=target
% lilypond will parse every lilypond enviroment in the tex file
% for each enviroment it will generate a pdf image
% then it will generate another tex file in the target folder,
% where it will replace the lilypond enviroments (which do not exist in LaTeX,
% compiling a document normally would give an error) with the pre compiled images
% So you're basically using an enviroment that doesn't exist, then using an external tool
% to parse the document and create another identical TeX file except it points
% to the precompiled music sheets
% I have yet to understand why they couldn't just \usepackage{lilypond} and that's it.

% lilypond-book --pdf *.tex --out=target
% lilypond-book --pdf *.tex --out=target
% cd target
% lualatex *.tex
% mv *.pdf ../
% cd ..
% # these files stack up, they have a diffent name each time
% find . -type f -name "tmp*.pdf" -delete
% find . -type f -name "tmp*.out" -delete

\hypersetup{
    colorlinks=true,
    linkcolor=black,
    urlcolor=blue,
    pdftitle={Music},
    pdfpagemode=FullScreen,
}

\title{Music}
\author{Paolo Bettelini}
\date{}

\begin{document}

\maketitle
\tableofcontents
\pagebreak

\section{Note}

A musical \textbf{note} is a symbol denoting a musical sound.
A note can both represent the pitch and the duration of the sound.

Notes are generally named as (\textit{A-B-C-D-E-F-G}) or (\textit{do-re-mi-fa-sol-la-si}).

Conventionally notes are defind around a central note \(A_4\), which has a pitch of
\(440\) Hz. Each note is an intger number \(n\) of half-steps away from this central note.

\[
    f = 2^\frac{n}{12} \cdot 440 \text{Hz}
\]

\section{Staff}

The \textbf{staff} or \textbf{stave} is a set of five horizontal lines and four spaces that each represent a different
note.

% empty staff
\begin{lilypond}
    \version "2.22.2"
    \paper {
      #(set-paper-size "letter")
      top-margin = 0.7\cm
    }
    \layout { 
      indent = 0.0\cm
      pagenumber = no
    }
    \new Score \with {
      \override TimeSignature #'transparent = ##t
      \override Clef #'transparent = ##t
      defaultBarType = #""
      \remove Bar_number_engraver
      \remove Clef_engraver
    } <<
        \context Staff {
            \repeat unfold 3 {
                s1
            }
        }
    >>
\end{lilypond}

\section{Clef}

A \textbf{clef} is a musical symbol used to indice which notes are represented
by the lines and spaces on a musical stave.
There are mainly 3 types of clefs.

% types of clefs
\begin{center}
    \begin{lilypond}
        \version "2.22.2"
        \score {
            \new StaffGroup <<
                \new Staff \with {
                    \override TimeSignature #'transparent = ##t
                    instrumentName = \markup \center-column {
                        "Treble"
                        "Clef"
                    }
                } {
                    \clef treble << \gtr >>
                }
                \new Staff \with {
                    \override TimeSignature #'transparent = ##t
                    instrumentName = \markup \center-column {
                        "Alto"
                        "Clef"
                    }
                } {
                    \clef alto << \gtr >>
                }
                \new Staff \with {
                    \override TimeSignature #'transparent = ##t
                    instrumentName = \markup \center-column {
                        "Bass"
                        "Clef"
                    }
                } {
                    \clef bass << \gtr >>
                }
            >>
            \layout {
                indent = 2\cm
                
            }
        }
    \end{lilypond}
\end{center}

The position of the clef may be shifted up or down to change the assignment of notes.

\section{Key Signature}

A \textbf{key signature} is a set of symbols placed
on the staff at the beginning of a section of music.
The initial key signature is placed right after the clef.

The symbols are:

\begin{itemize}
    \item \(\sharp\) sharp
    \item \(\flat\) sharp
    \item \(\natural\) natural
\end{itemize}

A \(\sharp\) or \(\flat\) symbol placed on a line or space
indicates that the note represented by that line or space is to be played
a semitone higher (\(\sharp\)) or lower (\(\flat\)).
This applies through the end of the piece or until another key signature is indicated.
The \(\natural\) symbol is used to cancel the signature.

\begin{center}
    \begin{lilypond}
        \version "2.22.2"
        \score {
            \new Staff {
                \relative {
                    \key d \major
                    \clef treble << \gtr >>
                    d'4 fis a c |
                }
            }
        }
    \end{lilypond}
\end{center}

It is also possible to write a key signature symbol right before a single note, this will apply
the pitch change only to it.

% order of appearance

\section{Pitch class}

Two notes with a waveform ratio of any integer power power of two
sound very similar. These notes can be grouped into \textbf{pitch classes}.

\[
    \frac{a}{b} = 2^n
\]

\end{document}
