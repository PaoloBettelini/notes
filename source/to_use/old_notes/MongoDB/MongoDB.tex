\documentclass[a4paper]{article}

\usepackage{amsmath}
\usepackage{amssymb}
\usepackage{parskip}
\usepackage{fullpage}
\usepackage{hyperref}
\usepackage{listings}
\usepackage{xcolor}

\definecolor{lightgray}{rgb}{0.9,0.9,0.9}
\definecolor{darkgray}{rgb}{0.2,0.4,0.2}
\definecolor{darkgreen}{rgb}{0.0,0.4,0.0}
\definecolor{olive}{rgb}{0.17,0.59,0.20}

\lstdefinelanguage{JavaScript}{
    keywords={typeof, new, true, false, catch, function, return, null, catch, switch, var, if, in, while, do, else, case, break},
    keywordstyle=\color{blue}\bfseries,
    ndkeywords={class, extends, export, boolean, throw, implements, import, this},
    ndkeywordstyle=\color{olive}\bfseries,
    sensitive=false,
    comment=[l]{//},
    morecomment=[s]{/*}{*/},
    commentstyle=\color{darkgray}\ttfamily,
    morestring=[b]',
    morestring=[b]",
    stringstyle=\color{red}\ttfamily
}

\lstdefinestyle{js} {
   language=JavaScript,
   backgroundcolor=\color{lightgray},
   extendedchars=true,
   basicstyle=\footnotesize\ttfamily,
   showstringspaces=false,
   showspaces=false,
   numberstyle=\footnotesize,
   numbersep=9pt,
   tabsize=2,
   breaklines=true,
   showtabs=false,
   captionpos=b
}

\author{Paolo Bettelini}
\date{}

\begin{document}

\maketitle
\tableofcontents
\pagebreak

\section{Introduction}

\subsection{MondoDB}

MongoDB is a NoSQL document database (no relations)

A document is a way to store data in keys-values.
\\
A collection contains multiple documents
\\
A database contains multiple collections

BSON (Binary JSON)
\\
BSON also supports Dates, Raw Binary types

\subsection{Documents}

Each document must have a unique "\_id" value
The type ObjectId is used by default to ensure unique values,
however it is not mandatory.
If no "\_id" is provided, it will be automatically popoulated with an ObjectId

\begin{lstlisting}[style=js]
    "_id": ObjectId("5ec5f1b710ca9222e6a46c42")
    "_id": "10"
\end{lstlisting}

\subsection{Data types}

\begin{itemize}
    \item Array
    \item Binary
    \item Boolean
    \item Code
    \item Date
    \item Decimal128
    \item Double
    \item Int32
    \item Int64
    \item MaxKey
    \item MinKey
    \item Null
    \item Object
    \item ObjectID
    \item BSONRegexp
    \item String
    \item BSONSymbol
    \item BSONMap
    \item Timestamp
    \item Undefined
\end{itemize}

\pagebreak

\subsection{Commands}

\subsection{Load from JSON}

\begin{lstlisting}[style=js]
    mongoimport --uri "<Cluster URI>"
                 --drop <filename>.json
\end{lstlisting}

\subsection{Export in JSON}

\begin{lstlisting}[style=js]
    mongoexport --uri "<Cluster URI>"
                --collection=<collection name>
                --out=<filename>.json
\end{lstlisting}

\subsection{Load from BSON}

\begin{lstlisting}[style=js]
    mongorestore --uri "<Cluster URI>"
                 --drop <dump>
\end{lstlisting}

\subsection{Export in BSON}

\begin{lstlisting}[style=js]
    mongodump --uri "<Cluster URI>"
\end{lstlisting}

\subsection{Show databases}

\begin{lstlisting}[style=js]
    show dbs
\end{lstlisting}

\subsection{Use databases}

\begin{lstlisting}[style=js]
    use <database>
\end{lstlisting}

\subsection{Show collections}

\begin{lstlisting}[style=js]
    show collections
\end{lstlisting}

\pagebreak

\section{Basic operations}

\subsection{Search for filter}

returns a cursor

\begin{lstlisting}[style=js]
    db.<collection>.find( {"state":"NY"} )
    db.<collection>.find( {"state":"NY", "city": "ALBANY"} )
\end{lstlisting}

\subsection{Search one}

\begin{lstlisting}[style=js]
    db.<collection>.findOne( {"state":"NY"} )
\end{lstlisting}

\subsection{Iterate through the cursor}

\begin{lstlisting}[style=js]
    it
\end{lstlisting}

\subsection{Count cursor length}

\begin{lstlisting}[style=js]
    db.<collection>.find( <query> ).count()
\end{lstlisting}

\subsection{Insert}

\begin{lstlisting}[style=js]
    db.<collection>.insert( <document> )
    db.<collection>.insert( [ <document>, <document>, ... ] )
\end{lstlisting}

\subsection{Insert without order}

When ordered=true as default, the insertions will stop at the first error

\begin{lstlisting}[style=js]
    db.<collection>.insert( [ <document>, <document>, ... ], { "ordered": false } )
\end{lstlisting}

\subsection{Update all}

\begin{lstlisting}[style=js]
    // Increment by 10 every "pop" field of every document matching the filter
    db.<collection>.updateMany( {"city": "HUDSON"}, {"$inc": {"popoulation": 10}} )
\end{lstlisting}

\subsection{Update one}

\begin{lstlisting}[style=js]
    db.<collection>.updateOne( <query>, <updates> )
\end{lstlisting}

\subsection{Delete all documents}

\begin{lstlisting}[style=js]
    db.<collection>.deleteMany( <query> )
\end{lstlisting}

\subsection{Delete one document}

should only be used when filtering by "\_id".

\begin{lstlisting}[style=js]
    db.<collection>.deleteOne( <query> )
\end{lstlisting}

\subsection{Drop collection}

\begin{lstlisting}[style=js]
    db.<collection>.drop()
    // Deleting all collections will also result in the deletion of the container database
\end{lstlisting}

\section{Syntax}

\subsection{Dollar operator}

The \$ \(\rightarrow\) operator represents the value of a field rather than the field itself

\subsection{Updates}

\begin{lstlisting}[style=js]
    {"$set": {"field1": value, "field2": value, ...}} // Sets the field value
    {"$inc": {"field1": value, "field2": value, ...}} // Increment the field value
    {"$push": {"field1": value, "field2": value, ...}} // Adds an element to an array
\end{lstlisting}

\subsection{Queries}

\begin{lstlisting}[style=js]
    {"field": {"$eq": 60}, ...} // Equals
    {"field": {"$ne": 60}, ...} // Not equals
    {"field": {"$gt": 60}, ...} // Greater than
    {"field": {"$lt": 60}, ...} // Less than
    {"field": {"$gte": 60}, ...} // Greater than or equal
    {"field": {"$lte": 60}, ...} // Less than or equal
\end{lstlisting}

\subsection{Logic Operators}

\begin{lstlisting}[style=js]
    {"$not": {statmement}}
    {"$and": [{statmement}, {statement}, ...]} // Default operator if not specified
    {"$or": [{statmement}, {statement}, ...]}
    {"$nor": [{statmement}, {statement}, ...]}
\end{lstlisting}

\subsection{Expressions}

\begin{lstlisting}[style=js]
	{"$expr": {<expression>}}
\end{lstlisting}

\subsection{Queries for arrays}

\begin{lstlisting}[style=js]
	{"array field": {"$size": 20}}
	{"array field": {"$all": <array>}} // field must contains all the elements in the <array>
\end{lstlisting}

\section{Operators}

\subsection{Projections}

Projections are used to see only certain fields in the resulting output.

\begin{lstlisting}[style=js]
    db.<collection>.find({<query>}, {<projection>})
\end{lstlisting}

\begin{lstlisting}[style=js]
    {"field1": 1, "field2": 1}
    0 -> exclude
    1 -> include
\end{lstlisting}

\subsection{Array operator}

Maches array with at least one element that match the criteria.

\begin{lstlisting}[style=js]
    "$elemMatch": {<field>:<value>}
\end{lstlisting}

\begin{lstlisting}[style=js]
    db.companies.find(
	    {"offices": {"$elemMatch": {"city": "Seattle"}} }
    ).count()
\end{lstlisting}

\subsection{Querying sub-documents}

\begin{lstlisting}[style=js]
    {"doc.field1.subfield1": "value"}
    {"array.0": "value"} // first element of array
    
    db.inspections.find({"address.city": "NEW YORK"})
    
    db.trips.find({"start station location.coordinates.0":
    {"$lt": -74} })
\end{lstlisting}

\subsection{Regular Expressions}

\begin{lstlisting}[style=js]
    {"field": {"$regex": "<regex>"}}
\end{lstlisting}

\subsection{Aggregations}

\begin{lstlisting}[style=js]
    db.<collection>.aggregate([{...},{...},{...}])
\end{lstlisting}

The output of each stage is the output to the next stage.

Stages:
\begin{itemize}
    \item \$project
	\item \$match
	\item \$group
\end{itemize}

\begin{lstlisting}[style=js]
    db.listingsAndReviews.aggregate([
        { "$project": { "address": 1, "_id": 0 }},
        { "$group": {
            "_id": "$address.country",
            "total ": {"$sum": "$price"}
        }}
    ])
    
    db.listingsAndReviews.aggregate([
        { "$project": { "room_type": 1, "_id": 0 }},
        { "$group": {
            "_id": "$room_type",
            "count": {"$sum": 1}
        }}
    ])
\end{lstlisting}

\subsection{Limit}

\begin{lstlisting}[style=js]
    db.<collection>.find(<query>).limit(1)
\end{lstlisting}

\subsection{Sort}

\begin{lstlisting}[style=js]
    db.<collection>.find(<query>).sort(<sorting>)
\end{lstlisting}

Sorting:
\begin{lstlisting}[style=js]
	{"value": 1} // Increasing
	{"value": -1} // Decreascing
\end{lstlisting}

\subsection{Indexes}

\begin{lstlisting}[style=js]
    db.<collection>.createIndex({"field": 1})
\end{lstlisting}

\begin{align*}
	\begin{cases}
		+1,\quad &\rightarrow \text{Increasing} \\
		-1,\quad &\rightarrow \text{Decreasing}
	\end{cases}
\end{align*}

\subsection{Compound Index}

\begin{lstlisting}[style=js]
    db.<collection>.createIndex({"field1":1, "field2":1})
\end{lstlisting}

\subsection{Upsert}

if there is a match (upsert == true), update. Otherwise, insert.

By default upsert is false

\begin{lstlisting}[style=js]
    db.iot.updateOne({ "sensor": r.sensor, "date": r.date,
    "valcount": { "$lt": 48 } },
          { "$push": { "readings": { "v": r.value, "t": r.time } },
         "$inc": { "valcount": 1, "total": r.value } },
  { "upsert": true })
\end{lstlisting}

\end{document}
