\documentclass[a4paper]{article}

\usepackage{amsmath}
\usepackage{amssymb}
\usepackage{parskip}
\usepackage{fullpage}
\usepackage{hyperref}

\author{Paolo Bettelini}
\date{}

\newcommand{\braket}[1]{\left\langle#1\right\rangle}
\newcommand{\innerprod}[2]{\braket{#1\,|\,#2}}

\begin{document}

\maketitle
\tableofcontents
\pagebreak

\section{Complex Vector Space}

A complex vector space is a field where the scalars are complex numbers.

\[
    {\complexnumbers}^n \equiv
    \left\{
        \begin{pmatrix}
            c_0 \\
            c_1 \\
            \vdots \\
            c_{n-1}
        \end{pmatrix}
        ,\quad c_k \in \complexnumbers
    \right\}
\]

The basic operations in \({\complexnumbers}^n\) are the same as in \({\realnumbers}^n\) except for the inner product. 

\section{Inner Product}

\subsection{Definition}

The inner product is basically the same as the dot product between two vectors.
However, when we dot a vector to its self (square of its length) the result should be strictly positive

\[
    \vec{a} \cdot \vec{a} \geq 0
\]

This does not usually happen when the components of the vector are complex numbers.
The product between two arbitrary vectors in the complex space might still be negative or even complex,
but the product between two vectors that are the same should be positive.\\
To solve this problem we use the complex conjugate of the first vector instead of the vector itself.

\[
    \vec{a} \cdot \vec{b} \equiv \sum_{k=0}^{n-1} {(\vec{a_k})}^{*} \vec{b}_k
\]

Another notation for the complex inner product is \(\innerprod{\vec{a}}{\vec{b}}\).

\subsection{Properties}

The complex inner product is not commutative

\[
    \innerprod{\vec{a}}{\vec{b}} \neq \innerprod{\vec{b}}{\vec{a}}
\]

however the order can be inverted if the then take the complex conjugate

\[
    \innerprod{\vec{a}}{\vec{b}}^{*} = \innerprod{\vec{b}}{\vec{a}}
\]

The Cauchy-Schwarz inequality

\[
    \left|\innerprod{\vec{a}}{\vec{b}}\right| \leq {||\vec{a}||}^2 \cdot {||\vec{a}||}^2
\]

these values become equal if one vector is a scalar multiple of the other.

\pagebreak

\section{Definition}

A Hilbert Space is a real or complex vector space that has an inner product and is complete.

\section{Unitary operators}

A linear transformation \(U\) is unitary if it preserves inner products

\[
    \innerprod{U\vec{a}}{U\vec{b}} = \innerprod{\vec{a}}{\vec{b}}
\]

This statement implies that the length of the vector is also preserved

\[
    ||U\vec{a}|| = ||\vec{a}||
\]

Operations such as rotation or phase change are unitary operators.

% https://lapastillaroja.net/wp-content/uploads/2016/09/Intro_to_QC_Vol_1_Loceff.pdf
% A Course in Quantum Computing for the Community College Volume 1 - Michael Loceff - Foothill College

\end{document}