\documentclass[a4paper]{article}

\usepackage{amsmath}
\usepackage{amssymb}
\usepackage{makecell}
\usepackage{parskip}
\usepackage{fullpage}
\usepackage{hyperref}
\usepackage{hologo}

\newcommand{\tabitem}{~~\llap{\textbullet}~~}

\newcommand{\xetex}{\hologo{XeTeX}\ }
\newcommand{\xelatex}{\hologo{XeLaTeX}\ }
\newcommand{\tex}{\hologo{TeX}\ }
\newcommand{\latex}{\hologo{LaTeX}\ }

\author{Paolo Bettelini}
\date{}

\begin{document}

\maketitle
\tableofcontents
\pagebreak

\section{Introduction}

\subsection{What is \TeX ?}

\tex is a typesettimg software (and language) written in 1978 by Donald Knuth.

\tex defines all the primitive commands to produce high quality books (especially mathematical formulae).
The original output file format is the \texttt{.dvi} (Device independent file format).

\subsection{What is \LaTeX ?}

\latex is a logical extension of \tex language adding lots of general purpose macros to it.

A macro is a new command containing primitive commands and/or other macros.

\subsection{\tex Engines}

\tex engines are the executable binaries which implement different \tex variants.

The most common ones are:

\begin{center}
    \bgroup{}
    \def\arraystretch{2}
    \begin{tabular}{ |l|l| }
        \hline
        \textbf{e-\tex} &
        \makecell[l] {
            \tabitem Additional primitive commands to TeX
        } \\
        \hline
        \textbf{pdf\tex} &
        \makecell[l] {
            \tabitem Incorporates the e-TeX extension  \\
            \tabitem Direct access to PDF-specific features
        } \\
        \hline
        \textbf{\xetex} &
        \makecell[l] {
            \tabitem Incorporates the e-TeX extension  \\
            \tabitem Works with native UTF-8 \\
            \tabitem Can access system fonts \\
            \tabitem Sophisticated handling of multilingual typesetting
        } \\
        \hline
        \textbf{Lua\tex} &
        \makecell[l] {
            \tabitem Incorporates the e-TeX extension  \\
            \tabitem Works with native UTF-8 \\
            \tabitem Lua scripting language \\
            \tabitem extensibility through C/C++ plugins
        } \\
        \hline
        \textbf{\makecell[l]{p\tex \\ up\tex}} &
        \makecell[l] {
            \tabitem Variants of \tex written for (primarily) Japanese typesetting
        } \\
        \hline
        \textbf{\makecell[l]{cs\latex \\ pdfcs\latex}} &
        \makecell[l] {
            \tabitem Czech and Slovak \tex support
        } \\
        \hline
    \end{tabular}
    \egroup{}
\end{center}

\subsubsection{\latex versions}

We can also talk about the \latex versions of some of these programs, for instance

\begin{itemize}
    \item \textbf{pdf\latex} means using the \latex macro package with the pdf\tex engine.
    \item \textbf{\xelatex} means using the \latex macro package with the \xetex engine.
    \item \textbf{Lua\latex} means using the \latex macro package with the Lua\tex engine.
    \item \textbf{p\latex} means using the \latex macro package with the p\tex engine.
    \item \textbf{up\latex} means using the \latex macro package with the up\tex engine.
\end{itemize}

\pagebreak

\subsection{Distributions}

MiK\TeX, \TeX Live, W32\TeX, Mac\TeX are a large collection of \TeX-related software to be downloaded and installed.

\subsection{\tex Formats}

\latex is by far the most common \tex format, however, there are a lot more, such as \hologo{ConTeXt}, Op\TeX.
Sometimes we refer to Plain \tex as a small set of macros developed by Knuth himself to make \tex more usable.

\pagebreak

\end{document}

FOrmats:

OpTeX               is a LuaTeX format based on Plain TeX macros
LaTeX
Plain TeX
ConTeXt

%https://latexref.xyz/TeX-engines.html