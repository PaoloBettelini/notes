\documentclass[a4paper]{article}

\usepackage{amsmath}
\usepackage{amssymb}
\usepackage{parskip}
\usepackage{fullpage}
\usepackage{hyperref}
\usepackage{listings}
\usepackage{listings-rust}

\author{Paolo Bettelini}
\date{}

\begin{document}

\maketitle
\tableofcontents
\pagebreak

\section{Data Types}

\subsection{Pritmive Types}

\begin{lstlisting}[language=Rust, style=boxed, numbers=none]
  // boolean
  bool

  // signed integers
  i8, i16, i32, i64, i128, isize

  // unsigned integers
  u8, u16, u32, u64, u128, usize

  // floating points
  f32, f64

  // Text
  char, str
\end{lstlisting}

\subsection{Tuples}

Tuples are a combination of multiple types.
Tuples can contain any number of types and/or other tuples.

\begin{lstlisting}[language=Rust, style=boxed, numbers=none]
    let coordinates = (101, 3, 4);
    let person = ("Paolo", "Bettelini", 18);
    let status: (bool, (u128, i32)) = (true, (1u128, 2));
\end{lstlisting}

\subsection{Arrays}

\subsubsection{Definition}

An array is defined by its type and length.

\begin{lstlisting}[language=Rust, style=boxed, numbers=none]
  let values = [1, 2, 3, 4, 5];
  // with explicit type
  let values: [i32; 5] = [1, 2, 3, 4, 5];
\end{lstlisting}

We can also initialize an array by specifying its default value and length

\begin{lstlisting}[language=Rust, style=boxed, numbers=none]
  let values = [0; 5]; // [0, 0, 0, 0, 0]
\end{lstlisting}

\subsubsection{Indexing}

We can index an array element using the square brackets

\begin{lstlisting}[language=Rust, style=boxed, numbers=none]
  let first = values[0];
  let second = values[1];
\end{lstlisting}

\subsubsection{Slices}

We can point to a portion of the array using slices

\begin{lstlisting}[language=Rust, style=boxed, numbers=none]
  let slice = &values[1..5];
  let slice = &values[1..=5];
  let slice = &values[..5];
  let slice = &values[1..];
  let slice = &values[..];
\end{lstlisting}

\subsection{Struct}

\subsubsection{Definition}

Structs are a way to group multiple values
into a single definition.

\begin{lstlisting}[language=Rust, style=boxed, numbers=none]
    struct Measurement {
        id: u32,
        weight: f64,
        velocity: f64
    }
\end{lstlisting}

Structs can be initialized as follows

\begin{lstlisting}[language=Rust, style=boxed, numbers=none]
    let result = Measurement {
        id: 0,
        weight: 55.5,
        velocity: 22.0
    };
\end{lstlisting}

Variables can accessed

\begin{lstlisting}[language=Rust, style=boxed, numbers=none]
    let id = result.id;
    let weight = result.weight;
\end{lstlisting}

A struct can omit the names of its fields

\begin{lstlisting}[language=Rust, style=boxed, numbers=none]
    struct MyStruct (i32, f64);

    fn main() {
        let result = MyStruct (0, 5.5);
        let a = result.0; // accessing
        let b = result.1;
    }
\end{lstlisting}

\subsection{Union}

A union allows to store different data types
in the same memory location. Every field must have
the same size.

\begin{lstlisting}[language=Rust, style=boxed, numbers=none]
    union Num {
        f: f32,
        i: i32
    }
\end{lstlisting}

\pagebreak

\section{Loops}

\subsection{Loop}

An infinite loop

\begin{lstlisting}[language=Rust, style=boxed, numbers=none]
    loop {
        // ...
    }
\end{lstlisting}

\subsection{While}

A while loop

\begin{lstlisting}[language=Rust, style=boxed, numbers=none]
    while a > 0 {
        // ...
    }
\end{lstlisting}

\subsection{For}

A for loop

\begin{lstlisting}[language=Rust, style=boxed, numbers=none]
    for i in 0..10 {
        // ...
    }
\end{lstlisting}

\subsection{Returning from loops}

\begin{lstlisting}[language=Rust, style=boxed, numbers=none]
  let mut counter = 0;

  let result = loop {
      counter += 1;

      if counter == 10 {
          break counter;
      }
  };
\end{lstlisting}

\subsection{Labels}

\begin{lstlisting}[language=Rust, style=boxed, numbers=none]
    'outer: loop {
        'inner: loop {
            // This breaks the inner loop
            break;
            // This breaks the outer loop
            break 'outer;
        }
    }
\end{lstlisting}

\subsection{Returning from labelled loops}

\begin{lstlisting}[language=Rust, style=boxed, numbers=none]
    let mut counter = 0;

    let result = 'outer: loop {
        counter += 1;

        if counter == 10 {
            break 'outer counter;
        }
    };
\end{lstlisting}

\section{Pattern Matching}

\subsection{Basic}

\begin{lstlisting}[language=Rust, style=boxed, numbers=none]
  let x = 5;

  match x {
    // matching literals
    1 => println!("one"),
    // matching multiple patterns
    2 | 3 => println!("two or three"),
    // matching ranges
    4..=9 => println!("within range"),
    // matching named variables
    x => println!("{}", x),
    // default case (ignores value)
    _ => println!("default Case")
  }
\end{lstlisting}

\pagebreak

\subsection{Destructuring}

\begin{lstlisting}[language=Rust, style=boxed, numbers=none]
  struct Point {
      x: i32,
      y: i32,
    }
    
    let p = Point { x: 0, y: 7 };
    
    match p {
      Point { x, y: 0 } => {
        println!("{}" , x);
      },
      Point { x, y } => {
        println!("{} {}" , x, y);
      },
    }
    
    enum Shape {
      Rectangle { width: i32, height: i32 },
      Circle(i32),
    }
    
    let shape = Shape::Circle(10);
    
    match shape {
      Shape::Rectangle { x, y } => //...
      Shape::Circle(radius) => //...
    }
\end{lstlisting}

\subsection{Ignoring values}

\begin{lstlisting}[language=Rust, style=boxed, numbers=none]
  struct SemVer(i32, i32, i32);

  let version = SemVer(1, 32, 2);
  
  match version {
    SemVer(major, _, _) => {
      println!("{}", major);
    }
  }
  
  let numbers = (2, 4, 8, 16, 32);
  
  match numbers {
    (first, .., last) => {
      println!("{}, {}", first, last);
    }
  }
\end{lstlisting}

\pagebreak

\subsection{Match guards}

\begin{lstlisting}[language=Rust, style=boxed, numbers=none]
  let num = Some(4);

  match num {
    Some(x) if x < 5 => println!("less than five: {}", x),
    Some(x) => println!("{}", x),
    None => (),
  }
\end{lstlisting}

\subsection{@ bindings}

Bind value to a name

\begin{lstlisting}[language=Rust, style=boxed, numbers=none]
  match beaufort() {
    v @ 0..1    => println!("Calm : {} km/h", v),
    v @ 1..=5   => println!("Light Air : {} km/h", v),
    v @ 5..=11  => println!("Light Breeze : {} km/h", v),
    v @ 11..=19 => println!("Gentle Breeze : {} km/h", v)
  }
\end{lstlisting}

\pagebreak

\section{Common types}

\subsection{Option}

A function that may fail might enclose its return value in an
\textbf{Option} enum, to notify wheter the action was successful.

\begin{lstlisting}[language=Rust, style=boxed, numbers=none]
  fn sqrt(v: f64) -> Option<(f64, f64)> {
    if v < 0.0 {
      return None;
    }

    let sqrt = v.sqrt();
    Some((sqrt, -sqrt))
  }
\end{lstlisting}

\subsection{Result}

\subsubsection{Definition}

The \textbf{Result} enum is similar to \textbf{Option} but it
specifies why the function has failed.

When the function doesn't really need to return anything other than the
\textbf{Result} status, \textbf{()} can be used.

\begin{lstlisting}[language=Rust, style=boxed, numbers=none]
  enum ErrorType {
    NegativeBase,
    NegativeArgument,
    BaseOne
  }
  
  fn log(base: f64, arg: f64) -> Result<f64, ErrorType> {
    if base <= 0.0 {
      return Err(ErrorType::NegativeBase);
    }

    if base == 1.0 {
      return Err(ErrorType::BaseOne);
    }

    if arg <= 0.0 {
      return Err(ErrorType::NegativeArgument);
    }

    let result = arg.log(base);
    Ok(result)
  }
\end{lstlisting}

\pagebreak

\subsubsection{? operator}

The \textbf{?} operator is syntax sugar for \textbf{Result} handling. \\
This operator can be placed at the end of a \textbf{Result} type.
If the result is an error, the functions returns it, otherwise unwraps
its value.

\begin{lstlisting}[language=Rust, style=boxed, numbers=none]
  fn log(base: f64, arg: f64) -> Result<f64, ErrorType> { ... }

  fn something() -> Result<f64, ErrorType> {
    let v = match log(2.718, 3.14) {
      Ok(v) => v,
      Err(e) => return Err(e)
    };

    // use `v`
  }
\end{lstlisting}

can be written as

\begin{lstlisting}[language=Rust, style=boxed, numbers=none]
  fn log(base: f64, arg: f64) -> Result<f64, ErrorType> { ... }

  fn something() -> Result<f64, ErrorType> {
    let v = log(2.718, 3.14)?;

    // use `v`
  }
\end{lstlisting}

\subsection{Box}

\texttt{Box<T>} is a smart pointer used for heap allocation.
You can dereference a \texttt{Box} to access its value.

\begin{lstlisting}[language=Rust, style=boxed, numbers=none]
  // Moving a value from stack to heap
  let boxxed = Box::new(num);
  // or
  let boxxed = box num;

  let a = *boxxed + 42;
\end{lstlisting}

\subsection{Rc}

The \texttt{Rc<T>} smart pointer (Reference Counted) is a type that provides
shared ownership of a heap allocated value. Cloning an \texttt{Rc} produces a shallow copy.
This data type keeps count of all the owners, and drops the value when there are 0 owners.

\begin{lstlisting}[language=Rust, style=boxed, numbers=none]
  let foo = Rc::new(value);
  let bar = foo.clone();
  // both point to `value`
\end{lstlisting}

\subsection{Arc}

The \texttt{Arc<T>} smart pointer (Atomic Reference Counted) is the same as \texttt{Rc<T>}
but uses atomic operations to increment the owner counter, so it is thread-safe.

\subsection{UnsafeCell}

\texttt{UnsafeCell<T>} is the core primitive that enables
\textbf{inner mutability}. This means that the value inside it can be
mutated even with a shared reference. This is a special type and the compiler
has special knowledge about it.

Inner mutability is accomplished by using \texttt{std::mem::replace()}
to mutate the value.

\subsection{Cell}

A \texttt{Cell<T>} enables interior mutability using an \texttt{UnsafeCell}.

\begin{lstlisting}[language=Rust, style=boxed, numbers=none]
  let cell = Cell::new(10); // not mutable
  cell.set(42);
  let v = cell.get() + 24;
\end{lstlisting}

\subsection{RefCell}

A \texttt{RefCell<T>} is like \texttt{Cell} but it will enforce
borrowing rules at runtime. This will make it impossible
to have multiple mutable reference to the data.

\subsubsection{Ref and RefMut}

\texttt{Ref} and \texttt{RefMut} are wrappers around a \texttt{RefCell}
and they are used to update the share state of the \texttt{RefCell}
when they are dropped.

\subsection{Mutex}

\subsection{RwLock}

\subsection{AsRef}

\pagebreak

\section{Cargo tools}

\subsection{cargo clippy}

A collection of lints to catch common mistakes and improve your Rust code.

\subsection{cargo fmt}

A tool for formatting Rust code according to style guidelines.

\subsection{cargo fix}

Automatically fixes lint warnings repoted by the compiler.

\subsection{cargo tree}

Shows a dependency graph

\subsection{cargo expand}

A tool that prints the result of the full macro expansion of the current crate.

\subsection{cargo modules}

Generates a tree of a crate's modules.

\end{document}
% https://youtu.be/8O0Nt9qY_vo?t=35ust of Rust: Smart Pointers and Interior Mutabi80
% Deref trait is called when you .smth() on a struct.

% https://docs.google.com/document/d/1kQidzAlbqapu-WZTuw4Djik0uTqMZYyiMXTM9F21Dz4/edit