\documentclass[a4paper]{article}

\usepackage{amsmath}
\usepackage{amssymb}
\usepackage{parskip}
\usepackage{fullpage}
\usepackage{hyperref}
\usepackage{xcolor,listings}

\hypersetup{
    colorlinks=true,
    linkcolor=black,
    urlcolor=blue,
    pdftitle={SQL},
    pdfpagemode=FullScreen,
}

\author{Paolo Bettelini}
\date{}

\lstdefinestyle{sql} {
    language=SQL,
  	backgroundcolor=\color{lightgray},
  	breakatwhitespace=false,
    basicstyle=\footnotesize,
  	breaklines=true,
  	captionpos=b,
    commentstyle=\color{dkgreen},
  	deletekeywords={...},
  	escapeinside={\%*}{*)},
  	extendedchars=true,
  	keepspaces=true,
  	keywordstyle=\color{blue},
  	morekeywords={*,modify,MODIFY,...},
  	numbers=none,
  	showspaces=false,
  	showstringspaces=false, 
  	showtabs=false,
  	stepnumber=1,
}

\begin{document}

\maketitle
\tableofcontents
\pagebreak

\section{Users}

\begin{lstlisting}[style=sql]
    CREATE USER 'username'@'localhost' IDENTIFIED BY 'password';
    % access from everywhere
    CREATE USER 'username'@'\%' IDENTIFIED BY 'password';
    % grant all + can grant
    GRANT ALL ON database.* TO 'username'@'localhost' [WITH GRANT OPTION];
    % grant single actions
    GRANT SELECT, UPDATE, INSERT, DELETE ON database.table TO 'username'@'localhost';
    % for triggers
    GRANT EXECUTE ON database.* TO 'username'@'localhost';
    % grant action on column
    GRANT UPDATE(column) ON database.table TO 'username'@'localhost';
\end{lstlisting}

\section{Triggers}

\begin{lstlisting}[style=sql]
    DELIMITER //
    CREATE
    [DEFINER = <user>]
    TRIGGER triggerName
    BEFORE [INSERT | UPDATE | DELETE]
    ON some_table
    FOR EACH ROW BEGIN
        IF  NEW.field != 'P' AND NEW.field != 'D' THEN
            SET NEW.ruolo = 'NV';
        END IF;
    END;
    // 
    DELIMITER ;
\end{lstlisting}

\section{Cursors}

\begin{lstlisting}[style=sql]
    BEGIN
        % variables
        
        DECLARE done INT DEFAULT FALSE;
        DECLARE cur CURSOR FOR SELECT id, nome, cognome FROM persone;
        DECLARE CONTINUE HANDLER FOR NOT FOUND SET done = 1;

        OPEN cur;
        iter: WHILE NOT done DO
            FETCH cur INTO var1, var2, var3;

            % ...
        END WHILE iter;
    END;
\end{lstlisting}

\section{Stored Procedures}

\subsection{Definition}

\begin{lstlisting}[style=sql]
    DELIMITER //
    CREATE PROCEDURE addDummyData()
    BEGIN
        INSERT INTO dummyTable VALUES ();
    END
    //
    DELIMITER ;
\end{lstlisting}

\begin{lstlisting}[style=sql]
    DELIMITER //
    CREATE PROCEDURE log(var1 VARCHAR(25), var2 VARCHAR(25), var3 VARCHAR(25))
    BEGIN
        INSERT INTO dummyTable (field1, field2, field3) VALUES (var1, var2, var3);
    END;
    // DELIMITER ;
\end{lstlisting}

\begin{lstlisting}[style=sql]
    DELIMITER //
    CREATE PROCEDURE doSomething(IN inVar INT, OUT outVar INT)
    BEGIN    
        SELECT COUNT(id) INTO outVar FROM dummyTable WHERE field = inVar;
    END;
    //
    DELIMITER ;
\end{lstlisting}

\subsection{Call}

\begin{lstlisting}[style=sql]
    CALL addDummyData();
    CALL doSomething(42, @res);
    SELECT @res;
\end{lstlisting}

\section{Events}

\subsection{Enable}

\begin{lstlisting}[style=sql]
    SET GLOBAL event_scheduler = ON;
    SET @@global.event_scheduler = ON;
    SET GLOBAL event_scheduler = 1;
    SET @@global.event_scheduler = 1;
\end{lstlisting}

\subsection{Definition}

\begin{lstlisting}[style=sql]
    CREATE EVENT eventName
    ON SCHEDULE EVERY 1 SECOND
    COMMENT 'Adds data'
    DO
        CALL someProcedure();
\end{lstlisting}

\begin{lstlisting}[style=sql]
    CREATE EVENT eventName
    ON SCHEDULE EVERY 1 SECOND
    DO BEGIN
        DELETE FROM dummyTable;
    END
\end{lstlisting}

\begin{lstlisting}[style=sql]
    CREATE EVENT eventName
    ON SCHEDULE AT CURRENT_TIMESTAMP + INTERVAL 10 SECOND
    DO BEGIN
        DELETE FROM dummyTable;
    END
\end{lstlisting}

ON SCHEDULE EVERY interval STARTS timestamp [+INTERVAL] ENDS timestamp [+INTERVAL]

\pagebreak

\section{Functions}

\subsection{Definition}

\begin{lstlisting}[style=sql]
    DELIMITER //
    CREATE FUNCTION countSomething (name VARCHAR(25))
    RETURNS INT
    [DETERMINISTIC]
    [READS SQL DATA]
    BEGIN
        RETURN (SELECT COUNT(*) FROM dummyTable WHERE user_name = name);
    END;
    // DELIMITER ;
\end{lstlisting}

\subsection{Call}

\begin{lstlisting}[style=sql]
    SELECT countSomething('Paolo');
\end{lstlisting}

\pagebreak

\end{document}





