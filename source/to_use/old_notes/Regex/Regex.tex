\documentclass[a4paper]{article}

\usepackage{amsmath}
\usepackage{amssymb}
\usepackage{parskip}
\usepackage{fullpage}
\usepackage{hyperref}
\usepackage{yFlatTable} % https://github.com/HarveySheppard/yLaTeX
\usepackage{makecell}

\author{Paolo Bettelini}
\date{}

\begin{document}

\maketitle
\tableofcontents
\pagebreak

\section{Operators}

\subsection{Anchors}

\begin{tabular}{_r^l}
    \tableHeaderStyle
    Symbol & Description \\
    \^{} & Start of string or line \\
    \$ & End of string or line \\
    \textbackslash A & Start of string \\
    \textbackslash Z & End of string \\
    \textbackslash b & Word boundary \\
    \textbackslash B & Not word boundary \\
    \textbackslash < & Start of word \\
    \textbackslash > & End of word \\
\end{tabular}

\subsection{Quantifiers}

\begin{tabular}{_r^l}
    \tableHeaderStyle
    Symbol & Description \\
    * & Match 0 or more times \\
    + & Match 1 or more times \\
    ? & Match 0 or 1 time \\
    \{n\} & Match n times \\
    \{n,\} & Match n or more times \\
    \{n,m\} & Match from n to m times \\
    \hline
    ? & \makecell[l]{
        You can add a ? after a quantifier \\
        to make it \textit{un-greedy}. \\
        The engine will match as few \\
        characters as possible.} \\
\end{tabular}

\subsection{Escaping}

\begin{tabular}{_r^l}
    \tableHeaderStyle
    Symbol & Description \\
    \textbackslash & Escape character \\
    \textbackslash Q & Start literal sequence \\
    \textbackslash E & End literal sequence \\
\end{tabular}

\subsection{Ranges}

\begin{tabular}{_r^l}
    \tableHeaderStyle
    Symbol & Description \\
    \([\cdots]\) & \makecell[l]{
            A range. \\
            E.g. [abcz], [0-9], [0-9A-Z], [a-f].} \\
    \([\text{\^{}}\cdots]\) & Not range \\
\end{tabular}

\subsection{Characters}

\begin{tabular}{_r^l}
    \tableHeaderStyle
    Symbol & Description \\
    \textbackslash c & Control character \\
    \textbackslash s & White space \\
    \textbackslash S & Not white space \\
    \textbackslash d & Digit \\
    \textbackslash D & Not digit \\
    \textbackslash w & Word \\
    \textbackslash W & Not word \\
    \textbackslash x & Hexadecimal digit \\
    \textbackslash O & Octal digit \\
\end{tabular}

\subsection{POSIX}

\begin{tabular}{_r^l}
    \tableHeaderStyle
    Symbol & Description \\
    \([\text{:upper:}]\) & Upper case letters \\
    \([\text{:lower:}]\) & Lower case letters \\
    \([\text{:alpha:}]\) & All letters \\
    \([\text{:alnum:}]\) & Digits and letters \\
    \([\text{:digit:}]\) & Digits \\
    \([\text{:xdigit:}]\) & Hexadecimal digits \\
    \([\text{:punct:}]\) & Punctuation \\
    \([\text{:blank:}]\) & Space and tab \\
    \([\text{:blank:}]\) & Blank characters \\
    \([\text{:cntrl:}]\) & Control characters \\
    \([\text{:graph:}]\) & Printed characters \\
    \([\text{:print:}]\) & Printed characters and spaces \\
    \([\text{:word:}]\) & Digits, letters and underscore \\
\end{tabular}

\subsection{Special characters}

\begin{tabular}{_r^l}
    \tableHeaderStyle
    Symbol & Description \\
    \textbackslash n & New line \\
    \textbackslash r & Carriage return \\
    \textbackslash t & Tab \\
    \textbackslash v & Vertical tab \\
    \textbackslash f & Form feed \\
    \textbackslash XXX & Octal character \textit{XXX} \\
    \textbackslash xHH & Hex character \textit{HH} \\
\end{tabular}

\subsection{String Replacement}

\begin{tabular}{_r^l}
    \tableHeaderStyle
    Symbol & Description \\
    \$n & Nth non-passive group \\
    \$2 & \textbf{xyz} in \textbf{/\^{}(abc(xyz))\$/} \\
    \$1 & \textbf{­xyz­} in \textbf{/\^{}(?:abc(xyz))\$/} \\
    \$` & Before matched string \\
    \$' & After matches string \\
    \$+ & Last matches string \\
    \$\& & Entire matched string \\
\end{tabular}

\subsection{Assertions}

\begin{tabular}{_r^l}
    \tableHeaderStyle
    Symbol & Description \\
    ?= & Lookahead assertion \\
    ?! & Negative lookahead \\
    ?<= & Lookbehind assertion \\
    ?!= & Negative lookbehind \\
    ?<! & Negative lookbehind \\
    ?> & Once-only Subexp­ression \\
    ?() & Condition \([\)if then\(]\) \\
    ?()| & Condition \([\)if then else\(]\) \\
    ?\# & Comment \\
\end{tabular}

\subsection{Pattern modifiers}

\begin{tabular}{_r^l}
    \tableHeaderStyle
    Symbol & Description \\
    g & Global match \\
    i * & Case-insensitive \\
    m * & Multiple lines \\
    s * & Treat string as single line \\
    x * & Allow comments and whitespace in pattern \\
    e * & Evaluate replac­ement \\
    U * & Un-greedy pattern \\
\end{tabular}


\subsection{Other} % i don't know where to put these

\begin{tabular}{_r^l}
    \tableHeaderStyle
    Symbol & Description \\
    (...) & Group \\
    ?: & Non-capturing group \\
    | & Or operator \\
\end{tabular}

\pagebreak

\end{document}
