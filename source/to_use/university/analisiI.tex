\documentclass[a4paper]{article}

\usepackage{amsmath}
\usepackage{amssymb}
\usepackage{stellar}
\usepackage{parskip}
\usepackage{fullpage}
\usepackage{wrapfig}
\usepackage{tikz}

\usetikzlibrary{arrows}
\usetikzlibrary{decorations.pathreplacing}

\title{Analisi I}
\author{Paolo Bettelini}
\date{}

\begin{document}

\maketitle
\tableofcontents

\section{Assiomi di Peano}

\sdefinition{Assiomi di Peano}{
    Gli \textit{assiomi di Peano} incudono i numeri naturali:
    \begin{itemize}
        \item il valore \(1\) è un numero;
        \item ogni numero \(n\) ha il suo successore \(S(n) = n+1\);
        \item se \(m\neq n\), allora \(S(m) \neq S(n)\);
        \item il numero \(1\) non è il successore di alcun numero;
        \item \textbf{assioma induttivo:} sia \(E \subseteq \mathbb{N}\) tale che \(1\in E\), allora
        \[
            n\in E \implies S(n) \in E
        \]
    \end{itemize}
    L'insieme \(E\) è l'insieme \(\mathbb{N}\).
}

La funzione successore è initettiva.

\sdefinition{Sottoinsieme finale}{
    Un sottoinsieme \(E \subseteq \mathbb{N}\) si dice \textit{finale} se
    \(E=\{n_0, n_0+1, n_0+2, \cdots\}\)
    per qualche \(n_0 \in \mathbb{N}\).
}

Esiste quindi un valore \(n\in \mathbb{N}\) tale che
\[
    E= \{ n \in \mathbb{N} \,|\, n \geq n_0 \}
\]

\sproposition{}{
    Usando l'assioma indutivo si deduce che se \(A\) è un insieme tale che \(n_0\in A\)
    e \(\forall n \in A, S(n) \in A\), allora \(A\) è finale.
}

\pagebreak

\section{Principio di induzione}

\stheorem{Principio di induzione}{
    Sia \(P(n)\) una proposizione dove \(n\in\mathbb{N}\), allora
    \[
        P(0) \land (P(n) \implies P(n+1)) \implies \forall n\in\mathbb{N}, P(n)
    \]
}

\stheorem{Equivalenza principio e assioma di induzione}{
    L'assioma induttivo è equivalente al principio di induzione.
}

\sproof{Equivalenza principio e assioma di induzione}{
    \iffproof{
        Sia
        \[
            E= \{ n\in\mathbb{N} \,|\, P(n) \}
        \]
        Se \(P(1)\) è vera e cioè \(1\in E\), e che per ogni \(n\) per cui \(P(n)\) è vera,
        e cioè \(n\in E\), abbiamo \(n+1\in E\).
        Allora \(E=\mathbb{N}\), che è la conclusione dell'assioma induttivo.
    }{
        TODO: Dimostrare che \(E=\mathbb{N}\) sapendo che vale il principio di induzione.
    }
}

\sproposition{Principio di induzione forte}{
    Il principio di induzione è equivalente alla seguente forma:
    sia \(P(n)\) una proposizione dove \(n\in\mathbb{N}\) tale che
    \begin{itemize}
        \item \(P(1)\) è vera;
        \item \(P(k)\) è vera per tutte le \(k \leq n\), allora \(P(n+1)\) è vera.
    \end{itemize}
    Allora \(P(n)\) è vera per tutte le \(n\).
}

\sexample{Principio di induzione}{
    Dimostrare che per ogni \(n \geq 1\), la somma \[ \sum_{k=1}^n k = \frac{n(n+1)}{2} \].
    \begin{itemize}
        \item Il caso base è dato da \(n=1\) dove \(1 = \frac{2}{2} = 1\).
        \item Il caso induttivo è dato dato da \(\xi = n+1\)
        \begin{align*}
            \frac{n(n+1)}{2} + \xi &= \frac{n(n+1)}{2} + \frac{2n}{2} + \frac{2}{2} \\
            &= \frac{n^2 + 3n + 2}{2} \\
            &= \frac{(n+1)(n+2)}{2} \\
            &= \frac{\xi(\xi+1)}{2}
        \end{align*}
    \end{itemize}
}

Considerando la serie
\[
    \sum_{k=1}^n a_k
\]
e impostiamo \(j=n-k+1\), abbiamo che la sommatoria è pari a
\[
    \sum_{j=1}^n a_{n-j+1}
\]

\sexample{Principio di induzione}{
    Dimostrare che
    \[
        \sum_{k+1}^n k^2 = \frac{n(n+1)(n+2)}{6}
    \]
}

\sexample{Principio di induzione}{
    Per ogni \(n \geq 0\) e per ogni \(h > -1\),
    \[
        {(1+h)}^n \geq 1 + nh
    \]
}

% se a,b,c sono numeri, a >= b \implies a+c >= b+c
% e se c>=0, imlica anche ac>=bc


\end{document}