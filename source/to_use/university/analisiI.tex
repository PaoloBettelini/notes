\documentclass[a4paper]{article}

\usepackage{amsmath}
\usepackage{amssymb}
\usepackage{stellar}
\usepackage{parskip}
\usepackage{fullpage}
\usepackage{stellar}
\usepackage{wrapfig}
\usepackage{tikz}

\usetikzlibrary{arrows}
\usetikzlibrary{decorations.pathreplacing}

\title{Analisi I}
\author{Paolo Bettelini}
\date{}

\begin{document}

\maketitle
\tableofcontents

\section{Assiomi di Peano}

\sdefinition{Assiomi di Peano}{
    Gli \textit{assiomi di Peano} incudono i numeri naturali:
    \begin{itemize}
        \item il valore \(1\) è un numero;
        \item ogni numero \(n\) ha il suo successore \(S(n) = n+1\);
        \item se \(m\neq n\), allora \(S(m) \neq S(n)\);
        \item il numero \(1\) non è il successore di alcun numero;
        \item \textbf{assioma induttivo:} sia \(E \subseteq \mathbb{N}\) tale che \(1\in E\), allora
        \[
            n\in E \implies S(n) \in E
        \]
        Allora l'insieme \(E\) è l'insieme \(\mathbb{N}\).
    \end{itemize}
}

La funzione successore è initettiva.

%%%% TODO
\sdefinition{Sottoinsieme finale}{
    Un sottoinsieme \(E \subseteq \mathbb{N}\) si dice \textit{finale} se
    \(E=\{n_0, n_0+1, n_0+2, \cdots\}\)
    per qualche \(n_0 \in \mathbb{N}\).
}

%%%% TODO
Esiste quindi un valore \(n\in \mathbb{N}\) tale che
\[
    E= \{ n \in \mathbb{N} \,|\, n \geq n_0 \}
\]

%%%% TODO
\sproposition{}{
    Usando l'assioma indutivo si deduce che se \(A\) è un insieme tale che \(n_0\in A\)
    e \(\forall n \in A, S(n) \in A\), allora \(A\) è finale.
}

\pagebreak

\section{Principio di induzione}

\stheorem{Principio di induzione}{
    Sia \(P(n)\) una proposizione dove \(n\in\mathbb{N}\), allora
    \[
        P(0) \land (P(n) \implies P(n+1)) \implies \forall n\in\mathbb{N}, P(n)
    \]
}

\stheorem{Equivalenza principio e assioma di induzione}{
    L'assioma induttivo è equivalente al principio di induzione.
}

\sproof{Equivalenza assioma e principio di induzione}{
    \newcommand{\successor}{S}
    Given a proposition \(P(n)\), let \[
        E= \{ n\in\mathbb{N} \,|\, P(n) \}
    \]
    \iffproof{
        If \(0\in E\), then \(P(0)\) is true. \\
        If \(n \in E \implies \successor(n) \in E\), then \(P(n) \implies P(\successor(n))\).\\
        If the latter conditions are satisfied, then by the axiom of induction,
        \(E=\mathbb{N}\), and thus \[\forall n\in \mathbb{N}, P(n)\]
    }{
        If \(P(0)\) is true, then \(0\in E\).\\
        If \(P(n) \implies P(\successor(n)) \), then if \(n\in E \implies \successor(n) \in E\).\\
        If the latter conditions are satisfied, then by the principle of induction
        \[\forall n\in\mathbb{N}, n \in E\] and thus \(E=\mathbb{N}\).
    }
}

\sproposition{Principio di induzione forte}{
    Il principio di induzione è equivalente alla seguente forma:
    sia \(P(n)\) una proposizione dove \(n\in\mathbb{N}\) tale che
    \begin{itemize}
        \item \(P(1)\) è vera;
        \item \(P(k)\) è vera per tutte le \(k \leq n\), allora \(P(n+1)\) è vera.
    \end{itemize}
    Allora \(P(n)\) è vera per tutte le \(n\).
}

\sexample{Principio di induzione}{
    Dimostrare che per ogni \(n \geq 1\), la somma \[ \sum_{k=1}^n k = \frac{n(n+1)}{2} \].
    \begin{itemize}
        \item Il caso base è dato da \(n=1\) dove \(1 = \frac{2}{2} = 1\).
        \item Il caso induttivo è dato dato da \(\xi = n+1\)
        \begin{align*}
            \frac{n(n+1)}{2} + \xi &= \frac{n(n+1)}{2} + \frac{2n}{2} + \frac{2}{2} \\
            &= \frac{n^2 + 3n + 2}{2} \\
            &= \frac{(n+1)(n+2)}{2} \\
            &= \frac{\xi(\xi+1)}{2}
        \end{align*}
    \end{itemize}
}

Considerando la serie
\[
    \sum_{k=1}^n a_k
\]
e impostiamo \(j=n-k+1\), abbiamo che la sommatoria è pari a
\[
    \sum_{j=1}^n a_{n-j+1}
\]

\sexample{Principio di induzione}{
    Dimostrare che
    \[
        \sum_{k+1}^n k^2 = \frac{n(n+1)(2n+2)}{6}
    \]
}

\sexample{Principio di induzione}{
    Per ogni \(n \geq 0\) e per ogni \(h > -1\),
    \[
        {(1+h)}^n \geq 1 + nh
    \]
}

% se a,b,c sono numeri, a >= b \implies a+c >= b+c
% e se c>=0, imlica anche ac>=bc

\section{Combinatorica}

Il valore \(n!\) è pari alla cardinalità dell'insieme di tutte le funzioni fa \(F_n\) a \(F_n\)
che sono biettive. Dove \(F_n = \{ 1,2,3 \cdots, n \}\).
\[
    n! = \left| \left\{ f \colon F_n \to F_n \right\} \right|
\]

\sproof{Cardinalità di queste funzioni}{
    \begin{itemize}
        \item Il caso base è \(F_1\), che contiene solo 1 elemento e \(1!=1\).
        \item Caso induttivo: notiamo che dato l'insieme \(F_n\), aggiungendo un oggetto
        quest'ultimo possiamo posizionarlo in \(n+1\) posizioni. Di conseguenza, il nuovo numero di
        permutazioni è \(n!(n+1) = (n+1)!\).
    \end{itemize}    
}

La funzione \(\sigma(n)\) è una funzione di permutazione (funzione biettiva che permuta \(n\) elementi).
Infatti, le permutazione di \(n\) sono \(n!\), ossia la cardinalità, cioè tutte le funzioni biettive possibili per permutare gli oggetti.

\sdefinition{Disposizioni}{
    Le \textit{disposizioni} di \(k\) oggetti scelti fra \(n\) oggetti,
    dove \(1\leq k \leq n\), sono il numero delle funzioni
    iniettive \(f\colon F_k \to F_n\).
    \[
        D_{n,k} = \frac{n!}{(n-k)!}
    \]
}

\sdefinition{Combinazioni}{
    Le \textit{combinazioni} di \(k\) oggetti scelto fra \(n\) oggetti,
    dove \(1\leq k \leq n\), sono il numero di sottoinsiemi di \(F_n\)
    di cardinalità \(k\).
    \[
        C_{n,k} = \binom{n}{k} = \frac{n!}{k!(n-k)!}
    \]
}

Abbimao che
\[
    D_{n,k} = k! \cdot C_{n,k}
\]

\slemma{Proprietà dei coefficienti binomiali}{
    Per ogni \(0 \leq k \leq n\)
    \[
        \binom{n}{k} = \binom{n}{n-k}
    \]
}

\pagebreak

\stheorem{Leggi di De Morgan}{
    \[
        {(A \cap B)}^c = A^c \cup B^c
    \]
    e
    \[
        {(A \cup B)}^c = A^c \cap B^c
    \]
    con il complementare rispetto a qualche insieme \(X\).
}

\sproof{Leggi di De Morgan}{
    \(x\in {(A \cap B)}^c\) è equivalente a  \(x \notin A \cap B\),
    che è equivalente a \(x\notin A\) o \(x\notin B\).
    Allora \(x\in A^c\) o \(x\in B^c\), e quindi \(x \in A^c \cup B^c\).
}

\stheorem{Teorema del binomio}{
    Let \(n \in \mathbb{N}\) and \(x,y \in \mathbb{R}\).
    \[
        {(x+y)}^n = \sum_{k=0}^n \binom{n}{k} x^{n-k}y^k
    \]
}

\section{Funzione indicatrice}

\sdefinition{Funzione indicatrice}{
    Sia \(X\) un insieme e \(E \subseteq X\). La \textit{funzione caratteristica}
    di \(E\) è data da
    \[
        1_E = \begin{cases}
            1 & x \in E \\
            0 & x \notin E
        \end{cases}
    \]
}

Dati due insiemi \(E\) e \(F\), abbiamo \(E \neq F \implies 1_E \neq 1_F\).

La notazione \(y^x\) indica \(\{ f\colon x \to y \}\), cioè tutte le funzioni da \(x\) a \(y\).

La funzione \(\Xi \colon \mathcal{P}(X) \to {\{ 0,1 \}}^X\)
è biettiva. La funzione \(f\colon X \to \{0,1\}\) è pari a \(f=1_E\) per
\(E = \{ x \,|\, f(x)=1 \}\).
Una funzione che ti dice \(1\) se l'elemento sta nel sottoinsieme, 0 altrimenti.
Quindi \[ |\mathcal{P}(X)| = |{\{0,1\}}^X| = 2^n \]

\section{Altre proprietà}

\[
    \sum_{k=0}^n \binom{n}{k}\cdot {(-1)}^k = 0
\]
Questa è la somma dei sottoinsiemi con un numero pari di elementi meno quelli con un numero dispari.

\pagebreak

\section{Interi relativi}

In \(\mathbb{N}\) è definita la funzione \(+\colon {\mathbb{N}}^2 \to \mathbb{N}\)
dove \((m,n) \to m+n\).

Abbiamo chiaramente che \((a,b) = (a', b') \iff a=a' \land b=b'\).

Le prorpietà sono:
\begin{itemize}
    \item è associativa;
    \item è distributiva;
    \item esiste un elemento neutro \(0\) tale che \(m+0=m, \forall m \in \mathbb{N}\)
\end{itemize}

Tuttavia, \(m-n\) è definito solo per \(m \geq n\).
%Per costruire gli interi \(\mathbb{Z}\) da \(\mathbb{N}\) si definisce la sottrazione prendendo
%le coppie \((m,n) ~ (m',n')\)
%se e solo se \(m+n' = m' + n\). Allora \(m-n = m'-n'\).
%Abbiamo quindi, per esempio, che \(-1 \equiv (0,1)\).

Definiamo \(\mathbb{Z}\) come l'insieme 
\[
    \mathbb{Z} = \{ 0, \pm 1, \pm 2, \pm 3, \cdots \}
\]
Abbiamo allora \(\forall n \in \mathbb{Z}, \exists_{=1} n'=-n \,|\, n+(-n) = 0\),
e quindi
\[
    n-m \triangleq n + (-m)
\]
Abbiamo quindi la somma \(+\colon {\mathbb{Z}}^2 \to \mathbb{Z}\)
che gode di tutte le proprietà precedenti ma in più 
\[
    \forall n \in \mathbb{Z}, \exists -n \,|\, n+(-n) = 0
\]

\sdefinition{Gruppo}{
    Un insieme \(G\) con un operazione binaria \(\circ\)
    tale che
    \begin{itemize}
        \item \textbf{associativa:} \((a\circ b)\circ c = a\circ (b\circ c)\)
        \item \textbf{elemento neutro:} \(\forall a \in A, \exists 0 \in G \,|\, 0\circ a=a\circ 0 = a\)
        \item \textbf{elemento opposto:} \(\forall a \in G, \exists a' \,|\, a+a' = a'\circ a=0\)
    \end{itemize}
}

Se aggiungiamo la commutatività viene detto \textit{gruppo abeliano}.

Per esempio \((\mathbb{Z}, +)\) è un gruppo abeliano.

La struttura algebrica \((\mathbb{Z}, \circ)\) dove \((a,b) \to a\cdot b\) non è un gruppo abeliano,
in quanto non c'è un inverso \(n^{-1}\) (c'è solamente per \(1\) e \(-1\)).
La divisione si può fare solo se uno è un multiplo dell'altro.

TODO: definizione di anello

Per definire gli inversi di tutti i numeri \(\neq 0\), si introducono le frazioni \(\frac{m}{n}\)
con \(m\in\mathbb{Z}\) e \(n\in{\mathbb{N}}^+\).

Si dice che due frazioni sono equivalenti \(\frac{m'}{n'}\) e \(\frac{m}{n}\)
se \(mn' = m'n\).
I numeri razionali sono descritti dalle frazioni quando si identificano
con frazioni equivalenti (classe di equivalenza), e le operazioni vengono fatte sulle frazioni.
La classe di equivalenza è quindi data relazione \(\frac{m}{n} \sim \frac{m'}{n'} \iff mn'=m'n\).

Abbiamo che
\[
    \frac{m}{n} \cdot \frac{p}{q} \to \frac{mq+pn}{nq}
\]

Risulta che i razionali \(\mathbb{Q}\) con le operazioni \(+\) e \(\cdot\) introdotte.
Quindi \((\mathbb{Q}, +)\) è un gruppo abeliano, \(({\mathbb{Q}}^*, \cdot)\) è anch'esso un gruppo abeliano
(da notare l'assenza dello 0).

Vale la proprietà distributiva di prodotto rispetto alla somma
\[ r\cdot (s+t) = r\cdot s + r\cdot t \]

Quindi \((\mathbb{Q}, +, \cdot)\) è un campo, per cui possiede le operazioni \(+\) e \(\cdot\)
con le prorpietà alle quali siamo abituati.

\pagebreak

In particolare, in \(\mathbb{Q}\) si possono risolvere le equazioni di primo grado.
\[
    ax+b=0
\]
con \(a,b,x\in\mathbb{Q}\), \(x\neq 0\).
\begin{align*}
    ax+b+(-b)&=-b \\
    ax&= -b \\
    a^{-1}(ax) &= -a^{-1}b \\
    a^{-1}ax &= -a^{-1}b \\
    x &= -\frac{b}{a} 
\end{align*}

Il campo di \(\mathbb{Q}\) ha un ordinamento totale dove \(r \leq s\) se e solo se \(r - s\) è
non-negativa.

In \(\mathbb{Q}\) è definito un ordinamento che è compatibile ocn le operazioni \(+\) e \(\cdot\),
cioè soddisfa le condizioni
\begin{align*}
    r \leq s \implies t + r \leq t + s \\
\end{align*}
con \(t\in \mathbb{Q}\) e con \(t \geq 0\) abbiamo \(tr \leq ts\).

\sdefinition{Campo ordinato}{
    Un campo \(F\) nel quale è definito un ordinamento per il quale valgono le proprietà
    appena date, viene detto \textit{ordinato}.
}

Non tutte le equazioni in \(\mathbb{Q}\) sono risolvibili.

\stheorem{Radice di due}{
    L'equazione \[ x^2 = 2 \] non ha soluzioni in \(\mathbb{Q}\).
}

\sproof{Radice di due}{
    Supponiamo che esista una frazione ridotta ai minimi termini \(r = \frac{m}{n}\),
    tale che \(r^2 = 2\). Abbiamo quindi che \(\frac{m^2}{n^2} = 2\), quindi \(m^2 = 2n^2\).
    Ciô ci dice che \(m^2\) è pari. Allora, \(2\) è un fattore anche di \(m\) (siccome la fattorizazzione è unica e non cambia),
    quindi \(m\) è pari. Di conseguenza, se \(m\) è divisibile per \(2\), allora \(m^2\) è divisibile per \(4\).
    Abbiamo quindi \(4k=n^2\) e quindi \(n^2\) è divisibile per \(2\), anche \(n\),
    contro l'ipotesi del fatto che i due numeri fossero coprimi.
}

%%%%%%% ========0%

\pagebreak 

\section{Definizioni con ordini}

\sdefinition{Insieme totalmente ordinato}{
    Un \textit{insieme ordinato} è una tupla \((X, \leq)\)
    dove \(X\) è un insieme e \(\leq\) è un ordinamento totale.
}

Sia anche \(E \subseteq X\) un insieme dove \(E \neq \emptyset\).

Si dice che \(m\in X\) è \textit{maggiorante} di \(E\) se \(\forall x \in E, x \leq m\). \\
Se un tale valore esiste, \(E\) si dice \textit{superiormente limitato}.

Si dice che \(m\in X\) è \textit{minorante} di \(E\) se \(\forall x \in E, x \geq m\). \\
Se un tale valore esiste, \(E\) si dice \textit{inferiormente limitato}.

L'insieme \(E\) si dice \textit{limitato} se è limitato sia inferiormente che superiormente.

Un valore \(m\in X\) si dice \textit{massimo} di \(E\) se \(M\) è un maggiorante di \(E\) e \(m\in E\). \\
Un valore \(m\in X\) si dice \textit{minimo} di \(E\) se \(M\) è un minorante di \(E\) e \(m\in E\).

\subsection{Considerazioni}

Nel caso in cui l'insieme \(E\) sia finito, vi è un massimo ed un minimo.
Tuttavia, in caso contrario, valori massimi e minimi non esistono necessariamente.

Consideriamo per esempio \(X=\mathbb{Q}\) ed
\[
    E = \left\{ r_n = \frac{n-1}{n}, \quad n\in{\mathbb{N}}^* \right\}
\]
Possiamo notare che il valore \(0\) è il minimo di \(E\).
Vi sono diversi minoranti di \(E\), come \(-1\), \(-30\) etc. In generale, tutti i \(x\leq 0\) sono
dei minoranti di \(E\).
I maggioranti di \(E\) sono tutti i valori \( x \geq 1\).

Tuttavia, non vi è un massimo. Per dimostrarlo prendiamo \(r_n \in E\).
È facile vedere che \(r_n\) non può essere maggiorante in quando se \(n'>n\), \(r_{n'}>r_n\).
Dato qualsiasi \(r_n\), è possibile trovare un altro elemento in \(E\) che è maggiore, e per cui non
esistono maggioranti.

Notiamo che il numero \(1\), che è il maggiorante, è infatti il più piccolo dei maggioranti:
supponiamo che \(z<1\), verifichiamo quindi che \(z\) non è un maggiorante.
Il valore \(z\) non è maggiorante di \(E\) se esiste una \(x \in E\) tale che \(x > z\).
Esiste infatti \(n\) tale che \(r_n > z\), studiamo quindi la disequazione
\begin{align*}
    r_n - z = 1 - \frac{1}{n} - z = (1-z) - \frac{1}{n} > 0
\end{align*}
purché \(1-z>1\). Qualcunque numero più piccolo di \(z\) sia dato, si possono fare
altri valori maggiori, dati quindi da
\[
    n > \frac{1}{1-z}
\]

\pagebreak

\subsection{Estremi superiori e inferiori}

\sdefinition{Estremo superiore}{
    Sia \(E \subseteq X\) un sottoinsieme non-vuoto, diciamo che \(\mu\) è
    l'\textit{estremo superiore} di \(E\) se \(\mu\) è un maggiorante di \(E\) e
    \(\mu\) è il più piccolo del maggioranti.
    Scriviamo quindi
    \[
        \mu = \sup E
    \]
}

\sdefinition{Estremo inferiore}{
    Sia \(E \subseteq X\) un sottoinsieme non-vuoto, diciamo che \(\mu\) è
    l'\textit{estremo inferiore} di \(E\) se \(\mu\) è un minorante di \(E\) e
    \(\mu\) è il più grande del minoranti.
    Scriviamo quindi
    \[
        \mu = \inf E
    \]
}

I valori di minimo, massimo, estremo inferiore, estremo superiore, sono unici se esistono.
Ci sono sottoinsiemi di \(\mathbb{Q}\) che non hanno estremi superiori (e quindi ci sono tante funzioni senza limiti, derivate e integrali.
L'analisi in \(\mathbb{Q}\) sarebbe quindi un disastro per questo motivo).

\stheorem{}{
    Sia
    \[
        E = \left\{ r \in \mathbb{Q} \,|\, r \geq 0 \land r^2 \leq 2 \right\}
    \]
    allora, \(E\) è non-vuoto, limitato superiormente, ma non esiste il suo estremo superiore.
}

\sproof{}{
    \begin{itemize}
        \item Per dimostrare che \(E \neq \emptyset\) possiamo semplicemente darne un elemento, come
        per esempio \(1\).
        \item L'insieme \(E\) è banalmente limitato superiormente
            da tutti i valori \(x \geq 2\). 
        \item Supponiamo per assurdo che esista un \(\mu = \sup E\). Notiamo che ovviamente \(\mu >0\).
        Possiamo notare che \(\mu^2 = 2\) è impossibile per il teorema di Euclide. Allora, \(\mu\)
        potrebbe essere minore di 2 oppure maggiore di 2.
        Supponiamo che \(\mu^2 < 2\), allora dimostro che \(\exists x \in E\) tale che \(x > \mu\)
        e quindi che \(\mu\) non è maggiorante.
        Consideriamo quindi i numeri razionali della forma
        \[
            \mu + \frac{1}{n}
        \]
        che sono chiaramente più grandi di \(\mu\).
        Possiamo quindi scegliere \(n\) sufficientemente grande tale che \({(\mu + \frac{1}{n})}^2 < 2\),
        e quindi \(\mu + \frac{1}{n} \in E\) in quanto
        \begin{align*}
            2 - {\left(\mu + \frac{1}{n}\right)}^2 &= 2 - \mu^2 + \frac{2\mu}{n} + \frac{1}{n^2} \\
            &= (2-\mu^2) - \frac{2\mu}{n} - \frac{1}{n^2}
        \end{align*}
        è chiaramente più grande di \((2-\mu^2) - \frac{2\mu}{n} - \frac{1}{n}\).
        Ciò è dato dal fatto che \(\frac{1}{n} > \frac{1}{n^2}\).
        \[
            \frac{2\mu + 1}{n} < 2-\mu^2, \quad n > \frac{2-\mu^2}{2\mu + 1}
        \]
        Analogamente, si dimostra che \(\mu^2\) non può essere nemmeno maggiore di \(2\),
        e quindi \(\mu\) non esiste.
    \end{itemize}
}

\pagebreak

È facile verificare che inf, sup, min, max se esistono sono unici.
Se esiste il massimo di \(E\), allora esiste il \(\sup E\) e coincidono.
Infatti, il massimo esiste se esiste \(\sup E\) e \(\sup E \in E\).

In \(\mathbb{Q}\) (e poi in \(\mathbb{R}\)), se \(E\) non è limitato superiormente (cioè non ha maggiornate
cioè \(\forall M \in \mathbb{Q}, \exists e \in E\) tale che \(e > M\)) si dice che
\[
    \sup E = +\infty
\]
Analogamente se \(E\) non è limitato inferiormente si dice che
\[
    \inf E = -\infty
\]

Possiamo quindi notare che
\[
    \sup \emptyset = -\infty
\]
e
\[
    \inf \emptyset = +\infty
\]

\sdefinition{Numeri reali}{
    Definiamo \(\mathbb{R}\) come un campo totalmente ordinato nel quale
    vale la seguente proprietà del \(\sup\):
    \[
        \forall E \subseteq \mathbb{R}, \quad E \neq \emptyset \land E \text{ limitato sup. esiste}
    \]
}

Bisogna tuttavia dimostrare l'unicità di questa costruzione e la sua esistenza.

\stheorem{Teorema di unicità}{
    Siano \(F_1\) e \(F_2\) due campi ordinati nei quali
    vale la proprietà del sup di prima. Allora, esiste una biezione
    \(\phi\colon F_1 \to F_2\) tale che è un isomorfismo del gruppo additivo
    \(\phi(x+_{F_1}y) = \phi(x) +_{F_2} \phi(y)\) per ogni \(x,y\in F_1\)
    e \(\phi(-x) = -\phi(x)\) per ogni \(x\in F_1\).
    % che si possono scrivere anche in un modo solo.
    Se aggiungiamo anche che \(\phi(x \cdot_{F_1} y) = \phi(x) \cdot_{F_2} \phi(y)\)
    per tutte le \(x,y \in F_1\) e \(\phi(x^{-1}) = {\phi(x)}^{-1}\)
    abbiamo un isomorfismo di campo.
    Se aggiungiamo anche che \(x \leq y \iff \phi(x) \leq \phi(y)\), abbiamo quindi un isomorfismo
    di campo ordinato.
}

Date le proprietà di un campo, ogni campo genera un insieme dei razionali \(\mathbb{Q}\).
Chiaramente, diversi campi generano \(\mathbb{Q}\) diversi ma con gli stessi elementi in un certo senso.
Possiamo mappare un insieme dei razionali di un campo a quello di un altro.

È facile definire \(\phi_0\colon \mathbb{Q}_1 \subseteq F_1 \to \mathbb{Q}_2 \subseteq F_2\).
Usando la proprietà del sup possiamo eseguire tale mappatura.
Dato \(x\in F_1\), abbiamo \(x = \sup \{ r \in \mathbb{Q}_1 \,|\, r \leq x \} = \sup E_x\).
Allora \(\phi (x) = \sup \{ \phi_0(r) \,|\, r \in E_x \}\).
Così viene esteso \(\phi\) a tutto. Bisognerebbe tuttavia dimostrare che le proprietà classiche vengano
preservate.

Per dimostrare l'esistenza è necessario considerare
\[\mathbb{R} = \{ \text{ numeri decimali } n, a_1, a_2, a_3, \cdots \} \text{ finiti o infiniti periodici o meno }\]
dove \(a_k \in \{0,1,2,3,4,5,6,7,8,9\}\).

Con la prescrizione che \(n,a_1, a_2, a_3, \cdots, a_k, \overline{9} = n, a_1, \cdots, a_{k-1}, (a_k+1)\).

I numeri reali possono essere anche definiti mediante le sezioni di Dedekind.
Alternativamente si possono definire mediante le successioni di Cauchy.

\paragraph{Definizione di somma e prodotto:}

Prendiamo \(x = n,a_1,\cdots,a_k\cdots\) e \(y = m,b_1,\cdots,b_k\cdots\)
che sono due numeri decimali, nessuno dei quali con period \(9\),
allora
\[
    x=y \iff n=m \land a_k=b_k
\]
e
\[
    x<y \iff n < m \lor (n=m \land a_i = b_i, i < k \land a_k < b_k)
\]

Le operazioni sono definite mediante troncamenti.
Verificiamo che questo modello di \(\mathbb{R}\) soddisfi la proprietà del sup.

Prendiamo quindi  \(E\subseteq \mathbb{R}\) non vuoto e sup limitato. Costruiamo
il sup mediante un algoritmo.
\[
    \sup E = \mu = n,a_1,a_2,a_3,\cdots,a_k,\cdot
\]
Per ogni \(x\in E\) scriveremo \(n_x,a(x)_1, a(x)_2,\cdots\).
\(E\) è non-vuoto e limitato sup, per cui
\[
    \{ n_x \,|\, x\in E \}
\]
è un insieme di numeri in \(\mathbb{Z}\) limitato superiormente.
Sia
\[
    N = \max\{ n_x \colon x \in E \}
\]
Prendiamo ora tutti gli insiemi
\[
    E_0 = \{ x\in E \,|\, n_x = N \} \neq \emptyset
\]
Poniamo \(a_1 = \max \{ a(x)_1 \,|\, x \in E_0 \}\)
Abbiamo quindi
\[
    E_1 = \{ x \in E_0 \,|\, a(x)_1 = a_1 \} \neq \emptyset
\]
Poniamo ora \(a_2 = \max \{ a(x)_2 \,|\, x \in E_1 \}\).
Con lo stesso metodo troviamo \(a_3, a_4, \cdots\), ossia
\[
    a_k = \max \{ a(x)_k \,|\, x \in E_{k-1} \} \quad a_{k+1} = \max\{ a(x)_{k+1} \,|\, x \in E_k \}
\]
Trovando quindi
\[
    \mu = N,a_1,a_2,\cdots
\]

Dico che \(\mu\) è un maggiorante di \(E\), e che se \(z<\mu\), \(z\) non è maggiorante.
Sia allora \(\overline{x} \in E\), quindi
\[
    \overline{x} = n_{\overline{x}}, a(\overline{x})_1, a(\overline{x})_2, \cdots
\]
Allora \(n_{\overline{x}}\leq N\) se \(n_{\overline{x}} < N\). \(\overline{x} < \mu\).
Gli elementi in \(E_0\) sono al massimo \(a_1\).
Se \(n_{\overline{x}} = N\) e \(n_{\overline{x}} \in E_0\) e \(a_1(\overline{x}) = a_1\). \\
Se \(a(\overline{x})_1 < a_1 \implies \overline{x} < \mu\).\\
Se invece \(a(\overline{x})_1=a_1 \implies \overline{x} \in E_1\) e \(a(\overline{x})_2 \leq a_2\) \\
Fino che ad un certo punto non trovo un decimale diverso.

Iterando, se \(\exists k\) tale che \(a(\overline{x})_k < a_k \implies \overline{x} < \mu\).
Se \(\forall k, a(\overline{x})_k = a_k\), allora \(\overline{x} = \mu\) e \(\mu\) è il max di \(E\).
Questo procedimento non dimostra che \(\mu \in E\).

Mostriamo ora che è il più piccolo dei maggioranti. Sia
\[
    z = n_z, a(z)_1, a(z)_2, \cdots < \mu
\]
Deve quindi succedere che o \(n_z < N\), e allora \(\forall x\in E_0 \neq \emptyset, z<x\),
oppure \(n_z = N\) e \(a(z)_j = a_j\) per tutte le \(j<k\) ma \(a(z)_k < a_k\).
Allora \(\mu = \sup E\).

\subsection{Conseguenze della proprietà del sup}

Le conseguenze della prorpietà del sup sono:
\begin{itemize}
    \item \textbf{proprietà archimedea:} \(\forall x \in \mathbb{R}, \forall a > 0, \exists n\in\mathbb{N} \,|\, na>x\) (in realtà vale anche in \(\mathbb{Q}\)).
    \item \textbf{densità dei razionali nel reali:} \(\forall x,y \in \mathbb{R}\) dove \(x<y\), esiste \(r\in\mathbb{Q}\,|\, x<r<y\).
\end{itemize}

\stheorem{Esistenza delle radici nei reali}{
    \[
        \forall y>0, \forall n \in \mathbb{N}, n \geq 1,
        \exists_{=1} x>0 \,|\, x^n = y
    \]
}

\sproof{}{
    Sia \[ E = \{ z \in \mathbb{R} \,|\, z>0 \land z^n \leq y \} \]
    Dobbiamo quindi mostrare che \(E\) non è vuoto, ed è limitato superiormente.
    Definiamo \(x = \sup E\) e mostriamo che \(x^n = y\).
    \begin{itemize}
        \item \textbf{Non vuoto:} se \(y \geq 1\), basta scegliere \(x=1\) in quanto \(x^n=1 \leq y\).
        Altrimenti, se \(y < 1\), poniamo \(x=y\) e notiamo che, perché \(y < 1\), allora
        \(y^n < y\), e quindi \(y \in E\).
        \item \textbf{Limitato superiormente:} \(E\) è limitato superiormente, infatti \(1+y\)
        è un maggiorante di \(E\). Se \(z \geq (1+y)\), poiché la funzione \(t\to t^2\) è crescente per \(t > 0\),
        si ha \(z^n \geq {(1+y)}^n > {(1+y)} > y \implies z \notin E\).
        Sia \(x =\sup E\). Dico che \(x^n = y\). Dimostro che se suppongo \(x^n > y\)
        allora per \(k\) grande
        \[
            {\left(x- \frac{1}{k}\right)}^n > y
        \]
        e quindi \(x-\frac{1}{k}\) è ancora un maggiorante di \(E\), contro l'ipotesi impossibile
        perché \(x\), che è il \(\sup E\), è il più piccolo maggiorante.
        Invece, se \(x^n < y\) allora per \(k\) grande
        \[
            {\left(x + \frac{1}{k}\right)}^n < y
        \]
        allora \(x+\frac{1}{k}\in E\) ed è più grande di \(x\), e \(x\) non è quindi un maggiorante (assurdo).
        Visto che \(x\) non può essere nè più grande nè più piccolo, \(x^n=y\).
        \item \textbf{Unicità:} notiamo che se \(0 < t_1 < t_2 \implies t_1^n < t_2^n\).
    \end{itemize}
}

Possiamo anche mettere \(z\geq 0\) così dimostrare che \(E\neq \emptyset\) è più facile.

Esercizio: dimostrazione per induzione che \(0<y<1 \implies y^n < y\), per \(n > 1\).
(Che abbiamo usato nell'ultima dimostrazione).

\subsection{Esercizi sup}

\sexercise{}{
    Let
    \[
        E = \left\{ x\in\mathbb{R} \,|\, \frac{1}{2} \leq x < 5 \right\}
    \]
    and the sequence \[
        F = \{ x = x_n \,|\, x_n = \frac{n + 1}{n + 2}, \quad n\in\mathbb{N}^* \}
    \]
    %%%%%%%
    Trova inf, sup, min, max (se esistono) di \(E\), \(F\), \(E \cup F\) e \(E\cap F\).
    \begin{itemize}
        \item \(E\) è limitato superiormente e inferiormente.
        Il minimo è \(\frac{1}{2}\), mentre \(5\) è un maggiorante, è il più piccolo
        dei maggioranti quindi \(\sup E = 5\), ma non vi è un massimo.
        \item \(F\) è limitato superiormente in quanto 
        \[
            x_n = \frac{n+1}{n+2} < \frac{n+2}{n+2}  =1
        \]
        È limitato inferiormente perché \(x_n > 0\).
        Per verificare sup e inf, è comodo riscrivere \[
            x_n = 1 - \frac{1}{n+2}
        \]
        Il temrine \(n+2\) cresce con \(n\), quindi \(\frac{1}{n+2}\) decresce al crescere di \(n\)
        e quindi \(x_n\) cresce approcciando \(1\). Allora con \(n=1\) il termine assume il valore più piccolo,
        ossia \(\frac{2}{3}\), quindi il minimo di \(F\). Allora siccome ci avviciamo arbitrariamente a
        \(1\), è lecito ipotizzare \(\sup F = 1\).
        Il massimo di \(F\) non esiste.
        Rimane da far vedere che se \(z < 1\) allora \(z\) non è maggiorante di \(F\)
        cioè
        \[
            x_n - z = (1-z) - \frac{1}{n+2} > 0
        \]
        purché \(\frac{1}{n+2} < 1-z\) cioè \(n > \frac{1}{1-z}-2\).
        Quindi \(z\) non è maggiorante e \(\sup E = 1\).
        \item Verificare che \(\sup (E \cup F) = \max \{ \sup E,\sup F \}\).
        Abbiamo che \(\sup E \leq \sup F\).
        In sup è il massimo dei due in quanto uno è maggiore dell'altro,
        e fa parte dell'insieme, quindi \(\sup E \cup F = 5\).
        Tuttavia, il max non esiste in quando \(5\notin  E \cup F\).
        Analogamente, \(\inf E \cup F = \frac{1}{2}\). Questo valore è anchde il minimo
        in quanto fa parte dell'insieme.
        \item Mostrare con un esempio che non c'è qualcosa di analogo per l'intersezione.
        \[
            E \cap F = \left\{ x_n = \frac{x+1}{x+2} \ \middle|\ \frac{1}{2} \leq \frac{x+1}{x+2} \leq 5 \right\}
        \]
        Quindi \(F \subseteq E\). Consideriamo allora \(E_1 = [\frac{4}{5}, 5)\)
        \[
            E_1 \cap F = \left\{ x_n = \frac{x+1}{x+2} \ \middle|\ \frac{4}{5} \leq x_n \leq 5 \right\}
        \]
        Per quali \(n\) vale che \(\frac{4}{5} \leq \frac{x+1}{x+2} = x_n\)?
        Abbiamo \(4(n+2) \leq 5(n+1)\) e quindi \(n \geq 3\).
        Allora \(\sup E_1 \cap F = 1\) e non vi è massimo, mentre \(\inf E_1 \cap F = \frac{4}{5}\)
        che è anche il minimo.
        \item Posto \(E+F = \{ x + y \,|\, x \in E, y \in F \}\)
        mostrare \(\sup E + F = \sup E + \sup F\). Supponiamo quindi che \(\sup E\) e \(\sup F\)
        siano finiti. Siccome, per definizione, \(\forall e \in E, e \leq \sup E\)
        e \(\forall f \in F, f \leq \sup F\), abbiamo che \[\forall e \in E, \forall f\in F, e + f \leq \sup E + \sup F\]
        Per mostrare che questo è il più piccolo dei maggioranti, è comodo riscrivere la definizione di
        sup dicendo che \(\mu\) è pari a \(\sup E\) se:
        \begin{enumerate}
            \item \(\forall x \in E, x\leq \mu\);
            \item \(\forall \epsilon > 0, \mu - \epsilon\) non è maggiorante.
        \end{enumerate}
        \textbf{Nota:} se \(x < \mu\) allora posto \(\epsilon = \mu - x\) risulta \(x = \mu - \epsilon\).
        Allora sia \(\epsilon > 0\). Diciamo che esistono \(e_1\in E\) e \(f_1\in F\) tali che
        \(e_1 + f_1 > \sup E + \sup F - \epsilon\).
        Poiché \(\sup E\) è, appunto, il supremum, esiste per definizione una \(e_1 \in E\) tale che
        \(e_1 > e_1 > \sup E . \frac{\epsilon}{2}\).
        Analogamente, esiste \(f_1 \in F\) tale che \(f_1 > \sup F - \frac{\epsilon}{2}\).
        Da cui \(e_1 + f_2 > \sup E - \frac{\epsilon}{2} + \sup F - \frac{\epsilon}{2} = \sup E + \sup F - \epsilon\).
        \item Posto \(-E = \{ -x \,|\, x \in E\}\) mostrare che
        \(\sup -E = -\inf E\) e \(\inf -E = -\sup E\).
    \end{itemize}
}

Dimostrare che il max esiste se e solo se \(\sup E\) è finito e appartiene a \(E\).
Analogamente per il min.

\pagebreak

\sexercise{}{
    Trovare sup, inf, min, max dell'insieme
    \[
        E = \left\{ x_n = \frac{n-7}{x^2 + 1} \ \middle|\ n \geq 1 \right\}
    \]
    Questa successione ha sicuramente un minimo in quanto ci sono solamente \(6\) numeri negativi.
    Possiamo notare che il denominatore cresce più velocemente del numeratore.

    Studiamo quindi per quali indici vale \(x_n \leq x_{n+1}\). Otteniamo quindi
    \begin{align*}
        \frac{n-7}{n^2 + 1} &\leq \frac{(n+1)-7}{{(n+1)}^2 + 1} \\
        \frac{(n-7)(n^2 + 2n + 2) - (n-6)(n^2+1)}{(n^2 + 1)(n^2 + 2n + 2)} &\leq 0
    \end{align*}
    Il denominatore è positivo, quindi studiamo il numeratore
    \begin{align*}
        n^2 - 13n - 8 \leq 0
    \end{align*}
    Le radici di questo polinomio sono \(n_{1,2}= \frac{13\pm\sqrt{201}}{2}\).
    Di conseguenza, l'espressione è negativa per \(\frac{13-\sqrt{201}}{2} < n < \frac{13+\sqrt{201}}{2}\).
    Notiamo che l'estremo di sinistra è negativo. Notiamo anche che \(14^2 < 201 < 15^2\),
    e quindi l'estremo di destra è compreso fra \(14\) e \(\frac{27}{2}\).
    Allora, tutte le \(n\) intere che soddisfano l'equazione sono
    \(n=13\). Ne consegue che se \(n \geq 14\), \(x_n > x_{n+1}\).
    Il maggiornate e supremum è quindi \(x_{14}\).

}

\pagebreak

\section{Esponenziali}

\subsection{Potenze ad esponente reale e esponziali e logaritmi}

Abbiamo definito le radici n-esime come
\[
    x^\frac{m}{n} \triangleq \sqrt[n]{x^m}
\]
Si dimostra inoltre che per ogni \(p\) intero positivo,
\[
    x^{\frac{x\cdot p}{n\cdot p}} = x^\frac{m}{n}
\]
La potenza \(x^r\) è quindi ben definita con \(r\in\mathbb{Q}^{> 0}\).
Successivamente, definiamo le potenze negative
\[
    x^{-r} = {(x^-1)}^r
\]
Abbiamo le consuete proprietà:
\begin{enumerate}
    \item \(\forall x>0, x^0 = 1\);
    \item \(\forall r,s\in\mathbb{Q}, x^rx^s = x^{r+s}\);
    \item \(\forall r,s\in\mathbb{Q}, {(x^r)}^s = x^{rs}\);
\end{enumerate}

Con \(r>0\) posso definire \(0^r = 0\)
e se \(r=\frac{m}{n}\) (ridotta ai minimi termini) con \(n\) dispari
posso definire \(x^\frac{m}{n}\) se \(x<0\).

\subsection{Potenze a esponente reale}

Se \(x=1\), \(\forall a\in \mathbb{R}, x^a = 1\).
Se \(x>1\) e \(r<s\), allora \(x^r < x^s\)
\[
    r = \frac{m}{p} < s = \frac{n}{p}, m < n
\]
\[
    x^r = {(\sqrt[p]{x})}^m < {(\sqrt[p]{x})}^n
\]

Definiamo quindi la potenza reale con \(a>1\) e \(x>1\)
\[
    x^a = \sup \{ x^r \,|\, r \leq a \}
\]
Estendiamo la definizione ad \(a < 0\) come
\[
    x^a = {(x^{-1})}^{-a}
\]
E infinie se \(0<x<1\)
\[
    x^a = {(x^{-1})}^{-a}
\]

\pagebreak

\subsection{Esponenziali}

Fissata una base \(a>0\) abbiamo poi l'esponenziale che è definita da \(a^x\),
\(x\in\mathbb{R}\).

Risulta che se \(a=1\), allora la funzione è sempre \(1\).
Se \(a>1\) la funzione è stretta crescente, e strettamente descrescente se \(0<a<1\).

La funzione è biettiva tra \(\mathbb{R}\) e \((0, +\infty)\), quindi è invertibile.
La funzione inversa è \(y=\log_a(x)\).

Le proprietà dei logaritmi sono analoghe a quelle degli esponenti.
\sproposition{Proprietà dei logaritmi} {
    \[
        \log_a(xy) = \log_a(x) + \log_a(y)
    \]
    \[
        \log_a(x^y) = y\log_a(x)
    \]
    \[
        \log_a(b) = \frac{\log_c(a)}{\log_c(b)}
    \]
}

% Come trovare la terza


Il passaggio da moltiplicazione e somma di logaritmi, potrebbe non avere senso
nella seconda forma.
E.g \(\ln(x(x-1))\) non is può riscrivere come \(\ln(x) + \ln(x-1)\)
perché, se sono positivi quando moltiplicati, non è detto che lo siano separatamente.

Se abbiamo \(\log_2(x^2)\), possiamo riscriverlo come \(2 \log_2|x|\).

% Per negare una proposizione si inverte \forall con \exists e si nega il predicato

\subsection{Assioma di continuità}

\[
    \bigcup_{n\neq 2} \left[\frac{1}{n}; 1-\frac{1}{n}\right] = (0,1)
\]

\saxiom{Assioma di Dedekind}{
    Se \(\{I_n\}\) è una successione di intervalli chiusi della forma
    \(I_n = [a_n; b_n]\) tali che \(I_{n+1} \subseteq I_n\) e
    \[
        l(I_{n+1}) = b_{n+1}-a_{n+1} = \frac{1}{2}l(I_n)
    \]
    e quindi \[l(I_n) = \frac{1}{2^{n-1}}l(I_1)\] allora esiste \(c\in \mathbb{R}\)
    tale che
    \[
        \bigcap_{n\in\mathbb{N}} I_n = \{c\}
    \]
}

Questo assioma non vale nei razionali.

\stheorem{Assioma di Dedekind equivalenza assioma di completezza} {
    L'assioma di Dedekind è equivalente all'assioma della completezza.
}

\pagebreak

\sproof{Assioma di continuità equivalenza assioma di Dedeking}{
    \iffproof{
        Sia \(E\) un insieme non vuoto e limitato superiormente, dimostriamo che
        esiste \(c=\sup E\).
        Poiché \(E\) è limitato superiormente esiste un \(b_1\) maggiorante di \(E\)
        dove \(b_1 \geq e, \forall e \in E\).
        Poiché \(E\) è non vuoto esiste \(\overline{e}\in E\) e
        poniamo \(a_1 = \overline{e} - 1\) cosicché \(a_1 < \overline{e}\)
        e \(a_1\) non è maggiornate. Sia \(I_1 = [a_1, b_1]\) e sia \(m_1=\frac{a_1 + b_1}{2}\),
        allora vi sono due casi:
        \begin{itemize}
            \item \(m_1\) è un maggiorante e allora poniamo \(a_2 = a_1\) e \(b_2 = m_1\);
            \item \(m_1\) non è un maggiorante e allora poniamo \(a_2 = m_1\) e \(b_2 = b_1\).
        \end{itemize}
        Sia \(I_2 = [a_2, b_2]\). Iteriamo allora il procedimento
        otteniamo una successione di intervalli
        \[
            I_n = [a_n, b_n]
        \]
        tali che \(I_{n+1} \subseteq I_n\) e \(l(I_{n+1}) = \frac{1}{2}l(I_n)\).
        Per ogni \(n\), \(a_n\) non è maggiorante di \(E\), \(b_n\) è maggiorante di \(E\).
        Per l'assioma di continuità \(\exists c\in \mathbb{R}\) tale che
        \[
            \bigcap_{n\in\mathbb{N}} I_n = \{c\}
        \]

        La nostra tesi è quindi \(c = \sup E\).
        Supponiamo quindi per assurdo che non sia un maggiorante, allora che esiste un elemento
        \(e\in E\) dove \(e > c\).
        Per definizione \(c\) è in almeno un \(I_n\) quindi \(a_n \leq c \leq b_n\)
        e poiché \(e > c\) abbiamo \(a_1 \leq c < e \leq b_n\) perché \(e\in E\)
        e \(b_n\) è maggiorante di \(E\) per ogni \(n\).
        Quindi \[
            [c, e] \subseteq [a_n, b_n]
        \]
        e allora
        \[
            [c, e] \in \bigcap_{n\in\mathbb{N}} I_n = \{c\}
        \]
        che è quindi una contraddizione.
        Dobbiamo ora mostrare che \(c\) è il più piccolo dei maggioranti.
        Supponiamo quindi per assurdo che ci sia un altro maggiorante
        \(x<c\).
        Poiché \(b_n \geq c\) per ogni \(n\) abbiamo che
        \(x < c \leq b\). Per ipotesi, \(x\) è maggiorante di \(E\)
        mentre \(a_n\) non è maggiorante di \(E\)
        per ogni \(n\). Quindi per tutte le \(n\) esiste un elemento \(e_n\) tale che
        \[ a_n < e_n \leq x < c \leq b_n \]
        e quindi si deduce che l'intervallo \([x,c] \subseteq [a_n, b_n] = I_n\)
        da cui
        \[
            [c, e] \in \bigcap_{n\in\mathbb{N}} I_n = \{c\}
        \]
        che quindi è assurdo.
    }{
        TODO (in the future)
    }
}

\pagebreak

\section{Numeri complessi}

In un campo ordinato e quindi in \(\mathbb{R}\), \(x^2 \geq 0\) e vale \(x^2 = 0 \iff x=0\).
Quindi l'equazione \(x^2 = -1\) non ha soluzione in \(\mathbb{R}\).
Estendiamo il campo \(\mathbb{R}\) costruendo un campo \(\mathbb{C}\)
che contiene una immagine isomorfa di \(\mathbb{R}\) nel quale \(z^2 = -1\)
ha soluzioni.

Tuttavia, tale campo non ammette il medesimo ordinamento che avevamo.
Definiamo quindi
\[
    \mathbb{C} = \mathbb{R} \times \mathbb{R}
\]

Definiamo l'operazione di addizione
\[
    +\colon \mathbb{C} \times \mathbb{C} \to \mathbb{C}
\]
in maniera tale che
\[
    (a,b) + (c,d) \triangleq (a+c, b+d)
\]

\begin{enumerate}
    \item anche questa somma è associativa, e commutativa come in \(\mathbb{R}\);
    \item l'elemento neutro \(0\) è la coppia \(0,0\);
    \item l'opposto di \((a,b)\) è \(-(a,b)\);
\end{enumerate}

Si può rappresentare \(\mathbb{C}\) come punti nel piano.
La moltiplicazione è definita come
\[
    (a,b) \cdot (c,d) \triangleq (ac-db,ad+bc)
\]
Questo prodotto è
\begin{enumerate}
    \item è associativo;
    \item è commutativo;
    \item l'elemento \((1,0)\) è l'elemento neutro;
    \item esiste un elemento inverso
\end{enumerate}

\[
    \forall z=(a,b) \in \mathbb{C} \,|\, (a,b) \neq (0,0),
    \exists z^{-1} = \left(\frac{a}{a^2 + b^2}, \frac{-b}{a^2 + b^2}\right) \,|\,
    zz^{-1} = (1,0)
\]

Per determinare questa forma basta risolvere \(z^{-1} = (x,y)\) dove \((a,b)(x,y) = (1,0)\).

Abbiamo quindi un campo.

Adesso, notiamo che \((0,1)(0,1) = (-1, 0)\).

\subsection{Inclusione dei reali}

Ogni number \(r\in\mathbb{R}\) può essere identificato con il numero complesso \((r, 0)\).
Cosifacendo, l'applicazione \(\varphi\colon \mathbb{R} \to \mathbb{C}\)
tale che \(\varphi(a) = (a,0)\) preserva le operazioni.

Possiamo poi scrivere \(z=(a,b)\) come \(a(1,0) + b(0,1)\).
Se identifichiamo \(i=(0,1)\), possiamo scrivere
\[
    (a,b) = a+bi
\]
che viene detta forma algebrica.
Le operazioni di numeri complessi in forma algebrica si forma con le consuete regole del calcolo letterale
e l'identità \(i^2 = -1\).

\subsection{Operazioni algebriche}

\[
    \begin{cases}
        i^0=+1\\
        i^1=+i\\
        i^2=-1\\
        i^3=-i\\
    \end{cases}
    \quad
    \begin{cases}
        i^4=+1\\
        i^5=+i\\
        i^6=-1\\
        i^7=-i\\
    \end{cases}
    \quad
    \cdots
\]

Dato \(z=a+bi\), diciamo che \(\Re(z) =a\) e \(\Im(z) = b\).

\sdefinition{Coniugio}{
    Dato \(z=a+bi\in\mathbb{Z}\),
    \[
        \overline{z} = a-bi
    \]
}

Chiaramente, \(z + \overline{z} = 2\Re(z)\). Possiamo quindi dire che

\[
    \Re z = \frac{z + \overline{z}}{2}
\]
e
\[
    \Im z = \frac{z-\overline{z}}{2i}
\]

\sproposition{Proprietà del coniugio}{
    \begin{itemize}
        \item \textbf{involutivo:} \(\overline{\overline{z}} = z\);
        \item \(\overline{z+w} = \overline{z} + \overline{w}\);
        \item \(\overline{zw} = \overline{z}\cdot\overline{w}\);
        \item \(w \neq 0 \implies \overline{z^{-1}} = {\left(\overline{z}\right)}^{-1}\);
        \item \(w \neq 0 \implies \overline{\left(\frac{z}{w}\right)} = \frac{\overline{z}}{\overline{w}}\);
        \item \(\overline{z^n} = {(\overline{z})}^n\) per \(n\in\mathbb{Z}\).
    \end{itemize}
}

Per ogni numero complesso \(z\),
\[
    {|z|}^2 = z\overline{z}
\]
e per ogni numero complesso \(w\)
\[
    \overline{wz} = wz\overline{wz} = z\overline{z}w\overline{w} = |z|^2|w|^2
\]

In particolare, \(|z^n| = |z|^n\).

La disuguaglianza \(||z| - |w|| \leq |z-w|\).

\begin{itemize}
    \item \(|wz| = |w|\cdot|z|\);
    \item \(|w+z| \leq |w| + |z|\).
\end{itemize}

% Il modulo è minore o uguale all'abs reale + abs imm

Da dimostrare: \(|z+w|^2 \leq {(|z| + |w|)}^2\).

\subsection{Passaggio polari e cartesiane}

Dato \(x+iy = r(\cos\theta + i\sin\theta)\) e il punto polare \((r, \theta)\)
abbiamo
\[
    r = \sqrt{x^2 + y^2}
\]
e \[
    \theta = \arctan\left(\frac{y}{x}\right)
\]

\subsection{De Moivre}

\[
    z^n = r^n {(\cos \theta + i\sin\theta)}^n = \sum_{k=0}^n \binom{n}{k}
    i^k {(\cos\theta)}^{n-k}{(\sin\theta)}^k
\]

\section{Distanza fra due insiemi}

La distanza (minima) fra due insiemi è definita come
\[
    \text{dist}(S, R) = \inf \{ d(z,w) \,|\, z \in S \land w \in R \}
\]

\section{Teorema di Ruffini}

Dato un polinomio \(p(z)\), \(z_0\) è una radice di \(p(z)\) se esiste
un polinomio \(q(z)\) con \(\deg q(z) = \deg p(z) - 1\)
tale che \[p(z) = (z-z_0)q(z)\], cioè se \(p(z)\) è divisibile per \(z-z_0\).

La radice \(z_0\) ha moltiplicità \(m \geq 1\) se
\(p(z)\) è divisibile per \((z-z_0)^m\) ma non per \((z-z_0)^{m+1}\).

% Esercizio: dimostrare per induzione che l'enunciato del teorema fondamentale dell'algebra
% è equivalente a: Se P(z) è un polinomio di grado n \geq 1, allora P(z) ha almeno una radice.
% => è banale. <= per induzione (per Ruffini)

\pagebreak

\section{Spazi metrici}

\sdefinition{Metrica della valle}{
    La funzione di distanza della metrica della valle fra due punti \(a=(x,y)\) e \(b=(t,u)\)
    è definita come
    \[
        d(a, b) = \begin{cases}
            |y-u| & x = t \\
            |y| + |x-t| + |u| & x \neq t
        \end{cases}
    \]
}

% TODOURGENTE: metterla come esempio e verificare che sia una distanza

\sdefinition{Distanza in un grafo}{
    Distanza in un grafo
    \[
        d(a, b) = \inf \{ \text{lunghezza dei cammini che congiungono a e b} \}
    \]
}

\section{Spazi topologici}

Un punto \(x_0\) è isolato in \(E\) se \(\exists r > 0\) tale che \((x_0 - r, x_0 + r) \cap E = \{x_0\}\).

% TODOURGENT mettere esempio e disegno!!! 0
Esercizio: considera l'insieme
\[
    E = \left\{ x_n = \frac{1}{n} \,|\, n\in {\mathbb{N}}^* \right\} \cup (2,3) \cup \{4\}
\]
\begin{itemize}
    \item Per ogni \(x\in (2,3)\), esiste \(r = \min\{3-x, x-2\}\)
    dove chiaramente \((x-r, x+r) \subseteq E\)
    \item I punti di frontiera sono \(2\), \(3\), \(4\) e tutti i punti della forma \(\frac{1}{n}\) per \(n\in {\mathbb{N}}^*\).
    Anche \(0\) è un punto di frontiera.
    \item I punti isolati sono quelli della forma \(\frac{1}{n}\) con \(n\in {\mathbb{N}}^*\) e \(4\).
    \item I punti esterni sono
    \[
        (1,2) \cup (4,+\infty) \cup (-\infty, 0) \cup \bigcup_{n=1}^\infty \left(\frac{1}{n+1}, \frac{1}{n}\right)
    \]
    \item I punti di accumulazione sono
    \([2,3] \cup \{0\} \)
\end{itemize}

\stheorem{}{
    Sia \(E \subseteq \mathbb{R}\) (vale in qualsiasi spazio metrico)
    e sia \(x_0 \in \mathbb{R}\). Sono equivalenti:
    \begin{enumerate}
        \item \(x_0\) è di accumulazione cioè \(\forall r > 0\), \[\left((x_0-r,x_0+r) \backslash \{x_0\}\right) \cap E \neq \emptyset\]
        \item \(\forall r>0\), \((x_0 -r, x_0 + r) \cap E\) è infinito (ogni intorno contiene infiniti punti di \(E\)).
    \end{enumerate}
}

% Contronominale: A => B \iff not B => not A

\sproof{}{
    \iffproof{
        Dimostriamo la contronominale.
        Assumiamo quindi che \(\exists r > 0\) tale che \(A=(x_0 -r, x_0 + r) \cap E\) è finito,
        e quindi \(A = \{ x_0, x_1, \cdots, x_n \}\) dove \(x_1, x_2, \cdots, x_n\)
        sono gli elementi di \((x_0 + r, x_0 - r) \cap E\) diversi da \(x_0\).
        Chiaramente, esiste un \(0 < \epsilon < \min\{ |x_0 - x_1|, |x_0 - x_2|, \cdots, |x_0 - x_n| \}\).
        Siccome l'insieme è finito, \(\epsilon\) esiste ed è strettamente positivo.
        Quindi, per definizione \(x_0\) non è di accumulazione.
    }{
        Trivial.
    }
}

% TODOURGENT come negare una proposizione.

\end{document}