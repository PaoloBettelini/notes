\documentclass[a4paper]{article}

\usepackage{amsmath}
\usepackage{amssymb}
\usepackage{stellar}
\usepackage{parskip}
\usepackage{fullpage}
\usepackage{wrapfig}
\usepackage{tikz}

\usetikzlibrary{arrows}
\usetikzlibrary{decorations.pathreplacing}
\usetikzlibrary{cd}

\title{Algebra I}
\author{Paolo Bettelini}
\date{}

\begin{document}

\maketitle
\tableofcontents


\section{Esercizi}

%\[
%    \sin(x) \sim x(1 + o(1))
%\]

\subsection{Limiti}

\sexercise{}{
    \begin{align*}
        \lim_{x\to 0} \frac{x^2 - x\ln(1 + x) + 3x}{\sin(x) + 3x^2}
        &= \lim_{x\to 0} \frac{x(x - \ln(1 + x) + 3)}{x(\frac{\sin x}{x} + 3x)} \\
        &= \lim_{x\to 0} \frac{0-0+3}{1 + 0} = 3
    \end{align*}
}

\sexercise{}{
    \begin{align*}
        \lim_{x\to 0} \frac{2\sin (x) \left(e^{x^2} - 1\right)}{(1-\cos(x)){\left[\ln(1 + \sqrt{x})\right]}^2}
        &= \lim_{x\to 0} \frac{
            2x(1 + o(1))x^2(1 + o(1))
        }{
            \frac{x^2}{2}(1 + o(1)) \sqrt{x}(1 + o(1))
        } \\
        &= \lim_{x\to 0} \frac{4(1 + 2o(1) + o^2(1))}{1 + o^2(1) + 3o(1) + o^3(1)} = 4
    \end{align*}
}

\sexercise{}{
    Considera \(n \in \mathbb{R}\)
    \begin{align*}
        \lim_{x\to 0^+} \frac{2}{x^2} \left[
            e^{-x^n} - 1 + \ln\left(\cos(x^n)\right)
        \right]
        &= \lim_{x\to 0^+} \frac{2}{x^2} \left[
            -x^n(1 + o(1)) + \ln\left(1 + \cos(x^n) - 1\right)
        \right] \\
        &= \lim_{x\to 0^+} \frac{2}{x^2} \left[
            -x^n(1 + o(1)) + \ln\left(1 - \frac{x^{2n}}{2}(1 + o(1))\right)
        \right] \\
        &= \lim_{x\to 0^+} \frac{2}{x^2} \left[
            -x^n(1 + o(1)) -\frac{x^{2n}}{2}(1 + o(1))
        \right] \\
        &= \lim_{x\to 0^+} - \frac{2}{x^2} \left[
            x^n \left(1 + \frac{x^n}{2}\right)(1 + o(1))
        \right]
    \end{align*}
    Nel caso \(n>0\) abbiamo
    \[
        \begin{cases}
            0 & n>2 \\
            -2 & n=2 \\
            -\infty & 0 < m < 2
        \end{cases}
    \]
    Nel caso \(n=0\) il limite è \(-\infty\), mentre se \(n<0\) il limite
    non è ben definito.
}

\sexercise{}{
    \begin{align*}
        \lim_{x\to 0} \ln(\cos x + x^2) \frac{e^{-\frac{x^2}{2}} + 1}{1-e^{-x^2}}
    \end{align*}
    Sostituendo troviamo la forma di indeterminazione \(\frac{0}{0}\).
    \begin{align*}
        \lim_{x\to 0} ln(1 + \cos x - 1 + x^2) \frac{e^{-\frac{x^2}{2}} + 1}{\left(1-e^{-\frac{x^2}{2}}\right)\left(1+e^{-\frac{x^2}{2}}\right)}
        &= \lim_{x\to 0} \frac{
            \ln\left(
                1 - \frac{x^2}{2}(1 + o(1)) + x^2
            \right)
        }{
            \frac{x^2}{2}(1 + o(1))
        } = 1
    \end{align*}
}

\sexercise{}{
    \begin{align*}
        \lim_{x\to \frac{\pi}{4}} {\left(
            \sin(2x)
        \right)}^{\frac{1}{\ln\left(1 + \cos\left(x + \frac{\pi}{4}\right)\right)}}
    \end{align*}
    Sostituendo troviamo la forma di indecisione \(1^\infty\).
    Facciamo un cambio di variabile \(t = x-\frac{\pi}{4}\)
    \begin{align*}
        \lim_{x\to 0} {\left(
            \sin\left(2\left(t + \frac{\pi}{4}\right)\right)
        \right)}^{\frac{1}{\ln\left(1 + \cos\left(x + \frac{\pi}{2}\right)\right)}}
        &= \lim_{t\to 0} \exp\left\{\frac{1}{\ln(1 - \sin t)} \cdot \ln(\cos(2t))\right\} \\
        &= \lim_{t\to 0} \exp\left\{\frac{\ln(1 + \cos(2t) - 1)}{\ln(1 - \sin t)}\right\} \\
        &= \lim_{t\to 0} \exp\left\{\frac{\ln(1 - 2t^2)}{1 - t}\right\} \\
        &= \lim_{t\to 0} \exp\left\{\frac{-2t}{t}\right\} = e^0 = 1
    \end{align*}
}

\pagebreak

\subsection{Continuità}

\sexercise{}{
    Studia la continuità di
    \[
        f(x) = {\left(\ln|x|\right)}^{-1}
    \]
}

\end{document}