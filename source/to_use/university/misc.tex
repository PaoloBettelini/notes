\documentclass[a4paper]{article}

\usepackage{amsmath}
\usepackage{amssymb}
\usepackage{stellar}
\usepackage{parskip}
\usepackage{fullpage}
\usepackage{wrapfig}
\usepackage{tikz}

\usetikzlibrary{arrows}
\usetikzlibrary{decorations.pathreplacing}
\usetikzlibrary{cd}

\title{Algebra I}
\author{Paolo Bettelini}
\date{}

% Integral command
\ExplSyntaxOn
\DeclareDocumentCommand{\integral}{d[] d[] d[] d[]}
{
    \IfNoValueTF { #4 }
    {
        \IfNoValueTF { #3 }
        { \int #1\,d#2 }
        { \text{Either\,\,2\,\,or\,\,4\,\,arguments\,\,must\,\,be\,\,passed} }
    }
    {
        \int\limits\c_math_subscript_token{#1}^{#2} #3 \,d#4
        % In Expl Syntax characters as '_' or ':' are used as letters in command names
        % hence the \c_math_subscript_token{#1} rather than _{#1}
    }
}
\ExplSyntaxOff

\begin{document}

\maketitle
\tableofcontents


\section{Esercizi}

%\[
%    \sin(x) \sim x(1 + o(1))
%\]

\subsection{Limiti}

\sexercise{}{
    \begin{align*}
        \lim_{x\to 0} \frac{x^2 - x\ln(1 + x) + 3x}{\sin(x) + 3x^2}
        &= \lim_{x\to 0} \frac{x(x - \ln(1 + x) + 3)}{x(\frac{\sin x}{x} + 3x)} \\
        &= \lim_{x\to 0} \frac{0-0+3}{1 + 0} = 3
    \end{align*}
}

\sexercise{}{
    \begin{align*}
        \lim_{x\to 0} \frac{2\sin (x) \left(e^{x^2} - 1\right)}{(1-\cos(x)){\left[\ln(1 + \sqrt{x})\right]}^2}
        &= \lim_{x\to 0} \frac{
            2x(1 + o(1))x^2(1 + o(1))
        }{
            \frac{x^2}{2}(1 + o(1)) \sqrt{x}(1 + o(1))
        } \\
        &= \lim_{x\to 0} \frac{4(1 + 2o(1) + o^2(1))}{1 + o^2(1) + 3o(1) + o^3(1)} = 4
    \end{align*}
}

\sexercise{}{
    Considera \(n \in \mathbb{R}\)
    \begin{align*}
        \lim_{x\to 0^+} \frac{2}{x^2} \left[
            e^{-x^n} - 1 + \ln\left(\cos(x^n)\right)
        \right]
        &= \lim_{x\to 0^+} \frac{2}{x^2} \left[
            -x^n(1 + o(1)) + \ln\left(1 + \cos(x^n) - 1\right)
        \right] \\
        &= \lim_{x\to 0^+} \frac{2}{x^2} \left[
            -x^n(1 + o(1)) + \ln\left(1 - \frac{x^{2n}}{2}(1 + o(1))\right)
        \right] \\
        &= \lim_{x\to 0^+} \frac{2}{x^2} \left[
            -x^n(1 + o(1)) -\frac{x^{2n}}{2}(1 + o(1))
        \right] \\
        &= \lim_{x\to 0^+} - \frac{2}{x^2} \left[
            x^n \left(1 + \frac{x^n}{2}\right)(1 + o(1))
        \right]
    \end{align*}
    Nel caso \(n>0\) abbiamo
    \[
        \begin{cases}
            0 & n>2 \\
            -2 & n=2 \\
            -\infty & 0 < m < 2
        \end{cases}
    \]
    Nel caso \(n=0\) il limite è \(-\infty\), mentre se \(n<0\) il limite
    non è ben definito.
}

\sexercise{}{
    \begin{align*}
        \lim_{x\to 0} \ln(\cos x + x^2) \frac{e^{-\frac{x^2}{2}} + 1}{1-e^{-x^2}}
    \end{align*}
    Sostituendo troviamo la forma di indeterminazione \(\frac{0}{0}\).
    \begin{align*}
        \lim_{x\to 0} ln(1 + \cos x - 1 + x^2) \frac{e^{-\frac{x^2}{2}} + 1}{\left(1-e^{-\frac{x^2}{2}}\right)\left(1+e^{-\frac{x^2}{2}}\right)}
        &= \lim_{x\to 0} \frac{
            \ln\left(
                1 - \frac{x^2}{2}(1 + o(1)) + x^2
            \right)
        }{
            \frac{x^2}{2}(1 + o(1))
        } = 1
    \end{align*}
}

\sexercise{}{
    \begin{align*}
        \lim_{x\to \frac{\pi}{4}} {\left(
            \sin(2x)
        \right)}^{\frac{1}{\ln\left(1 + \cos\left(x + \frac{\pi}{4}\right)\right)}}
    \end{align*}
    Sostituendo troviamo la forma di indecisione \(1^\infty\).
    Facciamo un cambio di variabile \(t = x-\frac{\pi}{4}\)
    \begin{align*}
        \lim_{x\to 0} {\left(
            \sin\left(2\left(t + \frac{\pi}{4}\right)\right)
        \right)}^{\frac{1}{\ln\left(1 + \cos\left(x + \frac{\pi}{2}\right)\right)}}
        &= \lim_{t\to 0} \exp\left\{\frac{1}{\ln(1 - \sin t)} \cdot \ln(\cos(2t))\right\} \\
        &= \lim_{t\to 0} \exp\left\{\frac{\ln(1 + \cos(2t) - 1)}{\ln(1 - \sin t)}\right\} \\
        &= \lim_{t\to 0} \exp\left\{\frac{\ln(1 - 2t^2)}{1 - t}\right\} \\
        &= \lim_{t\to 0} \exp\left\{\frac{-2t}{t}\right\} = e^0 = 1
    \end{align*}
}

\pagebreak

\subsection{Continuità}

\sexercise{}{
    Studia la continuità di
    \[
        f(x) = {\left(\ln|x|\right)}^{-1}
    \]
}

\pagebreak

\subsection{Integrali}

\sexercise{}{
    \[
        \integral[\frac{1}{e^x + 1}][x]
    \]
    Allora sostituiamo \(t = e^x\) quindi
    \begin{align*}
        \integral[\frac{1}{e^x + 1}][x] &= 
        \integral[\frac{1}{t}\frac{1}{1+t}][t] \\
        &= \integral[\frac{1}{t} - \frac{1}{t+1}][t] \\
        &= \log|t| - \log|t + 1| + C \\
        &= \log\left(\frac{e^x}{e^x + 1}\right) + C
    \end{align*}
}

\sexercise{}{
    \[
        \integral[\frac{1}{e^x + 2 + e^{-x}}][x]
    \]
    Allora abbiamo
    \begin{align*}
        \integral[\frac{1}{e^x + 2 + e^{-x}}][x] &= \integral[\frac{e^x}{e^{2} + 2e^x + 5}][x]
    \end{align*}
    Sostituiamo \(t = e^x\) e otteniamo
    \begin{align*}
        \int \frac{dt}{t^2 + 2t + 5} &=\frac{1}{4}\int \frac{dt}{1 + {\left(\frac{t+1}{2}\right)}^2} \\
        &= \frac{1}{2} \arctan\left\{\frac{e^x + 1}{2}\right\}
    \end{align*}
}

\sexercise{}{
    \[
        \integral[x^{-\frac{3}{2}} \arctan(x^{-\frac{1}{2}})][x]
    \]
    Cominciamo sostituendo \(t=x^{-\frac{1}{2}}\) e proseguiamo per parti
    \begin{align*}
        \integral[x^{-\frac{3}{2}} \arctan(x^{-\frac{1}{2}})][x] &=
        \integral[t^3 \arctan(t)(-2t^{-3})][t] \\
        &= -2\integral[\arctan(t) \cdot 1][t] \\
        &= -2 \left[
            t\arctan(t) - \integral[\frac{t}{1 + t^2}][t]
        \right] \\
        &= -2t\arctan(t) + \integral[\frac{2t}{1 + t^2}][t]
    \end{align*}
    Sostituiamo \(v = 1 + t^2\)
    \begin{align*}
        \integral[\frac{2t}{1 + t^2}][t] &=
        \int \frac{dv}{v} = \log(1 + t^2) + C
    \end{align*}
    Allora il risultato è
    \[
        -2x^{-\frac{1}{2}}\arctan(x^{-\frac{1}{2}}) +
        \log(1 + x^{-1}) + C
    \]
}

\sexercise{}{
    \begin{align*}
        \integral[\log(x^2 + 4)][x] &=
        x\log(x^2 + 4) - 2\integral[\frac{x^2}{x^2 + 4}][x] \\
        &= x\log(x^2 + 4) - 2 \integral[1 - \frac{4}{x^2 + 4}][x] \\
        &= x\log(x^2 + 4) - 2x + \arctan\left(\frac{x}{2}\right) + C
    \end{align*}
}

\sexercise{}{
    \begin{align*}
        \integral[\frac{1}{x^2 + x + 1}][x] &= \integral[\frac{1}{{\left(x + \frac{1}{2}\right)}^2 +\frac{3}{4}}][x]
    \end{align*}
}

\sexercise{}{
    \begin{align*}
        \integral[\frac{1}{x\sqrt{x + 1}}][x]
    \end{align*}
    Sostituiamo \(x + 1 = t^2\)
    \begin{align*}
        \integral[\frac{1}{x\sqrt{x + 1}}][x]
        &= \integral[\frac{2t}{t\cdot(t^2 - 1)}][t] \\
        &= 2 \int \frac{dt}{(t-1)(t+1)} \\
        &= \int \frac{dt}{t-1} + \int \frac{dt}{t+1} \\
        &= \log|t-1|-\log|t+1| + C \\
        &= \log\left|\frac{\sqrt{x+1} - 1}{\sqrt{x+1} + 1}\right| + C
    \end{align*}
}


\sexercise{}{
    Procediamo per parti
    \begin{align*}
        \integral[x^{-\frac{1}{2}} \arctan(x^{-\frac{1}{2}})][x] &= 
        2x^{\frac{1}{2}}\arctan(x^{-\frac{1}{2}}) - \integral[\frac{2x^{\frac{1}{2}}(-\frac{1}{2}x^{-\frac{3}{2}})}{1 + x^{-1}}][x] \\
        &= 2x^{\frac{1}{2}} \arctan(x^{-\frac{1}{2}}) + \log|1 + x| + C
    \end{align*}
}

\sexercise{}{
    \begin{align*}
        \integral[\frac{\arcsin(\sqrt{x})}{\sqrt{x}}][x]
    \end{align*}
    Sostituiamo \(t = \sqrt{x}\) e poi procediamo per parti
    \begin{align*}
        2 \integral[\arcsin(t)][t] &=
        t\arcsin(t) - 2\integral[\frac{t}{\sqrt{1-t^2}}][t]
    \end{align*}
    Sostituiamo ora \(v = 1-t^2\)
    \begin{align*}
        t\arcsin(t) + \integral[v^{-\frac{1}{2}}][v]
        &= \sqrt{x}\arcsin(\sqrt{x}) + 2\sqrt{1 - x} + C
    \end{align*}
}

\sexercise{}{
    \[
        \int \frac{dx}{e^{2x} + 3e^x + 2}
    \]
    Sostituiamo \(t = e^x\)
    \begin{align*}
        \int \frac{dt}{t(t+1)(t+2)}
    \end{align*}
}

\sexercise{}{
    \[
        \integral[\frac{x + 2}{{(x + 1)}^{\frac{5}{2}}}][x]
    \]
    Sostituiamo \(t = x + 1\).
}

\sexercise{}{
    \[
        \integral[\frac{1 + e^x}{1 + e^{2x}}][x]
    \]
    Sostituiamo \(t = e^x\).
    \begin{align*}
        \integral[\frac{1 + t}{1 + t^2}\frac{1}{t}][t] &=
        \integral[\frac{1}{1 + t^2} \frac{1}{t}][t] + \int \frac{dt}{1 + t^2} \\
        &= \arctan(t) + \integral[\frac{-t}{1 + t^2} + \frac{1}{t}][t] \\
        &= -\frac{1}{2} \ln(1 + e^{2x}) + \ln(e^{x}) + \arctan(e^x)
    \end{align*}
}

\end{document}