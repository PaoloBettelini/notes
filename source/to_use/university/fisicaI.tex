\documentclass[a4paper]{article}

\usepackage{amsmath}
\usepackage{amssymb}
\usepackage{stellar}
\usepackage{parskip}
\usepackage{fullpage}
\usepackage{wrapfig}
\usepackage{tikz}

\usetikzlibrary{arrows}
\usetikzlibrary{decorations.pathreplacing}

\title{Fisica I}
\author{Paolo Bettelini}
\date{}

% libri
% Rosati

\begin{document}

\maketitle
\tableofcontents

\section{Introduzione}

\section{Vettori spostamento}

\begin{itemize}
    \item vettore spostamento: direzione, verso, lunghezza;
    \item somma di vettori;
    \item moltiplicazione con scalare reale \(\vec{a} + (-\vec{a}) = \vec{0}\);
    \item modulo di un vettore;
\end{itemize}

\sproposition{Proprietà distributiva del prodotto rispetto alla somma vettoriale}{
    \[\alpha(\vec{a} + \vec{b}) = \alpha\vec{a} + \alpha\vec{b}\]
}

\section{Sistemi di coordinate}

Il punto di origine è il posto in cui viene posizionato l'osservatore.
I sistemi di coordinate trattati sono esclusivamente cartesiani e con basi ortogonali.
L'oservatore ha i versori delle direzioni.

Si possono quindi individuare le componenti di un vettore lungo le sue direzioni, ossia
le proiezioni ortogonali del vettori lungo gli assi cartesiani.
Di conseguenza, le coordinate di un vettore hanno senso solamente rispetto ad una base.

\sdefinition{Prodotto scalare}{
    Il prodotto scalare fra due vettori risulta in un numero reale
    \[
        \vec{a} \cdot \vec{b} \in \mathbb{R}
    \]
    Dato l'angolo \(\theta\) fra \(\vec{a}\) e \(\vec{b}\),
    \[
        \vec{a}\cdot \vec{b} = |\vec{a}| \cdot|\vec{b}| \cdot \cos\theta
    \]
}

Chiaramente il prodotto scalare è commutativo.

\sproposition{Proprietà distributiva del prodotto scalare rispetto alla somma}{
    \[
        \vec{c} \cdot (\vec{a} + \vec{b}) = c\vec{a} + c\vec{b}
    \]
}

\section{Leggi di Newton}


\end{document}