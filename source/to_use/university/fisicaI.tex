\documentclass[a4paper]{article}

\usepackage{amsmath}
\usepackage{amssymb}
\usepackage{stellar}
\usepackage{parskip}
\usepackage{fullpage}
\usepackage{wrapfig}
\usepackage{tikz}

\usetikzlibrary{arrows}
\usetikzlibrary{decorations.pathreplacing}

\title{Fisica I}
\author{Paolo Bettelini}
\date{}

% libri
% Rosati

\begin{document}

\maketitle
\tableofcontents

\section{Introduzione}

\section{Vettori spostamento}

\begin{itemize}
    \item vettore spostamento: direzione, verso, lunghezza;
    \item somma di vettori;
    \item moltiplicazione con scalare reale \(\vec{a} + (-\vec{a}) = \vec{0}\);
    \item modulo di un vettore;
\end{itemize}

\sproposition{Proprietà distributiva del prodotto rispetto alla somma vettoriale}{
    \[\alpha(\vec{a} + \vec{b}) = \alpha\vec{a} + \alpha\vec{b}\]
}

\section{Sistemi di coordinate}

Il punto di origine è il posto in cui viene posizionato l'osservatore.
I sistemi di coordinate trattati sono esclusivamente cartesiani e con basi ortogonali.
L'oservatore ha i versori delle direzioni.

Si possono quindi individuare le componenti di un vettore lungo le sue direzioni, ossia
le proiezioni ortogonali del vettori lungo gli assi cartesiani.
Di conseguenza, le coordinate di un vettore hanno senso solamente rispetto ad una base.

\sdefinition{Prodotto scalare}{
    Il prodotto scalare fra due vettori risulta in un numero reale (in uno spazio euclideo \(\mathbb{R}^n\))
    \[
        \vec{a} \cdot \vec{b} \in \mathbb{R}
    \]
    Dato l'angolo \(\theta\) fra \(\vec{a}\) e \(\vec{b}\),
    \[
        \vec{a}\cdot \vec{b} = |\vec{a}| \cdot|\vec{b}| \cdot \cos\theta
    \]
}

Chiaramente il prodotto scalare è commutativo.

\sproposition{Proprietà distributiva del prodotto scalare rispetto alla somma}{
    \[
        \vec{c} \cdot (\vec{a} + \vec{b}) = c\vec{a} + c\vec{b}
    \]
}

\sproposition{Prodotto vettoriale con componenti}{
    TODO....
}

Da qui possiamo notare che il prodotto scalare ha lo stesso ridultato per ogni basta ortonormata.

% fai il prodotto fra c e a+b, e poi espandi la distributiva e semplifica i termini ortogonali.

\pagebreak

\section{Cinematica}

La cinematica è la parte della meccanica che descrive il moto di un punto materiale.
Per descrivere il moto di un oggetto è necessario procurarsi un sistema di riferimento.
Sceglieremo quindi un'origine e una base ortonormata.

\sdefinition{Posizione}{
    La \textit{posizione} di un punto è rappresentata unicamente da un vettore \(\vec{r}(t)\),
    che mostra lo spostamento fra l'origine e la sua posizione \(P(t)\) in un determinato istante di tempo.
}

Se vogliamo considerare la posizione solo nella direzione \(x\)
possiamo calcoalre
\[
    \hat{x}(t) = \vec{x}\vec{r}(t)
\]
In generale
\[
    \vec{r}(t) = \hat{x}\vec{r}(t) + \hat{x}\vec{y}(t) + \hat{z}\vec{r}(t)
\]

La relazione fra due osservatori diversi è data da \(\vec{R} + \vec{r'}(t) = \vec{r}(t)\).

La velocità è quindi relativa a due posizioni \(P(t)\) e \(P(t + \Delta t)\).
Lo spostamento è \(\vec{r}(t + \Delta t) = \vec{r}(t) + \vec{s}(t)\).

\sdefinition{Velocità}{
    La \textit{velocità} di un punto rappresenta lo spostamento che il punto materiale
    percorre in un unità di tempo \(\vec{v}(t)\).
    Allora la velocità è definita come
    \[
        \vec{v}(t) = \lim_{\Delta t \to 0} \frac{\vec{s}}{\Delta t}
        = \lim_{\Delta t \to 0} \frac{\vec{r}(t+\Delta t) - \vec{r}(t)}{\Delta t}
        = \frac{d\vec{r}(t)}{dt}
    \]
}

Il vettore della velocità si orienta verso la tangente della curva (cioè nella direzione in cui si sta spostando).
Chiaramente la derivata può essere separata nelle componenti
\[
    \vec{v}(t)=v_x\hat{x} + v_y\hat{y} + v_z\hat{z}
\]
dove possiamo anche dire che
\[
    v_x(t) = \frac{dx(t)}{dt}
\]

\sdefinition{Accelerazione}{
    L'\textit{accelerazione} di un punto rappresenta il cambiamento istantaneo della velocità
    \[
        \vec{a}(t) = \lim_{\Delta t \to 0} \frac{\vec{v}(t + \Delta t) - \vec{v}(t)}{\Delta t}
    \]
}

\section{Leggi orarie}

\sproposition{Caduta da una altezza}{
    Il tempo di caduta di un oggetto da un altezza \(h\), soggetto a gravità costante \(g\)
    è cado da
    \[
        t_\text{caduta} = \sqrt{\frac{2h}{g}}
    \]
    con velocità
    \[
        -\sqrt{2gh}
    \]
}

\section{Moto circolare}

\end{document}