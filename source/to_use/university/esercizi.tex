\documentclass[a4paper]{article}

\usepackage{amsmath}
\usepackage{amssymb}
\usepackage{stellar}
\usepackage{parskip}
\usepackage{fullpage}
\usepackage{wrapfig}
\usepackage{tikz}


\title{Esercizi I}
\author{Paolo Bettelini}
\date{}

\begin{document}

\maketitle
\tableofcontents

\pagebreak 

\section{Esercizi}

\subsection{Serie a coefficienti positivi}

\sexample{Serie telescopica}{
    Considera
    \[
        \sum_{n=1}^\infty \frac{1}{(4n-1)(4n+3)}
    \]
    manipoliamo la serie come telescopica
    \begin{align*}
        \sum_{n=1}^\infty \frac{1}{(4n-1)(4n+3)}
        &= \sum_{n=1}^\infty
        \frac{\frac{1}{4}}{4n-1} + \frac{-\frac{1}{4}}{4n+3} \\
        &= \frac{1}{4} \sum_{n=1}^\infty
        \frac{1}{4n-1} - \frac{1}{4n+3} \\
        &= \frac{1}{4} \sum_{n=1}^\infty
        \frac{1}{b_n} - \frac{1}{b_{n+1}}, \quad b_n = 4n-1 
    \end{align*}
    Quindi il risultato è dato da
    \begin{align*}
        \frac{1}{4} \left[
            \frac{1}{3} - \lim_{n\to \infty} \frac{1}{4n-1}
        \right] = \frac{1}{12}
    \end{align*}
}

\sexample{Serie geometrica}{
    Considera
    \[
    x    \sum_{n=2}^\infty \frac{2^{-2n+1}}{3^{n-2}} = a_n
    \]
    manipoliamo la serie come geometrica
    \begin{align*}
        a_n &= \frac{2 \cdot 4^{-n}}{9^{-1} \cdot 3^n} \\
        &= 18 \cdot {\left(\frac{1}{2}\right)}^n       
    \end{align*}
    la serie diventa allora
    \begin{align*}
        18 \sum_{n=2}^\infty {\frac{1}{12}}^n
        &= 18 \left[\sum_{n=0}^\infty \left({\frac{1}{12}}^n\right) - 1 - \frac{1}{12}\right]
        \\
        &= 18 \left[ \frac{1}{1 - \frac{1}{12}} - 1 - \frac{1}{12} \right] \\
        &= \frac{3}{22}
    \end{align*}
}

\sexample{Termine n-esimo}{
    Considera \[
        \sum_{n=1}^\infty \frac{16-n^4}{n^2 + 3}
    \]
    Notiamo che il termine n-esimo non tende a zero, quindi la serie non converge.
}

\sexample{p-q-serie}{
    Considera
    \[
        \sum_{n=1}^\infty \frac{\log n}{n+1}
    \]
    vogliamo confrontarla con una serie nota
    \begin{align*}
        \sum_{n=1}^\infty \frac{\log n}{n+1}
        = a_1 + \sum_{n=2}^\infty \frac{1}{(n+1){(\log n)}^{-1}}
    \end{align*}
    Ignoriamo il \(+1\) a denominatore. Questa serie può essere confrontata con una p-q-serie
    con \(p=1\) e \(q=-1\), quindi diverge.
}

\sexample{Test del confronto}{
    Considera
    \[
        \sum_{n=1}^\infty \frac{1}{\sqrt{n}}
    \]
    (con il confronto).
    Chiaramente, \(\frac{1}{\sqrt{n}} > \frac{1}{n}\)
    e quindi siccome la serie armonica diverge, anche questa diverge.
}

\sexample{Test del confronto asintotico}{
    Considera
    \[
        \sum_{n=1}^\infty \frac{
            n^2 + n\cos n + {\left(1 + \frac{1}{n}\right)}^{\sqrt{n}}
        }{
            {\left(n+\frac{1}{2}\right)}^4
            + n\log n + e^{-n}
        }
    \]
    Abbiamo il termine
    \begin{align*}
        a_n &= \frac{
            n^2 \left[1 + \frac{\cos n}{n} + \frac{1}{n^2} {\left(1 + \frac{1}{n}\right)}^{\sqrt{n}}\right]
        }{
            n^4 \left[
                {\left(1 + \frac{1}{2n}\right)}^4
                + \frac{\log n}{n^3} + \frac{e^{-n}}{n^4}
            \right]
        } \sim \frac{
            1
        }{
            n^2
        }
    \end{align*}
    con
    \[
        {\left(1 + \frac{1}{n}\right)}^{\sqrt{n}} 
        = e^{\sqrt{n} \cdot \frac{1}{n}(1 + o(1))}
        \to 1
    \]
    Questo ultimo passaggio è dato dal fato che \(\log (1 + \varepsilon_n) = \varepsilon(1 + o(1))\).
    Quindi, la serie ha lo steso carattere della p-serie con \(p=2\) che converge.
}

\sexample{Test del rapporto}{
    Considera
    \[
        \sum_{n=0}^\infty \frac{n^2}{3^n}
    \]
    applichiamo il test del rapporto
    \begin{align*}
        \frac{a_{n+1}}{a_n} &= \frac{{(n+1)}^2}{3^{n+1}}
        \cdot \frac{3^n}{n^2} \\
        &= {\left(\frac{n+1}{n}\right)}^2 \cdot \frac{1}{3}
        \\
        &= \frac{1}{3} {\left(1 1 \frac{1}{n}\right)}^2 \to \frac{1}{3}
    \end{align*}
    Quindi, siccome \(L < 1\), la serie converge.
}

\sexample{Test della radice ennesima}{
    Considera
    \[
        \sum_{n=1}^\infty \frac{n^{\sqrt{n}}}{2^n}
    \]
    applichiamo il tst della radice ennesima
    \begin{align*}
        \sqrt[n]{a_n} = \frac{
            n^{\frac{\sqrt{n}}{2}}
        }{2}
        &= \frac{1}{2} e^{\frac{1}{\sqrt{n}}\log n} \to \frac{1}{2}
    \end{align*}
    Quindi, siccome \(L < 1\), la serie converge.
}

%%%%

\sexercise{}{ % rapporto
    Studia \[
        \sum_{n=1}^\infty \frac{
            n!
        }{
            e^{n^2}
        }
    \]
}

\sexercise{}{ 
    Studia \[
        \sum_{n=1}^\infty \frac{
            2^n + 3^{2n}
        }{
            10^{n+2}
        }
    \]
    La serie è geometrica
    \begin{align*}
        \sum_{n=1}^\infty \frac{
            2^n + 3^{2n}
        }{
            10^{n+2}
        } &=
        \frac{1}{100} \left[
            \left(\sum_{n=1}^\infty \frac{
            2^n
        }{
            10^{n}
        }\right)
        +
        \left(\sum_{n=1}^\infty \frac{
            9^{n}
        }{
            10^{n}
        }\right)
        \right]
        \\
        &= 
        \frac{1}{100} \left[
            \left(\sum_{n=0}^\infty \frac{
            2^n
        }{
            10^{n}
        }\right)
        +
        \left(\sum_{n=0}^\infty \frac{
            9^{n}
        }{
            10^{n}
        }\right)
        -1-1
        \right]
        \\
        &= 
        \frac{1}{100} \left[
            \left(\frac{1}{1 - \frac{2}{10}}\right)
        +
        \left(\frac{1}{1 - \frac{9}{10}}\right)
        -1-1
        \right] \\
        &= \frac{37}{400}
    \end{align*}
}

\sexercise{}{ % radice
    Studia \[
        \sum_{n=1}^\infty \frac{
            1
        }{
            {(\log n)}^n
        }
    \]
}

\sexercise{}{ % converge / p-q-serie
    Studia \[
        \sum_{n=2}^\infty \frac{
            e^{1/n} - 1
        }{
            n^2 \log n
        }
    \]
}

\sexercise{}{
    Studia \[
        \sum_{n=1}^\infty \frac{
            \sin^2(n) + 1
        }{
            3^n + n
        }
    \]
    Siccome \(0 \leq \sin^2 (n) \leq 1\), la serie ha lo stesso carattere di
    \begin{align*}
        \sum_{n=1}^\infty \frac{
            2
        }{
            3^n + n
        } &=
        \sum_{n=1}^\infty \frac{
            2
        }{
            3^n \left(1 + \frac{n}{3^n}\right)
        } \\
        &\sim 
        \sum_{n=1}^\infty \frac{
            1
        }{
            3^n 
        }
    \end{align*}
    che converge per gerarchia degli infiniti.
}

\sexercise{}{ % telescopica
    Studia \[
        \sum_{n=1}^\infty \frac{
            1
        }{
            n(n+1)(n+2)
        }
    \]
    Abbiamo
    \begin{align*}
        \sum_{n=1}^\infty \frac{
            1
        }{
            n(n^2 + 3n + 2)
        }
    \end{align*}
}

\end{document}