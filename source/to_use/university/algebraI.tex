\documentclass[a4paper]{article}

\usepackage{amsmath}
\usepackage{amssymb}
\usepackage{stellar}
\usepackage{parskip}
\usepackage{fullpage}
\usepackage{wrapfig}
\usepackage{tikz}

\usetikzlibrary{arrows}
\usetikzlibrary{decorations.pathreplacing}
\usetikzlibrary{cd}

\title{Algebra I}
\author{Paolo Bettelini}
\date{}

% iscriz. Esse3
\begin{document}

\maketitle
\tableofcontents

\section{Richiami di teoria degli insiemi}

% Fu introdorra da Cantor. Poi la storia con Russel e Hilbert

% prove that this doesn't work with the difference. (che non è associativa e commutativa)

% (A \ B) \ C = (A \ C) \ B

Data una famiglia finita o infinite di insiemi \({\{A_i\}}_{i\in I}\), la loro intersection
\[
    \bigcap_{i\in I} A_i
\]
è l'insieme degli elementi che stanno in tutti gli insiemi \(A_i\), mentre la loro unione
\[
    \bigcup_{i\in I} A_i
\]
è l'insieme degli elementi che stanno in almeno uno degli insiemi \(A_i\).

\pagebreak 

\section{Classi di equivalenza}

Esempio insieme quoziente \(\sim\) su \(\mathbb{Z}\) dove \(a\sim b \iff |a|=|b|\)
è dato da
\[
    \{
        \{0 \}, \{1,-1 \}, \{2, -2 \}, \cdots    
    \}
\]

L'unica relazione di equivalenza che è un ordine è l'uguaglianza.

\pagebreak 

\section{Esempi di maggiorante etc.}

In \(\mathbb{R}\) consideriamo l'usuale ordinamento. Consideriamo i sottoinsiemi 
\[
    A = \{ x \in \mathbb{R} \,|\, x > 0 \}
\]
\[
    B = \{ x \in \mathbb{R} \,|\, x \geq 0 \}
\]
e
\[
    C = \{ x \in \mathbb{R} \,|\, 0 < x \leq 2 \}
\]

Il sottoinsieme \(A\) non ha maggioranti. Ogni numero non-positivo è minorante di \(A\).
\(A\) non ha nè massimo nè minimo.

Il sottoinsieme \(B\) non ha maggioranti. Ogni numero non-positivo è minorante di \(B\).
\(B\) ha \(0\) come minimo.

Il sottoinsieme \(C\) ha minoranti e maggioranti ma non minimo e ho \(2\) come massimo.

Consideriamo ora la relazione di divisibilità in \(\mathbb{N}\).
L'unico maggiorante è \(0\) in quanto tutti dividono zero, ed è un massimo.
Il numero \(1\) è minorante, ed è un minimo.

Se ora prendiamo l'insieme \(\{2,3,4,5\}\), i maggioranti sono mulitpli del minimo comune multiplo (\(60\)),
i minoranti sono i divisori comuni. Non ci sono massimo e minimo.

\sproposition{Il massimo è unico}{
    Il massimo, se esiste, è unico.
}

\sproof{Il massimo è unico}{
    Diciamo che \(a,b\) sono due massimi di \(A\), cioè maggioranti di \(A\) che appartiene ad \(A\).
    Abbiamo allora \(a \geq b\) (in quanto \(a\) è un maggiorante) e \(b \geq a\) (in quando \(b\) è un maggiorante).
    Abbiamo quindi che \(a=b\).
}

\sdefinition{Massimale}{
    Un elemento \(a\in A\) con \(A\) insieme partzialmente ordinato è detto massimale in \(A\)
    se non esiste alcun \(b \in A\) tale che \(a \leq b\) dove \(a\neq b\).
}

\sdefinition{Minimale}{
    Un elemento \(a\in A\) con \(A\) insieme partzialmente ordinato è detto minimale in \(A\)
    se non esiste alcun \(b \in A\) tale che \(a \geq b\) dove \(a\neq b\).
}

Ogni massimo è massimale, ogni minimo è minimale.

Esempio in cui i massimali non sono massimi: in \(\mathbb{N}\), rispetto alla divisibilità,
consideriamo l'insieme \(A= \{2,3,4,5,6\}\).

\begin{itemize}
    \item Il numero \(2\) è minimale ma non massimale.
    \item Il numero \(3\) è minimale ma non massimale.
    \item Il numero \(4\) è massimale perché non divide nient'altro, ma non minimale.
    \item Il numero \(5\) è sia massimale che minimale.
    \item Il numero \(6\) è massimale ma non minimale.
\end{itemize}

In una relazione d'ordine totale un eventuale elemento massimale è massimo.
Infatti, se \(a\) è massimale per \(A\), preso un qualsiasi elemento
\(b\in A\), sappiamo che vale almeno una tra \(a \leq b\) e \(b \leq a\).
Se vale la prima, per la definizione di massimalità di \(a\), non può essere \(a\neq b\).
Nel secondo caso, \(b \leq a\) e quindi \(a\) è un massimo.
Analogamente per i minimali.

\subsection{Relazioni irriflessiva}

Data una relazione d'ordine \(\leq\), possiamo ottenere la relazione d'ordine stretta \(<\)
dicendo che \(a<b \) se \(a\leq b\) e \(a\neq b\).

Si può definire l'ordine stretto rimpiazzando la proprietà riflessiva con quella irriflessiva.

\section{Funzioni}

Una funzione \(\phi\colon A \to B\) dove \(A\) è il dominio mentre \(B\) è il codominio,
preso un elemento \(a\in A\), la sua immagine viene denotata \(\phi(a)\) oppure \(af\).

Se \(C \subseteq A\), la sua immagine tramite \(\phi\) è indicata come \(C\phi\)
che è un sottoinsieme di \(B\).
\[
    C\phi = \{  c\phi \,|\, c\in C \}
\]

Se \(D\) è un sottoinsieme di \(B\), la sua immagine inversa tramite \(\phi\)
è il sottoinsieme \(D\phi^{-1}\) di \(A\) degli elementi la cui immagine appartiene a \(D\).
\[
    D\phi^{-1} = \{  a \in A \,|\, a\phi \in D \}
\]

\sexample{Funzione}{
    Sia \(\phi\colon \mathbb{R} \to \mathbb{R}\) definita ponendo \(\phi x \triangleq x^2\).
}

Consideriamo ora \(A=\{-1,0,1,2\}\). Abbiamo allora \(A\phi = \{1,0,4\}\).
Consideriamo poi \(B=\{-1, 0, 2, 9\}\). Abbiamo allora \(B\phi^{-1} = \{0, \sqrt{2}, 3, -3\}\).

L'immagine di una funzione è chiaramente l'immagine per il suo dominio come insieme considerato.

\subsection{Proprietà}

\sproposition{}{
    Se \(C \subseteq D \subseteq A\), abbiamo \(C\phi \subseteq D\phi\).
}

\sproof{}{
    Abbiamo che
    \[
        C\phi = \{c\phi \,|\, c\in C\}
    \]
    Dunque \(x\in C\phi\) se e solo se esiste \(c\in C\) tale che \(x=c\phi\).
    Ma \(C \subseteq D\), dunque \(c \in D\). Quindi, \(x = c\phi \in D\phi\).
}

Non è detto che se \(C \subset D\) allora \(C\phi \subset D\phi\).
Mostriamo un esempio in cui \(C \subset D\) ma \(C\phi = D\phi\).
Prendiamo \(C=\{1\} \subset D=\{1,-1\}\). Se prendiamo la funzione del quadrato, in ambo caso
trovo la stessa immagine per via di ambo gli insiemi.

Ciò non avviene nel caso in cui la funzione fosse iniettiva.

\sproposition{}{
    Se \(E\subseteq F \subseteq B\), abbiamo che \(E\phi^{-1} \subseteq F\phi^{-1}\).
}

TODO: esercizio proof.

Anche qui la medesima proposizione ma con l'inclusione stretta non è assicurata.

\sproposition{}{
    Se \(C\subseteq A\), allora \(C\phi \phi^{-1} \supseteq C\).
}

\sproof{}{
    Sia \(x \in C\). Bisogna mostrare \(x\in C\phi\phi^{-1}\).
    Ricordiamo che \(D\phi^{-1} = \{ y \in A \,|\, y\phi \in D \}\). Dunque
    \(Cy\phi = \{ y \in A \,|\, y\phi \in C\phi \}\). Ma ora \(x\phi \in C\phi\),
    perché \(x\in C\). Dunque \(x \in C\phi\phi^{-1}\).
}

Nel solito esempio
\[
    \{1,-1\}\phi\phi^{-1} = \{1,-1\}   
\]
e
\[
    \{1\}\phi\phi^{-1} = \{1,-1\}   
\]

\sproposition{}{
    Se \(D\subseteq B\) allora \(D\phi^{-1}\phi \subseteq D\). L'inclusione può essere stretta.
}
\sproof{}{
    Sia \(x\in D\phi^{-1}\phi\). Ciò significa che \(x = z\phi \)
    per qualche \(z\in D\phi^{-1}\). Ma \(D\phi^{-1} = \{y\,|\, y\phi \in D\}\).
    Dunque, \(z\in D\phi^{-1}\), allora \(z\phi \in D\), cioè \(x\in D\).
}

Con il solito esempio
\[
    \{1,2\}\phi^{-1}\phi = \{1\}
\]
\[
    \{-1\}\phi^{-1}\phi = \emptyset
\]

\sproposition{}{
    Siano \(\phi \colon A \to B\), \(\psi\colon B\to C\) e \(\theta \colon C \to D\) funzioni.
    allora
    \[
        (\phi\psi)\theta = \phi(\psi\theta)
    \]
}

\sproof{}{
    Notiamo che \(\phi\psi \colon A \to C\) e \(\theta\colon C \to D\). Dunque
    \(\phi\psi \colon A \to D\). Analogamente \(\phi \colon A \to B\), \(\psi\theta\colon B\to D\)
    e quindi \(\phi (\psi \theta ) \colon A \to D\). Per mostrare l'uguaglianua devo mostrare che  per ogni \(x\in A\)
    risulta
    \[
        a((\phi\psi)\theta) = a(\phi(\psi \theta))
    \]
    Abbiamo infatti \(a((\phi\psi)\theta) = (a(\phi\psi\theta)) = ((a\phi)\psi)\theta\)
    e \(a(\phi(\psi \theta)) = (a\phi)(\psi\theta) = ((a\phi)\psi)\theta\).
}

Dunque possiamo scrivere semplicemente \(\phi\psi\theta\) senza ambiguità.

Siano \(\phi \colon A \to B\), \(\psi\colon B\to C\) funzioni. 
Ci chiediamo ora che \(\psi\phi = \phi\psi\). Chiaramente, non è detto
che \(\phi\psi\) esista. Possiamo confrontarle solo che \(A=B\).

Allora guardiamo \(\phi \colon A\to A\) e \(\psi \colon A\to A\).
Non è comunque detto che \(\psi\phi = \phi\psi\) siano uguali.

\sdefinition{Funzione identità}{
    Dato un insieme \(A\), la \textit{funzione identica} di \(A\) è la funzione
    \(\text{Id}_A\colon A \to A\) definita come \(a\text{Id}_A \triangleq a\).
}

\sproposition{}{
    Sia \(\phi\colon A \to B\), allora \(\phi\text{Id}_B=\phi\) e \(\text{Id}_A\phi=\phi\).
    TODO: dimostrazione.
}

\subsection{Iniettività suriettività}

La definizione di iniettività è equivalente a dire che \(b\phi^{-1}\) contiene solo un elemento. \\
La definizione di suriettività è equivalente a dire che \(b\phi^{-1}\) contiene almeno un elemento. \\
La definizione di suriettività è equivalente a dire che \(b\phi^{-1}\) e \(b\phi\) contengono solo un elemento.

\subsection{Composizione}

% TODO stellar
Date \(f\) e \(g\) cosa possiamo dire di \(f\) e \(g\) sapendo che \(g(f)\) è suriettiva o iniettiva?

Supponiamo che \(g(f)\) sia suriettiva.
Dunque, per ogni \(c\in C\) esiste \(a\) tale che \(c=g(f(a))\).
In particolare, posto \(b=f(a) \in B\), abbiamo che \(g(b)=c\) cioè \(g\) è suriettiva.
% ma non possiamo dire che f sia suriettiva

Supponiamo che \(g(f)\) sia iniettiva.
Dunque, per ogni \(a_1, a_2 \in A\) dove \(a_1 \neq a_2\), risulta che \(g(f(a_1)) \neq g(f(a_2))\).
Sicuramente la prima funzione non può fare convergere i due elementi, in quando non potrebbero uscire separati
dopo la seconda funzione.
In particolare, \(f(a_1) \neq f(a_2)\). Quindi, \(f\) è iniettiva.
% ma non possiamo dire che g è iniettiva

\sexample{}{
    Siano \(A = \{a\}\) e \(B=\{b, b'\}\) con \(b\neq b'\), \(C=\{c\}\)e \(f \colon A \to B\)
    data \(f(a)=b\) e \(g\colon B \to C\) data \(g(b) = c\) e \(g(b') = c\).
    Allora \(g(f)\) è biettiva. \(f\) è iniettiva e \(g\) non è iniettiva.
    \(f\) non è suriettiva e \(g\) è suriettiva.
}

\subsection{Definizione di invertibilità}

Data \(f\colon A \to B\), allora \(f\) è invertibile se esiste \(g\colon B \to A\)
tale che \(g(f)\) è la funzione identità su \(A\) e \(f(g)\) è la funzione identità su \(B\).

\sproposition{}{
    Se \(f\) è invertibile, allora \(g\) è unica.
}

\sproof{}{
    Prendiamo \(h\colon B \to A\) tale che \(h(f(a)) = a\) e \(f(h(b)) = b\).
    Allora \(g = g\text{Id}_A = g(fh) = (gf)h = \text{Id}_Bh=h\) e quindi è la funzione identità.
}

\sproposition{}{
    Ogni inverso è anch'esso invertibile \({f^{-1}}^{-1}\).
}

\sproposition{}{
    Se \(f\colon A \to B\) e \(g\colon B \to C\) sono invertibili, allora \(g(f)\)
    è invertibile e \({\left(g(f)\right)}^{-1} = f^{-1}(g^{-1})\).
}

\sproof{}{
    Sappiamo che esistono \(f^{-1}\) e \(g^{-1}\). Dunque esiste \(f^{-1}(g^{-1})\).
    Mostriamo che componendo le due in maniera simmetrica si trovano le identità di \(A\) e di \(B\).
}

\sproof{Invertibilità è equivalente a biettività}{
    \iffproof{
        Sia \(f\) invertibile. Allora sappiamo che \(f(f^{-1})\) è la funzione identità di \(A\) e \(f^{-1}(f)\)
        è la funzione identità di \(B\).
        Ora, l'identità di \(A\) è iniettiva (anche biettiva), dunque \(f\) è iniettiva
        e l'identità di \(B\) è suriettiva (dalle due proposizioni di prima), dunque \(f\) è suriettiva.
    }{
        Sia \(f\) biettiva.
        Dobbiamo costruire \(g \colon B \to A\) tale che \(f(g)\) è l'identità di \(A\) e \(f(g)\) è l'identità di \(B\).
        Sappiamo che per ogni \(b\in B\) esiste un unico \(a\in A\) tale che
        \(f(a) = b\). Poniamo allora \(g(b)=a\). Se \(b \in B\), allora \(f(g(b)) = f(a) = b\).
        Se \(a\in A\), abbiamo che \(g(f(a))\) è per definizione di \(g\) l'unico elemento \(a'\in A\)
        tale che  \(f(a') = f(a)\). Siccome \(f\) è iniettiva, \(a'=a\), e quindi \(g(f(a)) = a\)
        e quindi \(g=f^{-1}\).
    }
}

%% Usare anche \inversefunction su stellar

\pagebreak

\section{Matrici}

Data una matrice \(A\) indichiamo con \(A_{i,j}\) l'emenento di posto \((i,j)\).

La trasporta di una triangolare inferiore è triangolare superiore, e viceversa.
La trasporta di una matrice diagonale rimane uguale.

Una matrice uguale alla sua trasporta è detta simmetrica.

\sdefinition{}{
    Dato un anello commutativo \(R\) diciamo \(M_{m,n}(R)\) l'insieme delle matricii \(m \times n\)
    a coefficienti in \(R\).
}

L'addizione è associativa e commutativa (come nell'anello commutativo).
Esiste l'elemento neutro (matrice nulla \(0_{m,n}\)).
Esiste l'elemento inverso \(-A = -1 \cdot A\).
Si dovrebbe dimostrare l'unicità dell'elemento inverso e del neutro.

\sproposition{}{
    Date matrici \(A\) e \(B\) della stessa dimensione, si ha
    \[
        {(A+B)}^t = A^t + B^t
    \]
}

\stheorem{Moltiplicazione associativa}{
    Se \(A \in M_{m,n}(R)\) e \(B \in M_{n,p} (R)\) e \(C \in M_{p,r} (R)\),
    allora
    \[
        (AB)C = A(BC)
    \]
}

\sproof{Moltiplicazione associativa}{
    \(AB\) è di tipo \(m \times p\). L'elemento di posto \((i,j)\) è
    \[
        \sum_{k=1}^n A_{i,j}B_{k,j} = D_{i,j}
    \]
    La matrice \((AB)C\) ha dimensione \(m \times r\). L'elemento di posto \((i,l)\)
    è
    \begin{align*}
        \sum_{j=1}^p D_{i,j}C_{j,l} &= \sum_{j=1}^n \left( \sum_{k=1}^n A_{i,k}B_{k,j} \right) C_{j,l} \\
        &= \sum_{j=1}^p \sum_{k=1}^n A_{i,k} B_{k,j} C_{j,l}
    \end{align*}
    \(BC\) è ha dimensione \(n \times r\). L'elemento di post \((k,l)\)
    è \[
        \sum_{j=1}^p B_{k,j} C_{j,e} = E_{k,e}
    \]
    \(A(BC)\) ha dimensione \(m \times r\). L'elemento di posto \((i,l)\) è
    \begin{align*}
        \sum_{k=1}^n A_{i,k}E_{k,l} &= \sum_{k=1}^n A_{i,k} \left( \sum_{j=1}^p B_{k,j}C_{j,e} \right)\\
        &= \sum_{k=1}^n \sum_{j=1}^p A_{i,k} B_{k,j} C_{j,e}
    \end{align*}
}

\sproposition{Distributività destra}{
    Con \(A, B \in M_{m,n}(R)\) e \(C \in M_{n,p}(R)\)
    \[
        (A+B)C = AC + BC
    \]
}

\sproposition{Distributività sinistra}{
    Con \(A, B \in M_{m,n}(R)\) e \(C \in M_{n,p}(R)\)
    \[
        A(B+C) = AB + AC
    \]
}

In generale non vale \(AB=BA\). Ambo le operazioni sono definite solo se ambo le matrici sono quadrate
con dimensione \(n\times n\). In tal caso, non è comunque detto che la proprietà valga.
Nel caso in cui \(n=1\) la proprietà commutativa vale necessariamente.

Il principio di annullamento del prodotto non vale.
\[
    \begin{bmatrix}
        1 & 0 \\
        0 & 0
    \end{bmatrix}
    \begin{bmatrix}
        0 & 0 \\
        0 & 1
    \end{bmatrix}
    =
    \begin{bmatrix}
        0 & 0 \\
        0 & 0
    \end{bmatrix}
\]
In questo caso il risultato è la matrice nulla ma nessuno dei due era nulla.

\sproposition{}{
    Se \(A\) e \(B\) sono invertibili e dello stesso ordine, allora
    \(AB\) è invertibile e \({(AB)}^{-1 = B^{-1}A^{-1}}\).
}

\sexample{Matrice non invertibile}{
    \[
        \begin{bmatrix}
            1 & 2 \\
            2 & 4
        \end{bmatrix}
        \begin{bmatrix}
            x & y \\
            z & w
        \end{bmatrix}
        =
        \begin{bmatrix}
            x+2z & y+2w \\
            2x+4z & 2y+4w
        \end{bmatrix}
    \]
    Notiamo che i punti dove dovrebbe esserci uno \(0\) sono il doppio di quelli con \(1\),
    quindi non vi è soluzione e non è invertibile.
}

Se \(A\) e \(B\) sono due matrici quadrate della stessa dimensione
tali che \(AB = I_n\) allora anche \(BA = I_n\) (La dimostrazione non è banale).
% Quindi la definizione vale solo con uno.

\sproposition{}{
    Se \(A \in M_{m,n}(R)\) e \(B \in B_{n,p}(R)\) allora
    \(B^tA^t \in M_{p,m}(R)\). Abbiamo quindi che
    \[
        B^tA^t = {(AB)}^t
    \]
}

\sproposition{}{
    Se \(A\) è invertibile, allora
    \[
        {(A^t)}^{-1} = {(A^{-1})}^t
    \]
}

\pagebreak

\section{Numeri naturali}

\sdefinition{Assiomi di Peano}{
    I numeri naturali sono un insieme \(\mathbb{N}\) dotati di una funzione successore
    \(S \colon \mathbb{N} \to \mathbb{N}\)
    e di un elemento fissato \(0\) tali che:
    \begin{itemize}
        \item la funzione \(S\) è iniettiva;
        \item \(0 \not\in \text{Im}_S\);
        \item se \(A\subseteq \mathbb{N}\) tale che \(0\in A\) e \(As\subseteq A\), allora \(A = \mathbb{N}\);
    \end{itemize}
}

L'esistenza di un tale insieme è garantita dalla teoria assiomatica.
Tuttavia, dobbiamo garantire che i modelli degli assiomi di Peano siano isomorfi,
quindi trovare una funzione biettiva fra tutti i modelli.
Quindi, dati due modelli \((\mathbb{N}, S, 0)\) e \((\mathbb{N}', S', 0')\)
bisogna trovare una funzione biettiva \(f\colon \mathbb{N} \to \mathbb{N}'\)
tale che \(f(0) = 0'\) e \(nfs' = nsf\)

\begin{center}
% https://tikzcd.yichuanshen.de/#N4Igdg9gJgpgziAXAbVABwnAlgFyxMJZABgBpiBdUkANwEMAbAVxiXBAF9T1Nd9CUZAIxVajFmzAIuPbHgJEhpEdXrNWicAHI4WztxAY5-ReVFqJmsHo6iYUAObwioAGYAnCAFskZEDggkAGZVcQ0QV303Tx9EEP9AxAAmUPU2XSiImKQlBKQUsTTNSOoGOgAjGAYABV55ARB3LAcACxxMj29fagCc1MsQBFKKqtrjBU0m1vbbDiA
\begin{tikzcd}
    n \arrow[r, "f"] \arrow[d, "s"'] & n' \arrow[d, "s'"] \\
    ns \arrow[r, "f"']               & n's'              
\end{tikzcd}
\end{center}

Questo può essere fatto con un procedimento cosidetto per ricorrenza, dipende fortemente
dall'assioma 3. In generale gli assiomi di Peano mi permettono di definire successioni di oggetti
per ricorrenza, cioè assegnando un oggetto associato a \(0\)
e il modo di costruire l'oggetto associato (come per esempio il fattoriale o l'addizione nei naturali).

La somma è definita nel seguente modo ricorrente: \(m+n=0\) e \(m+S(n) = S(m+n)\).

Usango gli assiomi posso dimsotrare varie proprietà dell'addizione, detta moltiplicazione
(da definire anch'esso per ricorrenza) e dell'ordine (anch'esso da definire per ricorrenza).

L'ordine è definito solamente da \(n \leq S(n)\).

Le proprietà per \(m,n,p\in\mathbb{N}\) sono:
\begin{enumerate}
    \item \textbf{somma associativa:} \((m+n)+p = m + (n+p)\);
    \item \textbf{somma distributiva:} \(m + n = n + m\);
    \item \textbf{somma nulla:} \(m + 0 = m\);
    \item \textbf{prodotto associativo:} \((mn)p = m(np)\);
    \item \textbf{prodotto distributivo:} \(mn = nm\);
    \item \textbf{prodotto nullo:} \(m S(0)= m\);
    \item \textbf{distributiva:} \((m+n)p = mp + np\);
    \item \textbf{cancellazione somma:} \(m+n = m+p \implies n=p\);
    \item \textbf{cancellazione prodotto:} \(mn = mp \land m \neq 0 \implies n=p\);
    \item \textbf{compatibilità tra somma e ordine:} \(m\leq n \implies m+p \leq n + p\);
    \item \textbf{compatibilità tra prodotto e ordine:} \(m\leq n \implies mp \leq np\);
\end{enumerate}

Detto \(1\) il numero \(S(0)\) risulterà che \(S(n) = n+1\).

\paragraph{Assioma 3:}

\sproposition{}{
    Un altro modo per dire l'assioma 3 è che ogni sottoinsieme non vuoto di \(\mathbb{N}\)
    ammette un minimo.
}

\sproof{}{
    Per dimostrarlo sia \(A\subseteq B\) l'insieme di tutti i minoranti.
    L'insieme \(A\) contiene sicuramente \(0\). Infatti, \(0 \leq n\) per ogni \(n\in\mathbb{N}\).
    L'insieme \(A\) è diverso da \(\mathbb{N}\). Infatti, preso un \(n \in B\),
    sappiamo che \(B \neq \emptyset\), abbiamo che \(n+1\) non è minore o uguale di \(n\),
    quindi non è un minorante di \(B\). Pertanto \(n+1 \notin A\).
    Sappiamo per gli assiomi di Peano che un sottoinsieme di \(\mathbb{N}\) che contiene \(0\)
    e contiene il successore di ogni elemento, coincide con \(\mathbb{N}\).
    Poiché \(0\in A\) e \(A\neq \mathbb{N}\), possiamo concludere che esiste \(k\in A\)
    tale che \(k+1 \notin A\), cioè \(k\) è minorante di \(B\) ma \(k+1\) no.
    Ma allora esiste \(i \in A\) tale che \(k+1 \not\leq i\).
    Poiché l'ordine è totale, ciò significa che \(i < k+1\). D'altra parte
    \(k\) è minorante di \(B\). In particolare, \(k \leq i\), che è minore di \(k+1\).
    Per la proprietà dell'ordine dei naturali, si ha che \(k=1\), cioè \(k\in B\).
    Dunque, \(k\) è minorante di \(B\) che appartiene a \(B\) come volevamo (è il nostro minimo).
}

Non è necessario l'assioma della scelta per prendere \(n\in B\)
in quando \(B\) è ben definito e sappiamo come sceglierlo.

\section{Numeri interi}

Fatto l'anello commutativo degli interi si possono dimostrare delle proprietà come ad esempio
\(n\cdot0 = 0\) per ogni \(n\).

Per dimostrare invece il principio di annullamento del prodotto, cioè che \(mn=0\)
se e solo se almeno uno tra \(m\) e \(n\) è \(0\).
In alcuni anelli commutativi il principio di annullamento del prodotto non vale.

Si dimostra poi che dato \(n \in \mathbb{Z}\), si ha che
\(n\in\mathbb{Z}\) oppure \(0-n n\in\mathbb{Z}\) per ogni \(n\) intero.
Si pone allora
\[
    |n| = \begin{cases}
        n & \text{è un naturale} \\
        -n & \text{altrimenti}
    \end{cases}
\]

Una volta introdotto l'ordine negli interi (compatibile con quello dei naturali),
si dimostrano queste proprietà:
\begin{enumerate}
    \item \(a \leq b \implies a+c \leq b+c\);
    \item \(a \leq b \land c \geq 0 \implies ac \leq bc\);
    \item \(|a+b| \leq |a| + |b|\);
    \item \(|a\cdot b| \leq |a|\cdot|b|\).
\end{enumerate}

\subsection{Divisione con resto}

Estendiamo l'mcd a valori tutti nulli.
Dati due interi il loro mcd è il numero naturali
\(d\) che divide entrambi ed è multiplo di tutti i divisori comuni.
Se almeno uno tra questi è nullo, questo coincide con la definizione precedente.
Se tutti sono zero, definiamo l'mcd come zero, in quanto zero è un multiplo di zero.

\subsection{Massimo comun divisore}

Dimostrare l'esistenza di un massimo comune divisore su più interi
per induzione: il caso base è quello in cui il numero di interi è 2.
Usare l'esistenza del membro a destra e verificare che soddisfa la definizione.

% Dopo prime-number-divides-product
% Viceversa, mostriamo la contronominale, cioè che se \(p \not\divides a\) e \(p \not\divides b\)
% allora \(p \not\divides ab\).
%  I divosiri di \(p\) sono \(1\), \(p\) e i loro opposti. Se \(p\) non divide \(a\),
% i divisori comuni tra \(p\) e \(a\) sono solo \(1\) e \(-1\).
% Dunque gcd(p,a) = 1
% Riassunto:
% Se p divide a, allora gcd(p, a) = p, altrimenti gcd(p, a) = 1.
% Ora, io sto assumendo che p non divide a e p non divide b, cioè che
% p è coprimo con a e p è coprimo con b. Per la proposizione precedente,
% p è coprimo con ab e quindi, e quindi p non divide ab.
% In generale per induzione si trova che (esercizio) che se un primo p
% divide il prodotto di a_1a_2...a_n, allora ne divide almeno uno.

\section{Classi di resto}

Consideriamo i non-multipli di \(3\). La differenza fra un non-multiplo di 3 e quello dopo è 
o \(1\) o \(2\). Dividiamo allora i non-multipli di 3 saltando \(2\) a \(2\), ossia
\[
    -5, -2, +1, +4, +7, +10
\]
e
\[
    -4, -1, +2, +5, +8, +11
\]
La somma di due numeri corrispondenti è sempre un numero di 3.
In generale, se considero le tre liste
\begin{align*}
    -6, -3, +0, +3, +6, +9
    -5, -2, +1, +4, +7, +10 \\
    -4, -1, +2, +5, +8, +11 \\
\end{align*}
Se facciamo la somma di due termini, la lista in cui è il risutato è dato solamente dalle liste
dei due addenti.

\subsection{Criteri di divisibilità}

Consideriamo un numero in base \(10\)
\[
    m = a_r a_{r-1} \cdots
\]

Allora, \(n\) è multiplo di \(9\) se e solo se la somma delle cifre è multiplo di \(9\)
(analogamente per 3).

Allora, \(n\) è multiplo di \(11\) se e solo se \(a_0 - a_1 + a_2 - a_3 \cdots\)
è multiplo di \(11\).

Abbiamo che 
\[
    n = \sum a_r 10^r
\]

\sproposition{}{
    Le potenze di 10 sono tutte congrue a \(1 \pmod{9}\)
}

\sproof{}{
    \begin{itemize}
        \item Il caso base è \(10^0 = 0 \equiv 1 \pmod{9}\);
        \item Se \(10^k \equiv \pmod{9}\), allora \(10^{k+1} \equiv \pmod{9}\).
    \end{itemize}
}

Dunque,
\begin{align*}
    n &= a_r 10^r + a_{r-1} 10^r{r-1} \cdots + a_0 10^0 \\
    &\equiv a_r \cdot 1 + a_{r-1} 1 + \cdots + a_0 + 1 \pmod{9} \\
    &\equiv a_r + a_{r-1} + \cdots + a_0 \pmod{9}
\end{align*}

Analogamente per le potenze di \(3\).
Per \(11\) le cose sono più simili se calcoliamo le potenze di \(10\)
modulo \(11\).
\(10^0 = 1 \equiv 1 \pmod{11}\), \(10^1 = 10 \equiv -1 \pmod{11}\)
e così via. Si mostra per induzione che
\[
    10^k \equiv {(-1)}^k \pmod{11}
\]
per ogni \(k \geq 0\).

\subsection{Funzione di Eulero}

% TODO URGENT
Sia \({[a]}_n\) invertibile.
Classi inveritbili di \(\mathbb{Z} / n\) siano
\[
    {[b_1]}_n, {[b_2]}_n, \cdots, {[b_{\varphi(n)}]}_n
\]
Ora
\[
    {[a]}_n{[b_1]}_n, {[a]}_n{[b_2]}_n, \cdots, {[a]}_n{[b_{\varphi(n)}]}_n
\]

\begin{enumerate}
    \item Siccome il prodoto di due invertibili è invertibile, queste due sono tutte invertibili
    \item Poiché \({[a]}_n\) è invertibile, per la legge di cancellazione, le classi della seconda lista sono tutte diverse.
    \item (1+2) implicano che la prima e la seconda lista coincidano a mano dell'ordinamento.
\end{enumerate}
Dunque,
\[
    \cdots
\]

Abbiamo quindi dimostrato il teorema di Eulero

\stheorem{}{
    Se \(a\) è un intero coprimo con \(n\), allora
    \[
        a^{\varphi(n)} \equiv 1 \pmod{n}
    \]
}

\scorollary{Piccolo teorema di Fermat}{
    Sia \(p\) un primo e \(a\) un intero. Allora
    \[
        a^p \equiv a \pmod{p}
    \]
}

\sproof{}{
    % enumerate
    Se \(a\) è coprimo con \(p\), per il teorema di Eulero abbiamo che \(a^{\varphi(p)} \equiv 1 \pmod{p}\).
    Ma \(\varphi(p) = p-1\) poiché \(p\) è primo e, dunque, \(a^{p-1} \equiv 1 \pmod{p}\)
    da cui, moltiplicando per \(a\), si ottiene \(a^p \equiv a \pmod{p}\).

    Se \(a\) non è coprimo con \(p\), allora \(p \,|\, a\), cioè \(a \equiv 0 \pmod {p}\)
    e, quindi, ogni potenza con esponente positivo di \(a\) è congruo a \(0 \pmod{p}\).
    In particolare, \(a^p \equiv 0 \pmod{p}\).
}

\end{document}