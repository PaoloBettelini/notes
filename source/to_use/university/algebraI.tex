\documentclass[a4paper]{article}

\usepackage{amsmath}
\usepackage{amssymb}
\usepackage{stellar}
\usepackage{parskip}
\usepackage{fullpage}
\usepackage{wrapfig}
\usepackage{tikz}

\usetikzlibrary{arrows}
\usetikzlibrary{decorations.pathreplacing}

\title{Algebra I}
\author{Paolo Bettelini}
\date{}

% iscriz. Esse3
\begin{document}

\maketitle
\tableofcontents

\section{Richiami di teoria degli insiemi}

% Fu introdorra da Cantor. Poi la storia con Russel e Hilbert

% prove that this doesn't work with the difference. (che non è associativa e commutativa)

% (A \ B) \ C = (A \ C) \ B

Data una famiglia finita o infinite di insiemi \({\{A_i\}}_{i\in I}\), la loro intersection
\[
    \bigcap_{i\in I} A_i
\]
è l'insieme degli elementi che stanno in tutti gli insiemi \(A_i\), mentre la loro unione
\[
    \bigcup_{i\in I} A_i
\]
è l'insieme degli elementi che stanno in almeno uno degli insiemi \(A_i\).

\pagebreak 

\section{Classi di equivalenza}

Esempio insieme quoziente \(\sim\) su \(\mathbb{Z}\) dove \(a\sim b \iff |a|=|b|\)
è dato da
\[
    \{
        \{0 \}, \{1,-1 \}, \{2, -2 \}, \cdots    
    \}
\]

L'unica relazione di equivalenza che è un ordine è l'uguaglianza.

\pagebreak 

\section{Esempi di maggiorante etc.}

In \(\mathbb{R}\) consideriamo l'usuale ordinamento. Consideriamo i sottoinsiemi 
\[
    A = \{ x \in \mathbb{R} \,|\, x > 0 \}
\]
\[
    B = \{ x \in \mathbb{R} \,|\, x \geq 0 \}
\]
e
\[
    C = \{ x \in \mathbb{R} \,|\, 0 < x \leq 2 \}
\]

Il sottoinsieme \(A\) non ha maggioranti. Ogni numero non-positivo è minorante di \(A\).
\(A\) non ha nè massimo nè minimo.

Il sottoinsieme \(B\) non ha maggioranti. Ogni numero non-positivo è minorante di \(B\).
\(B\) ha \(0\) come minimo.

Il sottoinsieme \(C\) ha minoranti e maggioranti ma non minimo e ho \(2\) come massimo.

Consideriamo ora la relazione di divisibilità in \(\mathbb{N}\).
L'unico maggiorante è \(0\) in quanto tutti dividono zero, ed è un massimo.
Il numero \(1\) è minorante, ed è un minimo.

Se ora prendiamo l'insieme \(\{2,3,4,5\}\), i maggioranti sono mulitpli del minimo comune multiplo (\(60\)),
i minoranti sono i divisori comuni. Non ci sono massimo e minimo.

\sproposition{Il massimo è unico}{
    Il massimo, se esiste, è unico.
}

\sproof{Il massimo è unico}{
    Diciamo che \(a,b\) sono due massimi di \(A\), cioè maggioranti di \(A\) che appartiene ad \(A\).
    Abbiamo allora \(a \geq b\) (in quanto \(a\) è un maggiorante) e \(b \geq a\) (in quando \(b\) è un maggiorante).
    Abbiamo quindi che \(a=b\).
}

\sdefinition{Massimale}{
    Un elemento \(a\in A\) con \(A\) insieme partzialmente ordinato è detto massimale in \(A\)
    se non esiste alcun \(b \in A\) tale che \(a \leq b\) dove \(a\neq b\).
}

\sdefinition{Minimale}{
    Un elemento \(a\in A\) con \(A\) insieme partzialmente ordinato è detto minimale in \(A\)
    se non esiste alcun \(b \in A\) tale che \(a \geq b\) dove \(a\neq b\).
}

Ogni massimo è massimale, ogni minimo è minimale.

Esempio in cui i massimali non sono massimi: in \(\mathbb{N}\), rispetto alla divisibilità,
consideriamo l'insieme \(A= \{2,3,4,5,6\}\).

\begin{itemize}
    \item Il numero \(2\) è minimale ma non massimale.
    \item Il numero \(3\) è minimale ma non massimale.
    \item Il numero \(4\) è massimale perché non divide nient'altro, ma non minimale.
    \item Il numero \(5\) è sia massimale che minimale.
    \item Il numero \(6\) è massimale ma non minimale.
\end{itemize}

In una relazione d'ordine totale un eventuale elemento massimale è massimo.
Infatti, se \(a\) è massimale per \(A\), preso un qualsiasi elemento
\(b\in A\), sappiamo che vale almeno una tra \(a \leq b\) e \(b \leq a\).
Se vale la prima, per la definizione di massimalità di \(a\), non può essere \(a\neq b\).
Nel secondo caso, \(b \leq a\) e quindi \(a\) è un massimo.
Analogamente per i minimali.

\subsection{Relazioni irriflessiva}

Data una relazione d'ordine \(\leq\), possiamo ottenere la relazione d'ordine stretta \(<\)
dicendo che \(a<b \) se \(a\leq b\) e \(a\neq b\).

Si può definire l'ordine stretto rimpiazzando la proprietà riflessiva con quella irriflessiva.

\section{Funzioni}

Una funzione \(\phi\colon A \to B\) dove \(A\) è il dominio mentre \(B\) è il codominio,
preso un elemento \(a\in A\), la sua immagine viene denotata \(\phi(a)\) oppure \(af\).

Se \(C \subseteq A\), la sua immagine tramite \(\phi\) è indicata come \(C\phi\)
che è un sottoinsieme di \(B\).
\[
    C\phi = \{  c\phi \,|\, c\in C \}
\]

Se \(D\) è un sottoinsieme di \(B\), la sua immagine inversa tramite \(\phi\)
è il sottoinsieme \(D\phi^{-1}\) di \(A\) degli elementi la cui immagine appartiene a \(D\).
\[
    D\phi^{-1} = \{  a \in A \,|\, a\phi \in D \}
\]

\sexample{Funzione}{
    Sia \(\phi\colon \mathbb{R} \to \mathbb{R}\) definita ponendo \(\phi x \triangleq x^2\).
}

Consideriamo ora \(A=\{-1,0,1,2\}\). Abbiamo allora \(A\phi = \{1,0,4\}\).
Consideriamo poi \(B=\{-1, 0, 2, 9\}\). Abbiamo allora \(B\phi^{-1} = \{0, \sqrt{2}, 3, -3\}\).

L'immagine di una funzione è chiaramente l'immagine per il suo dominio come insieme considerato.

\subsection{Proprietà}

\sproposition{}{
    Se \(C \subseteq D \subseteq A\), abbiamo \(C\phi \subseteq D\phi\).
}

\sproof{}{
    Abbiamo che
    \[
        C\phi = \{c\phi \,|\, c\in C\}
    \]
    Dunque \(x\in C\phi\) se e solo se esiste \(c\in C\) tale che \(x=c\phi\).
    Ma \(C \subseteq D\), dunque \(c \in D\). Quindi, \(x = c\phi \in D\phi\).
}

Non è detto che se \(C \subset D\) allora \(C\phi \subset D\phi\).
Mostriamo un esempio in cui \(C \subset D\) ma \(C\phi = D\phi\).
Prendiamo \(C=\{1\} \subset D=\{1,-1\}\). Se prendiamo la funzione del quadrato, in ambo caso
trovo la stessa immagine per via di ambo gli insiemi.

Ciò non avviene nel caso in cui la funzione fosse iniettiva.

\sproposition{}{
    Se \(E\subseteq F \subseteq B\), abbiamo che \(E\phi^{-1} \subseteq F\phi^{-1}\).
}

TODO: esercizio proof.

Anche qui la medesima proposizione ma con l'inclusione stretta non è assicurata.

\sproposition{}{
    Se \(C\subseteq A\), allora \(C\phi \phi^{-1} \supseteq C\).
}

\sproof{}{
    Sia \(x \in C\). Bisogna mostrare \(x\in C\phi\phi^{-1}\).
    Ricordiamo che \(D\phi^{-1} = \{ y \in A \,|\, y\phi \in D \}\). Dunque
    \(Cy\phi = \{ y \in A \,|\, y\phi \in C\phi \}\). Ma ora \(x\phi \in C\phi\),
    perché \(x\in C\). Dunque \(x \in C\phi\phi^{-1}\).
}

Nel solito esempio
\[
    \{1,-1\}\phi\phi^{-1} = \{1,-1\}   
\]
e
\[
    \{1\}\phi\phi^{-1} = \{1,-1\}   
\]

\sproposition{}{
    Se \(D\subseteq B\) allora \(D\phi^{-1}\phi \subseteq D\). L'inclusione può essere stretta.
}
\sproof{}{
    Sia \(x\in D\phi^{-1}\phi\). Ciò significa che \(x = z\phi \)
    per qualche \(z\in D\phi^{-1}\). Ma \(D\phi^{-1} = \{y\,|\, y\phi \in D\}\).
    Dunque, \(z\in D\phi^{-1}\), allora \(z\phi \in D\), cioè \(x\in D\).
}

Con il solito esempio
\[
    \{1,2\}\phi^{-1}\phi = \{1\}
\]
\[
    \{-1\}\phi^{-1}\phi = \emptyset
\]

\sproposition{}{
    Siano \(\phi \colon A \to B\), \(\psi\colon B\to C\) e \(\theta \colon C \to D\) funzioni.
    allora
    \[
        (\phi\psi)\theta = \phi(\psi\theta)
    \]
}

\sproof{}{
    Notiamo che \(\phi\psi \colon A \to C\) e \(\theta\colon C \to D\). Dunque
    \(\phi\psi \colon A \to D\). Analogamente \(\phi \colon A \to B\), \(\psi\theta\colon B\to D\)
    e quindi \(\phi (\psi \theta ) \colon A \to D\). Per mostrare l'uguaglianua devo mostrare che  per ogni \(x\in A\)
    risulta
    \[
        a((\phi\psi)\theta) = a(\phi(\psi \theta))
    \]
    Abbiamo infatti \(a((\phi\psi)\theta) = (a(\phi\psi\theta)) = ((a\phi)\psi)\theta\)
    e \(a(\phi(\psi \theta)) = (a\phi)(\psi\theta) = ((a\phi)\psi)\theta\).
}

Dunque possiamo scrivere semplicemente \(\phi\psi\theta\) senza ambiguità.

Siano \(\phi \colon A \to B\), \(\psi\colon B\to C\) funzioni. 
Ci chiediamo ora che \(\psi\phi = \phi\psi\). Chiaramente, non è detto
che \(\phi\psi\) esista. Possiamo confrontarle solo che \(A=B\).

Allora guardiamo \(\phi \colon A\to A\) e \(\psi \colon A\to A\).
Non è comunque detto che \(\psi\phi = \phi\psi\) siano uguali.

\sdefinition{Funzione identità}{
    Dato un insieme \(A\), la \textit{funzione identica} di \(A\) è la funzione
    \(\text{Id}_A\colon A \to A\) definita come \(a\text{Id}_A \triangleq a\).
}

\sproposition{}{
    Sia \(\phi\colon A \to B\), allora \(\phi\text{Id}_B=\phi\) e \(\text{Id}_A\phi=\phi\).
    TODO: dimostrazione.
}

\subsection{Iniettività suriettività}

La definizione di iniettività è equivalente a dire che \(b\phi^{-1}\) contiene solo un elemento. \\
La definizione di suriettività è equivalente a dire che \(b\phi^{-1}\) contiene almeno un elemento. \\
La definizione di suriettività è equivalente a dire che \(b\phi^{-1}\) e \(b\phi\) contengono solo un elemento.

\end{document}