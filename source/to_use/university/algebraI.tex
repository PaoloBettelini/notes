\documentclass[a4paper]{article}

\usepackage{amsmath}
\usepackage{amssymb}
\usepackage{stellar}
\usepackage{parskip}
\usepackage{fullpage}
\usepackage{wrapfig}
\usepackage{tikz}

\usetikzlibrary{arrows}
\usetikzlibrary{decorations.pathreplacing}

\title{Algebra I}
\author{Paolo Bettelini}
\date{}

% iscriz. Esse3
\begin{document}

\maketitle
\tableofcontents

\section{Richiami di teoria degli insiemi}

% Fu introdorra da Cantor. Poi la storia con Russel e Hilbert

% prove that this doesn't work with the difference. (che non è associativa e commutativa)

% (A \ B) \ C = (A \ C) \ B

Data una famiglia finita o infinite di insiemi \({\{A_i\}}_{i\in I}\), la loro intersection
\[
    \bigcap_{i\in I} A_i
\]
è l'insieme degli elementi che stanno in tutti gli insiemi \(A_i\), mentre la loro unione
\[
    \bigcup_{i\in I} A_i
\]
è l'insieme degli elementi che stanno in almeno uno degli insiemi \(A_i\).

\pagebreak 

\section{Classi di equivalenza}

Esempio insieme quoziente \(\sim\) su \(\mathbb{Z}\) dove \(a\sim b \iff |a|=|b|\)
è dato da
\[
    \{
        \{0 \}, \{1,-1 \}, \{2, -2 \}, \cdots    
    \}
\]

L'unica relazione di equivalenza che è un ordine è l'uguaglianza.

\pagebreak 

\section{Esempi di maggiorante etc.}

In \(\mathbb{R}\) consideriamo l'usuale ordinamento. Consideriamo i sottoinsiemi 
\[
    A = \{ x \in \mathbb{R} \,|\, x > 0 \}
\]
\[
    B = \{ x \in \mathbb{R} \,|\, x \geq 0 \}
\]
e
\[
    C = \{ x \in \mathbb{R} \,|\, 0 < x \leq 2 \}
\]

Il sottoinsieme \(A\) non ha maggioranti. Ogni numero non-positivo è minorante di \(A\).
\(A\) non ha nè massimo nè minimo.

Il sottoinsieme \(B\) non ha maggioranti. Ogni numero non-positivo è minorante di \(B\).
\(B\) ha \(0\) come minimo.

Il sottoinsieme \(C\) ha minoranti e maggioranti ma non minimo e ho \(2\) come massimo.

Consideriamo ora la relazione di divisibilità in \(\mathbb{N}\).
L'unico maggiorante è \(0\) in quanto tutti dividono zero, ed è un massimo.
Il numero \(1\) è minorante, ed è un minimo.

Se ora prendiamo l'insieme \(\{2,3,4,5\}\), i maggioranti sono mulitpli del minimo comune multiplo (\(60\)),
i minoranti sono i divisori comuni. Non ci sono massimo e minimo.

\sproposition{Il massimo è unico}{
    Il massimo, se esiste, è unico.
}

\sproof{Il massimo è unico}{
    Diciamo che \(a,b\) sono due massimi di \(A\), cioè maggioranti di \(A\) che appartiene ad \(A\).
    Abbiamo allora \(a \geq b\) (in quanto \(a\) è un maggiorante) e \(b \geq a\) (in quando \(b\) è un maggiorante).
    Abbiamo quindi che \(a=b\).
}

\sdefinition{Massimale}{
    Un elemento \(a\in A\) con \(A\) insieme partzialmente ordinato è detto massimale in \(A\)
    se non esiste alcun \(b \in A\) tale che \(a \leq b\) dove \(a\neq b\).
}

\sdefinition{Minimale}{
    Un elemento \(a\in A\) con \(A\) insieme partzialmente ordinato è detto minimale in \(A\)
    se non esiste alcun \(b \in A\) tale che \(a \geq b\) dove \(a\neq b\).
}

Ogni massimo è massimale, ogni minimo è minimale.

Esempio in cui i massimali non sono massimi: in \(\mathbb{N}\), rispetto alla divisibilità,
consideriamo l'insieme \(A= \{2,3,4,5,6\}\).

\begin{itemize}
    \item Il numero \(2\) è minimale ma non massimale.
    \item Il numero \(3\) è minimale ma non massimale.
    \item Il numero \(4\) è massimale perché non divide nient'altro, ma non minimale.
    \item Il numero \(5\) è sia massimale che minimale.
    \item Il numero \(6\) è massimale ma non minimale.
\end{itemize}

In una relazione d'ordine totale un eventuale elemento massimale è massimo.
Infatti, se \(a\) è massimale per \(A\), preso un qualsiasi elemento
\(b\in A\), sappiamo che vale almeno una tra \(a \leq b\) e \(b \leq a\).
Se vale la prima, per la definizione di massimalità di \(a\), non può essere \(a\neq b\).
Nel secondo caso, \(b \leq a\) e quindi \(a\) è un massimo.
Analogamente per i minimali.

\subsection{Relazioni irriflessiva}

Data una relazione d'ordine \(\leq\), possiamo ottenere la relazione d'ordine stretta \(<\)
dicendo che \(a<b \) se \(a\leq b\) e \(a\neq b\).

Si può definire l'ordine stretto rimpiazzando la proprietà riflessiva con quella irriflessiva.

\end{document}