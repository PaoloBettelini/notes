\documentclass[a4paper]{article}

\usepackage{amsmath}
\usepackage{amssymb}
\usepackage{stellar}
\usepackage{parskip}
\usepackage{fullpage}
\usepackage{wrapfig}
\usepackage{tikz}

\usetikzlibrary{arrows}
\usetikzlibrary{decorations.pathreplacing}

\title{Algebra I}
\author{Paolo Bettelini}
\date{}

% iscriz. Esse3
\begin{document}

\maketitle
\tableofcontents

\section{Richiami di teoria degli insiemi}

% Fu introdorra da Cantor. Poi la storia con Russel e Hilbert

% prove that this doesn't work with the difference. (che non è associativa e commutativa)

% (A \ B) \ C = (A \ C) \ B

Data una famiglia finita o infinite di insiemi \({\{A_i\}}_{i\in I}\), la loro intersection
\[
    \bigcap_{i\in I} A_i
\]
è l'insieme degli elementi che stanno in tutti gli insiemi \(A_i\), mentre la loro unione
\[
    \bigcup_{i\in I} A_i
\]
è l'insieme degli elementi che stanno in almeno uno degli insiemi \(A_i\).

\end{document}