\documentclass[a4paper]{article}

\usepackage{amsmath}
\usepackage{amssymb}
\usepackage{parskip}
\usepackage{fullpage}
\usepackage{hyperref}

\author{Paolo Bettelini}
\date{}

\begin{document}


\section{Curvature}

The tangent vector is given by
\[
    \vec{T} = \frac{d\vec{r}}{ds}
\]

TODO: a smooth curve has \(r'(t)\) continuous and \(r'(t) \neq 0\)

\newcommand\norm[1]{\left\lVert#1\right\rVert}
The curvature of a curve is given by
\[
    \kappa = \norm{\frac{d\vec{T}}{ds}}
\]
where \(\vec{T}\) is the unit tangent and s is the arc length.
By the chain rule this can also be written as
\[
    \norm{\frac{d\vec{T}}{dt}\frac{dt}{ds}} = \frac{1}{\norm{\vec{v}}}\norm{\frac{d\vec{T}}{dt}}
\]
where \(\vec{v}\) is the velocity vector.

% TODO: example, a circle

\section{Principle unit normal}

When \(\kappa \neq 0\)
\[
    \vec{N} = \frac{1}{k}\frac{d\vec{r}}{ds}
\]

The curvature tells how much the curve is curving while the principle
unit normal tells in which direction it is curving (normal to the curvature).

\section{Vectors describing motion}

Given a vector-valued curve \(r(t)\), the motion can be described by
\begin{itemize}
    \item the tangent vector (derivative of the position
        with respect to the arclength parameter), which the direction in which the curve is going.
    \item the normal vector captures the way in which the tangent vector is itself changing.
    \item Binormal vector is the cross product, which represents the vector normal to the plane
        created by the tangent vector and the normal vector. That is, it captures the torsion of said plane.
\end{itemize}

\begin{align*}
    \vec{T} &= \frac{d\vec{r}}{ds} \\
    \vec{N} &= \frac{d\vec{T}}{ds} \\
    \vec{B} &= \vec{T} \times \vec{N}
\end{align*}

The change in the binormal vector is given by
\begin{align*}
    \frac{d\vec{B}}{ds} &= \frac{d(\vec{T} \times \vec{N})}{ds}
    = \frac{d\vec{T}}{ds} \times \vec{N} + \vec{T} \times \frac{d\vec{N}}{ds} \\
    &= \kappa\vec{N} \times \vec{N} + \vec{T} \times \frac{d\vec{N}}{ds} \\
    &= \vec{T} \times \frac{d\vec{N}}{ds}
\end{align*}

Clearly, \(\frac{d\vec{B}}{ds}\) is orthogonal to both \(\vec{B}\)
and \(\vec{T}\) and thus it is parallel to \(\vec{N}\).
Indeed,
\begin{align*}
    &\tau \vec{N} = -\frac{d\vec{B}}{ds} \\
    &\tau = -\frac{d\vec{B}}{ds} \cdot \vec{N}
\end{align*}
note that the minus sign is by convention.
This value is called torsion.

\section{Acceleration}

The acceleration has both a tangential and normal component.

$\begin{aligned} \vec{a} & =\frac{d \vec{v}}{d t}=\frac{d|\vec{v}|}{d t} \vec{T}+|\vec{v}| \frac{d \vec{T}}{d t} \\ & =\frac{d|\vec{v}|}{d t} \vec{T}+|\vec{v}|\left|\frac{d \vec{T}}{d t}\right| \vec{N} \\ & =\frac{d|\vec{v}|}{d t} \vec{T}+\kappa|\vec{v}|^2 \vec{N}\end{aligned}$

\section{Differentiability}

% TODO definition

Let \(f\colon {\mathbb{R}}^m \to {\mathbb{R}}^n\) be a function.
If the partial derivatives \(\frac{\partial f}{\partial x_1}\), \(\frac{\partial f}{\partial x_2}\),
\(\cdots\), \(\frac{\partial f}{\partial x_m}\) exist and are continuous in an open region \(R\), then \(f\) is differentiable in \(R\).

\pagebreak

\section{Exercises}

\subsection{Open set 1}

Prove that the set \(A=\{(x,y) \in {\mathbb{R}}^2 \,|\, 2<x^2+y^2<4\}\)
is open.

Let \(p = (x,y)\) where \(p \in A\).
The set \(A\) is open iff \(\exists \epsilon > 0 \,|\, B_\epsilon(p) \subset A \).
Let \(d = \sqrt{x^2 + y^2}\). For a radius \(\epsilon \leq \min(d-\sqrt{2}, d-\sqrt{4})\),
the open ball \(B_\epsilon(p) \subset A\).

\subsection{Norm 1}

TODO

\subsection{Countable set}

TODO % https://planetmath.org/mathbbr2setminuscispathconnectedifciscountable

\subsection{Open set 2}

Prove that the set \(A=\{(x,y) \in {\mathbb{R}}^2 \,|\, x^2<y<x\}\)
is open.

TODO

\subsection{Sequence 1}

Consider the sequence \(\{x_k\}\) in \({\mathbb{R}}^2\)
defined by \[ x_k = \left( \sin\left(\frac{\pi k}{2}\right), \frac{{(-1)}^k}{\sqrt{k}} \right) \]
for each \(k \in {\mathbb{N}}^*\).
Determine whether \(\{x_k\}\) is bounded, and if so, find a convergent subsequence
and identity its limit.

The sinusoidal part of the pair of the sequence is bounded because \(-1 \leq \sin\theta \leq 1\).
The other part has a numerator oscillating between \(1\) and \(-1\),
and the denominator goes from \(1\) to \(+\infty\) in the limit.
Thus, the sequence is absolutely decreasing and \(-1 \leq \{x_k\} \leq \frac{1}{\sqrt{2}}\).
We now notice that
\[
    \sin\left(\frac{\pi k}{2}\right)
    = \begin{cases}
        1 \text{ or } -1, & k \text{ odd} \\
        0, & k \text{ even}
    \end{cases}
\]
By considering the subsequence where \(k\) is even we get a converging sequence
\[
    \lim_{k \to \infty} \{x_{2k}\} = (0, 0) 
\]

\subsection{Induction}

Prove \(n! > n^2\) for \(n \geq 4\).

The base case is \(4!=24 > 4^2 = 16\).

The induction step is to prove \(n! > n^2 \implies (n +1)! > {(n+1)}^2\).
Note that \((n+1)!=(n+1)n!\).
Since \(n! > n^2\), then
\begin{align*}
    n!(n+1) &> n^2(n+1) \\
    n!(n+1) &> n^3 + n^2
\end{align*}
Since \(n \geq 4\), \(n^3 + n^2 > {(n+1)}^2=n^2+2n+1\).
Thus, by the transitive property, \((n+1)! > {(n+1)}^2\).

\subsection{Sheet 6}

Let \(f(x,y,z)=yxz\) be a function and let \(\vec{a}=(1,-1,2)\) be a point.

a. Find the directional derivative of \(f\) at \(\vec{a}\) along the vector \(\vec{v}=(2,-1,2)^T\)
and along the unit vector \(\vec{e} = \frac{1}{3}(1,-2,2)^T\).

The directional derivative is given by \(\nabla f(\vec{a}) \cdot \vec{v} = -8\)
and \(\nabla f(\vec{e}) \cdot \vec{v} = -\frac{8}{3}\).

b. Let \(\vec{u}\) be a unit vector expressed in spherical coordinates, that is,
\[ \vec{u} = (\sin(\theta)\cos(\phi), \sin(\theta)\sin(\phi), \cos(\theta))^T \]
Calculate the slope of \(f\) at \(\vec{a}\) along the vector \(\vec{u}\) as a function of \((\theta, \phi)\).

The directional derivative is given by \(\nabla f(\vec{a}) \cdot \vec{u} = -2\sin\theta\cos\phi+2\sin\theta\sin\phi-\cos\theta\).

c. TODO

\pagebreak

\section{Esame Fisica}

\subsection{Es 1}

\begin{enumerate}
    \item 50 Ohm.
    \item \(\frac{120}{50}=2.4\).
    \item R2//R3 sono 6 Ohm. Quindi 2.4 * 6 / (30+6) = 0.4 Ohm
    \item 25 Ohm * 2.5 Ampere = 60 Volt.
    % Pot = R * I^2 = V * I
    \item R5, 144 W
    \item 2.4 * 5 / (1.6 * 10^-19) = 7.5 * 10^19
\end{enumerate}

\end{document}