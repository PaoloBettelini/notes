\documentclass[preview]{standalone}

\usepackage{amsmath}
\usepackage{amssymb}
\usepackage{parskip}
\usepackage{fullpage}
\usepackage{hyperref}
\usepackage{tikz}
\usepackage{stellar}
\usepackage{bettelini}

\hypersetup{
    colorlinks=true,
    linkcolor=black,
    urlcolor=blue,
    pdftitle={Assets},
    pdfpagemode=FullScreen,
}

\begin{document}

\title{Sequences}
\id{sequences}
\genpage

\section{Definition}

\begin{snippetdefinition}{sequence-definition}{Sequence}{
    A sequence denotes a series of indexed values.
    A sequence may be written as
    \[
        \{a_n\} \quad \quad {\{a_n\}}_{n=1}^\infty
    \]
}
\end{snippetdefinition}

\begin{snippetdefinition}{sequence-convergence-definition}{Sequence convergence}{
    If \(\lim_{n \to \infty}a_n\) exists and is finite
    we say that the sequence is \textit{convergent}. If 
    \(\lim_{n \to \infty}a_n\) doesn't exist or is infinite
    we say that the sequence \textit{diverges}.
}
\end{snippetdefinition}

\begin{snippetcorollary}{function-sequence-limit-equivalence}{}{
    If \(f(x)\) is a function such that \(f(n)=a_n\)
    \[
        \lim_{x\to\infty}f(x)=L \implies
        \lim_{n\to\infty}a_n=L
    \]
}
\end{snippetcorollary}

\subsection{Properties}

\begin{snippet}{sequences-properties}
if \(\{a_n\}\) and \(\{b_n\}\) are convergent sequences, then

\[
    \lim_{n\to\infty} (a_n \pm b_n) = \lim_{n\to\infty} a_n \pm
    \lim_{n\to\infty} b_n
\]
\[
    \lim_{n\to\infty} ca_n = c \lim_{n\to\infty} a_n
\]
\[
    \lim_{n\to\infty} (a_n b_n) =
    \left(\lim_{n\to\infty} a_n\right)
    \left(\lim_{n\to\infty} b_n\right)
\]
% 2 omitted
\end{snippet}

\section{Squeeze Theorem}

\begin{snippettheorem}{sequence-squeeze-theorem}{Sequence Squeeze Theorem}{
    If \(a_n \leq c_n \leq b_n\) for sufficiently large \(n>N\) for some \(N\)
    and \(\lim_{n\to\infty}a_n =\lim_{n\to\infty}b_n=L\)
    then \(\lim_{n\to\infty} c_n =L\)
}
\end{snippettheorem}

\section{Absolute Value}

\begin{snippet}{abs-value-sequence-limit}
Note the following
\[
    -|a_n| \leq a_n \leq |a_n|
\]
Then, if we assume
\[
    \lim_{n\to\infty} (-|a_n|) = - \lim_{n\to\infty} |a_n| =0 
\]
by the Squeeze Theorem we get
\[
    \lim_{n\to\infty} a_n =0
\]

We conclude that if \(\lim_{n\to\infty} |a_n|=0\) then
\(\lim_{n\to\infty} a_n=0\).
\end{snippet}

\section{Exponential sequence} % right name?

\begin{snippettheorem}{exponential-sequence-convergence}{Exponential Sequence Convergence}{
    The sequence \({\{a^n\}}_{n=0}^\infty\) converges iff \(-1<r\leq 1\)
    \[
        \lim_{n\to\infty} a^n = \begin{cases}
            0 & \text{if } -1 < a < 1 \\
            1 & \text{if } a=1
        \end{cases}
    \]
}
\end{snippettheorem}

\section{Convergence of even and odd indexes}

\begin{snippettheorem}{convergence-of-even-and-odd-indexes-theorem}{}{
    If \(\lim_{n\to\infty}a_{2n}=L\) and \(\lim_{n\to\infty}a_{2n+1}=L\)
    then \(\{a_n\}\) is convergent and \(\lim_{n\to\infty}a_n=L\).
}
\end{snippettheorem}

\begin{snippetproof}{convergence-of-even-and-odd-indexes-proof}{}{
    Let \(\epsilon>0\). \\
    Since \(\lim_{n\to\infty}a_{2n}=L\) there exists an \(N_1\) such that
    if \(n>N_1\) then
    \[
        |a_{2n}-L|<\epsilon
    \]
    Also, since \(\lim_{n\to\infty}a_{2n+1}=L\) there exists an \(N_2\) such that
    if \(n>N_2\) then
    \[
        |a_{2n+1}-L|<\epsilon
    \]
    Let \(N=\max(2N_1, 2N_2+1)\) and let \(n>N\).
    Then either \(a_n=a_{2k}\) for some \(k>N_1\) or \(a_n=a_{2k+1}\)
    for some \(k>N_2\). Either way we have
    \[
        |a_n-L|<\epsilon
    \]
    which satisfies the convergence of \(\{a_n\}\).
}
\end{snippetproof}

\section{Properties of a sequence}

\subsection{Increasing}

\begin{snippetdefinition}{increasing-sequence-definition}{Increasing Sequence}{
    A sequence is \textit{increasing} if \(\forall n, a_n<a_{n+1}\).
}
\end{snippetdefinition}

\subsection{Decreasing}

\begin{snippetdefinition}{decreasing-sequence-definition}{Decreasing Sequence}{
    A sequence is \textit{decreasing} if \(\forall n, a_n>a_{n+1}\).
}
\end{snippetdefinition}

\subsection{Monotonic}

\begin{snippetdefinition}{monotonic-sequence-definition}{Monotonic Sequence}{
    If \(\{a_n\}\) is increasing or decrasing it is also called \textit{monotonic}.
}
\end{snippetdefinition}

\subsection{Bounded below}

\begin{snippetdefinition}{bounded-below-sequence-definition}{Bounded Below Sequence}{
    If there exists a number \(m\) such that \(\forall n, m \leq a_n\)
    the sequence is \textit{bounded below} by a lower bound.
}
\end{snippetdefinition}

\subsection{Bounded above}

\begin{snippetdefinition}{bounded-above-sequence-definition}{Bounded Above Sequence}{
    If there exists a number \(m\) such that \(\forall n, m \geq a_n\)
    the sequence is \textit{bounded above} by an upper bound.
}
\end{snippetdefinition}

\subsection{Bounded}

\begin{snippetdefinition}{bounded-sequence-definition}{Bounded Sequence}{
    If the sequence is both bounded above and bounded below it is
    said \textit{bounded}.
}
\end{snippetdefinition}

\section{Convergence of bounded and monotonic sequences}

\begin{snippettheorem}{bounded-and-monotonic-sequences-convergence}{Convergence of bounded and monotonic sequences}{
    If \(a_n\) is bounded and monotonic then it is convergent.
}
\end{snippettheorem}

\begin{snippettheorem}{sequence-convergence-of-abs}{Convergence of absolute value}{
    Le \(a_n\) be a sequence.
    \[ \lim_{n\to\infty} |a_n| = 0 \implies \lim_{n\to\infty} a_n = 0 \]
}
\end{snippettheorem}

% Calculus I ok
% Calculus II
    % Series and Sequences

\end{document}