\documentclass[preview]{standalone}

\usepackage{amsmath}
\usepackage{amssymb}
\usepackage{parskip}
\usepackage{fullpage}
\usepackage{hyperref}
\usepackage{wrapfig}
\usepackage{bettelini}
\usepackage{makecell}
\usepackage{stellar}

\hypersetup{
    colorlinks=true,
    linkcolor=black,
    urlcolor=blue,
    pdftitle={ComplexAnalysis},
    pdfpagemode=FullScreen,
}

\begin{document}
  
\title{Complex Analysis}
\id{complexanalysis-differentiation}
\genpage

\section{Derivative}

\begin{snippetdefinition}{complex-derivative-definition}{Complex Derivative}{
    Let \(f(z)\) be a complex-valued function
    of a complex value which can be written as
    \(f(x+iy)=u(x,y)+iv(x,y)\). Then
    \[
        f'(z) = \lim_{\Delta z \to 0} \frac{f(z_0 + \Delta z)-f(z)}{\Delta z}
    \]
}
\end{snippetdefinition}

\section{Cauchy-Riemann Equations}

\begin{snippetdefinition}{cauchy-riemann-equations-definition}{Cauchy-Riemann Equations}{
    Let \(f(x+iy)=u(x,y)+iv(x,y)\) be a complex values function
    where \(u(x,y)\) and \(v(x,y)\) are real functions.
    If \(f'(z)\) exists at a point, then the equations
    \[
        \frac{\partial u}{\partial x}=\frac{\partial v}{\partial y},
        \quad
        -\frac{\partial u}{\partial y}=\frac{\partial v}{\partial x}
    \]
    must hold at that point.
}
\end{snippetdefinition}

\begin{snippetproof}{cauchy-riemann-equations-proof}{Cauchy-Riemann Equations}{
    Let us write \(\Delta z = \Delta x + i\Delta y\). \\
    We now compute \(f'(z)\) by approaching \(z\) from the
    horizontal direction \((\Delta y=0)\).
    \begin{align*}
        f'(z_0) &= \lim_{\Delta x \to 0} \frac{f(z + \Delta x) - f(z)}{\Delta x} \\
        &= \lim_{\Delta x \to 0}
        \frac{f(x + \Delta x + iy) - f(x + iy)}{\Delta x} \\
        &= \lim_{\Delta x \to 0}
        \frac{(u(x + \Delta x, y) + iv(x + \Delta x, y)) - (u(x,y)+iv(x,y))}{\Delta x} \\
        &= \lim_{\Delta x \to 0}
        \frac{u(x + \Delta x, y) - u(x,y)}{\Delta x} + i\frac{v(x + \Delta x, y) - v(x,y)}{\Delta x} \\
        &= \frac{\partial u}{\partial x} + i \frac{\partial v}{\partial x}
    \end{align*}
    We now compute \(f'(z)\) by approaching \(z\) from the
    vertical direction \((\Delta x=0)\).
    \begin{align*}
        f'(z_0) &= \lim_{\Delta y \to 0} \frac{f(z + \Delta y) - f(z)}{i\Delta y} \\
        &= \lim_{\Delta y \to 0}
        \frac{f(x + iy + i\Delta y) - f(x + iy)}{i\Delta y} \\
        &= \lim_{\Delta y \to 0}
        \frac{(u(x, y + \Delta y) + iv(x, y + \Delta y)) - (u(x,y)+iv(x,y))}{i\Delta y} \\
        &= \lim_{\Delta y \to 0}
        \frac{u(x, y + \Delta y) - u(x,y)}{i\Delta y} + i\frac{v(x, y + \Delta y) - v(x,y)}{i\Delta y} \\
        &= \frac{\partial v}{\partial y} -i\frac{\partial u}{\partial y}
    \end{align*}
    
    We have found two different representations of \(f'(z)\) in terms
    of the partial derivatives of \(u\) and \(v\).
    \[
        f'(z)=\frac{\partial u}{\partial x} +i\frac{\partial v}{\partial x}
        = \frac{\partial v}{\partial y} -i \frac{\partial u}{\partial y}
    \]
    
    From this equality we can derive the Cauchy-Riemann equations.
}
\end{snippetproof}

\section{Sufficient condition}

\begin{snippettheorem}{complex-differentiability-sufficient-condition}{Complex differentiability sufficient condition}{
    A complex function \(f(z)\) is differentiable at a point \(z\)
    iff
    \begin{enumerate}
        \item The Cauchy-Riemann equations hold at \(z\).
        \item \(\frac{\partial u}{\partial x}, \frac{\partial v}{\partial y},
        \frac{\partial u}{\partial y}\) and \(\frac{\partial v}{\partial x}\)
        are continuous.
    \end{enumerate}
}
\end{snippettheorem}

\section{Holomorphic functions}

\subsection{Definition}

\begin{snippetdefinition}{holomorphic-function-definition}{Holomorphic Function}{
    A function is holomorphic in \(\Omega\) if it is complex differentiable
    in a neighbourhood of each point of \(\Omega\).
}
\end{snippetdefinition}

\begin{snippettheorem}{analyticity-of-holomorphic-functions}{Analyticity of holomorphic functions}{
    Let \(f(z)\) be a complex-valued function.
    Then, \(f(z)\) is holomorphic if and only if it is analytic.
}
\end{snippettheorem}

% TODO
% analytic -> f is infinitely differentiable
% analytic -> iff it is locally given by a convergent power series. 

\subsection{Sufficient condition for analytic functions}

A function \(f(z)\) is analytic in a region \(\Omega\) if it is differentable
in a neighborhood of every point in \(\Omega\).

\subsection{Entire functions}

\begin{snippetdefinition}{entire-function-definition}{Entire Function}{
    A complex-valued function \(f(z)\) is \textit{entire} if it is analytic on the complete complex plane.
}
\end{snippetdefinition}

\end{document}