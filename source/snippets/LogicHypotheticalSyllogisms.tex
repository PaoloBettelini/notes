\documentclass[preview]{standalone}

\usepackage{amsmath}
\usepackage{amssymb}
\usepackage{parskip}
\usepackage{fullpage}
\usepackage{hyperref}
\usepackage{stellar}

\hypersetup{
    colorlinks=true,
    linkcolor=black,
    urlcolor=blue,
    pdftitle={Logic},
    pdfpagemode=FullScreen,
}

\begin{document}

\title{Hypothetical Syllogisms}
\id{hypothetical-syllogisms}
\genpage

\section{Definition}

\begin{snippetdefinition}{hypothetical-syllogism-definition}{Hypothetical syllogism}{
    A \textit{hypothetical syllogism} is the name of a valid rule of inference on formulae.
    Any axiomatic system needs rules of inference to infer true propositions.
}
\end{snippetdefinition}


\section{Examples}

\subsection{Modus Ponens}

\begin{snippet}{logic-modus-ponens-desc}
\textit{Modus Ponens} or \textit{affirming the antecedent}
is a \underline{valid} form hypothetical syllogism. \\
If \(P\) implies \(Q\) and \(P\) is true, then \(Q\) is also true.
\end{snippet}

\includesnpt{logic-modus-ponens}

\subsection{Modus Tollens}

\begin{snippet}{logic-modus-tollens-desc}
\textit{Modus Tollens} or \textit{denying the consequent}
is a \underline{valid} form hypothetical syllogism. \\
If \(P\) implies \(Q\) and \(Q\) is false, then \(P\) is also false.
\end{snippet}

\includesnpt{logic-modus-tollens}

\subsection{Fallacy of affirming the consequent}

\begin{snippet}{logic-fallacy-of-affirming-the-consequent-desc}
\textit{Affirming the consequent}
is an \underline{invalid} form hypothetical syllogism. \\
If \(P\) implies \(Q\) and \(Q\) is true, then \(P\) is also true.
\end{snippet}

\includesnpt{logic-fallacy-of-affirming-the-consequent}

\subsection{Fallacy of denying the antecedent}

\begin{snippet}{logic-fallacy-of-denying-the-antecedent-desc}
\textit{Denying the antecedent}
is an \underline{invalid} form hypothetical syllogism. \\
If \(P\) implies \(Q\) and \(P\) is false, then \(Q\) is also false.
\end{snippet}

\includesnpt{logic-fallacy-of-denying-the-antecedent}

\end{document}
