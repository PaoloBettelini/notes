\documentclass[a4paper]{article}

\usepackage{amsmath}
\usepackage{amssymb}
\usepackage{parskip}
\usepackage{fullpage}
\usepackage{hyperref}

\hypersetup{
    colorlinks=true,
    linkcolor=black,
    urlcolor=blue,
    pdftitle={Generic},
    pdfpagemode=FullScreen,
}

\title{Generic}
\author{Paolo Bettelini}
\date{}

\begin{document}

\section{Plane}

A plane can be uniquely represented by its
normal vector \(\vec{n}\)
and a point on the plane \(P_0\).

To describe the plane using an equation, we can
consider an arbitrary point \(P=(x,y,z)\) on the plane.
There is always a 90 degrees angle between the normal
vector and the vector from \(P_0\) to \(P\) (i.e., their dot product is zero)

\[
    \vec{n} \cdot \overrightarrow{P_0 P} = 0
\]
By plugging in the values for \(\vec{n}\) and \(P_0\)
we get an equation in the form
\[
    Ax+By+Cz+D=0
\]

\section{Vector-Valued Function}

A vector-valued function is a function of a real parameter
which returns a vector
\[
    r(t) = \begin{pmatrix}
        f(t) \\
        g(t) \\
        h(t)
    \end{pmatrix}
\]

\section{Tangent Vector Vector-Valued Function}

Given a vector-valued function
\[
    r(t) = \begin{pmatrix}
        f(t) \\
        g(t) \\
        h(t)
    \end{pmatrix}
\]

where \(f\), \(g\) and \(h\) are differentiable,
then the Tangent vector to the curve is given b
\[
    r'(t) = \begin{pmatrix}
        f'(t) \\
        g'(t) \\
        h'(t)
    \end{pmatrix}
\]

\section{Curve length}

The length of the curve beteen a and b is the integral from a to b of \(\sqrt{f'(t)^2+g'(t)^2+h'(t)^2}\).
% https://youtu.be/40r56pX4mqA?list=PLHXZ9OQGMqxc_CvEy7xBKRQr6I214QJcd

\pagebreak

\section{Exercises}

\subsection{Open set 1}

Prove that the set \(A=\{(x,y) \in {\mathbb{R}}^2 \,|\, 2<x^2+y^2<4\}\)
is open.

Let \(p = (x,y)\) where \(p \in A\).
The set \(A\) is open iff \(\exists \epsilon > 0 \,|\, B_\epsilon(p) \subset A \).
Let \(d = \sqrt{x^2 + y^2}\). For a radius \(\epsilon \leq \min(d-\sqrt{2}, d-\sqrt{4})\),
the open ball \(B_\epsilon(p) \subset A\).

\subsection{Norm 1}

TODO

\subsection{Countable set}

TODO % https://planetmath.org/mathbbr2setminuscispathconnectedifciscountable

\subsection{Open set 2}

Prove that the set \(A=\{(x,y) \in {\mathbb{R}}^2 \,|\, x^2<y<x\}\)
is open.

TODO

\subsection{Sequence 1}

Consider the sequence \(\{x_k\}\) in \({\mathbb{R}}^2\)
defined by \[ x_k = \left( \sin\left(\frac{\pi k}{2}\right), \frac{{(-1)}^k}{\sqrt{k}} \right) \]
for each \(k \in {\mathbb{N}}^*\).
Determine whether \(\{x_k\}\) is bounded, and if so, find a convergent subsequence
and identity its limit.

The sinusoidal part of the pair of the sequence is bounded because \(-1 \leq \sin\theta \leq 1\).
The other part has a numerator oscillating between \(1\) and \(-1\),
and the denominator goes from \(1\) to \(+\infty\) in the limit.
Thus, the sequence is absolutely decreasing and \(-1 \leq \{x_k\} \leq \frac{1}{\sqrt{2}}\).
We now notice that
\[
    \sin\left(\frac{\pi k}{2}\right)
    = \begin{cases}
        1 \text{ or } -1, & k \text{ odd} \\
        0, & k \text{ even}
    \end{cases}
\]
By considering the subsequence where \(k\) is even we get a converging sequence
\[
    \lim_{k \to \infty} \{x_{2k}\} = (0, 0) 
\]

\subsection{Level curves}

% TODO: the level curve is the curve on the horizontal plane intersecting a 2 variables function

Find the equation of the level curve of the function \(f(x,y)\) that passes through
the given point p.
\begin{enumerate}
    \item \(f(x,y) = 16-x^2-y^2\) and \(p=(2\sqrt{2}, \sqrt{2})\);
    \item \(f(x,y) = \sqrt{x^2-1}\) and \(p=(1,0)\);
    \item \(f(x,y) = \int_x^y \frac{d\theta}{\sqrt{1-\theta^2}}\) and \(p=(0,1)\).
\end{enumerate}

\begin{enumerate}
    \item \(f(2\sqrt{2}, \sqrt{2}) = 6\), so the height of the plane is 6.
        By plugging \(z=6\) in we get \(6=16-x^2-y^2\)
        and thus the level curve is given by \(10=x^2+y^2\);
    \item \(f(1, 0) = 0\), so the height of the plane is 0.
        By plugging \(z=0\) in we get \(0 = \sqrt{x^2-1}\)
        and thus the level curve is given by \(x^2=1\), that is
        the lines \(x=1\) and \(x=-1\).
    \item We first solve the integral
    \[ \int_x^y \frac{d\theta}{\sqrt{1-\theta^2}} = \arcsin(y) - \arcsin(x) \]
    Now we can evaluate \(f(0,1) = \arcsin(1) - \arcsin(0) = \frac{\pi}{2}\),
    so the height of the plane is \(\frac{\pi}{2}\).
    By plugging \(z=\frac{\pi}{2}\) in we get
    \begin{align*}
        &\frac{\pi}{2} = \arcsin(y) - \arcsin(x) \\
        &y = \sin\left( \frac{\pi}{2} + \arcsin(x) \right) \\
        &y = \sqrt{1-x^2}
    \end{align*}
    with \(x < 0\).
\end{enumerate}

\subsection{Limits 1}

\begin{enumerate}
    \item \(\lim _{(x, y) \rightarrow(2,1)} \frac{x^2-3 y}{x+2 y^2}\);
    \item \(\lim _{(x, y) \rightarrow(0,0)} \frac{\sin \left(x^2+y^2\right)}{x^2+y^2}\);
    \item \(\lim _{(x, y) \rightarrow(0,0)} \frac{x y^2}{x^2+y^4}\);
    \item \(\lim _{(x, y) \rightarrow(0,0)} \frac{x^2 y}{x^2+y^4}\).
\end{enumerate}

\end{document}