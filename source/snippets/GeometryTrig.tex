\documentclass[preview]{standalone}

\usepackage{amsmath}
\usepackage{amssymb}
\usepackage{parskip}
\usepackage{fullpage}
\usepackage{hyperref}
\usepackage{stellar}
\usepackage{bettelini}

\hypersetup{
    colorlinks=true,
    linkcolor=black,
    urlcolor=blue,
    pdftitle={Geometry},
    pdfpagemode=FullScreen,
}

\begin{document}

\title{Geometry}
\id{trigonometry}
\genpage

\section{Laws}

\begin{snippettheorem}{law-of-sines}{Law of sines}
    Given a triangle with sides \(a\), \(b\) and \(c\) and their respective opposite angles
    \(\alpha\), \(\beta\) and \(\gamma\)
    \[
        \frac{\sin(\alpha)}{a} =
        \frac{\sin(\beta)}{b} =
        \frac{\sin(\gamma)}{c}
    \]
\end{snippettheorem}

\begin{snippettheorem}{law-of-cosines}{Law of cosines}
    Given a triangle with sides \(a\), \(b\) and \(c\)
    \[
        c^2 = a^2 + b^2 - 2ab\cos\gamma
    \]
    where \(\gamma\) is the angle between \(a\) and \(b\) (opposite of \(c\)).
\end{snippettheorem}

\section{Pythagorean identities}

\begin{snippettheorem}{pythagorean-identity}{Pythagorean identity}
    For any \(\theta \in \mathbb{R}\),
    \[
        \sin^2\theta + \cos^2\theta = 1
    \]
\end{snippettheorem}

\end{document}