\documentclass[preview]{standalone}

\usepackage{amsmath}
\usepackage{amssymb}
\usepackage{parskip}
\usepackage{fullpage}
\usepackage{hyperref}
\usepackage{bettelini}
\usepackage{stellar}

\hypersetup{
    colorlinks=true,
    linkcolor=black,
    urlcolor=blue,
    pdftitle={Metric Spaces},
    pdfpagemode=FullScreen,
}

\begin{document}

\title{Metric Spaces}
\id{metric-spaces}
\genpage

\section{Definitions}

\begin{snippetdefinition}{metricspaces-distance-function-definition}{Distance Function}{
    Let \(X\) be a set.
    A \textit{distance function} on \(X\) is
    a function \(d: X \times X \to \mathbb{R}\)
    with the following properties for \(x,y,z \in X\):
    \begin{itemize}
        \item \(d(x,y) = 0 \iff x = y\).
        \item \textbf{Positivity:} \(d(x,y) > 0 \iff x \neq y\).
        \item \textbf{Simmetry:} \(d(x,y) = d(x,y)\).
        \item \textbf{Triangle inequality:} \(d(x,z) \leq d(x,y) + d(y,z)\).
    \end{itemize}
}
\end{snippetdefinition}

\begin{snippetdefinition}{metricspaces-metric-space-definition}{Metric Space}{
    A \textit{metric space} is a tuple \((X, d)\)
    consisting of a set \(X\) and a distance function \(d\). % TODO cite
}
\end{snippetdefinition}

\section{Reverse triangle inequality}

\begin{snippetlemma}{metricspaces-reverse-triangle-inequality}{Reverse triangle inequality}{
    Given a metric space \((X,d)\). Let \(x,y,z \in X\).
    Then, \[|d(x,y) - d(x,z)| \leq d(y,z)\]
}
\end{snippetlemma}

\begin{snippetproof}{metricspaces-reverse-triangle-inequality-proof}{Reverse triangle inequality}{
    The absolute values can be removed and the lemma expressed as
    \[
        d(x,y) - d(x,z) \leq d(y,z)
    \]
    and
    \[
        d(x,z) - d(x,y) \leq d(y,z)
    \]
    Both follow from the triangle inequality.
}
\end{snippetproof}

\section{Examples}

% TODO

\section{Balls}

\subsection{Definitions}

\begin{snippetdefinition}{metricspaces-open-ball-definition}{Open Ball}{
    Let \((X, d)\) be a metric space.
    An \textit{open ball} of radius \(\epsilon\) around a point
    \(a \in X\) is defined as:
    \[
        B_\epsilon(a) = \{x \in X \suchthat d(x,a) < \epsilon\}.
    \]
}
\end{snippetdefinition}

\begin{snippetdefinition}{metricspaces-closed-ball-definition}{Closed Open Ball}{
    Let \((X, d)\) be a metric space.
    A \textit{closed ball} of radius \(\epsilon\) around a point
    \(a \in X\) is defined as:
    \[
        \overline{B}_\epsilon(a) = \{x \in X \suchthat d(x,a) \leq \epsilon\}.
    \]
}
\end{snippetdefinition}

\begin{snippetdefinition}{metricspaces-bounded-set}{Bounded Set}{
    Let \((X, d)\) be a metric space and \(Y \subseteq X\).
    The set \(Y\) is said to be \textit{bounded} if \(Y\)
    is contained in some open ball.
}
\end{snippetdefinition}

\subsection{Bounded sets}

\begin{snippetlemma}{metricspaces-boundness-equivalence}{todo}{
    Let \((X, d)\) be a metric space and \(Y \subseteq X\).
    The following statements are equivalence:
    \begin{itemize}
        \item \(Y\) is contained in some open ball (bounded).
        \item \(Y\) is contained in some closed ball.
        \item The set \(\{d(y_1,y_2) \suchthat y_1, y_2 \in Y\}\)
            is a bounded subset of \(\mathbb{R}\).
    \end{itemize}
}
\end{snippetlemma}

\begin{snippetproof}{metricspaces-boundness-equivalence-proof}{todo}{
    TODO
}
\end{snippetproof}

\end{document}
