\documentclass[preview]{standalone}

\usepackage{amsmath}
\usepackage{amssymb}
\usepackage{parskip}
\usepackage{fullpage}
\usepackage{hyperref}
\usepackage{wrapfig}
\usepackage{bettelini}
\usepackage{makecell}
\usepackage{stellar}

\hypersetup{
    colorlinks=true,
    linkcolor=black,
    urlcolor=blue,
    pdftitle={ComplexAnalysis},
    pdfpagemode=FullScreen,
}

\begin{document}

\title{Complex Analysis}
\id{complexanalysis-exercises}
\genpage

\section{De Moivre}

\begin{snippetexercise}{complex-analysis-demoivre-ex-1}{}
    Prove that \(\cos5\theta = 16\cos^5\theta - 20\cos^3\theta+5\cos\theta\).
\end{snippetexercise}

\begin{snippetsolution}{complex-analysis-demoivre-ex-1-sol}{}
    We start using De Moivre's theorem
    \begin{align*}
        \cos5\theta + i\sin5\theta &= (\cos\theta + i\sin\theta)^5 \\
        &= \cos^5\theta - 10\cos^3\theta\sin^2\theta + 5\cos\theta\sin^4\theta \\
            &+ i(5\cos^4\theta\sin\theta -10\cos^2\theta\sin^3\theta + \sin^5\theta)
    \end{align*}
    and thus
    \[ \cos5\theta = \cos^5\theta - 10\cos^3\theta\sin^2\theta + 5\cos\theta\sin^4\theta \]
\end{snippetsolution}

\begin{snippetexercise}{complex-analysis-demoivre-ex-2}{}
    Determine all the fifth roots of unity.
\end{snippetexercise}

\begin{snippetsolution}{complex-analysis-demoivre-ex-2-sol}{}
    \begin{align*}
        z^5 = 1 &= \cos (2k\pi) + i\sin(2k\pi) = e^{2k\pi i} \\
        z &= \cos \frac{2k\pi}{5} + i \sin \frac{2k\pi}{5} = e^{2k\pi i / 5}
    \end{align*}
    Let \(\omega = e^{2k\pi i}\). Using \(k=0,1,2,3,4\) since the others are redundant, we get
    \(1, \omega^1, \omega^2, \omega^3, \omega^4\).
\end{snippetsolution}

\begin{snippetexercise}{complex-analysis-demoivre-ex-3}{}
    Find all the complex solutions to
    \[ z^5 = -32 \]
\end{snippetexercise}

\begin{snippetsolution}{complex-analysis-demoivre-ex-3-sol}{}
    \[  z^5 = -32 = 32\left( \cos(\pi + 2k\pi) + i\sin(\pi + 2k\pi) \right) \]
    By De Moivre's theorem
    \[
        z = 2\left( 
            \cos\left(\frac{\pi + 2k\pi}{5}\right) + i\sin\left(\frac{\pi + 2k\pi}{5}\right)
        \right)
    \]
    for \(k=0,1,2,3,4\).
\end{snippetsolution}

\begin{snippetexercise}{complex-analysis-demoivre-ex-4}{}
    Find all the complex solutions to
    \[ z^3 = i-1 \]
\end{snippetexercise}

\begin{snippetsolution}{complex-analysis-demoivre-ex-4-sol}{}
    The point \(i-1\) has an absolute value of \(\sqrt{2}\)
    and an argument of \(3\pi/4\).
    \begin{align*}
        i-1 &= \sqrt{2} \left(
            \cos(3\pi/4 +2k\pi) + i\sin(3\pi/4 +2k\pi)
        \right) \\
        (i-1)^{1/3} &= 2^{1/6}
        \left(
            \cos\left( \frac{3\pi/4 + 2k\pi}{3} \right)
            + i \sin\left( \frac{3\pi/4 + 2k\pi}{3} \right)
        \right)
    \end{align*}
    for \(k=0,1,2\).
\end{snippetsolution}

\section{Limits}

\begin{snippetexercise}{complex-analysis-limits-ex-1}{}
    Prove that \[ \lim_{z \to 0} \dfrac{\overline{z}}{z} \]
    does not exist.
\end{snippetexercise}

\begin{snippetsolution}{complex-analysis-limits-ex-1-sol}{}
    If the limits exists, then it must be equal from all directions.
    If we approach the point from the real axis,
    \(z = \overline{z}\) and thus
    \[ \lim_{z \to 0} \dfrac{\overline{z}}{z} = \frac{z}{z} = 1 \]. \\
    If we approach the point from the imaginary axis,
    \( \overline{bi} = -bi \) and
    \[ \lim_{z \to 0} \dfrac{\overline{z}}{z} = \frac{-bi}{bi} = -1 \]
    and thus the limit does not exist.
\end{snippetsolution}

\section{Complex integrals}

\begin{snippetexercise}{complex-analysis-integrals-ex-1}{}
    Compute \[ \oint_\Omega \frac{e^z}{{(z^2 + \pi^2)}^2} \,dz = 0 \]
    where \(\Omega\) is the circle \(|z|=4\).
\end{snippetexercise}

\begin{snippetsolution}{complex-analysis-integrals-ex-1-sol}{}
    The poles of \(\frac{e^z}{{(z^2 + \pi^2)}^2} = \frac{e^z}{(z-\pi i)^2(z+\pi i)^2}\)
    are at \(z=\pm \pi i\) in \(\Omega\) and are both of order 2.

    The residue in \(z=\pm \pi i\) is
    \[ \lim_{z \to \pi i} \frac{1}{1!} \frac{d}{dz} \left[
        (z - \pi i)^2 \frac{e^z}{(z-\pi i)^2(z+\pi i)^2}
    \right] = \frac{\pi + i}{4\pi^3}\]

    The residue in \(z=-\pi i\) is
    \[ \lim_{z \to -\pi i} \frac{1}{1!} \frac{d}{dz} \left[
        (z + \pi i)^2 \frac{e^z}{(z-\pi i)^2(z+\pi i)^2}
    \right] = \frac{\pi - i}{4\pi^3}\]

    Thus, the integral is given by \(2\pi i \left( \frac{\pi + i}{4\pi^3} + \frac{\pi - i}{4\pi^3} \right) = \frac{i}{\pi}\).
\end{snippetsolution}

\begin{snippetexercise}{complex-analysis-integrals-ex-2}{}
    Compute \[ \oint_\Omega \frac{e^{iz}}{z^3} \,dz = 0 \]
    where \(\Omega\) is the circle \(|z|=2\).
\end{snippetexercise}

\begin{snippetsolution}{complex-analysis-integrals-ex-2-sol}{}
    The poles is at \(z=0\) and has an order of 3.

    The residue is
    \begin{align*}
        & \frac{1}{2!} \lim_{z \to 0} \frac{d^2}{dz^2} z^3 \frac{e^{iz}}{z^3} \\
        &-\frac{1}{2} e^{iz} = -\frac{1}{2}
    \end{align*}
    and thus the integral is \(2\pi i (-\frac{1}{2}) = -\pi i\).
\end{snippetsolution}

\end{document}