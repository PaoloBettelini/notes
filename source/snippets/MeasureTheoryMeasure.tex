\documentclass[preview]{standalone}

\usepackage{amsmath}
\usepackage{amssymb}
\usepackage{parskip}
\usepackage{fullpage}
\usepackage{hyperref}
\usepackage{bettelini}
\usepackage{stellar}

\hypersetup{
    colorlinks=true,
    linkcolor=black,
    urlcolor=blue,
    pdftitle={Measure Theory},
    pdfpagemode=FullScreen,
}

\begin{document}

\title{Measure Theory}
\id{measuretheory-measure}
\genpage

\begin{snippetdefinition}{measurable-space-definition}{Measurable space}
    A \textit{measurable space} is a tuple \((X, \Sigma)\)
    where \(X\) is a set and \(\Sigma\) or a \(\sigma\)-algebra over \(X\).
\end{snippetdefinition}

\begin{snippetdefinition}{measure-definition}{Measure}
    Let \(X\) be a set and \(\Sigma\) be a \(\sigma\)-algebra over \(X\).
    A \textit{measure} is a function
    \[
        \mu\colon \Sigma \to [-\infty; + \infty]
    \]
    where
    \begin{enumerate}
        \item \(\mu(\emptyset) = 0\);
        \item \textbf{non-negative:} \(\forall A \in \Sigma, \mu(A) \geq 0\);
        \item \textbf{countable additive:} for all countable collections \({\{A_k\}}_{k=1}^\infty\)
        of pairwise disjoint sets in \(\Sigma\)
        \[
            \mu\left( \bigcup_{k=1}^\infty E_k \right)
            =
            \sum_{k=1}^\infty \mu(A_k)
        \]
    \end{enumerate}
\end{snippetdefinition}

\begin{snippetdefinition}{measure-space-definition}{Measure space}
    A \textit{measure space} is a triple \((X, \Sigma, \mu)\)
    where \(X\) is a set, \(\Sigma\) or a \(\sigma\)-algebra on \(X\)
    and \(\mu\) is a measure on \((X, \Sigma)\).
\end{snippetdefinition}

\section{Examples}

\begin{snippetdefinition}{counting-measure-definition}{Counting measure space}
    Let \(X\) be a set and \(\Sigma = \mathcal{P}(X)\) be a \(\sigma\)-algebra.
    The \textit{counting measure space}
    is defined as the measure space \((X, \Sigma, \mu)\)
    where
    \[
        \mu(A) = \begin{cases}
            |A| & A \text{ is finite} \\
            +\infty & \text{ otherwise}
        \end{cases}
    \]
    for all \(A \in \Sigma\).
\end{snippetdefinition}

% dirac measure

\end{document}
