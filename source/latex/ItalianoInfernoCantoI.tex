\documentclass[preview]{standalone}

\usepackage{amsmath}
\usepackage{amssymb}
\usepackage{stellar}
\usepackage{bettelini}

\hypersetup{
    colorlinks=true,
    linkcolor=black,
    urlcolor=blue,
    pdftitle={Stellar},
    pdfpagemode=FullScreen,
}

\begin{document}

\title{Stellar}
\id{italiano-inferno-canto-i}
\genpage

\section{Canto I}

\plain{Il primo canto dell'inferno fa da proemio a tutta la Commedia.}

\begin{snippet}{inferno-canto-i-1}
    \StellarPoetry{1}{
        Nel mezzo del cammin di nostra vita \\
        mi ritrovai per una selva oscura, \\
        ché la diritta via era smarrita.
    }{
        Nell'età di mezzo della vita umana \\
        mi ritrovai in una buia boscaglia \\
        perché avevo smarrito il giusto percorso.
    }

    La vita media è di 70 anni. Questo è un dato biblico e non il valore della vita media.
    Dante è nato nel 1265, per cui 1265+35=1300. Il percorso inizia infatti esattamente nel 1300.
    Questo dato viene anche confermato in \textit{Inferno XXI, 112-114}.

    La selva oscura, dove non vi è luce, rappresenta una condizione di peccato.
    La \textbf{diritta via} è quella che conduce a Dio.
    Essa è smarrita, ma può essere appunto ritrovata.
    \\
    Dante non specifica il tipo di peccato, questo è dato dal fatto che Dante rappresenta allegoricamente
    l'interno dell'umanità (nel 1300), per cui il peccato di tutti gli uomini in quel periodo.
\end{snippet}

\begin{snippetdefinition}{giubileo-definition}{Giubileo}
    Il \textit{giubileo} è un anno di assoluzione collettiva di peccati.
\end{snippetdefinition}

\plain{Questa data è quella del primo Giubileo, indetto dal Papa Bonifacio VIII.}

\begin{snippetdefinition}{allegoria-definition}{Allegoria}
    Figura retorica per mezzo della quale l'autore esprime e il lettore ravvisa un significato riposto,
    diverso da quello letterale.
\end{snippetdefinition}

\begin{snippet}{inferno-canto-i-4}
    \StellarPoetry{4}{
        Ahi quanto a dir qual era è cosa dura \\
        esta selva \textbf{selvaggia} e \textbf{aspra} e \textbf{forte} \\
        che nel pensier rinova la paura!
    }{
        Ahimé, non è affatto facile descrivere \\
        questo bosco inospitale, impervio, difficile, \\
        del quale il solo pensiero mi fa tornare il timore!
    }

    I tre aggettivi; selvaggia (disumano), aspra (fitta) e forte (da cui è difficile uscire)
    sono disposti a climax.
\end{snippet}

\begin{snippet}{inferno-canto-i-7}
    \StellarPoetry{7}{
        Tant' è amara che poco è più morte; \\
        ma per trattar del ben ch'i' vi trovai, \\
        dirò de l'altre cose ch'i' v'ho scorte.
    }{
        [la selva] È tanto angosciante quasi quanto la morte; \\
        ma per dire ciò che di buono lì incontrai, \\
        parlerò [prima] delle altre cose che lì ho viste.
    }
    
    La morte, che è la cosa più terribile che ci sia, lo è solamente poco più della selva.
    \\
    I due verbi sui quali si chiude \textit{Vita Nova},
    \textbf{dire} e \textbf{trattare}, si ritrovano all'inizio della \textit{Commedia}.
    In \textit{Vita Nova} questi verbi si riferiscono all'io poetico, mentre all'inizio della \textit{Commedia}
    sono riferiti al \textbf{bene} e ad \textbf{altre cose}.
    Il bene si riferisce alla salvezza (Dio), mentre altre cose si riferisce a tutto ciò che trovò durante il viaggio.
    La parola \textit{vi} si riferisce probabilmente a tutto il viaggio compiuto da Dante.
\end{snippet}

\begin{snippet}{inferno-canto-i-10}
    \StellarPoetry{10}{
        Io non so ben ridir com' i' v'intrai, \\
        tant' era pien di sonno a quel punto \\
        che la verace via abbandonai.
    }{
        Non so descrivere il modo in cui vi entrai \\
        dato che il mio torpore era tale in quel momento \\
        che mi ero allontanato dalla verità.
    }
    
    Il sonno rappresenta in senso allegorico il sonno della coscienza,
    che porta al peccato, ossia la selva.
\end{snippet}

\begin{snippetnote}{funzioni-di-dante}{Funzioni di Dante}
    Date ha diverse funzioni che si intrecciano nel racconto.
    \begin{itemize}
        \item Dante personaggio, pellegrino che compie il viaggio
        \item Dante allegoria per tutta l'umanità
        \item Dante poeta fiorentino
    \end{itemize}
\end{snippetnote}

\begin{snippet}{inferno-canto-i-13}
    \StellarPoetry{13}{
        Ma poi ch'i' fui al piè d'un colle giunto, \\
        là dove terminava quella valle \\
        che m'avea di paura il cor compunto,
    }{
        Ma dopo che arrivai alle pendici d'una collina, \\
        nel luogo in cui finiva quel bosco \\
        che mi aveva impietrito il cuore di paura,
    }
\end{snippet}

\begin{snippet}{inferno-canto-i-16}
    \StellarPoetry{16}{
        guardai in alto e vidi le sue spalle \\
        vestite già de' raggi del pianeta \\
        che mena dritto altrui per ogne calle.
    }{
        alzai gli occhi e vidi la sua cima e il pendio \\
        già illuminati dai raggi di quel pianeta [il Sole] \\
        che guida ciascuno sulla giusta via.
    }

    Le spalle del colle sono il punto in cui la collina si piega.
    Questa perifrasi indica semplicemente che la collina era illuminata dalla luce solare.
    In alto vi è la luce divina, mentre in basso c'è il buio del peccato.
    Il colle rappresenta infatti il percorso difficile; è molto più facile
    cadere all'inferno che giungere a Dio. \\
    Il gesto di guardare in alto indica un progressivo distaccarsi dal peccato.
    Selva e sole sono due opposti, e Dante alzi gli occhi per guardare verso Dio.
    In questo punto vengono descritti i tre regni della Commedia; la selva rappresenta
    l'inferno, il colle il purgatorio e il Sole il paradiso.
\end{snippet}

\begin{snippet}{inferno-canto-i-19}
    \StellarPoetry{19}{
        Allor fu la paura un poco queta, \\
        che nel lago del cor m'era durata \\
        la notte ch'i' passai con tanta pieta.
    }{
        A quel punto si calmò quel timore \\
        che nel profondo dell'animo avevo sofferto \\
        durante la notte [precedente] trascorsa nel dolore.
    }
\end{snippet}

Dante alza quindi gli occhi verso il Cielo per avere speranza.

\begin{snippet}{inferno-canto-i-22}
    \StellarPoetry{22}{
        E come quei che con \textbf{lena affannata}, \\
        uscito fuor del pelago a la riva, \\
        si volge a l'acqua perigliosa e guata,
    }{
        E come colui che con respiro affaticato, \\
        uscito dal mare e arrivato alla spiaggia, si gira \\
        verso lo specchio d'acqua minaccioso e [lo] guarda;
    }

    % pelago = mare
    Il verbo \textbf{guatare} significa guardare con partecipazione, spesso paura.
    Questa similitudine mette in confronto un naufrago che scampa il pericolo dell'acqua,
    che come Dante scampa dalla selva e si gira a guardarla, con un sentimento di sollievo.
    Il corpo di Dante è fermo, ma il suo animo è ancora spaventato e vorrebbe continuare a scappare.
    
    Questo indica anche che indugiare nel peccato, come indugiare nella selva o nelle acque, porta alla morte.
    Questa è anche la prima similitudine della Commedia.
\end{snippet}

\begin{snippet}{inferno-canto-i-25}
    \StellarPoetry{25}{
        così l'animo mio, ch'ancor fuggiva, \\
        si volse a retro a rimirar lo passo \\
        che non lasciò già mai persona viva.
    }{
        Allo stesso modo il mio animo, che ancora fuggiva, \\
        si girò indietro a guardare il tragitto,\\
        che non abbandonò mai nessun vivo.
    }
\end{snippet}

\begin{snippet}{inferno-canto-i-28}
    \StellarPoetry{28}{
        Poi ch'èi posato un poco il corpo lasso, \\
        ripresi via per la piaggia diserta, \\
        sì che 'l piè fermo sempre era 'l più basso.
    }{
        Dopo che riposai per un po' il corpo stanco, \\
        ripresi il cammino lungo il pendio deserto [del colle],\\
        scalando la salita.
    }

    La piaggia è un leggero pendio che non è ancora l'effettiva salita.
    \\
    Il terzo verso, dove un piede è sempre più basso dell'altro,
    implica che ci sia ancora una zavorra che lo mantenga vicino al peccato (alla selva).
\end{snippet}

\begin{snippet}{inferno-canto-i-31}
    \StellarPoetry{31}{
        Ed ecco, quasi al cominciar de l'erta, \\
        una lonza leggiera e presta molto, \\
        che di pel macolato era coverta;
    }{
        E d'improvviso, quasi all'inizio del pendio, \\
        (arrivò) una lonza agile e molto veloce, \\
        dal pelo coperto di macchie;
    }

    Dante incontra la prima delle tre fiere, la lonza.
    La lonza è leggera (veloce, agile). Essa rappresenta infatti la lussuria.
\end{snippet}

\begin{snippetdefinition}{lussuria-definition}{Lussuria}
    La lussuria è un vizio inteso come l'abbandono alle proprie passioni o anche a divertimenti di natura generica, senza il controllo da parte della nostra ragione e della nostra morale.
\end{snippetdefinition}

\begin{snippet}{inferno-canto-i-34}
    \StellarPoetry{34}{
        e non mi si partia dinanzi al volto, \\
        anzi 'mpediva tanto il mio cammino, \\
        ch'i' fui per ritornar più volte vòlto.
    }{
        che non si scansava da davanti a me, \\
        e bloccava il mio cammino a tal punto \\
        che più volte mi voltai per tornare indietro.
    }
\end{snippet}

\begin{snippet}{inferno-canto-i-37}
    \StellarPoetry{37}{
        Temp' era dal principio del mattino, \\
        e 'l sol montava 'n sù con quelle stelle \\
        ch'eran con lui quando l'amor divino
    }{
        Era il principio del mattino, \\
        e il sole saliva in quella [stessa] costellazione \\
        in cui si trovava, quando Dio
    }

    Il tempo è il mattino, e la stazione è la primavera.
    Secondo la \textit{Genesi} il mondo è stato creato di primavera.
    La primavera rappresenta l'inizio della vita, che rappresenta anche speranza.
\end{snippet}

\begin{snippet}{inferno-canto-i-40}
    \StellarPoetry{40}{
        mosse di prima quelle cose belle; \\
        sì ch'a bene sperar m'era cagione \\
        di quella fiera a la gaetta pelle
    }{
        creò inizialmente i corpi celesti;  \\
        per cui mi dava ragione di non temere \\
        quella belva dalla pelle maculata
    }
\end{snippet}

\begin{snippet}{inferno-canto-i-43}
    \StellarPoetry{43}{
        l'ora del tempo e la dolce stagione; \\
        \textbf{ma} non sì che paura non mi desse \\
        la vista che m'apparve d'un leone.
    }{
        l'ora in cui [essa] comparve e la bella stagione;  \\
        finché non mi spaventò \\
        la presenza improvvisa di un leone
    }

    Dante ritrova speranza pensando che il mattino di primavera sia un momento propizio di inizio.
    \\
    Dante incontro la seconda delle fiere, il leone.
    Il \quotes{ma} avversativo inverte la speranza precedentemente acquisita.
\end{snippet}

\begin{snippet}{inferno-canto-i-46}
    \StellarPoetry{46}{
        Questi parea che contra me venisse \\
        con la test' alta e con rabbiosa fame, \\
        sì che parea che l'aere ne tremesse.
    }{
        Questo sembrava procedere contro di me \\
        superbo e affamato,\\
        al punto che sembrava far tremare l'aria.
    }
\end{snippet}

\begin{snippetdefinition}{superbia-definition}{Superbia}
    Radicata convinzione della propria superiorità (reale o presunta) che si traduce in atteggiamenti di orgoglioso distacco o anche di ostentato disprezzo verso gli altri.
\end{snippetdefinition}

\plain{Il leone rappresenta la superbia.}

\begin{snippet}{inferno-canto-i-49}
    \StellarPoetry{49}{
        Ed una lupa, che di tutte brame \\
        sembiava carca ne la sua magrezza, \\
        e molte genti fé già viver grame,
    }{
        Ed una lupa, che di tutti i desideri \\
        sembrava piena pur essendo magra,\\
        e già fece vivere molti popoli in miseria,
    }
\end{snippet}

\plain{Immediatamente Date incontro anche la terza fiera, la lupa.}

\begin{snippetdefinition}{avarizia-antica-definition}{Avarizia (antica)}
    La brama di possessi materialistici.
\end{snippetdefinition}

\plain{La lupa è molto magra. Essa rappresenta l'avariazia, la fame insaziabile e la brama di possessi materiali.}

\begin{snippet}{inferno-canto-i-52}
    \StellarPoetry{52}{
        questa mi porse tanto di gravezza \\
        con la paura ch'uscia di sua vista, \\
        ch'io perdei la speranza de l'altezza.
    }{
        questa vista mi trasmise tanta angoscia \\
        per la paura che mi diede la sua comparsa, \\
        che persi la speranza di arrivare in cima.
    }

    Questa gravezza (peso) simboleggia il ritorno verso la selva, perdendo la speranza di salire.
\end{snippet}

\begin{snippetnote}{alternanzi-speranza-dante}{Alternarsi della speranza}
    Dante viene attraversato da un continuo alternarsi fra speranza e disperazione.
\end{snippetnote}

\begin{snippet}{inferno-canto-i-55}
    \StellarPoetry{55}{
        E qual è quei che volontieri acquista, \\
        e giugne 'l tempo che perder lo face, \\
        che 'n tutti suoi pensier piange e s'attrista;
    }{
        E come [avviene a] colui che volentieri accumula denaro, \\
        arriva il momento che lo fa perdere,\\
        al punto che nell'animo si rattrista e piange;
    }

    Questa similitudine si riferisce agli avari (oppure potrebbe riferisci ai giocatori d'azzardo).
    Coloro che affidano la propria felicità ai beni materiali, e si disperano quando perdono tuto.
\end{snippet}

\begin{snippet}{inferno-canto-i-58}
    \StellarPoetry{58}{
        tal mi fece la bestia sanza pace, \\
        che, venendomi 'ncontro, a poco a poco \\
        mi ripigneva là dove 'l sol tace.
    }{
        così mi ridusse la belva che non ha pace, \\
        la quale, venedomi incontro, pian piano\\
        mi respingeva nell'ombra.
    }
    La lupa faceva provare a Dante la stessa sensazione dell'ultima terzina,
    riportandolo verso la selva (dove il sole non splende).
    
    È presente una sinestesia (dove 'l sol tace).
\end{snippet}

\begin{snippet}{inferno-canto-i-61}
    \StellarPoetry{61}{
        Mentre ch'i' rovinava in basso loco, \\
        dinanzi a li occhi mi si fu offerto \\
        chi per lungo silenzio parea fioco.
    }{
        Mentre ero ricacciato a forza in basso, \\
        mi si offrì alla vista colui che\\
        per un lungo silenzio era rimasto sfuocato.
    }

    Mentre Dante rotolava verso il basso, incontra \textbf{Virgilio}.
    La sua voce era bassa perché non aveva parlato per molto tempo.
    Questo indica anche che la sua parola non veniva ascoltata da molto,
    esso rappresenta infatti la ragione umana.
\end{snippet}

\begin{snippetcharacter}{virgilio}{Virgilio}
    \textit{Virgilio} fu un poeta
    vissuto tra il 70 a.C e il 19 a.C.
\end{snippetcharacter}

\begin{snippet}{inferno-canto-i-64}
    \StellarPoetry{64}{
        Quando vidi costui nel gran diserto, \\
        «Miserere di me», gridai a lui, \\
        «qual che tu sii, od ombra od omo certo!».
    }{
        Quando lo vidi nella grande spiaggia vuota, \\
        \quotes{Pietà di me}, gli gridai,\\
        \quotes{chiunque tu sia, fantasma o uomo vero!}
    }
    Questa è la prima volta che qualcuno parla.
    
    Dante chiede \quotes{Abbi pietà di me. Chiunque tu sia, anima o uomo}.
\end{snippet}

\begin{snippet}{inferno-canto-i-67}
    \StellarPoetry{67}{
        Rispuosemi: «Non omo, omo già fui, \\
        e li parenti miei furon lombardi, \\
        mantoani per patrïa ambedui.
    }{
        Mi rispose: “Non sono un uomo, uomo lo fui già, \\
        e i miei genitori furono lombardi,\\
        entrambi di Mantova.
    }
    Nel medioevo le persone si presentavano con la loro provenienza geografica.
    Ciò indica il nome della propria famiglia e l'appartenenza politica.
    
    La Lombardia era tutta l'Italia del Nord.
    I genitori erano mantovani.
\end{snippet}

\begin{snippet}{inferno-canto-i-70}
    \StellarPoetry{70}{
        Nacqui sub Iulio, ancor che fosse tardi, \\
        e vissi a Roma sotto 'l buono Augusto \\
        nel tempo de li dèi falsi e bugiardi.
    }{
        Nacqui sotto Giulio Cesare, ma troppo tardi, \\
        e vissi a Roma durante il regno del buon Augusto, \\
        all'epoca degli dei finti e impostori.
    }
    L'anima è nata durante il periodo di Giulio Cesare, ma visse
    a Roma sotto Augusto, a seguito della morte di Cesare nel 44 a.C.
    
    Virgilio ha vissuto in un periodo di Dei pagani (siccome Cristo non era ancora nato).
\end{snippet}

\begin{snippet}{inferno-canto-i-73}
    \StellarPoetry{73}{
        Poeta fui, e cantai di quel giusto \\
        figliuol d'Anchise che venne di Troia, \\
        poi che 'l superbo Ilïón fu combusto.
    }{
        Fui un poeta, e scrissi di quell'uomo giusto \\
        figlio di Anchise che arrivò da Troia, \\
        dopo che la superba Ilio venne bruciata.
    }

    Virgilio celebrò di Enea (Eneide) dopo che la fortezza fu bruciata.
    Qui termina la presentazione.
    
    È importante notare che la salvezza di Dante deriva da un poeta.
    La poesia era cruciale nel mondo medievale.
\end{snippet}

\begin{snippet}{inferno-canto-i-76}
    \StellarPoetry{76}{
        Ma tu perché ritorni a tanta noia? \\
        perché non sali il dilettoso monte \\
        ch'è principio e cagion di tutta gioia?».
    }{
        E tu, perché ridiscendi a tanta pena? \\
        Perché non scali il felice colle\\
        che è principio e causa di tutte le gioie?”
    }

    In italiano antico la noia indica tormento.
    Nonostante ci si aspetterebbe la risposta di Dante, esso
    riconosce Virgilio e lo elogia con le seguenti 3 terzine:
\end{snippet}

\begin{snippet}{inferno-canto-i-79}
    \StellarPoetry{79}{
        «Or se' tu quel Virgilio e quella fonte \\
        che spandi di parlar sì largo fiume?», \\
        rispuos' io lui con \textbf{vergognosa} fronte.
    }{
        “Sei tu dunque quel Virgilio e quella fonte \\
        che spande un fiume così ricco di parole?”\\
        Gli risposi con volto umile.
    }

    Dante si rivolge a Virgilio con vergogna, sentimento di deferenza e rispetto.
\end{snippet}

\begin{snippet}{inferno-canto-i-82}
    \StellarPoetry{82}{
        «O de li altri poeti onore e lume, \\
        vagliami 'l lungo studio e 'l grande amore \\
        che m'ha fatto cercar lo tuo volume.
    }{
        “Oh, gloria e luce per gli altri poeti, \\
        mi serva l'assiduo studio e il forte amore \\
        per il quale ho cercato le tue opere.
    }

    Dante dice di avere studiato la sua opera, e dichiara un debito poetico.
    L'amore poetico di Dante l'ha portato a studiare (a memoria) l'Eneide.
    Infatti, molte espressioni nella Commedia sono riprese dall'Eneide.
\end{snippet}

\begin{snippet}{inferno-canto-i-85}
    \StellarPoetry{85}{
        Tu se' lo mio maestro e 'l mio autore, \\
        tu se' solo colui da cu' io tolsi \\
        lo bello stilo che m'ha fatto onore.
    }{
        Tu sei il mio maestro e il mio autore [di riferimento], \\
        da te solo ho tratto\\
        lo stile eletto per cui sono conosciuto.
    }
\end{snippet}

\plain{<hr><br>}

\begin{snippet}{inferno-canto-i-88}
    \StellarPoetry{88}{
        Vedi la bestia per cu' io mi volsi; \\
        aiutami da lei, famoso saggio, \\
        ch'ella mi fa tremar le vene e i polsi».
    }{
        Guarda la belva per cui mi voltai indietro \\
        salvami da lei, celebrato poeta,\\
        poiché questa mi fa tremare le vene e i polsi.
    }

    La lonza e il leone non vengono nemmeno più nominati, \quotes{la bestia} è quella più difficile.
\end{snippet}

\begin{snippet}{inferno-canto-i-91}
    \StellarPoetry{91}{
        «A te convien tenere altro vïaggio», \\
        rispuose, poi che lagrimar mi vide, \\
        «se vuo' campar d'esto loco selvaggio;
    }{
        “Ti conviene intraprendere un itinerario diverso”, \\
        rispose, dopo che mi vide piangere,\\
        “se vuoi uscire da questo luogo selvaggio;
    }
    
    Il primo verso di questa terzina è quello più importante di tutto il canto.
    Virgilio risponde indicando un altro percorso da compiere
\end{snippet}

\begin{snippet}{inferno-canto-i-94}
    \StellarPoetry{94}{
        ché questa bestia, per la qual tu gride, \\
        non lascia altrui passar per la sua via, \\
        ma tanto lo 'mpedisce che l'uccide;
    }{
        Poiché questa belva, a causa della quale tu gridi, \\
        impedisce a tutti di superarla,\\
        e blocca chiunque fino a ucciderlo;
    }

    La lupa non lascia passare \textit{nessuno}.
    Indugiare qui significa morire. Le persone, più vanno verso il peccato
    rischiano di rimanerci e rimanere peccatori.
\end{snippet}

\begin{snippet}{inferno-canto-i-97}
    \StellarPoetry{97}{
        e ha natura sì malvagia e ria, \\
        che mai non empie la bramosa voglia, \\
        e dopo 'l pasto ha più fame che pria.
    }{
        e ha una natura così perversa e colpevole, \\
        che non riempie mai il suo desiderio ansioso,\\
        e dopo essersi cibata ha più fame di prima.
    }

    Viriglio descrive ulteriormente la lupa.
    La lupa ha ancora più fame dopo aver mangiato, questa è la cupidigia.
\end{snippet}

\begin{snippet}{inferno-canto-i-100}
    \StellarPoetry{100}{
        Molti son li animali a cui s'ammoglia, \\
        e più saranno ancora, infin che 'l veltro \\
        verrà, che la farà morir con doglia.
    }{
        Sono molti gli uomini a cui si lega, \\
        e ce ne saranno ancor di più, finché arriverà il veltro, \\
        che la farà morire con dolore.
    }

    Molte sono le vittime di questo peccato, ma Virgilio profetizza che il veltro sia l'unico
    a poterla superare.
    Possiamo capire che questa sia una profezia dal tempo futuro e linguaggio enigmatico.
\end{snippet}

\begin{snippet}{inferno-canto-i-103}
    \StellarPoetry{103}{
        Questi non ciberà terra né peltro, \\
        ma sapïenza, amore e virtute, \\
        e sua nazion sarà tra feltro e feltro.
    }{
        Costui non tratterà né terra, né denaro, \\
        ma sapienza, carità e virtù,\\
        e nascerà tra feltro e feltro.
    }
    %  TODO Dieresi - spezza le due vocali
    Il veltro non si ciberà nè di terra nè di peltro (lega metallica delle monete).
    Non avrà quindi fame di ricchezza materiale.
    Invece, si ciberà di sapienza, amore e virtù (i 3 attributi della trinità).

    La prima interpretazione è quella di intendere il feltro come un panno vile,
    mentre la seconda interpretazion, collegata alla prima, è che la lupa potrebbe
    essere stato sconfitta da un uomo della chiesa (un Francescano) che si occupa dei malati
    ed è molto umile. La terza interpretazione vede il feltro come il cielo, ovvero una figura provvidenziale
    mandata da Dio (facendo il veltro una figura divina). La quarta interpretazione
    lo collega alla collocazione geografica di Verona, la quale si situa tra Feltre (Italia) e Montefeltro, per cui
    nascerà a Verona.
    L'ultima interpretazione, quella più accreditata, dice che il veltro nascerà da un'elezione imperiale,
    probabilmente di Arrigo VIII, presumendo che il veltro sia un imperatore, siccome l'urna
    veniva foderata di feltro all'interno.
\end{snippet}

\begin{snippetdefinition}{veltro-definition}{Veltro}
    Il \textit{veltro} è un cane da caccia.
    Nella letteratura italiana, rappresenta un'azione di riforma,
    evidentemente promossa da Dio, che perseguiti la cupidigia in tutte le sue forme
    ristabilendo in tutto il mondo ordine e giustizia. 
\end{snippetdefinition}

\begin{snippet}{inferno-canto-i-106}
    \StellarPoetry{106}{
        Di quella umile Italia fia salute \\
        per cui morì la vergine Cammilla, \\
        Eurialo e Turno e Niso di ferute.
    }{
        Sarà salvezza di quella umile Italia \\
        per la quale morì la fanciulla Camilla,\\
        Eurialo, Turno e Niso per le ferite.
    }
    \textbf{\color{red}nota:} fia = sarà. \\
    Virgilio sta dicendo che il veltro sarà la salvezza dell'Italia.
\end{snippet}

\begin{snippet}{inferno-canto-i-109}
    \StellarPoetry{109}{
        Questi la caccerà per ogne villa, \\
        fin che l'avrà rimessa ne lo 'nferno, \\
        là onde 'nvidia prima dipartilla.
    }{
        Costui la caccerà di città in città\\
        finché non l'avrà ricacciata nell'Inferno,\\
        il luogo da cui in principio l'invidia la fece uscire.
    }
    Questa terzina termina la profezia.
    Il veltro sconfiggerà la lupa cacciandola ovunque fino all'inferno.
    L'ultimo verso sembrerebbe indicare il primo momento in cui la lupa è stata scagliata
    fra gli uomini, a seguito dell'invidia del demonio nei confronti di Dio.
\end{snippet}

\begin{snippet}{inferno-canto-i-111}
    \StellarPoetry{112}{
        Ond' io per lo tuo me' penso e discerno \\
        che tu mi segui, e io sarò tua guida, \\
        e trarrotti di qui per loco etterno;
    }{
        Per cui, riguardo te, penso e comprendo sia meglio \\
        che tu mi segua e io ti sia guida,\\
        portandoti da qui nell'oltretomba;
    }
    Virgilio dice che per il meglio di Dante, è auspicabile che lui lo segua.
    Questa è infatti la saggezza di Virgilio.
    Dante verrà salvato e portato via attraverso un luogo eterno (l'inferno).
\end{snippet}

\begin{snippet}{inferno-canto-i-115}
    \StellarPoetry{115}{
        ove udirai le disperate strida, \\
        vedrai li antichi spiriti \textbf{dolenti}, \\
        ch'a la seconda morte ciascun grida;
    }{
        in cui ascolterai le urla senza speranza, \\
        osserverai i vecchi spiriti sofferenti,\\
        che gridano per la morte dell'anima;
    }
    Queste terzine descrivono quindi l'inferno, dove Dante passerà.
    Esso viene rappresentato come pieno di anime dannate.
    
    La prima morte è quella fisica, mentre la seconda morte è quella dell'anima che
    diventa dannata.
\end{snippet}

\begin{snippet}{inferno-canto-i-118}
    \StellarPoetry{118}{
        e vederai color che son \textbf{contenti} \\
        nel foco, perché speran di venire \\
        quando che sia a le beate genti.
    }{
        e vedrai coloro che appaiono contenti \\
        nel dolore, perché hanno la speranza\\
        di accedere nel tempo al regno dei beati.
    }
    Le anime dannate sono distrutte nel dolore perché sanno che quella è la loro
    fine eterna, mentre le anime nel purgatorio sono contente di scontare la propria pena,
    perché sanno che essa avrà una fine (fino all'Apocalisse).

    Questo può essere sintetizzato dalla rima \quotes{dolenti}:\quotes{contenti}.
\end{snippet}    

\begin{snippet}{inferno-canto-i-121}
    \StellarPoetry{121}{
        A le quai poi \textbf{se} tu vorrai salire, \\
        anima fia a ciò più di me degna: \\
        con lei ti lascerò nel mio partire;
    }{
        A cui poi se tu vorrai accedere, \\
        ti porterà un'anima più degna di me:\\
        quando me ne andrò ti lascerò a lei;
    }
    Se Dante vorrà salire fra le genti beate (paradiso), non potrà farlo con Virgilio
    ma con un'altra persona (Beatrice).
    Dante avrà la scelta di percorrere anche i cieli. Non è infatti necessario
    farlo per essere salvi. Dante sarà salvo dopo aver percorso il purgatorio.
\end{snippet}

\begin{snippet}{inferno-canto-i-124}
    \StellarPoetry{124}{
        ché quello imperador che là sù regna, \\
        perch' i' fu' ribellante a la sua legge, \\
        non vuol che 'n sua città per me si vegna.
    }{
        Poiché quell'Imperatore che regna lassù, \\
        per via del fatto che fui ribelle alla sua dottrina, \\
        mi vieta di entrare nel suo Regno.
    }
    Il motivo per cui Dante dovrà essere seguire da un'altra anima
    è perché Virgilio è un dannato (non è battezzato). 
    Virgilio non può nominare direttamente Dio essendo tale.
    % virtù cardinali e teologate
\end{snippet}

\begin{snippet}{inferno-canto-i-127}
    \StellarPoetry{127}{
        In tutte parti impera e quivi regge; \\
        quivi è la sua città e l'alto seggio: \\
        oh felice colui cu' ivi elegge!».
    }{
        [Dio] regna in ogni luogo e qui dimora; \\
        questa è la sua città e il suo trono:\\
        felice è colui che viene chiamato nel suo Regno!”.
    }
    Beato colui che può accedere in paradiso.
\end{snippet}

\begin{snippet}{inferno-canto-i-130}
    \StellarPoetry{130}{
        E io a lui: «Poeta, io ti richeggio \\
        per quello Dio che tu non conoscesti, \\
        a ciò ch'io fugga questo male e peggio,
    }{
        Ed io gli dissi: “Poeta, io ti chiedo per quel Dio \\
        che tu non avesti modo di conoscere, affinché\\
        io mi allontani da questo male e dalla dannazione,
    }
    A differenza di Virgilio, Dante può pronunciare il nome di Dio.
    Nel nome del Dio che purtroppo Virgilio non potrà conoscere, Dante accetta.
\end{snippet}

\begin{snippet}{inferno-canto-i-133}
    \StellarPoetry{133}{
        che tu mi meni là dov' or dicesti, \\
        sì ch'io veggia la porta di san Pietro \\
        e color cui tu fai cotanto mesti».
    }{
        che tu mi conduca là dove dicesti, \\
        affinché io veda le porte del Paradiso\\
        e coloro i quali tu descrivi tanto tristi”.
    }
    Dante ripassa i posti che traverserà con Virgilio, ma in ordine opposto.
    La porta di san Pietro rappresenta il purgatorio, mentre le anime dannate
    sono nell'inferno.
\end{snippet}

\begin{snippet}{inferno-canto-i-136}
    \StellarPoetry{136}{
        Allor si mosse, e io li tenni dietro.
    }{
        [Virgilio] allora si mise in cammino, ed io lo seguii.
    }
    Come il canto comincia con un cammino (metaforico), viene terminato
    con un cammino (fisico).
\end{snippet}

\end{document}