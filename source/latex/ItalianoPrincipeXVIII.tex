\documentclass[preview]{standalone}

\usepackage{amsmath}
\usepackage{amssymb}
\usepackage{stellar}
\usepackage{definitions}

\begin{document}

\id{italiano-principe-capitolo-xviii}
\genpage

\section{Capitolo XVIII}

\begin{snippetnote}{c88770b0-a892-4ca8-a109-2455771ecd20}{}
    ciononostante \(\iff\) nondimanco
\end{snippetnote}

\begin{snippet}{il-principe-capitolo-xviii}
    % verità effettuale
    Il tema di questo capitolo è la lealtà, slealtà e mantenere
    la propria parola.
    Viene continuata la trama principale:
    nonostante un principe con tutte le virtù positive
    sarebbe utopico, nella realtà \textit{effettuale},
    i grandi principi sono stati quelli che hanno saputo essere sleali.
    \\\\
    Un principe può combattere o con le leggi (morali)
    o con la forza. Quel primo, è proprio dell'uomo, quel secondo
    è delle bestie.
    Pertanto ad un Principe è necessario saper ben usare la bestia e l'uomo.
    Questo principio era già stato insegnato implicitamente dagli scrittori antichi:
    viene citato come esempio l'eroe greco Achille, che fu allevato
    da un centauro (metà uomo, metà bestia).
    Chiaramente, questa argomentazione è piuttosto debole,
    soprattutto perché si tratta di un mito.
    \\\\
    La forza bestiale che un principe deve possedere è, a sua volta,
    suddivisa in astuzia (della volpe) e forza (del leone).
    Un principe non deve mantenere la parola data, se ciò
    gli si ritorce contro o se non valgono più i motivi per cui ha promesso.
    Se tutti gli uomini fossero buoni, idealmente, questo non varrebbe,
    ma dato che sono tristi (malvagi), e loro non manterrebbero
    a loro volta le loro promesse, allora è necessario.
    \\\\
    È importante notare che il principe non deve farsi notare
    quando inganna. Gli uomini sono semplici ed ingenui, e quindi
    chi inganna troverà sempre chi può essere manipolato.
    L'immagine che il principe deve dare
    è quella di essere completamente sulla parte virtuosa.
    L'uomo tende a giudicare sulla base delle apparenze.
    \\\\
    Il principe è giudicato sulla base dei fini (dei risultati).
    Tutto il mondo è popolo (e quindi manipolabile).
\end{snippet}

\end{document}