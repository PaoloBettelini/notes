\documentclass[preview]{standalone}

\usepackage{amsmath}
\usepackage{amssymb}
\usepackage{parskip}
\usepackage{fullpage}
\usepackage{hyperref}
\usepackage{stellar}
\usepackage{tikz}

\hypersetup{
    colorlinks=true,
    linkcolor=black,
    urlcolor=blue,
    pdftitle={NetwonsMethod},
    pdfpagemode=FullScreen,
}

\begin{document}

\id{newton-method}
\genpage

\section{Method}

\begin{snippetdefinition}{newton-method-definition}{Newton's method}
    \textit{Newton's method} is a numeric method to approximate the solution to equations
    in the form \(f(x)=0\). Starting from an initial approximation or guess \(x_0\),
    the next approximation is give by
    \[
        x_{n+1} = x_n - \frac{f(x_0)}{f'(x_n)}
    \]
\end{snippetdefinition}

\begin{snippetproof}{newton-method-definition-proof}{Newton's method}
    Starting from an initial approximation or guess \(x_0\),
    the tangent line at \(x=x=0\) is given by
    \[
        y=f(x_0) + f'(x_0)(x-x_0)
    \]

    By projecting the root of the tangent onto the graph, we get another tangen whose root is closer
    to the root we are looking for. By doing this process recursively we may approach the solution.

    The next approximation \(x_1\) occurs when the tangent of \(x_0\) is 0, so
    \begin{align*}
        0 &= f(x_0) + f'(x_0)(x_1 - x_0) \\
        x_1 - x_0 &= - \frac{f(x_0)}{f'(x_0)} \\
        x_1 &= x_0 - \frac{f(x_0)}{f'(x_0)}
    \end{align*}
    as long as \(f'(x_0) \neq 0\).

    If \(x_n\) is an approximation of the solution to \(f(x)=0\) and \(f'(x_n) \neq 0\) then a closer approximation is given by
    \[
        x_{n+1} = x_n - \frac{f(x_0)}{f'(x_n)}
    \]
\end{snippetproof}

\begin{snippet}{newton-method-illustration}
    \begin{center}
        \begin{tikzpicture}[
        scale=2,
        declare function={
            f(\x) = 0.25 * (\x - 0.5) * (\x - 0.5) - 0.5;
            % fPrime(\x) = (f(\x + 0.00025) - f(\x)) / 0.00025;
            x0 = 4;
            x1 = 2.5357;
            x2 = 2.0009;
            root = 1.914;
            }
        ]
            \draw[domain=0:4.25, smooth, variable=\x, blue, very thick] plot ({\x}, {f(\x)});
            
            \draw[domain=2:4.25, smooth, variable=\x, red]
            plot ({\x}, {f(x0) + 1.75 * (\x - x0)});
            
            \draw[domain=1.25:4.25, smooth, variable=\x, red]
            plot ({\x}, {f(x1) + 1.018 * (\x - x1)});
            
            \draw[->] (0, 0) -- (5, 0) node[above] {\(x\)};
            \draw[->] (0, -0.75) -- (0, 3) node[right] {\(y\)};
            
            \draw[-, dashed] (x0, 0) node[below] {\(x_0\)} -- (x0, {f(x0)});
            \filldraw [black] (x0, 0) circle (0.75pt);
            \filldraw [black] (x0, {f(x0)}) circle (0.75pt);

        \draw[-, dashed] (x1, 0) node[below] {\(x_1\)} -- (x1, {f(x1)});
        \filldraw [black] (x1, 0) circle (0.75pt);
        \filldraw [black] (x1, {f(x1)}) circle (0.75pt);

        \draw[-, dashed] (x2, 0) node[below] {\(x_2\)} -- (x2, {f(x2)});
        \filldraw [black] (x2, 0) circle (0.75pt);
        \filldraw [black] (x2, {f(x2)}) circle (0.75pt);

        \filldraw [red] (root, 0) circle (0.75pt);
        \end{tikzpicture}
    \end{center}
\end{snippet}

\end{document}
