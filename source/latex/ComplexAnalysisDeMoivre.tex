\documentclass[preview]{standalone}

\usepackage{amsmath}
\usepackage{amssymb}
\usepackage{parskip}
\usepackage{fullpage}
\usepackage{hyperref}
\usepackage{wrapfig}
\usepackage{bettelini}
\usepackage{makecell}
\usepackage{stellar}
\usepackage{definitions}

\usetikzlibrary{3d,arrows.meta,decorations.markings,perspective}
\tikzset{->-/.style={decoration={% https://tex.stackexchange.com/a/39282/194703
  markings,
  mark=at position #1 with {\arrow{>}}},postaction={decorate}},
  ->-/.default=0.55}

\begin{document}
  
\id{complexanalysis-demoivre}
\genpage

\section{De Moivre's Theorem}

\begin{snippettheorem}{de-moivres-theorem}{De Moivre's Theorem}
    For any \(\theta \in \realnumbers\) and \(n \in \naturalnumbers\)
    \[
        \cos(n\theta) + i\sin(n\theta) = \left(\cos(\theta) + i\sin(\theta)\right)^n 
    \]
\end{snippettheorem}

\begin{snippetproof}{de-moivres-theorem-proof}{De Moivre's Theorem}
    Using the property of exponentiation \({\left(a^b\right)}^c = a^{bc}\),
    we can see that \({\left(e^{i\theta}\right)}^n = e^{in\theta}\).
    \\
    Using Euler's formula we can deduce that
    
    \[
        {\left(\cos(\theta) + i\sin(\theta)\right)}^n = \cos(n\theta) + i\sin(n\theta),
        \quad n \in \naturalnumbers
    \]
\end{snippetproof}

\section{Roots of Unity}

\begin{snippetdefinition}{nth-root-of-unity-definition}{Nth root of unity}
    An \textit{nth root of unity} is a solution to
    \[
        z^n=1
    \]
    where \(z \in \complexnumbers\) and \(n \in \naturalnumbers\).
\end{snippetdefinition}

\begin{snippettheorem}{nth-root-of-unity-solution}{Nth roots of unity}
    The equation \(z^n = 1\) has \(n\) distinct solutions
    given by
    \[
        z = \cos\left(\frac{2k\pi}{n}\right) + i\sin\left(\frac{2k\pi}{n}\right)
    \]
\end{snippettheorem}

\begin{snippetproof}{nth-root-of-unity-solution-proof}{Nth roots of unity}
    We can extend De Moivre's Theorem for the integers powers or any complex number,
    rather than the ones on the unit circle \((r=1)\).

    \[
        \left(r\left(\cos(\theta) + i\sin(\theta)\right)\right)^n = 
        r^n\left(\cos(n\theta) + i\sin(n\theta)\right), \quad n \in \naturalnumbers
    \]

    The nth roots of \(1\) are the solutions to

    \[
        z^n=1
    \]

    for a given \(n\). We might write \(1\) as a complex number

    \[
        z^n = \cos(0) + i\sin(0)
    \]

    Comparing this to our extended De Moivre's theorem

    \[
        \cos(0) + i\sin(0) = r^n\left(\cos(n\theta) + i\sin(n\theta)\right)
    \]

    We can see that

    \begin{align*}
        r^n&=1 \\
        n\theta&=0
    \end{align*}

    As long as \(n \neq 0\)

    \begin{align*}
        r&=1 \\
        \theta&=0
    \end{align*}

    By plugging these values into

    \[
        z^n = \left(r\left(\cos(\theta) + i\sin(\theta)\right)\right)^n
    \]

    we get that \(z=1\).

    However we could also write \(1\) as

    \[
        \cos(2k\pi) + i\sin(2k\pi), \quad k\in \naturalnumbers
    \]

    We would then get that

    \begin{align*}
        r^n&=1 \\
        n\theta&=2k\pi
    \end{align*}

    When solving for \(z\) again we get

    \begin{align*}
        z^n &= \left(r\left(\cos(\theta) + i\sin(\theta)\right)\right)^n
        \\
        &= \left(\cos\left(\frac{2k\pi}{n}\right) + i\sin\left(\frac{2k\pi}{n}\right)\right)^n
    \end{align*}

    concluding that

    \[
        z = \cos\left(\frac{2k\pi}{n}\right) + i\sin\left(\frac{2k\pi}{n}\right)
    \]

    This gives us a solution for each \(k\), however the solutions are redundant for \(k \geq n\).
    In fact, the roots of unity of \(n\) are \(n\) distinct solutions (points on the unit circle).
\end{snippetproof}

\begin{snippet}{nth-roots-of-unity-illustration-expl}
The roots of unity have the same angle \(\alpha = \frac{2\pi}{n}\) between each other.
\\
The first root of unit counter-clockwise is denoted \(\zeta_n\) because each subsequent
root is a power of \(\zeta_n\).
\end{snippet}

\begin{snippet}{nth-roots-of-unity-illustration}
\def\n{7}
\begin{center}
\begin{tikzpicture}[
        dot/.style={draw,fill,circle,inner sep=1pt}
    ]
    \draw[->] (-2,0) -- (2,0) node[below] {\(\Re\)};
    \draw[->] (0,-2) -- (0,2) node[left] {\(\Im\)};
    \draw[help lines] (0,0) circle (1);
    
    \foreach \i in {1,...,\n} {
        \node[dot,label={\i*360/\n-(\i==\n)*45:\(\zeta_\n^{\i}\)}] (w\i) at (\i*360/\n:1) {};
        \draw[->] (0,0) -- (w\i);
    }
    \draw[->] (0:.3) arc (0:360/\n:.3);
    \node at (360/\n/2:.5) {\(\alpha\)};
\end{tikzpicture}
\end{center}
\end{snippet}


\end{document}