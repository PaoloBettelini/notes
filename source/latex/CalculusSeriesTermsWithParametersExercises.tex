\documentclass[preview]{standalone}

\usepackage{amsmath}
\usepackage{amssymb}
\usepackage{stellar}
\usepackage{definitions}

\begin{document}

\id{series-terms-with-parameters-exercises}
\genpage

\section{Series with parameters}


\begin{snippetexercise}{series-parameters-ex-1}{}
    Study the following \series
    \[
        \sum_{n=1}^\infty \frac{x^n}{n\cdot 2^n}
    \]
\end{snippetexercise}

\begin{snippetsolution}{series-parameters-ex-1-sol}{}
    Analizziamo la convergenza assoluta
    \begin{align*}
        \sum_{n=1}^\infty \left|\frac{x^n}{n\cdot 2^n}\right|
        &= \sum_{n=1}^\infty \frac{{|x|}^n}{n\cdot 2^n}
    \end{align*}
    Calcoliamo il limite
    \begin{align*}
        \lim_n |a_n| &= \lim_n {\left|\frac{x}{2}\right|}^n \cdot \frac{1}{n}
        \\
        &= \lim_n \frac{q^n}{n} = \begin{cases}
            0 & q \leq 1 \\
            +\infty & q > 1
        \end{cases}, \quad q = \left|\frac{x}{2}\right|
    \end{align*}
    Possiamo quindi notare che se \(q>1\), la serie non converge assolutamente.
    In particolare, il limite del termine non è pari a zero, e quindi la serie non converge.
    Applichiamo il criterio della radice n-esima,
    \begin{align*}
        \lim_n \sqrt[n]{|a_n|} &= \lim_n {\left(
            \frac{{|x|}^n}{n\cdot 2^n}
        \right)}^{1/n} \\
        &= \frac{|x|}{2} = q
    \end{align*}
    Allora, se \(q<1\), oppure \(|x| < 2\), la serie converge assolutamente,
    mentre nel caso \(q>1\), oppure \(|x| > 2\), la serie non converge.
    Infine, nel caso singolo \(q=1\), oppure \(|x| = 2\),
    il caso è inane. In questo caso la serie diventa
    \[
        \sum_{n=1}^\infty \frac{1}{n} {\left(\frac{|x|}{2}\right)}^n
        = \sum_{n=1}^\infty \frac{1}{n}
    \]
    che non converge assolutamente.
    La serie originale è invece
    \[
        \sum_{n=1} \frac{1}{n}
    \]
    che non converge, e
    \[
        \sum_{n=1} {(-1)}^n \frac{1}{n}
    \]
    con \(x=-2\), caso in cui converge.
    In conclusione la serie
    \[
        \begin{cases}
            \text{converge assolutamente} & -2 < x < 2 \\
            \text{non converge} & x < -2 \lor x \geq 2 \\
            \text{converge semplicemente} & x = -2
        \end{cases}
    \]
\end{snippetsolution}

\begin{snippetexercise}{series-parameters-ex-2}{}
    Study the following \series
    \[
        \sum_{n=1}^\infty {(-1)}^n
        \frac{e^{n \cdot \frac{x+1}{x-1}}}{n+\sqrt{n}}
    \]
\end{snippetexercise}

\begin{snippetsolution}{series-parameters-ex-2-sol}{}
    \todo
\end{snippetsolution}

\end{document}