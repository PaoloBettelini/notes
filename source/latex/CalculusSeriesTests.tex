\documentclass[preview]{standalone}

\usepackage{amsmath}
\usepackage{amssymb}
\usepackage{stellar}
\usepackage{definitions}

\begin{document}

\id{series-tests}
\genpage

\section{Comparison Test}

\begin{snippettheorem}{series-comparison-test}{Comparison test}
    Let \(\sum a_n\) and \(\sum b_n\) be two series with \(a_n, b_n \geq 0\)
    and \(\forall n, a_n < b_n\). Then,
    \begin{enumerate}
        \item If \(\sum b_n\) is convergent, so is \(\sum a_n\).
        \item if \(\sum a_n\) is divergent, so is \(\sum b_n\).
    \end{enumerate}
\end{snippettheorem}

\section{Limit comparison test}

\begin{snippettheorem}{limit-comparison-test}{Limit comparison test}
    Let \(\sum a_n\) and \(\sum b_n\) be two series with \(a_n \geq 0\)
    and \(b_n > 0\). Let
    \[ c = \lim_{n \to \infty} = \frac{a_n}{b_n} \]
    If \(0 > c > \infty\), then either both series converge or both series diverge.
\end{snippettheorem}

\section{Alternating series test}

\begin{snippettheorem}{alternating-series-test}{Alternating series test}
    Let \(\sum a_n\) be a series
    and either \(a_n = {(-1)}^n b_n\) or \(a_n = {(-1)}^{n+1} b_n\)
    where \(b_n \geq 0\).
    If the following conditions are met:
    \begin{enumerate}
        \item \(\lim_{n \to \infty} b_n = 0\);
        \item \(\{b_n\}\) is a decreasing sequence,
    \end{enumerate}
    then \(\sum a_n\) is convergent.
\end{snippettheorem}

\section{Ratio test}

\begin{snippettheorem}{ratio-test}{Ratio test}
    Let \(\sum a_n\) be a series and set
    \[
        L = \lim_{n \to \infty} \left| \frac{a_{n+1}}{a_n} \right|
    \]
    Then,
    \begin{enumerate}
        \item if \(L < 1\) the series is absolutely convergent;
        \item if \(L > 1\) the series is divergent;
        \item if \(L = 1\) the series may be divergent, conditionally convergent, or absolutely convergent.
    \end{enumerate}
\end{snippettheorem}

\section{Root test}

\begin{snippettheorem}{root-test}{Root test}
    Let \(\sum a_n\) be a series and set
    \[
        L = \lim_{n \to \infty} {|a_n|}^{\frac{1}{n}}
    \]
    Then,
    \begin{enumerate}
        \item if \(L < 1\) the series is absolutely convergent;
        \item if \(L > 1\) the series is divergent;
        \item if \(L = 1\) the series may be divergent, conditionally convergent, or absolutely convergent.
    \end{enumerate}
\end{snippettheorem}

\section{Integral Test}

\begin{snippetdefinition}{integral-test}{Integral Test}
    Let \(f(x)\) be a continuous function on \([k;\infty)\)
    such that it is decreasing and positive on the interval \([N; \infty)\)
    for some \(N\).
    \[
        \integral[k][\infty][f(x)][x] \text{ converges } \implies
        \sum_{n=k}^{\infty} f(n) \text{ converges}
    \]
    and
    \[
        \integral[k][\infty][f(x)][x] \text{ diverges } \implies
        \sum_{n=k}^{\infty} f(n) \text{ diverges}
    \]
\end{snippetdefinition}

\begin{snippetproof}{integral-test-proof}{integral-test}{Integral Test}{
    \todo
}
\end{snippetproof}

\end{document}