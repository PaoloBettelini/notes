\documentclass[preview]{standalone}

\usepackage{amsmath}
\usepackage{amssymb}
\usepackage{stellar}
\usepackage{definitions}
\usepackage{bettelini}

\begin{document}

\id{series-tests}
\genpage

\section{Comparison Test}

\begin{snippettheorem}{series-comparison-test}{Comparison test}
    Let \[\sum_{n=k}^\infty a_n\] and \[\sum_{n=k}^\infty b_n\]
    be \series where \(a_n, b_n \geq 0\) \eventually. Then,
    \begin{enumerate}
        \item \textbf{comparison:} if \(a_n \leq b_n\) \eventually we have
        \[
            \sum_{n=k}^\infty a_n \leq \sum_{n=k}^\infty b_n
        \]
        In particular,
        \[
            \sum_{n=1}^\infty b_n < +\infty \implies
            \sum_{n=1}^\infty a_n < +\infty
        \]
        and
        \[
            \sum_{n=1}^\infty a_n = \infty \implies
            \sum_{n=1}^\infty a_n = +\infty
        \]
        \item \textbf{asymptotic comparison:}
        if \(a_n \sim b_n\), we have
        \[
            \sum_{n=k}^\infty a_n < +\infty \iff
            \sum_{n=k}^\infty b_n < +\infty
        \]
    \end{enumerate}
\end{snippettheorem}

\begin{snippetproof}{series-comparison-test-proof}{series-comparison-test}{Comparison test}
    \begin{enumerate}
        \item Let \[
                A_N = \sum_{n=k}^N a_n \quad B_N = \sum_{n=k}^N b_n
            \]
            Since \(a_n \leq b_n\) eventually, we have \(A_N \leq B_N\) with \(N\to\infty\).
            By monotonie of the limit
            \[ \underset{n}{\lim}\, A_n = \sum_{n=k}^N a_n \leq \underset{n}{\lim}\, B_n = \sum_{n=k}^N b_n \]
        \item Assume that \(a_n \asymptotic b_n\) so that \(\frac{a_n}{b_n} \to 1\).
            By definition of limit with \(\varepsilon = \frac{1}{2}\), it follows
            \begin{align*}
                \exists n_0 \suchthat \forall n \geq b_0,
                -\frac{1}{2} < \frac{a_n}{b_n} - 1 < \frac{1}{2}
                &\iff \frac{1}{2} < \frac{a_n}{b_n} < \frac{3}{2} \\
                &\iff \frac{1}{2}b_n < a_n < \frac{3}{2}b_n
            \end{align*}
            By using the first point we have
            \[
                \frac{1}{2} \sum_{n=n_0}^\infty b_n < \sum_{n=n_0}^\infty a_n < \frac{3}{2} \sum_{n=n_0}^\infty b_n
            \]
            and the thesis follows.
    \end{enumerate}
\end{snippetproof}

\begin{snippetcorollary}{series-behavior-bigtheta}{}
    Let \[\sum_{n=k}^\infty a_n\] and \[\sum_{n=k}^\infty b_n\]
    be \series where \(a_n = \bigtheta(b_n)\). Then, both \series follow the same behavior.
\end{snippetcorollary}

\section{Limit comparison test}

\begin{snippettheorem}{limit-comparison-test}{Limit comparison test}
    Let \(\sum a_n\) and \(\sum b_n\) be two series with \(a_n \geq 0\)
    and \(b_n > 0\). Let
    \[ c = \lim_{n \to \infty} = \frac{a_n}{b_n} \]
    If \(0 > c > \infty\), then either both series converge or both series diverge.
\end{snippettheorem}

\section{Cauchy condensation test}

\begin{snippettheorem}{cauchy-condensation-test-theorem}{Cauchy condensation test}
    Let
    \[
        \sum_{n=1}^\infty a_n
    \]
    be a \series where \(a_n \geq 0\) and \(\forall n, a_n \geq a_{n+1}\).
    Then, the \series follows the same behavior as
    \[
        \sum_{k=0}^\infty 2^k a_{2^k}
    \]
\end{snippettheorem}

\begin{snippetproof}{cauchy-condensation-test-theorem-proof}{cauchy-condensation-test-theorem}{Cauchy condensation test}
    \todo
\end{snippetproof}

\section{Alternating series test}

\begin{snippettheorem}{alternating-series-test}{Alternating series test}
    Let \(\sum a_n\) be a series
    and either \(a_n = {(-1)}^n b_n\) or \(a_n = {(-1)}^{n+1} b_n\)
    where \(b_n \geq 0\).
    If the following conditions are met:
    \begin{enumerate}
        \item \(\lim_{n \to \infty} b_n = 0\);
        \item \(\{b_n\}\) is a decreasing sequence,
    \end{enumerate}
    then \(\sum a_n\) is convergent.
\end{snippettheorem}

\section{Ratio test}

\begin{snippettheorem}{ratio-test}{Ratio test}
    Let \(\sum a_n\) be a series and set
    \[
        L = \lim_{n \to \infty} \left| \frac{a_{n+1}}{a_n} \right|
    \]
    Then,
    \begin{enumerate}
        \item if \(L < 1\) the series is absolutely convergent;
        \item if \(L > 1\) the series is divergent;
        \item if \(L = 1\) the series may be divergent, conditionally convergent, or absolutely convergent.
    \end{enumerate}
\end{snippettheorem}

\section{Root test}

\begin{snippettheorem}{root-test}{Root test}
    Let \(\sum a_n\) be a series and set
    \[
        L = \lim_{n \to \infty} {|a_n|}^{\frac{1}{n}}
    \]
    Then,
    \begin{enumerate}
        \item if \(L < 1\) the series is absolutely convergent;
        \item if \(L > 1\) the series is divergent;
        \item if \(L = 1\) the series may be divergent, conditionally convergent, or absolutely convergent.
    \end{enumerate}
\end{snippettheorem}

\section{Integral Test}

\begin{snippettheorem}{integral-test}{Integral Test}
    Let \(f(x)\) be a continuous \function on \([k;\infty)\)
    such that it is decreasing and positive on the interval \([N; \infty)\)
    for some \(N\).
    \[
        \integral[k][\infty][f(x)][x] \text{ converges } \implies
        \sum_{n=k}^{\infty} f(n) \text{ converges}
    \]
    and
    \[
        \integral[k][\infty][f(x)][x] \text{ diverges } \implies
        \sum_{n=k}^{\infty} f(n) \text{ diverges}
    \]
\end{snippettheorem}

\begin{snippetproof}{integral-test-proof}{integral-test}{Integral Test}{
    \todo
}
\end{snippetproof}

\end{document}