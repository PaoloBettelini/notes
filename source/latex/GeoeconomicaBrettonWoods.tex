\documentclass[preview]{standalone}

\usepackage{amsmath}
\usepackage{amssymb}
\usepackage{stellar}
\usepackage{bettelini}

\hypersetup{
    colorlinks=true,
    linkcolor=black,
    urlcolor=blue,
    pdftitle={Stellar},
    pdfpagemode=FullScreen,
}

\begin{document}

\title{Geografia economica}
\id{geoeconomica-bretton-woods}
\genpage

\section{Accordi Bretton Woods (1944)}

\begin{snippet}{accordi-bretton-woods-expl}
    Gli \textbf{Accordi Bretton Woods} sono un accordo avente valenza economica
    che porta gli Stati Uniti a ricoprire un ruolo egemone. Con questi accordi, il
    dollaro assume valore. Vengono istanziati il \textbf{Fondo Monetario Internazionale},
    la \textbf{Banca Mondiale} ed il \textbf{General Agreement on Tariffs and Trade}.
\end{snippet}

\begin{snippetdefinition}{fondo-monetario-Internazionale-definizione}{Fondo Monetario Internazionale}
    Il \textit{Fondo Monetario Internazionale} (FMI) è un'organizzazione internazionale che mira a promuovere
    la cooperazione monetaria, la stabilità finanziaria,
    la crescita economica sostenibile, il libero scambio e la riduzione della povertà nel mondo.
    Gestisce crisi finanziarie fornendo prestiti ai paesi membri con difficoltà economiche e
    fornisce consulenza politica ed economica per favorire la stabilità economica globale.
\end{snippetdefinition}

\begin{snippetdefinition}{banca-Mondiale-definizione}{Banca Mondiale}
    La \textit{Banca Mondiale} è un'istituzione internazionale che mira a ridurre la povertà
    estrema e a promuovere lo sviluppo economico sostenibile nei paesi in via di sviluppo.
\end{snippetdefinition}

\begin{snippetdefinition}{gatt-definizione}{GATT}
    Il \textit{General Agreement on Tariffs and Trade} (GATT)
    è stato un accordo il quale obiettivo principale era quello di
    di promuovere il commercio internazionale riducendo le tariffe doganali,
    le discriminazioni commerciali e le restrizioni al commercio tra i paesi firmatari. 
\end{snippetdefinition}

\begin{snippet}{6c1f029e-1724-4090-89ee-3b3e2b87d5b7}
    Inoltre, gli Accordi di Bretton Woods sono anche connessi, ma non direttamente,
    alla \textbf{World Trade Organization (WTO, 1995)}.
\end{snippet}

\end{document}