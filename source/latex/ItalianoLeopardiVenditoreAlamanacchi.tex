\documentclass[preview]{standalone}

\usepackage{amsmath}
\usepackage{amssymb}
\usepackage{stellar}

\hypersetup{
    colorlinks=true,
    linkcolor=black,
    urlcolor=blue,
    pdftitle={Stellar},
    pdfpagemode=FullScreen,
}

\begin{document}

\title{Stellar}
\id{italiano-leopardi-venditore-almanacchi}
\genpage

\section{Dialogo di un venditore <br>d'almanacchi e di un passeggere}

\begin{snippet}{leopardi-venditore-almanacchi}
    Il venditore degli almanacchi, vende e crede che l'anno nuovo che sta per cominciare
    sarà bello, e il passante gli chiede se sarà bello come questo o gli ultimi anni, e
    lui risponde che lo sarà ancora di più. Tuttavia, il venditore non vorrebbe
    che l'anno nuovo fosse come quelli che ha già vissuto.
    Questo è una sorta di paradosso in quanto la medesima situazione avviene
    ogni anno.
    \\
    Il passante chiede a quale degli anni, fra quelli che ha vissuto il venditore, vorrebbe
    che il prossimo anno somigliasse.
    Tuttavia, il venditore è privo di risposta, nonostante tutti gli anni vissuti.
    \\\\
    La seguente domanda è quella del tornare indietro a rivivere la propria vita (in maniera identica).
    Il venditore non vorrebbe rivivere la propria vita nuovamente in maniera identica,
    nonostante vorrebbe tornare giovane.
    Addirittura, anche con la possibilità di scegliere di rivivere una vita altrui, il venditore
    non accetterebbe.
    Secondo il passante, nessuno accettere, nemmeno un principe.
    L'unica opzione per la quale si potrebbe rivivere una vita è senza la prospettiva di come
    essa si sviluppi.
    \\
    Le domande del passeggere colto sono destabilizzanti per il venditore, e mandano in crisi
    un'aspetto della vita che il venditore nona veva mai messo in dubbio.
    \\\\
    Questo paradosso è analogo a quello de \textit{Il sabato del villaggio}, 
    dove ogni settimana si trova piacere solo nell'attesa del piacere piuttosto
    che il piacere stesso.
    
    % Il sarcarmo di leopardi.
    % Se c'è un motivo per ammirare la natura, è ammirare qualcosa di cosÌ malvagio 1
\end{snippet}

\end{document}