\documentclass[preview]{standalone}

\usepackage{amsmath}
\usepackage{amssymb}
\usepackage{stellar}
\usepackage{bettelini}

\hypersetup{
    colorlinks=true,
    linkcolor=black,
    urlcolor=blue,
    pdftitle={Stellar},
    pdfpagemode=FullScreen,
}

\begin{document}

\title{Geografia economica}
\id{geoeconomica-concetto-sviluppo}
\genpage

\section{Storia e interpretazioni del concetto di sviluppo}

\begin{snippetdefinition}{indicatore-definition}{Indicatore}
    Un \textit{indicatore} è un valore o elemento/aspetto, come il PIL, il salario medio etc.
\end{snippetdefinition}

\begin{snippetdefinition}{indice-definition}{Indice}
    Un \textit{indice} è un insieme di 2 o più valori.
\end{snippetdefinition}

% https://moodle.edu.ti.ch/libe/pluginfile.php/119548/mod_resource/content/1/Pass%2004c%20Sviluppo-C%2023-24_schede.pdf

\begin{snippetdefinition}{pil-definition}{Prodotto Interno Lordo}
    Il \textit{prodotto interno lordo} (PIL) è pari alla somma dei beni e dei servizi finali
    prodotti da un paese in un dato periodo di tempo. Si dice interno perché si riferisce
    a quello che viene prodotto nel territorio del paese, sia da imprese nazionali sia
    da imprese estere. Se invece vogliamo riferirci solo a ciò che è prodotto da
    imprese nazionali, dobbiamo togliere dal PIL quel che è prodotto sul territorio
    nazionale da imprese estere e aggiungere quel che è prodotto all'estero da
    imprese nazionali: abbiamo così il prodotto nazionale lordo (PNL).
\end{snippetdefinition}

\plain{Quando c'è una guerra o una catastrofe il PIL cresce, ma non aumenta il tenore di vita.}

\begin{snippetdefinition}{isu-definition}{Indice di Sviluppo Umano}
    L'\textit{Indice di Sviluppo Umano} (ISU) si affianca al PIL e si inscrive
    nella logica della misurazione dello sviluppo umano che amplia la prospettiva
    della semplice crescita economica per definire il livello di sviluppo dei singoli
    paesi (è utilizzato con lo stesso fine anche per regioni e singole città).
    Questo indice si fonda sulla sintesi di tre diversi fattori: il PIL pro capite,
    l'alfabetizzazione e la speranza di vita.
\end{snippetdefinition}

\begin{snippetdefinition}{hpi-definition}{Happy Planet Index}
    L'\textit{Happy Planet Index} (HPI) è una misura dell'efficienza ambientale di una nazione,
    introdotto dalla New Economics Foundation
    (NEF) nel luglio 2006. Questo indice considera l'aspettativa di vita, la soddisfazione
    della vita soggettiva e una misura dei costi ambientali per considerare
    anche la sostenibilità globale.
    Questo indice aumenta con il benessere e la speranza di vita,
    ma diminuisce con l'impronta ecologica.
\end{snippetdefinition}

%% Indice di povertà normale

\begin{snippetdefinition}{mpi-definition}{Indice di Povertà Multidimensionale}
    Il MPI globale esamina le condizioni
    di deprivazione di una persona attraverso 10 indicatori relativi alla salute,
    all'istruzione e al tenore di vita e offre uno strumento per identificare chi è
    povero e in che misura lo è.
\end{snippetdefinition}

\end{document}