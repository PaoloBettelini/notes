\documentclass[preview]{standalone}

\usepackage{amsmath}
\usepackage{amssymb}
\usepackage{stellar}
\usepackage{definitions}
\usepackage{bettelini}

\begin{document}

\id{mechanics-ex-2}
\genpage

\section{Exercises - Batch 2}

\begin{snippetexercise}{mechanics-ex-2.1}{\underline{2.1}}
    An observer drops a stone into a well to measure its depth \(h\).
    The time interval between the initial moment and the instant the sound of the stone hitting the bottom
    is heard is \(\Delta t\). How much is \(h\)? (Take into account the speed of sound).
\end{snippetexercise}

\begin{snippetsolution}{mechanics-ex-2.1-sol}{\underline{2.1}}
    The value of \(\Delta t\) is given by the sum
    \begin{align*}
        \sqrt{\frac{2h}{g}} + \frac{h}{v_\text{suono}}
        = \Delta t
    \end{align*}
    We just need to solve for \(g\)
    \begin{align*}
        \sqrt{\frac{2h}{g}}
        &= \Delta t - \frac{h}{v_s} \\
        \frac{2h}{g} &= {\left(\Delta t - \frac{h}{v_s}\right)}^2 \\
        \Delta t^2 + \frac{h^2}{v_s^2} - \frac{2\Delta t h}{v_s} &= \frac{2h}{g} \\
        0 &= h^2 - 2 \Delta t v_s h - \frac{2v_s^2h}{g} + \Delta t^2 v_s^2 \\
        0 &= h^2 - \left(2\Delta t v_s + \frac{2v_s^2}{g} + \Delta t^2 v_s^2\right) \\
        h_{1,2} &= \Delta t v_s + \frac{v_s^2}{g} \left(
            1 \pm \sqrt{1 + \frac{2\Delta t g}{v_s}}
        \right)
    \end{align*}
    We consider the solution satisfying the original equation
    \[
        h = \Delta t v_s + \frac{v_s^2}{g} \left(
            1 + \sqrt{1 + \frac{2\Delta t g}{v_s}}
        \right)
    \]
\end{snippetsolution}

\begin{snippetexercise}{mechanics-ex-2.2}{\underline{2.2}}
    A projectile is fired at a target initially placed at a height \(h\) and dropped simultaneously with the shot.
    Show that the condition for the projectile to hit the target is that it is initially aimed at the target itself.
\end{snippetexercise}

\begin{snippetsolution}{mechanics-ex-2.2-sol}{\underline{2.2}}
    Let \(D\) be the horizontal distance between the target and the projectile
    and \(\alpha\) the launch angle.
    We want to show that
    \[
        \tan\alpha = \frac{h}{D}
    \]
    The equations for the projectile motions are
    \[
        \begin{cases}
            x_p = (v_0\cos\alpha)t \\
            y_p = (v_0\sin\alpha)t - \frac{1}{2}gt^2
        \end{cases}
    \]
    and the ones for the target are
    \[
        \begin{cases}
            x_t = D \\
            y_t = (v_0\sin\alpha)t - \frac{1}{2}gt^2
        \end{cases}
    \]
    The condition is that the two objects must meet.
    By setting \(x_p = x_t \land y_p = y_p\) we get
    \[
        \begin{cases}
            (v_0 \cos\alpha) t = D \\
            (v_0 \sin\alpha) t = h \\
        \end{cases}
    \]
    Without solving, we divide the conditions and get
    \[
        \tan\alpha = \frac{h}{D}
    \]
\end{snippetsolution}

\begin{snippetexercise}{mechanics-ex-2.3}{\underline{2.3}}
    A material point moves along a circular arc of radius \(R\) with the following equation of motion:
    \[
        s = s_0 \cos \omega t,
    \]
    where \(s\) is the curvilinear abscissa and \(s_0\) and \(\omega\) are given constants.
    Find the angular velocity and the normal and tangential components of the acceleration.
\end{snippetexercise}

\begin{snippetsolution}{mechanics-ex-2.3-sol}{\underline{2.3}}
    We know that \(s = R\theta\), and thus
    \[
        \theta(t) = \frac{s_0}{R} \cos \omega t = \theta_0 \cos \omega t
    \]
    The angular velocity is given by
    \[
        \Omega = \frac{d\theta}{dt} = -\omega\theta_0\sin\omega t
    \]
    Now, since \(v(t) = \Omega R\), we have
    \[
        \begin{cases}
            a_N = \frac{v^2}{R} = \omega^2 R\theta_0^2 \sin^2(\omega t) \\
            a_T = \frac{dv}{dt} = -\omega^2 R \theta_0 \cos(\omega t) = -\omega^2 s
        \end{cases}
    \]
\end{snippetsolution}

\begin{snippetexercise}{mechanics-ex-2.4}{\underline{2.4}}
    Two airplanes \(A\) and \(B\) have opposite velocities of magnitude \(v\) and their trajectories are two
    parallel straight lines separated by a distance \(d\).
    Let \(t=0\) be the instant when the line \(AB\) is perpendicular to the trajectories.
    The axis of the cannon mounted on \(A\) forms an angle \(\alpha\) with the axis of the airplane,
    and the projectiles are fired with a velocity of magnitude \(v_R\) relative to \(A\).
    At what instant \(t^*\) should airplane \(A\) fire to hit airplane \(B\)?
    [Do not consider gravitational acceleration].
\end{snippetexercise}

\begin{snippetsolution}{mechanics-ex-2.4-sol}{\underline{2.4}}
    The motion of \(A\) for \(t < t^*\) is given by
    \begin{align*}
        \begin{cases}
            x_A(t) = vt \\
            y_A(t) = 0
        \end{cases}
    \end{align*}
    while for \(t \geq t*\) we consider the projectile motion
    \begin{align*}
        \begin{cases}
            x_P(t) = vt^* + (v_r \cos \alpha + v)(t - t^*) \\
            y_P(t) = (v_r \sin \alpha)(t - t^*)
        \end{cases}
    \end{align*}
    The motion of \(B\) is given by
    \begin{align*}
        \begin{cases}
            x_B(t) = -vt \\
            y_B(t) = d
        \end{cases}
    \end{align*}
    We set them equal to eachother
    \begin{align*}
        \begin{cases}
            x_P(t) = x_B(t) \\
            y_P(t) = y_B(t) \\
        \end{cases}
    \end{align*}
    and thus we find
    \begin{align*}
        \begin{cases}
            vt^* + (v+v_r \cos \alpha)(t-t^*) = -vt \\
            (v_r \sin \alpha)(t-t*) = d
        \end{cases}
    \end{align*}
    From the second equation we get \(t-t^* = \frac{d}{v_r \sin \alpha}\).
    By substituting this value in the first we get
    \begin{align*}
        vt^* + (v + v_r \cos\alpha) \frac{d}{v_r \sin\alpha} &= -vt \\
         &= -v(t-t^*) - vt^* \\
         &= -v(t-t^*) - vt^* \\
         -\frac{d(2v + v_r\cos\alpha)}{2vv_r\sin\alpha} &= t^*
    \end{align*}
\end{snippetsolution}

\begin{snippetexercise}{mechanics-ex-2.5}{\underline{2.5}}
    A car starts from rest with uniformly accelerated motion with acceleration \(a\).
    After a time \(\tau\), a projectile is launched, assumed to move with constant velocity \(v_0\).
    Determine the minimum velocity \(v_0\) required to hit the car, as a function of \(a\) and \(\tau\).
    The motion can be considered purely one-dimensional.
\end{snippetexercise}

\begin{snippetsolution}{mechanics-ex-2.5-sol}{\underline{2.5}}
    The equations of motion are
    \begin{align*}
        \begin{cases}
            x_A(t) = \frac{1}{2}at^2 \\
            x_P(t) = v_0(t-\tau)
        \end{cases}
    \end{align*}
    We thus want \(x_A(t) = x_P(t)\)
    \begin{align*}
        \frac{1}{2}at^2 &= v_0(t-\tau) \\
        \frac{1}{2}at^2 + v_0\tau &= v_0t
    \end{align*}
    which means
    \begin{align*}
        t_{1,2} = \frac{v_0}{a} \pm \sqrt{\frac{v_0^2}{a^2} - \frac{2v_0\tau}{a}}
    \end{align*}
    the condition is given by the discriminant
    \[
        \frac{v_0^2}{a^2} > \frac{2v_0\tau}{a}
        \implies
        v_0 > 2a\tau
    \]
\end{snippetsolution}

\begin{snippetexercise}{mechanics-ex-2.6}{\underline{2.6}}
    A train moving in uniform rectilinear motion with velocity of magnitude \(v_0\) suddenly decelerates with
    constant deceleration of magnitude \(A\); as a consequence, a suitcase, precariously placed on the luggage rack,
    falls and lands on the floor of the train.
    Determine the trajectory of the suitcase as seen by an observer \(O\) stationary on the ground and by an
    observer \(O'\) on the train.
\end{snippetexercise}

\begin{snippetsolution}{mechanics-ex-2.6-sol}{\underline{2.6}}
    The stationary observer \(O\) sees the suitcase being launched with horizontal velocity \(v_0\)
    \[
        \begin{cases}
            x_O(t) = v_0 t \\
            y_O(t) = h- \frac{1}{2}gt^2
        \end{cases}
    \]
    By substituting \(t\) we get \(y_O = h-\frac{g}{2v_0^2} x_O^2\) which is a parabola.
    The moving observer \(O'\) feels a ficticious force with magnitude \(A\)
    \[
        \begin{cases}
            x_{O'}(t) = \frac{1}{2}At^2 \\
            y_{O'}(t) = h- \frac{1}{2}gt^2
        \end{cases}
    \]
    By substituting \(t\) we get \(y_{O'} = h-\frac{g}{A}x_{O'}\) which is a straight line.
\end{snippetsolution}

\end{document}