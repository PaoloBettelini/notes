\documentclass[preview]{standalone}

\usepackage{amsmath}
\usepackage{amssymb}
\usepackage{stellar}
\usepackage{definitions}

\begin{document}

\id{series-types}
\genpage

\section{Mengoli series}

\begin{snippetdefinition}{mengoli-series-definition}{Mengoli series}
    The \emph{Mengoli series} is the \series defined as
    \[
        \sum_{n=1}^\infty \frac{1}{n(n+1)}
    \]
\end{snippetdefinition}

\begin{snippetproposition}{mengoli-series-value}{Mengoli series value}
    \[
        \sum_{n=1}^\infty \frac{1}{n(n+1)} = 1
    \]
\end{snippetproposition}

\begin{snippetproof}{mengoli-series-value-proof}{mengoli-series-value}{Mengoli series value}
    We can expand the \partialsum
    \begin{align*}
        S_N &= \sum_{n=1}^N \frac{1}{n(n+1)} + \sum_{n=1}^N \frac{1}{n} - \frac{1}{n+1} \\
        &= \left(1 - \frac{1}{2}\right) + \left(\frac{1}{2} - \frac{1}{3}\right) \cdots \left(\frac{1}{N-1} - \frac{1}{N}\right)
        + \left(\frac{1}{N} - \frac{1}{N+1}\right) \\
        &= 1 - \frac{1}{N+1}
    \end{align*}
    Finally, we compute the limit
    \[
        \lim S_N = \lim \left(1 - \frac{1}{N+1}\right) = 1
    \]
\end{snippetproof}

\section{Geometric series}

\begin{snippetdefinition}{geometric-series-definition}{Geometric series}
    The \emph{geometric series} of \emph{common ratio} \(q\) is the \series of the form
    \[
        \sum_{n=0}^\infty q^n
    \]
\end{snippetdefinition}

\plain{The ratio between two adject terms is constant.}

\begin{snippetproposition}{geometric-series-partial-sum}{Geometric series partial sum}
    The \geometricseries of ratio \(q\) has \partialsum
    \[
        S_N = \begin{cases}
            \frac{q^{N+1}-1}{q-1} & q \neq 1 \\
            N + 1 & q = 1
        \end{cases}
    \]
\end{snippetproposition}

\begin{snippetproof}{geometric-series-partial-sum-proof}{geometric-series-partial-sum}{Geometric series partial sum}
    If \(q=1\), then \(S_N = N+1\).
    Otherwise,
    \begin{align*}
        S_N &= 1 + q + \cdots + q^N \\
        q S_n &= q + q^2 + \cdots + q^N + q^{N+1} \\
        &= (1 + q + \cdots + q^N) + q^{N+1} - 1 \\
        &= S_n + q^{N+1}-1 \\
        (q-1) S_N &= q^{N+1} - 1 \\
        S_N &= \frac{q^{N+1} - 1}{q-1}
    \end{align*}
\end{snippetproof}

\begin{snippettheorem}{geometric-series-behavior}{Geometric series behavior}
    A \geometricseries:
    \begin{enumerate}
        \item \seriesconverges for \(|q| < 1\) and always \convergesabsolutely
            \[
                \sum_{n=0}^\infty q^n = \frac{1}{1-q}
            \]
        \item \seriesdiverges for \(q \geq 1\);
        \item \seriesoscillates for \(q \leq -1\).
    \end{enumerate}
\end{snippettheorem}

\begin{snippetproof}{geometric-series-behavior-proof}{geometric-series-behavior}{Geometric series behavior}
    \begin{enumerate}
        \item If \(|q| < 1\),
            \[
                \lim \frac{q^{N+1} - 1}{q-1} = \frac{-1}{q-1} = \frac{1}{1-q}
            \]
        \item if \(q \geq 1\),
            \[ \lim S_N = +\infty \]
        \item if \(q \leq 1\),
            \[
                \sum_{n=0}^\infty q^n =
                \sum_{n=0}^\infty {(-1)}^n {|q|}^n
            \]
            and since \(|q|>1\), it \seriesoscillates.
    \end{enumerate}
\end{snippetproof}

\section{Telescoping series}

\begin{snippetdefinition}{telescoping-series-definition}{Telescoping series}
    A \textit{telescoping series} is a \series of general term \(\{a_n\}_{n=k}^\infty\)
    where \(a_n = b_n - b_{n+1}\) for some \sequence \(\{b_n\}_{n=k}^\infty\).
\end{snippetdefinition}

\begin{snippetproposition}{telescoping-series-partial-sum}{Telescoping series partial sum}
    Consider a \telescopicseries with general term \(\{a_n\}_{n=k}^\infty\)
    where \(a_n = b_n - b_{n+1}\) for some \sequence \(\{b_n\}_{n=k}^\infty\).
    Then, its \partialsum is given by
    \[ S_N = b_k - b_{N+1} \]
\end{snippetproposition}

\begin{snippetexample}{telescoping-series-example-1}{Telescoping Series}
    \begin{align*}
        &\sum_{n=0}^\infty \frac{1}{n^2 + 3n + 2}
        = \sum_{n=0}^\infty \left[ \frac{1}{n+1} - \frac{1}{n+2} \right]
        = \lim_{N \to \infty} \sum_{n=0}^N \left[ \frac{1}{n+1} - \frac{1}{n+2} \right] \\
        &= \lim_{N \to \infty} \frac{1}{1} - \frac{1}{2} + \frac{1}{2} - \frac{1}{3}
        + \frac{1}{3} - \frac{1}{4} + \cdots + \frac{1}{n} - \frac{1}{n+1} +
        \frac{1}{n+1} - \frac{1}{n+2} \\
        &= \lim_{N \to \infty} 1 - \frac{1}{n+2} = 1
    \end{align*}
\end{snippetexample}

\section{Harmonic series}

\begin{snippetdefinition}{harmonic-series-definition}{Harmonic series}
    The \textit{harmonic series} is the \series
    \[
        \sum_{n=1}^\infty \frac{1}{n}
    \]
\end{snippetdefinition}

\begin{snippettheorem}{harmonic-series-divergence-theorem}{The harmonic series diverges}
    The \harmonicseries diverges.
\end{snippettheorem}

\section{p-series}

\begin{snippetdefinition}{p-series-definition}{\(p\)-series}
    The \emph{\(p\)-series} is the \series of form
    \[\sum_{n=k}^\infty \frac{1}{n^p}\]
    for \(k > 0\).
\end{snippetdefinition}

\plain{The harmonic series is a p-series.}

\begin{snippettheorem}{p-series-test-theorem}{\(p\)-series test}
    The \(p\)-series converges if \(p > 1\) and diverges if \(p \leq 1\).
\end{snippettheorem}

\begin{snippetproposition}{nth-root-of-n-limit}{}
    \[ \lim_{n \to \infty} n^{\frac{1}{n}} = 1 \]
\end{snippetproposition}

\end{document}