\documentclass[preview]{standalone}

\usepackage{amsmath}
\usepackage{amssymb}
\usepackage{stellar}
\usepackage{definitions}

\begin{document}

\id{series-types}
\genpage

\section{Mengoli series}

\begin{snippetdefinition}{mengoli-series-definition}{Mengoli series}
    The \emph{Mengoli series} is the \series defined as
    \[
        \sum_{n=1}^\infty \frac{1}{n(n+1)}
    \]
\end{snippetdefinition}

\begin{snippetproposition}{mengoli-series-value}{Mengoli series value}
    \[
        \sum_{n=1}^\infty \frac{1}{n(n+1)} = 1
    \]
\end{snippetproposition}

\begin{snippetproof}{mengoli-series-value-proof}{mengoli-series-value}{Mengoli series value}
    We can expand the \partialsum
    \begin{align*}
        S_N &= \sum_{n=1}^N \frac{1}{n(n+1)} + \sum_{n=1}^N \frac{1}{n} - \frac{1}{n+1} \\
        &= \left(1 - \frac{1}{2}\right) + \left(\frac{1}{2} - \frac{1}{3}\right) \cdots \left(\frac{1}{N-1} - \frac{1}{N}\right)
        + \left(\frac{1}{N} - \frac{1}{N+1}\right) \\
        &= 1 - \frac{1}{N+1}
    \end{align*}
    Finally, we compute the limit
    \[
        \lim S_N = \lim \left(1 - \frac{1}{N+1}\right) = 1
    \]
\end{snippetproof}

\section{Geometric series}

\begin{snippetdefinition}{geometric-series-definition}{Geometric series}
    The \emph{geometric series} of \emph{common ratio} \(q\) is the \series of the form
    \[
        \sum_{n=0}^\infty q^n
    \]
\end{snippetdefinition}

\plain{The ratio between two adject terms is constant.}

\begin{snippetproposition}{geometric-series-partial-sum}{Geometric series partial sum}
    The \geometricseries of ratio \(q\) has \partialsum
    \[
        S_N = \begin{cases}
            \frac{q^{N+1}-1}{q-1} & q \neq 1 \\
            N + 1 & q = 1
        \end{cases}
    \]
\end{snippetproposition}

\begin{snippetproof}{geometric-series-partial-sum-proof}{geometric-series-partial-sum}{Geometric series partial sum}
    If \(q=1\), then \(S_N = N+1\).
    Otherwise,
    \begin{align*}
        S_N &= 1 + q + \cdots + q^N \\
        q S_n &= q + q^2 + \cdots + q^N + q^{N+1} \\
        &= (1 + q + \cdots + q^N) + q^{N+1} - 1 \\
        &= S_n + q^{N+1}-1 \\
        (q-1) S_N &= q^{N+1} - 1 \\
        S_N &= \frac{q^{N+1} - 1}{q-1}
    \end{align*}
\end{snippetproof}

\begin{snippettheorem}{geometric-series-behavior}{Geometric series behavior}
    A \geometricseries:
    \begin{enumerate}
        \item \seriesconverges for \(|q| < 1\) and always \convergesabsolutely
            \[
                \sum_{n=0}^\infty q^n = \frac{1}{1-q}
            \]
        \item \seriesdiverges for \(q \geq 1\);
        \item \seriesoscillates for \(q \leq -1\).
    \end{enumerate}
\end{snippettheorem}

\begin{snippetproof}{geometric-series-behavior-proof}{geometric-series-behavior}{Geometric series behavior}
    \begin{enumerate}
        \item If \(|q| < 1\),
            \[
                \lim \frac{q^{N+1} - 1}{q-1} = \frac{-1}{q-1} = \frac{1}{1-q}
            \]
        \item if \(q \geq 1\),
            \[ \lim S_N = +\infty \]
        \item if \(q \leq 1\),
            \[
                \sum_{n=0}^\infty q^n =
                \sum_{n=0}^\infty {(-1)}^n {|q|}^n
            \]
            and since \(|q|>1\), it \seriesoscillates.
    \end{enumerate}
\end{snippetproof}

\section{Telescoping series}

\begin{snippetdefinition}{telescoping-series-definition}{Telescoping series}
    A \textit{telescoping series} is a \series of general term \(\{a_n\}_{n=k}^\infty\)
    where \(a_n = b_n - b_{n+1}\) for some \sequence \(\{b_n\}_{n=k}^\infty\).
\end{snippetdefinition}

\begin{snippetproposition}{telescoping-series-partial-sum}{Telescoping series partial sum}
    Consider a \telescopicseries with general term \(\{a_n\}_{n=k}^\infty\)
    where \(a_n = b_n - b_{n+1}\) for some \sequence \(\{b_n\}_{n=k}^\infty\).
    Then, its \partialsum is given by
    \[ S_N = b_k - b_{N+1} \]
\end{snippetproposition}

\begin{snippetexample}{telescoping-series-example-1}{Telescoping Series}
    \begin{align*}
        &\sum_{n=0}^\infty \frac{1}{n^2 + 3n + 2}
        = \sum_{n=0}^\infty \left[ \frac{1}{n+1} - \frac{1}{n+2} \right]
        = \lim_{N \to \infty} \sum_{n=0}^N \left[ \frac{1}{n+1} - \frac{1}{n+2} \right] \\
        &= \lim_{N \to \infty} \frac{1}{1} - \frac{1}{2} + \frac{1}{2} - \frac{1}{3}
        + \frac{1}{3} - \frac{1}{4} + \cdots + \frac{1}{n} - \frac{1}{n+1} +
        \frac{1}{n+1} - \frac{1}{n+2} \\
        &= \lim_{N \to \infty} 1 - \frac{1}{n+2} = 1
    \end{align*}
\end{snippetexample}

\section{Harmonic series}

\begin{snippetdefinition}{harmonic-series-definition}{Harmonic series}
    The \textit{harmonic series} is the \series
    \[
        \sum_{n=1}^\infty \frac{1}{n}
    \]
\end{snippetdefinition}

\begin{snippettheorem}{harmonic-series-divergence-theorem}{The harmonic series diverges}
    The \harmonicseries diverges.
\end{snippettheorem}

\begin{snippetproof}{harmonic-series-divergence-theorem-proof}{harmonic-series-divergence-theorem}{The harmonic series diverges}
    The \series has positive terms, so it cannot oscillate and converges \ifandonlyif its \partialsum \sequence
    is \upperbound[bounded above]. We study \(S_{2^k}\):
    \begin{align*}
        S_1 &= 1 \\
        S_2 &= 1 + \frac{1}{2} \\
        S_4 &= 1 + \frac{1}{2} + \left(\frac{1}{3} + \frac{1}{4}\right) \\
        S_8 &= S_4 + \left(\frac{1}{5} + \frac{1}{6} + \frac{1}{7} + \frac{1}{8}\right) > 1 + \frac{3}{2} \\
        \cdots
    \end{align*}
    By \principleofinduction[induction], we have \(S_{2^n} > 1 + \frac{n}{2}\) (except for the first).
    Thus, \begin{align*}
        S_{2^{n+1}} = S_{2^n} + \left(\frac{1}{2^n + 1} + \frac{1}{2^n + 2} \cdots + \frac{1}{2^{n+1}}\right)
    \end{align*}
    We thus have \(2^n\) terms all bigger than the last one.
    This means that all of this is bigger than
    \[
        1 + n\frac{1}{2} + \frac{1}{2} = 1 + (n+1)\frac{1}{2}
    \]
    In conclusion, \(S_{2^n} > 1 + \frac{n}{2} \to \infty\) for \(n\to \infty\).
    The \sequence \(\{S_N\}\) is not \upperbound[bounded above].
\end{snippetproof}

\section{p-series}

\begin{snippetdefinition}{p-series-definition}{\(p\)-series}
    The \emph{\(p\)-series} is the \series of form
    \[\sum_{n=k}^\infty \frac{1}{n^p}\]
    for \(k > 0\).
\end{snippetdefinition}

\plain{The harmonic series is a p-series.}

\begin{snippettheorem}{p-series-test-theorem}{\(p\)-series test}
    The \(p\)-series converges if \(p > 1\) and diverges if \(p \leq 1\).
\end{snippettheorem}

\begin{snippetproof}{p-series-test-proof}{p-series-test}{\(p\)-series test}
    We start by assuming \(p \geq 2\).
    To start, \(\forall n \geq 2\) we have
    \[
        \frac{1}{n^p} \leq \frac{1}{n^2} < \frac{2}{n(n+1)}
    \]
    We confront the series with the Mengoli series.
    This works as \(\frac{n+1}{2} \leq n\) with \(n \geq 1\).
    Since the \series
    \[
        \sum^\infty \frac{1}{n(n+1)} = 1 \implies \sum^\infty \frac{1}{n^2} = \frac{\pi^2}{6} < 2
    \]
    Alternately, we note \(\frac{1}{n^2} \sim \frac{1}{n(n+1)}\).
    \todo
\end{snippetproof}

\section{p-q-series}

\begin{snippetdefinition}{p-q-series-definition}{\(p\text{-}q\)-series}
    The \emph{\(p\text{-}q\)-series} is the \series of form
    \[\sum_{n=2}^\infty \frac{1}{n^p{(\log n)}^q}\]
    for \(k > 0\).
\end{snippetdefinition}

\begin{snippettheorem}{p-q-series-test-theorem}{\(p\text{-}q\)-series test}
    The \(p\text{-}q\)-series converges if \(p > 1\), or if \(p=1 \land q>1\).
    Otherwise, it \seriesdiverges.
\end{snippettheorem}

\begin{snippetproof}{p-q-series-test-theorem-proof}{p-q-series-test-theorem}{\(p\text{-}q\)-series test}
    Consider the case \(p=1\), which is critical, and \(q \geq 0\).
    Then,
    \[
        \sum_{n=2}^\infty \frac{1}{n {\left(\log n\right)}^q}
    \]
    has clearly decreasing terms. Therefore, we can apply the condensation theorem.
    This series behaves like the series
    \[
        \sum_{n=2}^\infty 2^k \frac{1}{2^k \cdot {\left(\log 2^k\right)}^q}
        = \sum_{n=2}^\infty \frac{1}{{\left(\log 2\right)}^q \cdot k^q}
    \]
    which is a \(p\)-series and converges if and only if \(q > 1\).
    If \(q < 0\), then
    \[
        \frac{1}{n{(\log n)}^q} = \frac{{(\log n)}^{-q}}{n}
    \]
    is \eventually greater than or equal to \(\frac{1}{n}\) because \({(\log n)}^{-q} \to \infty\), and the \series
    \seriesdiverges by comparison with the harmonic series.

    For the case \(p > 1\), we show that the series converges for any \(q\).
    \begin{align*}
        \frac{
            1
        }{
            n^p {(\log n)}^q
        }
    \end{align*}
    At this point, the logarithm is less significant than the power (whether it is in the numerator for negative \(q\) or in the denominator for positive \(q\)), so the \series \seriesconverges.
    Since \(p > 1\), we can write \(p = p' + \varepsilon\)
    with \(p' > 1\) and \(\varepsilon > 0\) such that
    \begin{align*}
        a_n &= \frac{1}{n^{p'}\cdot n^\varepsilon {(\log n)}^q} \\
        &= \frac{1}{n^{p'}} \cdot \frac{1}{n^\varepsilon {(\log n)}^q}
    \end{align*}
    In any case, for every \(q\), the term
    \[
        \frac{1}{n^\varepsilon {(\log n)}^q} \to 0
    \]
    and it follows that \(a_n < \frac{1}{n^{p'}}\) \eventually, and the \series
    \seriesconverges by comparison with the \series
    \[
        \sum \frac{1}{n^{p'}} < +\infty
    \]
    because \(p' > 1\).

    The case \(p < 1\) is similar. We prove it by writing \(p = p' - \varepsilon\)
    with \(p' < 1\) and \(\varepsilon > 0\) so that
    \[
        a_n = \frac{1}{n^{p'}} \cdot \left(
            \frac{n^\varepsilon}{{(\log n)}^q}
        \right)
    \]
    where the second term diverges.
\end{snippetproof}

\end{document}