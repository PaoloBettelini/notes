\documentclass[preview]{standalone}

\usepackage{amsmath}
\usepackage{amssymb}
\usepackage{stellar}
\usepackage{definitions}
\usepackage{bettelini}

\begin{document}

\id{mechanics-energy}
\genpage

\section{Work}

\begin{snippetdefinition}{work-definition}{Work}
    Let \(\vec{F}\) be a constant force. The \emph{work}
    of \(\vec{F}\) over some displacement \(\vec{s}\) is defined as
    \[
        W \triangleq \vec{F} \cdot \vec{s}
    \]
    The unit measure of the work is the \emph{Joule}.
\end{snippetdefinition}

\plain{The path of the force is independent.}

\begin{snippettheorem}{work-non-constant-force-theorem}{Work of non-constant force}
    Let \(\vec{F}(t)\) be an arbitrary force. Then, the work
    of \(\vec{F}(t)\) over a path \(C\) is given by
    \[
        W = \int_C \vec{F}\,d\vec{s}
    \]
\end{snippettheorem}

\begin{snippetproof}{work-non-constant-force-theorem-proof}{work-non-constant-force-theorem}{Work of non-consntance force}
    Since the force is not constant, we subdivide the path into \(N\) intervals of size \(\Delta s\).
    The force is essentially constant for any given interval. The sum of the discrete works is given by
    \[
        \sum_{i=1}^N W_i = \sum_{i=1}^N \vec{F_i} \cdot \Delta s_i
    \]
    By taking the limit, we obtain the actual work
    \[
        W = \lim_{N \to \infty} \sum_{i=1}^N \vec{F_i} \cdot \Delta s_i
        = \int_C \vec{F}\,d\vec{s}
    \]
\end{snippetproof}

\plain{Constraint forces are always orthogonal to the movement, so they never do work.}

\subsection{Special case}

\begin{snippetproposition}{constant-force-work}{Constant force work}
    Given a constant force \(F\) and a path \(C\) from a point \(A\) to \(B\), the work is given by
    \[
        W = \vec{F} \cdot \vec{AB}
    \]
\end{snippetproposition}

\begin{snippetproof}{constant-force-work-proof}{constant-force-work}{Constant force work}
    The work is given by
    \[
        W = \int_C \vec{F}\,d\vec{s} = \vec{F} \int_C d\vec{s}
    \]
    In this case we need to sum the vectors of the infinitesimal subdivision.
    \[
        W = \vec{F} \cdot \vec{AB}
    \]
\end{snippetproof}

\begin{snippetproposition}{linear-path-work}{Linear path work}
    Given a force \(F\) and a linear path from a point \(A\) to \(B\), the work is given by
    \[
        W = \integral[A][B][F(x)][x]
    \]
\end{snippetproposition}

\begin{snippetproof}{linear-path-work-proof}{linear-path-work}{Linear path work}
    If the trajectory is linear we have
    \[
        W = \int_C \vec{F}\,d\vec{s} = \integral[A][B][F(x)][x]
    \]
\end{snippetproof}

\section{Power}

\begin{snippetdefinition}{power-definition}{Power}
    \emph{Power} is defined as the derivative of work with respect to time
    \[
        P \triangleq \frac{dW}{dt}
    \]
    The unit measure of power is the \emph{Watt}.
\end{snippetdefinition}

\begin{snippetproposition}{power-velocty-force}{}
    \[
        P(t) = \vec{F}(t) \cdot \vec{v}(t)
    \]
\end{snippetproposition}

\begin{snippetproof}{power-velocty-force-proof}{power-velocty-force}{}
    \begin{align*}
        \frac{dW}{dt} &= \lim_{h\to 0} \frac{W(t + h) - W(t)}{h} \\
        &= \lim_{h\to 0} \frac{\vec{F} \cdot \Delta \vec{s}}{h} = \vec{F} \vec{v}
    \end{align*}
\end{snippetproof}

\section{Kinetic energy}

\begin{snippettheorem}{work-energy-theorem}{Work-Energy Theorem }
    \[
        W = E_C(t_2) - E_C(t_1)
    \]
\end{snippettheorem}

\begin{snippetproof}{work-energy-theorem-proof}{work-energy-theorem}{Work-Energy Theorem}
    We start by using the definition of power and \(\vec{F}=m\vec{a}\). We have
    \begin{align*}
        \frac{dW}{dt} &= m\vec{a}\vec{v} = m\vec{v}\frac{d\vec{v}}{dt}
    \end{align*}
    Since
    \[
        \frac{d}{dt}\left(\vec{v}\cdot \vec{v}\right) = 2\vec{v}\frac{d\vec{v}}{dt}
    \]
    we thus have
    \begin{align*}
        \frac{dW}{dt} &= m \frac{1}{2} \frac{d(\vec{v} \cdot \vec{v})}{dt}
        = \frac{d}{dt} \left(\frac{1}{2} m v^2(t)\right)
    \end{align*}
    From the fundamental theorem of calculus
    \[
        W = \integral[t_1][t_2][\frac{dW}{dt}][t]
        = \integral[t_1][t_2][\frac{d}{dt}E_c][t]
        = E_C(t_2) - E_C(t_1)
    \]
\end{snippetproof}

\section{Examples}

\begin{snippettheorem}{escape-velocity-theorem}{Escape velocity}
    The minimum velocity required to escape the gravitational field
    of a planet of radius \(R\) and mass \(M\) is
    \[
        v = \sqrt{\frac{2GM}{R}}
    \]
\end{snippettheorem}

\begin{snippetproof}{escape-velocity-theorem-proof}{escape-velocity-theorem}{Escape velocity}
    In order for the body to escale, it must have enough energy to go from \(x = R\) to \(x = \infty\).
    The energy required to do so is
    \begin{align*}
        \integral[R][\infty][\frac{GMm}{r^2}][r]
        = - \frac{GMm}{R}
    \end{align*}
    Thus, the initial kinetic energy of the body must be equal to it
    \begin{align*}
        \frac{1}{2}mv^2 &= - \frac{GMm}{R} \\
        v &= \sqrt{\frac{2GM}{R}}
    \end{align*}
\end{snippetproof}

\plain{We will now derive the equation of a pendulum using the conservation of energy.}

\begin{snippetproof}{simple-pendulum-equation-energy-proof}{simple-pendulum-equation}{Simple pendulum equation}
    The total energy of the system is given by
    \[
        \frac{1}{2}m {\left(R\frac{d\theta}{dt}\right)}^2
        + mgR(1-\cos\theta)
    \]
    Since the energy is conserved,
    \begin{align*}
        \frac{dE}{dt} &= 0 \\
        0 &= \frac{d\theta}{dt}\left(mR^2 \frac{d^2\theta}{d^2t} + mgR\sin\theta\right) \\
        \frac{d\theta^2}{dt^2} &= -\frac{g}{L}\sin\theta
    \end{align*}
\end{snippetproof}

\end{document}