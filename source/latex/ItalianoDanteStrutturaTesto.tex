\documentclass[preview]{standalone}

\usepackage{amsmath}
\usepackage{amssymb}
\usepackage{stellar}
\usepackage{tikz}

\usetikzlibrary{cd}

\hypersetup{
    colorlinks=true,
    linkcolor=black,
    urlcolor=blue,
    pdftitle={Stellar},
    pdfpagemode=FullScreen,
}

\begin{document}

\title{Stellar}
\id{italiano-struttura-commedia}
\genpage

\section{Struttura del testo}

\begin{snippet}{struttura-testo-dante-illustration}
    % https://tikzcd.yichuanshen.de/#N4Igdg9gJgpgziAXAbVABwnAlgFyxMJZABgBoBGAXVJADcBDAGwFcYkQAdDnGADx2ABjemDyCAFjAC+IKaXSZc+QinIVqdJq3Zce-XMAAKzAE4Bzejggn8MuQux4CRNcQ0MWbRJ2588AgEkwADMYE0g7eRAMR2UXUgAmdy0vHz1-I3oTeigsOAhIhyVnFATSNxoPbW9dP2ByAGoAZiaAAmFRLELoxScVZDKqSpSdX35gFvaRPG6Y4v6ypOHPUfSJto6Z2Q0YKDN4IlBgkwgAWyQykCskABYaSRz2HAB3CAeoBHsQY7OkNSuIEgAKz3GCPbwvN5gj6yKI-c6IJo0a6IABsoPBV1e70+cJOCLIAL+X3hSEJKMuACMYGAoEgmsQSfiycjAYiaNTaUgALQMqSUKRAA
    \begin{center}
    \begin{tikzcd}
        & \textit{Inferno} \arrow[r, two heads]    & \text{1+33 canti} \\
        \text{Cantiche} \arrow[r] \arrow[ru, bend left] \arrow[rd, bend right] & \textit{Purgatorio} \arrow[r, two heads] & \text{33 canti}   \\
        & \textit{Paradiso} \arrow[r, two heads]   & \text{33 canti}  
    \end{tikzcd}
    \end{center}
    \phantom{}
\end{snippet}

\section{L'inferno}

\begin{snippet}{inferno-composizione}
    L'inferno è composto da settori sempre più stretti. Più lo spazio diminuisce e più i peccati sono immorali
    secondo Dante.
    Principalmente, l'inferno è suddiviso in 3 sezioni.
    Dall'alto verso il basso, ci sono gli \textit{incontinenti}, \textit{violenti} e
    i \textit{freudolenti}.
\end{snippet}

\begin{snippetdefinition}{legge-del-contrappasso-definition}{Legge del contrappasso}
    La \textit{legge del contrappasso} associa una pena
    legata alla colpa.
    Il nesso avviene o per \textit{analogia} o per \textit{opposizione}.
\end{snippetdefinition}

%\subsection{Struttura dell'inferno}
%\begin{figure}[h]
%    \centering
%    \includegraphics[width=0.5\textwidth]{./inferno.jpg}
%\end{figure}

\section{Lucifero}

\begin{snippet}{lucifero-expl}
    Lucifero viene rappresentato come un grande orrenda cretura con 3 bocche.
    In ogni bocca mastica per l'eternità i 3 peccatori più grandi.
    Al centro Giuda, mentre ai lati Bruto e Cassio.
\end{snippet}

\end{document}