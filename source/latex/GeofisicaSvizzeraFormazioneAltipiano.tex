\documentclass[preview]{standalone}

\usepackage{amsmath}
\usepackage{amssymb}
\usepackage{stellar}

\hypersetup{
    colorlinks=true,
    linkcolor=black,
    urlcolor=blue,
    pdftitle={Stellar},
    pdfpagemode=FullScreen,
}

\begin{document}

\id{geofisica-svizzera-formazione-altipiano}
\genpage

\section{Formazione dell'altipiano}

\begin{snippet}{formazione-altipiano-svizzero}
    L'Altopiano svizzero ha origini strettamente legate alla formazione delle Alpi. Originariamente, il
    massiccio alpino in formazione si rifletteva in un'ampia distesa di acque, chiamata Mare delle
    molasse, situata a nord delle Alpi. Queste acque, collegate occasionalmente all'oceano,
    alternavano tra acqua dolce e marina a causa di sollevamenti e sprofondamenti del fondale. Nelle
    ultime fasi della formazione montuosa, l'area si interrò completamente. Le rocce dell'Altopiano si
    dividono in quattro tipi di molasse: di mare inferiori, d'acqua dolce inferiori, di mare superiori e
    d'acqua dolce superiori. Questi strati sono composti dai detriti erosi dalle Alpi, con i fiumi che
    depositavano ghiaia vicino alla costa, sabbia più al largo e melma argillosa in zone più lontane. La
    solidificazione e cementazione di questi detriti con calcare formarono puddinghe, arenarie e
    marne, tipici elementi della molassa. La formazione delle montagne e la sedimentazione sono
    strettamente correlate, permettendo di ricostruire le fasi dell'orogenesi attraverso le stratificazioni
    dell'Altopiano. Ad esempio, la puddinga del Rigi mostra strati inferiori di puddinga calcarea grigia
    con ciottoli di antichi strati erosi, mentre strati superiori contengono ciottoli rossastri cristallini,
    indicando l'erosione avanzata dello zoccolo cristallino. Le ultime fasi di formazione montuosa
    influenzarono fortemente le zone antistanti. Ai margini delle Alpi, la molassa fu piegata e corrugata
    dalle falde di ricoprimento elvetiche in avanzamento. Il confine geologico tra Alpi e Altopiano è
    segnato dalla molassa giovane che emerge dalle vecchie rocce calcaree elvetiche. La molassa
    subalpina è caratterizzata da estesi piegamenti, con molassa intatta che si trova solo verso il Reno,
    avanzando leggermente verso nord in corrispondenza con i delta fluviali.
\end{snippet}

\section{Altopiano del giura}

\begin{snippet}{formazione-altopiano-giura}
    Anche il Giura è un prodotto secondario del corrugamento delle Alpi. Le ultime forti spinte hanno
    interessato il lato occidentale e settentrionale del complesso molassico, trasformando le masse
    calcaree della regione giurassica in pieghe di modesta altezza che avanzano a forma di falce tra il
    Massiccio centrale e i Vosgi francesi verso il bacino parigino. Questa prima fase di corrugamento, di
    per sé non molto rilevante, fu in gran parte cancellata da una seconda spinta molto più intensa,
    che fece emergere lungo il bordo settentrionale dell'Altopiano centrale catene montuose di
    notevole grandezza. Il Giura a pieghe presenta quindi due aspetti distinti: Giura ad altipiani:
    caratterizzato da una superficie leggermente ondulata, con alture a coste e dossi (Franches-
    Montagnes) che non sono pieghe anticlinali, ma parti rocciose più dure. Giura a catene: con creste
    che si allungano parallelamente, separate da valli longitudinali e frequentemente interrotte da
    profonde chiuse (dal Giura lionese fino ai Lägern). Basilea-campagna e il nord dell'Argovia fanno
    parte del Giura tabulare. In questa zona, di fronte al massiccio della Foresta Nera, i fenomeni di
    formazioni montuose non riuscirono a far scorrere gli strati calcarei sul sottosuolo cristallino,
    mantenendo una posizione più o meno orizzontale, ma fratturata da profonde faglie da sud verso
    nord. Ciò ha portato alla formazione di altipiani a tavolato con valli dai ripidi pendii. Verso ovest,
    paesaggi simili al Giura tabulare si trovano nell'Ajoie; verso est, il Giura tabulare continua
    attraverso il Randen (Sciaffusa) fino all'Altopiano svevo in Germania meridionale.
\end{snippet}

\end{document}