\documentclass[preview]{standalone}

\usepackage{amsmath}
\usepackage{amssymb}
\usepackage{bettelini}
\usepackage{stellar}
\usepackage{makecell}

\hypersetup{
    colorlinks=true,
    linkcolor=black,
    urlcolor=blue,
    pdftitle={English},
    pdfpagemode=FullScreen,
}

\begin{document}

\title{English}
\id{english-atonement-chapter11-14}
\genpage

\section{Summary}

\section{Exercises}

\begin{snippetexercise}{atonement-ex-9}%
    {What does chapter 11 reveal about the state of the Tallis' family?}
    The Tallis's family is very dysfunctional.
    For example, after the disappearance of the twins,
    the mother does not want to call the police because she does not want
    the outside to know about their problems, minimizing the problem.
    Emily's own self-absorption prevents her from recognizing any of the problems
    happening within her own household. The Tallis' family is not united.
\end{snippetexercise}

\begin{snippetexercise}{atonement-ex-10}%
    {Read the following sentences, what do they indicate?
    What is the name of this literary device? What does
    it add to the reader's interpretation of the events?}
    \begin{center}
        \quotes{\textit{This decision, as he was to acknowledge many times, transformed his life.}} \\
        \quotes{\textit{Within the half hour, Briony would commit her crime.}}
    \end{center}
    The decision of Briony will play a crucial role in shaping the character's future.
    This literary technique is called \textit{prolepsis}.
    where the author anticipates future events in the story, adding suspence.
\end{snippetexercise}

\begin{snippetexercise}{atonement-ex-11}%
    {Look closely at the description of the rape from Briony's point of view (chapter 13, pp. 154-5).
    What details are included? Which ones are omitted? Why is this relevant to the story?}
    Briony sees a bush which starts to move and shift. Given the weirdness of the phenomenon,
    Briony convinces herself that this is just an illusion.
    Briony finds Lola and sees a dark figure running away, without seeing his face.
    This is important because she thinks she has have seen Robbie.
    We could also say that she \textit{wants} him to be Robbie.
\end{snippetexercise}

\begin{snippetexercise}{atonement-ex-12}%
    {Does Lola know who the man who raped her was? What do we find out?}
    Lola does not know who raped her even thought she seems him walking away.
    The only clue within Lola is that she suspects to know the man, but is
    persuaded by Briony and convinced about Robbie's fault.
    It is possible that Lola suspected Paul but was too unsure or weak to accuse him.
    We are not told who the actual culprit is in this chapter,
    but we are already tuned to the fact that Paul was attracted to Lola.
\end{snippetexercise}

\begin{snippetexercise}{atonement-ex-13}%
    {What motivates Briony to maintain her version of the story with such steadfast determination?
    What misconceptions support Briony in her actions and behaviors?}
    Briony maintains her version of the story because she is certain who the culprit is,
    her misconceptions about Robbie given by her lack of understand about sexuality
    strengthen her belief.
    She sees Robbie as a maniac and is convinced of being helping her sister Cecilia.
\end{snippetexercise}

\begin{snippetexercise}{atonement-ex-14}{
    What is the narrator used in Part One? How does it contribute to our understanding of the
characters? }
    By employing an omniscient third-person narrator, McEwan grants readers
    access to the inner thoughts of various characters in different chapters of Part One
    (with an internal focalizer's perception).
    This narrative technique not only reveals Briony as an unreliable
    narrator but also fosters empathy for Cecilia and Robbie
    through their detailed character development.
\end{snippetexercise}

\end{document}
