\documentclass[preview]{standalone}

\usepackage{amsmath}
\usepackage{amssymb}
\usepackage{stellar}
\usepackage{bettelini}

\hypersetup{
    colorlinks=true,
    linkcolor=black,
    urlcolor=blue,
    pdftitle={Stellar},
    pdfpagemode=FullScreen,
}

\begin{document}

\title{Geografia economica}
\id{geoeconomica-societa-transnazionali}
\genpage

\section{Le società transnazionali}

\begin{snippet}{e4efe4d3-394e-4a18-8832-383d3dc0950a}
    Le imprese multinazionali sono tra gli attori più potenti dello spazio globale. La loro
    internazionalizzazione è sia la conseguenza che uno dei motori della globalizzazione. Di fronte alle loro
    strategie globali e ai loro modi di lavorare transnazionali, gli Stati hanno difficoltà a introdurre un
    sistema di governance che possa compensare le conseguenze sociali e ambientali delle loro attività.
\end{snippet}

% https://moodle.edu.ti.ch/libe/pluginfile.php/121154/mod_resource/content/0/Le%20multinazionali_AtlasEspaceMondial_Deepl_SoloTesto.pdf

\begin{snippetdefinition}{catena-produttiva-definizione}{Catena produttiva}
    La \textit{catena produttiva} (\textit{global commodity chain}) è l'insieme di relazioni e di flussi che collegnao
    fra loro le imprese e gli stabilimenti produttivi coinvolti nella 
    produzione, distribuzione e commercializzazione di un bene.
\end{snippetdefinition}

\begin{snippet}{9cd585fa-26ce-4732-8f11-400b8e6398ce}
    Le multinazionale possono consistere nelle seguenti realtà:
    \begin{enumerate}
        \item commercio materie prime \(\rightarrow\) settore I;
        \item delocalizzazione della produzione \(\rightarrow\) settore II;
        \item gestione internale dei servizi \(\rightarrow\) settore III;
        \item gruppi complessi \(\rightarrow\) (I+II+III).
    \end{enumerate}
    
    Le multinazionali hanno dei vantaggi:
    \begin{itemize}
        \item Produttivi (costo minore energia, leggi estere più deboli circa lo smaltimento e consumi);
        \item commerciali;
        \item fiscali (imposte fiscali, valori diversi della merce).
    \end{itemize}
    
    l'internazionalizzazione si è avvantaggiata
    soprattutto dell'apertura del commercio tra gli Stati nell'ambito degli accordi GATT
    e poi dell'OMC (Organizzazione Mondiale del Commercio),
    nonché della liberalizzazione finanziaria, che ha dato luogo a una maggiore mobilità dei capitali, a
    una tendenza alla diminuzione dei costi di trasporto e allo sviluppo delle tecnologie
    dell'informazione e delle telecomunicazioni.
    
    L'internazionalizzazione delle multinazionali ha certamente permesso ad alcuni Paesi del Sud di
    recuperare il ritardo economico - Paesi come la Cina, il cui sviluppo si basa sulla ricezione di IDE e
    sull'inserimento nel processo di globalizzazione. Ma contribuisce anche ad approfondire le
    disuguaglianze interne: mettendo in competizione i lavoratori dei Paesi ricchi con quelli dei Paesi
    in via di sviluppo, contribuisce ad aumentare la disoccupazione nei Paesi sviluppati, che si stanno
    deindustrializzando, favorendo al contempo la comparsa di classi agiate nei Paesi del Sud. Poiché
    le imprese multinazionali sono spesso in posizioni di potere rispetto agli Stati, mettono questi
    ultimi in competizione tra loro per l'attrattività dei loro territori (servizi, sussidi e persino
    l'allentamento delle norme fiscali, sociali e ambientali).
\end{snippet}

\begin{snippetdefinition}{greenwashing-definizione}{Greenwashing}
    Con il termine \textit{greenwashing} si indica la tecnica di
    spendere una buona parte di budget pubblicitario per pubblicizzare il fatto di essere green.
\end{snippetdefinition}

% INVESTIMENTI DIRETTI ESTERI
% Sono un indicatore dell'attività delle multinazionali.

\end{document}