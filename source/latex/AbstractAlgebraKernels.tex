\documentclass[preview]{standalone}

\usepackage{amsmath}
\usepackage{amssymb}
\usepackage{stellar}
\usepackage{definitions}
\usepackage{bettelini}
\usepackage{enumitem}

\begin{document}

\id{kernels}
\genpage

\section{Kernel of a group}

\begin{snippet}{group-homomorphism-preserves-subgroup-particular-cases}
    In the trivial cases where we consider \(1_G\) and \(G\) as the \subgroup[subgroups]
    of \(G\), we have \(\varphi(1_G)\) and \(\varphi(G)\) are \subgroup[subgroups]
    of \(G\). The case \(\varphi(1_G) = 1_H\) is not particularly interesting.
    The case \(\varphi(G) = \text{Im}\{\varphi\}\), meaning that the image of a \grouphomomorphism
    is a \subgroup. Now consider \(1_H\) and \(H\) as \subgroup[subgroups]
    of \(H\). We thus have \(\varphi^\inversefunction(1_H)\) and \(\varphi^\inversefunction(H)\),
    which are \subgroup[subgroups] of \(G\). The second case is not interesting as
    \(\varphi^\inversefunction(H) = G\). In the first case we only have a specific \subgroup.
\end{snippet}

\begin{snippetdefinition}{group-kernel-definition}{Kernel}
    Let \(\varphi \colon G \fromto H\) be a \grouphomomorphism.
    The \emph{kernel} of \(\varphi\), is defined as
    \[
        \text{ker}_\varphi \triangleq \{x\in G \suchthat \varphi(x) = 1_H\}
    \]
\end{snippetdefinition}

\begin{snippettheorem}{kernel-equivalent-properties-theorem}{Kernel coset properties}
    Let \(\varphi \colon G \fromto H\) be a \grouphomomorphism
    and let \(x,y \in G\). The following are equivalent:
    \begin{enumerate}[label=(\arabic*)]
        \item \(\varphi(x) = \varphi(y)\);
        \item \(\grpker_\varphi x = \grpker_\varphi y\);
        \item \(x\,\grpker_\varphi = y\,\grpker_\varphi\).
    \end{enumerate}
\end{snippettheorem}

\plain{In the kernel, the right and left cosets are the same.}

\begin{snippetproof}{kernel-equivalent-properties-theorem-proof}{kernel-equivalent-properties-theorem}{Kernel coset properties}
    \begin{itemize}
        \item \((1) \implies (2)\): consider \(k = xy^{-1}\). We have that
        \[\varphi(k) = \varphi(xy^{-1}) = \varphi(x)\varphi(y^{-1}) = (\varphi(x)){(\varphi(x))}^{-1}\]
        and thus \(k \in \grpker_\varphi\). Now, \(x = xy^{-1}y = ky \in \grpker_\varphi y\).
        and thus \(\grpker_\varphi x = \grpker_\varphi y\).
        \item \((2) \implies (1)\): we know that \(\grpker_\varphi x = \grpker_\varphi y\) and thus
        \(x = ky\) for some \(k\in \grpker_\varphi\). But then,
        \[
            \varphi(x) = \varphi(ky) = \varphi(k) \varphi(y) = 1_H \varphi(y) = \varphi(y)
        \]
        \item \((1) \implies (3)\): analogous;
        \item \((3) \implies (1)\): analogous.
    \end{itemize}
\end{snippetproof}

\begin{snippetcorollary}{homomorphism-injectivity-equivalence}{Homomorphism injectivity}
    Let \(\varphi \colon G \fromto H\) be a \grouphomomorphism.
    Then, \(\varphi\) is \injective \ifandonlyif \(\grpker_\varphi\) is trivial.
\end{snippetcorollary}

\begin{snippetproof}{homomorphism-injectivity-equivalence-proof}{homomorphism-injectivity-equivalence}{Homomorphism injectivity}
    \iffproof{
        Let \(x \in \grpker_\varphi\). We have that
        \(\varphi(x) = 1_H = \varphi(1_G)\), since the \grouphomomorphism is \injective
        we find \(x = 1_G\).
    }{
        Let \(x,y \in G\) such that \(\varphi(x) = \varphi(y)\).
        By \snippetref[kernel-equivalent-properties-theorem][this theorem],
        they are both in the same (right) coset. Since \(\grpker_\varphi\)
        is trivial, there is only one element in the coset, meaning \(x=y\).
    }
\end{snippetproof}

\plain{We do not define the kernel of a monoid analogously to the one of a group.
Such a definition would still be a submonoid, but if the kernel only contained the identity,
that would not imply that the homomorphism is injective (only the other way around).}

\begin{snippetexample}{monoid-homomorphism-kernel-example}{}
    Consider \((\integers \difference \{-1\}, \cdot)\)
    and the \monoidhomomorphism to the integers where \(\varphi(x) \triangleq x^2\).
    We have that \(\varphi(xy) = \varphi(x) \varphi(y)\).
    However, \(\varphi\) is not \injective. Only the element \(1\) goes into the identity,
    but the \function is not \injective like it would be in a \group.
\end{snippetexample}

\begin{snippettheorem}{generated-subgroup-homomorphism-theorem}{Generated subgroup homomorphism}
    Let \(\varphi \colon G \fromto H\) be a \grouphomomorphism
    and let \(X \subseteq G\). Then,
    \[
        \varphi(\gengrp{X}) = \gengrp{\varphi(X)}
    \]
    If \(Y \subseteq H\), it is not guaranteed that
    \[
        \varphi^\inversefunction(\gengrp{Y}) = \gengrp{\varphi^\inversefunction(Y)}
    \]
\end{snippettheorem}

\plain{The same with the inverse operation is not necessarily true.}

\begin{snippetproof}{generated-subgroup-homomorphism-theorem-proof}{generated-subgroup-homomorphism-theorem}{Generated subgroup homomorphism}
    If \(X = \emptyset\), we know \(\gengrp{X} = 1_G\) and, thus,
    \(\varphi(\gengrp{X}) = 1_H\). On the other hand, \(\gengrp{\varphi(X)} = 1_H\).
    Otheriwse, \(X \neq \emptyset\). We know that the element of \(\gengrp{X}\)
    have the form \(x_1\circ x_2 \cdots x_r\) where \(x_i \in X\) or \(x_i^{-1} \in X\).
    Now,
    \begin{align*}
        \varphi(\gengrp{X}) &= \{\varphi(x_1 \circ x_2 \cdots x_r) \suchthat
        x_i \in X \lor x_i^{-1} \in X\} \\
        &= \{ \varphi(x_1) \circ \varphi(x_2) \circ \varphi(x_r) \suchthat
        \varphi(x_i) \in \varphi(X) \lor \varphi(x_i^{-1}) \in \varphi(X) \} \\
        &= \{ \varphi(x_1) \circ \varphi(x_2) \circ \varphi(x_r) \suchthat
        \varphi(x_i) \in \varphi(X) \lor {(\varphi(x_i))}^{-1} \in \varphi(X) \} \\
        &= \gengrp{\varphi(X)}
    \end{align*}
    Let now \(Y \subseteq H\). We construct an example to show that the property does not hold
    necessarily. Consider
    \[
        \varphi\colon (\integers, +) \fromto \permgrp_3
    \]
    defined as \(\varphi(n) \triangleq {\kcycle{1, 2, 3}}^n\) which is a \grouphomomorphism.
    Also, consider \(Y = \{\kcycle{1,2}\}\).
    Thus, \(\gengrp{Y} = \{\identityfunc, \kcycle{1,2}\}\).
    On the other hand, \(\varphi^\inversefunction(Y) = \emptyset\)
    as now 3-cycle raised to a power will result in a \permswap.
    Thus, \(\gengrp{\varphi^\inversefunction(Y)} = \emptyset\).
    However, \(\varphi^\inversefunction(\gengrp{Y}) = 3\integers\).
\end{snippetproof}

\begin{snippetcorollary}{generated-subgroup-homomorphism-particular-case}{}
    Let \(\varphi \colon G \fromto H\) be a \grouphomomorphism
    and let \(g\in G\). Then,
    \[
        \varphi(\gengrp{g}) = \gengrp{\varphi(g)}
    \]
    Furthermore, if \(g\) has finite period, \(|\varphi(\gengrp{g})|\) divides
    \(|\gengrp{g}|\).
\end{snippetcorollary}

\begin{snippetproposition}{generated-subgroup-homomorphism-reverse}{}
    Let \(\varphi \colon G \fromto H\) be a \grouphomomorphism
    and let \(Y \subseteq H\). Then,
    \[
        \gengrp{\varphi^\inversefunction(Y)} \subgroupleq
        \varphi^\inversefunction(\gengrp{Y})
    \]
\end{snippetproposition}

\begin{snippetproposition}{homomorphism-kernel-product}{}
    Let \(\varphi \colon G \fromto H\) be a \grouphomomorphism
    and let \(K \subgroupleq G\). Then,
    \[
        \varphi^\inversefunction(\varphi(K)) = K\,\grpker_\varphi
    \]
    In particular,
    \[
        \varphi^\inversefunction(\varphi(K)) = K \iff \grpker_\varphi \subgroupleq K
    \]
\end{snippetproposition}

\begin{snippetproof}{homomorphism-kernel-product-proof}{homomorphism-kernel-product}{}
    If \(x \in \grpker_\varphi\), we have that
    \(\varphi(x) = 1_H \in \varphi(K)\). By definition,
    \(x\in \varphi^\inversefunction(\varphi(K))\).
    Thus, \(\grpker_\varphi \subgroupleq \varphi^\inversefunction(\varphi(K))\).
    Since \(K \subgroupleq \varphi^\inversefunction(\varphi(K))\), we have the thesis.
    Likewise, let \(x \in  \varphi^\inversefunction(\varphi(K))\). By definition, this means that
    \(\varphi(x) \in \varphi(K)\), meaning that there exist some \(k\in K\)
    such that \(\varphi(x) = \varphi(k)\). Since those are equal, \(x\) and \(k\)
    are in the same coset (right or left) of \(\grpker_\varphi\),
    meaning that there exist \(z \in \grpker_\varphi\) such that
    \(x = kz\). Thus, \(x\in K\,\grpker_\varphi\). \\
    In particular, \(K = \varphi^\inversefunction(\varphi(K))\) \ifandonlyif
    \(K\,\grpker_\varphi\) which happens \ifandonlyif \(\grpker_\varphi \subgroupleq K\).
\end{snippetproof}

\begin{snippetcorollary}{kernel-product-is-subgroup}{Kernel product is subgroup}
    Let \(\varphi \colon G \fromto H\) be a \grouphomomorphism
    and let \(K \subgroupleq G\).
    Then,
    \[
        K\,\grpker_\varphi \subgroupleq G
    \]
    and thus \(K\,\grpker_\varphi = \grpker_\varphi K\).
\end{snippetcorollary}

\begin{snippetproposition}{double-direction-homomorphism-image}{}
    Let \(\varphi \colon G \fromto H\) be a \grouphomomorphism
    and let \(L \subgroupleq H\). Then,
    \(\varphi(\varphi^\inversefunction(L)) = L \intersection \text{Im}\{\varphi\}\).
    In particular, \(L = \varphi(\varphi^\inversefunction(L))\) \ifandonlyif \(L \subgroupleq \text{Im}\{\varphi\}\).
\end{snippetproposition}

\begin{snippetproof}{double-direction-homomorphism-image-proof}{double-direction-homomorphism-image}{}
    Let \(x\in H\), we have that \(x\in \varphi(\varphi^\inversefunction(L))\) \ifandonlyif
    \(x = \varphi(y)\) for some \(y\in \varphi^\inversefunction(L)\).
    Now, \(\varphi^\inversefunction(L) = \{y \in G \suchthat \varphi(y) \in L\}\).
    We already noted that \(\varphi(\varphi^\inversefunction(L)) \subgroupleq L\).
    On the other hand, \(\varphi(\varphi^\inversefunction(L)) \subgroupleq \text{Im}\{\varphi\}\).
    Thus, \(\varphi(\varphi^\inversefunction(L)) \subgroupleq L \intersection \text{Im}\{\varphi\}\).
    If now \(x\in L \intersection \text{Im}\{\varphi\}\), we have that \(x = \varphi(y)\)
    for some \(y\in G\), but this is in \(\varphi^\inversefunction(L)\).
    Indeed, \(\varphi(y) \in L\). Thus, \(x\in \varphi(\varphi^\inversefunction(L))\).
\end{snippetproof}

\begin{snippetcorollary}{first-isomorphism-theorem-subgroup-correspondence}{}
    Let \(\varphi \colon G \fromto H\) be a \grouphomomorphism.
    Then, \(\varphi\) induces a \bijective[bijection] respecting the inclusion
    between the \subgroup[subgroups] of \(G\) containing \(\grpker_\varphi\)
    and the \subgroup[subgroups] of \(H\) in \(\text{Im}\{\varphi\}\).
\end{snippetcorollary}

\begin{snippetproof}{first-isomorphism-theorem-subgroup-correspondence-proof}{first-isomorphism-theorem-subgroup-correspondence}{}
    Let \(K \subgroupleq G\) such that \(\grpker_\varphi \in K\).
    Then, \(\varphi(K)\) is a \subgroup of \(H\) in \(\text{Im}\{\varphi\}\).
    Let \(L \subgroupleq H\) contained in \(\text{Im}\{\varphi\}\).
    Then, \(\varphi^\inversefunction(L)\) is a \subgroup of \(G\)
    containing \(\grpker_\varphi\).
    We now show that the two constructed functions are inverses of eachother.
    In one direction the composition goes from \(K\) to \(\varphi(K)\)
    to \(\varphi^\inversefunction(\varphi(K)) = K\) since \(\grpker_\varphi \subgroupleq K\).
    On the other hand the composition does from \(L\)
    to \(\varphi^\inversefunction(L)\) to \(\varphi(\varphi^\inversefunction(L)) = L\)
    since \(L \subgroupleq \text{Im}\{\varphi\}\).
\end{snippetproof}

\begin{snippetproposition}{product-kernel-image-size}{Product of image and kernel}
    Let \(\varphi \colon G \fromto H\) be a \grouphomomorphism
    where \(G\) is finite. Then,
    \[
        |G| = |\text{Im}\{\varphi\}| \cdot |\grpker_\varphi|
    \]
\end{snippetproposition}

\plain{This is because in general, elements with the same image form a coset of the kernel.
Thus, we have as many images as many there are cosets, which reconnects to Lagrange's theorem.}

\end{document}