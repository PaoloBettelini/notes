\documentclass[preview]{standalone}

\usepackage{amsmath}
\usepackage{amssymb}
\usepackage{parskip}
\usepackage{fullpage}
\usepackage{hyperref}
\usepackage{bettelini}
\usepackage{stellar}
\usepackage{definitions}

\begin{document}

\id{integers-modular-arithmetic}
\genpage

\section{Modular arithmetic}

\subsection{Congruence}

\begin{snippetdefinition}{modulo-congruence-definition}{Modulo congruence}
    Let \(a,b,n\in\integers\).
    We say that \(a\) and \(b\) are \textit{congruent modulo} \(n\)
    if \(a-b\) is a multiple of \(n\).
    \[ a \equiv b \pmod{n} \]
\end{snippetdefinition}

\plain{Note that every two integers are congruent modulo 1.}

\subsection{Congruence relation}

\begin{snippetproposition}{modulo-congruence-is-equivalence}{Modulo congruence is equivalence relation}
    The congruence relation modulo \(n\) is an \equivrelation.
\end{snippetproposition}

\begin{snippetproof}{modulo-congruence-is-equivalence-proof}{modulo-congruence-is-equivalence}{Modulo congruence is equivalence relation}
    \begin{itemize}
        \item \textbf{Reflexive}: \(\forall a, a-a = 0\), which is always a multiple of \(n\).
        \item \textbf{Symmetric}: \(a \equiv b \pmod{n} \implies \exists k \suchthat a-b=kn \implies b-a=-kn\).
        Since \(-kn\) is a multiple of \(n\), then \(b \equiv a \pmod{n}\).
        \item \textbf{Transitive}: \(a \equiv b \pmod{n} \land b \equiv c \pmod{n}\) implies that both
        \(a-b\) and \(b - c\) are multiples of \(n\).
        \(\exists h, k \suchthat nh=a-b \land nk=b-c \implies a-b+b-c=nh+nk \implies a-c=n(h+k)\)
        which means that \(a-c\) is also a multiple of \(n\), so \(a \equiv c \pmod{n}\).
    \end{itemize}
\end{snippetproof}

\begin{snippetproposition}{modulo-congruence-remainder-condition}{}
    Let \(a,b \in \integers\). Then, \(a \equiv b \pmod{n}\) \ifandonlyif
    their remainders when divided by \(n\) are equal.
\end{snippetproposition}

\begin{snippetproof}{modulo-congruence-remainder-condition-proof}{modulo-congruence-remainder-condition}{}
    Let \(a=nq + r\) and \(b = nq' + r'\) with \(0 \leq r < n- 1\)
    and \(0 \leq r' < n - 1\).
    Now, \(a \equiv b \pmod{n}\) \ifandonlyif \(a-b\) is a multiple of \(n\).
    That is, \(nq + r - (nq' + r')\) is a multiple of \(n\),
    which is true only when \(r-r'\) is a multiple of \(n\).
    Without loss of generality, suppose \(r \geq r'\). We have that
    \(0\leq r-r' < n - 1\). Between \(0\) and \(n-1\), \(0\) is the only multiple of \(0\).
    This means that \(r'-r = 0\) and thus \(r = r'\), which satisfies the congruence.
\end{snippetproof}

\subsection{Equivalence of summation and multiplication}

\begin{snippetproposition}{congruent-class-sum-mult-equiv}{Equivalence of summation and multiplication}
    If \(a \equiv a' \pmod{n}\) and if \(b \equiv b' \pmod{n}\), then
    \(a+b \equiv a' + b' \pmod{n}\) and \(ab \equiv a'b' \pmod{n}\).
\end{snippetproposition}

\begin{snippetproof}{congruent-class-sum-mult-equiv-proof}{congruent-class-sum-mult-equiv}{Equivalence of summation and multiplication}
    We can prove this by noting that there exists integers \(h\) and \(k\) such that
    \(a=a'+hn\) and \(b=b'+kn\).
    Now \(a+b = a'+b'+n(h+k)\) and \(ab=a'b' + n(hb0+ka'+hkn)\), meaning
    \(a+b \equiv a' + b' \pmod{n}\) and \(ab \equiv a'b' \pmod{n}\).
\end{snippetproof}

\subsection{Congruence class}

\begin{snippetdefinition}{congruence-class-definition}{Equivalence class}
    The \equivclass of \(a \in \integers\) with respect to modulo \(n\)
    is said to be a \textit{congruence class}, denoted \({[a]}_n\).
    \[
        {[a]}_n = \{a+kn \suchthat k \in \integers\}
    \]
\end{snippetdefinition}

\begin{snippetproposition}{periodic-equivalence-class}{Periodic equivalence classes}
    \[
        {[a]}_n = {[a + kn]}_n,\quad k \in \integers
    \]
\end{snippetproposition}

\subsection{Quotient set}

\begin{snippetdefinition}{modulo-quotient-set-definition}{Modulo quotient set}
    The set of all congruence classes modulo \(n\) is denoted \(\naturalnumbers / n\).
\end{snippetdefinition}

\begin{snippetproposition}{modulo-quotient-set-size}{Size of quotient set}
    The quotient set \(\naturalnumbers / n\) has \(n\) elements:
    \[
        {[0]}_n,{[1]}_n,\cdots,{[n-1]}_n
    \]
\end{snippetproposition}
% TODO proof

\subsection{Operations with congruent classes}

\begin{snippetdefinition}{congruent-classes-addition}{Congruent classes addition}
    Let \({[a]}_n\) and \({[b]}_n\) be congruence classes.  
    \[{[a]}_n \cdot {[b]}_n \triangleq {[a \cdot b]}_n\]
\end{snippetdefinition}

\begin{snippetdefinition}{congruent-classes-multiplication}{Congruent classes multiplication}
    Let \({[a]}_n\) and \({[b]}_n\) be congruence classes.  
    \[{[a]}_n + {[b]}_n \triangleq {[a + b]}_n\]
\end{snippetdefinition}

\subsection{Properties of congruent classes}

\begin{snippetproposition}{congruent-classes-properties}{Congruent classes properties}    
    For all \({[a]}_n, {[b]}_n, {[c]}_n \in \naturalnumbers/n\).

    \begin{enumerate}
        \item \textbf{Associative addition}: \({[a]}_n + ({[b]}_n + {[c]}_n) = ({[a]}_n + {[b]}_n) + {[c]}_n\)
        \item \textbf{Associative multiplication}: \({[a]}_n ({[b]}_n {[c]}_n) = ({[a]}_n {[b]}_n) {[c]}_n\)
        \item \textbf{Commutative addition}: \({[a]}_n + {[b]}_n = {[b]}_n + {[a]}_n\)
        \item \textbf{Commutative multiplication}: \({[a]}_n {[b]}_n = {[b]}_n {[a]}_n\)
        \item \textbf{Neutral addition element}: \({[a]}_n + {[0]}_n = {[a]}_n\)
        \item \textbf{Neutral multiplication element}: \({[a]}_n {[1]}_n = {[a]}_n\)
        \item \textbf{Inverse addition element}: \((-{[a]}_n) + {[a]}_n = {[0]}_n\)
        \item \textbf{Distributive property}: \(({[a]}_n + {[b]}_n) {[c]}_n = {[a]}_n{[c]}_n + {[b]}_n{[c]}_n\)
        \item \textbf{Cancellation law}: \({[a]}_n + {[b]}_n = {[a]}_n + {[c]}_n \implies {[b]}_n = {[c]}_n\)
    \end{enumerate}
\end{snippetproposition}

% TODO proofs, especially cancellation law

% TODO proof of [a][0] = [0] ?

\subsection{Invertible congruent classes}

\begin{snippetdefinition}{invertible-congruent-class-definition}{Invertible congruent class}
    A congruent class \({[a]}_n\) is \textit{invertible}
    if there exists an \({[b]}_n\) such that \({[a]}_n{[b]}_n={[1]}_n\).
    The inverse of \({[a]}_n\) is denoted \({[a]}_n^{-1}\).
\end{snippetdefinition}

\subsection{Properties of inverses}

\begin{snippetproposition}{inverse-congruent-class-properties}{Properties of inverse congruent classes}
    \begin{enumerate}
        \item If \({[a]}_n\) is invertible, then \({[a]}_n^{-1}\) is unique.
        \item If \({[a]}_n\) is invertible, then \({[a]}_n^{-1}\) is invertible and \(({[a]}_n^{-1})^{-1}={[a]}_n\).
        \item If \({[a]}_n\) and \({[b]}_n\) are invertible, then \({[a]}_n{[b]}_n\) is invertible and
        \({({[a]}_n{[b]}_n)}^{-1} = {[a]}_n^{-1}{[b]}_n^{-1}\).
    \end{enumerate}
\end{snippetproposition}

\end{document}
