\documentclass[preview]{standalone}

\usepackage{amsmath}
\usepackage{amssymb}
\usepackage{parskip}
\usepackage{fullpage}
\usepackage{hyperref}
\usepackage{bettelini}
\usepackage{stellar}
\usepackage{definitions}

\begin{document}

\id{fourieranalysis-complex-plot}
\genpage

\section{Complex plotting}

\begin{snippet}{fourier-analysis-complex-plotting-expl}
    To start, we will plot our signal in the complex plane.
    We will plot it while making it rotate around the origin \(0+0i\).
    Recall that Euler's Formula tells us that the function \(e^{it}\) is a rotation
    around the origin, also called the unit circle. If we multiply our signal by this circle,
    it will follow its path, achieving a polar plot-like graph: \(f(t)e^{it}\).
    Now, the rotational function makes a full cycle every \(2\pi\), we can plot our
    signal at a different speed (different frequencies). To do so, we will multiply
    the time of the rotational function by \(2\pi \xi\) where \(\xi\) is the frequency.
    We add the \(2\pi\) term so that if your frequency is \(1\), we will have a rotation each second
    rather then every \(2\pi\) seconds \((\sim 6.28\,s)\).
    Furthermore, we want the unit circle to rotate clockwise instead of counter-clockwise.
    We will just add the negative sign to the time \(t\) argument of the rotational function.
    Our final function for now is \(f(t)e^{-2\pi it\xi}\). You can see the animation next
    to this paragraph, you can vary the frequency or even draw your own signal.
\end{snippet}

\includesnpt{fourier-lib}
\includesnpt{fourier-complex-plot}

\end{document}
