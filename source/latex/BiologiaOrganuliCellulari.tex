\documentclass[preview]{standalone}

\usepackage{amsmath}
\usepackage{amssymb}
\usepackage{stellar}
\usepackage{definitions}
\usepackage{bettelini}

\begin{document}

\id{biologia-organuli-cellulari}
\genpage

\section{Gli organuli cellulari}

\begin{snippetdefinition}{membrana-plasmatica-definition}{Membrana plasmatica}
    La \textit{membrana plasmatica} è una parete sottile, a doppio strato,
    che regola il traffico delle molecole in entrata e in uscita.
\end{snippetdefinition}

\begin{snippet}{membrana-expl1}
    Tutte le cellule sono separate dall'ambiente esterno da
    una membrana plasmatica.
    La membrana plasmatica possiede canali, formati da proteine, che regolano il passaggio di molecole
    specifiche in entrata e in uscita dalla cellula.
    Si crea così una membrana semipermeabile (alcune sostanze possono traversarla) tra l'interno
    della cellula, il citoplasma, e l'ambiente esterno.
\end{snippet}

\begin{snippetdefinition}{involucro-nucleare-definition}{Involucro nucleare}
    L'\textit{involucro nucleare} è l'involucro contenente il nucleo
    di una cellula. Come la membrana plasmatica, l'involucro nucleare serve a
    regolare il traffico delle molecole, ossia il flusso delle molecole tra il nucleo
    e il resto della cellula.
\end{snippetdefinition}

\begin{snippetdefinition}{sistema-membrane-endocellulari-definition}{Sistema di membrane endocellulari}
    Il \textit{sistema di membrane endocellulari} è una rete di organuli che
    produce e distribuisce i prodotti cellulari.
\end{snippetdefinition}

\begin{snippet}{endocellulari-expl1}
    Molti dei materiali che escono dal nucleo vengono trasportati attraverso il sistema di
    membrane endocellulari.
    Questi organuli comprendono il reticolo endoplasmatico, l'apparato di Golgi e i lisosomi.
\end{snippet}

\begin{snippetdefinition}{ribosomi-definition}{Ribosomi}
    I \textit{ribosomi} sono delle strutture (non organelli) responsabili della produzione delle proteine.
\end{snippetdefinition}

\begin{snippetdefinition}{reticolo-endoplasmatico-definition}{Reticolo endoplasmatico}
    Il \textit{reticolo endoplasmatico} è il principale centro di produzione
    della cellula.
    In una cellula ne esistono di due tipi:
    \begin{itemize}
        \item \textit{ruvido:} provvisto di ribosomi sulla superficie esterna;
        \item \textit{liscio.}
    \end{itemize}
\end{snippetdefinition}

\begin{snippet}{endoplasmatico-expl1}
    La funzione principale del reticolo endoplasmatico liscio è
    la costruzione dei lipidi.\\
    Se il reticolo endoplasmatico è la \quotes{fabbrica} della cellula, i ribosomi sono i singoli macchinari che ne
    realizzano i prodotti.
\end{snippet}

\begin{snippetdefinition}{apparato-di-golgi-definition}{Apparato di Golgi}
    L'apparato di Golgi è uno spazio dove le proteine
    vengono modificate, etichettate e inviate alle aree di destinazione della cellula.
\end{snippetdefinition}

\plain{Spesso, i materiali prodotti nel reticolo endoplasmatico passano poi all'apparato di Golgi.
Qui la produzione della cellula viene quindi finalizzata.}

\begin{snippetdefinition}{centrosoma-definition}{Centrosoma}
    Il \textit{centrosoma} è una struttura coinvolta nel processo di divisione cellulare.
\end{snippetdefinition}

\begin{snippetdefinition}{lisosoma-definition}{Lisosoma}
    Il \textit{lisosoma} è un organulo presente solo nelle cellule animali costituito da enzimi digestivi
    racchiusi da una membrana.
    La sua funzione principale è la demolizione delle sostanze,
    come i prodotti di rifiuto o gli organuli vecchi.
\end{snippetdefinition}

\begin{snippetdefinition}{perossisoma-definition}{Perossisoma}
    Il \textit{perossisoma} è organulo con diverse funzioni metaboliche specifiche, tra cui la detossicazione di
    sostanze nocive.
\end{snippetdefinition}

\begin{snippetdefinition}{mitocondri-definition}{Mitocondri}
    I \textit{mitocondri} sono i principali organuli cellulari per la produzione di energia.
    Sono il centro primario in cui si svolge la respirazione cellulare, in cui l'energia degli
    zuccheri e delle altre
    molecole nutritive viene trasformata in ATP.
\end{snippetdefinition}

\plain{Le cellule che consumano molta energia, come
quelle del cervello o dei muscoli, hanno molti mitocondri, che producono molta ATP.}

\begin{snippetdefinition}{citoscheletro-definition}{Citoscheletro}
    Il \textit{citoscheletro} è la struttura di sostegno della cellula.
    Esso fornisce un punto di attacco per gli altri organuli così da non 
    lasciarli semplicemente fluttuare per la cellula.
\end{snippetdefinition}

\plain{Alcune proteine sfruttano il citoscheletro per trasportare materiali.}

\plain{Tutti questi organuli, fatta eccezione per i lisosomi, sono presenti
sia nelle cellule vegetali sia in quelle animali.}

\begin{snippetdefinition}{parete-cellulare-definition}{Parete cellulare}
    La \textit{parete cellulare} è uno strato rigido protettivo situato all'esterno della membrana
    plasmatica delle piante.
\end{snippetdefinition}

\plain{Nelle piante, la parete cellulare è formata dalla cellulosa.}

\begin{snippetdefinition}{vacuolo-centrale-definition}{Vacuolo centrale}
    Il \textit{vacuolo centrale} ha una struttura a sacco delimitata da una membrana che serve come
    magazzino e contribuisce a regolare la quantità di acqua nella cellula nelle piante.
\end{snippetdefinition}

\plain{All'interno delle cellule vegetali, il vacuolo centrale è spesso l'organulo più grande.}

\begin{snippetdefinition}{cloroplasto-definition}{Cloroplasto}
    Il \textit{cloroplasto} è un organulo per la produzione di
    energia nelle cellule vegetali.
\end{snippetdefinition}

\begin{snippet}{mitocondri-expl1}
    I mitocondri sono specializzati nella
    produzione di ATP a partire dagli zuccheri. I cloroplasti, invece, sono specializzati
    nell'utilizzare l'energia della luce solare per produrre gli zuccheri.
    I cloroplasti sfruttano l'energia luminosa per produrre zuccheri, e i mitocondri
    utilizzano gli zuccheri per produrre ATP. Le cellule vegetali hanno sia i cloroplasti sia i
    mitocondri: i cloroplasti per produrre gli zuccheri a partire dall'energia della luce solare, e i
    mitocondri che producono ATP a partire da questi zuccheri. Le cellule animali, tuttavia,
    hanno soltanto mitocondri, il che significa che non possono prodursi da sole gli zuccheri. Gli
    animali devono quindi mangiare piante o altri animali per ottenere gli zuccheri, necessari ai
    mitocondri per produrre l'ATP.
\end{snippet}

% TODO feedback negativo2

% TODO Pompa sodio-potassio porta fuori sodio e dento potassio (crea un gradiente), esempio di trasporto attivo primario

\end{document}