\documentclass[preview]{standalone}

\usepackage{amsmath}
\usepackage{amssymb}
\usepackage{stellar}

\hypersetup{
    colorlinks=true,
    linkcolor=black,
    urlcolor=blue,
    pdftitle={Stellar},
    pdfpagemode=FullScreen,
}

\begin{document}

\title{Stellar}
\id{italiano-duecento-trecento}
\genpage

\section{Passaggio fra il Duecento e Trecento}

\begin{snippet}{dante-boccaccio-transizione}
    Dante (1265-1321) e Boccaccio (1313 - 1375)
    si distinguono per le caratteristiche del cambio di secolo.
    
    \begin{itemize}
        \item Fino ai primi anni del secolo, si assiste ad un aumento della popolazione, favorito dall'accresciuto benessere e dalle migliorate condizioni sanitarie.
        \item I comuni si espandono sempre di più, grazie alla migliori condizioni economiche e all'immigrazione dal contado. La città diviene il vero e proprio centro della vita: civile, sociale, economica, culturale.
        \item L'esempio di Firenze e l'ascesa della borghesia mercantile.
    \end{itemize}
    
    Firenze aveva i fiorini ed era economicamente forte, comandata da poche famiglie come i Medici.
    Molti poeti e scrittori sono appunto provenienti da paesi del genere.
    In questi anni cominciano a venire gettate le basi per il capitalismo.
    \\\\
    A differenza di Dante, Boccaccio e Petrarca si distaccano dai loro comuni di appartenenza,
    mentre Dante centralizza Firenze nei suoi temi.
    \\\\
    La società comincia lentamente a diventare parzialmente laica.
    Il mondo passa dalla trascendenza (visione verticale, tutto è subordinato a Dio)
    ad un mondo di immanenza (ciò che è sulla terra).
    La teologia nelle università rimane la materia primaria e fondamentale.
    La medicina è l'unica materia autonoma (come facoltà universitaria), ma non è possibile sezionare i cadaveri (corpo sacro),
    rimanendo comunque vincolata dalla religione.
\end{snippet}

\section{Crisi economica e demografica}

\begin{snippet}{23528066-74d8-4b37-81f6-c89e79df3fa2}
    La produzione agricola entra progressivamente in crisi man mano che ci si inoltra nel Trecento.
    È una decisa inversione di tendenza rispetto al secolo precedente.\\
    In questa situazione già deteriorata, la diffusione in tutta Europa dellla peste nera (1347-1350) provocò
    un tracollo economico e una vera e propria crisi demografia (la popolazione si riduce di almeno un terzo).
\end{snippet}

\end{document}