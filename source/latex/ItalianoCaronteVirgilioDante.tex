\documentclass[preview]{standalone}

\usepackage{amsmath}
\usepackage{amssymb}
\usepackage{stellar}

\hypersetup{
    colorlinks=true,
    linkcolor=black,
    urlcolor=blue,
    pdftitle={Stellar},
    pdfpagemode=FullScreen,
}

\begin{document}

\title{Stellar}
\id{italiano-caronte-virgilio-dante}
\genpage

\section{Caronte Virgiliano e Dantesco}

\begin{snippetcharacter}{caronte}{Caronte}
    \textit{Caronte} è un personaggio che ha il ruolo di traghettare le anime
    oltre il primo fiume dell'inferno.
\end{snippetcharacter}

\begin{snippet}{caronte-virgiliano-dantesco-expl1}
    Vi sono delle differenza fra la descrizione di Caronte nell'\textit{Eneide}
    e la \textit{Commedia}.
    
    In ambo i casi, Caronte è un vecchio traghettatore che viene descritto come un personaggio con
    delle caratteristiche comuni.
    Entrambi sono caratterizzati da dei capelli bianchi (rr. 299-300, r. 83) e posseggono
    degli occhi di fiamme. Dante indica tuttavia quest'ultima caratteristica 2 volte.
    Il ruolo del personaggio è il medesimo; Caronte ha ruolo di ostacolo, ferma chiunque voglia passare
    (rr. 338-339, rr. 88-89).
    
    Nella \textit{Commedia}, la barca e Caronte stesso è descritto con più precisione
    rispetto all'\textit{Eneide}.
\end{snippet}

\begin{snippetnote}{dante-descrizione-expl}{Descrizione di Dante}
    Dante usa più verbi che aggettivi nelle varie descrizioni.
\end{snippetnote}

\begin{snippet}{caronte-virgiliano-dantesco-expl2}
    Dante descrive Caronte principalmente per quello che fa piuttosto che come è fatto.
    
    Il Caronte dantesco è più aggressivo (urla), sintatticamente le sue frasi sono
    più brevi e nette e viene descritto come indemoniato. Infatti, anche la sua entrata in scena
    è improvvisa e violenta. Questa caratteristica non è presente nel Caronte virgiliano,
    il quale è più pacato, e descritto come un Dio ma meno importante.
    
    La differenza più importante è che il Caronte dell'\textit{Eneide} fa parte di un mondo pagano.
    Non vi è ancora la concezione di salvezza dell'anima, la quale deriva dal cristianesimo.
\end{snippet}

\end{document}