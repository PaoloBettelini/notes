\documentclass[preview]{standalone}

\usepackage{amsmath}
\usepackage{amssymb}
\usepackage{stellar}
\usepackage{bettelini}

\hypersetup{
    colorlinks=true,
    linkcolor=black,
    urlcolor=blue,
    pdftitle={Biologia},
    pdfpagemode=FullScreen,
}

\begin{document}

\title{Biologia}
\id{biologia-carboidrati}
\genpage

\begin{snippetdefinition}{carboidrato-definizione}{Carboidrato}
    I \textit{carboidrati} sono dei tipi di biomolecole composti da carbonio, idrogeno e ossigeno
    \((CH_2O)_n\).
\end{snippetdefinition}

\plain{
    I monomeri di carboidrati si chiamano monosaccaridi.
    I polimeri di carboidrati si chiamano polisaccaridi (disaccaridi, trisaccaridi).
}

\begin{snippetdefinition}{maltosio-definizione}{Maltosio}
    Il \textit{maltosio} è composto da due molecole di glucosio (\(C_{12}H_{22}O_{11}\)).
\end{snippetdefinition}

\begin{snippet}{89123b29-4843-415e-8cbe-65c4a337c41c}
    Per unire 2 molecole di glucosio
è necessario perderne una di \(H_2O\). Per cui il maltosio è dato da \(C_{12}H_{22}O_{11}\).
\end{snippet}

\begin{snippetdefinition}{saccarosio-definizione}{Saccarosio}
    Il \textit{saccarosio} è composto da un glucosio e un fruttosio (\(C_{12}H_{22}O_{11}\)).
\end{snippetdefinition}

\begin{snippetdefinition}{lattosio-definizione}{Lattosio}
    Il \textit{lattosio} è composto da un glucosio e un galattosio (\(C_{12}H_{22}O_{11}\)).
\end{snippetdefinition}

\plain{I monosaccaridi sono glucosio, fruttosio, galattosio (isomeri).}

\begin{snippetdefinition}{amico-definizione}{Amido}
    L'\textit{amido} è un polisaccaride che viene prodotto dalle piante.
    Esso è composto da una catena di glucosi arrotolati ad elica.
\end{snippetdefinition}

\begin{snippetdefinition}{amilasi-definizione}{Amilasi}
    L'\textit{amilasi} è l'enzima che rompe l'amido.
    Esso fa parte della famiglia degli \textit{idrolasi}, ossia tutti gli enzimi che
    eseguono l'idrolisi.
\end{snippetdefinition}

\begin{snippetdefinition}{glicogeno-definizione}{Glicogeno}
    Il \textit{glicogeno} è un polisaccaride che viene prodotto dagli animali.
    Esso è composto da diverse diramazioni di catene di glucosio.
\end{snippetdefinition}

\begin{snippet}{0e9e8b4e-9f36-4f5d-8b31-3124454aba22}
    Amido e glicogeno occupano meno spazio dei monomeri da soli, per cui sono ottimali per immagazzinare
il glucosio.

Gli esseri umani immagazzinano il glucosio in eccesso nei muscoli e nel fegato, dove ci sono degli enzimi
che sono in grado di creare questi polimeri di glucosio.
\end{snippet}

\begin{snippetdefinition}{cellulosa-definizione}{Cellulosa}
    La \textit{cellulosa} è un polisaccaride di glucosio prodotto dalle piante.
    Esso è composto un insieme di fibre lineari.
\end{snippetdefinition}

\plain{La cellulosa serve per dare rigidità al tessuto delle piante.}

\plain{I polisaccaridi sono amido, glicogeno e cellulosa.}

% TODO: Sintesi dello sciroppo di mais

\end{document}