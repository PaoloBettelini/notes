\documentclass[preview]{standalone}

\usepackage{amsmath}
\usepackage{amssymb}
\usepackage{stellar}
\usepackage{bettelini}
\usepackage{chemfig}

\hypersetup{
    colorlinks=true,
    linkcolor=black,
    urlcolor=blue,
    pdftitle={Biologia},
    pdfpagemode=FullScreen,
}

\begin{document}

\title{Biologia}
\id{biologia-carboidrati}
\genpage

\section{Carboidrati}

\begin{snippetdefinition}{carboidrato-definition}{Carboidrato}
    I \textit{carboidrati} sono dei tipi di biomolecole composti da carbonio, idrogeno e ossigeno
    con formula \((CH_2O)_n\).
\end{snippetdefinition}

\plain{I monomeri di carboidrati si chiamano monosaccaridi. \\ I polimeri di carboidrati si chiamano polisaccaridi (disaccaridi, trisaccaridi).}

\subsection{Monosaccaridi e disaccaridi}

\begin{snippet}{lista-monosaccaridi-chemfig}
    I monosaccaridi sono:\\
    \phantom{}

    \begin{minipage}{0.2\textwidth}
        \centering
        \chemfig[cram width=2pt]{
            HO-[2,0.5,2]?<[7,0.7](-[2,0.5]OH)-[,,,,line width=2pt](-[6,0.5]OH)>[1,0.7](-[6,0.5]OH)
            -[3,0.7]O-[4]?(-[2,0.3]-[3,0.5]CH_2OH)
        } \phantom{}\\
        \hspace*{0.5cm} Glucosio
    \end{minipage}
    \hspace{3cm}
    \begin{minipage}{0.2\textwidth}
        \centering
        \chemfig[cram width=2pt]{
            HO-[2,-0.5,2]?<[7,0.7](-[2,0.5]OH)-[,,,,line width=2pt](-[6,0.5]OH)>[1,0.7](-[6,0.5]OH)
            -[3,0.7]O-[4]?(-[2,0.3]-[3,0.5]CH_2OH)
            } \phantom{}\\
            \hspace*{0.5cm} Galattosio
    \end{minipage}
    \hspace{3cm}
    \begin{minipage}{0.2\textwidth}
        \centering
        \chemfig[cram width=2pt]{
            (?[b](-[2]CH_2OH)
            <[7,0.7](-[6,0.5]OH)-[,,,,line width=2pt](-[2,0.5,,2]HO)>[1,0.7](-[6,0.5]CH_2OH)
            -[:150,1.15]O?[b])
            } \phantom{}\\
            \hspace*{-0.5cm} Fruttosio
    \end{minipage}
    \vspace{1cm}
\end{snippet}

\begin{snippetdefinition}{maltosio-definition}{Maltosio}
    Il \textit{maltosio} è un disaccaride composto da due molecole di glucosio (\(C_{12}H_{22}O_{11}\)).
\end{snippetdefinition}

\begin{snippet}{89123b29-4843-415e-8cbe-65c4a337c41c}
    Per unire 2 molecole di glucosio è necessario perderne una di \(H_2O\).
    Per cui il maltosio è dato da \(C_{12}H_{22}O_{11}\).
\end{snippet}

\begin{snippetdefinition}{saccarosio-definition}{Saccarosio}
    Il \textit{saccarosio} è un disaccaride composto da un glucosio e un fruttosio (\(C_{12}H_{22}O_{11}\)).
\end{snippetdefinition}

\begin{snippetdefinition}{lattosio-definition}{Lattosio}
    Il \textit{lattosio} è un disaccaride composto da un glucosio e un galattosio (\(C_{12}H_{22}O_{11}\)).
\end{snippetdefinition}

\subsection{Polisaccaridi}

\begin{snippetdefinition}{amico-definition}{Amido}
    L'\textit{amido} è un polisaccaride che viene prodotto dalle piante.
    Esso è composto da una catena di glucosi arrotolati a elica.
\end{snippetdefinition}

\begin{snippetdefinition}{amilasi-definition}{Amilasi}
    L'\textit{amilasi} è l'enzima che rompe l'amido.
    Esso fa parte della famiglia degli \textit{idrolasi}, ossia tutti gli enzimi che
    eseguono l'idrolisi.
\end{snippetdefinition}

\begin{snippetdefinition}{glicogeno-definition}{Glicogeno}
    Il \textit{glicogeno} è un polisaccaride che viene prodotto dagli animali.
    Esso è composto da diverse diramazioni di catene di glucosio.
\end{snippetdefinition}

\begin{snippet}{0e9e8b4e-9f36-4f5d-8b31-3124454aba22}
    Amido e glicogeno occupano meno spazio dei monomeri da soli, per cui sono ottimali per immagazzinare
    il glucosio.
    Gli esseri umani immagazzinano il glucosio in eccesso nei muscoli e nel fegato, dove ci sono degli enzimi
    che sono in grado di creare questi polimeri di glucosio.
\end{snippet}

\begin{snippetdefinition}{cellulosa-definition}{Cellulosa}
    La \textit{cellulosa} è un polisaccaride di glucosio prodotto dalle piante.
    Esso è composto un insieme di fibre lineari.
\end{snippetdefinition}

\plain{La cellulosa serve per dare rigidità al tessuto delle piante.}

% TODO: Sintesi dello sciroppo di mais

\end{document}