\documentclass[preview]{standalone}

\usepackage{amsmath}
\usepackage{amssymb}
\usepackage{stellar}
\usepackage{definitions}

\begin{document}

\id{integers-modular-exercises}
\genpage

\section{Exercises}

\begin{snippetexercise}{invertible-classes-ex-1}{}
    Consider the \congruenceclass \({[12]}_{1731}\).
    Determine whether it is \invertiblecongclass[invertible] and its inverse if any.
\end{snippetexercise}

\begin{snippetsolution}{invertible-classes-ex-1-sol}{}
    We compute \(\gcd(1731, 12)\), which is
    \(3\), so their are not \coprime and thus the class is not \invertiblecongclass[invertible].
\end{snippetsolution}

\begin{snippetexercise}{invertible-classes-ex-2}{}
    Consider the \congruenceclass \({[12]}_{1721}\).
    Determine whether it is \invertiblecongclass[invertible] and its inverse if any.
\end{snippetexercise}

\begin{snippetsolution}{invertible-classes-ex-2-sol}{}
    We compute \(\gcd(1721, 12)\), which is
    \(1\), so their are \coprime and thus the class is \invertiblecongclass[invertible].
    To compute the \invertiblecongclass[inverse], we find a Bezout's identity
    \begin{align*}
        1 &= 5 - 2 \cdot 2 = 5-(12 - 5\cdot 2)\cdot 2 \\
        &= 12 \cdot (-2) + 5 \cdot 5 = 12(-2) + (1721 - 12 \cdot 143) \cdot 5 \\
        &= 1721 \cdot 5 + 12 \cdot (-717)
    \end{align*}
    Moving on to the rest of module \(1721\), we find
    \({[1]}_{1721} = {[1721]}_{1721} \cdot {[5]}_{1721} + {[12]}_{1721} \cdot {[-717]}_{1721}\)
    from which \({[1]}_{1721} = {[12]}_{1721} \cdot {[-717]}_{1721}\).
    Thus, \({[12]}_{1721}^{-1} = {[-747]}_{1721} = {[1004]}_{1721}\).
\end{snippetsolution}

\begin{snippetexercise}{algebra-misc-ex3}{}
    Prove that
    \[
        \nexists n \in \integers \suchthat 3n^2 - 1 = k^2
    \]
    for \(k\in\integers\).
\end{snippetexercise}

\begin{snippetsolution}{algebra-misc-ex3-sol}{}
    Assume that there exist a \(k\in\integers\) such that
    \[
        k^2 = 3n^2 - 1
    \]
    Note that \(3n^2 - 1 \equiv 2 \pmod{3}\).
    We have different cases for \(k\):
    \begin{itemize}
        \item \(k \equiv 0 \pmod{3} \implies k^2 \equiv 0 \pmod{3}\);
        \item \(k \equiv 1 \pmod{3} \implies k^2 \equiv 1 \pmod{3}\);
        \item \(k \equiv 2 \pmod{3} \implies k^2 \equiv 1 \pmod{3}\).
    \end{itemize}
    This means that the other term cannot be equivalent to \(2\) modulo \(3\).
\end{snippetsolution}

\begin{snippetexercise}{algebra-misc-ex4}{}
    Find the remainder of \(227^{228^{229}}\) divided by \(117\).
\end{snippetexercise}

\begin{snippetsolution}{algebra-misc-ex4-sol}{}
    We note that \(227 \equiv -1 \pmod{114}\).
    This means that the value is congruent to \(1\) if \(228^{229}\) is even
    and \(-1\) if it is of. But \(228^{229}\) is clearly even, so the result,
    so
    \[
        227^{228^{229}} \equiv 1 \pmod{117}
    \]
    which is the last digit.
\end{snippetsolution}

\begin{snippetexercise}{algebra-misc-ex5}{}
    Prove that for \(n\in\naturalnumbers\),
    \[
        11^{n+2} + 12^{2n+1}
    \]
    is a multiple of \(133\).
\end{snippetexercise}

\begin{snippetsolution}{algebra-misc-ex5-sol}{}
    We first note that \(12^2 = 144 \equiv 11 \pmod{133}\).
    \begin{align*}
        11^{n+2} + 12^{2n+1} &= 11^{n+2} + 12 \cdot 12^{2n} \\
        &= 11^{n+2} + 12 \cdot 11^n \\
        &= 11^n (11^2 + 12) \\
        &= 133 \cdot 11^n
    \end{align*}
\end{snippetsolution}

\end{document}