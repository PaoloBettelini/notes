\documentclass[preview]{standalone}

\usepackage{amsmath}
\usepackage{amssymb}
\usepackage{stellar}
\usepackage{bettelini}

\hypersetup{
    colorlinks=true,
    linkcolor=black,
    urlcolor=blue,
    pdftitle={Stellar},
    pdfpagemode=FullScreen,
}

\begin{document}

\title{Geografia economica}
\id{geoeconomica-migrazioni}
\genpage

\section{Migrazioni}

\begin{snippetdefinition}{migrante-definition}{Migrante}
    Non esiste una definizione formale di migrante internazionale. Generalmente gli esperti tendono a considerare un migrante una persona che
    modifica il proprio Stato di residenza indipendentemente dalla ragione
    o dallo statuto legale.
\end{snippetdefinition}

\begin{snippetdefinition}{rifugiato-definition}{Rifugiato}
    Persona che si trova fuori dal proprio paese d'origine per paura di persecuzioni
    (nazionalità, razza, religione, appartenenza ad un gruppo sociale o politico),
    a causa di un conflitto, di violenza generalizzata o altre
    circostanze con impatto sull'ordine pubblico e che, pertanto, richiede
    protezione internazionale.
\end{snippetdefinition}

\begin{snippetdefinition}{richiedente-asilo-definition}{Richiedente l'asilo}
    Coloro che hanno lasciato il loro paese d'origine e hanno inoltrato una
    richiesta di asilo in un paese terzo, ma sono ancora in attesa di una
    decisione da parte delle autorità competenti riguardo al riconoscimento
    del loro status di rifugiati.
\end{snippetdefinition}

\begin{snippetdefinition}{profugo-definition}{Profugo}
    Si tratta di una parola usata in modo generico che deriva dal verbo latino
    profugere, «cercare scampo». Talvolta si intende profugo come
    colui che per diverse ragioni (guerra, povertà, calamità naturali, ecc.) ha
    lasciato il proprio Paese ma non è nelle condizioni di chiedere la protezione internazionale.
\end{snippetdefinition}

\newcommand{\greenbox}{
    \fcolorbox{black}{green}{\rule{0pt}{5pt}\rule{5pt}{0pt}}
}

\newcommand{\redbox}{
    \fcolorbox{black}{red}{\rule{0pt}{5pt}\rule{5pt}{0pt}}
}

\begin{snippet}{conseguenze-migrazioni}
    Le conseguenze sociali e culturali delle migrazioni
    possono essere sia positive che negative, sia per quanto
    riguarda i paesi di partenza che, sia per i paesi di arrivo.
    
    I paesi di arrivo hanno
    \begin{itemize}
        \item \greenbox più forza lavoro e diversità;
        \item \redbox più disoccupazione, difficoltà di integrazione,
        segregazione, discriminazione.
    \end{itemize}
    
    I paesi di partenza hanno
    \begin{itemize}
        \item \greenbox rimesse finanziarie;
        \item \redbox perdita di tradizioni,
        cambiamento della struttura demografica (meno nati, più invecchiamenti),
        fuga di cervelli.
    \end{itemize}

    \textbf{Ragioni delle migrazioni individuali:}
    \begin{itemize}
        \item \textbf{Ragioni forzate:} paura, persecuzione, guerra, catastrofi naturali;
        \item \textbf{Ragioni volontarie:} lavoro, stile di vita, cultura, famiglia e affetti;
        \item \textbf{Ragioni economiche e fiscali:} povertà.
    \end{itemize}
\end{snippet}

\begin{snippetdefinition}{flusso-definition}{Flusso}
    I \textit{flussi} indicano il numero di persone che oltrepassano il confine di uno
    Stato, indipendentemente se verso l'interno o verso l'esterno.
\end{snippetdefinition}

\plain{È come se ci fosse un contatore posto al confine che conteggia ogni persona che passa.}

\begin{snippetdefinition}{stock-definition}{Stock}
    Gli \textit{stocks} indicano il numero di tutte le persone che si trovano all'interno di uno Stato in un
    determinato momento.
\end{snippetdefinition}

\plain{È come se venisse scattata una fotografia dello Stato e si contassero
tutte le persone presenti nello scatto.}

\end{document}