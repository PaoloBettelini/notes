\documentclass[preview]{standalone}

\usepackage{amsmath}
\usepackage{amssymb}
\usepackage{stellar}
\usepackage{bettelini}

\hypersetup{
    colorlinks=true,
    linkcolor=black,
    urlcolor=blue,
    pdftitle={Stellar},
    pdfpagemode=FullScreen,
}

\begin{document}

\title{Stellar}
\id{italiano-canzoniere-rvf-310}
\genpage

\section{Rvf 310: Zephiro torna, e 'l bel tempo rimena}

\begin{snippet}{canzoniere-rvf-310}
    La metrica è un sonetto con schema di rima ABAB ABAB CDC DCD.
    \\\\
    \StellarPoetry{1}{
        Zephiro torna, e 'l bel tempo rimena, \\
        e i fiori et l'erbe, sua dolce famiglia, \\
        et garrir Progne et pianger Philomena, \\
        et primavera candida et vermiglia.
    }{XXX}
    \StellarPoetry{2}{
        Ridono i prati, e 'l ciel si rasserena; \\
        Giove s'allegra di mirar sua figlia; \\
        l'aria et l'acqua et la terra è d'amor piena; \\
        ogni animal d'amar si riconsiglia.
    }{XXX}
    \StellarPoetry{3}{
        Ma per me, lasso, tornano i piú gravi \\
        sospiri, che del cor profondo tragge \\
        quella ch'al ciel se ne portò le chiavi;
    }{XXX}
    \StellarPoetry{4}{
        et cantar augelletti, et fiorir piagge, \\
        e 'n belle donne honeste atti soavi \\
        sono un deserto, et fere aspre et selvagge.
    }{XXX}

    Zephiro è un vento caldo, tipico tiepido vento primaverile,
    che torna riportando il bel tempo, fiori ed erbe che sono la sua dolce famiglia.
    Inoltre, riporta anche il canto della rondine (Progne) e il pianto (canto malinconico)
    dell'usignolo (Philomena).
    \\\\
    Viene fatto un riferimento ad un mito dove Progne e Philomena erano sorelle.
    Progne è sposata con Tereo. Tereo, violenta la cognata, sorella della moglie Philomena,
    e le mozza la lingua per evitare che parli. Progne ne viene comunque a sapere,
    e per punirlo uccide il figlio che ha avuto con lui, dandoglielo in pasto.
    Le due donne chiedono agliuto agli Dei, ed essi per salvarle dalla violenza dell'uomo
    le trasformano in rondine ed usignolo.
    \\\\
    Anche la seconda strofa è descrittiva: i prati tornano ad essere rigogliosi, il cielo
    si rasserena.
    Nel medieovo si pensava che tutto il creato fosse composto da terra
    acqua aria e fuoco in forma di amore.
    Vi è una ciclicità in questa situazione.
    TODO
    \\\\
    Ma per l'autore, tornano invece i ricordi di Laura (morta).
    Le chiavi del cuore che l'amata porta con sè, sono dall'altra parte, e di conseguenza
    non l'amore è definitivamente chiuso e non può tornare.
    Il canto degli uccelli e le pianure fiorite appaiono all'autore con un deserto.
    Mentre vede la bellezza delle donne sono aspre e selvaggie.
\end{snippet}

\end{document}