\documentclass[preview]{standalone}

\usepackage{amsmath}
\usepackage{amssymb}
\usepackage{stellar}
\usepackage{bettelini}
\usepackage{wrapfig}

\hypersetup{
    colorlinks=true,
    linkcolor=black,
    urlcolor=blue,
    pdftitle={Biologia},
    pdfpagemode=FullScreen,
}

\begin{document}

\title{Biologia}
\id{biologia-teorie-evolutive}
\genpage

\begin{snippet}{fe987c05-9935-45a9-b7da-cef4ce67b824}
    L'idea che la natura si trasformi ha origine antiche: \\
    \begin{center}\begin{tabular}{|l|l|l|}
        \hline Esponente & Periodo & Principali Idee \\
        \hline Anassimandro & VI secolo a.C. & \begin{tabular}{l} 
        Origine degli esseri umani e degli animali \\
        dall'acqua e dalla terra riscaldata.
        \end{tabular} \\
        \hline \begin{tabular}{l} 
        Democrito ed \\
        Epicuro
        \end{tabular} & V-IV secolo a.C. & \begin{tabular}{l} 
        Teoria atomistica, suggerendo una \\
        concezione di trasformazione e cambiamento \\
        nel mondo naturale.
        \end{tabular} \\
        \hline Lucrezio & I secolo a.C. & \begin{tabular}{l} 
        Concetto pre-evoluzionistico, sosteneva che \\
        le specie si siano mutate nel corso del tempo.
        \end{tabular} \\
        \hline
    \end{tabular}\end{center}
    \vspace{0.25cm}
\end{snippet}

\begin{snippetdefinition}{fissismo-definition}{Fissismo}
    Il \textit{fissismo} è la teoria biologica secondo la quale le specie vegetali ed animali sono destinate a rimanere sempre uguali a se stesse. È una teoria biologica insieme al catastrofismo e al creazionismo. 
\end{snippetdefinition}

\begin{snippet}{2196c772-9f6a-4ae2-8883-b47401b471a2}
    Teoria del fissismo di Aristotele, sostenuta dal cristianesimo: \\
    \begin{center}\begin{tabular}{|l|l|l|}
        \hline Esponente & Periodo & Principali Idee \\
        \hline Aristotele & IV secolo a.C. & \begin{tabular}{l} 
        Descrisse le specie animali come fisse e \\
        immutabili, influenzando il pensiero \\
        medievale e cristiano.
        \end{tabular} \\
        \hline Linneo & XVIII secolo & \begin{tabular}{l} 
        Sostenitore del fissismo, credeva che le \\
        specie fossero destinate a rimanere sempre \\
        uguali a sé stesse.
        \end{tabular} \\
        \hline
    \end{tabular}\end{center}
    \vspace{0.25cm}
\end{snippet}

\begin{snippet}{2398bbb4-d36a-4ce4-b196-1b3d15333dd7}
    Nel XVIII secolo c'era però chi sosteneva le teorie evolutive (vs. Linneo) \\
    \begin{center}\begin{tabular}{|l|l|l|}
        \hline Esponente & Periodo & Principali Idee \\
        \hline Buffon & XVIII secolo & \begin{tabular}{l} 
        Propose una visione di natura in \\
        trasformazione, sfidando il fissismo \\
        aristotelico.
        \end{tabular} \\
        \hline Diderot & XVIII secolo & \begin{tabular}{l} 
        Sostenitore dell'idea di una natura e di un \\
        Universo regolati soltanto dal caso.
        \end{tabular} \\
        \hline Erasmus Darwin & XVIII secolo & \begin{tabular}{l} 
        Concepi idee evoluzionistiche, suggerendo \\
        che la varietà delle forme animali derivasse \\
        da modificazioni ambientali su un "filamento \\
        vivente" primordiale.
        \end{tabular} \\
        \hline
    \end{tabular}\end{center}
    \vspace{0.25cm}
\end{snippet}

\begin{snippetdefinition}{principio-attualismo-definition}{Principio attualismo}
    Il termine \textit{attualismo} afferma che i fenomeni geologici o fisici che operano adesso, hanno sempre agito, con la stessa intensità nel passato dei tempi geologici. 
\end{snippetdefinition}

\begin{snippet}{6c377972-572d-439f-aaf0-03be4eaa3e0e}
    \begin{center}\begin{tabular}{|l|l|l|}
        \hline Esponente & Periodo & Principali Idee \\
        \hline James Hutton & XVIII secolo & \begin{tabular}{l} 
        Formulò il principio dell'attualismo in \\
        geologia, sostenendo che la Terra sia stata \\
        modellata da forze che sono tuttora in atto.
        \end{tabular} \\
        \hline
    \end{tabular}\end{center}
    \vspace{0.25cm}
\end{snippet}

\begin{snippetdefinition}{catastrofismo-definition}{Catastrofismo}
    Il \textit{catastrofismo} è una teoria evolutiva scientifica per la quale
    l'esistenza dei fossili erano dati da delle catastrofi naturali, che portavano a
    diversità fra le specie nel tempo.
\end{snippetdefinition}

\begin{snippet}{c8616a6a-f7e4-440c-a868-37fa3cf7edca}
    \begin{center}
    \begin{tabular}{|l|l|l|}
        \hline Esponente & Periodo & Principali Idee \\
        \hline Georges Cuvier & XVIII-XIX secolo & \begin{tabular}{l} 
        Interpreta i fossili come vittime di catastrofi, \\
        sostenendo la teoria del catastrofismo.
        \end{tabular} \\
        \hline
    \end{tabular}\end{center}
    \vspace{0.25cm}
\end{snippet}

\begin{snippet}{00f75cbc-caf8-4e6a-b143-3eb3fa513437}
    \begin{center}
    \begin{tabular}{|l|l|l|}
        \hline Esponente & Periodo & Principali Idee \\
        \hline Charles Lyell & XIX secolo & \begin{tabular}{l} 
        Sostenitore del gradualismo e attualismo, \\
        dimostrò l'antichità della Terra.
        \end{tabular} \\
        \hline \begin{tabular}{l} 
        Jean-Baptiste de \\
        Lamarck
        \end{tabular} & XVIII-XIX secolo & \begin{tabular}{l} 
        Interpretò i fossili come testimonianza del \\
        fatto che le specie si siano modificate nel \\
        tempo, propose il primo modello \\
        evoluzionistico.
        \end{tabular} \\
        \hline
    \end{tabular}\end{center}
    \vspace{0.25cm}
\end{snippet}

\section{Teoria di Lamarck (1744-1829)}

\begin{snippetdefinition}{lamarckismo-definition}{Lamarckismo}
    Il lamarckismo fu la prima teoria evoluzionistica e fu elaborata dal naturalista francese Jean-Baptiste de Lamarck (1744-1829), attualmente considerata errata e non conforme alle prove disponibili in tutti i settori della biologia.
    Il lamarckismo considera l'evoluzione come lo sviluppo dell'uso e del disuso.
\end{snippetdefinition}

\begin{snippetexample}{teoria-lamarck-giraffa}{Lamarckismo}
    Partendo dai fossili, è possibile notare che il collo delle giraffe fosse meno lungo.
    Il lamarckismo considera 
    Secondo la teoria di lamarck, il collo delle giraffe si sarebbe allungato con
    l'uso e questo carattere sarebbe poi stato trasmesso alla discendenza.
\end{snippetexample}

\section{Teoria di Darwin (1809-82)}

\begin{snippet}{26bdd6d8-4987-4add-afc0-83b98af18602}
    Nel 1859 pubblicò \textit{L'origine delle specie} per selezione naturale.
    Secondo Darwin:
    \begin{itemize}
        \item \textbf{adattamenti}: caratteri ereditari vantaggiosi, che aumentano le
            probabilità di un individuo di sopravvivere e riprodursi in un dato
            ambiente;
        \item \textbf{cambiamento nell'ambiente o spostamento}: selezionate le
        caratteristiche più adatte alle nuove condizioni (selezione naturale);
        \item \textbf{estinzione}: causata da cambiamenti di grande portata, frequente nella
        storia della vita sulla Terra.
    \end{itemize}

    Dimostrò che le specie attuali derivano da una successione di specie
    antenate attraverso un processo di discendenza con modificazioni
    (teoria dell'evoluzione con selezione naturale come meccanismo).

    Individuò una conferma della sua teoria nel processo di selezione
    artificiale con cui gli esseri umani hanno modificato molte specie di
    animali e piante per ottenerne varietà con le caratteristiche desiderate.
\end{snippet}

\begin{snippetdefinition}{selezione-naturale-definition}{Selezione naturale}
    La \textit{selezione naturale}, concetto introdotto da Charles Darwin
    nel 1859 nel libro L'origine delle specie è un meccanismo chiave dell'evoluzione
    e secondo cui, nell'ambito della diversità genetica delle popolazioni,
    si ha un progressivo (e cumulativo) aumento degli individui con caratteristiche
    ottimali per l'ambiente in cui vivono.
\end{snippetdefinition}

\end{document}