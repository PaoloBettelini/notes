\documentclass[preview]{standalone}

\usepackage{amsmath}
\usepackage{amssymb}
\usepackage{stellar}
\usepackage{definitions}

\begin{document}

\id{groups-semidirect-product}
\genpage

\section{Semidirect product}

\begin{snippet}{semidirect-product-motivation}
    If the \group[groups] \(G_1, G_2, G_n\)
    are \abeliangroup[abelian], their product remains \abeliangroup[abelian].
    We would like to construct complicated groups from relatively simpler \group[groups],
    hence also non-abelian.
    We relax the condition that both groups must be \normalsubgrptext, and require that
    at least one of them to be \normalsubgrptext.
\end{snippet}

%
% con i simboli invertiti
%
%Nel prodotto diretto interno, la commutatività tra GG e HH (grazie alla loro normalità reciproca) garantisce che possiamo ricostruire KK unicamente dai sottogruppi.
%Nel prodotto semidiretto interno, invece, il fatto che GG non sia normale significa che gli elementi di HH possono "modificare" gli elementi di GG tramite coniugio. Questo coniugio introduce un'azione che determina come combinare gli elementi di GG e HH. Senza questa informazione, i sottogruppi GG e HH non sono sufficienti a determinare la struttura di KK.

\begin{snippetdefinition}{group-internal-semidirect-product-definition}{Internal semidirect product of groups}
    Let \(G\) be a \group and let \(H \subgroupleq G\)
    and \(N \normalsubgrp G\) such that:
    \begin{enumerate}
        \item \(N \intersection N = 1\);
        \item \(G = HN\).
    \end{enumerate}
    We then say that \(G\) is the \emph{(internal) semidirect product}
    of \(H\) and \(N\) and we write
    \[
        G = H \ltimes N
    \]
\end{snippetdefinition}

\begin{snippetdefinition}{group-external-semidirect-product-definition}{External semidirect product of groups}
    Let \(H,N\) be \group[groups] and \(\varphi \colon H \fromto \text{Aut}(N)\)
    be a \grouphomomorphism.
    Consider the \binoperation on \(H \cartesianprod N\) defined as
    \[
        (h_1, n_1) \star (h_2, n_2)
        \triangleq
        (h_1h_2, \varphi(h_2)(n_1)n_2)
    \]
    Then, the \emph{external semidirect product} is defined as
    \[
        H \ltimes_\varphi N \triangleq (H \cartesianprod N, \star)
    \]
\end{snippetdefinition}

\end{document}