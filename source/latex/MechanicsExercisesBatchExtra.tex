\documentclass[preview]{standalone}

\usepackage{amsmath}
\usepackage{amssymb}
\usepackage{stellar}
\usepackage{definitions}
\usepackage{bettelini}

\begin{document}

\id{mechanics-ex-extra}
\genpage

\section{Exercises - Extra}

\begin{snippetexercise}{mechanics-ex-orbital-path}{Orbital Path Length}
    A particle moves in the plane along a path described by 
    \[
    \vec{R}(t) = 
    \begin{pmatrix}
        a \cos(\omega t) \\
        b \sin(\omega t)
    \end{pmatrix}.
    \]
    Show that this is an elliptical orbit, calculate the time required for a complete orbit, and express its length as a definite integral (without solving it).
\end{snippetexercise}

\begin{snippetsolution}{mechanics-ex-orbital-path-sol}{Orbital Path Length}
    We can rewrite the trajectory in terms of its components:
    \[
        x(t) = a \cos(\omega t), \quad y(t) = b \sin(\omega t)
    \]
    from which it follows that
    \[
        \frac{x^2}{a^2} + \frac{y^2}{b^2} = 1
    \]
    which an ellipse with axes aligned with the coordinate axes and lengths \(2a\) and \(2b\). The time required to complete one full orbit is the period of \(\vec{R}(t)\), which is 
    \[
        T = \frac{2\picircle}{\omega}.
    \]
    To calculate the length of the orbit, we first determine the velocity:
    \[
    \vec{V}(t) = 
    \begin{pmatrix}
        -a\omega \sin(\omega t) \\
        b\omega \cos(\omega t)
    \end{pmatrix}
    \]
    and we integrate its module over a period
    \begin{align*}
        \ell &= \integral[0][T][|\vec{V}(t)|][t] = \integral[0][T][\sqrt{a^2\omega^2 \sin^2(\omega t) + b^2\omega^2 \cos^2(\omega t)}][t] \\
        &= \integral[0][2\picircle][\sqrt{a^2 \sin^2(u) + b^2 \cos^2(u)}][u]
    \end{align*}
    This integral cannot be express in terms of elementary functions, except for the trivial case
    \(a=b\) (circular trajectory) for which \(\ell = 2\picircle a\).
\end{snippetsolution}

\begin{snippetexercise}{mechanics-ex-snowplough}{Snowplough}
    At a certain time in the morning it starts
    snowing, and at noon a snowplough leaves to clear the roads. The snow continues
    falling with constant intensity. It is known that the speed at which the snowplough proceeds is
    inversely proportional to the height of the snow.
    In the first two hours of work the snowplough manages to clear 4 km of road. In the following two
    following two hours, however, only 2 km are cleared. One wants to know at what time it started to
    it started snowing.
\end{snippetexercise}

\begin{snippetsolution}{mechanics-ex-snowplough-sol}{Snowplough}
    Let \(t=0\) be noon. The height of the snow follows
    \[
        h \propto t-t_0
    \]
    The velocity of the snowplough is given by
    \[
        v = \frac{\xi}{t-t_0}
    \]
    for some \(\xi\) which we don't know.
    The space cleared in the first two hours is given by
    \[
        s_1 = \integral[0][\tau][\frac{\xi}{t-t_0}][t] = \xi\ln\frac{t_0-\tau}{t_0}
    \]
    where \(\tau = 2\text{h}\). In the following two hours we have
    \[
        s_2 = \integral[\tau][2\tau][\frac{\xi}{t-t_0}][t] = \xi\ln\frac{t_0-2\tau}{t_0-\tau}
    \]
    By putting these together we get 
    \begin{align*}
        \frac{\xi\ln\frac{t_0-\tau}{t_0}}{\xi\ln\frac{t_0-2\tau}{t_0-\tau}} &= \frac{s_1}{s_2} \\
        \ln\frac{t_0-\tau}{t_0} &= 2\ln\frac{t_0-2\tau}{t_0-\tau} \\
        t_0^2\tau - t_0\tau^2 - \tau^3 &= 0 \\
        t_0 = \frac{1\pm\sqrt{5}}{2}\tau
    \end{align*}
    We consider the negative solution, meaning that it started snowing at \(10:45:50\).
\end{snippetsolution}

\end{document}