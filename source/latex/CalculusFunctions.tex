\documentclass[preview]{standalone}

\usepackage{amsmath}
\usepackage{amssymb}
\usepackage{parskip}
\usepackage{fullpage}
\usepackage{hyperref}
\usepackage{stellar}
\usepackage{definitions}

\begin{document}

\id{functions}
\genpage

\section{Definition}

\begin{snippetdefinition}{function-definition}{Function}
    Let \(A\) and \(B\) be \set[sets].
    A \textit{function} from \(A\) to \(B\), denoted \(f\colon A\to B\),
    is a \binrelation \(f\) from \(A\) to \(B\) where
    \[
        \forall x \in A \exists_{=1} \, y \in B \suchthat (x,y) \in f
    \]
    The set \(A\) is called \textit{domain} and the set \(B\) is called \textit{codomain}.
\end{snippetdefinition}

\begin{snippetdefinition}{function-image-definition}{Image}
    Let \(f\colon X\to Y\) be a function.
    The \textit{image} of \(f\) is defined as
    \[ \text{Im}(f) = \{y \in Y \suchthat y=f(x), x \in X \} \]
\end{snippetdefinition}

\begin{snippetdefinition}{binary-operation-definition}{Binary operation}
    Let \(A\) be a \set.
    A \textit{binary operation} on \(A\)
    is a \function \(\circ\colon A \cartesianprod A \to A\).
\end{snippetdefinition}

\section{Properties}

\begin{snippetdefinition}{injectivity-definition}{Injectivity}
    A \function \(f \colon A\to B\) is \textit{injective} if
    \[
        \forall a,b \in A, f(a) = f(b) \implies a = b
    \]
\end{snippetdefinition}

\begin{snippetdefinition}{surjectivity-definition}{Surjectivity}
    A \function \(f \colon A\to B\) is \textit{surjectiv} if
    \[
        \forall b \in B \exists a \suchthat f(a)=b
    \]
\end{snippetdefinition}

\begin{snippetdefinition}{bijectivity-definition}{Bijectivity}
    A \function \(f \colon A\to B\) is \textit{bijective} if
    it has a one-to-one correspondence between each element of \(A\) and  \(B\).
\end{snippetdefinition}

\begin{snippetcorollary}{bijectivity-equiv-inj-and-surj}{Bijectivity properties}
    A \function \(f \colon A\to B\) is bijective iff it is both injective and surjective.
\end{snippetcorollary}

\subsection{Basic results}

\begin{snippetproposition}{injective-composition-is-injective}{Injective composition is injective}
    Let \(f\) and \(g\) be \function[functions] that are \injective.
    Then, the function \(f(g(x))\) is \injective.
\end{snippetproposition}

\begin{snippetproposition}{surjective-composition-is-surjective}{Surjective composition is surjective}
    Let \(f\) and \(g\) be \function[functions] that are \surjective.
    Then, the function \(f(g(x))\) is \surjective.
\end{snippetproposition}

\begin{snippetproposition}{bijective-composition-is-bijective}{Bijective composition is bijective}
    Let \(f\) and \(g\) be \function[functions] that are \bijective.
    Then, the function \(f(g(x))\) is \bijective.
\end{snippetproposition}

\subsection{Invertibility}

\begin{snippetdefinition}{inverse-function-definition}{Inverse function}
    Let \(A\) and \(B\) be sets. A \function \(f\colon A \to B\) is \textit{invertible}
    if there exist another function \(f^{-1}\colon B \to A\), called the \textit{inverse function},
    such that
    \begin{itemize}
        \item \(\forall x \in A, f^{-1}(f(x)) = x\);
        \item \(\forall y \in B, f(f^{-1}(y)) = y\)
    \end{itemize}
\end{snippetdefinition}

\begin{snippettheorem}{invertible-function-bijective-theorem}{Invertibility \(\iff\) bijectivity}
    A \function is \snippetref[inverse-function-definition][invertible]
    \ifandonlyif it is \snippetref[bijectivity-definition][bijective].
\end{snippettheorem}

\subsection{Periodic functions}

\begin{snippetdefinition}{periodic-function-definition}{Periodic Function}
    A \function \(f\) is periodic with a period \(T\) if
    \[
        f(x) = f(x + kT), \quad k \in \naturalnumbers
    \]
\end{snippetdefinition}

\begin{snippetdefinition}{odd-function-definition}{Odd Function}
    A \function \(f\) is \textit{odd} if
    \[
        f(-x) = -f(x)
    \]
\end{snippetdefinition}

\begin{snippetdefinition}{even-function-definition}{Even Function}
    A \function \(f\) is even if
    \[
        f(-x) = f(x)
    \]
\end{snippetdefinition}

\end{document}