\documentclass[preview]{standalone}

\usepackage{amsmath}
\usepackage{amssymb}
\usepackage{bettelini}
\usepackage{stellar}

\hypersetup{
    colorlinks=true,
    linkcolor=black,
    urlcolor=blue,
    pdftitle={Stellar},
    pdfpagemode=FullScreen,
}

\begin{document}

\id{chimica-acidi-basi}
\genpage

\begin{snippetexample}{acidi-esempi}{Acidi}
    Alcuni esempi di acidi sono:
    \begin{itemize}
        \item Acido citrico (farmacologico);
        \item acido cloridrico o muriatico (fisiologico, stomaco);
        \item acido lattico (fisiologico);
        \item acido carbonico
    \end{itemize}
    e cibi come pomodoro, cacao amaro, pompelmo.
\end{snippetexample}

\begin{snippetexample}{basi-esempi}{Basi}
    Alcuni esempi di basi sono:
    \begin{itemize}
        \item soda caustica (idrossido di sodio);
        \item bicarbonato di sodio.
    \end{itemize}
\end{snippetexample}

\plain{Il nostro stomaco contiene acido per denaturare le proteine
del cibo e proteggersi dai batteri.
Mangiare cibi basici inibirebbe lo stomaco.}

\begin{snippet}{acidi-comuni}
    Alcuni composti acidi comuni sono:
    \begin{itemize}
        \item \textbf{idracidi} (es. acido cloridrico)
        \item \textbf{ossiacidi} e gli ossidi non metallici da cui originano (es. acido nitrici, acido carbonico)
        \item \textbf{carbossilici} composti organici contenenti il gruppo —COOH,
            di solito acidi deboli (es. acido acetico)
    \end{itemize}
\end{snippet}

\begin{snippetexample}{acido-hcl}{Acido HCl}
    L'HCl è un idracido forte
    \[
        \text{HCl} + \text{H}_2\text{O} \rightleftharpoons \text{Cl}^- + \text{H}_3\text{O}^+
    \]
    e
    \[
        \text{HCl} \rightleftharpoons \text{H}^+ + \text{Cl}^-
    \]
    La sua costante di equilibrio è molto alta, perché il suo equilibrio è completamente spostato a destra.
\end{snippetexample}

\begin{snippet}{ph-expl2}
    Lo ione \(H^+\) in acqua non esiste, è sempre attaccato ad una molecola di acqua (idronio, \(H_3O^+\)).

    L'acido solforico H\({}_2\)SO\({}_4\) deriva dalla combustione dei fossili.

    L'acidificazione degli oceani è una diretta conseguenza dell'aumento delle concentrazioni di anidride carbonica (CO2) nell'atmosfera indotte dall'uomo.
\end{snippet}

\begin{snippet}{basi-comuni}
    Alcuni composti basici comuni sono:
    \begin{itemize}
        \item \textbf{idruri} (es. ammoniaca)
        \item \textbf{idrossidi} e gli ossidi metallici da cui originano. Basi forti (es. didrossido di calcio, idrossido di sodio)
        \item \textbf{ammine} composti organici contenenti il gruppo —NH\({}_2\) (es. metilammina)
    \end{itemize}
\end{snippet}

\begin{snippet}{ph-expl3}
    Un acido è tale perché \textbf{libera} ioni H\({}^+\),
    mentre una base è tale perché \textbf{produce} OH\({}^-\).
    Infatti, idruri e ammine non contengono OH\({}^-\), ma esso deriva dal legale dell'acqua.
    Gli iddrosidi invece possono liberare questo ione.
    Nessuna sostanza è acida o basica se non reagisce con qualcosa, solitamente l'acqua.

    Negli idrossidi, che sono basi forti, la costante di equilibrio è molto spostata a destra.
    Addirittura, la costante non c'è e tutti i reagenti diventano prodotti.
    negli idrossidi vi è una differenza di elettronegatività molto alta (ionico),
    e quindi quando viene a contatto con l'acqua le molecole viene strappate.
\end{snippet}

\section{Definizioni di acido e base secondo Brönsted-Lowry}

\begin{snippet}{bronsted-lowry-definition}
    \begin{itemize}
        \item un \textbf{acido} è un donatore di protoni (ioni H\({}^+\));
        \item una \textbf{base} è un accettore di protoni (ioni H\({}^+\)).
    \end{itemize}
    Questa definizione non richiede la presenza di acqua per determinare se una sostanza
    è acida o basica.
    Una reazione tra un acido e una base può avvenire anche
    in assenza di acqua e quindi senza la liberazione di H\({}^+\) in
    soluzione.
\end{snippet}

\section{Neutralizzazione}

\begin{snippetdefinition}{neutralizzazione-definizioe}{Neutralizzazione}
    La \textit{neutralizzazione} è il processo per giungere una soluzione
    neutra, aggiungendo una base ad un acido e viceversa.
\end{snippetdefinition}

\plain{La neutralizzazione porta sempre alla formazione di acqua
e ad un composto ionico detto sale.}

Equazione molecolare: TODO
Equazione ionica: TODO
Equazione ionica netta: TODO

TODO: slide dopo

TODO: acidi forte (costante equilibrio grandissima, si spacca tutto)
e acidi deboli (costante equilibrio piccola), e basi.

% quando è un idrossido nella reazione non si scrive H_2O nei prodotti (?)
%
%%%%%%%
% CH3OH 1) non è né un acido né una base perché non vi è reazione
%
% 2) non è un idrossido
%
%
%HCO+NaOH \(\rightleftharpoons\) H2O2+NaCl
%\\ H+ e OH- formano H2O e Cl- si lega con Na+ formando NaCl, neutralizzandosi.
%Questa è una equazione ionica.
%
%
%Acido acetico: CH3COOH. Copie coniugate

\begin{snippetdefinition}{sostanza-anfotera}{Sostanza anfotera}
    Una \textit{sostanza anfotera} è una sostanza che
    può comportarsi come acido o come base, a seconda della sostanza con cui reagisce.
\end{snippetdefinition}

\plain{Tutte le sostanze anfiprotiche sono anfotere, ma non viceversa.
L'acqua è anfotera}

\plain{Costanti acide. Costanti di ionizzazione di acidi e basi deboli.
La concentrazione dell'acqua può ritenersi costante perché ce n'è molto di più rispetto alle altre sostanze.}

\begin{snippetdefinition}{acido-poliprotico}{Acido poliprotico}
    Un \textit{acido monoprotico} è una sostanza che può rilasciare
    un solo protone per molecola, mentre un \textit{acido poliprotico}
    può rilasciare più protoni per molecola.
\end{snippetdefinition}

\plain{La conducibilità elettrica è una misura
che indica quanti ioni sono presenti in una soluzione.}

\begin{snippet}{acqua-conduce}
    Nonostante l'acqua sia prima di ioni, l'acqua conduce.
    L'acqua è in un equilibrio dinamico con sè stessa
    \[
        \text{H}_2\text{O} + \text{H}_2\text{O} \rightleftharpoons \text{OH}^- + \text{H}_3\text{O}^+
    \]
    e questa minoranza di prodotto permette la conduzione.
\end{snippet}

\begin{snippetdefinition}{costante-ionizzazione-acida-definition}{Costante di ionizzazione acida}
    Per calcolare la concentrazione molare di H\({}^+\) per un acido
    debole si deve tenere in considerazione che non si
    dissociano del tutto in acqua e che [H\({}_2\)O] si può
    considerare costante.
    Si utilizza quindi la \textit{costante di ionizzazione acida}
    \[
        K_a = \frac{[\text{H}+][\text{A}^-]}{\text{HA}}
    \]
    e
    \[
        pK_a=-\log_{10}(K_a)
    \]
    La forza dell'acido aumenta al diminuire del valore di
    \(pK_a\).
\end{snippetdefinition}

\begin{snippetdefinition}{costante-ionizzazione-basica-definition}{Costante di ionizzazione basica}
    TODO
\end{snippetdefinition}

\begin{snippetdefinition}{ph-level-definition}{pH level}
    Il \textit{livello di pH} è una misura dell'acidità o dell'alcalinità di una soluzione.
    È una scala logaritmica che va da 0 a 14, dove 7 è considerato neutro.
    Un valore di pH inferiore a 7 indica acidità, mentre un valore di pH superiore a 7 indica alcalinità.
\end{snippetdefinition}

\begin{snippet}{ph-expl1}
    OH sta per ione idrossido, che è una molecola carica negativamente costituita da un atomo di ossigeno e un atomo di idrogeno.
    È la base coniugata dell'acqua (\(\text{H}_2\text{O}\)) e svolge un ruolo nel determinare il livello di pH di una soluzione.
    La concentrazione di ioni \(\text{OH}^-\) in una soluzione è direttamente correlata alla sua alcalinità,
    poiché maggiore è la concentrazione di ioni \(\text{OH}^-\), più alcalina è la soluzione.

    \begin{align*}
        \text{pH} &= - \log_{10}(\text{H}^+) \\
        \text{pOH} &= - \log_{10}(\text{OH}^-) \\
        \text{pH} + \text{pOH} &= 14 \\
    \end{align*}
\end{snippet}

\begin{snippet}{acidita-soluzione-expl}
    In tutte le soluzioni acquose, sono sempre presenti
    entrambi gli ioni OH\({}^-\) + H\({}_3\)O\({}^+\),
    indipendentemente dalla presenza di altri soluti.
    Dunque,
    \begin{itemize}
        \item \textbf{soluzione neutra:} [H\({}_3\)O\({}^+\)] \(=\) [OH\({}^-\)]
            \(\iff\) pH = 7;
        \item \textbf{soluzione acida:} [H\({}_3\)O\({}^+\)] \(>\) [OH\({}^-\)]
            \(\iff\) pH < 7;
        \item \textbf{soluzione basica:} [H\({}_3\)O\({}^+\)] \(<\) [OH\({}^-\)]
            \(\iff\) pH > 7.
    \end{itemize}
\end{snippet}

\end{document}
