\documentclass[preview]{standalone}

\usepackage{amsmath}
\usepackage{amssymb}
\usepackage{stellar}
\usepackage{definitions}

\begin{document}

\id{integers-divisibility-rules}
\genpage

\section{Preliminaries}

\begin{snippetlemma}{powers-of-ten-modulo-9}{Powers of \(10\) modulo \(9\)}
    Let \(n\in\naturalnumbers\).
    Then, \[ 10^n \equiv 1 \pmod{9} \]
\end{snippetlemma}

\begin{snippetproof}{powers-of-ten-modulo-9-proof}{powers-of-ten-modulo-9}{Powers of \(10\) modulo \(9\)}
    \begin{itemize}
        \item The base case is \(10^0 = 1 \equiv 1 \pmod{9}\).
        \item The induction case is \(10^{k+1} \equiv 1 \cdot 10 \pmod{9} \equiv 1 \pmod{9}\).
    \end{itemize}
\end{snippetproof}

\begin{snippetlemma}{powers-of-ten-modulo-11}{Powers of \(10\) modulo \(11\)}
    Let \(n\in\naturalnumbers\).
    Then, \[ 11^n \equiv {(-1)}^n \pmod{1} \]
\end{snippetlemma}

\section{Divisibility rules}

\begin{snippetproposition}{divisibility-by-9-sum-base-10}{Divisibility by \(9\) (base \(10\) sum)}
    Let
    \[
        N = \sum_{k = 1}^n d_k \cdot 10^{n-k}
    \]
    be an integer. Then, \(N\) is divisible by \(9\) \ifandonlyif
    \[
        \sum_{k = 1}^n d_k
    \]
    is also divisible by \(9\).
\end{snippetproposition}

\begin{snippetproof}{divisibility-by-9-sum-base-10-proof}{divisibility-by-9-sum-base-10}{Divisibility by \(9\) (base \(10\) sum)}
    By \snippetref[powers-of-ten-modulo-9][this lemma] we have:
    \begin{align*}
        N &= \sum_{k = 1}^n d_k \cdot 10^{n-k} \\
        &\equiv \sum_{k = 1}^n d_k \cdot 1 \pmod{9} \\
        &\equiv \sum_{k = 1}^n d_k \pmod{9}
    \end{align*}
\end{snippetproof}

\begin{snippetproposition}{divisibility-by-11-alternating-sum-base-10}{Divisibility by \(11\) (alternating base \(10\) sum)}
    Let
    \[
        N = \sum_{k = 1}^n d_k \cdot 10^{n-k}
    \]
    be an integer. Then, \(N\) is divisible by \(11\) \ifandonlyif
    \[
        \sum_{k = 1}^n {(-1)}^{k+1} d_k
    \]
    is also divisible by \(11\).
\end{snippetproposition}

\begin{snippetproof}{divisibility-by-11-alternating-sum-base-10-proof}{divisibility-by-11-alternating-sum}{Divisibility by \(11\) (alternating base \(10\) sum)}
    By \snippetref[powers-of-ten-modulo-11][this lemma] we have:
    \begin{align*}
        N &= \sum_{k = 1}^n d_k \cdot 10^{n-k} \\
        &\equiv \sum_{k = 1}^n d_k \cdot {(-1)}^{k+1} \pmod{11}
    \end{align*}
\end{snippetproof}

\end{document}