\documentclass[preview]{standalone}

\usepackage{amsmath}
\usepackage{amssymb}
\usepackage{parskip}
\usepackage{fullpage}
\usepackage{hyperref}
\usepackage{bettelini}
\usepackage{stellar}

\hypersetup{
    colorlinks=true,
    linkcolor=black,
    urlcolor=blue,
    pdftitle={Topology},
    pdfpagemode=FullScreen,
}

\begin{document}

\id{topology-definitions}
\genpage

\begin{snippetdefinition}{topological-space-definition}{Topological space}
    A \textit{topological space} is a tuple \((X, \mathcal{T})\)
    where \(X\) is a non-empty set and \(\mathcal{T} \subseteq \mathcal{P}(X)\)
    with the following conditions:
    \begin{enumerate}
        \item \(\emptyset, X \in \mathcal{T}\);
        \item \(\forall U, V \in \mathcal{T}, U \cap V \in \mathcal{T}\);
        \item \({\{U_i\}}_{i \in I} \subseteq \mathcal{T} \implies \bigcup_{i \in I} U_i \in \mathcal{T}\).
    \end{enumerate}
    The sets in \(\mathcal{T}\) are called the \textit{open sets} of \(X\).
\end{snippetdefinition}

\plain{Note that the intersection of infinitely many open sets need not be open.}

\end{document}

*limit point* or *accumulation point* p for file in S
- is every neighborhood of p contains points in S that are not p

*closure* of a subset S is a topological space
- The intersection of all closed sets containing S 
- S U the boundary of S
- S U its limit points

set of limits points = closure - isolated points
