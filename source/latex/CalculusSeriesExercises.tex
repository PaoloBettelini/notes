\documentclass[preview]{standalone}

\usepackage{amsmath}
\usepackage{amssymb}
\usepackage{stellar}
\usepackage{definitions}

\begin{document}

\id{series-exercises}
\genpage

\section{Exercises}

\subsection{Series with positive terms}

\begin{snippetexercise}{seriex-ex-1}{}
    Study the following \series
    \[
        \sum_{n=1}^\infty \frac{1}{(4n-1)(4n+3)}
    \]
\end{snippetexercise}

\begin{snippetsolution}{series-ex-1-sol}{}
    The \series is telescopic
    \begin{align*}
        \sum_{n=1}^\infty \frac{1}{(4n-1)(4n+3)}
        &= \sum_{n=1}^\infty
        \frac{\frac{1}{4}}{4n-1} + \frac{-\frac{1}{4}}{4n+3} \\
        &= \frac{1}{4} \sum_{n=1}^\infty
        \frac{1}{4n-1} - \frac{1}{4n+3} \\
        &= \frac{1}{4} \sum_{n=1}^\infty
        \frac{1}{b_n} - \frac{1}{b_{n+1}}, \quad b_n = 4n-1 
    \end{align*}
    Quindi il risultato è dato da
    \begin{align*}
        \frac{1}{4} \left[
            \frac{1}{3} - \lim_{n\to \infty} \frac{1}{4n-1}
        \right] = \frac{1}{12}
    \end{align*}
\end{snippetsolution}

\begin{snippetexercise}{seriex-ex-2}{}
    Study the following \series
    \[
        \sum_{n=2}^\infty \frac{2^{-2n+1}}{3^{n-2}} = a_n
    \]
\end{snippetexercise}

\begin{snippetsolution}{series-ex-2-sol}{}
    The \series is geometric
    \begin{align*}
        a_n &= \frac{2 \cdot 4^{-n}}{9^{-1} \cdot 3^n} \\
        &= 18 \cdot {\left(\frac{1}{2}\right)}^n       
    \end{align*}
    la serie diventa allora
    \begin{align*}
        18 \sum_{n=2}^\infty {\frac{1}{12}}^n
        &= 18 \left[\sum_{n=0}^\infty \left({\frac{1}{12}}^n\right) - 1 - \frac{1}{12}\right]
        \\
        &= 18 \left[ \frac{1}{1 - \frac{1}{12}} - 1 - \frac{1}{12} \right] \\
        &= \frac{3}{22}
    \end{align*}
\end{snippetsolution}

\begin{snippetexercise}{seriex-ex-3}{}
    Study the following \series
    \[
        \sum_{n=1}^\infty \frac{16-n^4}{n^2 + 3}
    \]
\end{snippetexercise}

\begin{snippetsolution}{series-ex-3-sol}{}
    Notiamo che il termine n-esimo non tende a zero, quindi la serie non converge.
\end{snippetsolution}

\begin{snippetexercise}{seriex-ex-4}{}
    Study the following \series
    \[
        \sum_{n=1}^\infty \frac{\log n}{n+1}
    \]
\end{snippetexercise}

\begin{snippetsolution}{series-ex-4-sol}{}
    Vogliamo confrontarla con una serie nota
    \begin{align*}
        \sum_{n=1}^\infty \frac{\log n}{n+1}
        = a_1 + \sum_{n=2}^\infty \frac{1}{(n+1){(\log n)}^{-1}}
    \end{align*}
    Ignoriamo il \(+1\) a denominatore. Questa serie può essere confrontata con una p-q-serie
    con \(p=1\) e \(q=-1\), quindi diverge.
\end{snippetsolution}

\begin{snippetexercise}{seriex-ex-5}{}
    Study the following \series
    \[
        \sum_{n=1}^\infty \frac{1}{\sqrt{n}}
    \]
\end{snippetexercise}

\begin{snippetsolution}{series-ex-5-sol}{}
    (con il confronto).
    Chiaramente, \(\frac{1}{\sqrt{n}} > \frac{1}{n}\)
    e quindi siccome la serie armonica diverge, anche questa diverge.
\end{snippetsolution}

\begin{snippetexercise}{seriex-ex-6}{}
    Study the following \series
    \[
        \sum_{n=1}^\infty \frac{
            n^2 + n\cos n + {\left(1 + \frac{1}{n}\right)}^{\sqrt{n}}
        }{
            {\left(n+\frac{1}{2}\right)}^4
            + n\log n + e^{-n}
        }
    \]
\end{snippetexercise}

\begin{snippetsolution}{series-ex-6-sol}{}
    Abbiamo il termine
    \begin{align*}
        a_n &= \frac{
            n^2 \left[1 + \frac{\cos n}{n} + \frac{1}{n^2} {\left(1 + \frac{1}{n}\right)}^{\sqrt{n}}\right]
        }{
            n^4 \left[
                {\left(1 + \frac{1}{2n}\right)}^4
                + \frac{\log n}{n^3} + \frac{e^{-n}}{n^4}
            \right]
        } \asymptotic \frac{
            1
        }{
            n^2
        }
    \end{align*}
    con
    \[
        {\left(1 + \frac{1}{n}\right)}^{\sqrt{n}} 
        = e^{\sqrt{n} \cdot \frac{1}{n}(1 + \littleO(1))}
        \to 1
    \]
    Questo ultimo passaggio è dato dal fato che \(\log (1 + \varepsilon_n) = \varepsilon(1 + \littleO(1))\).
    Quindi, la serie ha lo steso carattere della p-serie con \(p=2\) che converge.
\end{snippetsolution}

\begin{snippetexercise}{seriex-ex-7}{}
    Study the following \series
    \[
        \sum_{n=0}^\infty \frac{n^2}{3^n}
    \]
\end{snippetexercise}

\begin{snippetsolution}{series-ex-7-sol}{}
    Applichiamo il test del rapporto
    \begin{align*}
        \frac{a_{n+1}}{a_n} &= \frac{{(n+1)}^2}{3^{n+1}}
        \cdot \frac{3^n}{n^2} \\
        &= {\left(\frac{n+1}{n}\right)}^2 \cdot \frac{1}{3}
        \\
        &= \frac{1}{3} {\left(1 1 \frac{1}{n}\right)}^2 \to \frac{1}{3}
    \end{align*}
    Quindi, siccome \(L < 1\), la serie converge.
\end{snippetsolution}

\begin{snippetexercise}{seriex-ex-8}{}
    Study the following \series
    \[
        \sum_{n=1}^\infty \frac{n^{\sqrt{n}}}{2^n}
    \]
\end{snippetexercise}

\begin{snippetsolution}{series-ex-8-sol}{}
    Applichiamo il tst della radice ennesima
    \begin{align*}
        \sqrt[n]{a_n} = \frac{
            n^{\frac{\sqrt{n}}{2}}
        }{2}
        &= \frac{1}{2} e^{\frac{1}{\sqrt{n}}\log n} \to \frac{1}{2}
    \end{align*}
    Quindi, siccome \(L < 1\), la serie converge.
\end{snippetsolution}

\begin{snippetexercise}{seriex-ex-9}{}
    Study the following \series
    \[
        \sum_{n=1}^\infty \frac{
            n!
        }{
            e^{n^2}
        }
    \]
\end{snippetexercise}

\begin{snippetsolution}{series-ex-9-sol}{}
    \todo
\end{snippetsolution}

\begin{snippetexercise}{seriex-ex-10}{}
    Study the following \series
    \[
        \sum_{n=1}^\infty \frac{
            2^n + 3^{2n}
        }{
            10^{n+2}
        }
    \]
\end{snippetexercise}

\begin{snippetsolution}{series-ex-10-sol}{}
    La serie è geometrica
    %\begin{align*}
    %    \sum_{n=1}^\infty \frac{
    %        2^n + 3^{2n}
    %    }{
    %        10^{n+2}
    %    } &=
    %    \frac{1}{100} \left[
    %        \left(\sum_{n=1}^\infty \frac{
    %        2^n
    %    }{
    %        10^{n}
    %    }\right)
    %    +
    %    \left(\sum_{n=1}^\infty \frac{
    %        9^{n}
    %    }{
    %        10^{n}
    %    }\right)
    %    \right]
    %    \\
    %    &= 
    %    \frac{1}{100} \left[
    %        \left(\sum_{n=0}^\infty \frac{
    %        2^n
    %    }{
    %        10^{n}
    %    }\right)
    %    +
    %    \left(\sum_{n=0}^\infty \frac{
    %        9^{n}
    %    }{
    %        10^{n}
    %    }\right)
    %    -1-1
    %    \right]
    %    \\
    %    &= 
    %    \frac{1}{100} \left[
    %        \left(\frac{1}{1 - \frac{2}{10}}\right)
    %    +
    %    \left(\frac{1}{1 - \frac{9}{10}}\right)
    %    -1-1
    %    \right] \\
    %    &= \frac{37}{400}
    %\end{align*}
\end{snippetsolution}

\begin{snippetexercise}{seriex-ex-11}{}
    Study the following \series
    \[
        \sum_{n=1}^\infty \frac{
            1
        }{
            {(\log n)}^n
        }
    \]
\end{snippetexercise}

\begin{snippetsolution}{series-ex-11-sol}{}
    Possiamo estrarre l'esponente
    \begin{align*} % radice
        \sum_{n=1}^\infty {\left(\frac{
            1
        }{
            \log n
        }\right)}^n
    \end{align*}
    \todo
\end{snippetsolution}

\begin{snippetexercise}{seriex-ex-12}{}
    Study the following \series
    \[
        \sum_{n=2}^\infty \frac{
            e^{1/n} - 1
        }{
            n^2 \log n
        }
    \] % converge / p-qseries
\end{snippetexercise}

\begin{snippetsolution}{series-ex-12-sol}{}
    \todo
\end{snippetsolution}

\begin{snippetexercise}{seriex-ex-13}{}
    Study the following \series
    \[
        \sum_{n=1}^\infty \frac{
            \sin^2(n) + 1
        }{
            3^n + n
        }
    \]
\end{snippetexercise}

\begin{snippetsolution}{series-ex-13-sol}{}
    Siccome \(0 \leq \sin^2 (n) \leq 1\), la serie ha lo stesso carattere di
    \begin{align*}
        \sum_{n=1}^\infty \frac{
            2
        }{
            3^n + n
        } &=
        \sum_{n=1}^\infty \frac{
            2
        }{
            3^n \left(1 + \frac{n}{3^n}\right)
        } \\
        &\asymptotic 
        \sum_{n=1}^\infty \frac{
            1
        }{
            3^n
        }
    \end{align*}
    che converge per gerarchia degli infiniti.
\end{snippetsolution}

\begin{snippetexercise}{seriex-ex-14}{}
    Study the following \series
    \[
        \sum_{n=1}^\infty \frac{
            1
        }{
            n(n+1)(n+2)
        }
    \]
\end{snippetexercise}

\subsection{Series with terms of any sign}

\begin{snippetsolution}{series-ex-14-sol}{}
    \todo % telescopic
\end{snippetsolution}

\begin{snippetexercise}{seriex-ex-15}{}
    Study the following \series
    \[
        \sum_{n=1}^\infty \frac{{(-1)}^n}{n^2}
    \]
\end{snippetexercise}

\begin{snippetsolution}{series-ex-15-sol}{}
    Analizziamo la convergenza assoluta
    \begin{align*}
        \sum_{n=1}^\infty \left|\frac{{(-1)}^n}{n^2}\right| = 
        \sum_{n=1}^\infty \frac{1}{n^2}
    \end{align*}
    che converge per confronto con \(p\)-serie.
    Quindi, la serie è assolutamente convergente e quindi convergente.
\end{snippetsolution}

\begin{snippetexercise}{seriex-ex-16}{}
    Study the following \series
    \[
        \sum_{n=1}^\infty {(-1)}^n \frac{1}{n}
    \]
\end{snippetexercise}

\begin{snippetsolution}{series-ex-16-sol}{}
    Analizziamo la convergenza assoluta
    \begin{align*}
        \sum_{n=1}^\infty \left| {(-1)}^n \frac{1}{n} \right|
        =
        \sum_{n=1}^\infty \frac{1}{n}
    \end{align*}
    questa serie non converge, quindi la nostra serie iniziale non converge assolutamente.
    Tuttavia, possiamo notare che la serie soddisfa il criterio di Leibniz, quindi converge.
    La serie ha infatti forma \(\sum {(-1)}^n a_n\) dove \(a_n \geq 0\)
    e \(a_n \geq a_{n+1}\). Inoltre abbiamo anche la condizione necessaria per la convergenza,
    ossia che \(\lim a_n = 0\).
\end{snippetsolution}

\begin{snippetexercise}{seriex-ex-17}{}
    Study the following \series
    \[
        \sum_{n=1}^\infty {(-1)}^n \frac{
            2^n + n
        }{
            3^n + n^2
        }
    \]
\end{snippetexercise}

\begin{snippetsolution}{series-ex-17-sol}{}
    Analizziamo la convergenza assoluta
    \begin{align*}
        \sum_{n=1}^\infty \left|{(-1)}^n \frac{
            2^n + n
        }{
            3^n + n^2
        }\right| &=
        \sum_{n=1}^\infty \frac{
            2^n + n
        }{
            3^n + n^2
        } \\
        &= \sum_{n=1}^\infty
        \frac{
            2^n(1 + \frac{n}{2^n})
        }{
            3^n(1 + \frac{n^2}{2^n})
        } \\
        &= \sum_{n=1}^\infty
        \frac{
            2^n(1 + \littleO(1))
        }{
            3^n(1 + \littleO(1))
        }
    \end{align*}
    Il carattere è quindi il medesimo di
    \begin{align*}
        \sum_{n=1}^\infty {\left(\frac{2}{3}\right)}^n
    \end{align*}
    che converge in quanto serie geometrica convergente.
\end{snippetsolution}

\begin{snippetexercise}{seriex-ex-18}{}
    Study the following \series
    \[
        \sum_{n=1}^\infty {(-1)}^n \frac{2 + n}{1 + n + n^2}
    \]
\end{snippetexercise}

\begin{snippetsolution}{series-ex-18-sol}{}
    Analizziamo la convergenza assoluta
    \begin{align*}
        \sum_{n=1}^\infty \left| {(-1)}^n \frac{2 + n}{1 + n + n^2} \right|
        &=
        \sum_{n=1}^\infty \frac{2 + n}{1 + n + n^2} \\
        &= \sum_{n=1}^\infty \frac{
            n(1 + \frac{2}{n})
        }{
            n^2(1 + \frac{1}{n^2} + \frac{1}{n})
        } \asymptotic \frac{1}{n}
    \end{align*}
    Allora la serie non converge assolutamente.
    Applichiamo il criterio di Leibniz.
    Notiamo che il termine è definitivamente decrescente
    \begin{align*}
        \frac{2+n}{1+n+n^2} &\geq \frac{3+n}{1+(n+1)+{(n+1)}^2} 
        = \frac{3+n}{3+3n+n^2} \\
        (3+3n+n^2)(2+n) &\geq (3+n)(1+n+n^2) \\
        3+5n + n^2 &\geq 0
    \end{align*}
    Il limite del termine tende a \(0\)
    ed è strettamente positivo. Allora, la serie converge.
\end{snippetsolution}

\subsection{Series with parameters}

\begin{snippetexercise}{seriex-ex-19}{}
    Study the following \series
    \[
        \sum_{n=1}^\infty \frac{x^n}{n\cdot 2^n}
    \]
\end{snippetexercise}

\begin{snippetsolution}{series-ex-19-sol}{}
    Analizziamo la convergenza assoluta
    \begin{align*}
        \sum_{n=1}^\infty \left|\frac{x^n}{n\cdot 2^n}\right|
        &= \sum_{n=1}^\infty \frac{{|x|}^n}{n\cdot 2^n}
    \end{align*}
    Calcoliamo il limite
    \begin{align*}
        \lim_n |a_n| &= \lim_n {\left|\frac{x}{2}\right|}^n \cdot \frac{1}{n}
        \\
        &= \lim_n \frac{q^n}{n} = \begin{cases}
            0 & q \leq 1 \\
            +\infty & q > 1
        \end{cases}, \quad q = \left|\frac{x}{2}\right|
    \end{align*}
    Possiamo quindi notare che se \(q>1\), la serie non converge assolutamente.
    In particolare, il limite del termine non è pari a zero, e quindi la serie non converge.
    Applichiamo il criterio della radice n-esima,
    \begin{align*}
        \lim_n \sqrt[n]{|a_n|} &= \lim_n {\left(
            \frac{{|x|}^n}{n\cdot 2^n}
        \right)}^{1/n} \\
        &= \frac{|x|}{2} = q
    \end{align*}
    Allora, se \(q<1\), oppure \(|x| < 2\), la serie converge assolutamente,
    mentre nel caso \(q>1\), oppure \(|x| > 2\), la serie non converge.
    Infine, nel caso singolo \(q=1\), oppure \(|x| = 2\),
    il caso è inane. In questo caso la serie diventa
    \[
        \sum_{n=1}^\infty \frac{1}{n} {\left(\frac{|x|}{2}\right)}^n
        = \sum_{n=1}^\infty \frac{1}{n}
    \]
    che non converge assolutamente.
    La serie originale è invece
    \[
        \sum_{n=1} \frac{1}{n}
    \]
    che non converge, e
    \[
        \sum_{n=1} {(-1)}^n \frac{1}{n}
    \]
    con \(x=-2\), caso in cui converge.
    In conclusione la serie
    \[
        \begin{cases}
            \text{converge assolutamente} & -2 < x < 2 \\
            \text{non converge} & x < -2 \lor x \geq 2 \\
            \text{converge semplicemente} & x = -2
        \end{cases}
    \]
\end{snippetsolution}

\begin{snippetexercise}{seriex-ex-20}{}
    Study the following \series
    \[
        \sum_{n=1}^\infty {(-1)}^n \frac{1}{\sqrt[2024]{n}}
    \]
\end{snippetexercise}

\begin{snippetsolution}{series-ex-20-sol}{}
    \todo
\end{snippetsolution}

\begin{snippetexercise}{seriex-ex-21}{}
    Study the following \series
    \[
        \sum_{n=2}^\infty {(-1)}^n \frac{1}{n {\left[\log n\right]}^2}
    \]
\end{snippetexercise}

\begin{snippetsolution}{series-ex-21-sol}{}
    \todo
\end{snippetsolution}

\begin{snippetexercise}{seriex-ex-22}{}
    Study the following \series
    \[
        \sum_{n=1}^\infty {(-1)}^n \frac{1}{n}
        \log\left(\frac{n+1}{n}\right)
    \]
\end{snippetexercise}

\begin{snippetsolution}{series-ex-22-sol}{}
    \todo
\end{snippetsolution}

\begin{snippetexercise}{seriex-ex-23}{}
    Study the following \series
    \[
        \sum_{n=1}^\infty {(-1)}^n
        \frac{\log(e^n + 1)}{n}
    \]
\end{snippetexercise}

\begin{snippetsolution}{series-ex-23-sol}{}
    \todo
\end{snippetsolution}

\begin{snippetexercise}{seriex-ex-24}{}
    Study the following \series
    \[
        \sum_{n=1}^\infty {(-1)}^n
        \frac{
            n^2\log\left(1 + \frac{1}{n}\right)
            + {\left(1 + \frac{1}{n^2}\right)}^{n^2}
        }{
            {(\sqrt{n^4+1}-n)}^2 + \frac{e^n}{n!}
        }
    \]
\end{snippetexercise}

\begin{snippetsolution}{series-ex-24-sol}{}
    \todo
\end{snippetsolution}

\begin{snippetexercise}{seriex-ex-25}{}
    Study the following \series
    \[
        \sum_{n=1}^\infty {(-1)}^n
        \frac{e^{n \cdot \frac{x+1}{x-1}}}{n+\sqrt{n}}
    \]
\end{snippetexercise}

\begin{snippetsolution}{series-ex-25-sol}{}
    \todo
\end{snippetsolution}

\end{document}