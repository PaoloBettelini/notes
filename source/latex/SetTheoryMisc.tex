\documentclass[preview]{standalone}

\usepackage{amsmath}
\usepackage{amssymb}
\usepackage{stellar}
\usepackage{definitions}

\begin{document}

\id{settheory-misc}
\genpage

\section{Definitions}

\begin{snippetdefinition}{ring-definition}{Ring}
    A \textit{ring} \((R, +, \circ)\) is a triple containing a set \(R\) and two \binoperation[binary operations]
    \(+\) and \(\cdot\) on \(R\) such that:
    \begin{enumerate}
        \item \((R, +)\) is an \abeliangroup;
        \item \((R, \cdot)\) is a \monoid;
        \item \textbf{left distributivity:} \(\forall a,b,c\in R, a\circ(b+c) = (a\circ b) + (a \circ c)\);
        \item \textbf{left distributivity:} \(\forall a,b,c\in R, (b+c)\circ a = (b\circ a) + (c \circ a)\).
    \end{enumerate}
\end{snippetdefinition}

\begin{snippetdefinition}{commutative-ring-definition}{Commutative ring}
    A ring \((R, +, \circ)\) is said to be \textit{commutative} if:
    \begin{enumerate}
        \item \textbf{distributivity:} \(\forall a,b\in R, a\circ b = b\circ a\).
    \end{enumerate}
\end{snippetdefinition}

\begin{snippetdefinition}{field-definition-definition}{Field}
    \todo
\end{snippetdefinition}

\section{Exercises}

\begin{snippetexercise}{algebra-misc-ex1}{}
    Find the last digit in base \(10\) of
    \[
        \sum_{n=1}^{100} n!
    \]
\end{snippetexercise}

\begin{snippetsolution}{algebra-misc-ex1-sol}{}
    Note that for \(n\geq 5\), \(n!\) is a multiple of \(10\).
    So, \(10 + 4! + 3! + 2! + 1! = 43\) and the last digit is \(3\).
\end{snippetsolution}

\begin{snippetexercise}{algebra-misc-ex2}{}
    Prove that for \(n \in \integers\),
    \[
        n^3 + {(n+1)}^3 + {(n+2)}^3
    \]
    is a multiple of \(3\).
\end{snippetexercise}

\begin{snippetsolution}{algebra-misc-ex2-sol}{}
    \begin{align*}
        n^3 + {(n+1)}^3 + {(n+2)}^3 &= 3n^3 + 9n^2 + 15n + 9 \\
        &= 3(n^3 + 3n^2 + 5n + 3)
    \end{align*}
\end{snippetsolution}

\begin{snippetexercise}{algebra-misc-ex3}{}
    Prove that
    \[
        \nexists n \in \integers \suchthat 3n^2 - 1 = k^2
    \]
    for \(k\in\integers\).
\end{snippetexercise}

\begin{snippetsolution}{algebra-misc-ex3-sol}{}
    Assume that there exist a \(k\in\integers\) such that
    \[
        k^2 = 3n^2 - 1
    \]
    Note that \(3n^2 - 1 \equiv 2 \pmod{3}\).
    We have different cases for \(k\):
    \begin{itemize}
        \item \(k \equiv 0 \pmod{3} \implies k^2 \equiv 0 \pmod{3}\);
        \item \(k \equiv 1 \pmod{3} \implies k^2 \equiv 1 \pmod{3}\);
        \item \(k \equiv 2 \pmod{3} \implies k^2 \equiv 1 \pmod{3}\).
    \end{itemize}
    This means that the other term cannot be equivalent to \(2\) modulo \(3\).
\end{snippetsolution}

\begin{snippetexercise}{algebra-misc-ex4}{}
    Find the remainder of \(227^{228^{229}}\) divided by \(117\).
\end{snippetexercise}

\begin{snippetsolution}{algebra-misc-ex4-sol}{}
    We note that \(227 \equiv -1 \pmod{114}\).
    This means that the value is congruent to \(1\) if \(228^{229}\) is even
    and \(-1\) if it is of. But \(228^{229}\) is clearly even, so the result,
    so
    \[
        227^{228^{229}} \equiv 1 \pmod{117}
    \]
    which is the last digit.
\end{snippetsolution}


\begin{snippetexercise}{algebra-misc-ex5}{}
    Prove that for \(n\in\naturalnumbers\),
    \[
        11^{n+2} + 12^{2n+1}
    \]
    is a multiple of \(133\).
\end{snippetexercise}

\begin{snippetsolution}{algebra-misc-ex5-sol}{}
    We first note that \(12^2 = 144 \equiv 11 \pmod{133}\).
    \begin{align*}
        11^{n+2} + 12^{2n+1} &= 11^{n+2} + 12 \cdot 12^{2n} \\
        &= 11^{n+2} + 12 \cdot 11^n \\
        &= 11^n (11^2 + 12) \\
        &= 133 \cdot 11^n
    \end{align*}
\end{snippetsolution}

\end{document}