\documentclass[preview]{standalone}

\usepackage{amsmath}
\usepackage{amssymb}
\usepackage{parskip}
\usepackage{fullpage}
\usepackage{hyperref}
\usepackage{bettelini}
\usepackage{stellar}

\hypersetup{
    colorlinks=true,
    linkcolor=black,
    urlcolor=blue,
    pdftitle={Zeta Function},
    pdfpagemode=FullScreen,
}

\begin{document}

\id{zetafunction-prime-counting}
\genpage

\begin{snippetdefinition}{prime-counting-function-definition}{Prime-counting function}
    The \textit{prime-counting function}, denoted \(\pi(x)\),
    is defined as the number of primes less or equals than \(x\).
\end{snippetdefinition}

\section{Properties of the prime-counting function}

\begin{snippetproposition}{series-over-primes}{Series over primes}
    A series over primes can be rewritten as a series over the naturals as follows:
    \[
        \sum_{p\in P}^{\infty}f(p)=\sum_{n=2}^{\infty}\left[\pi(n)-\pi(n-1)\right]f(n)
    \]
\end{snippetproposition}

\begin{snippetproof}{series-over-primes-proof}{Series over primes}
    Consider the difference between \(\pi(x)\) of two consecutive integers
    \begin{align*}
        \pi (x)-\pi (x-1)= 
        \begin{cases}
            1,& \text{if } x\in P
            \\
            0,& \text{otherwise}
        \end{cases}
    \end{align*}

    Given a series over all prime numbers, we can extend it to all integers and multiply each term by this difference.
    The terms whose index is not a prime number will be multiplied by 0.
    \begin{align*}
        \sum_{p\in P}^{\infty}f(p)=\sum_{n=2}^{\infty}\left[\pi(n)-\pi(n-1)\right]f(n)
    \end{align*}
    Here we start at 2 since there are no prime numbers less than 2.
\end{snippetproof}

\section{Relationship with the zeta function}

\begin{snippettheorem}{zeta-function-and-prime-counting-function}{Relation between \(\zeta(z)\) and \(\pi(x)\)}
    There is a relation between the zeta function and the prime-counting function
    \[
        \ln\left(\zeta(s)\right)=
        s\int\limits_2^\infty
        \frac{\pi(x)}{x(x^2-1)}\,dx
    \]
\end{snippettheorem}

\begin{snippetproof}{zeta-function-and-prime-counting-function-proof}{Relation between \(\zeta(z)\) and \(\pi(x)\)}
    Start with the Euler product for \(\zeta(s)\)

    \begin{align*}
        \zeta (s)=\prod_{p\in P}^{\infty}\frac{1}{1-p^{-s}}
    \end{align*}
    
    We want to convert this product into a series in order to apply the identity of the last paragraph.
    \\
    We can take the natural logarithm of both sides and use the multiplication property
    \begin{align*}
        \ln\left(\zeta (s)\right)&=\ln\prod_{p\in P}^{\infty}\frac{1}{1-p^{-s}}
        \\
        &=\sum_{p\in P}^{\infty}\ln\left(\frac{1}{1-p^{-s}}\right)
        \\
        &=\sum_{p\in P}^{\infty}-\ln\left(1-p^{-s}\right)
    \end{align*}
    Now we can apply the identity
    \begin{align*}
        \ln\left(\zeta (s)\right)=\sum_{n=2}^{\infty}-\ln\left(1-n^{-s}\right)\left[\pi (n) - \pi (n-1)\right]
    \end{align*}
    
    The next goal is to factor out \(\pi (n)\)
    \begin{align*}
        \ln\left(\zeta(s)\right)
        &=\sum_{n=2}^{\infty}\left[\pi (n-1)\ln\left(1-n^{-s}\right)\right]
        -\sum_{n=2}^{\infty}\left[\pi (n)\ln\left(1-n^{-s}\right)\right]
        \\&=\sum_{n=2}^{\infty}\left[\pi (n)\ln\left(1-{(n+1)}^{-s}\right)\right]
        -\sum_{n=2}^{\infty}\left[\pi (n)\ln\left(1-n^{-s}\right)\right]
        \\
        &=\sum_{n=2}^{\infty}\pi (n)\left[\ln\left(1-{(n+1)}^{-s}\right)-\ln\left(1-n^{-s}\right)\right]
    \end{align*}
    To simplify further more, we consider the derivative of the function \(\ln(1-x^{-s})\).
    \\
    Using the chain rule we get
    \begin{align*}
        \frac{d}{dx}\ln\left(1-x^{-s}\right)=
        \frac{s}{x(x^s-1)}
    \end{align*}
    Therefore,
    \begin{align*}
        \ln\left(1-x^{-s}\right)=
        \int \frac{s}{x(x^s-1)}\,dx+C
    \end{align*}
    
    Considering \(f(x)=\ln(1-x^{-s})\), our series can be expressed as
    \begin{align*}
        \ln\left(\zeta(s)\right)=
        \sum_{n=2}^{\infty}\pi(n)\left[f(n+1)-f(n)\right]
    \end{align*}
    which can be written as an integral from \(n\) to \(n+1\)
    \begin{align*}
        \ln\left(\zeta(s)\right)&=
        \sum_{n=2}^{\infty}\pi(n)
        \int\limits_n^{n+1} f'(x)\,dx
        \\
        &=
        \sum_{n=2}^{\infty}\pi(n)
        \int\limits_n^{n+1}
        \frac{s}{x(x^s-1)}\,dx
        \\
        &=
        \sum_{n=2}^{\infty}
        \int\limits_n^{n+1}
        \frac{s\pi(x)}{x(x^s-1)}\,dx
    \end{align*}
    Instead of taking the sum of each of these integrals (2 to 3, 3 to 4, \ldots), we can make a single integral
    \begin{align*}
        \ln\left(\zeta(s)\right)=
        s\int\limits_2^\infty
        \frac{\pi(x)}{x(x^2-1)}\,dx
    \end{align*}
\end{snippetproof}

\section{Approximations}

%%%%%%%%%%%%%

\begin{snippetdefinition}{logarithmic-integral-function-definition}{Logarithmic integral function}
    The \textit{logarithmic integral function} is defined as
    \[
        li(x)=\int\limits_0^{x} \frac{dt}{\ln\,t}
    \]
\end{snippetdefinition}

\plain{The logarithmic integral function is a pretty good approximation to the prime-counting function.}

\subsection{Exact form}

\begin{snippettheorem}{prime-counting-function-exact-form}{Exact form of \(\pi(x)\)}
    The prime-counting function is exactly given by
    \[
        \pi(x)=Re(x)-\sum_{p}R(x^p)
    \]
    where
    \[
        R(x)=\sum_{n=1}^{\infty}\frac{\mu(n)}{n}li(\sqrt[n]{x})
    \]
    \(\mu(x)\) is the Möbius function and
    \(p\) indexes every non-trivial zero of the zeta function.
\end{snippettheorem}


\end{document}
