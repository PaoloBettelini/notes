\documentclass[preview]{standalone}

\usepackage{amsmath}
\usepackage{amssymb}
\usepackage{bettelini}
\usepackage{stellar}
\usepackage{definitions}

\begin{document}

\id{chimica-trasformazioni}
\genpage

\section{Trasformazioni}

\plain{Le trasformazioni possono essere classificate come <i>chimiche</i>, <i>fisiche</i> oppure <i>nucleari</i>.}

\begin{snippetdefinition}{trasformazione-chimica-definition}{Trasformazione chimica}
    Una \textit{trasformazione chimica}
    è un processo attraverso il quale le sostanze modificano la loro identità chimica,
    si trasformano cioè in sostanze differenti.
\end{snippetdefinition}

\begin{snippetdefinition}{trasformazione-fisica-definition}{Trasformazione fisica}
    Una \textit{trasformazione fisica} è una trasformazione reversibile
    che non cambia la natura delle sostanze coinvolte ma ne modifica l'apparenza.
\end{snippetdefinition}

\begin{snippetdefinition}{trasformazione-nucleare-definition}{Trasformazione nucleare}
    Una \textit{trasformazione nucleare}
    è un processo che riguarda il nucleo di un atomo di uno specifico elemento chimico,
    che viene convertito in un altro a diverso numero atomico coinvolgendo le forze nucleari.
\end{snippetdefinition}

\begin{snippetexample}{trasformazioni-chimiche-esempio}{Trasformazioni chimiche}
    \begin{itemize}
        \item Combustione di una candela (anche fisica).
        \item Cottura di un uovo (le proteine cambiano).
        \item Formazione della ruggine.
    \end{itemize}
\end{snippetexample}

\begin{snippetexample}{trasformazioni-chimiche-esempio}{Trasformazioni chimiche}
    \begin{itemize}
        \item Combustione di una candela (anche chimica).
        \item Sbucciare una mela.
        \item Scaldare il tisolfato di sodio.
        \item Dissoluzione dello zucchero nell'acqua.
    \end{itemize}
\end{snippetexample}

\end{document}
