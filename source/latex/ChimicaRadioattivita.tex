\documentclass[preview]{standalone}

\usepackage{amsmath}
\usepackage{amssymb}
\usepackage{bettelini}
\usepackage{stellar}
\usepackage{definitions}
\usepackage[version=4]{mhchem}

\begin{document}

\id{chimica-radioattivita}
\genpage

\section{Definizione}

\begin{snippetdefinition}{radioattivita-definition}{Radioattività}
    Radioactivity is a set of physical-nuclear processes
through which some unstable or radioactive atomic nuclei decay,
in a certain period of time called decay time.
\end{snippetdefinition}

\plain{An unstable nuclei will keep emitting radiations
and transmuting to other nuclei until the atom is stable.}

\section{Decadimento}

\begin{snippet}{Decadimento}
    The mass of a radioactive material will decrease exponentially.
    \[
        M(t) = M_0 \cdot e^{-kt}
    \]

    \(M(t)\) is the mass (or number or particles)
    after a certain time \(t\). \(M_0\) is the initial mass
    and \(k\) is the rate of decay.
\end{snippet}

\section{Tempo di dimezzamento}

\begin{snippet}{tempo-di-dimezzamento}
    The time of half-life is given by \(t_\frac{1}{2} = \frac{\ln 2}{k}\).

    \begin{align*}
        \frac{1}{2}M_0 &= M_0 e^{-kt} \\
        \frac{1}{2} &= e^{-kt} \\
        \ln\left(\frac{1}{2}\right) &= -kt \\
        t &= \frac{\ln 2}{k}
    \end{align*}
\end{snippet}

\section{Tipi di radiazione}

\begin{snippet}{tipi-di-radiazione}
There are three types of radiations that can be emitted by an unstable nucleai.

\paragraph{\(\alpha\) particles}

An \(\alpha\) particle is a helium nuclei. For example

\[
    \ce{^238_92U -> ^4_2\alpha{} + ^234_90Th}
\]

\paragraph{\(\beta\) particles}

There are two types of \(\beta{}\) particles. \(\beta{}^+\) and \(\beta{}^-\).
A \(\beta{}^+\) particle is emitted when the nuclei is unstable due to
having too many protons, whist the \(\beta{}^-\) one is emitted when it has
too many neutrons.

\[
    \begin{cases}
        \beta{}^+,\quad \ce{^0_{+1}e} \text{ (positron)} \\
        \beta{}^-,\quad \ce{^0_{-1}e} \text{ (electron)}
    \end{cases}
\]

\paragraph{\(\gamma\) particles}

\(\gamma\) rays are photons of electromagnetic energy. They have \(0\) mass and \(0\) charge.
\end{snippet}

\includesnpt[width=75\%|src=/snippet/static/decadimento-alpha.png]{centered-img}

% Fusione, fissione

\end{document}
