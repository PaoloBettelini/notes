\documentclass[preview]{standalone}

\usepackage{amsmath}
\usepackage{amssymb}
\usepackage{stellar}
\usepackage{definitions}

\begin{document}

\id{cesare-beccaria-caffe}
\genpage

\section{Il caffè}

\begin{snippetdefinition}{accademia-dei-pugni-definition}{Accademia dei Pugni}
    L'\textit{Accademia dei Pugni}, anche chiamata \textit{Società dei Pugni},
    fu un'istituzione culturale fondata nel 1761 a Milano.
\end{snippetdefinition}

\begin{snippetdefinition}{il-caffe-definition}{Il caffè}
    Il Caffè (1764-1766) fu un periodico italiano, pubblicato dal giugno 1764 al maggio 1766.
    Nacque a Milano ad opera dei fratelli Pietro ed Alessandro Verri con il contributo del
    filosofo e letterato Cesare Beccaria e del gruppo di intellettuali che era solito raccogliersi
    all'Accademia dei Pugni. I fondatori del Caffè, pur provenendo dall'aristocrazia,
    furono i portavoce delle istanze culturali, sociali e politiche delle classi emergenti che
    puntavano allo svecchiamento delle istituzioni e alla razionalizzazione dell'apparato statale.
\end{snippetdefinition}

\plain{Il nome caffè deriva dal luogo, ossia la bottega, da dove nascono questi testi dell'illuminismo lombardo.}

\end{document}