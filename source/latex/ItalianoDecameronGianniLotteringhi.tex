\documentclass[preview]{standalone}

\usepackage{amsmath}
\usepackage{amssymb}
\usepackage{stellar}

\hypersetup{
    colorlinks=true,
    linkcolor=black,
    urlcolor=blue,
    pdftitle={Stellar},
    pdfpagemode=FullScreen,
}

\begin{document}

\id{italiano-decameron-gianni-lotteringhi}
\genpage

\section{Analisi}

\begin{snippet}{gianni-lotteringhi-analisi}
    \textbf{Rubrica:} Gianni Lotteringhi ode di notte toccar l'uscio suo; desta la moglie, ed ella gli fa accredere che egli è la fantasima; vanno ad incantare con una orazione, ed il picchiare si rimane.

    % Gianni ha successo nel suo lavoro ma è scarso altrove
    % I frati di un convento nominano Capo Gianni (per sfruttarlo un po' perché ricco)
    % Lui fornisce tessuti per i loro abiti in cambio di insegnamenti (preghiere)
    
    % personaggi
    La novella è caratterizzata da un triangolo amoroso fra Tessa, Gianni e Federigo.
    Gianni ha avuto molta fortuna con il suo lavoro, ma altrove è piuttosto scarso.
    Esso è una persona molto ingenua, e per quanto riguarda la religione è molto
    superstizioso e bigotto (§§4-5).
    Federigo è bello, giovane e fresco (§6).
    Rispetto a Gianni, Federgio è più intelligente e allegro.
    I due sono infatti molto distanti.
    % differenze
    Il marito Gianni si mangia la carne salata, mentre l'amante si gode la cena completa (§§12-13).
    La moglie con il marito dorme e basta, mentre con l'amante si sfoga in effusioni sessuali (§8 e §20).
    Inoltre, un altro elemento di differenza è quello delle preghiere. Infatti, Gianni intende le preghiere in maniera letteraria,
    mentre Federigo riesce ad interderne il significato (§8 e §20).
    %
    Tessa è bella, intelligente e saggia (§6). Rispetto al marito è molto più astuta, ma è innamorata di Federigo.
    Conosce bene il marito e sa di poterlo ingannare (§7).
    Il marito viene ingannato dicendogli che la preghiera può ora essere detto dal momento che sono
    entrambi presenti, facendolo sentire importante.
    
    
    % non è importante che ci siano "2 versioni"
    % Il consiglio di Emilia alle altre novellatrici è letterale; la beffa è utile per il tradimento
\end{snippet}


Tema: Beffe delle mogli ai mariti 
Analisi: 
	1) I personaggi
	2) La beffa

I personaggi
Hanno un rapporto triangolare; Gianni è bigotto, chiuso e superstizioso. Nella novella viene ridicolizzato. Federigo è una persona allegra (§28) con valori "nuovi".  Boccaccio non vuole far passare Gianni come martire. 
Gli elementi che oppongono i due sono innanzitutto il cibo; (§§ 12 e 13)  Gianni mangia la carne salata, mentre Federico mangia i capponi, uova e vino. il sesso (§§ 8 e 20) dato che Federico fa sesso mentre Gianni no.
Il terzo terreno divisivo sono le preghiere, perché quando sono riferite a Gianni lui prega veramente (§20) mentre quando sono riferite a Federico sono riferite in maniera erotica (§8). Si crea dunque un triangolo amoroso con Tessa sulla punta in cima e i due uomini in basso ai lati.
La moglie è intelligente, saggia, bella. Sa che può ingannare il marito con una certa facilità, escogita un piano per la beffa e per incontrare l'amante. Ha una buona capacità di pianificazione. All'inizio fa finta di dormire, poi quando continua lei reagisce; per non fare insospettire il marito dice al marito che era già successo nei giorni precedenti, l'orazione.  

Analisi e riassunto
[2] All'inizio Emilia non si ritiene degna di raccontare e pensa che forse ci sarebbe
qualcuno più degno di lei. [3] Si rivolge poi alle donne, sempre con un aggettivo positivo,
dicendo che insegnerà loro una preghiera per scacciare i fantasmi nel caso in cui gli sarà
utile in futuro. [4]. C'era a Firenze uno stamaiolo di nome Gianni Lotteringhi, che fa parte di
un gruppo religioso devoto a Santa Maria Novella. I frati della confraternita lo nominano
capo, con lo scopo di sfruttarlo. [5] Gianni fornisce ai frati i tessuti per i loro abiti e in
cambio loro gli insegnano le preghiere, ma lui è così ingenuo che non se ne accorge. [6]
Gianni è sposato con una bella donna, saggia e intelligente, che conosce l'ingenuità del
marito. È innamorata di un altro uomo di nome Federico, bello e giovane, con cui si
incontra nella casa di campagna in estate. Gianni, in quella casa, ogni tanto ci va per
cena, ogni tanto ci dorme e ogni tanto non ci va. [8] La serva avvisa Federico che il marito
non c'è, così lui viene per cena. [9] Entrambi vogliono che la relazione vada avanti e
vogliono evitare di usare la serva ogni volta. [10] la moglie allora mette un teschio, che se
girato verso Firenze il marito non c'è, se girato dalla parte opposta il marito è in casa. [11]
Questa situazione avviene tante volte. [12] La moglie prepara la cena per l'amante, ma il
marito, che non doveva arrivare, arriva molto tardi. [13] Allora dice alla fante di portare la
cena in un giardino per Federico. [14] era cosi dispiaciuta che si dimentico di dire alla fante
di avvertire Federico che c'è il marito. [15] Federico quindi arriva e bussa, ma Gianni lo
sente e fa finta di dormire. [16] Federico bussa di nuovo e Gianni chiede alla moglie se
anche lei sente bussare. [17] La moglie, che ha sentito, fa finta di svegliarsi e chiede al
marito di ripetere. [18] Gianni quindi ripete ciò che ha detto. [19] La moglie dice al marito
che c'è un fantasma, di cui ha avuto paura in queste notti. [20] Il marito dice alla moglie di
non preoccuparsi perché ha pregato, quindi il fantasma non può nuocergli. [21] La moglie
sa che deve in qualche modo avvertire Federico, quindi dice al marito di dire una
preghiera per esorcizzare il fantasma. [22] Gianni le chiede come si può fare per
esorcizzarlo. [23] La moglie gli dice che conosce una formula, che le ha insegnato una
donna del villaggio, che è efficace. [24] Lei dice non se la sentiva di farlo da sola, ma ora
che c'è lui, possono farlo. [25] Allora i due vanno davanti alla porta. [26] Gianni sputa. [27]
La donna comincia l'orazione e dice al fantasma, arrivato con l'erezione, che con
l'erezione deve tornarsene a casa. Poi aggiunge che nell'orto può trovare il cibo che ha
preparato. [28] Federico è geloso ma capisce la situazione, e anche se ha malinconia di
non poter stare con la donna, si deve trattenere dalle risate per la beffa. [29] La donna e il
marito, in seguito, tornano a letto. [30] Federico trova il cesto nell'orto e torna a casa per
cenare. Le volte successive lui e l'amante ridono dell'accaduto. [31] Emilia esplicita che
alcuni dicono, però, che la donna aveva avvertito Federico di non vedere sistemando il
teschio nella giusta direzione, ma un lavoratore lo aveva mosso e così Federico venne.
[32] Si dice anche che la donna aveva fatto l'orazione in questo modo: fantasma, non ho
girato io la testa dell'asino, ma qualcun altro, io sono qui con mio marito.
[33] Emilia dice che una vicina ha detto che sono capiate tutte e due le vicende ma con
personaggi diversi, entrambi stupidi. [34] In conclusione, si rivolge alle donne, dicendo loro
di scegliere la vicenda che gli piace di più oppure entrambe: impararle tutte e due può
essere utile.

% 
% Personaggi
% Gianni [4] è molto credulone, superstizioso e bigotto. Viene nella novella ridicolizzato.
% Federico [6] è bello, giovane, fresco. A differenza di Gianni, è più intelligente e allegro [28].
% Questi due personaggi sono opposti, appartengono a mondi diversi e hanno valori diversi:
% Gianni è un uomo vecchio, serio, legato al lavoro e agli obblighi; Federico è un uomo
% giovane e allegro. Un'altra caratteristica che divide questi due uomini è il cibo, perché uno
% dei due mangia molto meglio. Federico mangia i capponi, le uova e il vino [12]. Gianni
% mangia la carne salata [13]. Un'altra caratteristica che li distingue è l'esperienza sessuale
% con la donna, perché Federico consuma fisicamente l'amore con lei [8], mentre Gianni no
% [20]. Il terzo elemento divisivo è quello delle preghiere. Gianni prega davvero, quindi le
% preghiere rivolte a lui sono prese letteralmente [20]. Le preghiere riferite a Federico sono
% invece ambigue e illusive all'erotismo, sia quanto giace con la donna che quando si
% beffano del marito. Si crea dunque un triangolo amoroso con Tessa sulla punta in cima e i
% due uomini in basso ai lati. Tessa [6] è bella, saggia, intelligente e astuta rispetto al marito.
% È innamorata di Federico, con cui condivide forti emozioni mentali e sessuali. Le spie della
% sua scaltrezza sono che conosce bene il marito e sa che può ingannarlo facilmente perché
% legge il suo comportamento [6], escogita il piano per incontrare clandestinamente l'amante
% [10], dimostra sangue freddo simulando il sonno [17], fa sentire il marito importante
% risaldando la coppia [24] e reagisce alla situazione in modo improvviso e svelto
% inventandosi l'orazione [27].
%

\end{document}