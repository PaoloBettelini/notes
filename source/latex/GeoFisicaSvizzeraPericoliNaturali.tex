\documentclass[preview]{standalone}

\usepackage{amsmath}
\usepackage{amssymb}
\usepackage{tikz}
\usepackage{stellar}
\usepackage{bettelini}

\hypersetup{
    colorlinks=true,
    linkcolor=black,
    urlcolor=blue,
    pdftitle={Assets},
    pdfpagemode=FullScreen,
}

\begin{document}

\id{geofisica-svizzera-pericoli-naturali}
\genpage

\section{Definizione}

\begin{snippetdefinition}{pericolo-naturale-definition}{Pericolo naturale}
    Un \textit{pericolo naturale} è un evento naturale che può recare danno alla
    natura, alle persone o alle cose.
\end{snippetdefinition}

\begin{snippet}{pericoli-naturali-lista}
    Alcuni pericoli naturali possono essere:
    \begin{itemize}
        \item valanghe;
        \item alluvioni;
        \item terremoti;
        \item movimento di masse (frane e smottamenti).
    \end{itemize}
\end{snippet}

\begin{snippet}{pericoli-naturali-expl1}
    Per avere un rischio, oltre all'esistenza di un pericolo naturale, devono essere presi
    in considerazione la sua frequenza e il potenziale di danno (effetti che tale
    avvenimento può causare).

    Il rischio viene calcolato come il prodotto di
    esposizione, vulnerabilità e probabilità che un dato evento naturale avvenga.
\end{snippet}

\section{Gradi di pericolo}

\includesnpt[width=80\%|src=/snippet/static/gradi-pericolo-svizzera.png]{centered-img}

\end{document}