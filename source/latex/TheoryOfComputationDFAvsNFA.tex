\documentclass[preview]{standalone}

\usepackage{amsmath}
\usepackage{amssymb}
\usepackage{tikz}
\usepackage{wrapfig}
\usepackage{bettelini}
\usepackage{stellar}
\usepackage{definitions}

% =======
\usetikzlibrary{ % tikz packages
    automata,positioning,
    arrows.meta,bending
}
\tikzset{every state/.style={
    inner sep=2pt,
    minimum size=4pt
}}
\tikzset{>=stealth}  %latex, to, stealth
% Empty string symbol.
\newcommand{\emptyString}{\lambda}
% =======

\begin{document}

\id{theoryofcomputation-dfa-vs-nfa}
\genpage


\section{Equivalente of DFAs and NFAs}

\begin{snippettheorem}{dfa-nfa-equivalence}{DFA and NFA equivalence}
    Anything that can be computed by a NFA can also be computed by a DFA and vice versa.
\end{snippettheorem}

\begin{snippetproof}{dfa-nfa-equivalence-proof}{DFA and NFA equivalence}
    \begin{enumerate}
        \item \textbf{DFA to NFA conversion:}
        Let \(D=(Q, \Sigma, \delta, q, F)\) be a DFA.
        \(\delta\) is not a transition function of a NFA, so we need to redefine it as \(\delta'\).
        Since \(\delta\) cannot process \(\emptyString\), \(\delta'\)
        it is defined as
        \[
            \delta'(r, a)=
            \begin{cases} 
                \delta(r, a) & x \neq \emptyString \\
                \emptyset & x = \emptyString
            \end{cases}
        \]
        where \(r\) is a state in \(Q\) and \(a\) is a symbol in \(\Sigma_\emptyString\). \\
        We can conclude that \(N=(Q, \Sigma, \delta', q, F)\).
        \item \textbf{NFA to DFA conversion:}
        Let \(N=(Q, \Sigma, \delta, q, F)\) be a NFA. The idea is to construct a DFA \(M=(Q', \Sigma, \delta', q', F')\)
        that runs all the possible combinations that could be run by \(N\) at the same time.
        Any state of \(M\) is a set of states \(R \in \powerset(Q)\), so we will say that
        \(Q'=\powerset(Q)\). The set of accept states is any state \(R\) which contains an accept
        state of \(N\)
        \[
            F' = \{R \in Q' \suchthat R \intersection F \neq \emptyset\}
        \]
        Let's assume that \(N\) does not execute any \(\emptyString\)-transitions.
        \(q'\) would be \(\{q\}\) and \(\delta'\) would be
        \[
            \delta'(R, a) = \bigcup_{r\in R}\delta(r,a)
        \]
        which is the union of all possible states \(N\) could switch to.
        Recall that for every \(r\in R\), \(\delta(r,a)\) is a set
        of all possible states to switch to.
        
        Let's now remove the previous assumption. Now, \(M\)
        must also consider every state that could be reached by
        making zero or more \(\emptyString\)-transitions.
        The \(\emptyString\)-closure
        for a state \(r\), \(C_\emptyString(r)\), is defined as
        the set of all possible states that can be reached from \(r\) by making
        zero or more \(\emptyString\)-transitions.
        The \(\emptyString\)-closure for a set of states \(R\) is defined as
        \[
            C_\emptyString(R) = \bigcup_{r\in R}C_\emptyString(r)
        \]
        The initial state \(q'\) is now given by \(C_\emptyString(q)\)
        and the transition function
        \[
            \delta'(R, a) = \bigcup_{r\in R}C_\emptyString(\delta(r, a))
        \]
    \end{enumerate}
\end{snippetproof}

%\subsection{DFA to NFA conversion}
%\subsection{NFA to DFA conversion}

\end{document}