\documentclass[preview]{standalone}

\usepackage{amsmath}
\usepackage{amssymb}
\usepackage{stellar}
\usepackage{bettelini}

\hypersetup{
    colorlinks=true,
    linkcolor=black,
    urlcolor=blue,
    pdftitle={Biologia},
    pdfpagemode=FullScreen,
}

\begin{document}

\title{Biologia}
\id{biologia-replicazione}
\genpage

\section{Replicazione}

\begin{snippetdefinition}{mitosi-definition}{Mitosi}
    La \textit{mitosi} consiste nella duplicazione completa di una cellula.
\end{snippetdefinition}

\begin{snippet}{4c8d854e-4e8e-485a-97de-243088d665ec}
    Nella cellula eucariote umana il DNA è diviso in 46 filamenti, chiamati cromosomi.
    Prima di fare la mitosi è essenziale duplicare il materiale genetico, ossia il DNA.
    La mitosi può produrre solo cellule \textbf{somatiche}.

    Per quel che concerne la riproduzione sessuale,
    avviene la \textbf{meiosi}, che produce cellule \textbf{germinali}
    (ossia spermatozoi e ovuli), chiamati \textbf{gameti}.
    Essi contengono solamente 23 cromosomi ciascuno.

    La fecondazione è il processo di incontro fra ovulo e spermatozoo,
    dove il numero di cromosomi si somma.
    I 46 cromosomi sono in realtà 23 coppie, ogni coppia con una componente dal padre e uno dalla madre.
\end{snippet}

\begin{snippetdefinition}{zigote-definition}{Zigote}
    Lo \textit{zigote} è la prima cellula prodotta dalla fecondazione.
\end{snippetdefinition}

\begin{snippet}{225f2d52-9773-477e-999a-899f4218610d}
    Eseguendo la mitosi, lo zigote si duplica in \(2^n\) cellule.
\end{snippet}

\begin{snippetdefinition}{aploide-definition}{Aploide}
    Con \textit{aploide} si intende una cellula con solo un set (23) di cromosomi.
\end{snippetdefinition}

\plain{Ovuli e spermatozoi sono aploidi.}

\begin{snippetdefinition}{diploide-definition}{Diploide}
    Con \textit{diploide} si intende una cellula con 23 coppie (46) cromosomi.
\end{snippetdefinition}

\plain{Due cellule aploidi formano quattro cellule aploidi, replicandosi due volte.
Le uniche cellule che fanno la meiosi sono le cellule diploidi nei testicoli e nelle ovaie.}

\begin{snippetdefinition}{cromosomi-omologhi-definition}{Cromosomi omologhi}
    I cromosomi appartenenti ad una coppia di cromosomi con gli stessi geni vengono detti
    \textit{cromosomi omologhi}.
\end{snippetdefinition}

\plain{La meiosi consiste nel separare tutti i cromosomi omologhi per averne 23.
La meiosi crea 4 cellule, mentre la mitosi 2.}

\end{document}