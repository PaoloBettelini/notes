\documentclass[preview]{standalone}

\usepackage{amsmath}
\usepackage{amssymb}
\usepackage{stellar}
\usepackage{definitions}

\begin{document}

\id{series-arbitrary-terms-exercises}
\genpage

\section{Series with non-negative terms}

\begin{snippetexercise}{series-arbitrary-ex-1}{}
    Study the following \series
    \[
        \sum_{n=1}^\infty \frac{16-n^4}{n^2 + 3}
    \]
\end{snippetexercise}

\begin{snippetsolution}{series-arbitrary-ex-1-sol}{}
    Notiamo che il termine n-esimo non tende a zero, quindi la serie non converge.
\end{snippetsolution}

\begin{snippetexercise}{series-arbitrary-ex-2}{}
    Study the following \series
    \[
        \sum_{n=1}^\infty \frac{{(-1)}^n}{n^2}
    \]
\end{snippetexercise}

\begin{snippetsolution}{series-arbitrary-ex-2-sol}{}
    Analizziamo la convergenza assoluta
    \begin{align*}
        \sum_{n=1}^\infty \left|\frac{{(-1)}^n}{n^2}\right| = 
        \sum_{n=1}^\infty \frac{1}{n^2}
    \end{align*}
    che converge per confronto con \(p\)-serie.
    Quindi, la serie è assolutamente convergente e quindi convergente.
\end{snippetsolution}

\begin{snippetexercise}{series-arbitrary-ex-3}{}
    Study the following \series
    \[
        \sum_{n=1}^\infty {(-1)}^n \frac{1}{n}
    \]
\end{snippetexercise}

\begin{snippetsolution}{series-arbitrary-ex-3-sol}{}
    Analizziamo la convergenza assoluta
    \begin{align*}
        \sum_{n=1}^\infty \left| {(-1)}^n \frac{1}{n} \right|
        =
        \sum_{n=1}^\infty \frac{1}{n}
    \end{align*}
    questa serie non converge, quindi la nostra serie iniziale non converge assolutamente.
    Tuttavia, possiamo notare che la serie soddisfa il criterio di Leibniz, quindi converge.
    La serie ha infatti forma \(\sum {(-1)}^n a_n\) dove \(a_n \geq 0\)
    e \(a_n \geq a_{n+1}\). Inoltre abbiamo anche la condizione necessaria per la convergenza,
    ossia che \(\lim a_n = 0\).
\end{snippetsolution}

\begin{snippetexercise}{series-arbitrary-ex-4}{}
    Study the following \series
    \[
        \sum_{n=1}^\infty {(-1)}^n \frac{
            2^n + n
        }{
            3^n + n^2
        }
    \]
\end{snippetexercise}

\begin{snippetsolution}{series-arbitrary-ex-4-sol}{}
    Analizziamo la convergenza assoluta
    \begin{align*}
        \sum_{n=1}^\infty \left|{(-1)}^n \frac{
            2^n + n
        }{
            3^n + n^2
        }\right| &=
        \sum_{n=1}^\infty \frac{
            2^n + n
        }{
            3^n + n^2
        } \\
        &= \sum_{n=1}^\infty
        \frac{
            2^n(1 + \frac{n}{2^n})
        }{
            3^n(1 + \frac{n^2}{2^n})
        } \\
        &= \sum_{n=1}^\infty
        \frac{
            2^n(1 + \littleO(1))
        }{
            3^n(1 + \littleO(1))
        }
    \end{align*}
    Il carattere è quindi il medesimo di
    \begin{align*}
        \sum_{n=1}^\infty {\left(\frac{2}{3}\right)}^n
    \end{align*}
    che converge in quanto serie geometrica convergente.
\end{snippetsolution}

\begin{snippetexercise}{series-arbitrary-ex-5}{}
    Study the following \series
    \[
        \sum_{n=1}^\infty {(-1)}^n \frac{2 + n}{1 + n + n^2}
    \]
\end{snippetexercise}

\begin{snippetsolution}{series-arbitrary-ex-5-sol}{}
    Analizziamo la convergenza assoluta
    \begin{align*}
        \sum_{n=1}^\infty \left| {(-1)}^n \frac{2 + n}{1 + n + n^2} \right|
        &=
        \sum_{n=1}^\infty \frac{2 + n}{1 + n + n^2} \\
        &= \sum_{n=1}^\infty \frac{
            n(1 + \frac{2}{n})
        }{
            n^2(1 + \frac{1}{n^2} + \frac{1}{n})
        } \asymptotic \frac{1}{n}
    \end{align*}
    Allora la serie non converge assolutamente.
    Applichiamo il criterio di Leibniz.
    Notiamo che il termine è definitivamente decrescente
    \begin{align*}
        \frac{2+n}{1+n+n^2} &\geq \frac{3+n}{1+(n+1)+{(n+1)}^2} 
        = \frac{3+n}{3+3n+n^2} \\
        (3+3n+n^2)(2+n) &\geq (3+n)(1+n+n^2) \\
        3+5n + n^2 &\geq 0
    \end{align*}
    Il limite del termine tende a \(0\)
    ed è strettamente positivo. Allora, la serie converge.
\end{snippetsolution}

\begin{snippetexercise}{series-arbitrary-ex-6}{}
    Study the following \series
    \[
        \sum_{n=1}^\infty {(-1)}^n \frac{1}{\sqrt[2024]{n}}
    \]
\end{snippetexercise}

\begin{snippetsolution}{series-arbitrary-ex-6-sol}{}
    \todo
\end{snippetsolution}

\begin{snippetexercise}{series-arbitrary-ex-7}{}
    Study the following \series
    \[
        \sum_{n=2}^\infty {(-1)}^n \frac{1}{n {\left[\log n\right]}^2}
    \]
\end{snippetexercise}

\begin{snippetsolution}{series-arbitrary-ex-7-sol}{}
    \todo
\end{snippetsolution}

\begin{snippetexercise}{series-arbitrary-ex-8}{}
    Study the following \series
    \[
        \sum_{n=1}^\infty {(-1)}^n \frac{1}{n}
        \log\left(\frac{n+1}{n}\right)
    \]
\end{snippetexercise}

\begin{snippetsolution}{series-arbitrary-ex-8-sol}{}
    \todo
\end{snippetsolution}

\begin{snippetexercise}{series-arbitrary-ex-9}{}
    Study the following \series
    \[
        \sum_{n=1}^\infty {(-1)}^n
        \frac{\log(e^n + 1)}{n}
    \]
\end{snippetexercise}

\begin{snippetsolution}{series-arbitrary-ex-9-sol}{}
    \todo
\end{snippetsolution}

\begin{snippetexercise}{series-arbitrary-ex-10}{}
    Study the following \series
    \[
        \sum_{n=1}^\infty {(-1)}^n
        \frac{
            n^2\log\left(1 + \frac{1}{n}\right)
            + {\left(1 + \frac{1}{n^2}\right)}^{n^2}
        }{
            {(\sqrt{n^4+1}-n)}^2 + \frac{e^n}{n\factorial}
        }
    \]
\end{snippetexercise}

\begin{snippetsolution}{series-arbitrary-ex-10-sol}{}
    \todo
\end{snippetsolution}

\end{document}