\documentclass[preview]{standalone}

\usepackage{amsmath}
\usepackage{amssymb}
\usepackage{stellar}
\usepackage{definitions}
\usepackage{bettelini}
\usepackage{tikz}

\begin{document}

\id{mechanics-ex-5}
\genpage

\section{Exercises - Batch 5}

\begin{snippetexercise}{mechanics-ex-5.1}{\underline{5.1}}
    An elevator is rising with a constant acceleration \( A = -0.1g \); inside the elevator, there is an inclined plane with an inclination \( \alpha \) relative to the horizontal and a length \( l \). At the top of the inclined plane, a block of mass \( m \), initially at rest, begins sliding down the plane. Compute the magnitude \( v \) of the relative velocity (with respect to the elevator) of the block when it reaches the bottom of the plane, assuming there is kinetic friction between the block and the plane with coefficient \( \mu_D \).
\end{snippetexercise}

\begin{snippetsolution}{mechanics-ex-5.1-sol}{\underline{5.1}}
    \todo
\end{snippetsolution}

\begin{snippetexercise}{mechanics-ex-5.2}{\underline{5.2}}
    A ball is at rest at the base of an inclined plane with an angle \( \alpha \) relative to the horizontal and height \( h \), mounted on a cart. The cart starts moving with a constant acceleration \( A \) for a time interval \( \tau \), after which it moves with uniform velocity. Determine the values of \( A \) for which the ball, sliding with kinetic friction along the inclined plane, does not reach the top.
\end{snippetexercise}

\begin{snippetsolution}{mechanics-ex-5.2-sol}{\underline{5.2}}
    \todo
\end{snippetsolution}

\begin{snippetexercise}{mechanics-ex-5.3}{\underline{5.3} Rotating tube with ball}
    A rigid cylindrical tube, of negligible cross-section, rotates in a vertical plane with constant angular velocity
    \(\omega\) around a horizontal axis; inside the tube a small ball of mass \(m\) can move without friction.
    At the instant \(t=0\) the tube is in a vertical position, the ball is above the axis of rotation,
    at a distance \(d\) from it and with zero velocity with respect to the tube.
    Study the trajectory of the ball along the tube, approximating the ball to a material particle.
    Discuss the solution in the case
    \(\omega \to 0\) and \(\omega \gg 1\).
\end{snippetexercise}

\begin{snippetsolution}{mechanics-ex-5.3-sol}{\underline{5.3} Rotating tube with ball}
    We choose a reference system with the axes integral to the cylinder (x is the vertical direction of the cylinder).
    The weight force, the constraining reaction and the apparent force (the outward centrifugal force and the Coriolis force)
    act on the ball. The constraining reaction points on the y-axis.
    We therefore have
    \[
        \vec{F}_{CE} = -\vec{\omega} \wedge (\vec{\omega} \wedge \vec{r}), \qquad \vec{F}_{CO} = -2m\vec{\omega} \wedge \vec{v}
    \]
    Since \(\vec{r} = x\hat{x}\), \(\vec{v} = \dot{x}\hat{x}\) and \(\omega = w\hat{z}\),
    we have
    \begin{align*}
        \vec{F}_{CE} &= -\omega^2 x \hat{z} \wedge (\hat{z} \wedge \hat{x}) \\
        &= \omega^2 x \hat{z} \wedge \hat{y} \\
        &= \omega^2 x \hat{x}
    \end{align*}
    and
    \begin{align*}
        \vec{F}_{CO} &= -2m\omega \dot{x} \hat{z} \wedge \vec{x} \\
        &= 2m\omega \dot{x} \hat{y}
    \end{align*}
    Wr thus have the following equations:
    \[
        \begin{cases}
            m\ddot{x} = -mg\cos\theta + m\omega^2 x \\
            m\ddot{y} = -mg\sin\theta + 2m\omega \frac{dx}{dt} + R = 0
        \end{cases}
        \to
        \begin{cases}
            \ddot{x} - \omega^2 x = -g\cos(\omega t) \\
            \ddot{y} = -g\sin(\omega t) + 2\omega \frac{dx}{dt} + R = 0
        \end{cases}
    \]
    This is a non-homogeneous differential equation. The solution
    is given by the linear combination of the solution to the homogeneous equation and a particular
    general solution.
    \[
        x(t) = Ae^{\omega t} + Be^{-\omega t} + \frac{g}{2\omega^2}\cos(\omega t)
    \]
    with the initial conditions we find
    \[
        x(t) = \left(d-\frac{g}{2\omega^2}\right)
        \cosh(\omega t) + \frac{g}{2\omega^2}\cos(\omega t)
    \]
    where \(d\) is the length of the tube from which the ball starts.
    We now study \(\omega \to 0\).
    Note that
    \[
        \cos(\omega t) \asymptotic 1 - \frac{\omega^2 t^2}{2}
    \] and 
    \[
        \cosh(\omega t) \asymptotic 1 + \frac{\omega^2 t^2}{2}
    \]
    Then,
    \begin{align*}
        x(t) &\asymptotic \left(d - \frac{g}{2\omega^2}\right)
        \left(1 + \frac{\omega^2t^2}{2}\right)
        + \frac{g}{2\omega^2}\left(1 - \frac{\omega^2t^2}{2}\right) \\
        &= d - \frac{g}{2}t^2
    \end{align*}
    Thus, if the velocity is very low, the gravitational force is the strongest.
    If \(\omega \gg 1/t\) we have
    \begin{align*}
        x(t) \asymptotic \frac{1}{2}de^{\omega t}
    \end{align*}
    Thus is \(\omega\) is very big, the centrifugal force is the strongest and the ball is shot far away.
\end{snippetsolution}

\begin{snippetexercise}{mechanics-ex-5.4}{\underline{5.4} Falling raindrop}
    Suppose that a raindrop falling through a cloud accumulates mass at a rate \(kmv\),
    where \(k > 0\) is a constant, \(m\) is the mass of the droplet and \(v\) its velocity.
    Find the speed of the drop at a generic time \(t\)
    assuming it starts at rest, also find the mass of the drop at the same time.
\end{snippetexercise}

\begin{snippetsolution}{mechanics-ex-5.4-sol}{\underline{5.4} Falling raindrop}
    We have that
    \[
        \frac{dm(t)}{dt} = km(t)v(t)
    \]
    Since the mass is not constant,
    \[
        F = \frac{dQ}{dt} = \frac{dm}{dt}v + \frac{dv}{dt}m = gm
    \]
    By substituting we get
    \begin{align*}
        gm &= kmv^2 + \frac{dv}{dt} m \\
        g - kv^2&= \frac{dv}{dt}
    \end{align*}
    The equation is separable
    \begin{align*}
        \int \frac{dv}{g-kv^2} &= \int dt \\
        \frac{1}{g} \int \frac{dv}{1 - \frac{k}{g}v^2} &= t + C \\
    \end{align*}
    We use the substitution \(\varphi = \sqrt{\frac{k}{g}}v\) and thus \(d\varphi = \frac{\frac{k}{g}} dv\).
    \begin{align*}
        t + C &= \frac{1}{\sqrt{kg}} \int \frac{d\varphi}{1 - \varphi^2} \\
        t + C &= \frac{1}{\sqrt{kg}} \tanh^{-1}\left(\sqrt{\frac{k}{g}}v\right) \\
        (t + C) \sqrt{kg} &= \tanh^{-1}\left(\sqrt{\frac{k}{g}}v\right) \\
        v(t) &= \sqrt{\frac{g}{k}} \tanh((t + C)\sqrt{kg})
    \end{align*}
    Since \(v(0) = 0\), we have that \(\tanh\left(C\sqrt{kg}\right)\) and thus \(C=0\), so
    \[
        v(t) = \sqrt{\frac{g}{k}} \tanh(t\sqrt{kg})
    \]
    To find the mass, we separate the initial equation
    \begin{align*}
        \int \frac{dm}{m} &= \integral[kv(t)][t] \\
        \ln m &= \integral[k\sqrt{\frac{g}{k}} \tanh(t\sqrt{kg})][t] \\
        \ln m &= \ln(\cosh(t + \sqrt{kg})) + C \\
        m(t) &= m_0 \cosh(t\sqrt{kg})
    \end{align*}
\end{snippetsolution}

\begin{snippetexercise}{mechanics-ex-5.5}{\underline{5.5}}
    A point mass of \( m = 0.2 \, \text{kg} \) is hung from a spring with spring constant \( k = 25 \, \text{N/m} \) attached to a support that is accelerating with \( a = 4 \, \text{m/s}^2 \). Calculate the elongation of the spring.
\end{snippetexercise}

\begin{snippetsolution}{mechanics-ex-5.5-sol}{\underline{5.5}}
    \todo
\end{snippetsolution}

\begin{snippetexercise}{mechanics-ex-5.6}{\underline{5.6}}
    An inclined plane with dimensions \( 3-4-5 \) is fixed on a rotating platform. A block is placed at rest on the plane, and the coefficient of static friction between the block and the plane is \( \mu_s = 1/4 \). The block is initially at a distance of 40 cm from the center of the platform. Find the minimum value of the angular velocity \( \omega \) that prevents the block from sliding off the platform.
    % DISEGNO
\end{snippetexercise}

\begin{snippetsolution}{mechanics-ex-5.6-sol}{\underline{5.6}}
    \todo
\end{snippetsolution}

\end{document}