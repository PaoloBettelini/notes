\documentclass[preview]{standalone}

\usepackage{amsmath}
\usepackage{amssymb}
\usepackage{parskip}
\usepackage{fullpage}
\usepackage{hyperref}
\usepackage{stellar}

\hypersetup{
    colorlinks=true,
    linkcolor=black,
    urlcolor=blue,
    pdftitle={Logic},
    pdfpagemode=FullScreen,
}

\begin{document}

\title{Propositional Logic}
\id{propositional-logic}
\genpage

\section{Definition}

\begin{snippet}{propositional-logic-description}
\textit{Propositional logic}, also called \textit{zeroth-order logic},
is the logic dealing with propositional variables.
\end{snippet}

\includesnpt{logic-propositional-variable-example}

\subsection{Propositional formula}

\includesnpt{logic-propositional-formula}

\end{document}
