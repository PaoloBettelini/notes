\documentclass[preview]{standalone}

\usepackage{amsmath}
\usepackage{amssymb}
\usepackage{stellar}
\usepackage{definitions}

\begin{document}

\id{complexanalysis-complex-roots-exercises}
\genpage

\section{Exercises}

\begin{snippetexercise}{complex-roots-ex-1}{Roots of unity form abelian group}
    Show that for every \(n\), the nth roots of unity \(\{\zeta_n^k\}\)
    for \(k = 0,1,\cdots, n-1\) form an \abeliangroup with respect to multiplication.
\end{snippetexercise}

\begin{snippetsolution}{complex-roots-ex-1-sol}{Roots of unity form abelian group}
    Consider the structure \((\{\zeta_n^k\}, \cdot)\).
    Let \(z_k = \zeta_n^k\).
    \begin{enumerate}
        \item \textbf{Closure:} \(z_k \cdot z_l = z_{l+k}\) and \(z_{l+k \mod n}\) is an nth root of unity.
        \item \textbf{Associativity:} \((z_k \cdot z_l) \cdot z_w = z_l \cdot (z_k \cdot z_w)\) by the properties of exponents.
        \item \textbf{Identity:} \(z_0 \cdot z_k = z_k \cdot 1 = z_k\).
        \item \textbf{Inverse:} \(z_k \cdot z_{n-k} = z_0 = 1\) by the properties of exponents.
        \item \textbf{Commutativity:} \(z_l \cdot z_k = z_k \cdot z_k\) by the properties of exponents.
    \end{enumerate}
\end{snippetsolution}

\end{document}