\documentclass[preview]{standalone}

\usepackage{amsmath}
\usepackage{amssymb}
\usepackage{stellar}
\usepackage{bettelini}

\hypersetup{
    colorlinks=true,
    linkcolor=black,
    urlcolor=blue,
    pdftitle={Stellar},
    pdfpagemode=FullScreen,
}

\begin{document}

\title{Stellar}
\id{italiano-decameron-andreuccio}
\genpage

\section{Analisi}

\begin{snippet}{andreuccio-da-perugia-analisi}
    Il turno di parola viene dato immediatamente alla fine della novella precedente,
    il pezzo di cornice fra le due novelle è quindi assente. 
    
    \textbf{Rubrica:} Andreuccio da Perugia, venuto a Napoli a comperar cavalli, in una notte
    da tre gravi accidenti soprapreso, da tutti scampato con un rubino si torna a
    casa sua.
    
    % mercato, 500 fiorini d'oro, li mostra
    % giovane (prostituta) e vecchia siciliana | la vecchia riconosce adreuccio e spiega alla giovane ocme sono imparentati
    % la giovane torna a casa e mette e fa occupare la vecchia, mentre trama la trappola ad andreuccio
    % Ella manda qualcuno a chiamare andreuccio, dicendo che una donna gli vuole parlare ed è apparecchiato.
    % Il vanitoso andreuccio crede che lei sia innamorata di lui, e si reca in dimora.
    % Questa donna siciliana lo aspetta in cima alla scala per abbracciarlo calorosamente.
    % Piangendo di gioia, gli da il benvenuto e lo porta nella sua camera, la quale è stata preparata per essere ricca piuttosto che di una prostituta.
    % Gli comunica di essere sua sorella spiegando che il padre, Pietro, avrebbe vissuto da giovane a Palermo,
    % e che anreuccio sarebbe nato a Perugia da un'altra donna (sono fratellastri), per poi essere abbandonati dal padre.
    % - re carlo
    % Ella gli fa domande estensive circa i suoi parenti, rafforzando  la credibilità di tutta la questione
    
    % Lo invita a cena, e si offre di avvisare l'albero di Andreuccio di avvisare della sua assenza per restare da lei.
    % Lei fa durare tanto la cena, in maniera tale da far rimanere andreuccio a dormire
    % Da una camera ad Andreuccio con un fanticello a sua disposizione.
    % Si spoglia del farsello (tipico abito maschile) e dei pantaloni, appoggiandoli affianco al letto.
    % chiede al fanciullo dove fosse il bagno, e lui gli mostra l'uscio.
    % L'asse sul quale appoggia il piede è tagliato e casca nella latrina.
    % La donna si affretta a controllare se avesse lasciato i suoi averi in camera.
    % ANdreuccio si ritrova bloccato in strada, in mutande e imbrattato di feci.
    % pronta a rientrare bussando violentemente contro il portone, ma viene ignorato e scacciato dai vicini.
    % Giunge il rozzo e assonnato padrone della prostituta, che scaccia Andreuccio minacciandolo prima che possa finire di spiegarsi.
    
    % Andreuccio perde la speranza di recuperare i suoi fiorini e cerca di tornare al suo albergo
    % ma nella strada svolta per _Runa Catalana_ per andare a lavarsi.
    % Vede due signori con degli arnesi, ma non riesce a nascondersi per il proprio tanfo.
    % Andreuccio gli racconta tutta la storia, e i due capiscono da che casa provenisse.
    % Tuttavia, i due lo rassicurano perché se non fosse caduto sarebbe morto nel sonno.
    % I due gli propongono di andare con loro e ottenere più di quanto noa bbia perso.
    % Sicome disperato, Andreuccio accetta prima di sapere di cosa si tratti.
    
    % Il lavoro è il saccheggiamento della tomba di un vescovo morto recentemente
    % il quale possiede un preziosissimo rubino.
    % Prime di andare, si decidono di lavare Andreuccio in un pozzo.
    % Giunti al pozzo, il secchio manca e quindi legano andreuccio e lo calano nel pozzo.
    % Una volta lavato, la fune si rompe.
    % Arrivano delle due guardie nel pozzo per bere, le quali tirano sù andreuccio pensando fosse il secchio
    % Appena notano le mani di Andreuccio si spaventano e lasciano cadere la fune.
    % Andreuccio incontra nuovamente i due ladri per strada e si dirigono alla chiesa per saccheggiare la tomba.
    % Il coperchio è compsoto da una grande lastra di marmo.
    % Una volta sollevata la lastra, i due vogliono far entrare Andreuccio, il quale si rende conto dei loro intenti.
    % Alchè, una volta entrato, Andreuccio prende immediatamente l'anello e se lo nasconde
    % prima di passare ai ladri tutti gli altri beni sul vescovo.
    % I due ladri tolgono il supporto, facendo rimanere Andreuccio bloccato nella tomba.
    
    % dopo un po' di tempo giungono sul posto altri ladri, i quali aprono la lastra.
    % Andreuccio prende per le gambe il prete che stava entrando. I ladri fuggono lascindo la lastra aperta.
    % Andreuccio se ne va con il rubino molto prezioso.
    
    La vicenda può essere grossolanamente separata in tre avventure:
    quella con la donna siciliana nel chiassetto (§§1-55), quella del pozzo (§§56-70)
    ed infine quella della chiesa (§§71-89).
    Le tre disavventure sono ambientate in luoghi distinti e stretti,
    ma in tutti i posti vi è l'azione di cadere/scendere (latrina, pozzo, arca).
    Ognuna di queste caduta rappresenta una crescita personale per Andreuccio, infatti,
    ogni caduta è progressivamente più difficile da superare. Questo percorso rappresenta il suo
    sviluppo personale, dove impara ad essere più scaltro e astuto.
    Il sistema dei personaggi è costruito con Andreuccio al centro e tutti gli altri come antagonisti.
    Andreuccio è molto ingenuo e privo di esperienza (§§3, §§16) ma anche vanitoso (§§11).
    La prima azione da lui fatta mediante un ragionamento critico è quando si intasca l'anello per fregare i due ladri (§§77),
    questo suo genio è caratterizato dai verbi \quotes{pensò} e \quotes{s'avisò}.
    Prima di allora, Andreuccio aveva sempre agito in maniera completamente passiva o solo per disperazione (§§64).
    \\
    A differenza di Andreuccio, la prostituta è molto intelligente e possiede la capacità di agire in maniera critica e astuta.
    La prima dimostrazione di ciò è quando nota la borsa di Andreuccio senza essere vista (§§4),
    ma la dimostrazione della sua scaltrezza è data dalla sua recita dove convince Andreuccio di essere la sua sorellastra.
    \\
    La novella di Andreuccio ricorda ua fiaba ed un romanzo di formazione
    Questo romanzo di formazione porta Andreuccio a crescere, acquisendo una maggiore astuzia e anche più fortuna di
    quanta non ne avesse all'inizio, per cui Andreuccio acquista il pensiero di un mercante.
    A lui non interessa diventare più saggio o colto, diventa più furbo e usa questa qualità per ottenere un guadagno materiale
    Sia la prostituta che i ladri e Andreuccio guadagno qualcosa di materiale.
    Tutti questi personaggi si arricchiscono in maniera immorale, senza nessun giudizio negativo dalla parte di Boccaccio.
    
    Gli spazi della novella sono posti reali, la storia del protagonista comincia a Perugia,
    si svolge a Napoli per poi tornare nuovamente a Perugia.
    I posti descritti sono fattuali, come l'albergo e la casa siciliana, le vie di napoli citate
    e il mercato dei cavalli.
    Oltre ai posti, anche i riferimenti storici sono veri: Napoli non era infatti una città sicura.
    Il vescovo e la sua morte sono reali, anche se di un'altro tempo. I personaggi presenti nella Novella
    sono anch'essi probabilmente esistiti.
\end{snippet}

\subsection{Inferno, Canto XXIV}

\begin{snippet}{inferno-canto-xxiv-vanni-fucci}
    \href{https://en.wikipedia.org/wiki/Vanni_Fucci}{Vanni Fucci}
    incontra Dante nel canto XXIV. Fucci ha commesso il crimine di
    rubare da una chiesa, per questo soffro in eterno nell'ottavo cerchio dell'Inferno.
    Allo stesso modo, Andreuccio ruba in chiesa e se ne torna a casa con più soldi.
    La differenza è che a Boccaccio non interessano le implicazioni morali delle azioni
    delle persone, bensì considera solo la loro vita terrena.
    Infatti, i personaggi che commettono peccati non vengono necessariamente visti
    in maniera negativa.
    Contrariamente, Dante ritiene che la vita sia solo un piccolo segmento di tempo
    per prepararsi all'oltretomba, e si concentra proprio questo aspetto.
\end{snippet}

\end{document}