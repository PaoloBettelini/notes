\documentclass[preview]{standalone}

\usepackage{amsmath}
\usepackage{amssymb}
\usepackage{stellar}
\usepackage{epigraph}
\usepackage{bettelini}

\hypersetup{
    colorlinks=true,
    linkcolor=black,
    urlcolor=blue,
    pdftitle={Stellar},
    pdfpagemode=FullScreen,
}

\begin{document}

\title{Stellar}
\id{orlando-furioso-introduzione}
\genpage

\section{Introduzione}

\begin{snippet}{orlando-furioso-introduzione}
    L'\textit{Orlando furioso} succede un altro libro,
    \textit{Orlando Innamorato}.
    \\\\
    \textbf{Testimonianza I:} Isabelle d'Este scrive una lettera a Ippolito d'Este,
    ringraziandolo per avendole inviato l'Ariosto, il quale l'ha fatta
    divertire per due giorni.
    \\\\
    \textbf{Testimonianza II (1517):}  Machiavelli scrive una lettera all'amico
    Lodovico Alamanni, dove dice di aver letto il libro e
    che si tratta una bella opera, degna di ammirazione.
    L'unica cosa di cui si lamenta Machiavelli è di non essere presente nella lista di
    poeti apprezzati di Ariosto presente nel libro.
    
    \epigraph{\quotes{Io ho letto a questi dì Orlando Furioso dello Ariosto, e veramente il poema è bello tutto, e in molti luoghi è mirabile. Se lo incontrate raccomandatemi a lui, e ditegli che io mi dolgo solo che, avendo egli ricordato tanti poeti, m'abbia lasciato indietro come un cazzo.}}
    {\textit{Niccolò Machiavelli}}
    
    % 1516      40 canti
    % 1521      40 canti
    % 1525      prose della volgar lingua
    % 1532      46 canti
    
    La metrica è sempre ABABABCC.
    I canti 1-4 compongono il Proemio, quelli dal 5-9 compongono la \quotes{gionta}.
\end{snippet}

\end{document}