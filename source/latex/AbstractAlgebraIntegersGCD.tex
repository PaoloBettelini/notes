\documentclass[preview]{standalone}

\usepackage{amsmath}
\usepackage{amssymb}
\usepackage{stellar}
\usepackage{definitions}

\begin{document}

\id{integers-gcd}
\genpage

\section{Greatest common divisor of multiple integers}

\begin{snippetdefinition}{greatest-common-divisor-definition}{Greater common divisor}
    Let \(a_0, a_1, \cdots, a_n \in \integers\) where \((a_0, a_1, \cdots, a_n) \neq (0,0,0,\cdots, 0)\). \\
    The \textit{greatest common divisors} of \(a_0, a_1, \cdots, a_n\), denoted \(\origgcd(a_0, a_1, \cdots, a_n)\),
    is the greatest integer \(n\) such that \(n \divides a_k\). \\
    In the case where \((a_0, a_1, \cdots, a_n) = (0,0,0,\cdots, 0)\), we define \(\origgcd(a_0, a_1, \cdots, a_n) = 0\).
\end{snippetdefinition}

\plain{This definition is well-defined as the set of greatest common divisors is non-empty
and it is finite. If the numbers are not all zeroes, their common divisors are finite.}

\begin{snippetproposition}{gcd-of-dividind-pair}{}
    If \(a \divides b\) and \(a \neq 0\), \(\gcd(a,b) = |a|\).
\end{snippetproposition}

\begin{snippetproposition}{gcd-of-multiple-of-divisors}{}
    The value \(k = \gcd(a_0, a_1, \cdots, a_n)\) is a multiple of all the divisors of \(a_0, a_1, \cdots\).
\end{snippetproposition}

\begin{snippetproposition}{common-divisor-division-equivalence-property}{}
    Let \(a,b\in\integers\) with \(b>0\) and choose \(q\) and \(r\)
    such that \(a=bq+r\) (with \(0\leq r < b\)).
    Then, the common divisors of \(a\) and \(b\) are precisely the common divisors of
    \(b\) and \(r\).
\end{snippetproposition}

\begin{snippetproof}{common-divisor-division-equivalence-property-proof}{common-divisor-division-equivalence-property}{}
    If \(c\divides a\) and \(c\divides b\), then \(c\divides r = a-bq\).
    Likewise, if \(c \divides b\) and \(c \divides r\) then \(c\divides a = bq+r\)
\end{snippetproof}

\begin{snippetproposition}{greatest-common-divisor-exists-and-unique}{Uniqueness of GCD}
    The greatest common divisors of \(a_0, a_1, \cdots, a_n\) exists and is unique.
\end{snippetproposition}

\begin{snippetproposition}{greatest-common-divisor-multipliers}{}
    Given integers \(a_0, a_1, \cdots, a_n\), there exists integers \(u_k\) such that
    \[
        a_0u_0 + \cdots a_n u_n = \gcd(a_0, a_1, \cdots, a_n)
    \]
\end{snippetproposition}

\begin{snippetproposition}{composite-greatest-common-divisor}{Composite GCD}
    Given integers \(a_0, a_1, \cdots, a_n\)
    for \(n \geq 2\), \[\gcd(\gcd(a_0, \cdots, a_{n-1}), a_n) = \gcd(a_0, \cdots, a_n)\]
\end{snippetproposition}

\begin{snippetproposition}{greatest-common-divisor-scalar-multiplication}{Scalar multiplication of GCD}
    Given integers \(a_0, a_1, \cdots, a_n\) and  an integer \(c\),
    \[\gcd(ca_0, ca_1, \cdots, ca_n) = c \cdot \gcd(a_0, a_1, \cdots, a_n)\]
\end{snippetproposition}

\begin{snippetproof}{greatest-common-divisor-scalar-multiplication-proof}{greatest-common-divisor-scalar-multiplication}{Scalar multiplication of GCD}
    Consider the case \(\gcd(ca, cb) = c\gcd(a,b)\).
    Since \(\gcd(ca, \gcd(cb, cd)) = \gcd(ca, cb, cd)\) we proof will extend to multiple integers.
    Without loss of generality, assume \(a \geq 0\) and \(b \geq 0\) to be integers.
    Consider the trivial case \(c=0\), where \(\gcd(0a, 0b) = \gcd(0,0) = 0\) and \(0\gcd(a,b) = 0\).
    Consider now the case where \(c\neq 0\) and let \(\gcd(ca, cb) = e\) and \(\gcd(a,b) = d\).
    We need to show that \(e=cd\). Note that \(d\divides a\) and \(d \divides b\), and thus \(cd \divides ca\)
    and \(cd \divides cb\). This means that \(cd \divides e\) (i.e there exists \(f\) such that \(e=cdf\)).
    Now note that \(e \divides ca\), so \(cdf \divides ca\), which means
    \(ca = cdfg\) for some \(g \in \integers\). Since we are in the non-trivial case \(c\neq 0\),
    we can simplify as \(a=dfg\), from which \(df \divides a\). Likewise,
    \(df \divides b\). This also means that \(df \divides d\)
    and thus \(cdf \divides cd\) and \(e \divides cd\).
    We previously showed that \(cd\divides e\), so \(e\) and \(cd\) are two naturals
    which divide eachother. They are thus the same.

\end{snippetproof}

\section{Euclidean algorithm}

\begin{snippet}{euclidean-algorithm-expl}
    Euclid's algorithm, is an efficient method for computing the greatest common divisor of two integers
    \(a\) and \(b\) where \(b \geq 0\) and \(a \geq 0\) but \((a,b) \neq (0,0)\).
    If we want to use negative numbers, we just consider their absolute value and
    fix the sign accordingly at the end.

    Consider
    \[
        a = bq + r
    \]
    where \(0 \leq r < b\).

    If \(r=0\), then \(b \divides a\) and thus the greatest common divisor is \(b\).\\
    If \(r > 0\), the common divisors are the common divisors of \(b\) and \(r\).
    We thus divide \(b\) times \(r\).

    The process is iterative.
    For each iteration take the coefficient of the quotient (\(b\)) and divide it by the remainder.
    \begin{align*}
        &a = bq + r, &0 \leq r < b \\
        &b = rq_1 + r_1, &0 \leq r_1 < r \\
        &r = r_1q_2 + r_2, &0 \leq r_2 < r_1 \\
        \phantom{ } &\qquad \vdots & \\
        &r_n = r_{n+1}q_{n+2} + r_{n+1}, &0 \leq r_{n+2} < r_{n+1} \\
        &r_{n+1} = r_{n+2}q_{n+3} + 0&
    \end{align*}
    This sequence is strictly decreasing and will terminate with a null remainder.
    The last remainder \(r_{n+2}\) is then the greatest common divisor between \(a\) and \(b\).
    \begin{align*}
        r_{n+2} &= r_n - r_{n+1}q_{n+2} \\
        &= r_n - (r_{n-1} - r_nq_{n+1})q_{n+2} \\
        &= r_{n-1}(-q_{n+2}) + r_n(1 + q_{n+1}q_{n+2}) \\
        &\cdots 
    \end{align*}
    by continuing this process we get
    \[
        r_{n+2} = au+bv
    \]
    which is a Bézout's identity.
\end{snippet}

\plain{This algorithms also works for polynomials (by restraining the orders of the terms).}

\begin{snippetexample}{euclid-algorithm-example}{Euclid's algorithm}
    Consider \(a=1170\) and \(b=486\).
    We start with
    \(1170 = 482 \cdot 2 + 198\). So, now we consider \(486\) and \(198\).
    \begin{align*}
        1170 &= 482 \cdot 2 + 198 \\
        486 &= 198 \cdot 2 + 90 \\
        198 &= 90 \cdot 2 + 18 \\
        90 &= 18 \cdot 5 + 0
    \end{align*}
    The greatest common divisor is thus \(18\).
    Now we consider
    \begin{align*}
        18 &= 198 - 90\cdot 2 = 198 - (486 - 198 \cdot 2)\cdot 2 \\
        &= 486(-2) + 198(5) = 486(-2) + (1170 - 486\cdot 2) \cdot 5 \\
        &= 1170 \cdot 5 + 486(-12)
    \end{align*}
    which are Bezout's identities.
\end{snippetexample}

\section{Bézout's identity}

\begin{snippettheorem}{bezout-identity}{Bézout's identity}
    Let \(a\) and \(b\) be integers with greatest common divisor \(d\).
    Then, there exists integers \(x\) and \(y\) such that
    \[
        ax+by=d
    \]
    Furthermore, the integers \(az+bt\) are multiples of \(d\).
\end{snippettheorem}

\plain{These identities are not unique.}

\subsection{Coprime numbers}

\begin{snippetdefinition}{coprime-numbers-definition}{Coprime numbers}
    Two integers \(a\) and \(b\) are said to be \textit{coprime}
    if they have no common divisor other than \(1\), meaning that \(\gcd(a,b)=1\).
\end{snippetdefinition}

\begin{snippetproposition}{coprimes-given-by-gcd}{Coprimes given by GCD}
    Let \(d = \gcd(a, b) \neq 0\). Then, the integers \(a'\) and \(b'\) where \(a = da'\) and \(b = db'\)
    are coprime.
\end{snippetproposition}

\begin{snippetproof}{coprimes-given-by-gcd-proof}{coprimes-given-by-gcd}{Coprimes given by GCD}
    Let \(d = \gcd(a, b) \neq 0\). Then \(d = \gcd(da', db') = d\cdot \gcd(a', b') \implies \gcd(a', b') = 1\).
\end{snippetproof}

% TODO pag 35 fundamental theorem of arithmetic

\end{document}