\documentclass[preview]{standalone}

\usepackage{amsmath}
\usepackage{amssymb}
\usepackage{stellar}
\usepackage{bettelini}

\hypersetup{
    colorlinks=true,
    linkcolor=black,
    urlcolor=blue,
    pdftitle={Biologia},
    pdfpagemode=FullScreen,
}

\begin{document}

\id{biologia-riproduzione-esseri-viventi}
\genpage

\section{La riproduzione degli esseri viventi}

\begin{snippetdefinition}{riproduzione-definition}{Riproduzione}
    La \textit{riproduzione} è un processo per il quale viene data origine ad uno o più organismi
    discendenti.
\end{snippetdefinition}

\begin{snippetdefinition}{riproduzione-asessuata-definition}{Riproduzione asessuata}
    La \textit{riproduzione asessuata} non coinvolge l'impiego di cellule specializzate (gameti),
    e consiste nel clonarsi in maniera identica.
\end{snippetdefinition}

\plain{La riproduzione asessuata è veloce, non necessita un secondo individuo ma
produce poca variabilità genetica.}

\begin{snippetdefinition}{riproduzione-sessuata-definition}{Riproduzione sessuata}
    La \textit{riproduzione sessuata} coinvolge due individui di sesso opposto per produrre un discendente
    con una varietà genetica.
\end{snippetdefinition}

\begin{snippet}{c4838455-8714-43fc-98da-32cab553db17}
    La riproduzione sessuata si avvale di mutazioni genetiche.
    Essa costa molta più energia, è lenta, richiede due individui di sesso opposto ma
    genera una grande diversità genetica.
\end{snippet}

\end{document}