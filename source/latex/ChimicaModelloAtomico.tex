\documentclass[preview]{standalone}

\usepackage{amsmath}
\usepackage{amssymb}
\usepackage{stellar}

\hypersetup{
    colorlinks=true,
    linkcolor=black,
    urlcolor=blue,
    pdftitle={Stellar},
    pdfpagemode=FullScreen,
}

\begin{document}

\id{chimica-modello-atomico}
\genpage

\section{Modello atomico}

\includesnpt[src=/snippet/static/modelli-atomici.png|width=75\%]{centered-img}

\begin{snippet}{modello-quantomeccanico-expl}
    Secondo il modello quantomeccanico, le posizioni degli elettroni
    attorno al nucleo sono date da una funzione d'onda che modella la densità
    probabilistica \({|\Psi(x)|}^2\) di misurare un elettrone in una data posizione \(x\).
\end{snippet}

\begin{snippetdefinition}{ione-definition}{Ione}
    Gli \textit{ioni} si formano quando un atomo perde o guadagna uno o più elettroni.
    Questo fenomeno viene chiamato ionizzazione e gli atomi che la subiscono diventano
    particelle cariche.
    Gli ioni che acquisiscono elettroni, quindi presentano una carica negativa, sono chiamati
    \textit{anioni}.
    Al contrario, gli atomi che perdono elettroni, quindi presentano
    una carica negativa, sono definiti \textit{cationi}.
\end{snippetdefinition}

\begin{snippetdefinition}{isotopo-definition}{Isotopo}
    Un \textit{isotopo} è un atomo, di un qualunque elemento chimico, che mantiene
    lo stesso numero atomico ma differente numero di massa e di conseguenza differente massa atomica.
    La differenza nel numero di massa è dovuta al differente numero di neutroni che sono presenti nel
    nucleo di atomi aventi lo stesso numero atomico.
\end{snippetdefinition}

\end{document}