\documentclass[preview]{standalone}

\usepackage{amsmath}
\usepackage{amssymb}
\usepackage{stellar}
\usepackage{bettelini}

\hypersetup{
    colorlinks=true,
    linkcolor=black,
    urlcolor=blue,
    pdftitle={Stellar},
    pdfpagemode=FullScreen,
}

\begin{document}

\id{italiano-canzoniere-rvf-267}
\genpage

\section{Rvf 267: Oimè il bel viso, oimè il soave sguardo}

\begin{snippet}{canzoniere-rvf-267}
    \StellarPoetry{1}{
        \textbf{Oimè} il bel viso, \textbf{oimè} il soave sguardo, \\
        \textbf{oimè} il leggiadro portamento altero; \\
        \textbf{oimè} il parlar ch'ogni aspro ingegno et fero \\
        facevi humile, ed ogni huom vil gagliardo!
    }{XXX}
    \StellarPoetry{2}{
        et \textbf{oimè} il dolce riso, onde uscìo 'l dardo \\
        di che morte, altro bene omai non spero: \\
        alma real, dignissima d'impero, \\
        se non fossi fra noi scesa sì tardo!
    }{XXX}
    \StellarPoetry{3}{
        Per voi conven ch'io arda, e 'n voi respire, \\
        ch'i' pur \textbf{fui} vostro; et se di voi son privo, \\
        via men d'ogni sventura altra mi dole. % non vi può essere una sventura peggiore
    }{XXX}
    \StellarPoetry{4}{
        Di speranza m'empieste et di desire, \\
        quand'io partì' dal sommo piacer vivo; \\
        ma 'l vento ne portava le parole.
    }{XXX}

    Questo è il primo sonetto in cui Petrarca parla (implicitamente) di Laura morta.
    I primi elementi che fanno percepire il lutto sono i lamenti \quotes{oimè}.
    Il \quotes{fui vostro} è al passato remoto e indica la fine di questa situazione,

    L'autore lascia Laura con la sicurezza che l'avrebbe rivista, ma ciò non succede.
\end{snippet}

\end{document}