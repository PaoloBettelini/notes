\documentclass[preview]{standalone}

\usepackage{amsmath}
\usepackage{amssymb}
\usepackage{stellar}
\usepackage{bettelini}

\hypersetup{
    colorlinks=true,
    linkcolor=black,
    urlcolor=blue,
    pdftitle={Stellar},
    pdfpagemode=FullScreen,
}

\begin{document}

\title{Stellar}
\id{orlando-furioso-xix-analisi}
\genpage

\section{Analisi}

\begin{snippet}{orlando-furioso-xix-analisi}
    Tutto l'episodio è diviso in due,
    abbiamo la storia di Cloridano e Medoro (episodio epico e di guerra), ottave 1-16, 
    ed un episodio amoroso (Medoro e Angelica), ottave 16-42.
    \\\\
    All'ottava 17 si nota il cambiamento e passaggio tra i due episodi.
    Si passa in modo armonioso tra l'uno e l'altro.
    Cambia da un argomento all'altro per mantenere vivo l'interesse.
    \\\\
    Viene richiamata una delle storie d'amore più famose di tutta la letteratura mondiale,
    ossia quella dell'Eneide (35-37).
    La vicenda di Angelica viene descritta come l'amore di Petrarca, riprendendo sintagmi
    e le rime in maniera verbatim.
    L'elemento anormale è che i ruoli sono invertiti (rovesciamento); la donna
    fa ciò che farebbe l'uomo. Inoltre, il punto di prospettiva è quello della donna,
    lei è la protagonista.
    \\\\
    \textbf{Un amore normale con tratti di follia:}
    l'amore tra Angelica e Medoro hanno caratteristiche normali ma con tratti discrepanti.
    Tratti Tratti che rendono letterariamente il loro amore normale sono:
    \begin{itemize}
        \item (20.1) Il loro amore inizia con lo sguardo, innamoramento che passa dagli occhi di lei
        \item Il sentimento crea un dolore tremendo, una sofferenza data dal desiderio che porta anche
        alla morte (30.1)
        \item (30.7) Angelica se non vuol morire si dichiara
        \item (35.7) Viene richiamata la Grotta nella quale si innamorarono Enea e Didone, una delle opere più grandi letterali (Eneide, Virgilio).
        \item Il narratore riprende alla lettera i sintagmi de Il Canzoniere di Petrarca
    \end{itemize}
    Tratti che rendono il loro amore anormale:
    \begin{itemize}
        \item I ruoli sono rovesciati, invertiti tra uomo e donna
        \item Il punto di vista è sempre e solo di Angelica (della donna), non sappiamo mai cosa ne pensa Medoro del loro amore (sempre rovesciamento)
        \item Lei è una principessa che si innamora di un umile soldato, nonostante non le mancassero delle altissime alternative tra cui innamorarsi
        \item La principessa rifiuta l'amore dei parigrado e si unisce in matrimonio col più umile dei soldati
    \end{itemize}

    \textbf{L'intervento dell'autore al centro dell'episodio di Angelica e Medoro (31-32):}
    L'autore si rivolge a Orlando e a Sacripante alla ottava 31, mentre alla 32 si rivolge a Agricane e
    Ferraù e a mille altri grandi cavalieri che hanno fatto di tutto per Angelica e non hanno ricevuto
    niente in cambio. \\
    Il campo semantico delle due ottave attraversano il codice cortese, dell'onore cavalleresco.
    Agli ultimi due versi dell'ottava 32 si legge cosa pensa il narratore di Angelica; un'ingrata.
    \\
    Questo intervento serve a far capire la spaccatura tra il cavaliere e la donna:
    \begin{itemize}
        \item \textbf{Angelica idealizzata:} Una principessa bellissima da amare e fare di tutto per lei così da
        averla per loro per sempre, non accettando che la donna sia come non si vuole che sia.
        \item \textbf{Angelica reale:} Una persona ingrata, una sfruttatrice e priva di sentimenti
    \end{itemize}
    Si nota come tutti i cavalieri abbiano dei tratti di follia, nessuno si ferma mai a riflettere un
    attimo e nessuno impara dalle esperienze. Ciò anche fa capire che non è veramente Orlando
    l'unico pazzo dell'opera. L'unico ad accorgersene sarà Orlando.
    \\\\
    La donna amante: il rovesciamento dei ruoli e i suoi significati
    \begin{itemize}
        \item \textbf{Significato di libertà della donna:} Lo stato si laicizza, è figlio dei tempi che cambiano e le
        donne iniziano ad avere un ruolo nella società e quindi una lieve libertà, sicuramente
        maggiore di quella medievale
        \item \textbf{Significato letterale:} Ariosto si diverte a rovesciare alcuni schemi, segue tutta la tradizione
        dello Stil novo di Petrarca e altri, ma poi li rovescia per parodizzarli
        \item \textbf{Significato interno al libro:} Vedere cosa ne pensa lei della storia, la smaschera agli occhi del
        lettore, facendo capire com'è realmente l'Angelica reale.
        Da un \quotes{motore trasparente} durante l'innamoramento da \quotes{oggetto del desiderio passivo},
        viene resa il soggetto della storia e dunque un personaggio attivo.
    \end{itemize}
\end{snippet}

\end{document}