\documentclass[preview]{standalone}

\usepackage{amsmath}
\usepackage{amssymb}
\usepackage{bettelini}
\usepackage{stellar}

\hypersetup{
    colorlinks=true,
    linkcolor=black,
    urlcolor=blue,
    pdftitle={Chimica},
    pdfpagemode=FullScreen,
}

\begin{document}

\id{chimica-sistema-inernazionale}
\genpage

\section{Grandezze fondamentali}

\begin{snippet}{chimica-grandezze-fondamentali}
    \phantom{}$$
    \begin{array}{|c|l|c|c|l|c|}
    \hline \text { Sottomultiplo } & \text { Prefisso } & \text { Simbolo } & \text { Multiplo } & \text { Prefisso } & \text { Simbolo } \\
    \hline 10^{-1} & \text { deci- } & \mathrm{d}- & 10 & \text { deca- } & \text { da- } \\
    \hline 10^{-2} & \text { centi- } & \mathrm{c}- & 10^2 & \text { etto- } & \mathrm{h}- \\
    \hline 10^{-3} & \text { milli- } & \mathrm{m}- & 10^3 & \text { kilo- } & \mathrm{k}- \\
    \hline 10^{-6} & \text { micro- } & \mu- & 10^6 & \text { mega- } & \mathrm{M}- \\
    \hline 10^{-9} & \text { nano- } & \mathrm{n}- & 10^9 & \text { giga- } & \mathrm{G}- \\
    \hline 10^{-12} & \text { pico- } & \mathrm{p}- & 10^{12} & \text { tera- } & \mathrm{T}- \\
    \hline
    \end{array}
    $$
\end{snippet}

\section{Grandezze derivate}

\begin{snippet}{chimica-grandezze-derivate}
    \phantom{}$$
    \begin{array}{|l|l|c|c|}
    \hline \begin{array}{l}
    \text { Grandezza } \\
    \text { fisica }
    \end{array} & \text { Nome dell'unità di misura } & \begin{array}{c}
    \text { Simbolo } \\
    \text { dell'unità } \\
    \text { di misura }
    \end{array} & \begin{array}{c}
    \text { Definizione } \\
    \text { dell'unità } \\
    \text { di misura SI }
    \end{array} \\
    \hline \text { Area } & \text { metro quadrato } & \mathrm{m}^2 & \\
    \hline \text { Volume } & \text { metro cubo } & \mathrm{m}^3 & \\
    \hline \text { Densità } & \text { kilogrammo al metro cubo } & \mathrm{kg} / \mathrm{m}^3 & \\
    \hline \text { Forza } & \text { newton } & \mathrm{N} & \mathrm{N}=\mathrm{kg} \cdot \mathrm{m} / \mathrm{s}^2 \\
    \hline \text { Pressione } & \text { pascal } & \mathrm{Pa} & \mathrm{Pa}=\mathrm{N} / \mathrm{m}^2 \\
    \hline \text { Energia, lavoro, calore } & \text { joule } & \mathrm{J} & \mathrm{J}=\mathrm{N} \cdot \mathrm{m} \\
    \hline \text { Velocità } & \text { metri al secondo } & \mathrm{m} / \mathrm{s} & \\
    \hline
    \end{array}
    $$
\end{snippet}

\section{Misure}

\begin{snippet}{chimica-misure}
    \phantom{}$$
    \begin{array}{|l|c|l|c|}
    \hline \begin{array}{l}
    \text { Grandezza } \\
    \text { fisica }
    \end{array} & \begin{array}{c}
    \text { Simbolo della } \\
    \text { grandezza fisica }
    \end{array} & \begin{array}{l}
    \text { Nome dell'unità } \\
    \text { di misura }
    \end{array} & \begin{array}{c}
    \text { Simbolo dell'unità } \\
    \text { di misura }
    \end{array} \\
    \hline \text { Lunghezza } & l & \text { metro } & \mathrm{m} \\
    \hline \text { Massa } & m & \text { kilogrammo } & \mathrm{kg} \\
    \hline \text { Tempo } & t & \text { secondo } & \mathrm{s} \\
    \hline \text { Corrente elettrica } & I & \text { ampere } & \mathrm{A} \\
    \hline \text { Temperatura } & T & \text { kelvin } & \mathrm{K} \\
    \hline \text { Quantità di sostanza } & n & \text { mole } & \mathrm{mol} \\
    \hline \text { Intensità luminosa } & i_v & \text { candela } & \mathrm{cd} \\
    \hline
    \end{array}
    $$
\end{snippet}

\end{document}
