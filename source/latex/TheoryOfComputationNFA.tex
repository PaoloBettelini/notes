\documentclass[preview]{standalone}

\usepackage{amsmath}
\usepackage{amssymb}
\usepackage{tikz}
\usepackage{wrapfig}
\usepackage{bettelini}
\usepackage{stellar}
\usepackage{definitions}

% =======
\usetikzlibrary{ % tikz packages
    automata,positioning,
    arrows.meta,bending
}
\tikzset{every state/.style={
    inner sep=2pt,
    minimum size=4pt
}}
\tikzset{>=stealth}  %latex, to, stealth
% Empty string symbol.
\newcommand{\emptyString}{\lambda}
% =======

\begin{document}

\id{theoryofcomputation-nfa}
\genpage

\section{Nondeterministic Finite Automaton}

\begin{snippetdefinition}{nfa-definition}{Nondeterministic Finite Automaton}
    Nondeterministic finite automata (NFA) are state-machines
    like DFAs but can change multiple states at a time by processing
    empty strings \(\emptyString\) and when processing a symbol
    \(a\) may have multiple possible states to switch to.
    The NFA will choose the "correct" switch in order to end in an accept state, if possible.

    A NFA is defined as \(M=(Q, \Sigma, \delta, q, F)\) where
    \begin{itemize}
        \item \(Q\) is a finite set of \textit{states}
        \item \(\Sigma\) is an alphabet
        \item \(\delta \colon Q \cartesianprod \Sigma_\emptyString \fromto \powerset(Q)\) is the \textit{transition function}
        \item \(q\) is an element of \(Q\) called the \textit{start state}
        \item \(F\) is a subset of \(Q\) which contains the \textit{accept states}
    \end{itemize}
\end{snippetdefinition}

\begin{snippetexample}{nfa-multiple-of-2-of-3}{NFA example}
    The following automaton where \(\Sigma = \{1\}\) will end in an accept state
    if the input has length which is a multiple of \(2\) or \(3\).
    
    \newcommand\double[3][10]{%
      \draw (#2)
        edge [bend left=#1,draw=none]
        coordinate[at start](#2-#3-s)
        coordinate[at end](#2-#3-e)
        (#3)
        edge [bend right=#1,draw=none]
        coordinate[at start](#3-#2-e)
        coordinate[at end](#3-#2-s)
        (#3);
    }
    
    \begin{center}
        \begin{tikzpicture}[node distance=2cm,on grid,auto]
            \node[state] (0) {};
            \node (inv) [left=of 0] {};
    
            \node[state, accepting] (a1) [above right=of 0] {\(q_{a1}\)};
            \node[state] (a2) [right=of a1] {\(q_{a2}\)};
            
            \node[state, accepting] (b1) [below right=of 0] {\(q_{b1}\)};
            \node[state] (b2) [right=of b1] {\(q_{b2}\)};
            \node[state] (b3) [below=of b2] {\(q_{b3}\)};
            
            \double{a1}{a2};
    
            \path[->]
                (inv)
                    edge node {} (0)
                (0)
                    edge node {\(\emptyString\)} (a1)
                    edge node {\(\emptyString\)} (b1)
                (a1-a2-s)
                    edge node {\(1\)} (a1-a2-e)
                (a2-a1-s)
                    edge node {\(1\)} (a2-a1-e)
                (b1)
                    edge node {\(1\)} (b2)
                (b2)
                    edge node {\(1\)} (b3)
                (b3)
                    edge node {\(1\)} (b1);
        \end{tikzpicture}
    \end{center}
    The first switch is done by processing an empty string and the direction is chosen magically
    in order to end in an accept state.
\end{snippetexample}

\end{document}