\documentclass[preview]{standalone}

\usepackage{amsmath}
\usepackage{amssymb}
\usepackage{stellar}
\usepackage{chronology}

\hypersetup{
    colorlinks=true,
    linkcolor=black,
    urlcolor=blue,
    pdftitle={Stellar},
    pdfpagemode=FullScreen,
}

\begin{document}

\title{Stellar}
\id{storia-linea-temporale}
\genpage

\section{Linea temporale}

\begin{snippet}{storia-linea-temporale}
    \begin{chronology}[250]{-1251}{2010}{\textwidth}
        \event{-1250}{Caduta di Troia}
        \event{-753}{Romolo Re di Roma}
        \event{1}{Nascita di Gesù}
        \event{622}{Egira}
        \event{800}{Carlo Magno Imperatore}
        % rinascimento
        %\event[1450]{1700}{Rinascimento}
        \event{1517}{Riforma protestante}
        % illuminismo
        \event{1789}{Rivoluzione francese}
        % romanticismo
        \event{1922}{Marcia su Roma}
    \end{chronology}
    \begin{chronology}*[500]{-3000}{2010}{\textwidth}
        \event{476}{Crollo Impero Romano d'Occidente}
        \event{1453}{Crollo Impero Romano d'Oriente}
        \event[-3000]{476}{Età antica}
        \event[476]{1492}{Medioevo}
        \event[1492]{1789}{Età moderna}
        \event[1789]{2010}{Età contemporanea}
    \end{chronology}
\end{snippet}

% preistoria - fino a -3000

\begin{snippetdefinition}{Egira-definition}{Egira}
    Con \textit{egira} si intende l'abbandono della Mecca da parte di Maometto
    e il suo trasferimento a Medina,
    nel settembre del 622 d.C. (anno iniziale della cronologia islamica).
\end{snippetdefinition}

\end{document}