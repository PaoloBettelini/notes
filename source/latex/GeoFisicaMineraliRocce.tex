\documentclass[preview]{standalone}

\usepackage{amsmath}
\usepackage{amssymb}
\usepackage{tikz}
\usepackage{stellar}
\usepackage{bettelini}

\hypersetup{
    colorlinks=true,
    linkcolor=black,
    urlcolor=blue,
    pdftitle={Assets},
    pdfpagemode=FullScreen,
}

\begin{document}

\id{geofisica-minerali-e-rocce}
\genpage

\section{Minerali e rocce}

\begin{snippetdefinition}{minerale}{Minerale}
    I \textit{minerali} sono solidi cristallini inorganici, caratterizzati da composizione chimica ben definita.
\end{snippetdefinition}

\begin{snippetdefinition}{rocce}{Roccia}
    Le \textit{rocce} sono aggregati di uno o più minerali.
\end{snippetdefinition}

\begin{snippet}{processi-rocce}
    Le rocce si formano attraverso tre processi:
    \begin{itemize}
        \item \textbf{processo magmatico}, legato alla solidificazione di un magma;
        \item \textbf{processo sedimentario}, che avviene sulla superficie terrestre;
        \item \textbf{processo metamorfico}, che avviene all'interno della crosta terrestre.
    \end{itemize}
\end{snippet}

\section{Il processo magmatico}

\plain{
    Il <b>processo magmatico</b> inizia dalla solidificazione del magma
nell'astenosfera.
Il magma tende a risalire verso la superficie terrestre poiché ha una temperatura più elevata,
superiore a 1000°C, e una densità minore rispetto ai materiali rocciosi della litosfera.
Durante la risalita, il magma fonde parzialmente le rocce che incontra nel suo percorso e si
arricchisce di nuove sostanze chimiche.
Il magma in risalita può andare incontro a due diversi destini: fuoriuscire in superficie
o restare intrappolato nella crosta terrestre.
}

\begin{snippetdefinition}{rocce-magmatiche-effusive}{Rocce magmatiche effusive}
    Le \textit{rocce magmatiche effusive} sono rocce magmatiche generate dal rapido raffreddamento
    della lava in una eruzione effusiva.
\end{snippetdefinition}

\begin{snippetexample}{rocce-magmatiche-effusive-ex}{Rocce magmatiche effusive}
    Pietra pomice e ossidiana
\end{snippetexample}

\begin{snippetdefinition}{rocce-magmatiche-intrusive}{Rocce magmatiche intrusive}
    Le \textit{rocce magmatiche intrusive} sono rocce magmatiche generate
    dal magma che viene intrappolato fra le rocce in bolle o filoni magmatici
    e che si raffredda.
\end{snippetdefinition}

\plain{Il raffreddamento delle rocce magmatiche intrusive può durante millenni,
portando un'aspetto diverso alle rocce.}

\begin{snippetexample}{rocce-magmatiche-intrusive-ex}{Rocce magmatiche intrusive}
    Alcuni di esempi di rocce magmatiche intrusive sono il granito, la diorite
    e il gabbro.
\end{snippetexample}

\section{Il processo sedimentario}

\plain{
    Le rocce sulla superficie terrestre sono esposte all'azione degli agenti esogeni
dell'atmosfera, come le pioggie, e dell'idrosfera, come l'azione dei fiumi,
e anche della biosfera, poiché alcuni organismi come funghi, batteri o piante possono disgregare le rocce.
Tale interazione può durare milioni di anni e può essere descritta in diverse tappe,
quali <b>degradazione</b>, <b>trasporto</b>, <b>sedimentazione</b> e <b>diagenesi</b>.
}

\begin{snippetdefinition}{rocce-sedimentari}{Rocce sedimentarie}
    Le \textit{rocce sedimentarie} si formano spesso in una serie
    di strati e sono un tipo di rocce formate dall'accumulo
    di sedimenti di varia origine,
    derivanti in gran parte dalla degradazione e dall'erosione di rocce preesistenti,
    che si sono depositati sulla superficie terrestre. 
\end{snippetdefinition}

\section{Il processo metamorfico}

\plain{
    Le rocce superficiali, siano esse magmatiche o sedimentarie, a causa di imponenti
movimento della crosta, possono essere trasportate in profondità, dove incontrano
temperature e pressioni molto più elevate di quelle da cui provengono, e si
trasformano in <b>rocce metamorfiche</b>.
Il metamorfismo consiste nella trasformazione della struttura
o della composizione mineralogica delle rocce ed è causato dall'aumento della pressione e/o temperatura.
Le rocce metamorfiche costituiscono generalmente la parte più profonda dei continenti e le zone centrali
di molte catene montuose.
}

\begin{snippetdefinition}{rocce-metamorfiche-scistose}{Rocce metamorfiche scistose}
    Le \textit{rocce metamorfiche scistose} sono delle rocce metamorfiche
    che vengono stirate da movimento della crosta.
\end{snippetdefinition}

\begin{snippetdefinition}{rocce-metamorfiche-contatto}{Rocce metamorfiche di contatto}
    Le \textit{rocce metamorfiche di contatto} sono delle rocce metamorfiche
    che vengono trasformate dal calore del magma che si trova nelle vicinanze.
\end{snippetdefinition}

\begin{snippetexample}{rocce-metamorfiche-contatto-ex}{Rocce metamorfiche di contatto}
    Il calcare si trasforma in marmo e il granito in gneiss.
\end{snippetexample}

\section{Trasformazioni}

\includesnpt[width=85\%|src=/snippet/static/trasformazioni-rocce.png]{centered-img}

\includesnpt[width=85\%|src=/snippet/static/ciclo-litogenetico.png]{centered-img}

\end{document}