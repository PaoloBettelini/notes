\documentclass[preview]{standalone}

\usepackage{amsmath}
\usepackage{amssymb}
\usepackage{tikz}
\usepackage{stellar}
\usepackage{bettelini}

\hypersetup{
    colorlinks=true,
    linkcolor=black,
    urlcolor=blue,
    pdftitle={Stellar},
    pdfpagemode=FullScreen,
}

\begin{document}

\id{geofisica-biocapacita}
\genpage

\section{Biocapacità}

\begin{snippetdefinition}{biocapacita}{Biocapacità}
    La \textit{biocapacità} è la produttività biologica di una superficie. È
    l'insieme dei servizi ecologici erogati dagli ecosistemi locali, stimata
    attraverso la quantificazione della superficie dei terreni
    ecologicamente produttivi che sono presenti all'interno della regione
    in esame.
\end{snippetdefinition}

\plain{La biocapacità non dipende dalle sole condizioni naturali.
La biocapacità aumenta quando sale la produttività per unità di
superficie o si ingrandiscono le superfici produttive.}

\begin{snippet}{impronta-ecologica}
    L'\textit{impronta ecologica} è una sorta di contabilità delle risorse.
    \begin{itemize}
        \item Essa rileva quale parte della capacità rigenerativa
            dell'ambiente è sollecitata dall'essere umano.
        \item Il metodo converte l'intensità delle utilizzazioni e dei
            carichi esercitati sulla natura, quali la campicoltura, la
            produzione di fibre vegetali o l'assorbimento di CO 2 , in
            equivalenti di superficie necessari per produrre queste
            risorse in modo rinnovabile o per assorbire le emissioni.
        \item L'impronta ecologica esprime la totalità dei consumi, di qualunque
            genere, in superficie richiesta, chiamata ettaro globale, e mostra
            in quale misura l'utilizzazione fatta della natura supera o no la sua
            capacità di rigenerazione della biosfera (biocapacità). 
        \item In tal modo, un'utilizzazione delle risorse naturali è sostenibile
            fintanto che l'impronta ecologica non superi la biocapacità.
    \end{itemize}
\end{snippet}

\begin{snippetdefinition}{superamento-terrestre}{Giorno del Superamento Terrestre}
    Il \textit{giorno del superamento terrestre} (Earth Overshoot Day)
    indica a livello illustrativo il giorno dell'anno nel quale l'umanità
    consuma interamente le risorse prodotte dal pianeta nell'intero anno stesso. 
\end{snippetdefinition}

\plain{La costituzione federale imposta i principi per la pianificazione del territorio,
ma le leggi concrete sono delegate ai cantoni.}

\end{document}