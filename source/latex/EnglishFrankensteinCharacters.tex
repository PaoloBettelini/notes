\documentclass[preview]{standalone}

\usepackage{amsmath}
\usepackage{amssymb}
\usepackage{bettelini}
\usepackage{stellar}

\hypersetup{
    colorlinks=true,
    linkcolor=black,
    urlcolor=blue,
    pdftitle={English},
    pdfpagemode=FullScreen,
}

\begin{document}

\title{English}
\id{english-frankenstein-characters}
\genpage

\begin{snippetcharacter}{frankenstein-victor-frankenstein}{Victor Frankenstein}
    The doomed protagonist and narrator of the main portion of the story. Studying in Ingolstadt, Victor discovers the secret of life and creates an intelligent but grotesque monster, from whom he recoils in horror. Victor keeps his creation of the monster a secret, feeling increasingly guilty and ashamed as he realizes how helpless he is to prevent the monster from ruining his life and the lives of others.
\end{snippetcharacter}

\begin{snippetcharacter}{frankenstein-the-monster}{The Monster}
    The eight-foot-tall, hideously ugly creation of Victor Frankenstein. Intelligent, eloquent, and sensitive, the Monster attempts to integrate himself into human social patterns, but all who see him shun him. His feeling of abandonment compels him to seek revenge against his creator.
\end{snippetcharacter}

\begin{snippetcharacter}{frankenstein-robert-walton}{Robert Walton}
    The Arctic seafarer whose letters open and close Frankenstein. Walton picks the bedraggled Victor Frankenstein up off the ice, helps nurse him back to health, and hears Victor's story. He records the incredible tale in a series of letters addressed to his sister, Margaret Saville, in England.
\end{snippetcharacter}

\begin{snippetcharacter}{frankenstein-elizabth-lavenza}{Elizabeth Lavenza}
    An orphan, four to five years younger than Victor, whom the Frankensteins adopt. In the 1818 edition of the novel, Elizabeth is Victor's cousin, the child of Alphonse Frankenstein's sister. In the 1831 edition, Victor's mother rescues Elizabeth from a destitute peasant cottage in Italy. Elizabeth embodies the novel's motif of passive women, as she waits patiently for Victor's attention.
\end{snippetcharacter}

\begin{snippetcharacter}{frankenstein-henry-clerval}{Henry Clerval}
    Victor's boyhood friend, who nurses Victor back to health in Ingolstadt. After working unhappily for his father, Henry begins to follow in Victor's footsteps as a scientist. His cheerfulness counters Victor's moroseness.
\end{snippetcharacter}

\begin{snippetcharacter}{frankenstein-alphonse-frankenstein}{Alphonse Frankenstein}
    Victor's father, very sympathetic toward his son. Alphonse consoles Victor in moments of pain and encourages him to remember the importance of family.
\end{snippetcharacter}

\begin{snippetcharacter}{frankenstein-william-frankenstein}{William Frankenstein}
    Victor's youngest brother and the darling of the Frankenstein family. The monster strangles William in the woods outside Geneva in order to hurt Victor for abandoning him. William's death deeply saddens Victor and burdens him with tremendous guilt about having created the monster.
\end{snippetcharacter}

\begin{snippetcharacter}{frankenstein-justine-moritz}{Justine Moritz}
    A young girl adopted into the Frankenstein household while Victor is growing up. Justine is blamed and executed for William's murder, which is actually committed by the monster.
\end{snippetcharacter}

\begin{snippetcharacter}{frankenstein-beaufort}{Beaufort}
    A merchant and friend of Victor's father; the father of Caroline Beaufort.
\end{snippetcharacter}

\begin{snippetcharacter}{frankenstein-peasants}{Peasants}
    A family of peasants, including a blind old man, De Lacey; his son and daughter, Felix and Agatha; and a foreign woman named Safie. The monster learns how to speak and interact by observing them. When he reveals himself to them, hoping for friendship, they beat him and chase him away.
\end{snippetcharacter}

\begin{snippetcharacter}{frankenstein-m-waldman}{M. Waldman}
    The professor of chemistry who sparks Victor's interest in science. He dismisses the alchemists' conclusions as unfounded but sympathizes with Victor's interest in a science that can explain the “big questions,” such as the origin of life.
\end{snippetcharacter}

\begin{snippetcharacter}{frankenstein-m-krempe}{M. Krempe}
    A professor of natural philosophy at Ingolstadt. He dismisses Victor's study of the alchemists as wasted time and encourages him to begin his studies anew.
\end{snippetcharacter}

\begin{snippetcharacter}{frankenstein-mr-kirwin}{Mr. Kirwin}
    The magistrate who accuses Victor of Henry's murder.
\end{snippetcharacter}

% https://www.sparknotes.com/lit/frankenstein/characters/

\end{document}
