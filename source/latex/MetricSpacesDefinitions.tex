\documentclass[preview]{standalone}

\usepackage{amsmath}
\usepackage{amssymb}
\usepackage{bettelini}
\usepackage{stellar}
\usepackage{definitions}

\begin{document}

\id{metric-spaces-definitions}
\genpage

\section{Definitions}

\begin{snippetdefinition}{distance-function-definition}{Distance Function}
    Let \(X\) be a set.
    A \textit{distance function} on \(X\) is
    a \function \(d \colon X \cartesianprod X \fromto \realnumbers\)
    with the following properties for \(x,y,z \in X\):
    \begin{itemize}
        \item \(d(x,y) = 0 \iff x = y\).
        \item \textbf{Positivity:} \(d(x,y) > 0 \iff x \neq y\).
        \item \textbf{Simmetry:} \(d(x,y) = d(x,y)\).
        \item \textbf{Triangle inequality:} \(d(x,z) \leq d(x,y) + d(y,z)\).
    \end{itemize}
\end{snippetdefinition}

\begin{snippetdefinition}{metric-space-definition}{Metric Space}
    A \textit{metric space} is a tuple \((X, d)\)
    consisting of a set \(X\) and a \distancefunctiontext \(d\) on \(X\).
\end{snippetdefinition}

%\section{Examples}

% TODO

\section{Balls}

\begin{snippetdefinition}{metricspaces-open-ball-definition}{Ball}
    Let \((X, d)\) be a \metricspace.
    An \textit{open ball} of radius \(\epsilon > 0\) around a point
    \(a \in X\) is defined as
    \[
        B_\epsilon(a) = \{x \in X \suchthat d(x,a) < \epsilon\}.
    \]
\end{snippetdefinition}

\begin{snippetdefinition}{metricspaces-closed-ball-definition}{Closed Ball}
    Let \((X, d)\) be a \metricspace.
    A \textit{closed ball} of radius \(\epsilon\) around a point
    \(a \in X\) is defined as
    \[
        \overline{B}_\epsilon(a) = \{x \in X \suchthat d(x,a) \leq \epsilon\}.
    \]
\end{snippetdefinition}

\begin{snippetdefinition}{metricspaces-bounded-set-definition}{Bounded Set}
    Let \((X, d)\) be a \metricspace and \(Y \subseteq X\).
    The set \(Y\) is said to be \textit{bounded} if \(Y\)
    is contained in some \openball.
\end{snippetdefinition}

\begin{snippetdefinition}{metricspaces-open-set-definition}{Open set}
    Let \((X, d)\) be a \metricspace and \(U \subseteq X\).
    We say that \(U\) is \textit{open} in \((X, d)\) if
    \[ \forall x \in U, \exists \epsilon > 0 \suchthat \ball_\epsilon(x) \subseteq U\]
\end{snippetdefinition}

\end{document}
