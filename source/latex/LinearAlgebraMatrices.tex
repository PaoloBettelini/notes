\documentclass[preview]{standalone}

\usepackage{amsmath,stackengine}
\usepackage{amssymb}
\usepackage{stellar}
\usepackage{definitions}

\begin{document}

\id{matrices}
\genpage

\section{Definition}

\includesnpt{linearalgebra-matrix-definition}

% TODO linear transformation

\section{Elementary row operations}

\includesnpt{linearalgebra-elementary-row-operations}

\section{Systems of linear equations}

\begin{snippet}{systems-of-linear-equations}
A system of linear equations
\begin{align*}
    a_1x + b_1y + c_1z &= d_1 \\
    a_2x + b_2y + c_2z &= d_2 \\
    a_3x + b_3y + c_3z &= d_3 \\
\end{align*}

Can be represented by a matrix multiplication \(M\vec{x}=\vec{d}\)

\[
    \begin{bmatrix} 
        a_1 && b_1 && c_1 \\
        a_2 && b_2 && c_2 \\
        a_3 && b_3 && c_3
    \end{bmatrix}
    \begin{bmatrix}
        x \\ y \\ z
    \end{bmatrix}
    =
    \begin{bmatrix}
        d_1 \\ d_2 \\ d_3
    \end{bmatrix}
\]

The geometrical interpretation is to find the vector \(\vec{x}\)
such that when the matrix \(M\) is applied to it, the resulting vector is \(\vec{d}\).

We may represent the whole system just by
\[
    \begin{bmatrix} 
        a_1 && b_1 && c_1 && d_1 \\
        a_2 && b_2 && c_2 && d_2 \\
        a_3 && b_3 && c_3 && d_3
    \end{bmatrix}
\]
\end{snippet}

\subsection{Using elementary row operations}

\begin{snippet}{linearalgebra-expl1}
    Applying \snippetref[linearalgebra-elementary-row-operations][elementary row operations]
    does not change the solution of the linear system.
\end{snippet}

\includesnpt{linearalgebra-lin-sys-sol-amount}

\begin{snippet}{linearalgebra-expl2}
By applying these operations, our goal is to make the system matrix
look like the following:
\[
    \begin{bmatrix} 
        1 && 0 && 0 && e_1 \\
        0 && 1 && 0 && e_2 \\
        0 && 0 && 1 && e_3
    \end{bmatrix}
\]
which is the implicit solution
\(x=e_1\), \(y=e_2\) and \(z=e_3\).
If this is possible, then this is the only solution to our system.

\vspace{.25cm}

If it is possible to create a row whose elements are all \(0\)s,
then there are infinitely many solutions, because
there are infinitely many solutions for the equation
\(0x_1+0x_2+\cdots+0x_n = 0\).

\vspace{.25cm}

If it is possible to create a whose elements elements
are all \(0\)s expect for the last one there are zero solutions,
because there are zero solutions for the equation
\(0x_1+0x_2+\cdots+0x_n = a\) with \(a \neq 0\).
\end{snippet}

\subsubsection{Examples}

\includesnpt{linearalgebra-matrix-linear-system-1-sol-example-1}
\includesnpt{linearalgebra-matrix-linear-system-inf-sol-example-1}
\includesnpt{linearalgebra-matrix-linear-system-0-sol-example-1}

% the set of all m times n matrices?

% the properties like commutativity and associativity
% are there only if the elements have it

\section{Definitions}

\includesnpt{linearalgebra-zero-matrix-definition}

\includesnpt{linearalgebra-matrix-addition-definition}

\includesnpt{linearalgebra-matrix-scalar-multiplication-definition}

% TODO properties of addition
% A+0_{m,n} = A
% A+B = B + A
% 0A = 0_{m,n}
% A+(-A)= 0_{m,n}
% (A+B)+C = A + (B + C)
% 1A=A
% (a+b) A = aA + bA
% a(A+B) = aA+aB
% a(bA) = (ab)A

\includesnpt{linearalgebra-matrix-multiplication-definition}

\begin{snippet}{linearalgebra-expl3}
This means applying a `dot product' between every row and every column. \\

\[
    \begin{bmatrix} 
        a_1 && a_2 && a_3 && a_4 \\
        \mathbf{b_1} && \mathbf{b_2} && \mathbf{b_3} && \mathbf{b_4} \\
        c_1 && c_2 && c_3 && c_4
    \end{bmatrix}
    \cdot
    \begin{bmatrix} 
        d_1 && \mathbf{e_1} \\
        d_2 && \mathbf{e_2} \\
        d_3 && \mathbf{e_3} \\
        d_4 && \mathbf{e_4}
    \end{bmatrix}
    =
    \begin{bmatrix} 
        a \cdot d && a \cdot e \\
        b \cdot d && \mathbf{b \cdot e} \\
        c \cdot d && c \cdot e
    \end{bmatrix}
\]

Where \(a \cdot b\) denotes the `dot product' between a row and a column.

This operation is not commutative, but it is associative.
\end{snippet}

\includesnpt{linearalgebra-kronecker-delta-definition}

\includesnpt{linearalgebra-identity-matrix-definition}

% When \(I_n\) is applied to a matrix or vector, the matrix or vector remains the same.

\includesnpt{linearalgebra-kronecker-delta-sifting-property}

\begin{snippet}{linearalgebra-expl4}
This is given by the fact that \(\delta_{i,k}\)
will be \(1\) only when \(i = k\).
\end{snippet}

% TODO properties fo mul and proofs
% A0_{n,p} = 0_{m,p}
% 0_{p,m}A=0_{p,n}

\includesnpt{linearalgebra-matrix-premultiplication}
\includesnpt{linearalgebra-matrix-postmultiplication}

\includesnpt{linearalgebra-matrix-identity-postmultiplication}
\includesnpt{linearalgebra-matrix-identity-postmultiplication-proof}
\includesnpt{linearalgebra-matrix-identity-premultiplication}
\includesnpt{linearalgebra-matrix-identity-premultiplication-proof}

\includesnpt{linearalgebra-polynomial-of-a-matrix-definition}

\includesnpt{linearalgebra-matrix-exponentiation-definition}
\includesnpt{linearalgebra-matrix-exponentiation-example-1}
\includesnpt{linearalgebra-matrix-exponentiation-example-2}

\includesnpt{linearalgebra-matrix-inverse-definition}
\includesnpt{linearalgebra-invertible-matrix-definition}

\begin{snippetproposition}{matrix-invertibility}{Matrix invertibility}
    If \(\det(A) = 0\), meaning that the matrix collapses space into a lower dimension,
    thus resulting in a loss of information, the matrix is not invertible.
    Otherwise, the matrix has an inverse and it is unique.
    
    This is equivalent to saying that if \(A\vec{v}=0\) for some non-zero vector \(\vec{v}\),
    then \(A\) has no inverse.
\end{snippetproposition}

\includesnpt{linearalgebra-uniqueness-of-matrix-inverse}
\includesnpt{linearalgebra-uniqueness-of-matrix-inverse-proof}

\includesnpt{linearalgebra-product-rule-of-inverse-matrices}
\includesnpt{linearalgebra-product-rule-of-inverse-matrices-proof}

\includesnpt{linearalgebra-involution-rule-for-matrix-inverse}
\includesnpt{linearalgebra-involution-rule-for-matrix-inverse-proof}

\includesnpt{linearalgebra-matrix-left-and-write-inverses}
\includesnpt{linearalgebra-inverse-of-a-non-square-matrix}
\includesnpt{linearalgebra-inverse-of-a-square-matrix}

\includesnpt{linearalgebra-matrix-invertibility}
\begin{snippet}{linearalgebra-expl5}
If \(\det(M) = 0\), meaning that the matrix collapses space into a lower dimension,
thus resulting in a loss of information, the matrix is not invertible.

This is equivalent to saying that if \(M\vec{v}=0\) for some non-zero vector \(\vec{v}\),
then \(M\) has no inverse.
\end{snippet}

% TODO: commutability matrix theorem

\section{Cramer's rule}

\begin{snippettheorem}{cramer-rule-theorem}{Cramer's rule}
    Consider a system of \(n\) equations and unknowns
    \[
        A\vec{x}=\vec{b}
    \]
    The solution is given by
    \[
        \vec{x}_i = \frac{\det(A_i)}{\det(A)}
    \]
    where \(A_i\) is formed by replacing the \(i\)-th column
    of \(A\) by \(\vec{b}\).
\end{snippettheorem}

\section{Matrix transpose}

\begin{snippetdefinition}{matrix-transpose-definition}{Matrix transpose}
    The \textit{transpose} of a \(n \times m\) matrix results in a \(m \times n\) one.

    \[
        {\left(A_{ij}\right)}^t=A_{ji}
    \]
    
    The transpose of a matrix is just a flipped version of the original matrix.
    We can transpose a matrix by switching its rows with its columns.
    The original rows become the new columns and the original columns become the new rows.
    
    \[
        {\begin{bmatrix} 
            a_{1,1} & a_{1,2} & \cdots & a_{1,m} \\
            a_{2,1} & a_{2,2} & \cdots & a_{2,m} \\
            \vdots  & \vdots  & \ddots & \vdots  \\
            a_{n,1} & a_{n,2} & \cdots & a_{n,m} 
        \end{bmatrix}}^t
        =
        \begin{bmatrix} 
            a_{1,1} & a_{2,1} & \cdots & a_{m,1} \\
            a_{1,2} & a_{2,2} & \cdots & a_{m,2} \\
            \vdots  & \vdots  & \ddots & \vdots  \\
            a_{1,n} & a_{2,n} & \cdots & a_{m,n} 
        \end{bmatrix}
    \]
\end{snippetdefinition}

\section{Column space}

\begin{snippetdefinition}{column-space-definition}{Column space}
    The \textit{column space} (or range or image) of a matrix is the
    set of all possible vectors that can be generated using the transformation.
\end{snippetdefinition}

\section{Rank of a matrix}

\begin{snippetdefinition}{matrix-rank-definition}{Matrix rank}
    The \textit{rank} of a matrix \(M\) is the dimension of its column space.
\end{snippetdefinition}

\begin{snippet}{matrix-rank-expl}
    A rank \(n\) means that every vector after the transformation of \(M\)
    is projected in the \(n\)-dimension.
\end{snippet}

\begin{snippetdefinition}{matrix-full-rank-definition}{Matrix full rank}
    A matrix is \textit{full rank} if its rank is equal to the number of its columns, meaning
    the dimension does not change.
\end{snippetdefinition}

\begin{snippet}{full-rank-expl}
    A matrix is full rank iff \(\vec{v}=\vec{0}\) is the only vector such that \(M\vec{v}=\vec{0}\).
    
    Note that non-square matrices may still be full rank even if the vectors end up in a lower dimension.
    If there is no loss of information the matrix is still full rank.
    If \(M\) has dimensions \(m \times n\), then it is full rank if \(\text{Rank}(M)=\min(n,m)\).
\end{snippet}

\section{Null space}

\begin{snippetdefinition}{null-space-definition}{Null space}
    The \textit{null space} or the \textit{kernel} of a matrix is the set of all the vectors that land on the origin
    after the transformation.
    The kernel is the linear subspace of the domain of the transformation which is mapped to \(\vec{0}\).
\end{snippetdefinition}

\section{Eigenvalues and eigenvectors}

\begin{snippetdefinition}{eigenvalues-eigenvectors-definition}{Eigenvalues and eigenvectors}
    The \textit{eigenvectors} \(\vec{v}\) and \textit{eigenvalues} \(\lambda\) of a matrix \(M\)
    are values such that
    \[
        M\vec{v} = \lambda \vec{v}
    \]
\end{snippetdefinition}

\end{document}
