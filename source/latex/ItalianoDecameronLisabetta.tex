\documentclass[preview]{standalone}

\usepackage{amsmath}
\usepackage{amssymb}
\usepackage{stellar}

\hypersetup{
    colorlinks=true,
    linkcolor=black,
    urlcolor=blue,
    pdftitle={Stellar},
    pdfpagemode=FullScreen,
}

\begin{document}

\title{Stellar}
\id{italiano-decameron-lisabetta-da-messina}
\genpage

\section{Analisi}

\begin{snippetnote}{b45faf05-2655-4048-aa94-57461a68891f}{}
    Nonostante ci siano 3 uomini, chi narra si rivolge spesso alle donne.
    Ciò è dato dalla lode di Boccaccio verso le donne.
\end{snippetnote}

\begin{snippet}{lisabetta-da-messina-analisi}
    \textbf{Rubrica:} I fratelli d'Elisabetta uccidono
    l'amante di lei: egli l'apparisce
    in sogno e mostrale dove sia sotterato;
    ella occultamente disotterra la testa e mettela in un testo di bassilico,
    e quivi sù piagnendo ogni dì per una grande ora, i fratelli gliele tolgono,
    e ella se ne muore di dolor poco appresso.

    \begin{enumerate}
        \item \textbf{Premessa (§3) e antefatto (§§4-5)}
        \item \textbf{Svolgimento dell'azione (§§6-23)}
        \begin{enumerate}
            \item Protagonisti: fratelli (§§6-11)
            \begin{itemize}
                \item Scoperta della trasca (§§6-11)
                \item Omicidio (§§8-9)
                \item Domande di Lisabetta (§§10-11)
            \end{itemize}
            \item Protagonista: Lisabetta (§§11-18)
            \begin{itemize}
                \item Sogno (§§12-13)
                \item Scoperta del cadavere e recupero della testa (§§14-16)
                \item Culto per il vaso (§§17-18)
            \end{itemize}
            \item Protagonisti: fratelli (§§19-22): sottrazione del vaso
            \item Protagonista: Lisabetta (§23): morte
        \end{enumerate}
        \item \textbf{Conclusione che rivela l'origine della novella (§§23-24)}
    \end{enumerate}

    % Struttura
    % Sistema dei personaggi
    % Interpretazioni

    I protagonisti, essendo alternati, mostrano di non comunicare fra di loro.
    Non vi sono dialoghi fra i fratelli e le sorelle vi sono solo domande che non vengono nemmeno risposte.
    Piuttosto che essere verbali, i fratelli compiono azioni concrete. Sono uomini pratici.
    Al contrario, la sorella è più verbale, comunica con le sue domande (§10, §11, §13, §20) e il pianto (§11, §12, §14, §17, §18, §20, §23).
    L'unico punto in cui Elisabbeta sospende il pianto è quando trova il corpo deceduto e ne taglia la testa.

    I personaggi sono divisi in due, da un lato i 3 fratelli mentre dall'altra Lisabetta, dall'altro Lisabetta, Fante e Lorenzo.
    Il rapporto non è equilibrato, i fratelli sono gerarchicamente superiori per ragioni sociali.
    I loro valori sono completamente diversi, mettono il denaro in cima a tutto. Tuttavia, ciò non viene detto direttamente,
    bensì viene implicato dal loro interesse per la reputazione della loro famiglia ed attività.
    Se gli interessi della sorella si dovessero scoprire, la loro reputazione, assieme ai loro affari, crollerebbero (§7, §22).

    Quando uno dei fratelli scopre la sorella a letto con il garzone, non agisce di impulso ma ne parla con gli altri fratelli.
    La vicenda viene trattata come se fosse un affare di famiglia; ragionano in maniera fredda e come mercanti, bilanciando cosa fare in funzione della resa economica.
    La novella è tragica, nessuno ritrae un valore dalla vicenda.

    Boccaccio accusa la logica mercantile (non il mondo dei mercanti) con i suoi eccessi.
    Questa novella è in contrapposizione a quella dello Stalliere, dove viene posto un limite, mentre i fratelli non si fermano, non riuscendo ad equilibrare furbizia e sentimenti.

    % Da un punto di vista psicoanaliti, a differenza di quello sociale, i ruoli si ivnertono:
    % i 3 fratelli non hanno nome, indistinguibili e privi di identità.
    % Sono 3 attori, ma solo un attante
    % Non sanno amare, non sanno comunicare
    % Di lorenzo si dice che tutti lor fatti guidava, per cui i fratelli sono anche incapaci senza di lui.
    % Contrariamente, Lisabetta sa amare, sa comunicare (con le domande), e sa agire (prende l'iniziativa con Lorenzo, nonostante il divieto)
    % È come se i fratelli vedessero la loro inferiorià nella sessualità della propria sorella.
    % I fratelli uccidono Lorenzo 3vs1, ingannandolo e colpendolo alle spalle come dei vigliacchi.

    % Quando Lisabetta sogna Lorenzo, le sue parole non sono né di amore né di enfasi verso ciò che è successo.
    % Quando Lisabetta gli stacca la testa, la mette in grembo alla fante come se fosse un figlio.
    % Lisabetta trasforma quindi il trauma portando tutte le sue cure alla testa, come se fosse un bambino.
    % la pianta del basicili, mitologicamente, è legata alla fertilità. Ciô è legato alla cura e la crescita del figlio.
    % Il basilico è bellissimo e odorifero molto §19 che è sintatticamente la stessa descrizione di Lorenzo, §5
\end{snippet}

\end{document}