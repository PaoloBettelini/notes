\documentclass[preview]{standalone}

\usepackage{amsmath}
\usepackage{amssymb}
\usepackage{tikz}
\usepackage{stellar}
\usepackage{bettelini}

\hypersetup{
    colorlinks=true,
    linkcolor=black,
    urlcolor=blue,
    pdftitle={Assets},
    pdfpagemode=FullScreen,
}

\begin{document}

\title{I Vulcani}
\id{geofisica-vulcani}
\genpage

\begin{snippetdefinition}{eruzione}{Eruzione}
    La fuoriuscita di materiale da un vulcano è detta \textit{eruzione}
    e i materiali eruttati sono lava, cenere, lapilli, gas, scorie varie e vapore acqueo.
\end{snippetdefinition}

\plain{Quando i magmi sono viscosi e ricchi di gas, si può formare una sorta di tappo,
per poi portare ad <b>eruzioni esplosive</b>.
Quando invece ci sono magmi poco viscosi, i gas si liberano facilmente
permettendo uno scorrimento tranquillo e costante della lava (<b>eruzioni effusive</b>).}

\begin{snippetdefinition}{lava}{Lava}
    La \textit{lava} è il nome che viene dato al magma vulcanico
    dopo che ha perso i gas e gli altri componenti volatili sotto pressione che lo permeavano,
    ossia quando il magma raggiunge la superficie terrestre attraverso un condotto vulcanico.
\end{snippetdefinition}

\plain{Il basalto vulcanico, molto scuro compone la crosta oceanica terrestre.
Il rapido raffreddamento della lava porta alla formazione della pietra pomice o dell'ossidiana
(a dipendenza del tipo di lava).}

\begin{snippet}{vulcani-effetti-positivi}
    I vulcani possono avere degli effetti positivi, quali
    \begin{itemize}
        \item \textbf{Fertilità del suolo:} le eruzioni vulcaniche rilasciano minerali come potassio,
            fosforo e azoto nel suolo, rendendolo estremamente fertile per l'agricoltura.
        \item \textbf{Creazione di nuove terre:} le eruzioni vulcaniche possono creare nuove isole
            e terreni. Ad esempio, le isole dell'arcipelago delle Hawaii sono state formate
            da vulcani sottomarini.
        \item \textbf{Fonte di energia geotermica:} l'energia geotermica, prodotta dal calore
            del magma sottostante, può essere sfruttata per la produzione
            di energia elettrica e riscaldamento nelle vicinanze dei vulcani.
    \end{itemize}
\end{snippet}

\end{document}