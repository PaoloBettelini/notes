\documentclass[preview]{standalone}

\usepackage{amsmath}
\usepackage{amssymb}
\usepackage{stellar}
\usepackage{definitions}

\begin{document}

\id{infimum-and-supremum-exercises}
\genpage

\section{Exercises}

% TODO: big such that

\begin{snippetexercise}{infimum-supremum-ex-1}{}
    Let
    \[
        E = \left\{ x\in\mathbb{R} \,|\, 1 \leq 2x < 10 \right\}
    \]
    and the sequence \[
        F = \left\{ x = x_n \,|\, x_n = \frac{n + 1}{n + 2}, \quad n\in\mathbb{N}^\exceptzero \right\}
    \]
    TODO
\end{snippetexercise}

\begin{snippetexercise}{infimum-supremum-ex-2}{}
    Find the supremum and \greatestelement of the set
    \[
        E = \left\{ x_n = \frac{n-7}{x^2 + 1} \ \middle|\ n \geq 1 \right\}
    \]
    This sequence certainly has a \leastelement since there are only \(6\) negative numbers.
    We can see that the denominator grows faster than the numerator.

    Let us therefore study for which indices \(x_n \leq x_{n+1}\). We get
    \begin{align*}
        \frac{n-7}{n^2 + 1} &\leq \frac{(n+1)-7}{{(n+1)}^2 + 1} \\
        \frac{(n-7)(n^2 + 2n + 2) - (n-6)(n^2+1)}{(n^2 + 1)(n^2 + 2n + 2)} &\leq 0
    \end{align*}
    The denominator is positive, so we only study the numerator
    \begin{align*}
        n^2 - 13n - 8 \leq 0
    \end{align*}
    The roots of this polynomial are \(n_{1,2}= \frac{13\pm\sqrt{201}}{2}\).
    Consequently, the expression is negative for \(\frac{13-\sqrt{201}}{2} < n < \frac{13+\sqrt{201}}{2}\).
    We note that the extreme left is negative. We also note that \(14^2 < 201 < 15^2\),
    and therefore the right extremum is between \(14\) and \(\frac{27}{2}\).
    Then, all \(n\) integers satisfying the equation are
    \(n=13\). It follows that if \(n \geq 14\), \(x_n > x_{n+1}\).
    The supremum and \greatestelement is therefore \(x_{14}\).
\end{snippetexercise}

\begin{snippetexercise}{infimum-supremum-ex-3}{Infimum, supremum, greatest and least element}
    Let
    \[
        E = \left\{ x \in \realnumbers \suchthat \ln(x^3 + 1) \leq \right\}
    \]
    The domain of the logarithm is given by \(x^3 > -1\), so \(x>-1\).
    We also need to consider \(\ln(x^3 + 1) \leq 0\) which gives us
    \(x^3 \leq 0 \iff x \leq 0\). Therefore, the final condition is \(-1 < x \leq 0\).
    We thus have \(\inf E = -1\) and \(\sup E = 0\). The
    \greatestelement is \(0\) and the \leastelement does not exist.
\end{snippetexercise}

\begin{snippetexercise}{infimum-supremum-ex-4}{Infimum, supremum, greatest and least element}
    Let
    \[
        E = \left\{ \frac{1}{n} \suchthat n \in \naturalnumbers^\exceptzero \right\}
    \]
    Since the sequence is strictly decreasing, \(\sup E = 1\)
    which is also the \greatestelement. The sequence does not have a \leastelement
    and \(\inf E = 0\).
    We need to also show that the sequence is monotone decreasing.
\end{snippetexercise}

\begin{snippetexercise}{infimum-supremum-ex-5}{Infimum, supremum, greatest and least element}
    Let
    \[
        E = \left\{ \frac{{(-1)}^n}{n} \suchthat n \in \naturalnumbers^\exceptzero \right\}
    \]
    The sequence clearly converges with zero and is absolutely decreasing.
    We thus have \(\inf E = -1\) and \(\sup E = \frac{1}{2}\). These are also the \leastelement
    and \greatestelement.
    To prove that the sequence is absolutely decreasing we can distinguish the positive and negative cases.
\end{snippetexercise}

\begin{snippetexercise}{infimum-supremum-ex-6}{Infimum, supremum, greatest and least element}
    Let
    \[
        E = \left\{ \frac{9m}{m^2 + 9} \suchthat n \in \naturalnumbers^\exceptzero \right\}
    \]
    This sequences initially grows but then starts decreasing and tends to \(0\).
    Thus, \(\inf E = 0\) but it does not have a \leastelement.
    The supremum and \greatestelement are given by \(n=3\).
    To demonstrate this we prove
    \[
        \frac{9m}{m^2} \leq \frac{3}{2}
    \]
    for all \(m\). We get
    \begin{align*}
        18m^2 \leq 2m^2 + 27 \\
        m^2 - 6m + 9 \geq 0 \\
        (n-3)^2 \geq 0
    \end{align*}
    so it is always true.
    We also need to show that \(\inf E = 0\).
    To do so we let \(x > 0\) and we say that there exists an \(m \in\mathbb{N}\)
    such that \(\frac{9m}{m^2 + 9} > x \implies x > 0\) (i.e, \(x\) is not a \lowerbound).
    We get \(9m-xm^2 < 9x\) and
    \[
        m = \frac{9\pm \sqrt{81-96x}}{2x}
    \]
    so it is always true since there is no such \(m\).
\end{snippetexercise}

\begin{snippetexercise}{infimum-supremum-ex-7}{Infimum, supremum, greatest and least element}
    Let
    \[
        E = \left\{ \frac{3m^2 - 4}{m^2 + 1} \suchthat n \in \naturalnumbers \right\}
    \]
    The initial values are \(-4, - \frac{1}{2}, \frac{8}{5}, \frac{23}{10}\).
    By the limit of the sequence we get \(\sup E = 3\). To show this we set
    \begin{align*}
        \frac{3m^2 - 4}{m^2 + 1} \leq 3 \\
        -4 \leq 3
    \end{align*}
    However, \(3\) is not the \greatestelement.
    We also have \(\inf = -4\) which is also the \leastelement.
    To show this we show that for \(m \geq 2\)
    \begin{align*}
        \frac{3m^2-4}{m^2 + 1} &\geq 0 \\
        -\frac{1}{2} &\leq 0
    \end{align*}
\end{snippetexercise}

\end{document}