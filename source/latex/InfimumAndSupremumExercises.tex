\documentclass[preview]{standalone}

\usepackage{amsmath}
\usepackage{amssymb}
\usepackage{stellar}
\usepackage{definitions}

\begin{document}

\id{infimum-and-supremum-exercises}
\genpage

\section{Exercises}

% TODO: big such that

\begin{snippetexercise}{infimum-supremum-ex-1}{}
    Let
    \[
        E = \{ x\in\mathbb{R} \,|\, 1 \leq 2x < 10 \}
    \]
    and the sequence \[
        F = \{ x = x_n \,|\, x_n = \frac{n + 1}{n + 2}, \quad n\in\mathbb{N}^* \}
    \]
    TODO
\end{snippetexercise}

\begin{snippetexercise}{infimum-supremum-ex-2}{}
    Find the supremum and maximumo of the set
    \[
        E = \left\{ x_n = \frac{n-7}{x^2 + 1} \ \middle|\ n \geq 1 \right\}
    \]
    This sequence certainly has a minimum since there are only \(6\) negative numbers.
    We can see that the denominator grows faster than the numerator.

    Let us therefore study for which indices \(x_n \leq x_{n+1}\). We get
    \begin{align*}
        \frac{n-7}{n^2 + 1} &\leq \frac{(n+1)-7}{{(n+1)}^2 + 1} \\
        \frac{(n-7)(n^2 + 2n + 2) - (n-6)(n^2+1)}{(n^2 + 1)(n^2 + 2n + 2)} &\leq 0
    \end{align*}
    The denominator is positive, so we only study the numerator
    \begin{align*}
        n^2 - 13n - 8 \leq 0
    \end{align*}
    The roots of this polynomial are \(n_{1,2}= \frac{13\pm\sqrt{201}}{2}\).
    Consequently, the expression is negative for \(\frac{13-\sqrt{201}}{2} < n < \frac{13+\sqrt{201}}{2}\).
    We note that the extreme left is negative. We also note that \(14^2 < 201 < 15^2\),
    and therefore the right extremum is between \(14\) and \(\frac{27}{2}\).
    Then, all \(n\) integers satisfying the equation are
    \(n=13\). It follows that if \(n \geq 14\), \(x_n > x_{n+1}\).
    The supremum and maximum is therefore \(x_{14}\).
\end{snippetexercise}

\end{document}