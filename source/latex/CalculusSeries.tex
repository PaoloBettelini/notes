\documentclass[preview]{standalone}

\usepackage{amsmath}
\usepackage{amssymb}
\usepackage{tikz}
\usepackage{stellar}
\usepackage{definitions}
\usepackage{bettelini}

\begin{document}

\id{series}
\genpage

\section{Definition}

\begin{snippetdefinition}{series-definition}{Series}
    Let \(\{a_n\}_{n=k}^\infty\) be \sequence over a \field where \(k\in\integers\).
    The \(N\)-th \emph{partial sum} is the \sequence
    \[
        \{S_N\}_{N=k}^\infty \triangleq \sum_{n=k}^\infty a_n, \quad
        S_N \triangleq \sum_{n=k}^N a_n
    \]
    The \emph{series of general terms \(\{a_n\}_{n=k}^\infty\)} is defined as the sum
    \[
        \sum_{n=k}^\infty a_n
    \]
    We say that
    \[
        \sum_{n=k}^\infty a_n = S
    \]
    with \(S=\lim S_N\) if the limit exists:
    \begin{enumerate}
        \item the series \emph{converges} if \(\exists \lim S_N\) and \(S\in\realnumbers\);
        \item the series \emph{diverges} if \(\exists \lim S_N\) and \(S=\pm\infty\);
        \item the series \emph{oscillates} if \(\nexists \lim S_N\).
    \end{enumerate}
\end{snippetdefinition}

\plain{Note that this is a bit of abuse of notation; the same symbol represents both the sequence of partial sums
and its limit.}

\section{Covergence theorem}

\begin{snippettheorem}{convergence-theorem}{Convergence Theorem}
    Let \(\{a_n\}_{n=k}^\infty\) be \sequence.
    Then, if \[\sum_{n=k}^\infty a_n\] \seriesconverges
    we have that \[\lim a_n=0\]
\end{snippettheorem}

\begin{snippetproof}{convergence-theorem-proof}{convergence-theorem}{Convergence Theorem}
    Consider the \partialsum
    \[
        S_N = \sum_{k=1}^{M}a_k
    \]
    The sequence \(a_n\) can now be expressed as
    \[
        a_n = S_N - S_{N-1}
    \]
    Since \(\sum a_n\) converges, \(\lim S_N=L\) for \(L\) finite. \\
    The limit \(\lim S_{N-1}=L\) because \(N-1 \to \infty \text{ as } N \to \infty\).
    This implies the following
    \[
        \lim a_n = \lim S_N - S_{N-1} = L - L = 0
    \]
\end{snippetproof}

\section{Absolute and conditional convergence}

\begin{snippetdefinition}{absolute-convergence-definition}{Absolute convergence}
    A series \(\sum a_n\) is said to converge absolutely if
    \(\sum |a_n|\) converges.
\end{snippetdefinition}

\plain{This is a stronger type of convergence. Every absolutely convergent series is also convergent.}
\plain{A series that is convergent but not absolutely convergent is called conditionally convergent.}


\section{Riemann rearrangement theorem}

\begin{snippettheorem}{riemann-rearrangement-theorem}{Riemann rearranged theorem}
    If an infinite series is conditionally convergent, then its terms can be rearranged such that
    the series converges to any \(r\in \realnumbers\) or such that it diverges (to infinity or no finite value).
    If the series is absolutely convergent then any rearrangement of its terms will converge to the same value.
\end{snippettheorem}

\section{Product of series}

\begin{snippettheorem}{infinite-series-properties}{}
    Let \(\{a_n\}_{n=0}^\infty, \{b_n\}_{n=0}^\infty\) be \sequence[sequences].
    Then,
    \[
        \left(
            \sum_{n=0}^\infty a_n
        \right)
        \left(
            \sum_{n=0}^\infty b_n
        \right)
        =
        \sum_{n=0}^\infty \sum_{k=0}^n a_k b_{n-k}
    \]
\end{snippettheorem}

\end{document}