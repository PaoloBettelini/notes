\documentclass[preview]{standalone}

\usepackage{amsmath}
\usepackage{amssymb}
\usepackage{parskip}
\usepackage{fullpage}
\usepackage{hyperref}
\usepackage{tikz}
\usepackage{stellar}
\usepackage{definitions}
\usepackage{bettelini}

\begin{document}

\id{series}
\genpage

\section{Divergence and convergence}

\begin{snippet}{infinite-series-convergence}
    An infinite series converges if the limit
    of its partial sum sequence also converges,
    otherwise it diverges.
\end{snippet}

\section{Properties}

\begin{snippettheorem}{infinite-series-properties}{}
    \[
        \left(
            \sum_{n=0}^\infty a_n
        \right)
        \left(
            \sum_{n=0}^\infty b_n
        \right)
        =
        \sum_{n=0}^\infty \sum_{k=0}^n a_k b_{n-k}
    \]
\end{snippettheorem}

\section{Covergence theorem}

\begin{snippettheorem}{convergence-theorem}{Convergence Theorem}
    If \(\sum a_n\) converges then \(\lim_{n\to\infty}a_n=0\)
\end{snippettheorem}

\begin{snippetproof}{convergence-theorem-proof}{convergence-theorem}{Convergence Theorem}
    Consider the partial sum
    \[
        s_n = \sum_{k=1}^{n}a_k
    \]
    The sequence \(a_n\) can now be expressed as
    \[
        a_n = s_n - s_{n-1}
    \]
    Since \(\sum a_n\) converges, \(\lim_{n\to\infty}s_n=L\) for \(L\) finite. \\
    The limit \(\lim_{n\to\infty}s_{n-1}=L\) because \(n-1 \to \infty \text{ as } n \to \infty\).
    This implies the following
    \[
        \lim_{n \to \infty} a_n
        = \lim_{n \to \infty} s_n - s_{n-1} = L - L = 0
    \]
\end{snippetproof}

\section{Divergence test}

\begin{snippettheorem}{divergence-test}{Divergence test}
    If \(\lim_{n \to \infty} a_n \neq 0\) then \(\sum a_n\) diverges.
\end{snippettheorem}
% TODO proof

\section{Absolute and conditional convergence}

\begin{snippetdefinition}{absolute-convergence-definition}{Absolute convergence}
    A series \(\sum a_n\) is said to converge absolutely if
    \(\sum |a_n|\) converges.
\end{snippetdefinition}

\plain{This is a stronger type of convergence. Every absolutely convergent series is also convergent.}
\plain{A series that is convergent but not absolutely convergent is called conditionally convergent.}


\section{Riemann rearrangement theorem}

\begin{snippettheorem}{riemann-rearrangement-theorem}{Riemann rearranged theorem}
    If an infinite series is conditionally convergent, then its terms can be rearranged such that
    the series converges to any \(r\in \realnumbers\) or such that it diverges (to infinity or no finite value).
    If the series is absolutely convergent then any rearrangement of its terms will converge to the same value.
\end{snippettheorem}

\section{Geometric series}

\begin{snippetdefinition}{geometric-series-definition}{Geometric Series}
    A \textit{geometric series} is a series of the form
    \[
        \sum_{n=0}^\infty r^n
    \]
\end{snippetdefinition}

\plain{The ratio between two adject terms is constant.}

\begin{snippettheorem}{geometric-series-convergence}{Geometric Series Convergence}
    A \geometricseries converges for \(|r| < 1\) and always \convergesabsolutely.
    \[
        \sum_{n=0}^\infty r^n = \frac{1}{1-r}
    \]
\end{snippettheorem}

\section{Telescoping series}

\begin{snippetdefinition}{telescoping-series-definition}{Telescoping Series}
    A \textit{telescoping series} is a series where the terms in the partial sums cancel each other,
    leaving a finite number of terms.
\end{snippetdefinition}

\begin{snippetexample}{geometric-series-example-1}{Geometric Series}
    \begin{align*}
        &\sum_{n=0}^\infty \frac{1}{n^2 + 3n + 2}
        = \sum_{n=0}^\infty \left[ \frac{1}{n+1} - \frac{1}{n+2} \right]
        = \lim_{N \to \infty} \sum_{n=0}^N \left[ \frac{1}{n+1} - \frac{1}{n+2} \right] \\
        &= \frac{1}{1} - \frac{1}{2} + \frac{1}{2} - \frac{1}{3}
        + \frac{1}{3} - \frac{1}{4} + \cdots + \frac{1}{n} - \frac{1}{n+1} +
        \frac{1}{n+1} - \frac{1}{n+2} \\
        &= \lim_{N \to \infty} 1 - \frac{1}{n+2} = 1
    \end{align*}
\end{snippetexample}

\section{Harmonic series}

\begin{snippetdefinition}{harmonic-series-definition}{Harmonic series}
    The \textit{harmonic series} is the series
    \[
        \sum_{n=1}^\infty \frac{1}{n}
    \]
\end{snippetdefinition}

\begin{snippettheorem}{harmonic-series-divergence-theorem}{The harmonic series diverges}
    The \harmonicseries diverges.
\end{snippettheorem}

\section{Integral Test}

\begin{snippetdefinition}{integral-test}{Integral Test}
    Let \(f(x)\) be a continuous function on \([k;\infty)\)
    such that it is decreasing and positive on the interval \([N; \infty)\)
    for some \(N\).
    \[
        \integral[k][\infty][f(x)][x] \text{ converges } \implies
        \sum_{n=k}^{\infty} f(n) \text{ converges}
    \]
    and
    \[
        \integral[k][\infty][f(x)][x] \text{ diverges } \implies
        \sum_{n=k}^{\infty} f(n) \text{ diverges}
    \]
\end{snippetdefinition}

\begin{snippetproof}{integral-test-proof}{integral-test}{Integral Test}{
    \todo
}
\end{snippetproof}

\section{p-series}

\begin{snippetdefinition}{p-series-definition}{p-series Test}
    If \(k > 0\) then \[\sum_{n=k}^\infty \frac{1}{n^p}\]
    converges if \(p > 1\) and diverges if \(p \leq 1\).
\end{snippetdefinition}

\section{Comparison Test}

\begin{snippettheorem}{series-comparison-test}{Comparison test}
    Let \(\sum a_n\) and \(\sum b_n\) be two series with \(a_n, b_n \geq 0\)
    and \(\forall n, a_n < b_n\). Then,
    \begin{enumerate}
        \item If \(\sum b_n\) is convergent, so is \(\sum a_n\).
        \item if \(\sum a_n\) is divergent, so is \(\sum b_n\).
    \end{enumerate}
\end{snippettheorem}

\section{Limit comparison test}

\begin{snippettheorem}{limit-comparison-test}{Limit comparison test}
    Let \(\sum a_n\) and \(\sum b_n\) be two series with \(a_n \geq 0\)
    and \(b_n > 0\). Let
    \[ c = \lim_{n \to \infty} = \frac{a_n}{b_n} \]
    If \(0 > c > \infty\), then either both series converge or both series diverge.
\end{snippettheorem}

\section{Alternating series test}

\begin{snippettheorem}{alternating-series-test}{Alternating series test}
    Let \(\sum a_n\) be a series
    and either \(a_n = {(-1)}^n b_n\) or \(a_n = {(-1)}^{n+1} b_n\)
    where \(b_n \geq 0\).
    If the following conditions are met:
    \begin{enumerate}
        \item \(\lim_{n \to \infty} b_n = 0\);
        \item \(\{b_n\}\) is a decreasing sequence,
    \end{enumerate}
    then \(\sum a_n\) is convergent.
\end{snippettheorem}

\section{Ratio test}

\begin{snippettheorem}{ratio-test}{Ratio test}
    Let \(\sum a_n\) be a series and set
    \[
        L = \lim_{n \to \infty} | \frac{a_{n+1}}{a_n} |
    \]
    Then,
    \begin{enumerate}
        \item if \(L < 1\) the series is absolutely convergent;
        \item if \(L > 1\) the series is divergent;
        \item if \(L = 1\) the series may be divergent, conditionally convergent, or absolutely convergent.
    \end{enumerate}
\end{snippettheorem}

\section{Root test}

\begin{snippettheorem}{root-test}{Root test}
    Let \(\sum a_n\) be a series and set
    \[
        L = \lim_{n \to \infty} {|a_n|}^{\frac{1}{n}}
    \]
    Then,
    \begin{enumerate}
        \item if \(L < 1\) the series is absolutely convergent;
        \item if \(L > 1\) the series is divergent;
        \item if \(L = 1\) the series may be divergent, conditionally convergent, or absolutely convergent.
    \end{enumerate}
\end{snippettheorem}

\begin{snippetproposition}{nth-root-of-n-limit}{}
    \[ \lim_{n \to \infty} n^{\frac{1}{n}} = 1 \]
\end{snippetproposition}

\section{Derivative and integral of power series}

%\begin{snippettheorem}{}{}
%\end{snippettheorem}

% missing: https://tutorial.math.lamar.edu/Classes/CalcII/EstimatingSeries.aspx

\end{document}