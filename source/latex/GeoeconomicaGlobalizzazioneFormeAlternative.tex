\documentclass[preview]{standalone}

\usepackage{amsmath}
\usepackage{amssymb}
\usepackage{stellar}
\usepackage{tabularx}

\hypersetup{
    colorlinks=true,
    linkcolor=black,
    urlcolor=blue,
    pdftitle={Stellar},
    pdfpagemode=FullScreen,
}

\begin{document}

\title{Stellar}
\id{geoeconomica-forme-alternative-globalizzazione}
\genpage

% TODO cita il testo

\section{Forme alternative di globalizzazione}

\subsection{La globalizzazione e i suoi oppositori - J. Stiglitz}

\begin{snippet}{globalizzazione-alternative-j-stiglitz}
    \vspace{-0.25cm}
    \begin{itemize}
        \item Maggiore integrazione economica globale, riduzione dei costi delle merci;
        \item Malcontento nei paesi in via di sviluppo e nei ceti medi e bassi dei paesi avanzati;
        \item Trump critica gli accordi commerciali, ritenuti ingiunsti per gli Stati Uniti;
        \item Regole scritte dai paesi avanzati a discapito dei più deboli;
        \item Gli accordi favoriscono le multinazionali, penalizando i lavoratori;
        \item La globalizzazione può giovare a tutti se ben gestita, serve un sistema più equo.
    \end{itemize}
\end{snippet}

\subsection{L'Occidente e l'ascesa globale del resto del mondo - G. Sabattini}

\begin{snippet}{globalizzazione-alternative-g-sabattini}
    \vspace{-0.25cm}
    \begin{itemize}
        \item Supremazia occidentale per due secoli, valori occidentali prevalenti dopo la Guerra
            fredda;
        \item Declino della supremazia occidentale, crescita di potenze emergenti;
        \item Futuro multipolare e politicamente plurale, diverse potenze coesisteranno;
        \item Transizione pacifica richiede collaborazione e nuove regole per le relazioni
            internazionali;
        \item Accettazione della diverstià politica per una governance globale inclusiva.
    \end{itemize}
\end{snippet}

\subsection{Una nuova globalizzazione tra paesi amici - F. Rampini}

\begin{snippet}{globalizzazione-alternative-f-rampini}
    \vspace{-0.25cm}
    \begin{itemize}
        \item Concetto di ``Friend-shoring'': spostamento delle fabbriche verso paesi amici e
            alleati per ridurre la dipendenza da paesi considerati ostili (es. Russia e Cina);
        \item Pandemia e sanzioni evidenziano i rischi delle catene produttive globali;
        \item Delocalizzazioni hanno causato perdita di posti di lavoro e inflazione;
        \item Riportare produzioni nei paesi democratici può creare più posti di lavoro;
        \item Nuova geopolitica della globalizzazione complessa e costosa, ma potenzialmente
            benefica per la sicurezza economica.
    \end{itemize}
\end{snippet}

\subsection{Forme alternative di globalizzazione in breve}

\begin{snippet}{globalizzazione-alternative-riassunto}
    \phantom{}\\
    \begin{table*}[ht!]
        \centering
        \renewcommand{\arraystretch}{1.5}
        \begin{tabular}{| m{2cm} | m{4cm} | m{4cm} | m{4cm} |}
        \hline
        \textbf{} & \textbf{Principali critiche all'attuale realtà globale} & \textbf{Altre caratteristiche (neutre o positive) attribuite al mondo globalizzato attuale} & \textbf{Fondamentali cambiamenti auspicabili e possibili per una globalizzazione più soddisfacente} \\
        \hline
        \textbf{Joseph \ Stiglitz \ (2018)} & 
        - Malcontento nei paesi in via di sviluppo e nei ceti medi e bassi dei paesi avanzati. \newline
        - Accordi commerciali favorevoli alle multinazionali, non ai lavoratori. & 
        - Maggiore integrazione economica. \newline
        - Riduzione dei costi delle merci. & 
        - Gestire meglio la globalizzazione per renderla più equa ed efficiente. \\
        \hline
        \textbf{Charles Kupchan \ (2017)} & 
        - Declino dell'egemonia occidentale. \newline
        - Futuro multipolare con potenze emergenti che sfidano l'Occidente. & 
        - Varietà di concezioni politiche ed economiche nel futuro multipolare. & 
        - Collaborazione tra l'Occidente e le potenze emergenti. \newline
        - Accettazione della diversità politica. \\
        \hline
        \textbf{Federico Rampini \ (2022)} & 
        - Dipendenza economica da paesi ostili. \newline
        - Delocalizzazioni che causano perdita di posti di lavoro e inflazione. & 
        - Potenziale creazione di nuovi posti di lavoro nei paesi democratici. & 
        - "Friend-shoring" per ridurre la dipendenza da paesi ostili. \newline
        - Nuova geopolitica della globalizzazione più sicura ma complessa. \\
        \hline
        \end{tabular}
    \end{table*}
    \vspace{1cm}
\end{snippet}

\end{document}