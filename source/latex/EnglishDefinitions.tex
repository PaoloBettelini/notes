\documentclass[preview]{standalone}

\usepackage{amsmath}
\usepackage{amssymb}
\usepackage{bettelini}
\usepackage{stellar}

\hypersetup{
    colorlinks=true,
    linkcolor=black,
    urlcolor=blue,
    pdftitle={English},
    pdfpagemode=FullScreen,
}

\begin{document}

\id{english-definitions}
\genpage

\section{Narration}

\begin{snippet}{english-narration-terms}
The story is told by the narrator, who is not identical with the author. The narrator answers the
question \textit{Who speaks}?

There are essentially two types of narrators:
\begin{itemize}
    \item \textbf{first-person narrator:}
        a character in the story speaking as “I” or “we” is called a first-person
        narrator;
    \item \textbf{third-person narrator:}
        he/she is not a character in the story.
\end{itemize}

The third-person narrator knows the thoughts and emotions of all the characters; he is an
\textbf{omniscient narrator}. An omniscient narrator can move freely in time and space. He can shift from
character to character; reporting what he/she chooses of their speech, actions, thoughts, feelings
and emotions. He may give comments or decide to ''show`` the action without judgement.

If the narrator chooses to describe the thoughts and emotions of only one character, he is a \textbf{selective
narrator}: he presents the story through one character's eyes.

A first-person narrator is naturally \textbf{limited} in his perspective.
\end{snippet}

\section{Focalization}

\begin{snippet}{english-focalization-terms}
Choosing a perspective or point of view is separate from determining whether the narrator is a
character in the story. The focalizer answers the question \textit{Who sees or perceives}?

There are three kinds of focalizers:
\begin{itemize}
    \item \textbf{zero focalization} corresponds to the omniscient narrator. Here the narrator knows more
    than the characters;
    \item \textbf{an internal focalizer}'s perception belongs to a character within the story. Internal focalizers
    are also called character-focalizers. Here the narrator knows and says only what a given
    character knows;
    \item \textbf{an external focalizer} is a POV character external to the story. An external focalizer is called
    a narrator-focalizer because perception belongs to the narrator. Here the character knows
    more than the narrator.
\end{itemize}
\end{snippet}

\begin{snippetdefinition}{media-res-definition}{Medias res}
    XXXXXXXXXXXXXXXXXXXXXXXXXXXXXXXXXXXXXXXXXXXXXXXXX
\end{snippetdefinition}

\end{document}
