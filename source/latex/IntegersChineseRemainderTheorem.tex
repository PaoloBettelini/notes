\documentclass[preview]{standalone}

\usepackage{amsmath}
\usepackage{amssymb}
\usepackage{stellar}
\usepackage{definitions}
\usepackage{bettelini}

\begin{document}

\id{chinese-remainder-theorem}
\genpage

\section{Chinese remainder theorem}

\begin{snippet}{congruences-with-unknowns-expl1}
    Consider the equations in \(\integers / n\) and congruences with unknowns.
    Suppose to have a congruence with unknown of the form \(a x \equiv b \pmod{n}\)
    wiht \(a,b\in\integers\) fixed and \(n\in\integers^+\) fixed.

    If \(a\) is \invertiblecongclass[invertible] modulo \(n\), meaning there exist a \(c\) such that
    \(ac \equiv 1 \pmod{n}\), we can solve the congruence
    as we would do for a linear equation in the rationals, meaning \quotes{dividing} by \(a\),
    or, more precisely, by multiply by \(c\), and finding \(x \equiv bc \pmod{n}\).

\end{snippet}

\begin{snippetexample}{congruences-with-unknowns-example}{}
    For example, consider \(3x \equiv 5 \pmod{20}\). Since \(3\) is \coprime with \(20\),
    there exist an ineger \(c\) such that \(3c \equiv 1 \pmod{20}\).
    We thus easily find for instance \(c=7\), and we get
    \[
        3\cdot7x \equiv 5\cdot 7 \pmod{20}
    \]
    which means that \(x\equiv 35 \pmod{20} \equiv 15 \pmod{20}\)

    Equivalently, we can consider the equation in \(\mathbb{Z} / n\),
    \({[a]}_n x = {[b]}_n\) where we look for an \(x\) in \(\mathbb{Z} / n\).
    If, like above, \(a\) is \coprime with \(n\), and it exists, which means
    \({[c]}_n\) such that \({[a]}_n{[c]}_n = {[1]}_n\), the given equation has the unique solution
    \[
        x = {[b]}_n{[c]}_n = {[bc]}_n
    \]

    In out example we have that \({[3]}_{20} x = {[5]_{20}}\) has a unique solution
    \(x={[15]}_{20}\).
    \textbf{Note}: the congruence with unknowns has infinite solutions, all with integers 
    congruent to \(15\) modulo \(20\), the equation in \(\mathbb{Z} / 20\) has a single equation.
\end{snippetexample}

\begin{snippet}{congruences-with-unknowns-expl1}
    In the general case we want to solve
    \[
        ax\equiv b \pmod{n} \qquad \text{and} \qquad {[a]}_n x = {[b]}_n
    \]
    We are looking for an \(x\) such that \(ax-b\) is a multiple of \(n\), that is
    \(ax-b = ny\) for some \(y\in\integers\).
    This is a diophantine equation
    \[
        ax-ny = b
    \]
    which has solutions \ifandonlyif \(\gcd(a,n) \divides b\)
\end{snippet}

\begin{snippetexample}{chinese-remainder-example-1}{}
    Consider
    \[
        6x \equiv 5 \pmod{14}
    \]
    and, equivalently \[
        {[6]}_{14} x = {[5]}_{14}
    \]
    Since \(\gcd(6,14) = 2\) does not divide \(5\) the congruence
    and the corresponding equation have no solution.
\end{snippetexample}

\begin{snippetexample}{chinese-remainder-example-2}{}
    Consider
    \[
        6x \equiv 10 \pmod{14}
    \]
    and, equivalently \[
        {[6]}_{14} x = {[10]}_{14}
    \]
    Since \(\gcd(6,14) = 2 \divides 10\), the equation ha ssolutions
    which we can find by solving the diophantine equation
    \[
        6x - 14y = 10 \iff 3x - 7y = 5
    \]
    A Bezout's identity between \(3\) and \(7\) is given by
    \(3\cdot (-2) + 7 \cdot 1 = 1\).
    By multiplying by \(5\) we get \(3\cdot (-10) + 7 \cdot 5 = 5\),
    so a particular solution is given by \(x=-10\) and \(y = -5\)
    and the general one is \(x=-10 + 7h\) and \(y-5+3h\).
    This means that the congruence has solutions
    \(x \equiv -10 \pmod{7}\) (when I multiply it by \(6\) I get a number modulo \(14\))
    and the corresponding equation has as solutions the classes modulo \(14\)
    represented by integers congruence to \(-10\) modulo 3.
    \[
        x = {[-10]}_{14} = {[4]}_{14}
    \]
    and
    \[
        x = {[-3]}_{14}
    \]
    The amount of solutions is thus \(2 = \gcd(6, 14)\).
\end{snippetexample}

% questa cosa qua è data dal fatto che diviso l'equazione per d

\begin{snippetproposition}{congruence-with-unknown-solution}{}
    If we consider the congruence \(ax \equiv b \pmod{n}\)
    and the corresponding equation \({[a]}_n x = {[b]}_n\) e \(d = \gcd(a,b) \divides n\),
    then the congruence is solvable and is equivalent to
    \[
        a'x \equiv b' \pmod{n'}
    \]
    with \(a=a'd\), \(b = b'd\), \(n=n'd\).
    All the classes represented by integers satisfying the given congruence, are solutions
    of the equation in \(\integers / n \).
    If \(c\) is a particular solution of the congruence,
    we have the following classes:
    \[
        {[c]}_n, {[c + n']}_n, {[c + 2n']}_n, \cdots, {[c + (d-1)n']}_n
    \]
    We note that these classes are all distinct (the difference between their representatives
    are not multiples of \(n\)) and every other class of form
    \({[c+kn']}_n\) shows up in these.
    More precisely, \({[c+kn']}_n + {[c+rn']}_n\) with \(r\) being the remainder of the division
    of \(k\) by \(d\).
\end{snippetproposition}

\begin{snippettheorem}{chinese-remainder-theorem}{Chinese remainder theorem}
    Consider the system of congruences
    \[
        \begin{cases}
            x \equiv a_1 \pmod{n_1} \\
            x \equiv a_2 \pmod{n_2} \\
            \cdots \\
            x \equiv a_r \pmod{n_r}
        \end{cases}
    \]
    with \(a_i\in\integers\) and \(n_i \in\integers^+\)
    where \(n_i, n_{i+1}\) are \coprime.
    Then, the given system is equivalent to a congruence
    of the form \(x \equiv a \pmod{n}\)
    for some \(a\) and
    \[
        n = \prod_{i=1}^r n_i
    \]
    
\end{snippettheorem}

\begin{snippetproof}{chinese-remainder-theorem-proof}{chinese-remainder-theorem}{Chinese remainder theorem}
    \begin{itemize}
        \item The base case \(r=1\) is trivial (the congruence is equivalent to itself).
        \item Let \(r > 1\). We consider the first two congruences
        \[
            \begin{cases}
                x \equiv a_1 \pmod{n_1} \\
                x \equiv a_2 \pmod{n_2} \\
            \end{cases}
        \]
        We need to determine the integers \(x\) such that
        \(x = a_1 + n_1h\) for some \(h\in\integers\) and
        \(x = a_2 + n_2k\) for some \(k\in\integers\).
        We thus get
        \[
            a_1 + n_1 h = a_2 + n_2 k
        \]
        which is the diophantine equation
        \[
            n_1h - n_2 k = a_2 - a_1
        \]
        Since \(n_1\) and \(n_2\) are \coprime, \(\gcd(n_1, n_2) = 1 \divides a_2 - a_1\)
        and thus solutions exist.
        However, we want to show that the solutions of the system
        \[
            \begin{cases}
                x \equiv a_1 \pmod{n_1} \\
                x \equiv a_2 \pmod{n_2} \\
            \end{cases}
        \]
        are only the ones of type \(a + tn_1n_2\)
        with a particular solution (determined as done above)
        with \(t\in\integers\).
        In order to achieve this, we consider the following facts:
        \begin{enumerate}
            \item every integers of type \(a+tn_1n_2\) is a solution.
            Indeed, \(a+tn_1n_2 \equiv a \pmod{n_1} \equiv a_1 \pmod{n_1}\)
            and \(a+tn_1n_2 \equiv a \pmod{n_2} \equiv a_2 \pmod{n_2}\);
            \item every solution \(b\) of the system has form \(b = a + tn_1n_2\)
            for some \(t\in\integers\).
            Indeed, \(b\equiv a_1 \pmod{n_1}\), \(a \equiv a_1 \pmod{n_1}\)
            and thus \(b \equiv a \pmod{n_1}\) which means \(b-a\)
            is a multiple of \(n_1\).
            Likewise, \(b-a\) is a multiple of \(n_2\).
            It follows that \(b-a\) is a multiple of \(n_1\) and \(n_2\), which are
            \coprime, and \(b-a = tn_1n_2\).
        \end{enumerate}
        We just showed that the first two congruences are equivalent to a congruence of type
        \[
            x \equiv a \pmod{n_1n_2}
        \]
        Thus, the system is reduced to
        \[
            \begin{cases}
                x \equiv a_1 \pmod{n_1n_2} \\
                x \equiv a_3 \pmod{n_3} \\
                \cdots \\
                x \equiv a_r \pmod{n_r}
            \end{cases}
        \]
        We iterate this procedure to obtain the thesis.
    \end{itemize}
\end{snippetproof}

\section{Systems where congruences are not coprime}

\begin{snippet}{congruences-systems-not-coprime-expl1}
    Consider a system of the form
    \[
        \begin{cases}
            x \equiv a_1 \pmod{n_1} \\
            x \equiv a_3 \pmod{n_2} \\
            \cdots \\
            x \equiv a_r \pmod{n_r}
        \end{cases}
    \]
    where \(n_1\) cannot be arranged in pairs where every pair is \coprime.
\end{snippet}

\begin{snippetexample}{congruences-systems-not-coprime-example-1}{}
    Consider
    \[
        \begin{cases}
            x \equiv 3 \pmod{12} \\
            x \equiv 6 \pmod{15} \\
            x \equiv 1 \pmod{40}
        \end{cases}
    \]
    For each congruence, we express the module as a product of prime factors
    and we use the chinese remainder theorem in reverse such that the modules
    are powers of primes. For instance, \(12 = 2^3 \cdot 3\), thus \(x \equiv 3 \pmod{12}\)
    is equivalent to
    \[
        \begin{cases}
            x \equiv 3 \pmod{2^2} \\
            x \equiv 3 \pmod{3}
        \end{cases}
    \]
    We repeat this process for every congruence, and find a system like the following
    \[
        \begin{cases}
            x \equiv 3 \pmod{2^2} \\
            x \equiv 3 \pmod{3} \\
            x \equiv 6 \pmod{3} \\
            x \equiv 6 \pmod{5} \\
            x \equiv 11 \pmod{2^3} \\
            x \equiv 11 \pmod{5}
        \end{cases}
    \]
    Now, every pair is either \coprime or powers of the same prime.
    We confront the ones which are relative to powers of the same prime,
    and we check whether they're compatible or if one implies the other.
    We have that \(x\equiv 11 \pmod{2^3} \implies x \equiv 3 \pmod{2^2}\).
    We also have that \(x \equiv 6 \pmod{3} \iff x \equiv 3 \pmod{3}\).
    The same goes for \(x \equiv 6 \pmod{5}\) and \(x \equiv 11 \pmod{5}\).
    \[
        \begin{cases}
            x \equiv 3 \pmod{8} \\
            x \equiv 3 \pmod{3} \\
            x \equiv 1 \pmod{5}
        \end{cases}
    \]
    We can now use the chinese remainder theorem.
    We start with the first two congruences, which is trivial without extra steps.
    \[
        \begin{cases}
            x \equiv 3 \pmod{24} \\
            x \equiv 1 \pmod{5} \\
        \end{cases}
    \]
    Now, we must find \(x=3+24h = 1 + 5k\) with \(h,k \in \integers\).
    We have the linear diophantine equation
    \[ 24h - 5k = -2 \]
    A Bezout's identity is given by
    \[
        24(-1) + 5\cdot 5 = 1
    \]
    We multiply by \(-2\) and get
    \[
        24\cdot 2 - 5\cdot 10 = -2
    \]
    so \(h=2\) and \(k=10\) is a particular solution.
    Thus, we can consider \(x = 3 + 24\cdot 2 = 51\) or \(x = 1 + 10\cdot 5 = 51\)
    and get \(x \equiv 51 \pmod{24\cdot 5} \equiv 51 \pmod{120}\).
\end{snippetexample}

\begin{snippetexample}{congruences-systems-not-coprime-example-2}{}
    Consider
    \[
        \begin{cases}
            x \equiv 5 \pmod{12} \\
            x \equiv 4 \pmod{10} \\
            x \equiv 2 \pmod{15}
        \end{cases}
    \]
    we split this into
    \[
        \begin{cases}
            x \equiv 5 \pmod{2^2} \\
            x \equiv 5 \pmod{3} \\
            x \equiv 4 \pmod{5} \\
            x \equiv 4 \pmod{2} \\
            x \equiv 2 \pmod{5} \\
            x \equiv 2 \pmod{3}
        \end{cases}
    \]
    We note that \(x \equiv 4 \pmod{2}\) and \(x \equiv 5 \pmod{2^2}\)
    are incompatible, so the system has no solutions.
\end{snippetexample}

\end{document}