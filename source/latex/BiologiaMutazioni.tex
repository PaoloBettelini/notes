\documentclass[preview]{standalone}

\usepackage{amsmath}
\usepackage{amssymb}
\usepackage{stellar}
\usepackage{bettelini}

\hypersetup{
    colorlinks=true,
    linkcolor=black,
    urlcolor=blue,
    pdftitle={Stellar},
    pdfpagemode=FullScreen,
}

\begin{document}

\id{biologia-mutazioni}
\genpage

\section{Le mutazioni}

\plain{Le mutazioni geniche sono caratterizzate da mutazioni di uno o più nucleotidi.}

\begin{snippetdefinition}{mutazioni-definition}{Mutazioni}
    Una \textit{mutazione} è una qualsiasi modifica alla sequenza del DNA.
\end{snippetdefinition}

\plain{Una mutazione potrebbe consistere anche in una mutazione di un pezzo di cromosoma o, addirittura,
di un cromosoma in più.}

\begin{snippet}{mutazioni-lista}
    Le mutazioni possono occorrere:
    \begin{itemize}
        \item \textbf{nei gameti}: come mella meiosi.
            In questo caso il problema di vedrà nella generazione successiva;
        \item \textbf{nelle cellule somatiche}: come nella mitosi.
            In questo caso il problema è delle cellule figlie dello steso organismo.
    \end{itemize}
\end{snippet}

\begin{snippetdefinition}{mutazione-genica-definition}{Mutazione genica}
    Una \textit{mutazione genica} consiste nella mutazione
    di uno o qualche nucleotide.
\end{snippetdefinition}

\begin{snippetdefinition}{mutazioni-cromosomiche-definition}{Mutazioni cromosomiche}
    Una \textit{mutazione cromosomica} consiste nella mutazione
    del numero di cromosomi nei cariotipo.
\end{snippetdefinition}

\begin{snippetdefinition}{mutazione-genomica-definition}{Mutazione genomica}
    Una \textit{mutazione genomica} consiste nella mutazione
    di una grande parte di un cromosoma.
\end{snippetdefinition}

\begin{snippetexample}{mutazione-genica-example}{Mutazione genica}
    La \textit{fibrosi cistica} è una malattia dovuto all'accumulo di muco sul'epitelio dei tessuti.
    Il problema maggiore si presenta quando questo muco si posa sul polmone, causando problemi
    allo scambio di ossigeno. Il problema è dovuto alla carenza di ioni cloro che sciolgono il muco.
    Una mutazione genica della sequenza di DNA concernente questa proteina può causare
    una struttura diversa della proteina, e quindi una mancata funzionalità. La proteina mutata
    non riesce a far passare gli ioni cloro.
\end{snippetexample}

\begin{snippet}{tipi-mutazioni-geniche}
    Vi sono tre tipi di mutazioni geniche:
    \begin{itemize}
        \item \textbf{sostituzione:} un nucleotide viene sostituito;
        \item \textbf{delezione:} un nucleotide viene cancellato;
        \item \textbf{inserzione:} un nucleotide viene aggiunto.
    \end{itemize}
\end{snippet}

\begin{snippet}{0ed923ec-1279-45cb-bee1-9e647addcc13}
    La fibrosi cistica è dovuta ad una delezione, vengono infatti cancellati tre nucleotidi
    del gene CFTR (nucleotidi CTT nella sequenza di DNA).
\end{snippet}

\begin{snippetdefinition}{agente-mutagene-definition}{Agente mutagene}
    Il \textit{agente mutagene} è un agente chimico o fisico che altera permanentemente
    un pezzo di materiale genetico, spesso DNA.
\end{snippetdefinition}

\plain{Le mutazioni sono sempre dovuto ad agenti mutageni oppure ad errori nella duplicazione del DNA.}

\begin{snippetexample}{mutagene-example}{Mutagene}
    Un esempio di mutagene potrebbe essere per esempio una radiazione.
\end{snippetexample}

\end{document}