\documentclass[preview]{standalone}

\usepackage{amsmath}
\usepackage{amssymb}
\usepackage{stellar}

\hypersetup{
    colorlinks=true,
    linkcolor=black,
    urlcolor=blue,
    pdftitle={Stellar},
    pdfpagemode=FullScreen,
}

\begin{document}

\title{Stellar}
\id{italiano-decameron-introduzione}
\genpage

\section{Introduzione al Decameron}

\begin{snippet}{decameron-introduzione}
    L'opera è ambientata durante l'epidemia di peste del 1348 a Firenze
    e segue un gruppo di dieci giovani aristocratici (sette donne e tre uomini)
    che cercano rifugio nella campagna toscana per sfuggire alla peste.
    Per passare il tempo, ciascuno di loro racconta una novella ogni giorno,
    per un totale di cento novelle in dieci giorni (14 giorni totali).
    Queste novelle spaziano in temi e argomenti, toccando la vita, l'amore, l'umorismo,
    l'ingiustizia sociale, la morale, la religione e molti altri aspetti della società medievale.
    
    L'opera è strutturata nella seguente maniera:
    \begin{itemize}
        \item \textbf{Introduzione e Proemio:} Inizia con una breve introduzione che spiega le circostanze della fuga dei giovani dalla città e il loro soggiorno in campagna. Il proemio presenta anche i temi principali e lo scopo dell'opera.
            Boccaccio parla della propria sofferenza amorosa, e scrive il Decameron come omaggio alle donne (borghesi, sensibili, che non posso distrarsi essendo confinate in casa).
        \item \textbf{Dieci Giornate:} Ogni giornata rappresenta una sezione dell'opera in cui i personaggi raccontano le novelle. Ogni giornata è caratterizzata da un tema generale o un'idea predominante scelta da un re o una regina che viene eletto ogni giorno.
        \item \textbf{Novelle:} Ogni giornata contiene dieci novelle narrate dai personaggi, ciascuna di esse seguendo il tema giornaliero. Le novelle sono scritte in prosa ed esprimono la varietà delle esperienze umane.
        \item \textbf{Conclusione:} L'opera si conclude con una breve conclusione in cui Boccaccio parla dell'importanza dell'amicizia, dell'amore e della sorte.
    \end{itemize}
\end{snippet}

\begin{snippetdefinition}{metanarrazione-definizione}{Metanarrazione}
    La \textit{metanarrazione} è una tecnica narrativa,
    che consiste nell'intervento diretto dell'autore all'interno dello stesso testo che va componendo
    Si verifica così una narrazione che assume come proprio oggetto l'atto stesso del raccontare,
    così da sviluppare un romanzo nel romanzo.
\end{snippetdefinition}

\end{document}