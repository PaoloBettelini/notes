\documentclass[preview]{standalone}

\usepackage{amsmath}
\usepackage{amssymb}
\usepackage{parskip}
\usepackage{fullpage}
\usepackage{hyperref}
\usepackage{bettelini}
\usepackage{stellar}

\hypersetup{
    colorlinks=true,
    linkcolor=black,
    urlcolor=blue,
    pdftitle={Cyclic groups},
    pdfpagemode=FullScreen,
}

\begin{document}

\id{cyclid-groups}
\genpage

\begin{snippetdefinition}{cyclic-group-definition}{Cyclic Group}
    Let \(G\) be a group. For any element \(g\) in \(G\),
    the \textit{cyclic group} generated by \(g\), denoted \(\langle g \rangle\),
    is defined as the subgroup of \(G\) with element
    \[
        \{ g^k \suchthat k \in \mathbb{Z} \}
    \]
\end{snippetdefinition}

\begin{snippetcorollary}{cyclic-is-abelian}{Cyclic groups are abelian}
    Let \(G\) be a group. If \(G\) is cyclic, then it is abelian.
\end{snippetcorollary}

% PROOF

\begin{snippettheorem}{subgroups-of-cyclic-are-cyclic}{Subgroups of cyclic groups}
    Any subgroup of a cyclic group is cyclic.
\end{snippettheorem}

\end{document}
