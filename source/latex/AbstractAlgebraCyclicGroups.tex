\documentclass[preview]{standalone}

\usepackage{amsmath}
\usepackage{amssymb}
\usepackage{parskip}
\usepackage{fullpage}
\usepackage{hyperref}
\usepackage{bettelini}
\usepackage{stellar}
\usepackage{definitions}

\begin{document}

\id{cyclid-groups}
\genpage

\section{Cyclid groups}

\begin{snippetdefinition}{cyclic-group-definition}{Cyclic Group}
    Let \(G\) be a \group. For any element \(g\) in \(G\),
    the \textit{cyclic group} generated by \(g\)
    is defined as the \subgroup of \(G\) as follows:
    \[
        \langle g \rangle = \{ g^k \suchthat k \in \naturalnumbers \}
    \]
\end{snippetdefinition}

\begin{snippetcorollary}{cyclic-is-abelian}{Cyclic groups are abelian}
    Let \(G\) be a \group. If \(G\) is \cyclicgroup[cyclic], then it is \abeliangroup[abelian].
\end{snippetcorollary}

% PROOF

\begin{snippettheorem}{subgroups-of-cyclic-are-cyclic}{Subgroups of cyclic groups}
    Any \subgroup of a \cyclicgroup is \cyclicgroup[cyclic].
\end{snippettheorem}

\end{document}
