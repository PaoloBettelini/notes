\documentclass[preview]{standalone}

\usepackage{amsmath}
\usepackage{amssymb}
\usepackage{stellar}
\usepackage{definitions}

\begin{document}

\id{precession-motion}
\genpage

\section{Precession}

\begin{snippetdefinition}{precession-motion-definition}{Precession motion}
    \emph{Precession motion} refers to the phenomenon where the orientation of an axis of
    rotation or a vector changes over time due to an external torque or constraint.
\end{snippetdefinition}

\begin{snippetexample}{precession-motion-example}{}
    Consider \(\vec{a}(t)\) and \(\vec{w}\) with the condition that
    \[
        \frac{d\vec{a}}{dt} = \vec{w} \wedge \vec{a}
    \]
    where \(\vec{\omega}\) (fixed) is the angular velocity of the precession.
    We first note that \(|\vec{a}(t)|\) is constant.
    We have that
    \begin{align*}
        \frac{d}{dt} {|\vec{a}(t)|}^2 &= \frac{d}{dt} \vec{a}\cdot\vec{a}
        = \vec{a}\frac{d\vec{a}}{dt} + \frac{d\vec{a}}{dt} \vec{a} 
        = 2\vec{a}\frac{d\vec{a}}{dt}
        = 2 \vec{a} \cdot (\vec{w} \wedge \vec{a})
        = 0
    \end{align*}
    We define out cartesian system with the condition that \(\hat{z}\) has the direction
    direction and length as \(\vec{w}\), thus \(\vec{w} = w\hat{z}\).
    As a second fact we have that \(a_z\) is independent of time.
    Indeed,
    \[
        \frac{da_z}{dt} = \frac{d\vec{a}\hat{z}}{dt} =  \hat{z}\frac{d\vec{a}}{dt}
        = \hat{z} \cdot (\vec{\omega} \wedge \vec{a})
        = 0
    \]
    so it is constant. Geometrically, \(\vec{a}\) creates a cone.
    Now, \(\vec{a}_\perp^2 = a^2 - a_z^2\) which is independent of \(t\),
    and \(a_x = a_T \cos\varphi\) where \(\varphi\) is the angle between \(\hat{x}\)
    and the projection \(a_\perp\) (on the \(xy\) plane).
    \[
        \begin{cases}
            a_x(t) = a_\perp \cos\varphi(t) \\
            a_y(t) = a_\perp \sin\varphi(t) \\
            a_t
        \end{cases}
    \]
    We now have
    \begin{align*}
        \frac{da_x}{dt} &= {(\vec{w} \wedge \vec{a})}_x
        = -\omega a_y \\
        \frac{da_x}{dt} &= {(\vec{w} \wedge \vec{a})}_y
        = \omega a_x \\
        \frac{da_z}{dt} &= {(\vec{w} \wedge \vec{a})}_z
        = 0
    \end{align*}
    We can substitute the parametrization
    \begin{align*}
        \frac{da_x}{dt} &= -\omega a_y \implies a_\perp (-\sin(\varphi(t))) \cdot \frac{d\varphi}{dt} = -\omega a_\perp \sin\varphi(t) \\
        \frac{da_y}{dt} &= -\omega a_x \implies a_\perp \cos(\varphi(t)) \cdot \frac{d\varphi}{dt} = \omega a_\perp \cos\varphi(t) \\
        \frac{da_z}{dt} &= 0
    \end{align*}
    We note that simplifying these equations yields the same equation 
    \begin{align*}
        \frac{d\varphi}{dt} = \omega \\
        \frac{d\varphi}{dt} = \omega
    \end{align*}
    for \(a_\perp \neq 0\),
    which is obvious given the relation that we had established.
    Thus, the final solution is \(\varphi(t) = \varphi_0 + \omega t\).
    In conclusion,
    \[
        \begin{cases}
            a_x = a_\perp \cos(\omega t + \varphi_0) \\
            a_y = a_\perp \sin(\omega t + \varphi_0) \\
            a_z = a_z
        \end{cases}
    \]
\end{snippetexample}

\end{document}