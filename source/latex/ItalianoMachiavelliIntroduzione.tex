\documentclass[preview]{standalone}

\usepackage{amsmath}
\usepackage{amssymb}
\usepackage{stellar}

\hypersetup{
    colorlinks=true,
    linkcolor=black,
    urlcolor=blue,
    pdftitle={Stellar},
    pdfpagemode=FullScreen,
}

\begin{document}

\title{Stellar}
\id{italiano-machiavelli-introduzione}
\genpage

\section{Introduzione}

\begin{snippetdefinition}{trattatistica-definition}{La trattatistica}
    Un \textit{trattato} è un libro, solitamente scritto in latino,
    che descrive delle indicazioni comportamentali.
    Dietro ogni trattato vi è un'idea umanistica per perfezionare l'umano. \\
    Vige l'idea di demunicipalizzazione, ossia quella di distaccare
    il trattato dal municipio e corte locale, rendendoli più universali.
\end{snippetdefinition}

\begin{snippet}{machiavelli-introduzione}
    Un trattato può fissare uno dei seguenti modelli:
    \begin{enumerate}
        \item modelli di comportamento;
        \begin{enumerate}
            \item Machiavelli, \textit{Il Principe} (il perfetto principe);
            \item Castiglione, \textit{Il Cortigiano} (il perfetto uomo e donna di corte);
            \item Giovanni della Casa, \textit{Il galateo} (il perfetto uomo civile.)
        \end{enumerate}
        \item modelli artistico-letterari;
        \begin{enumerate}
            \item Castelvetro;
        \end{enumerate}
        \item modelli linguistici;
        \begin{enumerate}
            \item Pietro Bembo, \textit{Prose della 'volgar liunga},
                dove vengono indicati Boccaccio e Petrarca come modelli per i rispettivi tipi di testo.
                Dante nel 1500 viene inece considerato come un rozzo a livello linguistico;
            \item Machiavelli: di scrivere con la lingua odierna;
            \item Castiglione: la lingua perfetta nascerebbe dal
                meglio delle parlate di tutte le corti d'Italia (secondo la bellezza fonetica delle parole).
        \end{enumerate}
    \end{enumerate}
    
    Con Machiavelli nasce l'idea odierna di politica.
    I modelli scientifici sono trattati dai \textit{saggi} piuttosto che dai trattati.
\end{snippet}

\end{document}