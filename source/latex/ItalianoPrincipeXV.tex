\documentclass[preview]{standalone}

\usepackage{amsmath}
\usepackage{amssymb}
\usepackage{stellar}

\hypersetup{
    colorlinks=true,
    linkcolor=black,
    urlcolor=blue,
    pdftitle={Stellar},
    pdfpagemode=FullScreen,
}

\begin{document}

\title{Stellar}
\id{italiano-principe-capitolo-xv}
\genpage

\section{Capitolo XV}

\begin{snippet}{il-principe-capitolo-xv}
    Il capitolo XV risponde a come un principe si deve comportare
    con i sudditi e con gli amici (alleati).
    L'autore è consapevole di non essere il primo a parlare
    delle caratteristiche del principe (trattatistica politica).
    Infatti, in molti ne hanno già scritto e Machiavelli si
    aggiunge alla tradizione.
    Qui l'autore dichiara una novità: nonostante si inserisca in una tradizione,
    porta qualcosa di diverso.
    Il motivo di questa novità è che Machiavelli si distacca
    dagli orfini (categorie mentali, principi) altrui.
    Il suo scopo è quello di essere utile, ossia
    affinché i suoi printicipi abbiano una utilità pratica e applicata
    nella realtà.
    Per proporre il ritratto del principe perfetto, viene utilizzata
    la verità effettiva piuttosto che l'immaginazione.
    Secondo Machiavelli gli altri si sono sempre basati sull'immaginazione,
    ossia una idealizzazione del principe.
    Questi principi utopici erano sempre quelli che incorporavano tutte le virtù.
    Una sorta di catalogo astratto di virtù.
    Vi è talmente tanta differenza fra come si vive realmente (effettiva realtà)
    e come sarebbe bello vivere (immaginazione utopica) che
    individuo (in particolare, un partice), se si comporta in modo moralmente giusto
    perde il potere. La tesi è quindi che chi si comporta bene
    viene stroncato dagli altri, perché gli altri non si comportano bene.
    \\\\
    Uno è tenuto liberale (generoso), l'altro misero (avaro),
    l'uno effemminato e pusillanime (debole) e l'altro
    feroce ed animoso. L'uno umano l'altro superbo, l'uno lascivo (che si lascia andare alle voglie) e l'altro casto.
    L'uno intero (onesto), l'altro furbo (astuto).
    \\
    Tutte queste copie sono composte da parole antitetiche
    secondo un criterio morale. Di conseguenza abbiamo delle virtù associate a dei vizi.
    Secondo Machiavelli, sarebbe bellissimo se un principo avesse tutte queste virtù,
    ma dal momento che non si possono rispettare sempre, date le condizioni umane,
    va bene ignorarle se necessario.
    Inoltre, che il principe non si preoccupi di ricorrere a quei vizi,
    se sono necessari per salvare lo stato.
    \\\\
    % Dover giustificare presuppone che i mezzi siano sbagliati,
    ma nella prospettiva dell'autore, se le azioni sono necessarie
    sono corrette e non c'è nulla da giustificare.
    Questa prospettiva non è più morale ma sulla base di un criterio funzionale
    e politico.
    La morale ragiona su dei principi assoluti, ma Machiavelli porta i valori a
    dei principi relativi.
    A questo punto lealtà e svelatà sono dei termini neutri:
    possono essere positivi o negativi a seconda delle circostanze.
\end{snippet}

\end{document}