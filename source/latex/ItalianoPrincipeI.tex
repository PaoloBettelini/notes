\documentclass[preview]{standalone}

\usepackage{amsmath}
\usepackage{amssymb}
\usepackage{stellar}
\usepackage{definitions}

\begin{document}

\id{italiano-principe-capitolo-i}
\genpage

\section{Capitolo I}

\begin{snippet}{il-principe-capitolo-i-parte1}
    \begin{center}
        \begin{minipage}{0.75\textwidth}
            \itshape
            Tutti gli Stati, tutti i domini che hanno avuto, e hanno imperio sopra gli uomini, sono stati e sono o Repubbliche o Principati.
        \end{minipage}
    \end{center}
    \vspace{0.25cm}
    Un Principato si differenzia dalla repubblica in quanto la repubblica è più democratica e partecipativa,
    mentre il principato è incentrato su una singola persona o gruppo di persone. \\
    Come descritto dall'autore: I principati sono o nuovi (perché lo si conquista) o ereditari (lunga discendenza).
    I principati nuovi, a loro volta, hanno anch'essi due diramazioni:
    o sono completamente nuovi (\textit{come fu Milano a Francesco Sforza}),
    o sono come pezzi aggiunti ad una conquista
    (\textit{come è il Regno di Napoli al Re di Spagna}). \\
    Per quest'ultima diramazione, a differenza degli altri, Machiavelli cita degli esempi storici.
\end{snippet}

\begin{snippetnote}{machiavelli-parola-acquista}{}
    La parola \textit{acquista} è un termine tecnico (termine generico, ma in questo caso indica conquistare).
\end{snippetnote}

\begin{snippet}{il-principe-capitolo-i-parte2}
    O si conquista un popolo che è abituato ad essere libero, o che è abituato ad
    essere sotto il dominio di un principe.
    Inoltre, essi si conquistano o con le armi d'altri (mercenari, prestate o comprate),
    oppure con il proprio esercito di milizia.
    La terza ulteriore distinzione è quella più importante: si può conquistare per fortuna (tutto ciò che sfugge al calcolo umano),
    oppure per virtù.
    \\\\
    Questo concetto di virtù è riferito alle capacità tecniche del principe (non la connotazione odierna morale).
    Il principe virtuoso non è quindi necessariamente saggio, onesto etc., ma è colui che
    possiede le capacità tecniche.
\end{snippet}

\end{document}