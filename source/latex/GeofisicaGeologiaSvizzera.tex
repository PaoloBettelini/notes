\documentclass[preview]{standalone}

\usepackage{amsmath}
\usepackage{amssymb}
\usepackage{stellar}

\hypersetup{
    colorlinks=true,
    linkcolor=black,
    urlcolor=blue,
    pdftitle={Stellar},
    pdfpagemode=FullScreen,
}

\begin{document}

\title{Stellar}
\id{geofisica-geologia-svizzera}
\genpage

\section{Geologia della Svizzera}

\begin{snippet}{geologia-svizzera-expl}
    Il territorio svizzero si suddivide in Alpi, Altopiano centrale, Giura. I due terzi circa del territorio
    nazionale costituiscono le regioni alpine, solo una parte delle quali è permanentemente abitata,
    soprattutto le vallate e i pendii a terrazza esposti al sole; per il resto, la popolazione si distribuisce
    con densità minima sul rimanente territorio abitabile. Tra queste due vaste regioni
    morfologicamente determinate, si estende l'Altopiano, regione vitale della Svizzera. Qui si
    concentrano su uno spazio limitato i centri abitati: floridi paesi, piccole, medie e grandi città; qui
    abita la maggior parte della popolazione svizzera, qui si addensa la produzione industriale e si
    formano tuttora complessi agglomerati. Un buon dieci per cento della superficie totale è coperto
    dal Giura. La suddivisione della Svizzera in grandi regioni è stata determinata da processi geologici
    avvenuti prima che l'uomo esistesse.
\end{snippet}

\end{document}