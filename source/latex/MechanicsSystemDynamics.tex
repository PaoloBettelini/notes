\documentclass[preview]{standalone}

\usepackage{amsmath}
\usepackage{amssymb}
\usepackage{stellar}
\usepackage{definitions}
\usepackage{bettelini}

% https://tex.stackexchange.com/questions/120029/how-to-typeset-a-primed-vector
\newcommand{\pvec}[1]{\vec{#1}\mkern2mu\vphantom{#1}}

\begin{document}

\id{system-dynamics}
\genpage

\section{System dynamics}

\begin{snippetdefinition}{center-of-mass-n-particles-definition}{Center of mass}
    Consider \(n\) particles with masses \(m_1, m_2, \cdots, m_n\).
    The \emph{center of mass} is defined as
    \[
        \vec{R} \triangleq \frac{\sum_i m_1\vec{r}_i}{\sum_i m_i}
    \]
\end{snippetdefinition}

\begin{snippettheorem}{system-dynamics-first-cardinal-law-theorem}{First cardinal law}
    Consider \(n\) particles with masses \(m_1, m_2, \cdots, m_n\) mutually interacting and in the presence of
    external forces. Then,
    \[
        \frac{d\vec{p}}{dt} = \sum_i \vec{f}_i
    \]
    where \(\vec{f}_i\) are the external forces.
\end{snippettheorem}

\begin{snippetproof}{system-dynamics-first-cardinal-law-theorem-proof}{system-dynamics-first-cardinal-law-theorem}{First cardinal law}
    The Newton equation is given by
    \[
        m_i\vec{a}_i = \sum_{j \neq i} \vec{F}_{i,j} + \vec{f}_{i}
    \]
    We now write
    \[
        \sum_i m_i\vec{a}_i = \sum_i \sum_{j \neq i} \vec{F}_{i,j} + \sum_i \vec{f}_i
    \]
    Since \(\vec{F}_{i,j} = -\vec{F}_{j,i}\), we can simplify
    \[
        \sum_i \sum_{j \neq i} \vec{F}_{i,j} = 0
    \]
    We can define the total momentum
    \[
        \vec{p} = \sum_i m_i\vec{v}_i \qquad \frac{d\vec{p}}{dt} = \sum_i m_i\vec{a}_i
    \]
    Given these two facts we can note that the variation of the momentum is just the sum
    of the external forces.
\end{snippetproof}

\begin{snippettheorem}{system-dynamics-second-cardinal-law-theorem}{Second cardinal law}
    Consider \(n\) particles with masses \(m_1, m_2, \cdots, m_n\) mutually interacting and in the presence of
    external forces. Then,
    \[
        \frac{d\vec{L}}{dt} = \sum_i \vec{r}_i \wedge \vec{f}_i
    \]
    which is the momentum of the external forces
\end{snippettheorem}

\begin{snippetproof}{system-dynamics-second-cardinal-law-theorem-proof}{system-dynamics-second-cardinal-law-theorem}{Second cardinal law}
    The derivative of the center of mass is
    \begin{align*}
        \vec{V} = \frac{d\vec{R}}{dt} &= \frac{\sum_i m_1\vec{v}_i}{\sum_i m_i} = \frac{\vec{p}}{M}
    \end{align*}
    We now start with the Newton equation and multiply by \(\vec{r}_i\) on the left
    and sum with respect to \(i\)
    \begin{align*}
        m_i\vec{a}_i &= \sum_{j \neq i} \vec{F}_{i,j} + \vec{f}_{i} \\
        \sum_i \vec{r}_i \wedge (m_i\vec{a}_i) &= \sum_i \sum_{j \neq i} \vec{r}_i \wedge \vec{F}_{i,j} + \sum_i \vec{r}_i \wedge \vec{f}_{i}
    \end{align*}
    We now consider the angular momentum
    \[
        \vec{L}_i = \vec{r}_i \wedge m_1 \vec{v}_i
    \]
    whose derivative is
    \begin{align*}
        \frac{d\vec{L}_i}{dt} = \vec{v}_i \wedge m_i\vec{v}_i + \vec{r}_i \wedge m_1\vec{a}_i
    \end{align*}
    The first term is null and the angular momentum of the \(i\)-nth particle remans.
    We thus define the total angular momentum of the system
    \[
        \vec{L} = \sum_i \vec{L}_i = \sum_i \vec{r}_i \wedge m_1\vec{v}_i
    \]
    We now consider the term
    \[
        \sum_i \sum_{j\neq i} \vec{r}_i \wedge \vec{F}_{i,j}
    \]
    Given two particles \(a\) and \(b\) we have
    \begin{align*}
        \vec{r}_a \wedge \vec{F}_{a,b} - \vec{r}_b \wedge \vec{F}_{a,b}
        &= (\vec{r}_a - \vec{r}_b) \wedge \vec{F}_{a,b}
    \end{align*}
    If we impose the condition that a force between a pair of particles is oriented
    towards the joining of the two (which is valid for a large class of forces),
    then the product is zero.
    In such case,
    \[
        \sum_i \sum_{j\neq i} \vec{r}_i \wedge \vec{F}_{i,j} = 0
    \]
\end{snippetproof}

\end{document}