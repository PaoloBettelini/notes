\documentclass[preview]{standalone}

\usepackage{amsmath}
\usepackage{amssymb}
\usepackage{stellar}
\usepackage{bettelini}

\hypersetup{
    colorlinks=true,
    linkcolor=black,
    urlcolor=blue,
    pdftitle={Biologia},
    pdfpagemode=FullScreen,
}

\begin{document}

\id{biologia-processi-facolta-eucariote}
\genpage

\section{Le funzioni di una cellula eucariote}

\begin{snippet}{funzioni-cellula-eucariote}
    Le funzioni principali di una cellula eucariote possono essere:

    \begin{itemize}
        \item mitosi;
        \item meiosi;
        \item sopravvive (nutrimento, respirazione);
        \item essere uccisa (es. dal sistema immunitario);
        \item morire (anche suicida, apoptosi);
        \item svolge la propria mansione;
        \item mutare (diventare cancerogena).
    \end{itemize}
\end{snippet}

\end{document}