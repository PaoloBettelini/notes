\documentclass[preview]{standalone}

\usepackage{amsmath}
\usepackage{amssymb}
\usepackage{tikz}
\usepackage{stellar}
\usepackage{definitions}
\usepackage{bettelini}

\begin{document}

\id{geofisica-tempo-clima}
\genpage

\section{Tempo e clima}

\begin{snippetdefinition}{tempo-meteorologico}{Tempo meteorologico}
    Con \textit{tempo meteorologico}
    si intendono le condizioni momentanee
    dell'atmosfera, che variano di ora
    in ora, di giorno in giorno.
\end{snippetdefinition}

\begin{snippetdefinition}{clima}{Clima}
    Con \textit{clima}
    si intendono le condizioni meteorologiche
    medie di una regione, osservate
    su un arco di almeno 30 anni.
\end{snippetdefinition}

\begin{snippetdefinition}{koppen-classification}{La classificazione climatica di Köppen}
    La \textit{classificazione climatica di Köppen} è una delle classificazioni climatiche più usate.
    Questa classificazione include fattori come: temperatura media annua, precipitazioni medie mensili,
    precipitazioni medie annue, tipo di associazione vegetale corrispondente e temperature medie mensili.
\end{snippetdefinition}

\end{document}