\documentclass[preview]{standalone}

\usepackage{amsmath}
\usepackage{amssymb}
\usepackage{stellar}
\usepackage{definitions}

\begin{document}

\id{settheory-relations-examples}
\genpage

\section{Equivalence classes}

\begin{snippetexample}{equivalence-relation-example-1}{Equivalence relation}
    Given a \set \(A\), the equality relation is an \equivrelation \(\sim\) on \(A\)
    \[ \{ (a,a) \suchthat a \in A \} \]
\end{snippetexample}

\begin{snippetexample}{equivalence-relation-example-2}{Equivalence relation}
    Given a \set \(A\), the \binrelation where every element is in relation with every other element
    is an \equivrelation \(\sim\) on \(A\)
    \[ \{ (a,b) \suchthat a,b \in A \} \]
\end{snippetexample}

\begin{snippetexample}{equivalence-relation-example-3}{Equivalence relation}
    The \binrelation \(\sim\) on \(\integers\) defined as
    \[ \{ (a,b) \in \integers^2 \suchthat |a| = |b| \} \]
    is an \equivrelation.
\end{snippetexample}

%\begin{snippet}{greater-or-less-is-no-equivalence-relation}
%    The \binrelation defined by
%    \[ \{ (a,b) \in \integers^2 \suchthat ab > 0 \} \]
% is not an \equivrelation because it is not reflexive (when \(a=0\lor b=0\).
% If we replace the \(>\) sign with \(\geq\), is it reflexive but not transitive.

\begin{snippetexample}{equivalence-relation-example-5}{Equivalence relation}
    The \binrelation \(\sim\) on \(\naturalnumbers\) defined as
    \[ \{ (a,b) \in \naturalnumbers^2 \suchthat a+b = 2k, k \in \naturalnumbers \} \]
    is an \equivrelation.
    The only non-trivial property is transitivity; if \(a + b\) and \(b+c\) are even, then so is \(a+c+2b\).
    But since \(2b\) is even, then \(a+c\) must also be even.
\end{snippetexample}

\plain{The same would not work if the sum needed to be odd, as the relation would not be reflexive and transitive.}

\section{Orders}

\begin{snippetexample}{partial-order-example-1}{Non-total partial order}
    Let \(X\) be a \set and \(A,B \in \powerset(X)\).
    The ordering given by \(a \leq b \iff A \subseteq B\) is a \partialorder. \\
    This is a \totalorder only if \(\cardinality{X}=1\). % also emptyset?
\end{snippetexample}

\begin{snippetexample}{partial-order-example-2}{Partial Order}
    Let \(X\) be a \set and \(A = \powerset(X)\).
    The relation of inclusion \(\in\) is a \partialorder in \(A\). \\
    The relation is a \totalorder only if \(\cardinality{X} \leq 1\).
\end{snippetexample}

\begin{snippetexample}{partial-order-example-3}{Partial Order}
    The relation of divisibility \(a \divides b\) in \(\naturalnumbers\)
    is a \partialorder. It is not total because there are numbers which are not in relation
    which eachother.
\end{snippetexample}

\plain{In the integers this relation would not be antisymmetric.}

\end{document}