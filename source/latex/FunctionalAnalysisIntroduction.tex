\documentclass[preview]{standalone}

\usepackage{amsmath}
\usepackage{amssymb}
\usepackage{stellar}
\usepackage{definitions}
\usepackage{bettelini}

\begin{document}

\id{functional-analysis-introduction}
\genpage

\section{Introduction to Functional Analysis}

\begin{snippet}{functional-analysis-expl1}
    Functional Analysis is a branch of mathematical analysis, concerned with the study of vector spaces and operators acting upon them. It has profound applications in various fields such as quantum mechanics, signal processing, and more. This course will guide you through the fundamental concepts of functional analysis, starting from basic definitions to advanced theorems.
\end{snippet}

\begin{snippetdefinition}{normed-vector-space}{Normed Vector Space}
    A \textit{normed vector space} is a vector space \(V\) along with a function \(\|\cdot\|: V \to \realnumbers\), called the norm, which satisfies the following properties for all \(x, y \in V\) and \(a \in \realnumbers\):
    \begin{itemize}
        \item \(\|x\| \geq 0\) and \(\|x\| = 0\) if and only if \(x = 0\),
        \item \(\|ax\| = |a|\|x\|\),
        \item \(\|x + y\| \leq \|x\| + \|y\|\) (Triangle Inequality).
    \end{itemize}
\end{snippetdefinition}

\begin{snippetdefinition}{banach-space}{Banach Space}
    A \textit{Banach space} is a normed vector space \(V\) that is complete with respect to the norm \(\|\cdot\|\), meaning that every Cauchy sequence in \(V\) converges to an element in \(V\).
\end{snippetdefinition}

\begin{snippettheorem}{banach-space-completeness}{Completeness of Banach Spaces}
    Let \(V\) be a normed vector space. If \(V\) is a Banach space, then for every Cauchy sequence \(\{x_n\}\) in \(V\), there exists an \(x \in V\) such that \(\|x_n - x\| \to 0\) as \(n \to \infty\).
\end{snippettheorem}

\begin{snippetexample}{sequence-space-lp}{Sequence Spaces \(l^p\)}
    Consider the space \(l^p\) for \(1 \leq p < \infty\), defined as
    \[
        l^p = \left\{ \{x_n\} \subset \realnumbers : \left( \sum_{n=1}^{\infty} |x_n|^p \right)^{1/p} < \infty \right\}.
    \]
    The norm on \(l^p\) is given by
    \[
        \|\{x_n\}\|_p = \left( \sum_{n=1}^{\infty} |x_n|^p \right)^{1/p}.
    \]
    \(l^p\) is a Banach space with this norm.
\end{snippetexample}

\begin{snippetdefinition}{inner-product-space}{Inner Product Space}
    An \textit{inner product space} is a vector space \(V\) along with a function \(\langle \cdot, \cdot \rangle: V \times V \to \realnumbers\), called the inner product, which satisfies the following properties for all \(x, y, z \in V\) and \(a \in \realnumbers\):
    \begin{itemize}
        \item \(\langle x, x \rangle \geq 0\) and \(\langle x, x \rangle = 0\) if and only if \(x = 0\),
        \item \(\langle x, y \rangle = \langle y, x \rangle\),
        \item \(\langle ax, y \rangle = a \langle x, y \rangle\),
        \item \(\langle x + y, z \rangle = \langle x, z \rangle + \langle y, z \rangle\).
    \end{itemize}
    The norm induced by the inner product is \(\|x\| = \sqrt{\langle x, x \rangle}\).
\end{snippetdefinition}

\begin{snippetdefinition}{hilbert-space}{Hilbert Space}
    A \textit{Hilbert space} is an inner product space that is complete with respect to the norm induced by the inner product.
\end{snippetdefinition}

\begin{snippettheorem}{orthogonal-projection-theorem}{Orthogonal Projection Theorem}
    Let \(H\) be a Hilbert space and let \(M\) be a closed subspace of \(H\). For every \(x \in H\), there exists a unique \(y \in M\) such that \(x - y \in M^\perp\), where \(M^\perp\) is the orthogonal complement of \(M\).
\end{snippettheorem}

\begin{snippettheorem}{riesz-representation-theorem}{Riesz Representation Theorem}
    Let \(H\) be a Hilbert space. For every continuous linear functional \(f\) on \(H\), there exists a unique \(y \in H\) such that \(f(x) = \langle x, y \rangle\) for all \(x \in H\).
\end{snippettheorem}

\begin{snippetexample}{hilbert-space-l2}{Hilbert Space \(l^2\)}
    The space \(l^2\) is defined as
    \[
        l^2 = \left\{ \{x_n\} \subset \realnumbers : \sum_{n=1}^{\infty} |x_n|^2 < \infty \right\}.
    \]
    The inner product on \(l^2\) is given by
    \[
        \langle \{x_n\}, \{y_n\} \rangle = \sum_{n=1}^{\infty} x_n y_n.
    \]
    \(l^2\) is a Hilbert space with this inner product.
\end{snippetexample}

\begin{snippetdefinition}{bounded-operator}{Bounded Operator}
    Let \(V\) and \(W\) be normed vector spaces. A linear operator \(T: V \to W\) is called \textit{bounded} if there exists a constant \(C \geq 0\) such that \(\|T(x)\|_W \leq C\|x\|_V\) for all \(x \in V\).
\end{snippetdefinition}

\begin{snippettheorem}{bounded-operator-continuous}{Bounded Operators are Continuous}
    Let \(V\) and \(W\) be normed vector spaces. If \(T: V \to W\) is a bounded linear operator, then \(T\) is continuous.
\end{snippettheorem}

\begin{snippetproof}{bounded-operator-continuous-proof}{Proof of Continuity for Bounded Operators}
    Let \(T: V \to W\) be a bounded linear operator. By definition, there exists a constant \(C \geq 0\) such that \(\|T(x)\|_W \leq C\|x\|_V\) for all \(x \in V\). To show that \(T\) is continuous, let \(\{x_n\}\) be a sequence in \(V\) such that \(x_n \to x\). We need to show that \(T(x_n) \to T(x)\) in \(W\).
    \[
        \|T(x_n) - T(x)\|_W = \|T(x_n - x)\|_W \leq C\|x_n - x\|_V.
    \]
    Since \(x_n \to x\) in \(V\), \(\|x_n - x\|_V \to 0\). Therefore, \(\|T(x_n) - T(x)\|_W \to 0\), which implies \(T(x_n) \to T(x)\). Hence, \(T\) is continuous.
\end{snippetproof}

\begin{snippettheorem}{hahn-banach-theorem}{Hahn-Banach Theorem}
    Let \(V\) be a normed vector space, and let \(p: V \to \realnumbers\) be a sublinear function. If \(f: U \to \realnumbers\) is a linear functional defined on a subspace \(U \subseteq V\) such that \(f(x) \leq p(x)\) for all \(x \in U\), then there exists an extension \(F: V \to \realnumbers\) of \(f\) to the whole space \(V\) such that \(F(x) \leq p(x)\) for all \(x \in V\).
\end{snippettheorem}

\begin{snippetexample}{bounded-operator-example}{Example of Bounded Operators}
    Consider the space \(l^2\) and the shift operator \(S: l^2 \to l^2\) defined by \(S(\{x_n\}) = \{0, x_1, x_2, x_3, \ldots\}\). This operator is linear and bounded with \(\|S\| = 1\).
\end{snippetexample}

\end{document}
