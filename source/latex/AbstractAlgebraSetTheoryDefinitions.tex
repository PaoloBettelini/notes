\documentclass[preview]{standalone}

\usepackage{amsmath}
\usepackage{amssymb}
\usepackage{fullpage}
\usepackage{bettelini}
\usepackage{stellar}

\hypersetup{
    colorlinks=true,
    linkcolor=black,
    urlcolor=blue,
    pdftitle={SetTheory},
    pdfpagemode=FullScreen,
}

\begin{document}

\title{Set Theory Definitions}
\id{settheory-definitions}
\genpage

% Russel's paradox

\section{Definitions}

\subsection{Cardinality}

\begin{snippetdefinition}{cardinality}{Cardinality}
    The \textit{cardinality} of a set \(A\), denoted \(|A|\),
    is the amount of elements it contains.
\end{snippetdefinition}

\subsection{Subset}

\begin{snippetdefinition}{subset}{Subset}
    If \(A\) and \(B\) are sets, then \(A\) is a \textit{subset} of \(B\)
    (\(A\subseteq B\)), if all the elements of \(A\) are also in \(B\).
\end{snippetdefinition}

\begin{snippetcorollary}{subset-of-itself}{Every set is subset of itself}
    For every set \(A\), \(A \subseteq A\).
\end{snippetcorollary}

\subsection{Proper Subset}

\begin{snippetdefinition}{propert-subset}{Proper Subset}
    Given two sets \(A\) and \(B\), if \(A \subseteq B\) but \(A \neq B\),
    then \(A\) is a \textit{proper} (or \textit{strict}) subset of \(B\)
    \[
        A \subset B
    \]
\end{snippetdefinition}

\subsection{Empty Set}

\begin{snippetdefinition}{empty-set}{Empty Set}
    The empty set \(\emptyset\) is a subset of all other sets.
    \[
        |\emptyset|=0
    \]
\end{snippetdefinition}

\begin{snippetcorollary}{empty-set-is-subset-of-any-set}{Empty set is subset of any set}
    For every set \(A\),
    \(\emptyset \subseteq A\).
\end{snippetcorollary}

\subsection{Power Set}

\begin{snippetdefinition}{power-set}{Proper Subset}
    If \(B\) is a set, then the \textit{power set} \(\mathcal{P}(B)\)
    is defined as the set of all subsets of \(B\)
    \[
        \mathcal{P}(B)=\{A \suchthat A\subseteq B\}
    \]
\end{snippetdefinition}

\begin{snippetcorollary}{subset-of-powerset}{Subset of powerset}
    For every set \(B\), \(B\in\mathcal{P}(B)\).
\end{snippetcorollary}

\begin{snippettheorem}{cardinality-of-the-power-set}{Cardinality of the power set}
    The cardinality of \(\mathcal{P}(A)\) is given by
    \[
        |\mathcal{P}(A)| = 2^{|A|}
    \]
\end{snippettheorem}

\subsection{Union}

\begin{snippetdefinition}{union}{Union}
    If \(A\) and \(B\) are sets, then their \textit{union} is
    \[
        A \cup B = \{x \suchthat x \in A \lor x \in B\}
    \]
\end{snippetdefinition}

\subsection{Intersection}

\begin{snippetdefinition}{intersection}{Intersection}
    If \(A\) and \(B\) are sets, then their \textit{intersection} is
    \[
        A \cap B = \{x \suchthat x \in A \land x \in B\}
    \]
\end{snippetdefinition}

\subsection{Difference}

\begin{snippetdefinition}{difference}{Difference}
    If \(A\) and \(B\) are sets, then their \textit{difference} is
    \[
        A \backslash B = \{x \suchthat x \in A \land x \notin B\}
    \]
\end{snippetdefinition}

\begin{snippetcorollary}{dual-set-difference}{}
    Note that
    \[
        A \backslash B = B \backslash A
        \iff A = B
    \]
\end{snippetcorollary}

\subsection{Subset in terms of relationships}

\begin{snippetcorollary}{subset-in-terms-of-relationships}{Subset in terms of relationships}
    \[
        A \subseteq B
        \iff
        A \cup B = B
        \iff
        A \cap B = A
        \iff
        A \backslash B = \emptyset
    \]
\end{snippetcorollary}

\subsection{Disjoint Sets}

\begin{snippetdefinition}{disjoint-sets}{Disjoint Sets}
    If \(A\) and \(B\) are sets and \(A \cap B = \emptyset \), then \(A\)
    and \(B\) are \textit{disjoint sets}.
\end{snippetdefinition}

\subsection{Cartesian Product}

\begin{snippetdefinition}{cartesian-product-two-sets}{Cartesian Product of two sets}
    If \(A\) and \(B\) are sets, then their \textit{cartesian product} is
    \[
        A\times B = \{(x,y) \suchthat x \in A \land y \in B\}
    \]
    which is the set of all possible \textit{ordered pairs}.
\end{snippetdefinition}

\begin{snippetdefinition}{cartesian-product}{Cartesian Product}
    Given \(n\) sets \(A_1, A_2, \ldots, A_2\),
    their \textit{cartesian product} \(A_1 \times A_2 \times \cdots \times A_n\)
    is the set of ordered \(n\)-tuples \((a_1, a_2, \ldots, a_n)\) with \(a_i\in A_i\).
\end{snippetdefinition}

\subsection{Cartesian Power}

\begin{snippetdefinition}{cartesian-power}{Cartesian Power}
    Given a set \(A\), \(A^n=\underbrace{A\times A\times \cdots \times A}_n\).
\end{snippetdefinition}

\begin{snippet}{n-dimensional-plane-cartesian-power}
The \(n\)-dimensional plane of real numbers is a cartesian power \({\mathbb{R}}^n\).
\end{snippet}

\subsection{Disjoint union}

\begin{snippetdefinition}{disjoint-union}{Disjoint union}
    Given sets \(A_{i\in I}\), their disjoint union is
    \[
        \bigsqcup_{i\in I}A_i= \bigcup_{i\in I}\{(x, i) \suchthat x \in A_i\}
    \]
    which consists of prdered pairs where the second element
    is the index of the set.
\end{snippetdefinition}

\subsection{Complement}

\begin{snippetdefinition}{complement}{Complement}
    If \(A\) is a set, its \textit{complement} is
    \[
        \bar{A} = \{x \suchthat x \notin A\}
    \]
\end{snippetdefinition}

\subsection{Binary Relation}

\begin{snippetdefinition}{binary-relation}{Binary Relation}
    If \(A\) and \(B\) are sets, a function \(f:A\to B\)
    defines a \textit{binary relation} \(R\)
    \[
        R = \{(a,b) \suchthat f(a)=b\}
    \]
\end{snippetdefinition}

\begin{snippetcorollary}{binary-relation-is-subset-of-product}{Binary relation is subset of product}
    For every binary relation \(R\) given by \(f: A \times B\),
    \[R\subseteq A\times B\]
\end{snippetcorollary}

\subsection{Homogeneous Relation}

\begin{snippetdefinition}{homogeneous-relation}{Homogeneous Relation}
    A \textit{homogeneous relation} on a set \(S\) is a binary relation
    from a \(A\) to \(A\).
\end{snippetdefinition}

\subsection{Reflexive relation}

\begin{snippetdefinition}{refexive-relation}{Reflexive relation}
    A homogeneous relation \(R\) on a set \(A\) is \textit{reflexive}
    if
    \[
        \forall a\in A, (a,a) \in R
    \]
\end{snippetdefinition}

\subsection{Symmetric relation}

\begin{snippetdefinition}{symmetric-relation}{Symmetric relation}
    A homogeneous relation \(R\) on a set \(A\) is \textit{symmetric}
    if
    \[
        \forall (a,b) \in R, (b,a) \in R
    \]
\end{snippetdefinition}

\subsection{Transitive relation}

\begin{snippetdefinition}{transitive-relation}{Transitive relation}
    A homogeneous relation \(R\) on a set \(A\) is \textit{transitive}
    \[
        \forall a,b,c \in A, (a,b) \in R \land (b,c) \in R \implies (a,c) \in R 
    \]
\end{snippetdefinition}

\subsection{Equivalence relation}

\begin{snippetdefinition}{equivalence-relation}{Equivalence relation}
    An \textit{equivalence relation} is a homogeneous relation \(\sim\) on a set \(A\)
    that is
    \begin{enumerate}
        \item \textit{Reflexive}: \(\forall a \in A, a \sim a\)
        \item \textit{Symmetric}: \(\forall a,b \in A, a \sim b \iff b \sim a\)
        \item \textit{Transitive}: \(\forall a,b,c \in A, a \sim b \land b \sim c \implies a \sim c\)
    \end{enumerate}
\end{snippetdefinition}

\subsection{Equivalence class}

\begin{snippetdefinition}{equivalence-class-definition}{Equivalence class}
    Let \(\sim\) be an equivalence relation on a set \(A\).
    Given an element \(a\in A\), the \textit{equivalence class} of \(a\), is defined as
    \[
        {[a]}_{\sim} = \{x \in A \suchthat a \sim x\}
    \]
\end{snippetdefinition}

\begin{snippettheorem}{shared-element-in-equivalence-class-theorem}{Shared element in equivalence class}
    Let \(\sim\) be an equivalence relation on a set \(A\)
    and \(a,b \in A\).
    Then,
    \[
        b \in {[a]}_{\sim} \iff {[a]}_{\sim} = {[b]}_{\sim}
    \]
\end{snippettheorem}

\begin{snippetproof}{shared-element-in-equivalence-class-proof}{Shared element in equivalence class}
    By the symmetric property we have \(a \in {[a]}_{\sim}\).
    Let \(b \in {[a]}_{\sim}\), meaning \(a \sim b\). \(\forall c \in {[b]}_{\sim}\),
    meaning \(b \sim c\), we have \(a \sim c\) by the transitive property.
    Thus, \(c \in {[a]}_{\sim}\) and \({[b]}_{\sim} \subseteq {[a]}_{\sim}\).
    By the symmetric property we also have \(b \sim a\),
    \(\forall d \in {[a]}_{\sim}\), meaning \(a \sim d\), we have
    \(b \sim d\) by the transitive property. Thus, \(d \in {[b]}_{\sim}\)
    and \({[a]}_{\sim} \subseteq {[b]}_{\sim}\). Hence,
    \[
        b \in {[a]}_{\sim} \iff {[a]}_{\sim} = {[b]}_{\sim}
    \]
\end{snippetproof}

\begin{snippet}{settheory-1}
This means that every element of an equivalence class has the same equivalence class.
Thus, if two classes share an element they are the same.
\end{snippet}

\subsection{Partition of a set}

\begin{snippetdefinition}{partition-of-a-set}{Partition of a set}
    Given a set \(A\), a \textit{partition of a set} \(P={\{C_i\}}_{i\in I}\) is a collection of
    non-empty subsets of \(A\) such that \(\bigcup_{i\in I} C_i = P\) and
    \(C_i \cap C_j = \emptyset, i \neq j\).
\end{snippetdefinition}

\begin{snippet}{settheory-2}
In other words the sets \(C_i\)
contain every element of \(A\) exactly once.

Given an equivalence relationship \(\sim\) of a set \(A\),
the set of its equivalence classes form a partition of \(A\).
\end{snippet}

\subsection{Preorder}

\begin{snippetdefinition}{preorder-order}{Preorder order}
    A \textit{preorder} is a homogeneous relation \(\leq\) on a set \(A\)
    with the following properties:
    \begin{enumerate}
        \item \textit{Reflexive}: \(\forall a \in A, a \leq a\)
        \item \textit{Transitive}: \(\forall a,b,c \in A, a \leq b \land b \leq c \implies a \leq c\)
    \end{enumerate}
\end{snippetdefinition}

\subsection{Partial order}

\begin{snippetdefinition}{partial-order}{Partial order}
    A \textit{partial order} is a homogeneous relation \(\leq\) on a set \(A\)
    with the following properties:
    \begin{enumerate}
        \item \textit{Reflexive}: \(\forall a \in A, a \leq a\)
        \item \textit{Transitive}: \(\forall a,b,c \in A, a \leq b \land b \leq c \implies a \leq c\)
        \item \textit{Antisymmetric}: \(\forall a,b \in A, a \leq b \land b \leq a \implies a=b\)
    \end{enumerate}
\end{snippetdefinition}

\subsection{Total order}

\begin{snippetdefinition}{total-order}{Total order}
    A \textit{total order} is a homogeneous relation \(\leq\) on a set \(A\)
    with the following properties:
    
    \begin{enumerate}
        \item \textit{Reflexive}: \(\forall a \in A, a \leq a\)
        \item \textit{Transitive}: \(\forall a,b,c \in A, a \leq b \land b \leq c \implies a \leq c\)
        \item \textit{Antisymmetric}: \(\forall a,b \in A, a \leq b \land b \leq a \implies a=b\)
        \item \textit{Strongly connected} (or \textit{total}): \(\forall a,b\in A, a \leq b \lor b\leq a\)
    \end{enumerate}
\end{snippetdefinition}

\begin{snippet}{settheory-3}
A total order is a partial order where any two elements are comparable.
\end{snippet}

\subsection{Greatest element}

\begin{snippetdefinition}{greatest-element}{Greatest element}
    Given a partial order on a set \(A\), an element \(g\) is a \textit{greatest element}
    if \(\forall a\in A, a \leq g\).
\end{snippetdefinition}

\subsection{Least element}

\begin{snippetdefinition}{least-element}{Least element}
    Given a partial order on a set \(A\), an element \(g\) is a \textit{least element}
    if \(\forall a\in A, g \leq a\).
\end{snippetdefinition}

\subsection{Maximal element}

\begin{snippetdefinition}{maximal-element}{Maximal element}
    Given a partial order on a set \(A\), an element \(g\in A\) that is
    a greatest element is a \textit{maximal element}.
\end{snippetdefinition}

\subsection{Minimal element}

\begin{snippetdefinition}{minimal-element}{Minimal element}
    Given a partial order on a set \(A\), an element \(g\in A\) that is
    a least element is a \textit{minimal element}.
\end{snippetdefinition}

\end{document}
