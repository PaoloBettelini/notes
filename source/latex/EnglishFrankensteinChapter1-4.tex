\documentclass[preview]{standalone}

\usepackage{amsmath}
\usepackage{amssymb}
\usepackage{bettelini}
\usepackage{stellar}

\hypersetup{
    colorlinks=true,
    linkcolor=black,
    urlcolor=blue,
    pdftitle={Stellar},
    pdfpagemode=FullScreen,
}

\begin{document}

\id{english-frankenstein-chapter-1-4}
\genpage

\section{Exercises}

\begin{snippetexercise}{frankenstein-ex-6}
    {Who is the narrator in this part of the novel? What kind of narrator is that?}
    The narrator is Victor Frankenstein himself and is the protagonist.
    This means that he is unreliable and the reader cannot trust everything he says.
\end{snippetexercise}

\begin{snippetexercise}{frankenstein-ex-7}
    {Are the statements about chapters 1 and 2 true or false? Correct the false ones.}
    \begin{enumerate}
        \item \textbf{Frankenstein grew up in Lucerne.} False, Geneva
        \item \textbf{Frankenstein's parents travelled around Europe when he was a kid.} True
        \item \textbf{Elizabeth Lavenza is Frankenstein's younger sister.} False, about the same age
        \item \textbf{Frankenstein treasured his relationship with Elizabeth.} True
        \item \textbf{Frankenstein's was a happy childhood.} True
    \end{enumerate}
\end{snippetexercise}

\begin{snippetexercise}{frankenstein-ex-8}
    {What was Frankenstein interested in as a child?}
    He was interested in philosophy and alchemy, he then became interested in
    chemistry, mathematics and natural philosophy (physical secrets of the world).
\end{snippetexercise}

\begin{snippetexercise}{frankenstein-ex-9}
    {What does Frankenstein's mother ask Victor and Elizabeth on her deathbed?}
    She asks them to get married. Elizabeth has to take care of her brothers.
\end{snippetexercise}

\begin{snippetexercise}{frankenstein-ex-10}
    {Why is the episode of the thunderstorm in chapter 2, which takes place when Frankenstein is
    about fifteen, particularly important?}
    Because he found out that something as normal as a thunder could
    destroy a tree as it did  during the storm.
    With this episode, Victor discovered the power of electricity and energy.
\end{snippetexercise}

\begin{snippetexercise}{frankenstein-ex-11}
    {What happens on the \quotes{memorable day} that decides Frankenstein's \quotes{future destiny}
    soon after he starts studying at the university of Ingolstadt?}
    He met his Chemistry professor who encouraged Victor not to give up his studies
    but to see them in a different perspective and start new studies
    with more books.
\end{snippetexercise}

\begin{snippetexercise}{frankenstein-ex-12}
    {What is Frankenstein's ambition, and how does he pursue it?}
    He aims to find the origin of life, why and how life begins.
    He starts by studying dead bodies, being them human or animal.
    After that, he continues with animal bodies but this time they're alive and he also says that he
    used to torture them.
    He then found out the answer to his question and starts improving a technique that will help
    him giving live to dead bodies.
\end{snippetexercise}

\begin{snippetexercise}{frankenstein-ex-13}
    {In chapter 4, pages 64-5, Frankenstein interrupts his narrative. Why does he do that? What does
    he express in these interruptions?}
    He interrupts the narration because he doesn't want to reveal the recipe
    and because he wants Walton (or the reader) to follow the entirety
    of his biography so as not to let other people do the same mistakes as him.
\end{snippetexercise}

\begin{snippetexercise}{frankenstein-ex-14}
    {How does Frankenstein describe his physical and mental conditions at the end of chapter 4?}
    He describes himself as self-isolated, wrecked and completely immersed
    in his labour. Nothing matters, only his studies. He also
    avoided other people, as if he knew that the things he as doing
    were wrong and that people wouldn't understand.
    He can no longer enjoy his obsession.
\end{snippetexercise}

\end{document}