\documentclass[preview]{standalone}

\usepackage{amsmath}
\usepackage{amssymb}
\usepackage{stellar}
\usepackage{definitions}

\begin{document}

\id{complexanalysis-demoivre-exercises}
\genpage

\section{Exercises}

\begin{snippetexercise}{complex-analysis-demoivre-ex-1}{} % pag16.21a
    Prove that \(\cos5\theta = 16\cos^5\theta - 20\cos^3\theta+5\cos\theta\).
\end{snippetexercise}

\begin{snippetsolution}{complex-analysis-demoivre-ex-1-sol}{}
    We start using De Moivre's theorem
    \begin{align*}
        \cos5\theta + i\sin5\theta &= (\cos\theta + i\sin\theta)^5 \\
        &= \cos^5\theta - 10\cos^3\theta\sin^2\theta + 5\cos\theta\sin^4\theta \\
            &+ i(5\cos^4\theta\sin\theta -10\cos^2\theta\sin^3\theta + \sin^5\theta)
    \end{align*}
    and thus
    \[ \cos5\theta = \cos^5\theta - 10\cos^3\theta\sin^2\theta + 5\cos\theta\sin^4\theta \]
\end{snippetsolution}

\begin{snippetexercise}{complex-analysis-demoivre-ex-2}{} % pag21.37
    Determine all the fifth roots of unity.
\end{snippetexercise}

\begin{snippetsolution}{complex-analysis-demoivre-ex-2-sol}{}
    \begin{align*}
        z^5 = 1 &= \cos (2k\picircle) + i\sin(2k\picircle) = e^{2k\picircle i} \\
        z &= \cos \frac{2k\picircle}{5} + i \sin \frac{2k\picircle}{5} = e^{2k\picircle i / 5}
    \end{align*}
    Let \(\omega = e^{2k\picircle i}\). Using \(k=0,1,2,3,4\) since the others are redundant, we get
    \(1, \omega^1, \omega^2, \omega^3, \omega^4\).
\end{snippetsolution}

\begin{snippetexercise}{complex-analysis-demoivre-ex-3}{} % pag18.28a
    Find all the complex solutions to
    \[ z^5 = -32 \]
\end{snippetexercise}

\begin{snippetsolution}{complex-analysis-demoivre-ex-3-sol}{}
    \[  z^5 = -32 = 32\left( \cos(\picircle + 2k\picircle) + i\sin(\picircle + 2k\picircle) \right) \]
    By De Moivre's theorem
    \[
        z = 2\left( 
            \cos\left(\frac{\picircle + 2k\picircle}{5}\right) + i\sin\left(\frac{\picircle + 2k\picircle}{5}\right)
        \right)
    \]
    for \(k=0,1,2,3,4\).
\end{snippetsolution}

\begin{snippetexercise}{complex-analysis-demoivre-ex-4}{} % pag18.29a
    Find all the complex solutions to
    \[ z^3 = i-1 \]
\end{snippetexercise}

\begin{snippetsolution}{complex-analysis-demoivre-ex-4-sol}{}
    The point \(i-1\) has an absolute value of \(\sqrt{2}\)
    and an argument of \(3\picircle/4\).
    \begin{align*}
        i-1 &= \sqrt{2} \left(
            \cos(3\picircle/4 +2k\picircle) + i\sin(3\picircle/4 +2k\picircle)
        \right) \\
        (i-1)^{1/3} &= 2^{1/6}
        \left(
            \cos\left( \frac{3\picircle/4 + 2k\picircle}{3} \right)
            + i \sin\left( \frac{3\picircle/4 + 2k\picircle}{3} \right)
        \right)
    \end{align*}
    for \(k=0,1,2\).
\end{snippetsolution}

\begin{snippetexercise}{complex-analysis-demoivre-ex-4}{}
    Find all the complex solutions to
    \[ z^3 = i-1 \]
\end{snippetexercise}

\begin{snippetsolution}{complex-analysis-demoivre-ex-5-sol}{}
    Find the complex solutions to \(z^3 = -1\).
\end{snippetsolution}

\begin{snippetexercise}{complex-analysis-demoivre-ex-5}{}
    We first write \(-1\) as \(e^{i(\pi+2k\pi)}\) for \(k\in\naturalnumbers\).
    Using the exponent properties, we get
    \[
        z = e^{i(\pi+2k\pi)/3} = \cos\left(\frac{i(\pi+2k\pi)}{3}\right) + i \sin\left(\frac{i(\pi+2k\pi)}{3}\right)
    \]
    which gives \(3\) distinct solutions:
    \begin{itemize}
        \item \(k=0\): \(z = e^{\frac{\pi}{3}i} = \frac{1}{2} + \frac{i\sqrt{3}}{2}\);
        \item \(k=1\): \(z = e^{i\pi} = -1\);
        \item \(k=2\): \(z = e^{\frac{5}{3}\pi i} = \frac{1}{2} - \frac{i\sqrt{3}}{2}\).
    \end{itemize}
\end{snippetexercise}

\end{document}