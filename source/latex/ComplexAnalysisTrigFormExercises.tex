\documentclass[preview]{standalone}

\usepackage{amsmath}
\usepackage{amssymb}
\usepackage{stellar}
\usepackage{definitions}

\begin{document}

\id{complexanalysis-trigonometric-form-exercises}
\genpage

\section{Exercises}

\begin{snippetexercise}{complex-analysis-trig-form-ex-1}{}
    Find the trigonometric form of \(i+1\).
\end{snippetexercise}

\begin{snippetsolution}{complex-analysis-trig-form-ex-1-sol}{}
    The trigonometric form is \[
        \sqrt{2}\left(\cos\frac{\picircle}{4} + i\sin\frac{\picircle}{4}\right)
    \]
\end{snippetsolution}

\begin{snippetexercise}{complex-analysis-trig-form-ex-2}{}
    Find the trigonometric form of \(i\).
\end{snippetexercise}

\begin{snippetsolution}{complex-analysis-trig-form-ex-2-sol}{}
    The trigonometric form is \[
        i\sin\frac{\picircle}{2}
    \]
\end{snippetsolution}

\begin{snippetexercise}{complex-analysis-trig-form-ex-3}{}
    Find the trigonometric form of
    \[
        \frac{{(1+i)}^9}{{(\sqrt{3}-i)}^7}
    \]
\end{snippetexercise}

\begin{snippetsolution}{complex-analysis-trig-form-ex-3-sol}{}
    \begin{align*}
        \frac{{\left(\sqrt{2}e^{\frac{i\picircle}{4}}\right)}^9}{{\left(2e^{\frac{-i\picircle}{6}}\right)}^7}
        = 2^{\frac{9}{2}-7}e^{i\left(\frac{9\picircle}{4} + \frac{7\picircle}{6}\right)}
    \end{align*}
\end{snippetsolution}

\end{document}