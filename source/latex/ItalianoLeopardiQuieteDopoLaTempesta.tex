\documentclass[preview]{standalone}

\usepackage{amsmath}
\usepackage{amssymb}
\usepackage{stellar}
\usepackage{bettelini}

\hypersetup{
    colorlinks=true,
    linkcolor=black,
    urlcolor=blue,
    pdftitle={Stellar},
    pdfpagemode=FullScreen,
}

\begin{document}

\title{Stellar}
\id{italiano-leopardi-quiete-dopo-la-tempesta}
\genpage

\section{La quiete dopo la tempesta}

\includesnpt[text=Testo:|display=La quiete dopo la tempesta|href=https://liceocuneo.it/raimondo/wp-content/uploads/sites/7/La-quiete-dopo-la-tempesta.pdf]{file-url}

\begin{snippet}{quiete-dopo-la-tempesta-analisi-iniziale}
    La metrica consiste di tre strofe di endecasillabi e settenari.
    L'ultimo verso di ogni strofa rima sempre con uno precedente.
    \\\\
    Le strofe hanno i seguenti significati:
    \begin{enumerate}
        \item puramente descrittiva;
        \item di riflessione;
        \item sentenza morale.
    \end{enumerate}
\end{snippet}

\section{Prima strofa}

\begin{snippet}{quiete-dopo-la-tempesta-strofa1-analisi}
    \StellarPoetry{1}{
        Passata è la tempesta:
    }{
        La tempesta è passata
    }

    Si constante che un elemento negativo è passato.
    La tempesta non è subita passivamente, bensì è un momento potenzialmente mortale
    (come nell'\textit{Islandese}). Abbiamo quindi un rischio di morte superato.
    \\\\
    \StellarPoetry{2}{
        Odo augelli far festa, e la gallina, \\
        Tornata in su la via, \\
        Che ripete il suo verso. Ecco il sereno
    }{
        Sento gli uccelli far festa, e la gallina,
        \bfslash tornata sulla strada che ripete il suo verso.
        \bfslash XXXX
    }

    Per tutta la strofa si insiste molto sull'elemento sonore, più che immagini visive.
    Tempesta è in rima con festa; questa rima fa riferimento alla tempesta nefasta ormai superata.
    Il sereno è il cielo azzurro.
    \\\\
    \StellarPoetry{5}{
        \textbf{Rompe} là da ponente, alla montagna; \\
        Sgombrasi la campagna\\
        E chiaro nella valle il fiume appare.
    }{}

    Il cielo \textbf{erompe} verso la montagna.
    Questo è il momento, dopo la tempesta, dove il cielo torna azzurro.
    La conseguenza è che la campagna si sgombra dalle nubi, e il fiume torna visibile.
    Abbiamo quindi una mutazione di paesaggio dove il tempo torna bello.
    \\\\
    \StellarPoetry{8}{
        Ogni cor si rallegra, in ogni lato \\
        Risorge il romorio \\
        Torna il lavoro usato
    }{}

    Adesso, che la tempesta è alle spalle, il villaggio si riattiva con gioia.
    Per tutta la poesia le azioni vengono \textbf{ri}-compiute.
    Nessuno fa nulla di diverso rispetto a prima, riprendono a fare ciò che facevano prima,
    ma con una gioia assoluta.
    \\\\
    \StellarPoetry{11}{
        L'artigiano a mirar l'umido cielo, \\
        Con l'opra in man, cantando,\\
        Fassi in su l'uscio; a prova
    }{}

    L'artigiano si affaccia sull'uscio con in mano l'oggetto (al quale sta lavorando)
    e guarda il cielo umido (che porta ancora le tracce della tempesta).
    L'artigiano vede questa cosa cantando, e quindi in uno stato di quiete.
    \\\\
    \StellarPoetry{14}{
        Vien fuor la femminetta a còr dell'acqua \\
        Della novella piova;
    }{
        Esce la donzella a raccogliere l'acqua
        \bfslash della pioggia recente
    }

    L'espressione \quotes{a prova} significa come se facesse una gara, quindi in fretta.
    Le due figure, l'artigiano e la femminetta escono entrambi dalla loro casa.
    Vi è un'azione di riacquistare lo spazio esterno, che prima era prescluso.
    Abbiamo una inversione delle parole che cfrea un chiasmo fra soggetto e
    predicato (soggetto, predicato, predicato, soggetto).
    \\\\
    \StellarPoetry{16}{
        E l'erbaiuol rinnova \\
        Di sentiero in sentiero\\
        Il grido giornaliero.
    }{
        E l'erbivendolo riprende il suo lavoro
        \bfslash strada per strada
        \bfslash il suo consueto richiamo
    }

    Questa è la terza figura umana
    \\\\
    \StellarPoetry{19}{
        Ecco il Sol che ritorna, ecco sorride \\
        Per li poggi e le ville. Apre i balconi,\\
        Apre terrazzi e logge la famiglia:
    }{
        Ecco il sole che torna, sorride
        \bfslash per i casolari e rustici. Apre i balconi,
        \bfslash La servitù apre le terrazzi, balconi
    }

    Il sole torna a sorridere sui casolari e rustici.
    Il tempo scorre ed il tempo è sempre più bello. \\
    Le tre figure umuane sono racchiuse da delel notazioni di paesaggi.
    \\\\
    \StellarPoetry{22}{
        E, dalla via corrente, odi lontano \\
        Tintinnio di sonagli; il carro stride\\
        Del passeggier che il suo cammin ripiglia.
    }{
        E, dalla strada, si sente da lontano
        \bfslash il suono di sognagli; il carro cigola
        \bfslash del viandante che riprende il viaggio
    }

    Dalla strada si sente in lontananza il
    tintinnio delle sonagliere (dei cavalli etc.).
    La parola tintinnio è una onomatopea.
\end{snippet}

\section{Seconda strofa}

\begin{snippet}{quiete-dopo-la-tempesta-strofa2-analisi}
    Tutta la strofa è percorsa da questa idea della ciclicità, ma con una grande gioia.
    \\\\
    \StellarPoetry{25}{
        Si rallegra ogni core.\\
        Sì dolce, sì gradita
    }{
        Ogni animo si rallegra
        \bfslash così dolce, così gradita
    }

    La gioia è quella ch viene dopo una grande angoscia, che finalmente termina
    in positivo. Non abbiamo nulla di nuovo rispetto a prima; ma siete felici per ciò che non è
    successo.
    \\\\
    \StellarPoetry{27}{
        Quand'è, com'or, la vita? \\
        Quando con tanto amore \\
        L'uomo a' suoi studi intende? \\
        O torna all'opre? o cosa nova imprende? \\
        Quando de' mali suoi men si ricorda?
    }{
        Quando, come adesso, la vita è così?
        \bfslash Quando, con così tanto amore
        \bfslash L'uomo si dedica ai suoi studi (occupazioni)?
        \bfslash O riprende il lavoro? Quando fa qualcosa di nuovo?
        \bfslash Quando si ricorda un po' di meno dei suoi mali?
    }

    Perché si prova questo piacere? Soltanto quando le cose rischiamo di perderle,
    ci si rende conto di quanto valgono.
    \\\\
    \StellarPoetry{32}{
        Piacer figlio d'affanno; \\
        Gioia vana, ch'è frutto\\
        Del passato timore, onde si scosse
    }{
        Il piacere è figlio del dolore
        \bfslash è una gioia inconsistente
        \bfslash che nasce dalla paura appena passata, per cui
    }\\
    Il piacere è figlio del dolore (metafora), il piacere nasce dalla
    cessazione del dolore.
    % Zibaldone (p. 66) Se tu hai un nemico mortale...
    \\\\
    \StellarPoetry{35}{
        E paventò la morte \\
        Chi la vita abborria;\\
        Onde in lungo tormento,\\
        Fredde, tacite, smorte,\\
        Sudàr le genti e palpitàr, vedendo\\
        Mossi alle nostre offese \\
        Folgori, nembi e vento.
    }{
        E qualcuno iniziò a temere la morte,
        \bfslash Anche chi disprezzava la vita;
        \bfslash Così, in una lunga agonia,
        \bfslash Gelide, silenziose, pallide,
        \bfslash Le persone sudavano e tremavano, vedendo
        \bfslash I fulmini, le nuvole e il vento scatenati contro di noi
        \bfslash A causa delle nostre azioni (o peccati).
    }
    
    Fa paura anche a chi era annoiato dalla vita, a chi la disprezzava.
    La vita si riattacca a coloro, nel momento in cui vedi la morte.
    Forse abbiamo bisogno di avere paura di morire, per riattaccarsi alla vita,
    e forse la vita ci ha dato qusta offesa proprio per questo motivo.
\end{snippet}

\section{Terza strofa}

\begin{snippet}{quiete-dopo-la-tempesta-strofa3-analisi}
    \StellarPoetry{42}{
        O natura cortese, \\
        Son questi i doni tuoi,\\
        Questi i diletti sono\\
        Che tu porgi ai mortali. Uscir di pena\\
        È diletto fra noi.
    }{XXX}

    la terza e ultima strofa si apre con un apostrofe alla natura.
    Si rivolge alla natura con parole che sono esattamente il contrario di quelle che vuole comunicare.
    Le parole \textbf{cortese}, \textbf{doni} e \textbf{diletti} sono termini
    molto positivi, ma hanno un fondo di ironia.
    \\\\
    \StellarPoetry{47}{
        Pene tu spargi a larga mano; il duolo \\
        Spontaneo sorge e di piacer, quel tanto\\
        Che per mostro e miracolo talvolta\\
        Nasce d'affanno, è gran guadagno. Umana\\
    }{XXX}
    
    Utilizza termini crudi e la natura diffonde solamente sofferenza,
    e il dolore nasce spontaneo. Qui finisce il vocativo alla natura.
    \\\\
    \StellarPoetry{51}{
        Prole cara agli eterni! assai felice \\
        Se respirar ti lice\\
        D'alcun dolor: beata\\
        Se te d'ogni dolor morte risana.
    }{
        O stirpe umana cara agli dei! molto felice
        \bfslash puoi considerarti se hai una pausa
        \bfslash da qualche dolore: beata
        \bfslash se la morte ti guarisce
    }
\end{snippet}

\end{document}