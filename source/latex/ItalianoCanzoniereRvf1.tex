\documentclass[preview]{standalone}

\usepackage{amsmath}
\usepackage{amssymb}
\usepackage{stellar}
\usepackage{definitions}
\usepackage{bettelini}

\begin{document}

\id{italiano-canzoniere-rvf-1}
\genpage

\section{Rvf 1: Voi ch'ascoltate in rime sparse il suono}

\begin{snippet}{canzoniere-rvf-1-parte-1}
    La prima poesia è un sonetto con schema delle rime ABBA, ABBA, CDE, CDE.
    \\\\
    \StellarPoetry{1}{
        \textbf{Voi ch'ascoltate} in \textbf{rime sparse} il suono \\
        di quei sospiri ond'io nudriva 'l core \\
        in sul mio primo giovenile errore \\
        quand'era in parte altr'uom da quel ch'i' sono,
    }{
        Presso di voi che ascoltate in poesie staccate tra loro il suono
        \bfslash di quei sospiri d'amore di cui io nutrivo il mio animo,
        \bfslash al tempo del mio primo traviamento giovanile,
        \bfslash quando in parte ero un uomo diverso da quello che sono ora,
    }
    \StellarPoetry{5}{
        del vario stile in ch'io piango et ragiono \\
        fra le vane speranze e 'l van dolore, \\
        ove sia chi per prova intenda amore, \\
        \textbf{spero trovar} pietà, nonché perdono.
    }{
        (presso di voi che scoltate il suono) dei diversi stili, in cui io piango e mi esprimo 
        \bfslash fra le inutili speranze e l'inutile dolore,
        \bfslash se c'è qualcuno che sappia per esperienza che cos'è l'amore,
        \bfslash spero di trovare presso di lui compassione e perdono.
    }
    \StellarPoetry{9}{
        Ma ben veggio or sì come al popol tutto \\
        favola fui gran tempo, onde sovente \\
        di me medesmo meco mi vergogno;
    }{
        Ma ora mi accorgo chiaramente come per tutto il popolo 
        \bfslash sono stato per molto tempo oggetto di dicerie, motivo per cui spesso
        \bfslash ho vergogna di me stesso dentro di me;
    }
    \StellarPoetry{12}{
        et del mio vaneggiar vergogna è 'l frutto, \\
        e 'l pentersi, e 'l conoscer chiaramente \\
        che quanto piace al mondo è breve sogno.
    }{
        e la vergogna è il risultato del mio vaneggiare,
        \bfslash e il pentimento e il sapere con chiarezza
        \bfslash che tutto ciò che riguarda la vita terrena è di breve durata.
    }

    Questo primo sonetto fa da prologo, ma possiede anche degli elementi di bilancio,
    cioè degli elementi che leggono l'esperienza del libro retrospettivamente,
    e quindi fa anche da epilogo.
    Inizialmente l'autore si rivolge al lettore che ascolta (v. 1), come spesso di tradizione (es. Il Paradiso II).
    Le prime due quartine presentano un errore di sintassi; nonostante si rivolga inizialmente all'autore,
    il verbo principale dopo le varie subordinate è \quotes{spero} (v. 8), causando un incoerenza fra soggetto e verbo.
    Questo è un elemento che fa da prologo, ossia la forma del suo libro è data da \quotes{rime sparse} (v. 1).
    Nonostante ciò, vi è un significato che percorre le poesie nel loro ordine.
    Dopo aver annunciato la forma del libro, viene anche annunciato il tema, ossia
    un tema di una natura amorosa (v. 2).
    Il (v. 3) fa invece da epilogo, perché l'amore, ormai passato, viene definito come errore.
    L'\textit{errore} per questa donna viene giudicato tale perché esso risiede nell'atto stesso di amarla,
    ossia il fatto che si tratti di un amore nei confronti di una donna terrena, mentre l'unico amore
    vero può essere rivolto solo verso Dio. L'errore è appunto quello di essersi dedicati a qualcosa di fugace,
    uno sviamento rispetto alla via verso Dio ed il suo amore eterno.
    Tuttavia, questo errore non è ancora completamente superato poiché Petrarca
    era \textit{in parte} un uomo diverso da ciò che è oggi (v. 4), e quindi il cambiamento non è ancora
    completamente compiuto.
    In fondo, possiamo notare un conflitto intrinseco fra l'amore sacro e l'amore terreno.
    Questo sonetto è stato quindi scritto dopo tutto il resto, e potrebbe anche essere posizionato
    logicamente al termine.
    \\\\
    Nella seconda quartina il poeta dichiarare di sperare di trovare pietà e perdono
    presso un pubblico più ristretto rispetto a quello della prima quartina,
    ossia presso coloro che nell'amore hanno fatto esperienza (v. 7).
    Il \textit{vario stile} (v. 5) si riferisce all'intercambiarsi continuo fra
    razionalità e lucidità e pianto e dolore. Infatti, vi è chiasmo con le parole \textit{piango},
    \textit{ragion}, \textit{speranze} e \textit{dolore}, dove all'interno vi è la razionalità e all'esterno
    parole negative di dolore.
    L'errore è quello di essersi attaccati ad un qualcosa di \textit{vano}, ma non inutile, bensì effimera, e quindi
    indegna di un amore che dovrebbe essere dedicata solo alle cose Celesti.
    \\\\
    La seconda metà del testo inizia con \quotes{ma} (v. 9) avversativo.
    Il poeta dice che ora, da uomo maturo, gli è chiaro come lui per lungo tempo fu motivo
    di chiacchiera per tutto il popolo a causa di questo amore (v. 9).
    A questo punto il pubblico si riallarga di nuovo (\quotes{popol tutto}, v. 9).
    Il fatto che sia stato motivo di chiacchiere lo fa vergognare di se stesso con lui stesso
    (\quotes{onde sovente} \textrightarrow per cui spesso, v. 10).
    \\\\
    La parola \quotes{vaneggiare} (v. 12) fa riferimento all'attaccamento a qualcosa di effimero.
    Le conseguenze di questo vaneggiare sono tre.
    La prima è la vergogna, il secondo è il pentimento, il terzo è il conoscere chiaramente che tutto in
    terra è un breve sogno. Queste tre conseguenze sono disposte come climax. C'è una progressiva
    introspezione, un percorso di redenzione in cui alla fine c'è un assioma (verità assoluta indiscutibile,
    sentenza di valori universali).
    \begin{enumerate}
        \item \textbf{Contrapposizione tra presente e passato}:
            È un testo in cui l'esperienza viene riletta in termini critici. Il rapporto tra presente e passato è
            l'esperienza amorosa e la volontà di liberarsene. La passione amorosa è evocata nelle quartine che
            rievocano quella passione con già qualche traccia della volontà di liberarsene. È un misto tra
            coinvolgimento e volontà di liberarsene. La vera liberazione da essa è affermata in modo netto nelle
            terzine. I verbi al passato si riferiscono all'amore per Laura. Gli altri verbi sono al presente. I tempi
            verbali sono mescolati come per mostrare come questo distaccamento sia difficile e non ancora del
            tutto avvenuto. Nell'ultima terzina si passa ai verbi all'infinito per farle avere un valore universale
            che valga per tutti (generalizzazione dell'esperienza).
        \item \textbf{Il rapporto tra elementi cortesi e la prospettiva cristiana}:
            Tratti tipici della poesia cortese: l'apostrofe al lettore, sospiri come segno dell'innamoramento,
            rimanti core dolore e amore, parole tipiche come piango ragiono e speranza, la ricerca di
            comprensione presso un pubblico di intenditori. Questi elementi appaiono nelle quartine. La parola
            errore indica un primo riferimento del giudizio cristiano. Le speranze ed il dolore considerati vani ed
            il trovar pietà e perdono sono altri elementi del giudizio negativo cristiano. Il vergognarsi ed il
            vaneggiare così come il pentirsi e la sentenza finale al verso 14 sono ulteriori elementi riferiti al
            pensiero cristiano. Tra queste due spinte qui prevale la spinta religiosa, ovvero quella del
            superamento e del pentimento.
        \item \textbf{Aspetti formali}:
            Allitterazione della \quotes{m} al verso 11 non è l'unica catena allitterante. Al verso 1 e 2 si ripete la v
            così come nella seconda quartina. Infine, ai versi 13 e 14 la \quotes{c} dura.
    \end{enumerate}
\end{snippet}

\end{document}