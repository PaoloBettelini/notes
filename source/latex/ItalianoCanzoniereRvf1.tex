\documentclass[preview]{standalone}

\usepackage{amsmath}
\usepackage{amssymb}
\usepackage{stellar}
\usepackage{bettelini}

\hypersetup{
    colorlinks=true,
    linkcolor=black,
    urlcolor=blue,
    pdftitle={Stellar},
    pdfpagemode=FullScreen,
}

\begin{document}

\title{Stellar}
\id{italiano-canzoniere-rvf-1}
\genpage

\section{Rvf 1: Voi ch'ascoltate in rime sparse il suono}

\begin{snippet}{canzoniere-rvf-1-parte-1}
    La prima poesia è un sonetto con schema delle rime ABBA, ABBA, CDE, CDE.

    \begin{center}
        \textit{\textbf{Voi ch'ascoltate} in \textbf{rime sparse} il suono} \\
        \textit{di quei sospiri ond'io nudriva 'l core} \\
        \textit{in sul mio primo giovenile errore} \\
        \textit{quand'era in parte altr'uom da quel ch'i' sono,}
    \end{center}
    \begin{center}
        \textit{del vario stile in ch'io piango et ragiono} \\
        \textit{fra le vane speranze e 'l van dolore,} \\
        \textit{ove sia chi per prova intenda amore,} \\
        \textit{\textbf{spero trovar} pietà, nonché perdono.}
    \end{center}
    \begin{center}
        \textit{Ma ben veggio or sì come al popol tutto} \\
        \textit{favola fui gran tempo, onde sovente} \\
        \textit{di me medesmo meco mi vergogno;}
    \end{center}
    \begin{center}
        \textit{et del mio vaneggiar vergogna è 'l frutto,} \\
        \textit{e 'l pentersi, e 'l conoscer chiaramente} \\
        \textit{che quanto piace al mondo è breve sogno.}
    \end{center}

    Questo primo sonetto fa da prologo, ma possiede anche degli elementi di bilancio,
    cioè degli elementi che leggono l'esperienza del libro retroaspettivamente,
    e quindi fa anche da epilogo.
    Inizialmente l'autore si rivolge al lettore che ascolta (v. 1), come spesso di tradizione (es. Il Paradiso II).
    le prime due quartine presentano un errore di sintassi; nonostante si rivolga inizialmente all'autore,
    il verbo principale dopo le varie subordinate è \quotes{spero} (v. 8), causando un incoerenza fra soggetto e verbo.
    Questo è un elemento che fa da prologo, ossia la forma del suo libro è data da \quotes{rime sparse} (v. 1).
    Nonostante ciò, vi è un significato che percorre le poesie nel loro ordine.
    Dopo aver annunciato la forma del libro, viene anche annunciato il tema, ossia
    un tema di una natura amorosa (v. 2).
    Il (v. 3) fa invece da epilogo, perché l'amore, ormai passato, viene definito come errore.
    L'\textit{errore} per questa donna viene giudicato tale perché esso risiede nell'atto stesso di amarla,
    ossia il fatto che si tratti di un amore nei confronti di una donna terrena, mentre l'unico amore
    vero può essere rivolto solo verso Dio. L'errore è appunto quello di essersi dedicati a qualcosa di fugace,
    uno sviamento rispetto alla via verso Dio ed il suo amore eterno.
    Tuttavia, questo errore non è ancora completamente superato poiché Petrarca
    era \textit{in parte} un uomo diverso da ciò che è oggi (v. 4), e quindi il cambiamento non è ancora
    completamente compiuto.
    In fondo, possiamo notare un conflitto intrinseco fra l'amore sacro e l'amore terreno.
    Questo sonetto è stato quindi scritto dopo tutto il resto, e potrebbe anche essere posizionato
    logicamente al termine.
    \\
    Nella seconda quartina il poeta dichiarare di sperare di trovare pietà e perdono
    presso un pubblico più ristretto rispetto a quello della prima quartina,
    ossia presso coloro che nell'amore hanno fatto esperienza (v. 7).
    Il \textit{vario stile} (v .5) si riferisce all'intercambiarsi continuo fra
    razionalità e luciditià e pianto e dolore. Infatti, vi è chiasmo con le parole \textit{piango},
    \textit{ragion}, \textit{speranze} e \textit{dolore}, dove all'interno vi è la razionalità e all'esterno
    parole negative di dolore.
    L'errore è quello di essersi attaccati ad un qualcosa di \textit{vano}, ma non inutile, bensì effimera, e quindi
    indegna di un amore che dovrebbe essere dedicata solo alle cose Celesti.
\end{snippet}

\plain{TODO: parte due}

\end{document}