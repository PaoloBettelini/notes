\documentclass[preview]{standalone}

\usepackage{amsmath}
\usepackage{amssymb}
\usepackage{stellar}
\usepackage{definitions}
\usepackage{bettelini}

\begin{document}

\id{types-of-subgroups}
\genpage

\section{The centralizer subgroup}

\begin{snippetdefinition}{centralizer-group-definition}{The centralizer subgroup}
    Let \((H, \circ) \subgroupleq A = (G, \circ)\) be \group[groups].
    The \textit{centralizer} of \(H\) is defined as
    the \group with the elements of \(G\) that commute with every element of \(H\)
    with respect to \(\circ\).
    \[
        \text{C}_A(H) = \{
            g \in G \suchthat \forall h \in H, g\circ h=h\circ g
        \}
    \]
    \textit{Syntax:} if \(g \in G\), \(\text{C}_A(g) \triangleq \text{C}_A(\{g\})\).
\end{snippetdefinition}

\plain{The centralizer of one element contains at least the element itself, as it
commutes with itself.}

\begin{snippettheorem}{centralizer-of-subgroup-is-subgroup}{}
    Let \((H, \circ) \subgroupleq G\) be \group[groups]. Then, \(\groupcentralizer_G(H) \subgroupleq G\).
\end{snippettheorem}

\begin{snippetproof}{centralizer-of-subgroup-is-subgroup-proof}{centralizer-of-subgroup-is-subgroup}{Centralizer of subgroup is subgroup}
    Suppose \(a,b \in \groupcentralizer_G(H)\).
    We want to show \(ab^{-1} \in \groupcentralizer_G(H)\). \\
    Note that the condition \(gh=hg \iff hg^{-1}=g^{-1}h\). \\
    Consider the expression \((ab^{-1})h = a(b^{-1}h) = ahb^{-1} = h(ab^{-1})\).
    This means that \(ab^{-1} \in \groupcentralizer_G(H)\) and thus in \(H\).
\end{snippetproof}

\plain{We can also define the centralizer of a set as the intersection of
all singleton centralizers.}

\section{Center of a group}

\begin{snippetdefinition}{center-of-group-definition}{Center of a group}
    Let \(A=(G, \circ)\) be a \group. The \textit{center} of \((G, \circ)\) is defined as
    the \group with every element of \(G\) that commute with every other element
    \[
        \text{Z}(A) = \groupcentralizer_A(G)
    \]
\end{snippetdefinition}

\begin{snippet}{center-of-group-condition-alternative}
    The condition \(gx=xg\) is also sometimes expressed as \(gxg^{-1} = x\).
\end{snippet}

\begin{snippettheorem}{center-of-group-is-subgroup}{}
    Let \(G\) be a \group, then \(\groupcenter(G) \subgroupleq G\).
\end{snippettheorem}

\begin{snippetproof}{center-of-group-is-subgroup-proof}{center-of-group-is-subgroup}{}
    Assume \(a, b \in \groupcenter(G)\) meaning \(a = gag^{-1}\) and \(b = gag^{-1}\) for any \(g \in G\). \\
    We want to show \(ab^{-1} \in \groupcenter(G)\).
    \(ab^{-1} = (gag^{-1}){(gbg^{-1})}^{-1} = gag^{-1}gb^{-1}g^{-1}
    = g ab^{-1} g^{-1}\) which is precisely the requirement to be in \(\groupcenter(G)\).
\end{snippetproof}

\begin{snippetproposition}{generated-group-abelian-iff-subset-is}{Abelian group check}
    Let \(A=(G, \circ)\) be a \group generated by \(X \subseteq G\)
    \[
        A = \gengrp{X}
    \]
    Then, \(A\) is \abeliangroup[abelian] \ifandonlyif \((X, \circ)\) is \abeliangroup[abelian].
\end{snippetproposition}

\begin{snippetproof}{generated-group-abelian-iff-subset-is-proof}{generated-group-abelian-iff-subset-is}{Abelian group check}
    \iffproof{
        Trivial.
    }{
        We have that \(\groupcentralizer_A(X) \subseteq X\).
        However, \(\groupcentralizer_A(X) \subgroupleq A\), and so
        \[\groupcentralizer_A(X) \geq \gengrp{X} = (X, \circ)\]
        This means that every element in \(G\) commutes with every element of \(X\).
        Thus, \(X \subseteq \groupcenter(A)\).
        However, \(\groupcenter(A)\) is a \subgroup,
        and thus
        \[
            \groupcenter(A) \geq \gengrp{X} = A
        \]
        On the other hand, \(\groupcenter(A) \subgroupleq A\), and thus
        \(A = \groupcenter(A)\), meaning that \(A\) is \abeliangroup[abelian].
    }
\end{snippetproof}

\end{document}