\documentclass[preview]{standalone}

\usepackage{amsmath}
\usepackage{amssymb}
\usepackage{stellar}
\usepackage{definitions}
\usepackage{bettelini}

\begin{document}

\id{mechanics-exercises}
\genpage

\section{Exercises}

\begin{snippetexercise}{mechanics-ex-1}{Orbital Path Length}
    A particle moves in the plane along a path described by 
    \[
    \vec{R}(t) = 
    \begin{pmatrix}
        a \cos(\omega t) \\
        b \sin(\omega t)
    \end{pmatrix}.
    \]
    Show that this is an elliptical orbit, calculate the time required for a complete orbit, and express its length as a definite integral (without solving it).
\end{snippetexercise}

\begin{snippetsolution}{mechanics-ex-1-sol}{Orbital Path Length}
    We can rewrite the trajectory in terms of its components:
    \[
        x(t) = a \cos(\omega t), \quad y(t) = b \sin(\omega t)
    \]
    from which it follows that
    \[
        \frac{x^2}{a^2} + \frac{y^2}{b^2} = 1
    \]
    which an ellipse with axes aligned with the coordinate axes and lengths \(2a\) and \(2b\). The time required to complete one full orbit is the period of \(\vec{R}(t)\), which is 
    \[
        T = \frac{2\picircle}{\omega}.
    \]
    To calculate the length of the orbit, we first determine the velocity:
    \[
    \vec{V}(t) = 
    \begin{pmatrix}
        -a\omega \sin(\omega t) \\
        b\omega \cos(\omega t)
    \end{pmatrix}
    \]
    and we integrate its module over a period
    \begin{align*}
        \ell &= \integral[0][T][|\vec{V}(t)|][t] = \integral[0][T][\sqrt{a^2\omega^2 \sin^2(\omega t) + b^2\omega^2 \cos^2(\omega t)}][t] \\
        &= \integral[0][2\picircle][\sqrt{a^2 \sin^2(u) + b^2 \cos^2(u)}][u]
    \end{align*}
    This integral cannot be express in terms of elementary functions, except for the trivial case
    \(a=b\) (circular trajectory) for which \(\ell = 2\picircle a\).
\end{snippetsolution}


\end{document}