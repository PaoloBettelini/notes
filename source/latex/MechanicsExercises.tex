\documentclass[preview]{standalone}

\usepackage{amsmath}
\usepackage{amssymb}
\usepackage{stellar}
\usepackage{definitions}
\usepackage{bettelini}

\begin{document}

\id{mechanics-exercises}
\genpage

\section{Exercises}

\begin{snippetexercise}{mechanics-ex-free-fall}{Free fall}
    A body is dropped from a certain height with zero initial velocity: how much time
    \(\tau\) must be expected for the body to travel a distance of \(s = 20m\) in time from that instant
    \(\tau = 1s\).
\end{snippetexercise}

\begin{snippetsolution}{mechanics-ex-free-fall-sol}{Free fall}
    The falling function is given by \(s = \frac{1}{2}at^2 + v_0t + s_0\).
    We thus have
    \begin{align*}
        20 &= \integral[t][t+\tau][v_0 + ax][x] \\
        20 &= \frac{1}{2}a{(t+1)}^2 - \frac{1}{2}at^2 \\
        40 &= a{(t+1)}^2 - at^2 \\
        40 &= a(t^2 + 2t + 1) - at^2 \\
        40 &= at^2 + 2at + a - at^2 \\
        40 &= 2at + a \\
        40 - a &= 2at \\
        t &= \frac{40 - a}{2a}
    \end{align*}
\end{snippetsolution}

\begin{snippetexercise}{mechanics-ex-orbital-path}{Orbital Path Length}
    A particle moves in the plane along a path described by 
    \[
    \vec{R}(t) = 
    \begin{pmatrix}
        a \cos(\omega t) \\
        b \sin(\omega t)
    \end{pmatrix}.
    \]
    Show that this is an elliptical orbit, calculate the time required for a complete orbit, and express its length as a definite integral (without solving it).
\end{snippetexercise}

\begin{snippetsolution}{mechanics-ex-orbital-path-sol}{Orbital Path Length}
    We can rewrite the trajectory in terms of its components:
    \[
        x(t) = a \cos(\omega t), \quad y(t) = b \sin(\omega t)
    \]
    from which it follows that
    \[
        \frac{x^2}{a^2} + \frac{y^2}{b^2} = 1
    \]
    which an ellipse with axes aligned with the coordinate axes and lengths \(2a\) and \(2b\). The time required to complete one full orbit is the period of \(\vec{R}(t)\), which is 
    \[
        T = \frac{2\picircle}{\omega}.
    \]
    To calculate the length of the orbit, we first determine the velocity:
    \[
    \vec{V}(t) = 
    \begin{pmatrix}
        -a\omega \sin(\omega t) \\
        b\omega \cos(\omega t)
    \end{pmatrix}
    \]
    and we integrate its module over a period
    \begin{align*}
        \ell &= \integral[0][T][|\vec{V}(t)|][t] = \integral[0][T][\sqrt{a^2\omega^2 \sin^2(\omega t) + b^2\omega^2 \cos^2(\omega t)}][t] \\
        &= \integral[0][2\picircle][\sqrt{a^2 \sin^2(u) + b^2 \cos^2(u)}][u]
    \end{align*}
    This integral cannot be express in terms of elementary functions, except for the trivial case
    \(a=b\) (circular trajectory) for which \(\ell = 2\picircle a\).
\end{snippetsolution}

\begin{snippetexercise}{mechanics-ex-rotating-platform}{Rotating platform}
    A platform rotates with angular velocity
    \(\omega\) around a vertical central axis.
    At the instant \(t = 0\), a ball is launched horizontally with velocity \(v_0\)
    from the centre of the platform; the friction the ball encounters is negligible,
    so that it moves with respect to the earth in uniform rectilinear motion with velocity \(v_0\).
    Determine the acceleration of the ball, at a generic instant, with respect to a reference system
    integral with the platform.
\end{snippetexercise}

\begin{snippetsolution}{mechanics-ex-rotating-platform-sol}{Rotating platform}
    The position is given by
    \[
        \vec{R}(t) = \begin{pmatrix}
            v_0t\cos(\omega t) \\
            v_0t\sin(\omega t)
        \end{pmatrix}
    \]
    The velocity is given by
    \[
        \vec{v}(t) = v_0 \begin{pmatrix}
            -\omega t \sin(\omega t) + \cos(\omega t) \\
            \omega t \cos(\omega t) + \sin(\omega t)
        \end{pmatrix}
    \]
    The acceleration is given by
    \[
        \vec{a}(t) = v_0 \omega \begin{pmatrix}
            -\omega t \cos(\omega t) - 2\sin(\omega t) \\
            -\omega t \sin(\omega t) + 2\cos(\omega t)
        \end{pmatrix}
    \]
    and its module
    \begin{align*}
        |\vec{a}(t)| &= v_0 \omega \sqrt{
            {(-\omega t \cos(\omega t) - 2\sin(\omega t))}^2 + {(-\omega t \sin(\omega t) + 2\cos(\omega t))}^2
        } \\
        &= v_0 \omega \sqrt{4 + \omega^2t^2}
    \end{align*}
\end{snippetsolution}

\begin{snippetexercise}{mechanics-ex-falling-raindrop}{Falling raindrop}
    Suppose that a raindrop falling through a cloud accumulates mass at a rate \(kmv\),
    where \(k > 0\) is a constant, \(m\) is the mass of the droplet and \(v\) its velocity.
    Find the speed of the drop at a generic time \(t\)
    assuming it starts at rest, also find the mass of the drop at the same time.
\end{snippetexercise}

\begin{snippetsolution}{mechanics-ex-falling-raindrop-sol}{Falling raindrop}
    We have that
    \[
        \frac{dm(t)}{dt} = km(t)v(t)
    \]
    Since the mass is not constant,
    \[
        F = \frac{dQ}{dt} = \frac{dm}{dt}v + \frac{dv}{dt}m = gm
    \]
    By substituting we get
    \begin{align*}
        gm &= kmv^2 + \frac{dv}{dt} m \\
        g - kv^2&= \frac{dv}{dt}
    \end{align*}
    The equation is separable
    \begin{align*}
        \int \frac{dv}{g-kv^2} &= \int dt \\
        \frac{1}{g} \int \frac{dv}{1 - \frac{k}{g}v^2} &= t + C \\
    \end{align*}
    We use the substitution \(\varphi = \sqrt{\frac{k}{g}}v\) and thus \(d\varphi = \frac{\frac{k}{g}} dv\).
    \begin{align*}
        t + C &= \frac{1}{\sqrt{kg}} \int \frac{d\varphi}{1 - \varphi^2} \\
        t + C &= \frac{1}{\sqrt{kg}} \tanh^{-1}\left(\sqrt{\frac{k}{g}}v\right) \\
        (t + C) \sqrt{kg} &= \tanh^{-1}\left(\sqrt{\frac{k}{g}}v\right) \\
        v(t) = \sqrt{\frac{g}{k}} \tanh((t + C)\sqrt{kg})
    \end{align*}
    Since \(v(0) = 0\), we have that \(\tanh\left(C\sqrt{kg}\right)\) and thus \(C=0\), so
    \[
        v(t) = \sqrt{\frac{g}{k}} \tanh(t\sqrt{kg})
    \]
    To find the mass, we separate the initial equation
    \begin{align*}
        \int \frac{dm}{m} &= \integral[kv(t)][t] \\
        \ln m &= \integral[k\sqrt{\frac{g}{k}} \tanh(t\sqrt{kg})][t] \\
        \ln m &= \ln(\cosh(t + \sqrt{kg})) + C \\
        m(t) &= m_0 \cosh(t\sqrt{kg})
    \end{align*}
\end{snippetsolution}

\begin{snippetexercise}{mechanics-ex-snowplough}{Snowplough}
    At a certain time in the morning it starts
    snowing, and at noon a snowplough leaves to clear the roads. The snow continues
    falling with constant intensity. It is known that the speed at which the snowplough proceeds is
    inversely proportional to the height of the snow.
    In the first two hours of work the snowplough manages to clear 4 km of road. In the following two
    following two hours, however, only 2 km are cleared. One wants to know at what time it started to
    it started snowing.
\end{snippetexercise}

\begin{snippetsolution}{mechanics-ex-snowplough-sol}{Snowplough}
    Let \(t=0\) be noon. The height of the snow follows
    \[
        h \propto t-t_0
    \]
    The velocity of the snowplough is given by
    \[
        v = \frac{\xi}{t-t_0}
    \]
    for some \(\xi\) which we don't know.
    The space cleared in the first two hours is given by
    \[
        s_1 = \integral[0][\tau][\frac{\xi}{t-t_0}][t] = \xi\ln\frac{t_0-\tau}{t_0}
    \]
    where \(\tau = 2\text{h}\). In the following two hours we have
    \[
        s_2 = \integral[\tau][2\tau][\frac{\xi}{t-t_0}][t] = \xi\ln\frac{t_0-2\tau}{t_0-\tau}
    \]
    By putting these together we get 
    \begin{align*}
        \frac{\xi\ln\frac{t_0-\tau}{t_0}}{\xi\ln\frac{t_0-2\tau}{t_0-\tau}} &= \frac{s_1}{s_2} \\
        \ln\frac{t_0-\tau}{t_0} &= 2\ln\frac{t_0-2\tau}{t_0-\tau} \\
        t_0^2\tau - t_0\tau^2 - \tau^3 &= 0 \\
        t_0 = \frac{1\pm\sqrt{5}}{2}\tau
    \end{align*}
    We consider the negative solution, meaning that it started snowing at \(10:45:50\).
\end{snippetsolution}

\begin{snippetexercise}{mechanics-ex-rotating-tube-with-ball}{Rotating tube with ball}
    A rigid cylindrical tube, of negligible cross-section, rotates in a vertical plane with constant angular velocity
    \(\omega\) around a horizontal axis; inside the tube a small ball of mass \(m\) can move without friction.
    At the instant \(t=0\) the tube is in a vertical position, the ball is above the axis of rotation,
    at a distance \(d\) from it and with zero velocity with respect to the tube.
    Study the trajectory of the ball along the tube, approximating the ball to a material particle.
    Discuss the solution in the case
    \(\omega \to 0\) and \(\omega \gg 1\).
\end{snippetexercise}

\begin{snippetsolution}{mechanics-ex-rotating-tube-with-ball-sol}{Rotating tube with ball}
    We choose a reference system with the axes integral to the cylinder (x is the vertical direction of the cylinder).
    The weight force, the constraining reaction and the apparent force (the outward centrifugal force and the Coriolis force)
    act on the ball. The constraining reaction points on the y-axis.
    We therefore have
    \[
        \vec{F}_{CE} = -\vec{\omega} \wedge (\vec{\omega} \wedge \vec{r}), \qquad \vec{F}_{CO} = -2m\vec{\omega} \wedge \vec{v}
    \]
    Since \(\vec{r} = x\hat{x}\), \(\vec{v} = \dot{x}\hat{x}\) and \(\omega = w\hat{z}\),
    we have
    \begin{align*}
        \vec{F}_{CE} &= -\omega^2 x \hat{z} \wedge (\hat{z} \wedge \hat{x}) \\
        &= \omega^2 x \hat{z} \wedge \hat{y} \\
        &= \omega^2 x \hat{x}
    \end{align*}
    and
    \begin{align*}
        \vec{F}_{CO} &= -2m\omega \dot{x} \hat{z} \wedge \vec{x} \\
        &= 2m\omega \dot{x} \hat{y}
    \end{align*}
    Wr thus have the following equations:
    \[
        \begin{cases}
            m\ddot{x} = -mg\cos\theta + m\omega^2 x \\
            m\ddot{y} = -mg\sin\theta + 2m\omega \frac{dx}{dt} + R = 0
        \end{cases}
        \to
        \begin{cases}
            \ddot{x} - \omega^2 x = -g\cos(\omega t) \\
            \ddot{y} = -g\sin(\omega t) + 2\omega \frac{dx}{dt} + R = 0
        \end{cases}
    \]
    This is a non-homogeneous differential equation. The solution
    is given by the linear combination of the solution to the homogeneous equation and a particular
    general solution.
    \[
        x(t) = Ae^{\omega t} + Be^{-\omega t} + \frac{g}{2\omega^2}\cos(\omega t)
    \]
    with the initial conditions we find
    \[
        x(t) = \left(d-\frac{g}{2\omega^2}\right)
        \cosh(\omega t) + \frac{g}{2\omega^2}\cos(\omega t)
    \]
    where \(d\) is the length of the tube from which the ball starts.
    We now study \(\omega \to 0\).
    Note that
    \[
        \cos(\omega t) \sim 1 - \frac{\omega^2 t^2}{2}
    \] and 
    \[
        \cosh(\omega t) \sim 1 + \frac{\omega^2 t^2}{2}
    \]
    Then,
    \begin{align*}
        x(t) &\sim \left(d - \frac{g}{2\omega^2}\right)
        \left(1 + \frac{\omega^2t^2}{2}\right)
        + \frac{g}{2\omega^2}\left(1 - \frac{\omega^2t^2}{2}\right) \\
        &= d - \frac{g}{2}t^2
    \end{align*}
    Thus, if the velocity is very low, the gravitational force is the strongest.
    If \(\omega \gg 1/t\) we have
    \begin{align*}
        x(t) \sim \frac{1}{2}de^{\omega t}
    \end{align*}
    Thus is \(\omega\) is very big, the centrifugal force is the strongest and the ball is shot far away.
\end{snippetsolution}

\end{document}