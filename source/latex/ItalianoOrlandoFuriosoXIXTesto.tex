\documentclass[preview]{standalone}

\usepackage{amsmath}
\usepackage{amssymb}
\usepackage{stellar}
\usepackage{bettelini}

\hypersetup{
    colorlinks=true,
    linkcolor=black,
    urlcolor=blue,
    pdftitle={Stellar},
    pdfpagemode=FullScreen,
}

\begin{document}

\title{Stellar}
\id{orlando-furioso-xix-testo}
\genpage

\section{Testo}

\begin{snippet}{orlando-furioso-ottava-31-xix}
    \StellarPoetry{31}{
        O conte Orlando, o re di Circassia,\\
        vostra inclita virtù, dite, che giova?\\
        Vostro alto onor dite in che prezzo sia,\\
        o che mercé vostro servir ritruova.\\
        Mostratemi una sola cortesia\\
        che mai costei v'usasse, o vecchia o nuova,\\
        per ricompensa e guidardone e merto\\
        di quanto avete già per lei sofferto.
    }{Oh conte Orlando, oh Sacripante,
    il vostro illustre valore, ditemi, a cosa può giovare?
    Ditemi in che misura sia apprezzato il vostro sublime onore,
    o che riconoscenza ottenga la vostra servitù d'amore.
    Fatemi l'esempio anche di un solo gesto di cortesia
    che mai Angelica vi abbia fatto dono, o del passato o recente,
    come ricompensa, premio o per acquisizione di merito
    per tutto quello che per amore di lei avete sofferto.}
    \\\\
    Si rivolge a Orlando, si rivolge a Sacripante con una domanda retorica,
    ma il vostro onore, servizio e cortesia, a che cosa servono? (A niente).
\end{snippet}

\begin{snippet}{orlando-furioso-ottava-32-xix}
    \StellarPoetry{32}{
        Oh se potessi ritornar mai vivo,\\
        quanto ti parria duro, o re Agricane!\\
        che già mostrò costei sì averti a schivo\\
        con repulse crudeli ed inumane.\\
        O Ferraù, o mille altri ch'io non scrivo,\\
        ch'avete fatto mille pruove vane\\
        per questa ingrata, quanto aspro vi fôra,\\
        s'a costu' in braccio voi la vedesse ora!
    }{Oh, se potessi ritornare invita oh re Agricane, morto per l'amore di lei,
    quanto ti sembrerebbe duro il destino!
    che già ti manifestò tanto disprezzo nei tuoi confronti
    con crudeli ed inumani gesti di avversione.
    Oppure Ferraù, o mille alri dei quali non scrivo,
    che avete dato mille dimostrazioni del vostro valore
    per questa donna ingrata, quanto duro vi sarebbe
    se la vedeste ora tra le braccia di costui!}
    \\\\
    Il campo semantico che attraversa questa e l'ottava precedente fanno
    parte delle parole cavalleresche, il codice dell'onore, il codice cortese
    della generosità e del servizio verso qualcuno.
    \\\\
    A differenza del lettore, i personaggi del testo non conoscono il carattere di Angelica,
    ossia un'ingrata manipolatrice e sfruttatrice.
    Angelica si vergogna si sè stessa per essere abbassata a tal punto di essersi fatta accompagnare
    da personaggi come Orlando.
    Invece, per i personaggi, Angelica è l'ideale di principessa, per la quale fare tutto,
    perché prima o poi la si potrà ottenere.
    Queste ultime due ottave delineano proprio questo concetto dell'idealizzazione di Angelica
    e della differenza di cosa pensano i cavaliere in confronto alla realtà effettiva.
    In questi cavaliere vi è forse qualcosa di folle, ossia questo accanirsi a seguire qualcosa
    che vogliono che sia come loro vogliono, ma così non è.
    \\
    Tutti sono soggetti alla lima dell'amore, indipendentemente da quanto tu possa resistere.
    In questo canto viene delineato come anche Angelica possa essere vittima di questa cosa.
    \\
    Nonostante Ariosto segua la tradizione letteraria si diverse a rovesciare gli schemi.
    Questa è fondamentalmente la base sulle quali si fondano le parodie.
\end{snippet}

\begin{snippet}{orlando-furioso-ottava-37-xix}
    \StellarPoetry{37}{
        Poi che le parve aver fatto soggiorno\\
        quivi più ch'a bastanza, fe' disegno\\
        di fare in India del Catai ritorno,\\
        e Medor coronar del suo bel regno.\\
        Portava al braccio un cerchio d'oro, adorno\\
        di ricche gemme, in testimonio e segno\\
        del ben che 'l conte Orlando le volea;\\
        e portato gran tempo ve l'avea.
    }{Dopo che le sembrò di avere soggiornato
    in quel luogo a sufficienza, decise
    di fare ritorno in India nella regione del Catai,
    e di incoronare quindi Medoro re del suo bel regno.
    Portava al braccio un cerchio d'oro, adornato
    da gemme preziose, a testimonianza e simbolo
    del bene che il conte Orlando provava nei suoi confronti;
    e l'aveva al braccio da molto tempo.}
    \\\\
    In quella parte dell'India chiamata Catai (oggi Cina, Ariosto per India intendeva tutta l'asia).
\end{snippet}

\begin{snippet}{orlando-furioso-ottava-38-xix}
    \StellarPoetry{38}{
        Quel donò già Morgana a Ziliante, \\
        nel tempo che nel lago ascoso il tenne;\\
        ed esso, poi ch'al padre Monodante,\\
        per opra e per virtù d'Orlando venne,\\
        lo diede a Orlando: Orlando ch'era amante,\\
        di porsi al braccio il cerchio d'or sostenne,\\
        avendo disegnato di donarlo\\
        alla regina sua di ch'io vi parlo.
    }{Quel cerchio d'oro fu donato da Morgana, innamorata, a Ziliante,
    quando lo tenne nascosto sul fondo del lago;
    e Ziliante, dopo che dal padre Monodante
    poté tornare grazie all'opera ed al grande valore di Orlando,
    lo diede poi ad Orlando stesso: il paladino innamorato,
    tollerò di portare al braccio il cerchio, così poco virile,
    avendo deciso di portarlo in dono
    alla sua regina, Angelica, della quale vi sto ora raccontando.}
    \\\\
    Questa è una analessi sulla storia di quel braccialetto
\end{snippet}

\begin{snippet}{orlando-furioso-ottava-39-xix}
    \StellarPoetry{39}{
        Non per amor del paladino, quanto\\
        perch'era ricco e d'artificio egregio,\\
        caro avuto l'avea la donna tanto,\\
        che più non si può aver cosa di pregio.\\
        Se lo serbò ne l'Isola del pianto,\\
        non so già dirvi con che privilegio,\\
        là dove esposta al marin mostro nuda\\
        fu da la gente inospitale e cruda.
    }{Non per amore nei confronti del paladino, piuttosto
    perché era un ornamento prezioso e di ottima fattura,
    la donna l'aveva tanto caro,
    che di più non si potrebbe avere caro un oggetto di valore.
    Lo conservò con sé quando era sull'isola di Ebuba,
    non so come riuscì ad ottennere un tale privilegio,
    là dove venne esposta, completamente nuda, al mostro marino
    da parte della gente crudele e inospitale che abita l'isola.}
    \\\\
    Angelica non ha tenuto il braccialetto con molto riguardo perché fosse dono di Orlando, ma perché
    fosse estremamente prezioso.
    Qui viene mostrato il tratto di indifferenza e di avidità di Angelica.
\end{snippet}

\begin{snippet}{orlando-furioso-ottava-40-xix}
    \StellarPoetry{40}{
        Quivi non si trovando altra mercede\\
        ch'al buon pastor ed alla moglie dessi,\\
        che serviti gli avea con sì gran fede\\
        dal dì che nel suo albergo si fur messi,\\
        levò dal braccio il cerchio e gli lo diede,\\
        e volse per suo amor che lo tenessi.\\
        Indi saliron verso la montagna\\
        che divide la Francia da la Spagna.
    }{Ora, non disponendo di altra ricompensa
    da poter dare al buon pastore ed alla sua moglie,
    che li avevano serviti con così grande devozione
    dal giorno in cui entrarono nella loro dimora,
    si levò dal braccio il cerchio e lo diede loro,
    e vollè che lo tenessero come segno del suo amore.
    Quindi, iniziarono a risalire la montagna, i Pirenei,
    che divide la Francia dalla Spagna.}
    \\\\
    Non si separa a malincuore del braccialetto, per lei è solamente un valore economico alto.
    Angelica ha comunque un minimo di senso di gratitudine nei confronti del pastore.
\end{snippet}

\begin{snippet}{orlando-furioso-ottava-41-xix}
    \StellarPoetry{41}{
        Dentro a Valenza o dentro a Barcellona\\
        per qualche giorno avea pensato porsi,\\
        fin che accadesse alcuna nave buona\\
        che per Levante apparecchiasse a sciorsi.\\
        Videro il mar scoprir sotto a Girona\\
        ne lo smontar giù dei montani dorsi;\\
        e costeggiando a man sinistra il lito,\\
        a Barcellona andar pel camin trito.
    }{Dentro Valencia o dentro Barcellona
    aveva pensato di fermarsi per qualche giorno,
    finché non fosse capitata qualche buona nave
    che si fosse apprestata a salpare verso l'Asia.
    Videro apparire il mare in prossimità di Gerona,
    mentre scendevano dalle dorsali montuose;
    e costeggiando alla loro sinistra il litorale,
    lungo la via più battuta giunsero a Barcellona.}
    \\\\
    Il progetto è quello di scendere dai pirenei e aspettare per qualche giorno una nave.
\end{snippet}

\begin{snippet}{orlando-furioso-ottava-42-xix}
    \StellarPoetry{42}{
        Ma non vi giunser prima, ch'un uom pazzo\\
        giacer trovato in su l'estreme arene,\\
        che, come porco, di loto e di guazzo\\
        tutto era brutto e volto e petto e schene.\\
        Costui si scagliò lor come cagnazzo\\
        ch'assalir forestier subito viene;\\
        e diè lor noia, e fu per far lor scorno.\\
        Ma di Marfisa a ricontarvi torno.
    }{Ma non vi giunsero in tempo per evitare di incontrare un uomo folle
    che giaceva sulla battigia, sul limite della spiaggia,
    e che, come fosse un porco, di fango e di acqua
    era completamente sporco in volto, petto e schiena.
    Costui sì lanciò  con violenza contro di loro così come un cagnaccio
    subito va ad assalire uno straniero;
    e diede loro noia e fu sul punto di recare loro danno.
    Ma torno ora a raccontarvi nuovamente di Marfisa.}
    \\\\
    In questa ottava avviene un colpo di scena; trovano un pazzo lì sulla spiaggia,
    completamente imbrattato di fango e di terra (come un porco - simulitudine animalesca).
    Questo pazzo si scaglia alla coppia aggredendoli
    (come un cagnazzo salta addosso ad una persona che non conosce).
    \\\\
    L'ottava è divisa in sette versi più uno, dove nell'ultimo verso
    il narratore sospende la storia sul più bello.
    Non si sa chi sia il pazzo nè se farà effettivamente del male.
    Verrà svelato che l'uomo è in realtà Orlando.
\end{snippet}

\end{document}