\documentclass[preview]{standalone}

\usepackage{amsmath}
\usepackage{amssymb}
\usepackage{stellar}
\usepackage{bettelini}

\hypersetup{
    colorlinks=true,
    linkcolor=black,
    urlcolor=blue,
    pdftitle={Stellar},
    pdfpagemode=FullScreen,
}

\begin{document}

\title{Stellar}
\id{orlando-furioso-xix-testo}
\genpage

\section{Riassunto 1-16}

\begin{snippet}{italiano-orlando-riassunto-canto-xix-1-16}
    Cloridano e Medoro sono due soldati pagani molto amici. (I soldati hanno un ruolo più alto dei cavalieri).
    Avviene una battaglia contro i Cristiani scozzesi che termina con la sconfitta dei pagani con la morte e il ritiro dei soldati e l'uccisione del Re Dardinello.
    Entrambi i soldati battono in ritirata, ma Medoro non volendo lasciare sul campo il corpo del loro Re, porta in spalla la salma di Dardinello.
    Mentre fuggono, Cloridano si accorge di non avere più affianco Medoro.
    Cloridano, salvo dalla battaglia, torna indietro per andarlo a cercare e lo vede, nascosto tra gli alberi e quindi senza essere visto, accerchiato dai cristiani con il corpo del re affianco.
    Medoro difende il corpo del suo re dai cristiani \quotes{come orsa, che l'alpestre cacciatore \bfslash ne la pietrosa tana assalita abbia, \bfslash sta sopra i figli con incerto core, \bfslash e freme in suono di pietà e di rabbia} (vv. 7.1-4)
    Cloridano allora carica una freccia dopo l'altra e ammazza due nemici dal cespuglio, i quali si spaventano non capendo da dove arrivino i dardi.
    Zerbino, il re dei cristiani, sostiene che sicuramente c'era un pagano appostato a difendere Medoro, e allora il re lo prende in ostaggio.
    Quando però Zerbino alza la chioma d'oro di Medoro, si impietosisce dalla bellezza del volto del soldato e lo risparmia.
    In quel momento, un cavaliere cristiano villano arriva e infilza nel petto Medoro.
    Cloridano, vedendo la scena, istintivamente abbandona l'arco e sguaina la spada per andare a combattere, andando in contro a un semi-suicidio, morendo (15).
    L'esercito cristiano lascia tutto e parte all'inseguimento del cavaliere villano.
    Medoro sopravvive (16) e vede in lontananza arrivare una donzella, Angelica (17).
\end{snippet}

\section{Testo}

\begin{snippet}{orlando-furioso-ottava-29-xix}
    \StellarPoetry{29}{
        La sua piaga più s'apre e più incrudisce,\\
        quanto più l'altra si ristringe e salda.\\
        Il giovine si sana: ella languisce\\
        di nuova febbre, or agghiacciata, or calda.\\
        Di giorno in giorno in lui beltà fiorisce:\\
        la misera si strugge, come falda\\
        strugger di nieve intempestiva suole,\\
        ch'in loco aprico abbia scoperta il sole.
    }{La sua ferita si apre e diventa grave sempre di più,
    quanto più l'altra si restringe e si chiude.
    Il giovane guarisce: lei va perdendo le proprie forze
    a causa di una nuova febbre, ora fredda gelata ed ora calda.
    Di giorno in giorno rifiorisce in lui la bellezza:
    la povera ragazza si consuma lentamente, così come uno strato
    di neve caduta fuori stagione è solita consumarsi,
    quando viene scoperta dal sole in un luogo aperto.}
    \\\\
    Le ferite fra Medoro e Angelica sono inversamente proporzionali.
    Il cavaliere si sana dalle ferite, mentre Angelica si ammala dall'amore.
    Ai versi 6-8 c'è una similitudine: così come la
    neve si scioglie al sole, anche Angelica si scioglie.
\end{snippet}

\begin{snippet}{orlando-furioso-ottava-30-xix}
    \StellarPoetry{30}{
        Se di disio non vuol morir, bisogna \\
        che senza indugio ella se stessa aiti:\\
        e ben le par che di quel ch'essa agogna,\\
        non sia tempo aspettar ch'altri la 'nviti.\\
        Dunque, rotto ogni freno di vergogna,\\
        la lingua ebbe non men che gli occhi arditi:\\
        e di quel colpo domandò mercede,\\
        che, forse non sapendo, esso le diede.
    }{Se non vuole morire di desiderio, è necessario
    che, senza esitare oltre, inizi ad aiutare sé stessa:
    e crede quindi bene che quello che essa desidera ardentemente,
    non si debba aspettare che altri al invitino ad ottenere.
    Pertanto, rotto ogni freno della propria vergogna,
    mostrò una lingua non meno audace, coraggiosa, dei propri occhi:
    e chiese quindi misericordia per quel colpo ricevuto in cuore,
    che, forse non rendendosene conto, lui le aveva inflitto.}
    \\\\
    L'ottava è bipartita a metà. La prima parte esprime il
    desiderio di Angelica al concedersi a Medoro e,
    se non vuol morire di volontà, deve subito agire.
    Nella seconda parte, Angelica dichiara i propri
    sentimenti al Cavaliere.
    Anteticamente alle storie d'amore stilnovistiche,
    è la donna che arde di sentimento e che prende l'iniziativa
    di dichiararsi all'amata.
\end{snippet}

\begin{snippet}{orlando-furioso-ottava-31-xix}
    \StellarPoetry{31}{
        O conte Orlando, o re di Circassia,\\
        vostra inclita virtù, dite, che giova?\\
        Vostro alto onor dite in che prezzo sia,\\
        o che mercé vostro servir ritruova.\\
        Mostratemi una sola cortesia\\
        che mai costei v'usasse, o vecchia o nuova,\\
        per ricompensa e guidardone e merto\\
        di quanto avete già per lei sofferto.
    }{Oh conte Orlando, oh Sacripante,
    Il vostro illustre valore, ditemi, a cosa può giovare?
    Ditemi in che misura sia apprezzato il vostro sublime onore,
    o che riconoscenza ottenga la vostra servitù d'amore.
    Fatemi l'esempio anche di un solo gesto di cortesia
    che mai Angelica vi abbia fatto dono, o del passato o recente,
    come ricompensa, premio o per acquisizione di merito
    per tutto quello che per amore di lei avete sofferto.}
    \\\\
    Il narratore si rivolge a Orlando e a Sacripante con una domanda retorica:
    ``ma il vostro onore, servizio e cortesia, a che cosa sono serviti?'' (A niente).
    Inoltre, il narratore elogia il codice cavalleresco verso l'umiltà e 
    l'immensità dei cavalieri: ``inclita virtù'', ``Vostro alto onor'',
    ``o che mercé \underline{vostro servir} ritruova'', ``cortesia'', (vv. 2-5).
\end{snippet}

\begin{snippet}{orlando-furioso-ottava-32-xix}
    \StellarPoetry{32}{
        Oh se potessi ritornar mai vivo,\\
        quanto ti parria duro, o re Agricane!\\
        che già mostrò costei sì averti a schivo\\
        con repulse crudeli ed inumane.\\
        O Ferraù, o mille altri ch'io non scrivo,\\
        ch'avete fatto mille pruove vane\\
        per questa ingrata, quanto aspro vi fôra,\\
        s'a costu' in braccio voi la vedesse ora!
    }{Oh, se potessi ritornare invita oh re Agricane, morto per l'amore di lei,
    quanto ti sembrerebbe duro il destino!
    che già ti manifestò tanto disprezzo nei tuoi confronti
    con crudeli ed inumani gesti di avversione.
    Oppure Ferraù, o mille alri dei quali non scrivo,
    che avete dato mille dimostrazioni del vostro valore
    per questa donna ingrata, quanto duro vi sarebbe
    se la vedeste ora tra le braccia di costui!}
    \\\\
    Il campo semantico che attraversa questa e l'ottava precedente fanno
    parte delle parole cavalleresche, il codice dell'onore, il codice cortese
    della generosità e del servizio verso qualcuno.
    \\\\
    Le ottave 31 e 32 sono un intervento diretto crudele e sarcastico da parte del narratore.
    A differenza del lettore, i personaggi del testo non conoscono il carattere di Angelica,
    ossia un'ingrata manipolatrice e sfruttatrice.
    Angelica si vergogna di sé stessa per essersi abbassata a tal punto di essersi fatta accompagnare
    da personaggi come Orlando.
    Invece, per i personaggi, Angelica è l'ideale di principessa, per la quale fare tutto,
    perché prima o poi la si potrà ottenere.
    Queste ultime due ottave delineano proprio questo concetto dell'idealizzazione di Angelica
    e della differenza di cosa pensano i cavalieri nei confronti della realtà effettiva.
    In questi cavaliere vi è forse qualcosa di folle, ossia questo accanirsi a seguire di qualcosa
    che vogliono che sia come loro vogliono, ma così non è.
    \\
    Tutti sono soggetti alla lima dell'amore, indipendentemente da quanto tu possa resistere.
    In questo canto viene delineato come anche Angelica possa essere vittima di questa cosa.
    \\
    Nonostante Ariosto segua la tradizione letteraria si diverte a rovesciare gli schemi.
    Questa è fondamentalmente la base sulle quali si fondano le parodie.
\end{snippet}

\begin{snippet}{orlando-furioso-ottava-33-xix}
    \StellarPoetry{33}{
        Angelica a Medor la prima rosa\\
        coglier lasciò, non ancor tocca inante:\\
        né persona fu mai sì aventurosa,\\
        ch'in quel giardin potesse por le piante.\\
        Per adombrar, per onestar la cosa,\\
        si celebrò con cerimonie sante\\
        il matrimonio, ch'auspice ebbe Amore,\\
        e pronuba la moglie del pastore.
    }{Angelica la propria verginità da Medoro
    lasciò che venisse colta, mai prima di allora toccata:
    nessuna persona fu infatti mai tanto fortunata,
    da poter mettere piede in quel giardino.
    Per coprire, per rendere legittima la cosa,
    venne celebrato con santo cerimoniale
    il loro matrimonio, che ebbe il Dio Amore come testimone dello sposo,
    e la moglie del pastore a testimone della sposa.}
    \\\\
    Angelica si concede per la prima volta e per inaugurare l'evento
    i due si sposano. Così facendo, per titolo nobiliare,
    Medoro diventa Principe ed erede del Catai.
    Ai versi 7 e 8 si ha un contrasto di proporzionalità, poiché
    come testimone del matrimonio per Medoro è presente Amore
    in persona, dunque un Dio sceso in terra, mentre, per la fretta di
    Angelica di unirsi con lui, si accontenta di avere come testimone
    la moglie del pastore che li ospitava, dunque la più misera delle
    persone.
\end{snippet}

\begin{snippet}{orlando-furioso-ottava-34-xix}
    \StellarPoetry{34}{
        Fersi le nozze sotto all'umil tetto\\
        le più solenni che vi potean farsi;\\
        e più d'un mese poi stero a diletto\\
        i duo tranquilli amanti a ricrearsi.\\
        Più lunge non vedea del giovinetto\\
        la donna, né di lui potea saziarsi;\\
        né, per mai sempre pendergli dal collo,\\
        il suo disir sentia di lui satollo.
    }{Le nozze fuorono celebrato sotte l'umile tetto della dimora del pastore,
    furono le più fastose che si sarebbero potutote svolgere;
    e per più di un mese stettero piacevolemente
    i due tranquilli amanti a svagarsi.
    Non riusciva a vedere null'altro che il giovanotto
    la donna, e non poteva mai sentirsi sazia di lui;
    né, per quanto pendesse sempre dal suo collo,
    sentiva sazio il desiderio che provava nei suoi confronti.}
    \\\\
    In questa ottava ritorna il contrastro di proporzionalità
    tra la solennità delle loro nozze, dunque nobili, e
    il luogo in cui questo si svolge, cioè nella casa di
    un povero pastore.
    Si ha inoltre un riferimento temporale della loro
    permanenza nella casa del pastore dopo il loro
    connubio, che è di un mese.
\end{snippet}

\begin{snippet}{orlando-furioso-ottava-35-xix}
    \StellarPoetry{35}{
        Se stava all'ombra o se del tetto usciva,\\
        avea dì e notte il bel giovine a lato:\\
        matino e sera or questa or quella riva\\
        cercando andava, o qualche verde prato:\\
        nel mezzo giorno un antro li copriva,\\
        forse non men di quel commodo e grato,\\
        ch'ebber, fuggendo l'acque, Enea e Dido,\\
        de' lor secreti testimonio fido.
    }{Se stava al coperto o se usciva fuori casa,
    aveva sempre, giorno e notte, il bel giovane accanto a sé:
    dal mattino alla sera, ora questa ed ora quella riva del fiume
    andava percorrendo a passeggio, o altrimenti qualche prato verde:
    a mezzogiorno trovavano riparo in una grotta,
    forse non meno comoda e gradita di quella
    che ebbero, per evitare un temporale, Enea e Didone
    a fedele testimone dei loro segreti.}
    \\\\
    Le ottave 34 e 35 condividono l'inseparabilità
    della coppia, poiché viene spesso rammentato che
    essi non sono mai l'uno senza l'altro.
    Da verso 5 a verso 8 si ha un riferimento all'Eneide,
    nel quale Enea e Didone trovano riparo in una grotta
    durante una tempesta, testimone dei loro segreti amori.
    Questo riferimento preannuncia la sopraggiunta della
    coppia a un luogo di simile importanza.
\end{snippet}

\begin{snippet}{orlando-furioso-ottava-36-xix}
    \StellarPoetry{36}{
        Fra piacer tanti, ovunque un arbor dritto\\
        vedesse ombrare o fonte o rivo puro,\\
        v'avea spillo o coltel subito fitto;\\
        così, se v'era alcun sasso men duro:\\
        ed era fuori in mille luoghi scritto,\\
        e così in casa in altritanti il muro,\\
        Angelica e Medoro, in vari modi\\
        legati insieme di diversi nodi.
    }{Tra tanti piaceri, ovunque un dritto arbusto
    vedesse fare ombra ad una fonte od a un limpido ruscello,
    vi conficcava subito uno spillone od un coltello;
    allo stesso modo agiva se trovava qualche roccia poco dura:
    e vi erano scritti all'aperto in mille diversi luoghi,
    ed anche sul muro di casa in altrettanti luoghi ,
    i nomi di Angelica e Medoro, in diversi modi
    intrecciati tra di loro.}
    \\\\
    Muniti di punte di spilli e di coltelli, la coppia
    incide la scritta ``Angelica e Medoro'' su ogni
    albero, sasso e altri mille luoghi.
    Viene inoltre specificato che le incisioni non
    si limitano al mondo esterno, bensì anche sulle
    mura della casa del pastore.
\end{snippet}

\begin{snippet}{orlando-furioso-ottava-37-xix}
    \StellarPoetry{37}{
        Poi che le parve aver fatto soggiorno\\
        quivi più ch'a bastanza, fe' disegno\\
        di fare in India del Catai ritorno,\\
        e Medor coronar del suo bel regno.\\
        Portava al braccio un cerchio d'oro, adorno\\
        di ricche gemme, in testimonio e segno\\
        del ben che 'l conte Orlando le volea;\\
        e portato gran tempo ve l'avea.
    }{Dopo che le sembrò di avere soggiornato
    in quel luogo a sufficienza, decise
    di fare ritorno in India nella regione del Catai,
    e di incoronare quindi Medoro re del suo bel regno.
    Portava al braccio un cerchio d'oro, adornato
    da gemme preziose, a testimonianza e simbolo
    del bene che il conte Orlando provava nei suoi confronti;
    e l'aveva al braccio da molto tempo.}
    \\\\
    Dopo più di un mese di soggiorno dal pastore, Angelica
    progetta di tornare a casa per incoronare Medoro principe del Catai.
    Al polso la ragazza portava sempre con sé un bracciale regalatole
    da Orlando in segno di onore all'amore provato nei suoi confronti.
    Per Ariosto, l'India è tutta l'Asia, mentre il Catai è
    l'odierna Cina.
\end{snippet}

\begin{snippet}{orlando-furioso-ottava-38-xix}
    \StellarPoetry{38}{
        Quel donò già Morgana a Ziliante, \\
        nel tempo che nel lago ascoso il tenne;\\
        ed esso, poi ch'al padre Monodante,\\
        per opra e per virtù d'Orlando venne,\\
        lo diede a Orlando: Orlando ch'era amante,\\
        di porsi al braccio il cerchio d'or sostenne,\\
        avendo disegnato di donarlo\\
        alla regina sua di ch'io vi parlo.
    }{Quel cerchio d'oro fu donato da Morgana, innamorata, a Ziliante,
    quando lo tenne nascosto sul fondo del lago;
    e Ziliante, dopo che dal padre Monodante
    poté tornare grazie all'opera ed al grande valore di Orlando,
    lo diede poi ad Orlando stesso: il paladino innamorato,
    tollerò di portare al braccio il cerchio, così poco virile,
    avendo deciso di portarlo in dono
    alla sua regina, Angelica, della quale vi sto ora raccontando.}
    \\\\
    Questa è un'analessi sulla storia di quel bracciale che
    pareva ad Orlando poco virile, donatogli per il suo coraggio
    in passato, che decide di darlo ad Angelica in segno d'amore.
\end{snippet}

\begin{snippet}{orlando-furioso-ottava-39-xix}
    \StellarPoetry{39}{
        Non per amor del paladino, quanto\\
        perch'era ricco e d'artificio egregio,\\
        caro avuto l'avea la donna tanto,\\
        che più non si può aver cosa di pregio.\\
        Se lo serbò ne l'Isola del pianto,\\
        non so già dirvi con che privilegio,\\
        là dove esposta al marin mostro nuda\\
        fu da la gente inospitale e cruda.
    }{Non per amore nei confronti del paladino, piuttosto
    perché era un ornamento prezioso e di ottima fattura,
    la donna l'aveva tanto caro,
    che di più non si potrebbe avere caro un oggetto di valore.
    Lo conservò con sé quando era sull'isola di Ebuba,
    non so come riuscì ad ottennere un tale privilegio,
    là dove venne esposta, completamente nuda, al mostro marino
    da parte della gente crudele e inospitale che abita l'isola.}
    \\\\
    Angelica ha tenuto il braccialetto con molto riguardo non perché fosse dono di Orlando, ma perché
    era estremamente prezioso. Questo comportamento mostra il tratto avido
    e indifferente di Angelica. Viene inoltre accennato come lei sia riuscita
    a proteggere il bracciale sull'isola dell'episodio dell'orca non per il
    valore sentimentale ma per il puro valore economico.
\end{snippet}

\begin{snippet}{orlando-furioso-ottava-40-xix}
    \StellarPoetry{40}{
        Quivi non si trovando altra mercede\\
        ch'al buon pastor ed alla moglie dessi,\\
        che serviti gli avea con sì gran fede\\
        dal dì che nel suo albergo si fur messi,\\
        levò dal braccio il cerchio e gli lo diede,\\
        e volse per suo amor che lo tenessi.\\
        Indi saliron verso la montagna\\
        che divide la Francia da la Spagna.
    }{Ora, non disponendo di altra ricompensa
    da poter dare al buon pastore ed alla sua moglie,
    che li avevano serviti con così grande devozione
    dal giorno in cui entrarono nella loro dimora,
    si levò dal braccio il cerchio e lo diede loro,
    e vollè che lo tenessero come segno del suo amore.
    Quindi, iniziarono a risalire la montagna, i Pirenei,
    che divide la Francia dalla Spagna.}
    \\\\
    Non si separa a malincuore del braccialetto, per lei è solamente un valore economico alto.
    Angelica ha comunque un minimo di senso di gratitudine nei confronti del pastore, pagandogli
    il soggiorno con il bracciale d'oro di Orlando.
    In Francia, la coppia attraversa i Pirenei per andare in Spagna verso
    i grandi porti e salpare verso l'India.
\end{snippet}

\begin{snippet}{orlando-furioso-ottava-41-xix}
    \StellarPoetry{41}{
        Dentro a Valenza o dentro a Barcellona\\
        per qualche giorno avea pensato porsi,\\
        fin che accadesse alcuna nave buona\\
        che per Levante apparecchiasse a sciorsi.\\
        Videro il mar scoprir sotto a Girona\\
        ne lo smontar giù dei montani dorsi;\\
        e costeggiando a man sinistra il lito,\\
        a Barcellona andar pel camin trito.
    }{Dentro Valencia o dentro Barcellona
    aveva pensato di fermarsi per qualche giorno,
    finché non fosse capitata qualche buona nave
    che si fosse apprestata a salpare verso l'Asia.
    Videro apparire il mare in prossimità di Gerona,
    mentre scendevano dalle dorsali montuose;
    e costeggiando alla loro sinistra il litorale,
    lungo la via più battuta giunsero a Barcellona.}
    \\\\
    Il progetto è quello di scendere dai Pirenei
    e aspettare per qualche giorno una nave.
    La coppia, per essere certi di arrivare Barcellona
    costeggiano il mare a Girona, subito dopo i Pirenei.
\end{snippet}

\begin{snippet}{orlando-furioso-ottava-42-xix}
    \StellarPoetry{42}{
        \textbf{Ma} non vi giunser prima, ch'un uom pazzo\\
        giacer trovato in su l'estreme arene,\\
        che, come porco, di loto e di guazzo\\
        tutto era brutto e volto e petto e schene.\\
        Costui si scagliò lor come cagnazzo\\
        ch'assalir forestier subito viene;\\
        e diè lor \textbf{noia}, e fu per far lor scorno.\\
        Ma di Marfisa a ricontarvi torno.
    }{Ma non vi giunsero in tempo per evitare di incontrare un uomo folle
    che giaceva sulla battigia, sul limite della spiaggia,
    e che, come fosse un porco, di fango e di acqua
    era completamente sporco in volto, petto e schiena.
    Costui sì lanciò  con violenza contro di loro così come un cagnaccio
    subito va ad assalire uno straniero;
    e diede loro tormento e fu sul punto di recare loro danno.
    Ma torno ora a raccontarvi nuovamente di Marfisa.}
    \\\\
    In questa ottava avviene un colpo di scena; trovano un pazzo lì sulla spiaggia,
    completamente imbrattato di fango e di terra (come un porco - simulitudine animalesca).
    Questo pazzo si scaglia alla coppia aggredendoli
    (come un cagnazzo salta addosso ad una persona che non conosce).
    \\\\
    L'ottava è divisa in sette versi più uno, dove nell'ultimo verso
    il narratore sospende la storia sul più bello.
    Non si sa chi sia il pazzo nè se farà effettivamente del male.
    Verrà svelato che l'uomo è in realtà Orlando, evidentemente
    già privo di senno.

    Il passo si chiude con il cambio di focalizzazione del personaggio
    (Marfisa), utilizzando la tecnina dell'entrelacement.
\end{snippet}

\end{document}