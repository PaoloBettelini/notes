\documentclass[preview]{standalone}

\usepackage{amsmath}
\usepackage{amssymb}
\usepackage{stellar}
\usepackage{bettelini}
\usepackage{chronology}

\hypersetup{
    colorlinks=true,
    linkcolor=black,
    urlcolor=blue,
    pdftitle={Stellar},
    pdfpagemode=FullScreen,
}

\begin{document}

\title{Geografia economica}
\id{geoeconomica-edgar-morin}
\genpage

\section{La Francia ferita nell'epoca delle policrisi, Edgar Morin}

\includesnpt[href=/snippet/static/edgar-morin-policrisi.pdf|text=Testo:|display=Edgar Morin (pdf)]{file-url}
\plain{<span></span>}

\begin{snippetexercise}{edgar-morin-ex1}
{Identifica e sintetizza in una linea del tempo i principali riferimenti storici}
    \begin{chronology}[25]{1789}{2023}{\textwidth}
        \event{1789}{Rivoluzione Francese (1789)}
        \event{1830}{Rivoluzioni e crisi (1830-1906)}
        \event{1901}{Multiculturalismo francese (1901)}
        \event{1914}{Prima guerra mondiale (1901)}
        \event{1929}{Crisi economica (1929)}
        \event{1939}{Seconda guerra mondiale (1939)}
        \event{1947}{Guerra Fredda (1947)}
        \event{1956}{Crisi del comunismo (1956)}
        \event{1968}{Rivolta studentesca (1968)}
        \event{1974}{National Rally (1972)} % cambiato anno per non sovrascrivere
        \event{1989}{Caduta Muro Berlino (1989)}
        \event{2014}{Guerra Russo-Ucrania (2014)}
    \end{chronology}
\end{snippetexercise}

\begin{snippetexercise}{edgar-morin-ex2}
{Identifica i riferimenti spaziali e specifica quali sono le scale geografiche mobilitate dall'autore}
L'autore cita riferimenti spaziali su tre diverse scale geografiche.
Vengono citate per la scala nazionale nazioni con grandi influenze politiche, quali la Francia,
Russia e Stati Uniti.
Vengono citate l'Europa, Africa o l'Occidente.
Come scala più ampia viene citata quella globale.
\end{snippetexercise}

\begin{snippetdefinition}{policrisi-definizione}{Policrisi}
    Una \textit{policrisi} è un insieme di molteplici eventi nefasti e interdipendenti che potrebbero
    portare a danni su grande scala (planetaria).
\end{snippetdefinition}

\plain{Il termine policrisi sottolinea l'idea che il mondo moderno sia
caratterizzato da una profonda interconnettività.}

\begin{snippetexercise}{edgar-morin-ex3}
{Proponi una riflessione argomentata sui contenuti delle ultime 4 righe del testo}
    Evitare che la Francia (la Repubblica) si trasformi in uno stato del controllo
    è uno degli step fondamentali per ridurre la policrisi,
    essendo la Francia un importante tassello della policrisi globale.
\end{snippetexercise}

\begin{snippetexercise}{edgar-morin-ex4}
{Riassumi in due (o tre) frasi il contenuto del testo}
    Le relazioni fra eventi di scale diverse, sia a livello locale che a quello globale, 
    sono interdipendenti e portano a situazioni di crisi globali.

    Eventi che sono all'apparenza locali possono avere grandi effetti in crisi di scala maggiore.
\end{snippetexercise}

\end{document}