\documentclass[preview]{standalone}

\usepackage{amsmath}
\usepackage{amssymb}
\usepackage{stellar}

\hypersetup{
    colorlinks=true,
    linkcolor=black,
    urlcolor=blue,
    pdftitle={Stellar},
    pdfpagemode=FullScreen,
}

\begin{document}

\id{cesare-beccaria-proporzioni-delitti-e-pene}
\genpage

\section{Proporzione fra i delitti e le pene}

\begin{snippet}{proporzione-fra-delitte-e-pene}
    Non tutti i delitti sono uguali, e quelli più gravi è auspicabile che vengano
    commessi più raramente.
    Inoltre, più una società diventa grande e complessa, più i delitti aumentano
    e vi è il rischio che le pene debbano essere sempre più dure.
    Questi ostacolo non possono toglierti la libertà di delinquere, bensì minimizzano la
    gravità della situazione.
    Il criterio per stabilire la gravità di un delitto è la tutela del deposito (patto sociale).
    È impossibile creare una scala di delitti discreta (come in \(\mathbb{N}\)),
    perché è impossibile delineare dove un delitto termini e ne cominci un altro,
    bensì deve essere per forza continua e densa (come in \(\mathbb{R}\)).
    \\\\
    Siccome ci sono infinite delitti, è quindi impossibile catalogare
    con precisione le pene corrispondenti.
    \\\\
    Se questo sistema non dovesse funzionare, si arriverebbe addirittura ad una situazione
    dove i delitti sono creati dalle pene.
    Se una pena uguale è destinata a due delitti, che disugualmente offendono la società, gli
    uomini non troveranno un più forte ostacolo per commettere il
    maggior delitto, se con esso vi trovino unito un maggior vantaggio.
\end{snippet}

\end{document}