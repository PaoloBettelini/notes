\documentclass[preview]{standalone}

\usepackage{amsmath}
\usepackage{amssymb}
\usepackage{stellar}
\usepackage{definitions}
\usepackage{bettelini}

\begin{document}

\title{Biologia}
\id{biologia-lipidi}
\genpage

\section{Lipidi}

\begin{snippetdefinition}{lipidi-definition}{Lipidi}
    I \textit{lipidi} sono una classe di composti idrofobi (idrorepellenti)
    costituiti prevalentemente da atomi di carbonio e idrogeno.
\end{snippetdefinition}

\plain{I lipidi non sono strutturati con monomeri e polimeri.}

\section{Classi di lipidi}

\subsection{Trigliceridi}

\begin{snippetdefinition}{trigliceride-definition}{Trigliceride}
    Il \textit{trigliceride} costituisce una riserva energetica della cellula (comunemente grasso).
\end{snippetdefinition}

\begin{snippet}{trigliceride-expl1}
    Il monogliceride è composto da un glicerolo, attaccato (per condensazione) ad un acido grasso.
    Il trigliceride è attaccato a 3 catene di acido grasso.

    Le catene di acidi grassi possono essere dritti (saturi) oppure piegate (insaturi)
    in quanto posseggono un doppio legame.
    I grassi saturi sono generalmente solidi a temperatura ambiente (es. burro),
    mentre quelli insaturi sono liquidi (es. olio)
\end{snippet}

\includesnpt[width=50\%|src=/snippet/static/mono-trigliceride.png]{centered-img}

\subsection{Fosfolipidi}

\begin{snippetdefinition}{Fosfolipide-definition}{Fosfolipide}
    Il \textit{fosfolipide} sono composti da una testa idrofila e da due code idrofobe.
\end{snippetdefinition}

\plain{Le catarriteristiche idrofobe e idrofile permettono ai fosfolipidi di disporsi in maniera ordinata,
con la testa verso l'acqua e la coda rivolta in altra direzione.}

\subsection{Steroidi}

\begin{snippetdefinition}{steroide-definition}{Steroide}
    Lo \textit{steroide} è una molecola con una struttura di 4 anelli.
\end{snippetdefinition}

\plain{Alcuni esempi sono il colesterolo, testosterone ed estrogeno.}

\end{document}
