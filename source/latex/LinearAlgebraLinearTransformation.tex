\documentclass[preview]{standalone}

\usepackage{amsmath}
\usepackage{amssymb}
\usepackage{stellar}
\usepackage{definitions}

\hypersetup{
    colorlinks=true,
    linkcolor=black,
    urlcolor=blue,
    pdftitle={Stellar},
    pdfpagemode=FullScreen,
}

\stackMath{}

\begin{document}

\title{Stellar}
\id{linearalgebra-linear-transformation}
\genpage

\section{Linear Transformation}

\subsection{Definition}

\begin{snippetdefinition}{linear-transformation-definition}{Linear Transformation}
    Let \(\mathcal{V}\) and \(\mathcal{W}\) be vector spaces over a field \((K, +, \cdot)\).
    A \function \(f \colon \mathcal{V} \to \mathcal{W}\) is a
    \textit{linear transformation} if for any two vector \(\vec{u}, \vec{v} \in \mathcal{V}\)
    and any scalar \(c\in K\)
    \begin{enumerate}
        \item \(f(\vec{u} + \vec{v}) = f(\vec{u}) + f(\vec{v})\);
        \item \(f(c\cdot\vec{u}) = c\cdot f(\vec{u})\).
    \end{enumerate}
\end{snippetdefinition}

\subsection{The basis}

%\plain{Multiplying a vector by a matrix produces another vector. This is a linear transformation.
%The matrix contains the information about the transformation.}

\begin{snippettheorem}{linear-transformation-on-basis}{Linear transformation on basis}
    Given a basis for a vector space

    \[
        \mathcal{B}=\{\vec{b}_1, \vec{b}_2, \ldots, \vec{b}_n\}
    \]
    and a linear transformation \(T\), it suffices
    to know the value of \(T\mathcal{B}=\{T\vec{b}_1, T\vec{b}_2, \ldots, T\vec{b}_n\}\)
    to determine \(T\) applied to any vector on the vector space.
    Any transformation \(T\) is completely described by \(\mathcal{B}\)
    and \(T\mathcal{B}\).
\end{snippettheorem}

\begin{snippetproof}{linear-transformation-on-basis-proof}{linear-transformation-on-basis}{Linear transformation on basis}
    Given a basis for a vector space

    \[
        \mathcal{B}=\{\vec{b}_1, \vec{b}_2, \ldots, \vec{b}_n\}
    \]
    
    we can expand a vector \(\vec{a}\) along this basis
    
    \[
        \vec{a} = \sum_{k=1}^{n} \alpha_k \mathcal{B}_k
    \]
    
    We then apply a transformation \(T\) to the vector \(\vec{a}\) and use the properties of
    linear transformations
    
    \[
        T\vec{a}
        = T\sum_{k=1}^{n} \alpha_k \mathcal{B}_k
        = \sum_{k=1}^{n} \alpha_k T\mathcal{B}_k
    \]
\end{snippetproof}

\begin{snippet}{matrix-values-expl}
    Each column of a matrix is indeed the result of applying its transformation
    to the corresponding vector in the basis.
    Intuitively, this is given by the fact that we only need to know where the vectors of the basis
    end up after the transformation in order to represent the whole information.
\end{snippet}

\end{document}