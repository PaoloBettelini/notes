\documentclass[preview]{standalone}

\usepackage{amsmath}
\usepackage{amssymb}
\usepackage{stellar}
\usepackage{definitions}

\begin{document}

\id{topology-examples}
\genpage

\section{Examples}


\plain{The intersection of infinitely many open sets is not necessarily open, but their union is.}

\begin{snippetexample}{infinite-intersection-open-sets-not-open-example}{Intersection of infinite open sets is not open}
    Consider the \set[sets]
    \[
        A_k = \left(-\frac{1}{k}, \frac{1}{k}\right), \quad k\in\naturalnumbers^\exceptzero
    \]
    which are \msopenset[open].
    Then, their intersection is given by
    \[
        \bigcap_{k=1}^\infty A_k = \{0\}
    \]
    which is not an \msopenset.
\end{snippetexample}

\plain{The union of infinitely many closed sets is not necessarily closed, but their intersection is.}

\begin{snippetexample}{infinite-union-closed-sets-not-closed-example}{Union of infinite closed sets is not closed}
    Consider the \set[sets]
    \[
        A_k = \left[\frac{1}{k}, 1\right], \quad k\in\naturalnumbers^\exceptzero
    \]
    which are \msclosedset[closed].
    Then, their intersection is given by
    \[
        \bigcap_{k=1}^\infty A_k = (0, 1]
    \]
    which is not a \msclosedset.
\end{snippetexample}

\end{document}