\documentclass[preview]{standalone}

\usepackage{amsmath}
\usepackage{amssymb}
\usepackage{bettelini}
\usepackage{stellar}
\usepackage{definitions}

\begin{document}

\id{zetafunction-zeroes}
\genpage

\section{Trivial zeroes}

\begin{snippetdefinition}{trivial-zero-zeta-function-definition}{Trivial zero of \(\zeta(z)\)}
    A \textit{trivial zero} of the zeta function is a point of the form \(p = 2n\)
    where \(n \in \mathbb{Z}^-\) such that \(\zeta(p)=0\).
\end{snippetdefinition}

\begin{snippetproposition}{trivial-zeroes-zeta-function}{Trivial zeroes of \(\zeta(z)\)}
    Every negative even integer is a trivial zero of the zeta function.
\end{snippetproposition}

\begin{snippetproof}{trivial-zeroes-zeta-function-proof}{Trivial zeroes of \(\zeta(z)\)}
    Consider the functional equation and analytic continuation of the zeta function
    \begin{align*}
        \zeta(s)=2^s\pi^{s-1}\sin\left(\frac{\pi s}{2}\right)\Gamma(1-s)\zeta(1-s)
    \end{align*}
    We can notice that the term \(\sin\left(\frac{\pi s}{2}\right)\) equals \(0\) when \(s\) is a multiple of \(2\).
    \\
    However, the gamma function has a pole for every negative integer, this constrains our zeroes to be less or equal to 1.\\
    Furthermore, the zeta function has a pole at \(s=1\), excluding the value \(s=0\) from the zeroes, leaving \(\{-2;-4;-6;\cdots\}\)
    \begin{align*}
        \zeta(2k)=0,
        \quad k\in \mathbb{Z}^{-}
    \end{align*}
\end{snippetproof}

\section{Non-trivial zeroes}

\begin{snippetdefinition}{non-trivial-zero-zeta-function-definition}{Non-trivial zero of \(\zeta(z)\)}
    A \textit{non-trivial zero} of the zeta function is a zero of \(\zeta(z)\)
    that is not a trivial zero.
\end{snippetdefinition}

\begin{snippetproposition}{riemann-hypothesis}{Riemann Hypothesis}
    Every non-trivial zero of \(\zeta(z)\) has the form \(\frac{1}{2} + it\)
    for some \(t \in \mathbb{R}\).
\end{snippetproposition}

% TODO
% 4 symmetries
% critical strip contians all the non trivial zeroes

\end{document}
