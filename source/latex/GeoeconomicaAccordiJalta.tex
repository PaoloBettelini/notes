\documentclass[preview]{standalone}

\usepackage{amsmath}
\usepackage{amssymb}
\usepackage{stellar}
\usepackage{bettelini}

\hypersetup{
    colorlinks=true,
    linkcolor=black,
    urlcolor=blue,
    pdftitle={Stellar},
    pdfpagemode=FullScreen,
}

\begin{document}

\title{Geografia economica}
\id{geoeconomica-accordi-jalta}
\genpage

\section{Gli accordi di Jalta}

\begin{snippetdefinition}{accordi-jalta-definizione}{Accordi di Jalta}
    Gli \textit{Accordi di Jalta} furono un incontro tenutosi nel febbraio 1945 tra i leader delle tre principali potenze alleate della Seconda Guerra Mondiale: Stati Uniti, Regno Unito e Unione Sovietica, rappresentati rispettivamente da Franklin D. Roosevelt, Winston Churchill e Joseph Stalin.
\end{snippetdefinition}

\begin{snippet}{accordi-jalta-expl}
    Durante questa conferenza, i tre leader discussero principalmente il destino dell'Europa dopo la fine imminente della guerra. Le principali decisioni includevano:

    \begin{itemize}
        \item suddivisione della Germania in quattro zone di occupazione controllate rispettivamente
            da Stati Uniti, Regno Unito, Unione Sovietica e Francia;
        \item creazione di un'organizzazione internazionale per promuovere la pace e la cooperazione tra le nazioni, che in seguito divenne l'\textit{Organizzazione delle Nazioni Unite} (ONU);
        \item i paesi liberati possono istituire liberamente i loro governi (possibilmente democratici) con elezioni libere.
    \end{itemize}
\end{snippet}

\end{document}