\documentclass[preview]{standalone}

\usepackage{amsmath}
\usepackage{amssymb}
\usepackage{stellar}
\usepackage{bettelini}

\hypersetup{
    colorlinks=true,
    linkcolor=black,
    urlcolor=blue,
    pdftitle={Biologia},
    pdfpagemode=FullScreen,
}

\begin{document}

\title{Biologia}
\id{biologia-proteine}
\genpage

\section{Proteine}

\plain{I monomeri di proteine si chiamano <b>amminoacidi</b>.}

\plain{Ci sono 20 possibili amminoacidi diversi.}

\begin{snippetdefinition}{catena-polipetidica-definition}{Catena Polipeptidica}
    Una \textit{catena polipeptidica} è una catena di amminoacidi.
\end{snippetdefinition}

\begin{snippetdefinition}{proteina-definition}{Proteina}
    Le \textit{proteine} sono delle biomolecole costruite
    da una o più catene polipeptidiche.
\end{snippetdefinition}

\begin{snippet}{proteine-classi}
    Le proteine si distinguono in 7 classi per funzione
    \begin{enumerate}
        \item \textbf{Strutturali:} es. unghie (cheratina), elastina, collagene.
        \item \textbf{Contrattili:} costituiscono il muscolo. Es. actina, miosina.
        \item Di \textbf{riserva:} costituiscono una riserva di amminoacidi (specialmente per l'embrione).
        Es. albumina, caseina.
        \item Di \textbf{difesa:} costituiscono gli anticorpi, neutralizzano gli agenti patogeni.
        \item Di \textbf{trasporto:} trasportano l'ossigeno all'interno del sistema circolatorio. Es. Emoglobina.
        \item \textbf{Regolatrici:} costituiscono alcuni ormoni. Es. insulina.
        \item \textbf{Enzimi:} costituiscono gli enzimi.
    \end{enumerate}
\end{snippet}

\subsection{Struttura}

\includesnpt[width=40\%|src=/snippet/static/composizione-amminoacidi.png]{centered-img}

\includesnpt[width=70\%|src=/snippet/static/struttura-proteina.png]{centered-img}

\begin{snippet}{struttura-proteina-expl}
    \begin{itemize}
        \item \textbf{struttura primaria:} polipeptide \(\rightarrow\) aminoacidi creati tramite legami peptidici;
        \item \textbf{struttura secondaria:} alpha eliche e foglietti elica
        \item \textbf{struttura terziaria:} più alpha eliche rivoluzionate (che prendono una forma) \(\rightarrow\) ha una funzione specifica
        e quindi si può chiamare proteina;
        \item \textbf{struttura quaternaria:} aggregazione di 1 o più strutture terziarie (polipeptidi).
    \end{itemize}
\end{snippet}

\end{document}