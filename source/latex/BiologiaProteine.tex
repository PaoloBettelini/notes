\documentclass[preview]{standalone}

\usepackage{amsmath}
\usepackage{amssymb}
\usepackage{stellar}
\usepackage{bettelini}

\hypersetup{
    colorlinks=true,
    linkcolor=black,
    urlcolor=blue,
    pdftitle={Biologia},
    pdfpagemode=FullScreen,
}

\begin{document}

\title{Biologia}
\id{biologia-proteine}
\genpage

\plain{I monomero di proteine si chiamano <b>amminoacidi</b>.}

% TODO foto

\plain{Ci sono 20 possibili amminoacidi diversi.}

\begin{snippetdefinition}{catena-polipetidica-definition}{Catena Polipeptidica}
    Una \textit{catena polipeptidica} è una catena di amminoacidi.
\end{snippetdefinition}

\begin{snippetdefinition}{proteina-definition}{Proteina}
    Le \textit{proteine} sono delle biomolecole costruite
    da una o più catene polipeptidiche.
\end{snippetdefinition}

\begin{snippet}{proteine-classi}
    Le proteine si distinguono in 7 classi per funzione
    \begin{enumerate}
        \item \textbf{Strutturali:} es. unghie (cheratina).
        % Elastina, Collagene

        \item \textbf{Contrattili:} costituiscono il muscolo.
        % Actina e Miosina

        \item Di \textbf{riserva:} costituiscono una riserva di amminoacidi (specialmente per l'embrione).
        % Es. Albumina, Caseina

        \item Di \textbf{difesa:} costituiscono gli anticorpi, neutralizzano gli agenti patogeni.
        
        \item Di \textbf{trasporto:} trasportano l'ossigeno all'interno del sistema circolatorio.
        % Emoglobina

        \item \textbf{Regolatrici:} costituiscono alcuni ormoni.
        % Insulina

        \item \textbf{Enzimi:} costituiscono gli enzimi.
    \end{enumerate}
\end{snippet}

\end{document}