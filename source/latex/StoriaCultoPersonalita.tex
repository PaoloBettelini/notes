\documentclass[preview]{standalone}

\usepackage{amsmath}
\usepackage{amssymb}
\usepackage{stellar}

\hypersetup{
    colorlinks=true,
    linkcolor=black,
    urlcolor=blue,
    pdftitle={Stellar},
    pdfpagemode=FullScreen,
}

\begin{document}

\title{Stellar}
\id{storia-culto-personalita}
\genpage

\section{Culto della personalità}

\begin{snippetdefinition}{realismo-socialista-definizione}{Realismo Socialista}
    Il \textit{realismo socialista} è un movimento artistico e culturale nato nell'Unione Sovietica nel 1934 e poi allargatosi a tutti i paesi socialisti del centro ed est Europa. La funzione principale era quella di avvicinare l'espressione artistica alla cultura delle classi proletarie e celebrare il progresso socialista.
\end{snippetdefinition}

\begin{snippetdefinition}{culto-personalita-definizione}{Culto della personalità}
    Il \textit{culto della personalità} è una forma di idolatria sociale che
    generalmente si configura nell'assoluta devozione a un leader,
    solitamente politico o religioso, attraverso l'esaltazione del pensiero e 
    delle capacità, tanto da attribuirgli doti di infallibilità.
\end{snippetdefinition}

\begin{snippet}{0c6f8469-2ce4-4624-beef-ff605b7bf56b}
    Questa espressione indica la tendenza dei regimi totalitari
    a celebrare in modo acritico la persona, l'azione e il pensiero del proprio leader.
\end{snippet}

\includesnpt[src=/snippet/static/stalin.jpg|width=50\%]{centered-img}

\end{document}