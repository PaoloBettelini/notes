\documentclass[preview]{standalone}

\usepackage{amsmath}
\usepackage{amssymb}
\usepackage{stellar}
\usepackage{definitions}

\begin{document}

\id{metric-spaces-sequences}
\genpage

\section{Sequences}

\begin{snippetdefinition}{metric-space-convergence-definition}{Metric space convergence}
    Let \((X, d)\) be a \metricspace.
    A sequence \(\{x_n\}\) is said to \textit{converge}
    to \(\alpha \in X\) (written \(x_n\to\alpha\)) if, for any \(\epsilon > 0\),
    there exists an integer \(N_\epsilon \in \naturalnumbers\) such that
    \[\forall n \geq N_\epsilon, d(x_n, \alpha) < \epsilon\]
\end{snippetdefinition}

\plain{This means that the sequence approaches some specific point as it progresses.}

\begin{snippetdefinition}{metricspaces-cauchy-sequence-definition}{Cauchy sequence}
    Let \((X, d)\) be a \metricspace.
    A sequence \(\{x_n\}\) is said to be \textit{Cauchy}
    if \(\forall \epsilon > 0\), there is an \(N\in{\naturalnumbers}^+\)
    such that \[ \forall m,n \in {\naturalnumbers}^+ \suchthat m,n > N, \quad d(x_m, x_n) < \epsilon \]
\end{snippetdefinition}

\plain{This means that the terms of the sequence get arbitrarily close to each other as the sequence progresses.}

\begin{snippetproposition}{convergent-sequence-is-cauchy}{Convergent sequences are Cauchy}
    Let \(M\) be a \metricspace.
    Every convergent sequence is \mscauchy.
\end{snippetproposition}

\begin{snippetdefinition}{metric-space-complete-definition}{Complete metric space}
    A \metricspace \((X, d)\) is \textit{complete} if every \mscauchy[Cauchy sequence]
    converges to an element of \(X\).
\end{snippetdefinition}

%Example:
%Consider the space of rational numbers QQ with the standard metric.
%There are Cauchy sequences of rationals (e.g., sequences that approach sqrt{2})
%that do not converge to a rational number (since sqrt{2} is not rational).
%Thus, in Q, a Cauchy sequence is not necessarily convergent.

\section{Contractions}

\begin{snippetdefinition}{contraction-definition}{Contraction}
    Let \((X, d)\) be a \metricspace.
    A \textit{contraction} on \((X, d)\) is a \function \(f\colon X \to X\),
    with the following property:
    \[
        \exists k \in [0;1) \suchthat \forall x,y \in X, \quad d(f(x), f(y)) \leq k\cdot d(x,y)
    \]
\end{snippetdefinition}

\plain{A contraction brings two points always closer to eachother, according to the distance function.}

\begin{snippettheorem}{banach-fixed-point-theorem}{Banach Fixed-Point Theorem}
    TODO.
\end{snippettheorem}

\end{document}