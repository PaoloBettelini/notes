\documentclass[preview]{standalone}

\usepackage{amsmath}
\usepackage{amssymb}
\usepackage{stellar}
\usepackage{definitions}
\usepackage{tikz}
\usepackage{chemfig}
\usepackage{makecell}
\usepackage{soul}

% === COMMANDS ===
\newcommand*\circled[1]{\tikz[baseline=(char.base)]{
    \node[shape=circle,draw,inner sep=1.1pt] (char) {#1};}
}
\newcommand\angstrom{\mbox{\normalfont\AA}}
\newcommand\namebond[4][5pt]{\chemmove{\path(#2)--(#3)node[midway,sloped,yshift=#1]{#4};}}
\newcommand\arcbetweennodes[3]{
    \pgfmathanglebetweenpoints{\pgfpointanchor{#1}{center}}{\pgfpointanchor{#2}{center}}
    \let#3\pgfmathresult
}
\newcommand\arclabel[8][stealth-stealth,shorten <=1pt,shorten >=1pt]{
    \chemmove{
        \arcbetweennodes{#4}{#3}\anglestart \arcbetweennodes{#4}{#5}\angleend
        \draw[#1]([shift=(\anglestart:#2)]#4)arc(\anglestart:\angleend:#2);
        \pgfmathparse{(\anglestart+\angleend)/2}\let\anglestart\pgfmathresult
        \node[shift=(\anglestart:#2+1pt)#4,anchor=\anglestart+180,rotate=\anglestart#7,
        inner sep=0pt,outer sep=#8]at(#4){#6};
    }
}

\begin{document}

\id{chimica-vsepr}
\genpage

\section{Geometria molecolare VSEPR}

\begin{snippetdefinition}{vsepr-definition}{VSEPR}
    La geometria di una molecola può essere rappresentata ricorrendo al modello VSEPR
    (\textit{Valence Shell Electron Pair Repulsion}), il quale principio è basato sulla repulsione degli
    elettroni di valenza, i quali tendono a collocarsi alla maggior distanza possibile l'uno
    dall'altro.
\end{snippetdefinition}

\subsection{Lineare}

\begin{snippet}{vsepr-lineare-illustration}
    \begin{center}
        \chemfig{
            \Charge{90=\|,-90=\|}{O}=C=\Charge{90=\|,-90=\|}{O}
        } \qquad
        \chemfig{
            @{1}\circled{O}=@{2}\circled{C}=@{3}\circled{O}
        } \arclabel{-0.5cm}{1}{2}{3}{\footnotesize180\textdegree}{-90}{-8.5pt}
    \end{center}
    \phantom{}
\end{snippet}

\includesnpt[cid=280]{molecule}

\subsection{Piegata}

\begin{snippet}{vsepr-piegata-illustration}
    \begin{center}
        \chemfig{
            \Charge{45=\|,135=\|}{O}(-[:-45]H)(-[:-135]H)
        }
        \hspace*{1.15cm}
        \chemfig{
            @{1}\circled{O}(-[:-30]@{2}\circled{H})(-[:-150]@{0}\circled{H})
        } \arclabel{.6cm}{0}{1}{2}{\footnotesize104.5\textdegree}{-270}{.5}
    \end{center}
    \phantom{}
\end{snippet}

\includesnpt[cid=962]{molecule}

\subsection{Triangolare planare}

\begin{snippet}{vsepr-triangolare-planare-illustration}
    \begin{center}
        \chemleft[
            \chemfig{
                {\Charge{90=\|,-180=\|}{O}}
                =[:-45]C(-[:-90]{\Charge{0=\|,-90=\|,-180=\|}{O}})
                -[:45]\Charge{45=\|,135=\|,-45=\|}{O}
            } 
        \chemright]$^{2-}$
        \hspace*{.35cm}
        \chemfig{
            @{0}\circled{O}
            -[:-45]@{1}\circled{C}(-[:-90]@{2}\circled{O})
            =[:45]@{3}\circled{O}
        } \arclabel{0.5cm}{0}{1}{2}{\footnotesize120\textdegree}{-202.5}{0}
    \end{center}
    \phantom{}
\end{snippet}

\includesnpt[cid=6356]{molecule}

\subsection{Piramidale}

\begin{snippet}{vsepr-piramidale-illustration}
    \begin{center}
        \chemfig{
            \Charge{90=\|}{N}(<:[:-135]H)(<[:-90]H)(-[:-45]H)
        }
        \hspace*{1cm}
        \chemfig{
            @{0}\circled{N}(-[:-165]\circled{H})(-[:-90]@{1}\circled{H})(-[:-15]@{2}\circled{H})
        } \arclabel{.7cm}{1}{0}{2}{\footnotesize109\textdegree}{-270}{.5}
    \end{center}
    \phantom{}
\end{snippet}

\includesnpt[cid=222]{molecule}

\subsection{Tetraedrica}

\begin{snippet}{vsepr-tetraedrica-illustration}
    \begin{center}
        \chemfig{
            H-[:-90]C(<:[:-135]H)(<[:-90]H)(-[:-45]H)
        }
        \hspace*{1cm}
        \chemfig{
            @{0}\circled{H}-[:-90]@{1}\circled{C}(-[:-165]@{2}\circled{H})(-[:-90]\circled{H})(-[:-15]\circled{H})
        } \arclabel{.6cm}{0}{1}{2}{\footnotesize107.3\textdegree}{-142.5}{2pt}
    \end{center}
    \phantom{}
\end{snippet}

\includesnpt[cid=297]{molecule}

\subsection{Bipiramidale trigonale}

\begin{snippet}{vsepr-bipiramidale-trigonale-illustration}
    \begin{center}
        \chemfig{
            Cl-[:-90]P(-Cl)(-[:-90]Cl)(<[:-135]Cl)(<:[:135]Cl)
        }
        \hspace*{.95cm}
        \chemfig{
            @{3}\circled{Cl}-[:-90]@{1}\circled{P}(-@{4}\circled{Cl})(-[:-90]\circled{Cl})
            (-[:-135]@{0}\circled{Cl})(-[:135]@{2}\circled{Cl})
        }
        \arclabel{.5cm}{0}{1}{2}{\footnotesize120\textdegree}{-180}{.5} 
        \arclabel{.6cm}{3}{1}{4}{\footnotesize90\textdegree}{-45}{2}
    \end{center}
    \phantom{}
\end{snippet}

\includesnpt[cid=24819]{molecule}

\subsection{Ottaedrica}

\begin{snippet}{vsepr-ottaedrica-illustration}
    \begin{center}
        \chemfig{
            F-[:-90]S(-[:-90]F)(<:[:30]F)(<:[:150]F)(<[:-150]F)(<[:-30]F)
        }
        \hspace*{1.05cm}
        \chemfig{
            \circled{F}-[:-90]@{1}\circled{S}(-[:-90]@{0}\circled{F})(-[:30]\circled{F})(-[:150]\circled{F})
            (-[:-150]\circled{F})(-[:-30]@{2}\circled{F})
        }
        \arclabel{.6cm}{0}{1}{2}{\footnotesize90\textdegree}{-299}{1}
    \end{center}
    \phantom{}
\end{snippet}

\includesnpt[cid=17358]{molecule}

\end{document}