\documentclass[preview]{standalone}

\usepackage{amsmath}
\usepackage{amssymb}
\usepackage{bettelini}
\usepackage{stellar}
\usepackage[version=4]{mhchem}

\hypersetup{
    colorlinks=true,
    linkcolor=black,
    urlcolor=blue,
    pdftitle={Stellar},
    pdfpagemode=FullScreen,
}

\begin{document}

\id{chimica-redox}
\genpage

\begin{snippetdefinition}{redox-definition}{Reazione di ossidoriduzione}
    Le reazioni di \textit{ossidoriduzione} (\textit{redox}) sono una reazione chimica in cui gli elettroni
    vengono trasferiti tra due reagenti che vi partecipano.
    Le reazioni di ossidoriduzione sono composte da un
    \begin{itemize}
        \item \textit{agente riducente:} la sostanza che si ossida donando elettroni;
        \item \textit{agente ossidante:} la sostanza ossida prendendo elettroni.
    \end{itemize}
\end{snippetdefinition}

\plain{Una reazione redox comporta un cambiamento nello stato di ossidazione di uno o più atomi.}

\section{Numeri di ossidazione}

\begin{snippetdefinition}{numero-ossidazione-definition}{Numero di ossidazione}
    Il \textit{numero di ossidazione} è la carica elettrica virtuale che si può
    attribuire a un atomo o a uno ione impegnato in un legame chimico,
    immaginando di spostare tutti gli elettroni del legame sull'atomo più elettronegativo.
\end{snippetdefinition}

\begin{snippet}{numeri-ossidazione-expl1}
Lo stato di ossidazione o numero di ossidazione di un atomo in una molecola rappresenta la sua capacità di perdere o guadagnare
elettroni in una reazione chimica. In una molecola neutra, la somma degli stati di ossidazione di tutti gli atomi è
sempre uguale a zero. Ciò significa che la somma degli elettroni persi da alcuni atomi è uguale alla somma
degli elettroni acquistati da altri atomi. In una molecola ha l'atomo con la più alta elettronegatività
sempre il numero di ossidazione negativo.

\begin{enumerate}
    \item Elementi singoli e liberi hanno sempre un numero di ossidazione di \(0\);
    \item ioni monoatomici hanno sempre un numero di ossidazione pari alla propria carica;
    \item il fluoro ha sempre numero di ossidazione di \(-1\) essendo il più elettronegativo;
    \item la somma di tutti i numeri di ossidazione degli atomi in una molecola deve essere
        uguale alla carica della particella;
    \item l'idrogeno ha numero di ossidazione \(+1\) nei composti con i non metalli e \(-1\)
        nei composti con i metalli;
    \item l'ossigeno ha quasi sempre un numero di ossidazione di \(-2\), \(-1\) per i perossidi;
    \item i metalli dei gruppi I e II hanno rispettivamente numeri di ossidazione di \(+1\) e \(+2\);
    \item gli alogeni hanno numero di ossidazione \(-1\) nei composti con i metalli e un numero di ossidazione
        positivo quando si legano a \(O\) o \(F\).
\end{enumerate}

%i metalli dei primi gruppi tendono a cedere i propri elettroni esterni OSSIDANDOSI;
%i non metalli, molto elettronegativi, acquistano invece elettroni, RIDUCENDOSI;

Se lo stato di ossidazione aumenta, la molecola si ossida (perde elettroni). \\
Se lo stato di ossidazione diminuisce, la molecola si riduce (acquista elettroni).

Tutte le combustioni sono reazioni ossidoriduzione, dove l'ossigeno è ossidante.
La formazione della ruggine è una ossidazione, dove l'ossigeno è ossidante.
\end{snippet}

\begin{snippetdefinition}{reazione-spontanea-definition}{Reazione spontanea}
    Una reazione è \textit{spontanea} se procede spontaneamente.
\end{snippetdefinition}

\section{Semireazioni}

\begin{snippetdefinition}{semireazioni-definition}{Semireazioni}
    Le \textit{semireazioni} (di \textit{ossidazione} e di \textit{riduzione})
    sono delle reazioni che indicano il numero di elettroni che vengono
    ceduti o accettati.
\end{snippetdefinition}

\end{document}
