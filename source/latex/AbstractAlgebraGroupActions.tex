\documentclass[preview]{standalone}

\usepackage{amsmath}
\usepackage{amssymb}
\usepackage{stellar}
\usepackage{definitions}
\usepackage{bettelini}

\begin{document}

\id{group-actions}
\genpage

\section{Group actions}

\begin{snippetdefinition}{right-group-action-definition}{(Right) group action}
    Let \(\Omega\) be a non-empty \set and let \((G, \circ)\) be a \group.
    A \emph{right action} of \(G\) on \(\Omega\) is a \function
    \[
        \Omega \times G \fromto \Omega
    \]
    such that:
    \begin{enumerate}
        \item \(\forall x\in \Omega, x1_G = x\);
        \item \(\forall x\in \Omega, \forall h \in G, (xg)h = x(gh)\).
    \end{enumerate}
    \emph{Syntax:} \(xg\) denoted the image of the pair \((x, g)\).
\end{snippetdefinition}

\begin{snippetdefinition}{left-group-action-definition}{(Left) group action}
    Let \(\Omega\) be a non-empty \set and let \((G, \circ)\) be a \group.
    A \emph{left action} of \(G\) on \(\Omega\) is a \function
    \[
        G \times \Omega \fromto \Omega
    \]
    such that:
    \begin{enumerate}
        \item \(\forall x\in \Omega, 1_Gx = x\);
        \item \(\forall x\in \Omega, \forall h \in G, g(hx) = (gh)x\).
    \end{enumerate}
    \emph{Syntax:} \(gx\) denoted the image of the pair \((g, x)\).
\end{snippetdefinition}

\plain{Translations, isometries, rotations with fixed center are all examples of group actions on a plane.}
\plain{The action that sends the tuple to the first element of the tuple, discarding the other, is a right action.}

\begin{snippet}{group-action-restrictions}
    \textbf{Subgroup restriction:} let \(G\) be a \group with an action on a \set \(\Omega\)
    and let \(H \subgroupleq G\). Then,
    an action of \(H\) on \(\Omega\) is naturally induced. \\
    \textbf{Subset restriction:} let \(G\) be a \group with an action on a \set \(\Omega\)
    and let \(\Theta \subseteq \Omega\). Then,
    it is not guaranteed that an action of \(G\) on \(\Theta\) can be induced.
    If the action is closed, then an action is induced.
\end{snippet}

\begin{snippetproposition}{group-action-properties-1}{Group action properties}
    Let \(G\) be a \group with an action on a \set \(\Omega\)
    and let \(g\in G\). Consider the \function \(\varphi_G\)
    from \(\Omega\) to itself which sends \(x\) to \(xg\).
    Then, \(\varphi_g\) is \bijective and
    \({(\varphi_g)}^{-1} = \varphi_{g^{-1}}\).
\end{snippetproposition}

\begin{snippetproof}{group-action-properties-1-proof}{group-action-properties-1}{Group action properties}
    We wanto to show that \(\varphi_g\) and \(\varphi_{g^{-1}}\) are inverses:
    \begin{align*}
        x(\varphi_g \varphi_{g^{-1}}) &= (x\varphi_g)\varphi_{g^{-1}} = (xg)g^{-1} \\
        &= x(gg^{-1}) = x1_G = x = x\text{Id}
    \end{align*}
    The other direction is analogous.
\end{snippetproof}

\section{Examples}

\begin{snippetdefinition}{regular-right-group-action}{Regular right action}
    Let \(\Omega\) be a non-empty \set and let \((G, \circ)\) be a \group.
    The \function
    \[
        (x, g) \fromto x \circ g
    \]
    is a right group action on the \group itself called \emph{regular right action}.
\end{snippetdefinition}

\plain{This is an automorphism if and only if the fixed element is the identity.}

\begin{snippetdefinition}{conjugate-action-definition}{Conjugate action}
    Let \(G\) be a \group.
    \[
        x^g \triangleq (x, g) \fromto g^{-1}xg
    \]
\end{snippetdefinition}

\plain{This action is always an automorphism.} %TODOURGENT proofs, 28 nov

\plain{In such case, we have a function thar associaites an element of the group
to an automorphism. Such function is a group homomorphism. Thus, the image of this
homomorphism forms a subgroup of the automorphisms of the group.}

\begin{snippetproposition}{conjugate-group-properties}{Properties of conjugate group}
    Let \(G\) be a \group.
    \begin{enumerate}
        \item \(x^{1_G} = x\);
        \item \({(x^g)}^h = (h^{-1}g^{-1})x(gh)\).
    \end{enumerate}
\end{snippetproposition}

\section{Action orbits}

\begin{snippetproposition}{bin-relation-group-action-is-equiv}{}
    Let \(G\) be a \group with an action on a \set \(\Omega\).
    Let \(\sim_G\) be a \binrelation in \(\Omega\) such that
    \(x \sim_G y\) \ifandonlyif there exist \(g\in G\)
    such that \(x=yg\). Then, \(\sim_G\) is an \equivrelation.
\end{snippetproposition}

\begin{snippetproof}{bin-relation-group-action-is-equiv-proof}{bin-relation-group-action-is-equiv}{}
    The \binrelation is an \equivrelation:
    \begin{enumerate}
        \item \emph{reflexive:} \(x \sim_G x\), indeed \(x1_G = x\);
        \item \emph{symmetric:} if \(x \sim_G y\), meaning there exist \(g\in G\)
        such that \(x=yg\), then \(xg^{-1} = (yg)g^{-1} = y(gg^{-1}) = y\).
        Thus, \(y \sim_G x\).
        \item \emph{transitivity:} if \(x \sim_G y\) and \(y\sim_G z\),
        there exist \(h,g \in G\) such that \(x=yg\) and \(y=zh\).
        Thus,
        \[
            x = yg = (zh)g = z(hg)
        \]
        and thus \(x \sim_G z\).
    \end{enumerate}
\end{snippetproof}

\begin{snippetdefinition}{group-action-orbit-definition}{Orbit}
    Let \(G\) be a \group with an action on a \set \(\Omega\).
    The \equivclass[equivalence classes] of \(\sim_G\) defined
    as \(y \sim_G y\) \ifandonlyif there exist \(g\in G\)
    such that \(x=yg\), are called \emph{orbits}. \\
    \emph{Syntax:} let \(y\in G\). The \equivclass of \(y\)
    is formed by the elements of the form \(yg\) for \(g\in G\)
    and denoted \(yG\).
\end{snippetdefinition}

\begin{snippet}{group-action-orbit-expl}
    Given a fixed \(g\in G\), the elements of form \(yg\)
    are all the elements of \(\Omega\) as the function \(y \to yg\)
    is \bijective. \\
    Given a fixed \(y\in \Omega\), the elements of form \(yg\)
    are the ones of an orbit (could be all the elements of \(\Omega\) if there is only an orbit).
\end{snippet}

\subsection{Orbits of actions on groupselves}

\begin{snippetproposition}{regular-right-action-orbit}{Orbits of regualr right action}
    Let \(G\) be a \group and consider the regular action on itself.
    Given \(x\in G\), its orbit is formed by the elements of \(xg\) for \(g\in G\).
    By the cancellation laws, there exist only one orbit.
\end{snippetproposition}

\plain{Thus, every element cn be send into any other element.}

\begin{snippetproposition}{conjugate-action-orbit}{Orbits of conjugate action}
    Let \(G\) be a \group and consider the conjugate action on itself.
    There is always an orbit containing only \(1_G\).
    If the \group is not trivial, there are always more orbits,
    which are called \emph{conjugate classes} \(x^G\).
\end{snippetproposition}

\begin{snippetproposition}{group-center-conjugate-class}{Group center conjugate class}
    Let \(G\) be a \group.
    \(X \in \groupcenter(G)\) \ifandonlyif its conjugate class
    contains only \(X\).
\end{snippetproposition}

\section{Stabilizer}

\begin{snippetdefinition}{stabilizer-definition}{Stabilizer}
    Let \(G\) be a \group with an action on \(\Omega\)
    and let \(x\in\Omega\).
    The \emph{stabilizer} of \(x\) is the subset
    \[
        \text{Stab}_G(x) \triangleq \{
            g\in G \suchthat xg = x    
        \}
    \]
\end{snippetdefinition}

\begin{snippetproposition}{stabilizer-is-subgroup}{}
    Let \(G\) be a \group. Then,
    \[
        \text{Stab}_G(x) \subgroupleq G
    \]
\end{snippetproposition}

\begin{snippetproof}{stabilizer-is-subgroup-proof}{stabilizer-is-subgroup}{}
    \begin{enumerate}
        \item \(1_G \in \text{Stab}_G(x)\) as \(x1_G = x\);
        \item if \(g \in \text{Stab}_G(x)\) and \(h \in \text{Stab}_G(x)\),
        meaning \(xg=x\) ahd \(xh=x\), then \(x(gh) = (xg)h = xh = x\),
        meaning \(gh \in \text{Stab}_G(x)\);
        \item if \(g\in\text{Stab}_G(x)\), meaning \(xg = x\), then
        \(xgg^{-1} = xg^{-1}\) and \(xg^{-1} = x\) meaning \(g^{-1} \in \text{Stab}_G(x)\).
    \end{enumerate}
\end{snippetproof}

\begin{snippetproposition}{stabilizer-right-lateral-equivalence}{}
    Let \(G\) be a \group with an action on \(\Omega\)
    and let \(x\in\Omega\). The following are equivalent
    \begin{enumerate}
        \item \(xg = xh\);
        \item \(\text{Stab}_G(x)g = \text{Stab}_G(x)h\) meaning \(h\) and \(g\)
        are in the same right lateral of the stabilizer.
    \end{enumerate}
\end{snippetproposition}

\begin{snippetproof}{stabilizer-right-lateral-equivalence-proof}{stabilizer-right-lateral-equivalence}{}
    \iffproof{
        From \(xg=xh\) we get \(xgh^{-1} = x\), meaning \(gh^{-1} \in \text{Stab}_G(x)\)
        and thus \[
            g = {(gh^{-1})}h \in \text{Stab}_G(x)h
            \implies \text{Stab}_G(x)g = \text{Stab}_G(x)h
        \]
    }{
        Let \(g = sh\) with \(s\in\text{Stable}_G(x)\).
        Then, \(xg = x(sh) = (xs)h = xh\).
    }
\end{snippetproof}

\begin{snippetcorollary}{stabilizer-index-size}{}
    Let \(G\) be a \group with an action on \(\Omega\)
    and let \(x\in\Omega\). Then,
    \[
        |G\,:\,\text{Stab}_G(x)| = |xG|
    \]
\end{snippetcorollary}

\plain{The bigger the index, the bigger the orbit.}

\begin{snippetproposition}{stabilizer-cardinality-finite-group}{}
    Let \(G\) be a \group with an action on \(\Omega\)
    where \(G\) is finite. Then,
    \(|xG|\) divides \(|G|\).
\end{snippetproposition}

\begin{snippet}{stabilizer-of-conjugate-action}
    Consider the conjugate action on a \group \(G\).
    We have that
    \[
        \text{Stab}_G(x) = \{g \in G \suchthat g^{-1} x g = x\}
    \]
    Note that \(g^{-1} x g = x\) \ifandonlyif \(xg = gx\).
    Thus,
    \[
        \text{Stab}_G(x) = \groupcentralizer_G(x)
    \]
    and
    \[
        |x^G| = |G \,:\, \groupcentralizer_G(x)|
    \]
    In particular, \(|x^G| = 1\) \ifandonlyif \(\groupcentralizer_G(x) = G\),
    meaning \(x \in \groupcenter(G)\).
\end{snippet}

\plain{The more elements commute, the more classes we have, and the closer the group is to being abelian.}

\begin{snippettheorem}{conjugate-classes-equations-theorem}{}
    Let \(G\) be a finite group. We have
    \[
        |G| = |\groupcenter(G)| + \sum_i C_i
    \]
    where \(C_i\) are divisors of \(|G|\), greater than \(1\),
    and are precisely the orders of the conjugate classes
    of non-central elements.
\end{snippettheorem}

\begin{snippetproof}{conjugate-classes-equations-theorem-proof}{conjugate-classes-equations-theorem}{}
    Note that \(|G|\) is the sum of the orders
    of the conjugate classes, as they form a partition since they are \equivclass[equivalent classes].
    These cardinalities divide \(|G|\).
    The classes with only one element are the ones in the center,
    and thus the amount of indexes for which \(C_i = 1\) is \( |\groupcenter(G)|\).
\end{snippetproof}

\begin{snippetdefinition}{p-group-definition}{P-groups}
    Let \(G\) be a finite \group whose order is a power of a \primen.
    Then, \(G\) is a \emph{p-group}.
\end{snippetdefinition}

\begin{snippettheorem}{center-of-p-group-not-trivial-theorem}{Center of p-group is not trivial}
    Let \(G\) be a non-trivial p-group. Then,
    \(\groupcenter(G)\) is not trivial.
\end{snippettheorem}

\begin{snippetproof}{center-of-p-group-not-trivial-theorem-proof}{center-of-p-group-not-trivial-theorem}{Center of p-group is not trivial}
    We have the equation
    \[
        |G| = |\groupcenter(G)| + \sum_i C_i
    \]
    Now, \(|G| = p^n\) for some \(n \geq 1\).
    In particlar, \(|G|\) is a multiple of \(p\).
    Furthermore, \(C_i\) is a divisor of \(p^n\) greater than \(1\)
    and thus \(C_i\) are all multiples of \(p\).
    Thus,
    \[
        |\groupcenter(G)| = |G| - \sum_i C_i
    \]
    is a multiple of \(p\) and \(|\groupcenter(G)| \neq 0\) impliest that \(|\groupcenter(G)| \geq p\).
    In particular, \(|\groupcenter(G)| \neq 1\). 
\end{snippetproof}

\section{Conjugate subgroup}

\begin{snippetdefinition}{conjugate-subgroup-definition}{The conjugate subgroup}
    Let \(H \subgroupleq G\) be \group[groups]. Then, the \textit{conjugate subgroup} is defined as
    \[
        H^g = \{
            g^{-1}hg \suchthat h \in H
        \}
    \]
\end{snippetdefinition}

\begin{snippettheorem}{conjugate-subgroup-is-subgroup}{}
    Let \(H \subgroupleq G\), then \(g^{-1}Hg \subgroupleq G\).
\end{snippettheorem}

\begin{snippetproof}{conjugate-subgroup-is-subgroup-proof}{conjugate-subgroup-is-subgroup}{}
    Suppose \(a,b \in g^{-1}Hg\).
    We want to show \(ab^{-1} \in g^{-1}Hg\).\\
    Note that \(a = g^{-1}h_1g\) and \(b = g^{-1}h_2g\)
    for some \(h_1, h_2 \in H\). \\
    This means that \(ab^{-1}=a{(g^{-1}h_2g)}^{-1} = a(g^{-1}h_2^{-1}g)
    =g^{-1}h_1gg^{-1}h_2^{-1}g = g^{-1} (h_1h_2) g \in g^{-1}Hg \)
    because \(h_1h_2 \in H\).
\end{snippetproof}

\begin{snippetdefinition}{normal-subgroup-definition}{Normal subgroup}
    Let \(H \subgroupleq G\) be \group[groups]. Then, \(H\)
    is said to be \emph{normal}
    \[
        H \unlhd G
    \]
    if \(h^g \in H\) per all \(h\in H\) and \(g\in G\).
\end{snippetdefinition}

\begin{snippetproposition}{subgroups-of-center-are-normal}{}
    If \(H \subgroupleq \groupcenter(G)\), then \(H \unlhd G\).
    In particular, \(\groupcenter(G) \unlhd G\).
\end{snippetproposition}

\begin{snippetcorollary}{abelian-subgroups-are-normal}{}
    If \(G\) is an \abeliangroup, every \subgroup of \(G\)
    is normal in \(G\)
\end{snippetcorollary}

\plain{We are now going to characterize normal groups}

\begin{snippettheorem}{normal-subgroup-equivalences-theorem}{}
    Let \(H \subgroupleq G\) be \group[groups].
    The following are equivalent:
    \begin{enumerate}
        \item \(H \unlhd G\);
        \item \(H^g \subgroupleq H\) for all \(g\in G\);
        \item \(H^g = H\) for all \(g\in G\);
        \item \(Hg \subseteq gH\) for all \(g\in G\);
        \item \(Hg = hG\) for all \(g\in G\).
    \end{enumerate}
\end{snippettheorem}

\begin{snippetproof}{normal-subgroup-equivalences-theorem-proof}{normal-subgroup-equivalences-theorem}{}
    \begin{itemize}
        \item \((1) \implies (2)\): trivial;
        \item \((3) \implies (2)\): trivial;
        \item \((2) \implies (3)\): we need to show that \(H \subgroupleq H^g\)
        per every \(g\in G\). We know that \(H^g \subgroupleq H\) per every \(g \in G\).
        In particular, \(H^{g{-1}} \subgroupleq H\) for every \(g \in G\).
        But then
        \[
            {\left(H^{g^{-1}}\right)}^g \subgroupleq H^g
        \]
        for every \(g\in G\). However, clearly \({\left(H^{g^{-1}}\right)}^g = H\),
        from which \(H \subgroupleq H^g\) for every \(g\in G\);
        \item \((2) \implies (4)\): we know that \(H^g \subgroupleq H\) for every \(g\in G\),
        meaning \(g^{-1}Hg \subgroupleq H\), from which \(gg^{-1} H g \subseteq gH\),
        meaning \(Hg \subseteq gH\) for every \(g\in G\).
        \item \((4) \implies (5)\): we know that \(Hg\ subseteq gH\) for every \(g\in G\).
        We want to show that \(gH \subseteq Hg\) for every \(g\in G\).
        We need to show that given \(g\in G\) and \(h\in H\),
        we have \(gh \in Hg\), meaning \(gh = h'g\) for some \(h' \in H\).
        Now \begin{align*}
            gh = {({(gh)}^{-1})}^{-1} = {(h^{-1} g^{-1})}^{-1}
        \end{align*}
        However, \(h^{-1} g^{-1}\) and \(Hg^{-1} \subseteq g^{-1} H\),
        meaning \(h^{-1}g^{-1} = g^{-1} h''\) for some \(h'' \in H\).
        Thus, \[gh = {(h^{-1}g^{-1})}^{-1} = {(g^{-1}h'')}^{-1} = {(h'')}^{-1} g \in Hg\]
        \item \((4) \implies (2)\): if \(Hg \subseteq gH\) we have \(g^{-1}hg \in g^{-1} g H = H\),
        meaning \(H^g \subgroupleq H\) for very \(g \in G\).
    \end{itemize}
\end{snippetproof}

\begin{snippetcorollary}{kernel-group-homomorphism-normal-subgroup}{}
    Let \(\varphi \colon G \to H\) be a \grouphomomorphism.
    Then,
    \[
        \grpker_\varphi \unlhd G
    \]
\end{snippetcorollary}

\plain{Indeed, we proved that the right and left laterals of the kernel coincide.}

% Esercizio: mostrare che il nucleo di phi è normale in G senza usare i laterali, ma solo al def. di normalità

% i sottogruppi normali sono tutti e solo i sottogruppi di omomorfismi

\begin{snippetcorollary}{index-two-implies-normal-subgroups}{}
    Let \(H \subgroupleq G\) and \(|G \,:\, H| = 2\).
    Then, \(H \unlhd G\).
\end{snippetcorollary}

\begin{snippetproof}{index-two-implies-normal-subgroups-proof}{index-two-implies-normal-subgroups}{}
    \(H\) itself is one of the two right laterals of \(G\).
    The other right lateral is \(G \difference H\).
    The same goes for the left laterals.
\end{snippetproof}

\end{document}