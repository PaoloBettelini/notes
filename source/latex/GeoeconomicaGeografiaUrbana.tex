\documentclass[preview]{standalone}

\usepackage{amsmath}
\usepackage{amssymb}
\usepackage{stellar}
\usepackage{bettelini}

\hypersetup{
    colorlinks=true,
    linkcolor=black,
    urlcolor=blue,
    pdftitle={Stellar},
    pdfpagemode=FullScreen,
}

\begin{document}

\id{geoeconomica-geografia-urbana}
\genpage

\section{Geografia urbana}

\begin{snippetdefinition}{citta-definition}{Città}
    Non esiste una definizione universale di città.
    Nonostante la loro grande varietà, tutte le città condividono queste
    caratteristiche di base:
    \begin{enumerate}
        \item un'elevata densità di popolazione;
        \item una certa dimensione demografica che la distingue dagli
        insediamenti rurali;
        \item una complessità di funzioni culturali, sociali, economiche a cui
        corrispondono usi del suolo specializzati;
        \item l'essere centri dei poteri connessi all'esercizio di queste varie
        funzioni;
        \item l'essere ambienti dinamici e creativi;
        \item l'essere connesse ad altri luoghi urbani e rurali attraverso una
        fitta rete di relazioni e di flussi di persone, beni, servizi,
        informazioni e denaro;
        \item l'essere luoghi di grandi contraddizioni e di conflitti. Le città
        offrono infatti opportunità e speranze, ma sono nello stesso
        tempo luoghi di povertà, privazioni, disperazione e rivolte.
    \end{enumerate}
\end{snippetdefinition}

\section{Urbanesimo vs urbanizzazione}

\begin{snippet}{urbanesimo-urbanizzazione-expl}
    L'urbanesimo è legato allo spostamento delle persone,
    mentre l'urbanizzazione fa riferimento alla creazione delle varie infrastrutture.
    Per esempio, durante la Rivoluzione industriale, le persone lasciano le campagne
    perché meno persone sono necessarie in campagna.
    % le terre venogno vendute alle aziende (land grabbing)
    Da un periodo attorno al 2007, più della metà della popolazione vive in città.
\end{snippet}

\begin{snippetdefinition}{urbanizzazione-definition}{Urbanizzazione}
    L'\textit{urbanizzazione} è il processo di sviluppo e organizzazione
    che porta un centro abitato ad assumere le caratteristiche tipiche di una città.
    Il termine include sia la costruzione di strutture (opere di urbanizzazione),
    come reti di trasporti e sistema fognario, sia i cambiamenti di comportamento e
    costume della società. 
\end{snippetdefinition}

\begin{snippetdefinition}{urbanesimo-definition}{Urbanesimo}
    L'\textit{urbanesimo} o \textit{inurbamento} è quel processo che consiste nella migrazione
    di grandi masse di popolazioni dalle campagne alle città.
    Da un punto di vista sociale, essa è riconducibile all'assunzione di uno stile
    di vita urbano da parte di masse contadine (urbanizzazione). 
\end{snippetdefinition}

\plain{Solitamente, urbanesimo e urbanizzazione vanno di pari passi,
a differenza di una città che vine costruita prima di essere abitata.}

\section{Tipologie di zone}

\begin{snippetdefinition}{metropoli-definition}{Metropoli}
    Una \textit{metropoli} è una città che possiede più di un milione di abitanti.
\end{snippetdefinition}

\begin{snippetdefinition}{mega-citta-definition}{Mega città}
    Una \textit{mega città} è una città che possiede più di dieci milioni di abitanti.
\end{snippetdefinition}

\begin{snippetdefinition}{megalopoli-definition}{Megalopoli}
    Una \textit{megalopoli} è l'unione di più città dove il confine
    tra di esse è difficilmente tracciabile.
\end{snippetdefinition}

% Il ciclo di vita delle città non c'è da sapere

\begin{snippetnote}{06546afe-61de-49ff-9abf-d965424a9a8e}{Precisazioni}
    \begin{itemize}
        \item Metropolizzazione \(\neq\) Urbanizzazione;
        \item Metropoli / megalopoli / megacity \(\neq\) città globale / global
        city.
    \end{itemize}
\end{snippetnote}

\end{document}