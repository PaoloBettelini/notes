\documentclass[preview]{standalone}

\usepackage{amsmath}
\usepackage{amssymb}
\usepackage{stellar}

\hypersetup{
    colorlinks=true,
    linkcolor=black,
    urlcolor=blue,
    pdftitle={Stellar},
    pdfpagemode=FullScreen,
}

\begin{document}

\title{Stellar}
\id{italiano-decameron-nastagio-degli-onesti}
\genpage

\section{Analisi}

\begin{snippet}{nastaglio-degli-onesti-analisi}
    \textbf{Rubrica:} Nastagio degli Onesti, amando una de'Traversari,
    spende le sue ricchezze senza essere amato.
    Vassene, pregato da'suoi, a Chiassi; quivi vede cacciare ad un
    cavaliere una giovane e ucciderla e divorarla da due cani.
    Invita i parenti suoi e quella donna amata da lui ad un desinare,
    la quale vede questa medesima giovane sbranare; e temendo di simile avvenimento
    prende per marito Nastagio.
    
    % 1 riga di cornice
    
    % Personaggi 
    I personaggi vengono descritti poco. Lei è \underline{molto} nobile (§4), e siccome sa di essere tale e bella
    è molto cruda nei confronti di Nastagio (sdegnosa). Nastagio è nobile, giovane, molto ricco (dalla morte del padre),
    spende senza misura e si toglierebbe la vita per quanto la ami. Il suo vizio di spendere senza ottenre nulla lo sta rovinando.
    
    % I due racconti
    La storia di Nastagio potrebbe essere quella di Guido, ed è identica rispetto alla sua.
    Infatti, i loro nomi di famiglia coincidono. Entrambi sono gentiluomini di Ravenna
    che amavano non essendo essi amati.
    Le due donne non hanno nome, e il loro comportamento è simile (§6, §21).
    La storia di Guido è tuttavia andata più avanti dal momento che lui si è effettivamente ucciso (§21), 
    mentre l'altro ne ha solo avuto la tentazione (§6).
    Inoltre, Nastagio si sforza di odiarla (§7) e Guido la odia e la uccide (§26).
    Guido è quindi l'evoluzione possiible della storia di Nastagio.
    La donna di Nastagia non lo odia, bensì ne è solamente indifferente (§6), mentre l'altra
    non solo è indifferente, ma gode anche della sua morte (§22).
    
    Nastagio vede la proiezione della sua storia in quella di Guido, e cambia il corso degli eventi per far sì che
    la sua prenda un cammino diverso.
    Dopo la sua visione, sfrutta la situazione di caccia che gli si presenta (§32) mediantre la propria furbizia.
    
    Rovesciamento di quello che la religione indica come un obbliga, che qui viene espresso come un vizio: \\
    La novella di Nastagio è una parodia di exemplum (storia di modello morale. E.g. fiaba, con fine didascalico).
    Agire contro le forze di natura è sempre sbagliato. In questo caso, l'attrazione, è giusto che trovi sfogo.
    Questa è una nuova morale insegnata da Boccaccio.
    Il cambiamento inverosimile è che tutte le donne di Ravenna si concedono agli uomini (§44)
    e X in una sera tramuta l'odio in amore.
    È Dio che vuole che la crudeltà sia da condannare e che la pietà sia da lodare.
    Tuttavia, viene successivamente indicata la crudeltà come il fatto di resistere alle forze della natura, per cui
    vi è un rovescimaneto di ciò che insegna la religione.
    Al contrario di Dante, finisce all'inferno chi resiste all'amore.
    
    % La parodia
    % L'industria di Nestagio
    % L'aldilà
    % Le fonti della novella

    Nell'(anti-)exemplum di Passavanti la morale viene esplicitata alla fine: resistere è un valore,
    mentre cedere è un disvalore.

    % FONTI. Chiedi a qualcuno perché stavo studiando bio
\end{snippet}

\begin{snippetnote}{rovesciamento-dante-boccaccio}{Rovesciamento Dante \(\rightarrow\) Boccaccio}
    Dante scrive 100 canti nell'oltretomba e ne ambienta 2 sulla terra, mentre Boccaccio scrive 100 novelle di cui solamente
    due sono ambientate nell'oltretomba (tra l'altro, Inferno sulla terra).
    Il centro della vita non è più la preparazione all'oltretomba, bensì la vita stessa.
\end{snippetnote}

\end{document}