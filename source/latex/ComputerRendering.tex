\documentclass[preview]{standalone}

\usepackage{amsmath}
\usepackage{amssymb}
\usepackage{parskip}
\usepackage{fullpage}
\usepackage{hyperref}
\usepackage{tikz}
\usepackage{stellar}

\usetikzlibrary{angles,quotes}

\hypersetup{
    colorlinks=true,
    linkcolor=black,
    urlcolor=blue,
    pdftitle={Computer Rendering},
    pdfpagemode=FullScreen,
}

\begin{document}

\title{Computer Rendering}
\id{computerrendering}
\genpage

\section{Rendering}

\begin{snippet}{gpu-usage}
The GPU is able to perform computation on a lot of data simultaneously
(SIMD, Single Instruction stream, Multiple Data stream).
\end{snippet}

\includesnpt{teapot}

\begin{snippetdefinition}{index-buffer-definition}{Index Buffer}
    An \textit{index buffer} is a data structure used to store and organize indices of vertices.
    Objects are often represented by a collection of vertices and the connections between them (edges).
    These connections are typically defined by specifying the indices of the vertices that
    form each polygon (such as triangles) in the mesh.

    An index buffer contains an array of indices that reference the vertices in the vertex buffer. Instead of specifying the complete set of vertex data for each polygon, you can reference the vertices using their indices. This can result in more efficient memory usage and bandwidth, especially when rendering complex 3D models with many shared vertices.
\end{snippetdefinition}

\end{document}
