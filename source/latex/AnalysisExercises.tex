\documentclass[preview]{standalone}

\usepackage{amsmath}
\usepackage{amssymb}
\usepackage{stellar}
\usepackage{definitions}
\usepackage{bettelini}

\begin{document}

\id{analysis-exercises}
\genpage

\section{Exercises}

\begin{snippetexercise}{analysis-misc-ex1}{}
    Find the value of
    \[
        \lim_{x\to0} \frac{1}{\sqrt{\ln(x)}} \integral[0][x][\frac{e^{-1/t}\arctan(t^\alpha)}{t^{1+\alpha}{\left[\ln(1+t)\right]}^{2+\alpha}}][t]
    \]
    for \(\alpha\in\realnumbers\).
\end{snippetexercise}

\begin{snippetproof}{analysis-misc-ex1-proof}{analysis-misc-ex1}{}
    We first study the integrand \(f_\alpha(t)\).
    In the interval \((0, +\infty)\), \(f_\alpha(t)\) is continuous and positive, meaning that the integral
    exists.
    \begin{itemize}
        \item for \(t\to0^+\), \[
            f_\alpha(t)
            \asymptotic \frac{t^\alpha e^{-1/t}}{t^{1+\alpha}\cdot t^{2 + \alpha}}
        \]
        meaning that for all \(\alpha\in\realnumbers\), \(f_\alpha(t)\) is integrable in \((0, \varepsilon)\)
        for some \(\varepsilon > 0\).
        \item for \(t\to+\infty\), \[
            f_\alpha(t) \asymptotic \begin{cases}
                \frac{\pi/2}{t^{1+\alpha}{(\ln t)}^{2+\alpha}} & \alpha>0 \\
                \frac{\pi/4}{t{(\ln t)}^2} & \alpha=0 \\
                \frac{t^\alpha}{t^{1+\alpha}{(\ln t)}^{2+\alpha}} & \alpha<0
            \end{cases}
        \]
        We will determine convergence using \(p-q\)-integrals.
        In the first case, we need \(\alpha\geq 0\) for the integral to converge, so it always converges.
        If \(\alpha = 0\), we have the \(p=1\) and \(q=2\), so it does converge.
        If \(\alpha<0\), we need \(\alpha>-1\), meaning the integral converges for \(-1<\alpha<0\).
        In conclusion, the integral converges for \(\alpha>-1\).
    \end{itemize}
    If \(\alpha>-1\), the limit becomes
    \[
        \lim_{x\to+\infty} \frac{I}{\sqrt{\ln x}} = 0
    \]
    where \(I\) is the value of the integral, which is finite.
    Otherwise, let \(a\leq -1\). We have
    \begin{align*}
        \lim_{x\to+\infty} \frac{\integral[0][x][f_\alpha(t)][t]}{\sqrt{\ln x}}
        &\overset{H}{=} \lim_{x\to+\infty} \frac{f_\alpha(x)}{\frac{1}{2} \frac{1}{\sqrt{\ln x}} \cdot \frac{1}{x}} \\
        &= \lim_{x\to+\infty} \frac{e^{-1/x}\arctan(x^\alpha)}{x^{1+\alpha} {[\ln(1+x)]}^{2+\alpha}} 2x{(\ln x)}^{1/2} \\
        &= \lim_{x\to+\infty} \frac{2}{{(\ln x)}^{\alpha + 3/2}}
        = \begin{cases}
            0 & -\frac{3}{2} < \alpha \leq -1 \\
            2 & \alpha = -\frac{3}{2} \\
            + \infty & \alpha<-\frac{3}{2}
        \end{cases}
    \end{align*}
\end{snippetproof}

\begin{snippetexercise}{analysis-misc-ex2}{}
    Find the value of
    \[
        \lim_{x\to0} \frac{1}{\ln(x)} \integral[1][x][\frac{\arctan(t)}{t^\alpha{(1+t)}^{\alpha / 2}}][t]
    \]
    for \(\alpha\in\realnumbers\).
\end{snippetexercise}

\begin{snippetproof}{analysis-misc-ex2-proof}{analysis-misc-ex2}{}
    We first study the integrand \(f_\alpha(t)\). We note that it is always non-negative and continuous.
    \begin{itemize}
        \item for \(t\to0^+\),
            \[
                f_\alpha(t) \asymptotic
                \frac{t^\alpha}{t^\alpha \cdot 1} = t^{1-\alpha}
            \]
            meaning that \(f_\alpha(t)\) converges \ifandonlyif \(1-\alpha>-1 \iff \alpha<2\).
        \item for \(t\to+\infty\),
            \[
                f_\alpha(t) \asymptotic t^{-3\alpha/2}
            \]
            meaning that \(f_\alpha(t)\) converges \ifandonlyif \(\alpha > \frac{2}{3}\).
    \end{itemize}
    Thus, the integral converges for \(\frac{2}{3} < \alpha < 2\). In such case,
    the limit becomes
    \[
        \lim_{x\to+\infty} \frac{I}{\ln x} = 0
    \]
    where \(I\) is the value of the integral, which is finite.
    Otherwise, we have
    \begin{align*}
        \lim_{x\to+\infty} \frac{\integral[0][x][f_\alpha(t)][t]}{\ln x}
        &\overset{H}{=} \lim_{x\to+\infty} \frac{f_\alpha(x)}{1/x} \\
        &= \lim_{x\to+\infty} \frac{\pi}{2} t^{1-\frac{3}{2}\alpha}
        = \begin{cases}
            0 & \alpha > \frac{2}{3} \\
            \frac{\pi}{2} & \alpha = \frac{2}{3} \\
            +\infty & \alpha < \frac{2}{3}
        \end{cases}
    \end{align*}
\end{snippetproof}

\end{document}