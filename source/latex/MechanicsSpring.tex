\documentclass[preview]{standalone}

\usepackage{amsmath}
\usepackage{amssymb}
\usepackage{stellar}
\usepackage{definitions}

\begin{document}

\id{mechanics-spring}
\genpage

\section{Elastic force}

\begin{snippettheorem}{hookes-law-theorem}{Hook's law}
    The elastic force of a spring is given by
    \[
        \vec{F_e} = -k \vec{x}
    \]
    where \(\vec{x}\) is the stretching with respect to the equilibrium position
    and \(k\) is the \emph{elastic constant}.
\end{snippettheorem}

\plain{This is actually an approximation, and assumes relatively small stretching.}

\section{Simple harmonic motion}

\begin{snippetdefinition}{simple-harmonic-motion-definition}{Simple harmonic motion}
    The motion of a particle moving along a straight line with an acceleration whose direction is always towards a fixed point on the line and whose magnitude is proportional to the displacement from the fixed point is called simple harmonic motion.
\end{snippetdefinition}

\begin{snippettheorem}{spring-mass-motion-theorem}{Spring-mass motion}
    Given a mass \(m\) attached to a spring of constant \(k\),
    the oscillating motion is a simple harmonic motion and the position is given by
    \[
        A\cos(\omega t + \varphi), \quad \omega = \sqrt{\frac{k}{m}}
    \]
    where \(\omega\) is the angular frequency, \(A\) the amplitude
    and \(\varphi\) the phase shift. The frequency of the motion is given by
    \[
        \frac{\omega}{2\pi}
    \]
\end{snippettheorem}

\begin{snippetproof}{spring-mass-motion-theorem-proof}{spring-mass-motion-theorem}{Spring-mass motion}
    We combine Newton's law \(F = ma\) in a direction and \(F = -kx\).
    We thus get the differential equation
    \[
        \frac{d^2 x}{dt} + \frac{k}{m}x = 0 
    \]
    which has solution
    \[
        x(t) = A \cos(\omega t + \varphi)
    \]
\end{snippetproof}

\end{document}