\documentclass[preview]{standalone}

\usepackage{amsmath}
\usepackage{amssymb}
\usepackage{stellar}
\usepackage{bettelini}

\hypersetup{
    colorlinks=true,
    linkcolor=black,
    urlcolor=blue,
    pdftitle={Stellar},
    pdfpagemode=FullScreen,
}

\begin{document}

\title{Geografia economica}
\id{geoeconomica-fordismo-postfordismo}
\genpage

\section{Fordismo e postfordismo}

\begin{snippetdefinition}{fordismo-definition}{Fordismo}
    Il \textit{Fordismo} è un modello economico e produttivo sviluppato da Henry Ford,
    fondatore della Ford Motor Company, che ha rivoluzionato il settore automobilistico
    e ha avuto un impatto significativo sull'industria globale nel XX secolo.
\end{snippetdefinition}

\begin{snippet}{fordismo-caratteristiche}
    Il fordismo è caratterizzato dalle seguenti proprietà:
    \begin{itemize}
        \item catena di montaggio per la produzione in serie;
        \item specializzazione del lavoro;
        \item salario elevato e standardizzazione dei prodotti, riducendone i costi e permettendo ai dipendenti di acquistare i prodotti che producevano;
        \item Produce-to-Stock - si basava sulla produzione di beni in grandi quantità prima che ci fosse una domanda effettiva.
    \end{itemize}
\end{snippet}

\begin{snippetdefinition}{postfordismo-definition}{Postfordismo}
    Con \textit{Postfordismo} si intende il periodo di tempo
    successivo al declino del Fordismo, dove esso viene rivoluzionato per avere una maggiore efficienza
    sulla società ormai cambiata.
\end{snippetdefinition}

\begin{snippet}{postfordismo-caratteristiche}
    Le due caratteristiche sono:
    \begin{itemize}
        \item flessibilità produttiva per soddisfare le esigenze velocemente;
        \item automazione tecnologica;
        \item decentramento della produzione e delocalizzazione delle fabbrice;
    \end{itemize}
\end{snippet}

\begin{snippet}{07acb3c8-5fed-45f2-9726-573143653e2e}
    Il Post-Fordismo ha portato a una nuova organizzazione economica e produttiva, evidenziando la necessità di adattarsi rapidamente ai cambiamenti del mercato e incorporare innovazione e conoscenza come elementi centrali dello sviluppo economico.

    \begin{figure}[h]
        \centering
        \begin{tabular}{|l|l|}
            \hline 
            \textbf{Fordismo} & \textbf{Post-fordismo} \\
            \hline 
            \begin{tabular}{@{}l@{}}
                - Modello di organizzazione industriale \\ 
                \; tipico nel Novecento
            \end{tabular} & 
                - Modello organizzativo post-industriale \\
            \hline 
            \begin{tabular}{@{}l@{}}
                - Predominantemente nel Settentrione \\
                \; europeo e in America nel dopoguerra
            \end{tabular} & 
            \begin{tabular}{@{}l@{}}
                - Emergenza a partire dagli anni '70 e ' 80 \\
                \; del Novecento
            \end{tabular} \\
            \hline 
            \begin{tabular}{@{}l@{}}
                - Produzione in serie di grandi \\
                \; stabilimenti
            \end{tabular} & 
                - Flessibilità produttiva \\
            \hline 
            \begin{tabular}{@{}l@{}}
                - Produzione basata sulla \\
                \; standardizzazione e sull'efficienza
            \end{tabular} & 
            \begin{tabular}{@{}l@{}}
                - Produzione differenziata, mirata alla \\
                \; personalizzazione
            \end{tabular} \\
            \hline 
            \begin{tabular}{@{}l@{}}
                - Economia basata sull'espansione di \\
                \; mercati potenzialmente infiniti
            \end{tabular} & 
            \begin{tabular}{@{}l@{}}
                - Produzione di piccoli lotti, spesso in \\
                \; base alla domanda
            \end{tabular} \\
            \hline 
            \begin{tabular}{@{}l@{}}
                - Uso intensivo della forza lavoro \\
                \; umana
            \end{tabular} & 
            \begin{tabular}{@{}l@{}}
                - Maggiore uso di tecnologie e \\
                \; automatismi
            \end{tabular} \\
            \hline 
            \begin{tabular}{@{}l@{}}
                - Stabilimenti come quelli della Ford \\
                \; negli Stati Uniti
            \end{tabular} & 
            \begin{tabular}{@{}l@{}}
                - Decentrare la produzione spostandola \\
                \; vicino ai mercati target
            \end{tabular} \\
            \hline 
            \begin{tabular}{@{}l@{}}
                - Innovazione focalizzata su processi e \\
                \; tecnologia che ottimizzano la \\
                \; produzione in serie \\
                - Vertice del modello con le grandi
            \end{tabular} & 
            \begin{tabular}{@{}l@{}}
                - Innovazione orientata alla \\
                \; diversificazione e all'adattamento rapido \\
                - Emergenza di nuovi poli produttivi \\
                \; flessibili e decentrati
            \end{tabular} \\
            \hline 
            & 
            \begin{tabular}{@{}l@{}}
                - Maggiore interconnessione con \\
                \; l'economia globale e con le reti di \\
                \; informazione \\
                - Importanza crescente delle società \\
                \; transnazionali e delle multinazionali
            \end{tabular} \\
            \hline
            \begin{tabular}{@{}l@{}}
                - Catena di montaggio (prod. in serie)
            \end{tabular} & 
                - Flessibilità, just-in-time \\
            \hline 
            \begin{tabular}{@{}l@{}}
                - Domanda in funzione all'offerta
            \end{tabular} & 
                - Offerta in funzione della domanda \\
            \hline 
            \begin{tabular}{@{}l@{}}
                - Territorializzazione
            \end{tabular} & 
                - Delocalizzazione \\
            \hline 
            \begin{tabular}{@{}l@{}}
                - Mercati specifici ``nazionali'', seppur \\
                \; comunicanti
            \end{tabular} & 
            \begin{tabular}{@{}l@{}}
                - Mercati ``transnazionali'' \\
                \; (Stati come ostacolo)
            \end{tabular} \\
            \hline
        \end{tabular}
    \end{figure}
    \vspace{0.25cm}
\end{snippet}

\end{document}