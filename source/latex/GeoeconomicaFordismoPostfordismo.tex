\documentclass[preview]{standalone}

\usepackage{amsmath}
\usepackage{amssymb}
\usepackage{stellar}
\usepackage{bettelini}

\hypersetup{
    colorlinks=true,
    linkcolor=black,
    urlcolor=blue,
    pdftitle={Stellar},
    pdfpagemode=FullScreen,
}

\begin{document}

\title{Geografia economica}
\id{geoeconomica-fordismo-postfordismo}
\genpage

\begin{snippetdefinition}{fordismo-definizione}{Fordismo}
    Il \textit{Fordismo} è un modello economico e produttivo sviluppato da Henry Ford,
    fondatore della Ford Motor Company, che ha rivoluzionato il settore automobilistico
    e ha avuto un impatto significativo sull'industria globale nel XX secolo.
\end{snippetdefinition}

\begin{snippet}{fordismo-caratteristiche}
    Il fordismo è caratterizzato dalle seguenti proprietà:
    \begin{itemize}
        \item catena di montaggio per la produzione in serie;
        \item specializzazione del lavoro;
        \item salario elevato e standardizzazione dei prodotti, riducendone i costi e permettendo ai dipendenti di acquistare i prodotti che producevano;
        \item Produce-to-Stock - si basava sulla produzione di beni in grandi quantità prima che ci fosse una domanda effettiva.
    \end{itemize}
\end{snippet}

\begin{snippetdefinition}{postfordismo-definizione}{postfordismo}
    Con \textit{Postfordismo} si intende il periodo di tempo
    successivo al declino del Fordismo, dove esso viene rivoluzionato per avere una maggiore efficienza
    sulla società ormai cambiata.
\end{snippetdefinition}

\begin{snippet}{postfordismo-caratteristiche}
    Le due caratteristiche sono:
    \begin{itemize}
        \item flessibilità produttiva per soddisfare le esigenze velocemente;
        \item automazione tecnologica;
        \item decentramento della produzione e delocalizzazione delle fabbrice;
    \end{itemize}
\end{snippet}

\begin{snippet}{07acb3c8-5fed-45f2-9726-573143653e2e}
    Il Post-Fordismo ha portato a una nuova organizzazione economica e produttiva, evidenziando la necessità di adattarsi rapidamente ai cambiamenti del mercato e incorporare innovazione e conoscenza come elementi centrali dello sviluppo economico.

    \begin{table}[h]
        \centering
        \begin{tabular}{|c|c|}
        \hline
        \textbf{Fordismo} & \textbf{Post-Fordismo} \\ \hline
        Molta forza lavoro umana & Uso delle tecnologie \\ \hline
        Domanda = f(offerta), potenzialmente illimitata & Offerta o prod = f(domanda) \\ \hline
        Territorializzazione & Delocalizzazione \\\hline
        Catena di montaggio in serie & Flessibilità, just-in-time (catena di montaggio concettuale) \\\hline
        Mercati specifici, nazionali, seppur comunicanti & Mercati "transnazionali", Stati come ostacolo \\ \hline
        \end{tabular}
    \end{table}
\end{snippet}

\end{document}