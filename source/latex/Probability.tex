\documentclass[preview]{standalone}

\usepackage{amsmath}
\usepackage{amssymb}
\usepackage{stellar}
\usepackage{definitions}
\usepackage{bettelini}

\begin{document}

\id{probability}
\genpage

% 
% 1) https://courses.maths.ox.ac.uk/pluginfile.php/11726/mod_resource/content/4/ProbabilityNotes2021-10-11.pdf
% 2) https://courses.maths.ox.ac.uk/pluginfile.php/7683/mod_resource/content/5/A8LectureNotes_MT21.pdf

\section{Definition}

\begin{snippetdefinition}{probability-space-definition}{Probability Space}
    A \textit{probability space} is a tuple \((\Omega, \mathcal{F}, \mathbb{P})\)
    where
    \begin{itemize}
        \item \(\Omega\) is the set of all possible outcomes, called \textit{sample space};
        \item \(\mathcal{F}\) is a set of subsets of \(\Omega\), called \textit{event};
        \item \(\mathbb{P}\) is a function \(\mathbb{P}\colon \mathcal{F} \to [0;1]\), called a \textit{probability measure}. 
    \end{itemize}
    The set \(\mathcal{F}\) satisfies the following axioms:
    \begin{enumerate}
        \item it contains the sample space: \(\Omega \in \mathcal{F}\);
        \item it is closed under complements: \(\forall A \in \mathcal{F}, \Omega \setminus A \in \mathcal{F}\);
        \item it is closed under countable unions: \({\{A_i\}}_{i \in I} \subseteq \mathcal{F} \implies \bigcup_{i \in I}A_i \in \mathcal{F}\).
    \end{enumerate}
    The probability measure \(\mathbb{P}\) follows the probability axioms:
    \begin{enumerate}
        \item \(\mathbb{P}(\Omega) = 1\);
        \item it is countably additive: if \({\{A_i\}}_{i \in I} \subseteq \mathcal{F}\)
        is a countable collection of pairwise disjoint sets, then
        \[
            \mathbb{P}\left(\bigcup_{i \in I}A_i\right) = \sum_{i \in I} \mathbb{P}(A_i)
        \]
    \end{enumerate}
\end{snippetdefinition}

\begin{snippettheorem}{measure-of-inverse-theorem}{Measure of inverse}
    Let \((\Omega, \mathcal{F}, \mathbb{P})\) be a probability space and \(A \in \mathcal{F}\).
    \[
        \mathbb{P}(\Omega \setminus A) = 1 - \mathbb{P}(A)
    \]
\end{snippettheorem}

\begin{snippetproof}{measure-of-inverse-proof}{measure-of-inverse}{Measure of inverse}
    Since \(\mathbb{P}(\Omega) = \mathbb{P}(A \union (\Omega \setminus A)) = 1\),
    we have \(\mathbb{P}(A) + \mathbb{P}(\Omega \setminus A) = 1\)
    and thus the theorem.
\end{snippetproof}

\begin{snippettheorem}{measure-of-subset-theorem}{Measure of subset}
    Let \((\Omega, \mathcal{F}, \mathbb{P})\) be a probability space and \(A, B \in \mathcal{F}\).
    \[
        A \subseteq B \implies \mathbb{P}(A) \leq \mathbb{P}(B)
    \]
\end{snippettheorem}

\begin{snippetproof}{measure-of-subset-proof}{measure-of-subset}{Measure of subset}
    Assume \(A \subseteq B\). Since \(\mathbb{P}(B) = \mathbb{P}(A) + \mathbb{P}(B \intersection (\Omega \setminus A))\)
    and \(\mathbb{P}(B \intersection (\Omega \setminus A)) \geq 0\), we have \(\mathbb{P}(B) \geq \mathbb{P}(A)\).
\end{snippetproof}

\section{Conditional probability}

\begin{snippetdefinition}{conditional-probability-definition}{Conditional Probability}
    Let \((\Omega, \mathcal{F}, \mathbb{P})\) be a probability space
    and \(A, B \in \mathcal{F}\).
    If \(\mathbb{P}(B) > 0\), then the \textit{conditional probability} of \(A\) given \(B\)
    is given by
    \[
        \mathbb{P}(A|B) = \frac{\mathbb{P}(A \intersection B)}{\mathbb{P}(B)}
    \]
\end{snippetdefinition}

\begin{snippetdefinition}{sample-space-partition-definition}{Sample Space Partition}
    Let \(\Omega\) be a sample space.
    A family of events \(\{B_1,B_2, \cdots\}\)
    is a \textit{partition} of \(\Omega\) if
    \begin{enumerate}
        \item at least one event happens: \(\Omega = \bigcup_{i \geq 1} B_i\);
        \item no pair can happen together: \(i \neq j \implies B_i \intersection B_j = \emptyset\).
    \end{enumerate}
\end{snippetdefinition}

\begin{snippettheorem}{total-probability-law}{Law of Total Probability}
    Let \((\Omega, \mathcal{F}, \mathbb{P})\) be a probability space.
    Suppose \(\{B_1,B_2, \cdots\}\) is a partition of \(\Omega\) by sets from \(\mathcal{F}\)
    such that \(\mathbb{P}(B_i) > 0\) for \(i \geq 1\).
    Then, for any \(A \in \mathcal{F}\)
    \[
        \mathbb{P}(A) = \sum_{i \geq 1} \mathbb{P}(A|B_i) \mathbb{P}(B_i)
    \]
\end{snippettheorem}

% proof

\begin{snippettheorem}{bayes-theorem}{Bayes' Theorem}
    Let \((\Omega, \mathcal{F}, \mathbb{P})\) be a probability space.
    Suppose \(\{B_1,B_2, \cdots\}\) is a partition of \(\Omega\) by sets from \(\mathcal{F}\)
    such that \(\mathbb{P}(B_i) > 0\) for \(i \geq 1\).
    Then, for any \(A \in \mathcal{F}\) such that \(\mathbb{P}(A) > 0\)
    \[
        \mathbb{P}(B_k|A) = \frac{\mathbb{P}(A|B_k)\mathbb{P}(B_k)}{\sum_{i \geq 1}\mathbb{P}(A|B_i)\mathbb{P}(B_i)}
    \]
\end{snippettheorem}

% proof

\end{document}