\documentclass[preview]{standalone}

\usepackage{amsmath}
\usepackage{amssymb}
\usepackage{stellar}

\hypersetup{
    colorlinks=true,
    linkcolor=black,
    urlcolor=blue,
    pdftitle={Stellar},
    pdfpagemode=FullScreen,
}

\begin{document}

\title{Stellar}
\id{italiano-principe-capitolo-vii}
\genpage

\section{Capitolo VII}

\begin{snippet}{il-principe-capitolo-vii-parte1}
    Questo capitolo tratta l'opposto di quello precedente, ossia quando si giunge al potere
    con la fortuna.
    È possibile mantenere il potere una volta conquistato senza sforzo, si può
    anche avere successo a condizione che la virtù la si dimostri dopo.
    \\\\
    L'esempio di Cesare Borgia è uno di quelli di chi ha fallito.
    Tuttavia, Machiavelli lo propone comunque come modello poiché
    il motivo per il quale Cesare Borgia ha fallito è dato da una sfortuna immensa,
    una sfortuna quasi impossiible da riavere.
\end{snippet}

\begin{snippetdefinition}{cesare-borgia-definition}{Cesare Borgia}
    \textit{Cesare Borgia} è stato un generale, cardinale, nobile e politico italiano.
\end{snippetdefinition}

\begin{snippetnote}{7bdd428d-eaa8-4e43-8386-c549b379b7b8}{}% grazie al papa si ritrova la romagna
    Guidò l'esercito francese alla conquista del Ducato di Milano e,
    con l'appoggio del papa, incominciò la riconquista dei territori della Romagna,
    battendo i vari signorotti locali, fra cui la celebre Caterina Sforza,
    ricevendo in seguito dal padre il titolo di Duca di Romagna.
    Successivamente invase il Regno di Napoli guidando le truppe francesi.
    Nel 1502, raggiunto rapidamente un grande potere politico,
    riuscì a difendersi dalla congiura della Magione,
    traendo in inganno i traditori e facendoli strangolare a Senigallia.
    Questa vendetta colpì molto l'opinione pubblica.
\end{snippetnote}

\begin{snippet}{il-principe-capitolo-vii-parte2}
    Machiavelli si prepara alla morte futura del padre, per cui a quando
    non avrà più il sostegno del padre e del papa.
    La sfortuna risiede nel fatto che in concomitanza con la morte del padre,
    Cesare Borgia si ammala e muore per la debolezza.
\end{snippet}

\end{document}