\documentclass[preview]{standalone}

\usepackage{amsmath}
\usepackage{amssymb}
\usepackage{stellar}
\usepackage{definitions}

\begin{document}

\id{italiano-machiavelli-biografia}
\genpage

\section{Biografia}

\begin{snippet}{machiavelli-biografia}
    Niccolò Machiavelli è stato un influente pensatore, diplomatico e scrittore fiorentino nato nel 1469 e deceduto nel 1527. È noto soprattutto per il suo capolavoro \textit{Il Principe}, un trattato politico che esplora il potere, la politica e la leadership.
    Nato a Firenze in una famiglia di modeste origini, Machiavelli ricevette un'educazione umanistica e divenne coinvolto nella politica fiorentina. La sua carriera diplomatica lo portò a interagire con figure di spicco del suo tempo, inclusi i Medici e alcuni degli esponenti più importanti della politica europea del Rinascimento.
    Dopo la caduta dei Medici e l'instaurazione della Repubblica a Firenze, Machiavelli subì l'esilio e si ritirò a vita privata, dedicandosi alla scrittura. Ha continuato a scrivere e a studiare politica fino alla sua morte nel 1527.
    La sua influenza si è estesa attraverso i secoli, e il suo nome è diventato sinonimo di astuzia politica e realismo nel campo della politica. La sua opera continua a essere studiata e discussa in ambito accademico e politico, lasciando un'impronta duratura nella storia del pensiero politico occidentale.
\end{snippet}

\end{document}