\documentclass[preview]{standalone}

\usepackage{amsmath}
\usepackage{amssymb}
\usepackage{stellar}

\hypersetup{
    colorlinks=true,
    linkcolor=black,
    urlcolor=blue,
    pdftitle={Stellar},
    pdfpagemode=FullScreen,
}

\begin{document}

\title{Stellar}
\id{italiano-leopardi-islandese}
\genpage

\section{Dialogo della Natura di un Islandese}

\includesnpt[text=Testo:|display=Dialogo della natura di un islandese|href=https://laspada.altervista.org/wp-content/uploads/2018/05/Dialogo-della-Natura.pdf]{file-url}

\begin{snippet}{leopardi-dialogo-natura-islandese-parte1}
    L'islandese sta fuggendo dalla natura ma la trova in persona.
    Secondo l'islandese, vivere una vita felice è utopico.
    Se non può essere felice, potrà per lo meno non soffrire,
    ossia vivere in maniera tranquilla (non ozio), ma vivere quietamente.
    Tuttavia, vivendo in un mondo di uomini, non si può evitare 
    che gli altri ti offendano.
    Quindi l'islandese si isola.
    Forse, per vivere senza soffrire, basta vivere isolato.
    \\\\
    La natura è crudele, ma forse non è lei ad essere crudele.
    Forse, sei tu natura che ha stabilito il lugo dell'uomo,
    è l'uomo che si è spostato. L'islandese si prende questa responsabilità perché
    riconosce di essere ambiozioso. Quindi, cerca la zone che la natura gli avrebbe posto.
    Non è un attacco frontale alla natura, è solamente una considerazione
    che lo porta al pellegrinaggio.
    \\\\
    La natura è indifferente e non si bada della felicità o infelicità
    degli umani, non se ne accorge nemmeno del suo effetto sugli uomini.
    Se l'umano si estinguesse, la natura non se ne accorgerebbe nemmeno.
    Questa è una critica all'antropocentrismo.
\end{snippet}

\begin{snippetdefinition}{antropocentrismo-definition}{Antropocentrismo}
    L'\textit{antropocentrismo} è la tendenza -
    che può essere propria di una teoria, di una religione o di una semplice opinione -
    a considerare l'essere umano, e tutto ciò che gli è proprio,
    come centrale nell'Universo.
    Una centralità che può essere intesa secondo diversi accenti e sfumature:
    semplice superiorità rispetto al resto del mondo animale o preminenza ontologica su
    tutta la realtà, in quanto si intende l'uomo come espressione immanente
    dello spirito che è alla base dell'Universo. 
\end{snippetdefinition}

\begin{snippet}{leopardi-dialogo-natura-islandese-parte2}
    Come risposta a queste critiche, la natura indica come l'universo sia sostanzialmente
    una ciclo di creazione e distruzione, e per cui la sofferenza dell'islandese
    è necessaria. Questa risposta non è sufficiente; l'islandese capisce questo concetto,
    capisce come funziona l'universo ma non ne comprende il senso.
    La domanda circa il senso rimane in sospeso, e la natura non risponde.
    \\\\
    Due leoni affamati mangiano l'islandese.
    Alcuni negano tuttavia la storia del leone ma afferiscono che il vento abbia steso l'islandese,
    e un velo di sabbia l'abbia ricoperto. Alcuni viaggiatori lo ritrovarono in futuro, ben
    preservato, e lo portano in un museum in europa.
    In entrambi i casi, l'identità dell'islandese diventa dimenticata.
\end{snippet}

\end{document}