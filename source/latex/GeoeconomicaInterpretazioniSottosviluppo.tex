\documentclass[preview]{standalone}

\usepackage{amsmath}
\usepackage{amssymb}
\usepackage{stellar}
\usepackage{bettelini}

\hypersetup{
    colorlinks=true,
    linkcolor=black,
    urlcolor=blue,
    pdftitle={Stellar},
    pdfpagemode=FullScreen,
}

\begin{document}

\title{Geografia economica}
\id{geoeconomica-interpretazioni-sottosviluppo}
\genpage

% TODO https://moodle.edu.ti.ch/libe/pluginfile.php/119046/mod_resource/content/1/Pass_04a_Sviluppo-A-Teorie.pdf
% Lo sviluppo di sto Rostow

\begin{snippet}{ragioni-sottosviluppo}
    Ragioni del sottosviluppo:
    \begin{itemize}
        \item Una teoria del secolo scorso lega ragioni culturali al sottosviluppo (inferiorità culturale),
            ossia la cultura non permette lo sviluppo;
        \item (geo)determinismo: clima caldo, suolo arido, morfologia difficile,
            scarse risorse minerarie ed energtiche sarebbero i fattori che, secondo questa teoria, causano il sottosviluppo;
        \item Rostow: mancanza di conoscenze, infrastrutture (per produrre);
        \item Dipendenza: l'esistenza di paesi sviluppati è la causa del blocco dei paesi sottosviluppati;
        \item Integrazione: scarsa integrazione nel sistema (per raggiungere lo sviluppo per tutti basterebbe integrare tutti nel sistema).
    \end{itemize}
\end{snippet}

\begin{snippet}{e04a8798-7f9e-4e8d-b5ee-93a2dc9fe8d1}
    Secondo Truman (scorso inaugurale del Presidente degli Stati Uniti d'America Harry Truman),
    la causa del sottosviluppo è la mancanza di conoscenze tecniche dei popoli in questione.
    
    Queste ragioni per il sottosviluppo sono chiaramente poco precise: ci sono
    evidentemente delle situazioni dove lo sviluppo non è stato fermato dalla geografia difficile.
\end{snippet}

% testo truman 1949

% https://moodle.edu.ti.ch/libe/pluginfile.php/119047/mod_resource/content/1/ElementidiGeografia_V11.1%20156-158_2pp.pdf

\begin{snippetdefinition}{visione-ambientalista-definizione}{Visione ambientalista}
    La \textit{visione ambientalista} o \textit{sviluppo sostenibile}
    parte dall'idea che la biosfera è un sistema
    dotato di risorse limitate e che occorra porre dei limiti allo sviluppo in
    funzione della disponibilità delle risorse. Quelle scelte in cui la differenza tra
    i tempi biologici e i tempi di produzione è tanto grande da non permettere
    la rinnovabilità delle risorse e la compatibilità con i ritmi naturali, non
    possono essere considerate sostenibili. Le relazioni tra attività umane e
    biosfera devono allora essere tali da permettere di soddisfare i bisogni e lo
    sviluppo delle culture e, nel contempo, non compromettere il contesto
    biofisico globale.
\end{snippetdefinition}

\begin{snippetdefinition}{visione-latouche-definizione}{Visione di Latouche}
    Secondo Latouche, non esistono quindi modelli universali
    ma le nozioni di sviluppo devono essere messe in relazione con la diversità
    delle culture e delle civiltà.
\end{snippetdefinition}

\begin{snippet}{6fafd2ab-f112-4d30-a261-222dfd510880}
    Opposte alle \textbf{visioni critiche} (visione ambientalista e visione di Latouche),
    vi sono le \textbf{visioni ortodosse} (3, 4, 5).
    Le visioni ortodosse hanno in comune il fatto che lo sviluppo sia direttamente legato crescita economica,
    mentre le altre due indicano che lo sviluppo non è solo necessariamente legato alla crescita economia,
    bensì non sono nemmeno un criterio oggettivo.
\end{snippet}

\begin{snippetdefinition}{pragmatismo-scientifico-definizione}{Pragmatismo scientifico}
    Il \textit{pragmatismo scientifico} è una concezione nata dalla metà degli anni Ottanta.
    Essa spiega il sottosviluppo con un insime di cause esterne (la storia, situazione mondiale, azione del Nord, delle multinazionali, ecc.)
    e interne (regimi inefficienti, ecc.).
\end{snippetdefinition}

% https://moodle.edu.ti.ch/libe/pluginfile.php/119472/mod_resource/content/1/Latouche_Estratto_40_segg.pdf

\begin{snippet}{9c8bc7f7-3cd0-4778-8dc1-2dd2597a0138}
    Secondo Latouche lo sviluppo non può essere sostenibile in quanto
    un pianeta finito non può accomodare uno sviluppo illimitato.
\end{snippet}

\end{document}