\documentclass[preview]{standalone}

\usepackage{amsmath}
\usepackage{amssymb}
\usepackage{stellar}
\usepackage{definitions}
\usepackage{bettelini}

\begin{document}

\id{settheory-basic-results}
\genpage

\section{Basic results}

\begin{snippetcorollary}{subset-of-itself}{Every set is subset of itself}
    For every \set \(A\), \(A \subseteq A\).
\end{snippetcorollary}

\begin{snippetcorollary}{empty-set-is-subset-of-any-set}{Empty set is subset of any set}
    For every \set \(A\),
    \(\emptyset \subseteq A\).
\end{snippetcorollary}

% For every property P, all the elements of the empty set satisfy P

\begin{snippetcorollary}{subset-of-powerset}{Subset of powerset}
    For every \set \(B\), \(B\in\powerset(B)\).
\end{snippetcorollary}

\begin{snippetcorollary}{set-equivalence-with-subsets}{Set equivalence with subsets}
    Let \(A\) and \(B\) be \set[sets].
    \[ A = B \iff A \subseteq B \land B \subseteq A \]
\end{snippetcorollary}

\begin{snippetcorollary}{subseteq-transitivity}{Subset transitivity}
    Let \(A\), \(B\) and \(C\) be \set[sets].
    \begin{align*}
        &1. \quad A \subseteq B \land B \subseteq C \implies A \subseteq C \\
        &2. \quad A \subset B \land B \subseteq C \implies A \subset C \\
        &3. \quad A \subseteq B \land B \subset C \implies A \subset C \\
        &4. \quad A \subset B \land B \subset C \implies A \subset C
    \end{align*}
\end{snippetcorollary}

\begin{snippetproof}{subseteq-transitivity-proof}{subseteq-transitivity}{Subset transitivity}
    Without loss of generality (1) we have \(B \subseteq C \implies \forall b \in B, b \in C\)
    and \(A \subseteq B \implies \forall a \in A, a \in B\).
    This implies that \(\forall a \in A, a \in C\) and thus \(A \subseteq C\).
\end{snippetproof}

\begin{snippetcorollary}{proper-subsets-with-subsets}{Proper subsets with subsets}
    Let \(A\) and \(B\) be \set[sets].
    \[ A \subset B \iff A \subseteq B \land B \notsubseteq A \]
\end{snippetcorollary}

\begin{snippetcorollary}{dual-set-difference}{}
    Note that
    \[
        A \difference B = B \difference A
        \iff A = B
    \]
\end{snippetcorollary}

\begin{snippetcorollary}{subset-in-terms-of-relationships}{Subset in terms of relationships}
    \[
        A \subseteq B
        \iff
        A \union B = B
        \iff
        A \intersection B = A
        \iff
        A \difference B = \emptyset
    \]
\end{snippetcorollary}

\begin{snippetproposition}{set-equivalence-power-set-equivalence}{Set and power set equivalence}
    Let \(A\) and \(B\) be \set[sets].
    \[ A = B \iff \powerset(A) = \powerset(B) \]
\end{snippetproposition}

\begin{snippetproof}{set-equivalence-power-set-equivalence-proof}{set-equivalence-power-set-equivalence}{Set and power set equivalence}
    \iffproof{
        Assume \(A=B\), then \(\powerset(A) = \powerset(B)\).
    }{
        Assume \(\powerset(A) = \powerset(B)\),
        \(A \in \powerset(A)\) and \(A \in \powerset(B)\), and thus \(A \subseteq B\).
        Likewise, we can show that \(B \subseteq A\), and by
        \snippetref[set-equivalence-with-subsets][set equivalence with subsets] we have \(A=B\).
    }
\end{snippetproof}

\begin{snippettheorem}{cardinality-of-the-power-set}{Cardinality of the power set}
    Let \(A\) be a \set where \(\cardinality{A} < \infty\). The cardinality of \(\powerset(A)\) is given by
    \[
        \cardinality{\powerset(A)} = 2^{\cardinality{A}}
    \]
\end{snippettheorem}

\begin{snippetproof}{cardinality-of-the-power-set-proof}{cardinality-of-the-power-set}{Cardinality of the power set}
    Let \(A\) be a \set where \(\cardinality{A} < \infty\).
    To find \(\cardinality{\powerset(A)}\) we can count all the ways of selecting
    \(k\) out of \(n=\cardinality{S}\) elements with \(k=0,1,\cdots, n\).
    \[
        \cardinality{\powerset(A)} = \sum_{k=0}^n \binom{n}{k}
    \]
    From the \snippetref[binomial-theorem][binomial theorem] we have that
    \[
        \cardinality{\powerset(A)} = \sum_{k=0}^n \binom{n}{k} 1^{n-k}1^k = {(1+1)}^n = 2^n
    \]
\end{snippetproof}

\begin{snippetcorollary}{infinite-set-power-set-is-infinite}{Power set of infinite set is infinite}
    Let \(A\) be a \set where \(\cardinality{A} = \infty\).
    \[ \cardinality{\powerset(A)} =\infty \]
\end{snippetcorollary}

\begin{snippetproof}{infinite-set-power-set-is-infinite-proof}{infinite-set-power-set-is-infinite}{Power set of infinite set is infinite}
    Let \(A\) be a \set where \(\cardinality{A} = \infty\).
    Let's assume that \(\cardinality{\powerset(A)} < \infty\), then there would be a finite number of subsets of \(A\).
    This implies that there is only a maximum numbers of distinct subsets that can be formed from \(A\),
    but an infinite set should always be able to create an infinite amount of distinct subsets \lightning.
\end{snippetproof}

\begin{snippetcorollary}{set-union-intersection-commutative}{Union and intersection is commutative}
    Let \(A\) and \(B\) be \set[sets].
    \[ A \union B = B \union A \qquad A \intersection B = B \intersection A \]
\end{snippetcorollary}

\begin{snippetcorollary}{set-union-intersection-associative}{Union and intersection is associative}
    Let \(A\), \(B\) and \(C\) be \set[sets].
    \[ (A \union B) \union C = A \union (B \union C) \]
    and
    \[ (A \intersection B) \intersection C = A \intersection (B \intersection C) \]
\end{snippetcorollary}

\begin{snippetproposition}{subset-relation-union-intersection}{Subset relations in union and intersection}
    Let \(A\), \(B\) and \(C\) be \set[sets].
    \begin{enumerate}
        \item \(A \subseteq A \union B\);
        \item \(B \subseteq A \union B\);
        \item \(A \supseteq A \intersection B\);
        \item \(B \supseteq A \intersection B\).
    \end{enumerate}
\end{snippetproposition}

\begin{snippetproposition}{set-distributivity-intersection-over-union}{Distributivity of Intersection over Union}
    Let \(A\), \(B\) and \(C\) be \set[sets].
    \[ (A \union B) \intersection C = (A \intersection C) \union (B \intersection C) \]
\end{snippetproposition}

\begin{snippetproposition}{set-distributivity-union-over-intersection}{Distributivity of Union over Intersection}
    Let \(A\), \(B\) and \(C\) be \set[sets].
    \[ (A \intersection B) \union C = (A \union C) \intersection (B \union C) \]
\end{snippetproposition}

\begin{snippetproposition}{subset-characterization-union-and-intersection}{}
    Let \(A\), \(B\) and \(C\) be \set[sets].
    \begin{enumerate}
        \item \(A \union B \subseteq C \iff A \subseteq C \land B \subseteq C\);
        \item \(A \intersection B \supseteq C \iff A \supseteq C \land B \supseteq C\).
    \end{enumerate}
\end{snippetproposition}

\end{document}