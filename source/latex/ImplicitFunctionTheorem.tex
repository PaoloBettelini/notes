\documentclass[preview]{standalone}

\usepackage{amsmath}
\usepackage{amssymb}
\usepackage{stellar}
\usepackage{bettelini}

\hypersetup{
    colorlinks=true,
    linkcolor=black,
    urlcolor=blue,
    pdftitle={Assets},
    pdfpagemode=FullScreen,
}

\begin{document}

\id{implicit-function-theorem}
\genpage

\begin{snippet}{implicit-function-theorem-expl1}
    Consider the circle equation \(F(x,y)=x^2+y^2-1=0\).
    This equation is a function locally everywhere except at the points \(x=\pm 1\).
    Indeed, it is possible to represent part of the circle as the graph
    of a function of one variable \(g_1(x) = \sqrt(1-x^2)\) and
    \(g_2(x) = \sqrt(1-x^2)\).
    The implicit function theorem tells a criterion under which
    functions like \(g_1, g_2\) exist (except at a finite set of points). 
\end{snippet}

\end{document}