\documentclass[preview]{standalone}

\usepackage{amsmath}
\usepackage{amssymb}
\usepackage{bettelini}
\usepackage{stellar}
\usepackage{definitions}

\begin{document}

\id{chimica-tecniche-separazione}
\genpage

\section{Metodi meccanici}

\plain{Le componenti di un miscuglio eterogeneo possono essere separate con metodi meccanici:}

\begin{snippetdefinition}{decantazione-definition}{Decantazione}
    La \textit{decantazione} si usa di solito per separare due liquidi di densità diversa
    sfruttando la gravità.
\end{snippetdefinition}

\begin{snippetexample}{decantazione-example}{Decantazione}
    la separazione dell'olio e l'acqua.
\end{snippetexample}

\begin{snippetdefinition}{filtrazione-definition}{Filtrazione}
    Versando un fluido con un materiale sospeso al suo interno attraverso un setto poroso,
    permette di separare il fluido dal materiale.
\end{snippetdefinition}

\begin{snippetdefinition}{centrifugazione-definition}{Centrifugazione}
    Inserendo un miscuglio eterogeneo in una centrifuga, esso verrà roteato ponendo il
    fluido a una forza centrifuga, rendendo più rapida la sedimentazione della sostanza con densità maggiore
\end{snippetdefinition}

\section{Metodi fisici}

\plain{Le componenti di un miscuglio omogeneo (soluzioni) possono essere separate con metodi fisici:}

\begin{snippetdefinition}{distillazione-definition}{Distillazione}
    La \textit{distillazione} sfrutta i diversi punti di ebollizione di due liquidi per separarli.
    La miscela viene riscaldata fino a quando solo uno delle due componenti diventa vapore, per poi
    spostarla e raffreddarla nuovamente.
\end{snippetdefinition}

\begin{snippetdefinition}{cromatografia-definition}{Cromatografia}
    La \textit{cromatografia} sfrutta la tendenza delle sostanze a sciogliersi o interagire
    con diverse specie chimiche.
\end{snippetdefinition}

\begin{snippetdefinition}{estrazione-definition}{Estrazione}
    L'\textit{estrazione} si basa sulla maggiore o minore solubilità di un componente di un miscuglio in una certa miscela.
\end{snippetdefinition}

\section{Metodi chimici}

\plain{Le componenti di una soluzione possono essere separate con metodi chimici:}

\begin{snippetdefinition}{dialisi-definition}{Dialisi}
    La \textit{dialisi} separa i soluti in base alla loro dimensione e alla loro capacità di diffondere attraverso una
    membrana semipermeabile.
\end{snippetdefinition}

\begin{snippetdefinition}{precipitazione-definition}{Precipitazione}
    Vengono aggiunti reagenti che causano la formazione di un solido insolubile (precipitato)
    da una soluzione, che può essere separato per filtrazione.
\end{snippetdefinition}

\end{document}
