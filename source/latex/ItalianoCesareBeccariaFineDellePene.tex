\documentclass[preview]{standalone}

\usepackage{amsmath}
\usepackage{amssymb}
\usepackage{stellar}
\usepackage{definitions}

\begin{document}

\id{cesare-beccaria-fine-delle-pene}
\genpage

\section{Fine delle pene}

\begin{snippet}{fine-delle-pene}
    Lo scopo delle pene \textbf{non} è
    quello di torturare una persona o di cancellare un delitto
    (e ricompensare la vittima), in quanto non c'è legame fra la sofferenza
    del reo e il delitto che ha commesso, bensì quello di
    evitare la recidiva e fare da deterrente su persone nefaste.
    \\\\
    Uno Stato non può agire per passione o per fanatismo, ma dev'essere moderato
    e oggettivo.
    \\\\
    Secondo Beccaria, la pena perfetta deve avere le seguenti tre caratteristiche:
    \begin{itemize}
        \item proporzionalità al delitto;
        \item deterrente agli altri;
        \item priva di violenza.
    \end{itemize}
    Questo punto è stato molto criticato dagli stessi illuministi
    dal momento che non vi è una sola parola a favore della vittima (risarcimento).
    Oggi i risarcimenti consistono in denaro o lavori socialmente utili, che risarciscono
    la società.
    
    % Scopo delle pene per linee negative
    % Argomenti (domande retoriche)
    % Scopo delle pene per linee positivie
    % 
\end{snippet}

\end{document}