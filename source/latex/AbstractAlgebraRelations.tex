\documentclass[preview]{standalone}

\usepackage{amsmath}
\usepackage{amssymb}
\usepackage{bettelini}
\usepackage{stellar}
\usepackage{definitions}
\usepackage{cancel}

\begin{document}

\id{settheory-relations}
\genpage

\section{Relations}

\begin{snippetdefinition}{binary-relation-definition}{Binary relation}
    Let \(A\) and \(B\) be \set[sets].
    A \textit{binary relation} between \(A\) and \(B\) is a set \(R\)
    where
    \[ R \subseteq A \cartesianprod B \]
    Given two elements \(a\in A\) and \(b\in B\), we say that
    \(a\,R\,b\) if \((a,b) \in R\) and \(a\,\cancel{R}\,b\) if \((a,b) \notin R\).
\end{snippetdefinition}

% a more generic relation is an n-tuple

\begin{snippetdefinition}{relation-composition-definition}{Relation composition}
    Let \(A\), \(B\) and \(C\) be \set[sets], \(S\) be a \binrelation between
    \(A\) and \(B\) and \(T\) a \binrelation between \(B\) and \(C\).
    The \textit{composition} of \(S\) and \(T\) is the \binrelation \(ST\)
    defined as follows:
    \[
        (a,c) \in ST \iff \exists b \in B \suchthat (a,b) \in S \land (b,c) \in T
    \]
\end{snippetdefinition}

\begin{snippetdefinition}{homogeneous-relation-definition}{Homogeneous Relation}
    Let \(A\) be a set. A \textit{homogeneous relation} on \(A\) is a \binrelation
    from a \(A\) to \(A\).
\end{snippetdefinition}

%\subsection{Reflexive relation}
%
%\begin{snippetdefinition}{refexive-relation-definition}{Reflexive relation}
%    A \homrelation \(R\) on a set \(A\) is \textit{reflexive}
%    if
%    \[
%        \forall a\in A, (a,a) \in R
%    \]
%\end{snippetdefinition}
%
%\subsection{Symmetric relation}
%
%\begin{snippetdefinition}{symmetric-relation-definition}{Symmetric relation}
%    A \homrelation \(R\) on a set \(A\) is \textit{symmetric}
%    if
%    \[
%        \forall (a,b) \in R, (b,a) \in R
%    \]
%\end{snippetdefinition}
%
%\subsection{Transitive relation}
%
%\begin{snippetdefinition}{transitive-relation-definition}{Transitive relation}
%    A \homrelation \(R\) on a set \(A\) is \textit{transitive}
%    \[
%        \forall a,b,c \in A, (a,b) \in R \land (b,c) \in R \implies (a,c) \in R 
%    \]
%\end{snippetdefinition}

\begin{snippetdefinition}{equivalence-relation-definition}{Equivalence relation}
    An \textit{equivalence relation} is a \homrelation \(\sim\) on a set \(A\)
    that is
    \begin{enumerate}
        \item \textit{Reflexive}: \(\forall a \in A, a \sim a\)
        \item \textit{Symmetric}: \(\forall a,b \in A, a \sim b \iff b \sim a\)
        \item \textit{Transitive}: \(\forall a,b,c \in A, a \sim b \land b \sim c \implies a \sim c\)
    \end{enumerate}
\end{snippetdefinition}

\section{Equivalence class}

\begin{snippetdefinition}{equivalence-class-definition}{Equivalence class}
    Let \(\sim\) be an \equivrelation on a set \(A\).
    Given an element \(a\in A\), the \textit{equivalence class} of \(a\), is defined as
    \[
        {[a]}_{\sim} = \{x \in A \suchthat a \sim x\}
    \]
\end{snippetdefinition}

\begin{snippettheorem}{shared-element-in-equivalence-class-theorem}{Shared element in equivalence class}
    Let \(\sim\) be an \equivrelation on a set \(A\)
    and \(a,b \in A\).
    Then,
    \[
        b \in {[a]}_{\sim} \iff {[a]}_{\sim} = {[b]}_{\sim}
    \]
\end{snippettheorem}

\begin{snippetproof}{shared-element-in-equivalence-class-proof}{shared-element-in-equivalence-class-theorem}{Shared element in equivalence class}
    By the symmetric property we have \(a \in {[a]}_{\sim}\).
    Let \(b \in {[a]}_{\sim}\), meaning \(a \sim b\). \(\forall c \in {[b]}_{\sim}\),
    meaning \(b \sim c\), we have \(a \sim c\) by the transitive property.
    Thus, \(c \in {[a]}_{\sim}\) and \({[b]}_{\sim} \subseteq {[a]}_{\sim}\).
    By the symmetric property we also have \(b \sim a\),
    \(\forall d \in {[a]}_{\sim}\), meaning \(a \sim d\), we have
    \(b \sim d\) by the transitive property. Thus, \(d \in {[b]}_{\sim}\)
    and \({[a]}_{\sim} \subseteq {[b]}_{\sim}\). Hence,
    \[
        b \in {[a]}_{\sim} \iff {[a]}_{\sim} = {[b]}_{\sim}
    \]
\end{snippetproof}

\begin{snippet}{equivalence-expl-1}
    This means that every element of an \equivclass has the same \equivclass.
    Thus, if two classes share an element they are the same.
\end{snippet}

\section{Partition of a set}

\begin{snippetdefinition}{set-partition-definition}{Partition of a set}
    Given a set \(A\), a \textit{partition} of a set \(P={\{C_i\}}_{i\in I}\) is a collection of
    non-empty subsets of \(A\) such that \(\bigcup_{i\in I} C_i = P\) and
    \(C_i \intersection C_j = \emptyset, i \neq j\).
\end{snippetdefinition}

\begin{snippet}{settheory-2}
    In other words, the sets \(C_i\)
    contain every element of \(A\) exactly once.
\end{snippet}

\begin{snippettheorem}{equivalence-classes-form-partition-theorem}{Equivalence classes form partitions}
    Let \(\sim\) be an \equivrelation on a set \(A\).
    The set of its \equivclass[equivalence classes] is a \partition of \(A\).
\end{snippettheorem}

\subsection{Types of orders}

\begin{snippetdefinition}{preorder-order-definition}{Preorder order}
    A \textit{preorder} is a \homrelation \(\leq\) on a set \(A\)
    with the following properties:
    \begin{enumerate}
        \item \textit{Reflexive}: \(\forall a \in A, a \leq a\)
        \item \textit{Transitive}: \(\forall a,b,c \in A, a \leq b \land b \leq c \implies a \leq c\)
    \end{enumerate}
\end{snippetdefinition}

\begin{snippetdefinition}{partial-order-definition}{Partial order}
    A \textit{partial order} is a \homrelation \(\leq\) on a set \(A\)
    with the following properties:
    \begin{enumerate}
        \item \textit{Reflexive}: \(\forall a \in A, a \leq a\)
        \item \textit{Transitive}: \(\forall a,b,c \in A, a \leq b \land b \leq c \implies a \leq c\)
        \item \textit{Antisymmetric}: \(\forall a,b \in A, a \leq b \land b \leq a \implies a=b\)
    \end{enumerate}
\end{snippetdefinition}

\begin{snippetdefinition}{total-orderv}{Total order}
    A \textit{total order} is a \homrelation \(\leq\) on a set \(A\)
    with the following properties:
    
    \begin{enumerate}
        \item \textit{Reflexive}: \(\forall a \in A, a \leq a\)
        \item \textit{Transitive}: \(\forall a,b,c \in A, a \leq b \land b \leq c \implies a \leq c\)
        \item \textit{Antisymmetric}: \(\forall a,b \in A, a \leq b \land b \leq a \implies a=b\)
        \item \textit{Strongly connected} (or \textit{total}): \(\forall a,b\in A, a \leq b \lor b\leq a\)
    \end{enumerate}
\end{snippetdefinition}

\begin{snippet}{settheory-3}
    A total order is a partial order where any two elements are comparable.
\end{snippet}

\subsection{Elements characterization}

\begin{snippetdefinition}{greatest-element-definition}{Greatest element}
    Given a partial order on a set \(A\), an element \(g\) is a \textit{greatest element}
    if \(\forall a\in A, a \leq g\).
\end{snippetdefinition}

\begin{snippetdefinition}{least-element-definition}{Least element}
    Given a partial order on a set \(A\), an element \(g\) is a \textit{least element}
    if \(\forall a\in A, g \leq a\).
\end{snippetdefinition}

\begin{snippetdefinition}{maximal-element-definition}{Maximal element}
    Given a partial order on a set \(A\), an element \(g\in A\) that is
    a greatest element is a \textit{maximal element}.
\end{snippetdefinition}

\begin{snippetdefinition}{minimal-element-definition}{Minimal element}
    Given a partial order on a set \(A\), an element \(g\in A\) that is
    a least element is a \textit{minimal element}.
\end{snippetdefinition}

\end{document}