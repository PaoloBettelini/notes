\documentclass[preview]{standalone}

\usepackage{amsmath}
\usepackage{amssymb}
\usepackage{tikz}
\usepackage{stellar}
\usepackage{bettelini}
\usepackage{makecell}

\hypersetup{
    colorlinks=true,
    linkcolor=black,
    urlcolor=blue,
    pdftitle={Assets},
    pdfpagemode=FullScreen,
}

\begin{document}

\title{Margini}
\id{geofisica-margini}
\genpage

\section{Margini}

\begin{snippet}{tipi-di-margine}
    I tipi possibile di margine sono
    \begin{itemize}
        \item Oceanica - Oceanica
        \item Oceanica - Continentale
        \item Continentale - Continentale
    \end{itemize}
\end{snippet}

\subsection{Margini divergenti}

\begin{snippetdefinition}{margine-divergente}{Margine divergente}
    I \textit{margini divergenti} o \textit{margini costruttivi} avvengono se
    la densità del mantello diminuisce in un punto quando vi è una risalita di magma.
\end{snippetdefinition}

\includesnpt{rift-valley}

\subsection{Margini convergenti}

\begin{snippetdefinition}{margine-convergente}{Margine convergente}
    I \textit{margini convergenti} o \textit{margini di subduzione} (collisione) sono distruttivi e
    avvengono quando due margini si scontrano.
\end{snippetdefinition}

\includesnpt{margini-convergenti}

\plain{Il materiale che fonde, risale dando origine ad attività vulcaniche.
Se i margini sono oceanica-oceanica, abbiamo una formazione di isole vulcaniche,
altrimenti si tratta di margini oceanica-continentale.}

\subsection{Margini trasformi}

\begin{snippetdefinition}{margine-trasforme}{Margine trasforme}
    I \textit{margini trasformi} o \textit{margini conservativi} e di \textit{scorrimento},
    avvengono quando le placche si muovono in senso opposto a velocità differenti.
\end{snippetdefinition}

\plain{I margini trasformi non provocano vulcanismo ma violenti terremoti.}

\includesnpt{margini-trasformi}

\end{document}