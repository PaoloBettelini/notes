\documentclass[preview]{standalone}

\usepackage{amsmath}
\usepackage{amssymb}
\usepackage{parskip}
\usepackage{fullpage}
\usepackage{hyperref}
\usepackage{bettelini}
\usepackage{stellar}

\hypersetup{
    colorlinks=true,
    linkcolor=black,
    urlcolor=blue,
    pdftitle={Measure Theory},
    pdfpagemode=FullScreen,
}

\begin{document}

\id{measuretheory-other}
\genpage

% https://courses.maths.ox.ac.uk/course/view.php?id=1041
% Zhongmin Qian's Notes from 2017 
% https://www.youtube.com/playlist?list=PLBh2i93oe2qvMVqAzsX1Kuv6-4fjazZ8j

\section{Example of non Riemann integrable function}

\begin{snippetdefinition}{dirichlet-function-definition}{Dirichlet function}
    The \textit{Dirichlet function} is defined as the indicator function of the rational numbers
    \[
        1_{\mathbb{Q}}(x) \triangleq \begin{cases}
            1 & x \in \mathbb{Q} \\
            0 & x \notin \mathbb{Q}
        \end{cases}
    \]
\end{snippetdefinition}

\begin{snippet}{dirichlet-function-is-not-riemann-integrable}
    Since there are uncountably many irrational numbers and countably many ration numbers,
    and thus there is a 0\% probability of picking a rational number, the integral
    of the Dirichlet function in the interval \([0; 1]\) is 1.
    \[
        \integral[0][1][1_{\mathbb{Q}}(x)][x] = 1
    \]
    However, the Dirichlet function is not Riemann integratable.
    The Lebesgue measure extends the class of integratable functions, like the Dirichlet function.
\end{snippet}


\end{document}
