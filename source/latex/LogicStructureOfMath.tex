\documentclass[preview]{standalone}

\usepackage{amsmath}
\usepackage{amssymb}
\usepackage{parskip}
\usepackage{fullpage}
\usepackage{hyperref}
\usepackage{stellar}

\hypersetup{
    colorlinks=true,
    linkcolor=black,
    urlcolor=blue,
    pdftitle={Logic},
    pdfpagemode=FullScreen,
}

\begin{document}

\id{structure-of-math}
\genpage

\section{What is math}

\begin{snippet}{what-is-math}
Consider any science, like psychology. Psychology is based on
neuroscience, neuroscience is based on biology,
biology is based on organic chemistry, organic chemistry is based on chemistry,
chemistry is based on physics and finally, physics, is expressed with mathematics.
In general, mathematics is always at the top of the pyramid.
But what is math based on?
\\\\
Math is essentialy the study of propositions,
it is a tool to study whether given propositions are true of false.
Theorems, which are just important propositions, are proved using other theorems,
which are proved using other theorems and so on. This cannot go on forever, and at the
beginning of this chain there must be some sort of assumption(s), on which everything
is based on.
Math ought to be based on something that is \textit{true} in our universe.
But can we really know anything to a certainty?
This is more of a philosophical question about \href{https://en.wikipedia.org/wiki/Epistemology}{epistemology}.
To evade this problem and avoid uncertainty, we make some assumptions which seem reasonable to our human mind.
Indeed, those assumptions are called \textit{axioms} or \textit{postulates}.
\end{snippet}

\includesnpt{logic-axiom-definition}

\subsection{Types of propositions}

\begin{snippet}{types-of-propositions}
The following terms all refer to a proposition, but very slightly in their importance.
From a mathematical perspectiver, they are all the same.
\\
\begin{itemize}
    \item \textbf{Theorem}: an important proposition that can be proved.
    \item \textbf{Lemma}: an intermediary proposition used to prove a theorem.
    \item \textbf{Corollary}: a proposition that directly follows from a theorem or lemma.
\end{itemize}
\end{snippet}

\end{document}
