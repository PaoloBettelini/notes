\documentclass[preview]{standalone}

\usepackage{amsmath}
\usepackage{amssymb}
\usepackage{bettelini}
\usepackage{stellar}

\hypersetup{
    colorlinks=true,
    linkcolor=black,
    urlcolor=blue,
    pdftitle={English},
    pdfpagemode=FullScreen,
}

\begin{document}

\id{english-aviewfromthebridge-characters}
\genpage

\begin{snippetcharacter}{aviewfromthebridge-eddie-carbone}{Eddie Carbone}
    A Longshoreman. Eddie lives with his wife, Beatrice,
    and orphaned niece, Catherine, in Red Hook Brooklyn.
    Eddie is an inarticulate character and is powerless in the face of his tragic fate.
    He harbors a secret lust for his niece Catherine which causes his eventual destruction.
\end{snippetcharacter}

\begin{snippetcharacter}{aviewfromthebridge-catherine}{Catherine}
    The niece of Eddie Carbone and Beatrice. Catherine is a beautiful, smart, young Italian girl who is very popular among the boys in the community. Catherine seeks approval from her uncle and struggles when Eddie does not approve of Rodolpho, the man she intends to marry.
\end{snippetcharacter}

\begin{snippetcharacter}{aviewfromthebridge-beatrice}{Beatrice}
    The wife of Eddie Carbone and aunt of Catherine. Beatrice has raised Catherine from the time she was very young and acts as Catherine's mother. Beatrice is a warm and caring woman, more reasonable than Eddie. Like Catherine, Beatrice is not a well-developed character in the play.
\end{snippetcharacter}

\begin{snippetcharacter}{aviewfromthebridge-marco}{Marco}
    The cousin of Beatrice. Marco comes to the U.S. to work and make money to send back to his wife and children in Italy. Marco is a hard-working Italian man who is a powerful, sympathetic leader.
\end{snippetcharacter}

\begin{snippetcharacter}{aviewfromthebridge-rodolpho}{Rodolpho}
    Beatrice's young, blonde cousin from Italy. Rodolpho prefers singing jazz to working on the ships. To Eddie and the other Longshoremen, Rodolpho seems effeminate because he also cooks, sews and loves to dance. Rodolpho desires to be an American and have all the privileges of Western society including wealth and fame.
\end{snippetcharacter}

\begin{snippetcharacter}{aviewfromthebridge-alfieri}{Alfieri}
    An Italian-American lawyer. Alfieri is the narrator of the story. He speaks directly to the audience and attempts to make clear the greater social and moral implications of the story.
\end{snippetcharacter}

\begin{snippetcharacter}{aviewfromthebridge-mike}{Mike}
    A Longshoreman and friend of Eddie's. Mike is often seen with Louis outside the Carbone home.
\end{snippetcharacter}

\begin{snippetcharacter}{aviewfromthebridge-louis}{Louis}
    A Longshoreman and friend of Eddie's. Louis hangs out with Mike outside Eddie's home.
\end{snippetcharacter}

\begin{snippetcharacter}{aviewfromthebridge-tony}{Tony}
    A friend of the Carbones. He assists Marco and Rodolpho off the ship and brings them safely to Beatrice's home.
\end{snippetcharacter}

\begin{snippetcharacter}{aviewfromthebridge-first-immigration-office}{First Immigration Office}
    One of two officers from the Immigration Bureau who comes to look for Marco and Rodolpho at Eddie's request.
\end{snippetcharacter}

\begin{snippetcharacter}{aviewfromthebridge-second-immigration-officer}{Second Immigration Officer}
    One of two officers from the Immigration Bureau who comes to look for Marco and Rodolpho at Eddie's request.
\end{snippetcharacter}

\begin{snippetcharacter}{aviewfromthebridge-mr-lipari}{Mr. Lipari}
    A butcher who lives upstairs from the Carbones. Eddie blames Mr. Lipari for the arrest of Marco and Rodolpho.
\end{snippetcharacter}

\begin{snippetcharacter}{aviewfromthebridge-mrs-lipari}{Mrs. Lipari}
    The upstairs neighbor of the Carbones. Mrs. Lipari agrees to give Marco and Rodolpho a room in her home when Eddie kicks the men out of his house.
\end{snippetcharacter}

\begin{snippetcharacter}{aviewfromthebridge-two-submarines}{Two ``Submarines''}
    Two illegal immigrants hiding upstairs in the Lipari house.
\end{snippetcharacter}

% https://www.sparknotes.com/drama/viewbridge/characters/

\end{document}
