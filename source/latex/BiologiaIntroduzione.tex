\documentclass[preview]{standalone}

\usepackage{amsmath}
\usepackage{amssymb}
\usepackage{stellar}
\usepackage{definitions}
\usepackage{bettelini}

\begin{document}

\id{biologia-introduzione}
\genpage

\section{Introduzione}

\begin{snippetdefinition}{cellula-definition}{Cellula}
    La \textit{cellula} è la più piccola unità funzionale degli esseri
    viventi. È delimitata da una membrana plasmatica
    che racchiude varie strutture, dette organuli cellulari.
\end{snippetdefinition}

\begin{snippetdefinition}{organulo-definition}{Organulo}
    Un \textit{organulo} è una struttura racchiusa da una membrana che svolge
    una funzione specifica all'interno della cellula.
\end{snippetdefinition}

\begin{snippetdefinition}{tessuto-definition}{Tessuto}
    Un \textit{tessuto} è costituito da gruppi
    di cellule simili che svolgono
    una specifica funzione.
\end{snippetdefinition}

\begin{snippet}{biologia-definizioni-expl1}
    % pag 14
Tutte le specie viventi sono suddivise in tre grandi gruppi:
\textbf{eubatteri} (Bacteria), \textbf{archebatteri} (Archaea)
ed \textbf{eucarioti} (Eukarya).
I primi due domini, eubatteri e archebatteri, identificano due gruppi di organismi
unicellulari molto diversi formati da cellule
procariote.
Il dominio degli eucarioti, invece, comprende organismi sia unicellulari
sia pluricellulari costituiti da cellule eucariote ed è,
a sua volta, suddiviso in più regni: \textbf{protisti},
\textbf{piante}, \textbf{funghi} e \textbf{animali}.
\end{snippet}

\section{Il microscopio}

\begin{snippetdefinition}{microscopio-ottima-definition}{Microscopio ottico}
    Tipologia di microscopio in cui la luce visibile, concentrata da un condensatore, attraversa
    il campione da esaminare. La luce attraversa poi due serie di lenti (obiettivo e oculare) e
    viene deviata prima di raggiungere l'occhio dell'esaminatore su cui forma un'immagine
    ingrandita del preparato.
\end{snippetdefinition}

\begin{snippet}{biologia-microscopi-ottici-tipi}
    Il microscopio ottico di distingue in:
    \begin{itemize}
        \item \textit{ottico a contrasto di fase (moderno):} amplifica le differenze di densità delle varie parti di
        un campione e permette comunque di esaminare le cellule vive;
        \item \textit{confocale a fluorescenza:} usando il microscopio ottico confocale a fluorescenza, molecole specifiche vengono
        selettivamente marcate con sostanze fluorescenti ed è perciò possibile mettere a fuoco
        sezioni particolarmente sottili della cellula.
    \end{itemize}
\end{snippet}

\includesnpt[width=50\%|src=/snippet/static/microscopio.png]{centered-img}

\begin{snippetdefinition}{microscopio-elettronico-definition}{Microscopio elettronico}
    Al posto della luce visibile, il microscopio elettronico utilizza un fascio di elettroni; le lenti
    sono sostituite da elettromagneti che deviano il fascio di elettroni ottenendo
    l'ingrandimento dell'immagine. Il microscopio elettronico ha una risoluzione molto
    maggiore rispetto al microscopio ottico; può infatti distinguere strutture biologiche di soli
    0.2 nm, dunque 1000 volte più piccole di quelle distinguibili con un microscopio ottico.
\end{snippetdefinition}

\begin{snippet}{biologia-microscopi-elettronici-tipi}
    Il microscopio elettronico si distingue in:
    \begin{itemize}
        \item \textit{a scansione:} utilizza un fascio di elettroni per esplorare la
        superficie di una cellula o di un gruppo di cellule precedentemente rivestite da una sottile
        lamina metallica. Colpendo la superficie metallica, gli elettroni vengono riflessi e rilevati da
        un dispositivo in grado di convertire il segnale in un'immagine tridimensionale proiettata su
        uno schermo;
        \item \textit{a trasmissione:} viene utilizzato per studiare la struttura interna
        della cellula. I campioni vengono tagliati in sezioni estremamente sottili e “colorati” con
        atomi di metalli pesanti, come l'oro, che si depositano soltanto su alcune strutture cellulari.
        Si ottiene un'immagine ingrandita e a due dimensioni che evidenzia la struttura interna
        della cellula.
    \end{itemize}
\end{snippet}

\end{document}