\documentclass[preview]{standalone}

\usepackage{amsmath}
\usepackage{amssymb}
\usepackage{stellar}
\usepackage{bettelini}

\hypersetup{
    colorlinks=true,
    linkcolor=black,
    urlcolor=blue,
    pdftitle={Stellar},
    pdfpagemode=FullScreen,
}

\begin{document}

\title{Stellar}
\id{english-atonement-part-three}
\genpage

\section{Exercises}

\begin{snippetexercise}{atonement-ex-21}
    {What is the significance of the physical discomfort experienced by Briony and the other nurses at
    the hospital?}
    The physical discomfort humanizes Briony and the other nurses.
    It highlights their vulnerabilities and humanity. Regardless of their social
    status or background, they all suffer together in the face of the war's brutality.
    There is a parallel between being a soldier and a nurse. You have to obey
    rules from the higher-ups, you have to wear a univerform and become a number.
\end{snippetexercise}

\begin{snippetexercise}{atonement-ex-22}
    {\quotes{This was her student life now, these four years, this enveloping regime, and she had no will,
    no freedom to leave.} (p. 260, emphasis added) Explain the quote. Why would she express \quotes{no will
    to leave}?}
    She doesn't want to rebel, she just accepts it. She has become like a robot
    following orders.
    She feels guilty and thinks that this punishment is fair,
    she has to \textbf{atone} for her wrongdoings.
\end{snippetexercise}

\begin{snippetexercise}{atonement-ex-23}
    {What is Briony only escape from reality?}
    Briony's only escpape is writing about her patients and her days.
    She enjoys handwriting, which is all that remains left in her spark.
\end{snippetexercise}

\begin{snippetexercise}{atonement-ex-24}
    {What news does his father's letter contain? How does Briony react to it?}
    Her father delivers the news that Lola Quincey and Paul Marshall are to
    be married the following week.
    Though he does not provide any reason for telling her this,
    Briony understands what she has known for some time:
    Paul was the one who assaulted Lola that summer night in 1935.
    After receiving the letter, Briony feels her years-old guilt
    even more acutely and understands that no matter how good a nurse she is
    or whatever opportunities she has given up,
    she will never be able to make up for what she did to Robbie.
    Briony wonders if her father sent her the letter because he, too, has figured out the truth.
    She tries to call him but cannot get through.
    As Briony walks back to the hospital,
    she sees two army medics smiling at her, but she cannot bring herself to meet their eyes.
    Briony wonders what it would be like to live a carefree life.
\end{snippetexercise}

\begin{snippetexercise}{atonement-ex-25}
    {At the hospital, Briony deals with increasingly more severe injuries and her mental and physical
    strength is tested. She even acknowledges that \quotes{all the training she had received […] had been useful
    preparation, especially in obedience, but everything she understood about nursing she learned that
    night. She had never seen men crying before. It shocked her at first, and within the hour she was
    used to it} (286). What is the impact of the French soldier's death on her?}
    Briony has to deal with the death of the French soldier. She would like to cry for him.
    but she wouldn't feel a thing, she was empty.
\end{snippetexercise}

\begin{snippetexercise}{atonement-ex-26}
    {At the end of the section, Briony receives a letter from the publishing company to whom she had
    sent one of her short stories some time before, namely \quotes{Two Figures at a Fountain}. What are their
    main criticisms of Briony's story?}
    They do give her some appluase in the sense that it is well written,
    but they say that it's highly unrealistic and nothing really happens.
    This is important because we can see a little spark of hope towards her future as a writer.
\end{snippetexercise}

\end{document}