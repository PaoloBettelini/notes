\documentclass[preview]{standalone}

\usepackage{amsmath}
\usepackage{amssymb}
\usepackage{stellar}
\usepackage{definitions}
\usepackage{tikz}
\usetikzlibrary{cd}

\begin{document}

\id{permutations-and-k-cycles}
\genpage

\section{Permutations}

\begin{snippet}{permutation-notation-example}
    Let \(\sigma \in \text{Sym}_n\), we can explicit the permutation as
    \[
        \sigma \triangleq \begin{pmatrix}
            1 & 2 & \cdots & n \\
            \sigma(1) & \sigma(2) & \cdots & \sigma(n)
        \end{pmatrix}
    \]
    where \(1,2,\cdots, n\) are the letters.
\end{snippet}

\begin{snippetproposition}{permutation-group-non-abelian}{}
    The \group \(\permgrp_n\) is an \abeliangroup \ifandonlyif \(n<3\).
\end{snippetproposition}

\begin{snippetproof}{permutation-group-non-abelian-proof}{permutation-group-non-abelian}{}
    If \(n=1\), the case is trivial. If \(n=2\), the permutation can only exchange two letters.
    For \(n\geq 3\), consider
    \[
        \sigma \triangleq \begin{pmatrix}
            1 & 2 & 3 & \cdots & n \\
            2 & 1 & 3 & \cdots & n
        \end{pmatrix}
    \]
    and
    \[
        \tau \triangleq \begin{pmatrix}
            1 & 2 & 3 & 4 & \cdots & n \\
            3 & 2 & 1 & 4 & \cdots & n
        \end{pmatrix}
    \]
    Then,
    \begin{align*}
        \tau(\sigma) &= \begin{pmatrix}
            1 & 2 & 3 & 4 & \cdots & n \\
            2 & 3 & 1 & 4 & \cdots & n
        \end{pmatrix} \\
        \sigma(\tau) &= \begin{pmatrix}
            1 & 2 & 3 & 4 & \cdots & n \\
            3 & 1 & 2 & 4 & \cdots & n
        \end{pmatrix}
    \end{align*}
\end{snippetproof}

\begin{snippetdefinition}{permutation-support-definition}{Support of a permutation}
    Let \(\sigma \in \permgrp_n\). Then, the \emph{support of \(\sigma\)}
    is defined as
    \[
        \text{Supp}(\sigma) \triangleq \{ i \in \{1,2,\cdots,n\} \suchthat \sigma(i) \neq i \}
    \]
\end{snippetdefinition}

\plain{The support is the empty set if and only if the permutation is the identity function.
Indeed, if the permutation is not the identity function, then its cycle is greater or equal than 2.}

\begin{snippetproposition}{permutation-supp-closure}{}
    Let \(\sigma \in \permgrp_n\).
    Then,
    \[
        i \in \supp(\sigma) \implies \sigma(i) \in \supp(\sigma)
    \]
\end{snippetproposition}

\begin{snippetproof}{permutation-supp-closure-proof}{permutation-supp-closure}{}
    By the hypothesis, we have \(i \neq \sigma(i)\).
    We need to show that \(\sigma(i) \neq \sigma(\sigma(i))\).
    However, this is true since \(\sigma\) is \injective.
\end{snippetproof}

\begin{snippetproposition}{properties-of-supp}{}
    Let \(\sigma, \tau \in \permgrp_n\). Then:
    \begin{enumerate}
        \item \(\supp(\tau(\sigma)) \subseteq \supp(\sigma) \union \supp(\tau)\);
        \item \(\supp(\sigma^{-1}) = \supp(\sigma)\).
    \end{enumerate}
\end{snippetproposition}

\begin{snippetproof}{properties-of-supp-proof}{properties-of-supp}{}
    \begin{enumerate}
        \item Let \(i\in\supp(\tau(\sigma))\) (i.e. \(\tau(\sigma(i)) \neq i\)).
            We need to show that \(i\in\supp(\sigma)\) (meaning \(\sigma(i) \neq i\)) or
            \(i\in \supp(\tau)\) (meaning \(\tau(i) \neq i\)).
            Assume that this is not true, meaning \(i\notin\supp(\sigma) \land i\notin\supp(\tau)\).
            Thus, \(\sigma(i)=i\) and \(\tau(i)=i\) implying \(\tau(\sigma(i)) = \tau(i) = i\),
            which is a contradiction \lightning.
        \item Let \(i\in\supp(\sigma)\), meaning \(\sigma(i) \neq i\) or \(i\neq \sigma^\inversefunction(i)\)
            which also means \(i\in\supp(\sigma^\inversefunction)\).
            Thus, \(\supp(\sigma)\subseteq \supp(\sigma^\inversefunction)\).
            The reverse inclusion is obtained by swapping \(\sigma\) and \(\sigma^\inversefunction\).
    \end{enumerate}
\end{snippetproof}

\begin{snippetdefinition}{disjoint-permutation-definition}{Disjoint permutations}
    Let \(\sigma, \tau \in \permgrp_n\). Then, \(\sigma\) and \(\tau\)
    are said to be \emph{disjoint permutations} if \(\supp(\sigma)\) and \(\supp(\tau)\)
    are \disjoint.
\end{snippetdefinition}

\begin{snippetproposition}{disjoint-permutations-commute}{Disjoint permutations commute}
    Let \(\sigma, \tau \in \permgrp_n\) where \(\sigma\) and \(\tau\) are \disjointperm. Then,
    \(\sigma(\tau) = \tau(\sigma)\).
\end{snippetproposition}

\begin{snippetproof}{disjoint-permutations-commute-proof}{disjoint-permutations-commute}{Disjoint permutations commute}
    We need to show that for every \(i\in\{1,2,\cdots. n\}\), we have \(\tau(\sigma(i)) = \sigma(\tau(i))\).
    Consider the cases:
    \begin{enumerate}
        \item \(i\in\supp(\sigma) \land i\notin\supp(\tau)\): since \(i\in\supp(\sigma)\)
            we have \(\sigma(i)\in\supp(\sigma)\) and, thus, \(\sigma(i)\notin\supp(\tau)\),
            meaning \(\tau(\sigma(i)) \neq \sigma(i)\).
            On the other hand, \(\tau(i) = i\) and, thus, \(\tau(\sigma(i)) = \sigma(i)\).
            Thus, \(\tau(\sigma(i)) = \sigma(i) = \sigma(\tau(i))\);
        \item \(i\notin\supp(\sigma) \land i\in\supp(\tau)\): analogous to the last case;
        \item \(i\notin\supp(\sigma) \land i\notin\supp(\tau)\): we have that \(\tau(\sigma(i)) = \tau(i) = i\)
            and \(\sigma(\tau(i))=\sigma(i)=i\).
    \end{enumerate}
\end{snippetproof}

\begin{snippetproposition}{disjoint-permutation-period}{Period of disjoint permutations}
    Let \(\sigma, \tau \in \permgrp_n\) where \(\sigma\) and \(\tau\) are \disjointperm.
    Then,
    \[
        |\tau(\sigma)| = \lcm(|\tau|, |\sigma|)
    \]
\end{snippetproposition}

\begin{snippetproof}{disjoint-permutation-period-proof}{disjoint-permutation-period}{Period of disjoint permutations}
    Let \(|\sigma| = m\), \(|\tau| = n\) and \(t = \text{lmc}(m,n)\).
    Since the permutations are \disjointperm, we know that the commute and thus
    \({(\tau(\sigma))}^r = \tau^r(\sigma^r)\) for every \(r\in\integers\).
    In particular, \[{(\tau(\sigma))}^t = \tau^t(\sigma^t) = \text{Id}(\text{Id}) = \text{Id}\]
    This means that \(|\tau(\sigma)|\) is a divisor of \(t\).
    Let \(d=|\tau(\sigma)|\). We need to show that \(d=t\).
    In order to do this, we show that \(d\) is a multiple of \(m\) and \(n\).
    We ave that \({(\tau(\sigma))}^d = \text{Id}\). Then, \(\tau^d(\sigma^d) = \text{Id}\),
    from which \(\sigma^d = \tau^{-d}\). Now, \(\supp(\sigma^d) \subseteq \supp(\sigma)\)
    and likewise \(\supp(\tau^{-d}) \subseteq \supp(\tau)\).
    Since \(\sigma^d = \tau^{-d}\), obviously \(\supp(\sigma^d) = \supp(\tau^{-d})\),
    but \[\supp(\sigma^d) \intersection \supp(\tau^{-d}) \subseteq \supp(\sigma) \intersection \supp(\tau) = \emptyset\]
    Therefore, \(\supp(\sigma^d) = \supp(\tau^{-d}) = \emptyset\), meaning \(\sigma^d = \text{Id}\)
    and \(\tau^{-d} = \text{Id}\), from which it follows that \(d\) is a multiple of \(|\sigma|\)
    and \(-d\) is a multiple of \(|\tau|\) (and thus multiple of \(m\) and \(n\)).
\end{snippetproof}

\begin{snippetdefinition}{k-cycle-definition}{\(k\)-cycle}
    Let \(\sigma \in \permgrp_n\). Then, \(\sigma\) is said to be a \emph{\(k\)-cycle}
    if there exist \(a_0, a_1, \cdots, a_k\) distinct letters
    such that \[
        \bigwedge\limits_{i=0}^{k-1} \sigma(a_k) = a_{k+1 \bmod{k}}
    \]
    and \(\sigma(i) = i\) for all \(i\notin \{a_0, a_1, \cdots, a_k\}\). \\
    \emph{Notation:} such permutation is denoted \((a_1\;a_2\;\cdots\;a_k)\).
\end{snippetdefinition}

\plain{Any k-cycle can be written in k different ways and it has a period of k.}

\begin{snippetproposition}{supp-of-permutation}{Support of permutation}
    Let \(\sigma = \kcycle{a_1,a_2,\cdots,a_k}\). Then,
    \[
        \supp(\sigma) = \{a_1, a_2, \cdots, a_k\}
    \]
\end{snippetproposition}

\begin{snippetlemma}{unique-k-cycle}{}
    Let \(\sigma \in \permgrp_n\) and \(a \in \{1,2,\cdots, n\}\).
    Then, there exist a unique \(k\)-cycle \(\rho = \kcycle{a_1, a_2, \cdots, a_r}\)
    such that:
    \begin{enumerate}
        \item \(a\) is one of \(a_i\);
        \item \(\rho(a_i) = \sigma(a_i)\) for \(1 \leq i \leq r\).
    \end{enumerate}
\end{snippetlemma}

\begin{snippetproof}{unique-k-cycle-proof}{unique-k-cycle}{name}
    Consider the the following sequence:
    \[
        \begin{cases}
            a_1 \triangleq a \\
            a_{i+1} \triangleq \sigma(a_i)
        \end{cases}
    \]
    Since this sequence is defined over a finite \set,
    it will present repretitions at some point.
    Let \(a_i = a_j\) with \(i<j\) be the first repetition.
    We want to show that \(i=1\).
    Assume \(i>1\), then \(\sigma(a_{i-1}) = a_i = a_j = \sigma(a_{j-1})\)
    and since \(\sigma\) is \injective, \(a_{i-1} = a_{j-1}\),
    which is against the hypothesis of \(a_i = a_j\) being the first repetition \lightning.
    We thus have that \(\rho =\kcycle{a_1, a_2, \cdots, a_j}\)
    is a \(k\)-cycle by construction. The element \(a\) is one of \(a_i\).
    The uniqueness of \(\rho\) follows immediately by construction.
\end{snippetproof}

\begin{snippettheorem}{permutation-factorization-theorem}{Factorization of a permutation}
    Let \(\sigma \in \permgrp_n\). Then, \(\sigma\)
    can be uniquely written as a product of \disjointperm \(k\)-cycles
    up to ordering.
\end{snippettheorem}

\plain{The uniqueness is given by the fact that disjoint cycles commute.}

\begin{snippetproof}{permutation-factorization-theorem-proof}{permutation-factorization-theorem}{Factorization of a permutation}
    Take \(a \in \{1,2,\cdots,n\}\)
    and construct the \(k\)-cycle \(\kcycle{a_1, a_2, \cdots, a_r}\)
    as in \snippetref[unique-k-cycle3][this lemma].
    If \(\sigma = \kcycle{a_1, a_2, \cdots, a_r}\) we are done.
    Otherwise, consider \(b\in\supp(\sigma) \difference \{a_1, a_2, \cdots, a_r\}\)
    and construct a \(k\)-cycle \(\kcycle{b_1, b_2, \cdots, b_s}\)
    which permutes \(b\) as in \snippetref[unique-k-cycle3][this lemma].
    The two \(k\)-cycles \(\kcycle{b_1, b_2, \cdots, b_s}\) and \(\kcycle{a_1, a_2, \cdots, a_r}\)
    are \disjointperm (by \snippetref[unique-k-cycle3][this lemma], as otherwise they would be the same).
    We proceed in the same way until we exhaust all letters in the \group.
    The permutation \(\sigma\) is thus the product of \(k\)-cycles found in this way.
    This permutation is unique as every letter can appear only in one \(k\)-cycle.
\end{snippetproof}

\begin{snippetexample}{permutation-as-disjoint-cyclic-product-example}{}
    Write the permutation
    \[
        \sigma = \begin{pmatrix}
            1 & 2 & 3 & 4 & 5 & 6 & 7 \\
            3 & 1 & 4 & 2 & 7 & 6 & 5
        \end{pmatrix}
    \]
    as a product of \disjointperm \(k\)-cycles.
    Take \(1\) and consider its image \(\sigma(1) = 3\).
    Proceed like this with the next images.
    \begin{center}
        % https://tikzcd.yichuanshen.de/#N4Igdg9gJgpgziAXAbVABwnAlgFyxMJZABgBpiBdUkANwEMAbAVxiRAEYQBfU9TXfIRTtyVWoxZsAzN14gM2PASIAmUdXrNWiEABZZfRYKJT14rWxUH5-JUOS6zmyTs5cxMKAHN4RUADMAJwgAWyQyEBwIJHYeAOCwxBFI6MQVOJAg0KQ1FKQpDKzE0zzEXXcuIA
        \begin{tikzcd}
            1 \arrow[r] & 3 \arrow[r] & 4 \arrow[r] & 2 \arrow[r] & 1
        \end{tikzcd}
    \end{center}
    We take the next element that is not being permutates by the last \(k\)-cycle
    \begin{center}
        % https://tikzcd.yichuanshen.de/#N4Igdg9gJgpgziAXAbVABwnAlgFyxMJZABgBpiBdUkANwEMAbAVxiRAFYQBfU9TXfIRQBGclVqMWbAOzdeIDNjwEiAJjHV6zVog7dxMKAHN4RUADMAThAC2SUSBwQkqnhet3EZR88TCuFFxAA
        \begin{tikzcd}
            5 \arrow[r] & 7 \arrow[r] & 5
        \end{tikzcd}
    \end{center}
    and
    \begin{center}
        \begin{tikzcd}
            6 \arrow[r] & 6
        \end{tikzcd}
    \end{center}
    This gives the \(k\)-cycles \(\sigma = \kcycle{1, 3, 4, 2}\kcycle{5, 7}\kcycle{6}= \kcycle{1, 3, 4, 2}\kcycle{5, 7}\).
\end{snippetexample}

\begin{snippetproposition}{product-disjoint-cycles-period}{Period of product of disjoint cycles}
    Let \(\sigma \in \permgrp_n\) where \(\sigma=\rho_1 \rho_2 \cdots \rho_t\)
    where \(\rho_i\) are \disjointperm \(k\)-cycles.
    Then, the period of the permutation is the least common multiple
    of the lengths of the \(k\)-cycles.
\end{snippetproposition}

\begin{snippetproof}{product-disjoint-cycles-period-proof}{product-disjoint-cycles-period}{Period of product of disjoint cycles}
    We already noted that the period of the product of \disjointperm \(k\)-cycles is
    equal to the least common multiple.
\end{snippetproof}

\begin{snippetexample}{period-of-non-disjoint-cycles-example}{Period of non-disjoint \(k\)-cycles}
    Consider \(\sigma = \kcycle{1, 3, 2, 4}\kcycle{5, 4}\kcycle{2, 7, 6}\).
    We need to transform this permutation into a product if \disjointperm \(k\)-cycles.
    \begin{center}
        % https://tikzcd.yichuanshen.de/#N4Igdg9gJgpgziAXAbVABwnAlgFyxMJZABgBpiBdUkANwEMAbAVxiRAEYQBfU9TXfIRTtyVWoxZsAzN14gM2PASIAmUdXrNWiEDJ59FgolPXit07mJhQA5vCKgAZgCcIAWyQiQOCEjVnJHQAKAFYAHTCAbgAWAEpZJ1cPRDJvX0QvBjoAIxgGAAV+JSEQBhhHHBANCW0QIPYIyKlGlUa4hJAXdz9qHyQTANqg1qiAdkaANniuCi4gA
        \begin{tikzcd}
            1 \arrow[r, "\kcycle{1,3,2,4}"] & 3 \arrow[r, "\kcycle{5,4}"] & 3 \arrow[r, "\kcycle{2,7,6}"] & 3
        \end{tikzcd}
    \end{center}
    \begin{center}
        \begin{tikzcd}
            3 \arrow[r] & 2 \arrow[r] & 2 \arrow[r] & 7
        \end{tikzcd}
        \\
        \begin{tikzcd}
            7 \arrow[r] & 7 \arrow[r] & 7 \arrow[r] & 6
        \end{tikzcd}
        \\
        \begin{tikzcd}
            6 \arrow[r] & 6 \arrow[r] & 6 \arrow[r] & 2
        \end{tikzcd}
        \\
        \begin{tikzcd}
            2 \arrow[r] & 4 \arrow[r] & 5 \arrow[r] & 5
        \end{tikzcd}
        \\
        \begin{tikzcd}
            5 \arrow[r] & 5 \arrow[r] & 4 \arrow[r] & 4
        \end{tikzcd}
        \\
        \begin{tikzcd}
            4 \arrow[r] & 1 \arrow[r] & 1 \arrow[r] & 1
        \end{tikzcd}
    \end{center}
    Thus, \(\kcycle{1, 3, 2, 4}\kcycle{5,4}\kcycle{2,7,6} = \kcycle{1,3,7,6,2,5,4}\)
    has a period of \(7\).
\end{snippetexample}

\begin{snippetexercise}{permutations-products-ex1}{}
    Let \(\sigma = \kcycle{1,2,3}\kcycle{4,5}\)
    and \(\tau = \kcycle{1,4}\kcycle{2,5}\). Compute \(\tau(\sigma)\)
    and \(\sigma(\tau)\)
    as a product of \disjointperm \(k\)-cycles.
\end{snippetexercise}

\begin{snippetsolution}{permutations-products-ex1-sol}{}
    We have \(\tau(\sigma) = \kcycle{1,2,3}\kcycle{4,5}\kcycle{1,4}\kcycle{2,5} = \kcycle{1,5}\)
    \begin{center}
        \begin{tikzcd}
            1 \arrow[r] & 2 \arrow[r] & 2 \arrow[r] & 2 \arrow[r] & 5
        \end{tikzcd}
        \\
        \begin{tikzcd}
            5 \arrow[r] & 5 \arrow[r] & 4 \arrow[r] & 1 \arrow[r] & 1
        \end{tikzcd}
        \\
        \begin{tikzcd}
            2 \arrow[r] & 3 \arrow[r] & 3 \arrow[r] & 3 \arrow[r] & 3
        \end{tikzcd}
        \\
        \begin{tikzcd}
            3 \arrow[r] & 1 \arrow[r] & 1 \arrow[r] & 4 \arrow[r] & 4
        \end{tikzcd}
        \\
        \begin{tikzcd}
            4 \arrow[r] & 4 \arrow[r] & 5 \arrow[r] & 5 \arrow[r] & 2
        \end{tikzcd}
    \end{center}
    and \(\sigma(\tau) = \kcycle{1,4}\kcycle{2,5}\kcycle{1,2,3}\kcycle{4,5} = \kcycle{1,5,3}\kcycle{2,4}\).
\end{snippetsolution}

\begin{snippetdefinition}{permutation-type-definition}{Permutation type}
    Let \(\sigma \in \permgrp_n\) where \(\sigma=\gamma_1 \gamma_2 \cdots \gamma_t\)
    where \(\gamma_i\) are \disjointperm \(k\)-cycles of lengths \(r_i \geq r_j\)
    for \(i<j\) where \(r_i >1\). Then, the \emph{type}
    of \(\sigma\) is \((r_1, r_2, \cdots, r_t)\).
    The type of the identity permutation is \(()\).
\end{snippetdefinition}

\begin{snippet}{permutation-amount-of-given-type}
    We want to study the amount of permutations with a given \permtype.
    For the type \((r)\) there are \(\binom{n}{r}\) permutations in \(\permgrp_n\)
    with \(r>1\).
    For \((r_1, r_2, \cdots, r_t)\) where \(r_1>r_2>\cdots r_t>1\) we have
    \[
        \kcycle{a_1,a_2,\cdots,a_{r_1}}\kcycle{a_{r_1+1},\cdots;a_{r_1+r_2}}\cdots
        \kcycle{\cdots,a_{r_1 + r_2+\cdots+r_t}}
    \]
    However, every cycle can be written in different ways (as many as its length).
    So we need to divide the amount by its length
    \[
        \frac{
            n(n-1)\cdots(n-r_1-r_2-\cdots-r_t + 1)
        }{
            r_1r_2\cdots r_t
        }
    \]
    In general, there could be multiple cycles with the same length.
    In this case, the cycles of same length can be ordered in multiple ways.
    If there are \(l\) cycles of the same length, they can be rearanged in \(l!\)
    different ways. Thus, the amount must be divided by the product of these factorials.
    \[
        \frac{
            n(n-1)\cdots(n-r_1-r_2-\cdots-r_t + 1)
        }{
            r_1r_2\cdots r_t \cdot l_1!l_2!\cdots l_f
        }
    \]
    where \(l_i\) are the amounts of cycles with the same length.
\end{snippet}

%\begin{snippet}{permutation-amount-of-given-type-small-n}
%    We study the amount of permutations of a certain \permtype in \(\permgrp_n\)
%    for small \(n\). Since being of the same \permtype is an \equivrelation,
%    we only write a representative of the class.
%    TODO
%    %TODOURGENT.
%    % Sym_1 1
%    % Sym_2 2
%    % Sym_3 6
%    % Sym_4 24
%    % Sym_5 120
%\end{snippet}

\begin{snippetdefinition}{swap-permutation-definition}{Swap permutation}
    A \(k\)-cycle with \(k=2\) is said to be a \emph{swap}.
\end{snippetdefinition}

\begin{snippettheorem}{permutations-as-swap-product-theorem}{Permutations as products of swaps}
    Let \(\sigma \in \permgrp_n\). Then, \(\sigma\) can be written
    as a product of \permswap[swaps].
\end{snippettheorem}

\begin{snippetproof}{permutations-as-swap-product-theorem-proof}{permutations-as-swap-product-theorem}{Permutations as products of swaps}
    We proceed by \principleofinduction[induction] on
    \(\cardinality{\supp(\sigma)}\).
    \begin{enumerate}
        \item the base case is the identity, which is a product of zero \permswap[swaps].
        \item let \(\cardinality{\supp(\sigma)}>0\). There exist \(i\in\{1,2,\cdots, n\}\)
        such that \(i\in\supp(\sigma)\), meaning \(\sigma(i) = j\) with \(i \neq j\).
        We consider \(\tau = \sigma\kcycle{i,j}\). We know that
        \[
            \supp(\tau) \subseteq \supp(\kcycle{i,j}) \union \supp(\sigma)
            = \supp(\sigma) \union \{i, j\}
        \]
        However, \(i\in\supp(\sigma)\) and \(j=\sigma(i) \in \supp(\sigma)\).
        Thus, \(\supp(\tau) \subseteq \supp(\sigma)\).
        Now, \(\tau(i) = \sigma(i)\kcycle{i,j} = i\), and since \(i \in \supp(\sigma)\),
        we have \(\cardinality{\supp(\tau)} < \cardinality{\supp(\sigma)}\).
        By the induction hypothesis, \(\tau\) is a product of \permswap[swaps], meaning
        \(\tau = S_1S_2\cdots S_r\).
        Now, \(\tau{i,j} = \sigma\kcycle{i,j}{i,j}\) and thus \(\tau = S_1S_2\cdots S_r {i,j}\).
    \end{enumerate}
\end{snippetproof}

\begin{snippet}{permutations-as-swap-product-expl}
    For \(r>2\) consider \(\sigma(a_r) = a_1\).
    We explicity have \[
        \sigma\kcycle{a_1, a_r} = \kcycle{a_1, a_2, \cdots, a_r}\kcycle{a_1, a_r}
        = \kcycle{a_1, a_2 \cdots a_{r-1}}
    \]
    and thus \(\sigma = \sigma\kcycle{a_1, a_r}\kcycle{a_1, a_r}\).
    Proceeding by induction we get
    \begin{align*}
        \sigma &= \kcycle{a_1, \cdots, a_{r-2}}\kcycle{a_1, a_{r-1}}\kcycle{a_1, a_r} \\
        &= \kcycle{a_1, a_2}\kcycle{a_1, a_3}\cdots \kcycle{a_1, a_r}
    \end{align*}
\end{snippet}

\begin{snippettheorem}{permutation-swap-products-parity-theorem}{Permutation as swap product parity}
    Let \(\sigma \in \permgrp_n\). Then, \(\sigma\) can be written as a product
    of \permswap[swaps] in infinite ways (for \(r>1\))
    and every way of writing \(\sigma\) has always the same parity.
\end{snippettheorem}

\begin{snippetlemma}{permutation-swap-products-parity-theorem-identity-case}{}
    Let \(S_1, S_2, \cdots, S_r\) be \permswap[swaps]
    such that \(S_1S_2\cdots S_r = \text{Id}\).
    Then, \(r\) is even.
\end{snippetlemma}

\begin{snippetproof}{permutation-swap-products-parity-theorem-identity-case-proof}{permutation-swap-products-parity-theorem-identity-case}{}
    We proceed by \principleofinduction[induction]
    on \(r\).
    \begin{itemize}
        \item the base case is trivial;
        \item let \(r>0\) and \(a\) be a letter appearing in at least one of
            \(S_1, S_2, \cdots, S_r\). Let \(S_h = \kcycle{a, b}\)
            be the first \permswap in which \(a\) appears.
            Note that \(h \neq r\) since it would not be the identity.
            Thus, \(S_h\) is followed by at least another \permswap.
            We consider various cases:
            \begin{enumerate}
                \item \(S_h = S_{h+1}\): in this case we can simplify \(S_h\)
                and \(S_{h+1}\) are a \permswap is the inverse of itself.
                We would have \(\text{Id} = S_1\cdots S_{h-1}S_{h+2}\cdots S_r\)
                which is a product of \(r-2\) \permswap[swaps] and thus \(r\) is even
                by the inductive hypothesis.
                \item \(S_h\) and \(S_{h+1}\) are \disjointperm: in this case we can swap
                the two permutations. Since \(S_h\) is the first containing \(a\),
                if we swap if with \(S_{h+1}\), then the first occurence of \(a\) is moved \(1\) towards the last position
                The letter \(a\) cannot be moved until the last position at it would be a contradiction.
                \item \(S_{h+1} = \kcycle{a, c} \lor S_{h+1} = \kcycle{b, c}\)
                with \(c \neq a\) and \(c \neq b\):
                in this case \[S_hS_{h+1} = \kcycle{a, b}\kcycle{a, c} = \kcycle{b, c}\kcycle{a, b} = \kcycle{a,b,c}\]
                or \[S_hS_{h+1} = \kcycle{a, b}\kcycle{b, c} = \kcycle{b, c}\kcycle{a, c} = \kcycle{a, c, b}\]
                The first occurence of \(a\) is moved \(1\) towards the last position.
                The letter \(a\) cannot be moved until the last position at it would be a contradiction.
            \end{enumerate}
            Note that if we are not in the first case, and cannot simply the \permswap[swaps],
            the first ocucrence of \(a\) can always be moved one step closed to the last index.
            However, \(a\) can never reach the last position as it would be a contradiction,
            meaning that this procedure must terminate before the last index. This means
            that at some point, two \permswap[swaps] must simplify
            and the thesis is given.
    \end{itemize}
\end{snippetproof}

\begin{snippetproof}{permutation-swap-products-parity-theorem-proof}{permutation-swap-products-parity-theorem}{Permutation as swap product parity}
    Let \(\sigma = S_1 S_2 \cdots S_t = C_1 C_2 \cdots C_u\)
    be a product of \permswap[swaps].
    We have that \begin{align*}
        \text{Id} = \sigma^\inversefunction(\sigma) &= \left(S_1S_2\cdots S_t\right)
        {\left(C_1C_2\cdots C_u\right)}^\inversefunction \\
        &= S_1S_2\cdots S_t C_u C_{u-1} \cdots C_n
    \end{align*}
    by \snippetref[permutation-swap-products-parity-theorem-identity-case][this lemma],
    \(t+u\) is even, meaning \(t\) and \(u\) have the same parity.
\end{snippetproof}

\begin{snippetdefinition}{alternating-group-definition}{Alternating group}
    The \emph{alternating group} \(A_n\) is the \group
    of even permutations on \(n\) letters.
\end{snippetdefinition}

\begin{snippetproposition}{alternating-group-order}{Order of alternating group}
    \[
        \cardinality{A_n} = \begin{cases}
            1 & n=1 \\
            \frac{n!}{2} & n>1
        \end{cases}    
    \]
\end{snippetproposition}

\begin{snippet}{amount-of-permutations}
    Half of the permutations are even and half are odd (meaning outside of \(A_n\)).
    Let \(\sigma \in \permgrp_n\) and consider \(\sigma\kcycle{1,2}\).
    Clearly, \(\sigma\kcycle{1,2} \notin A_n\).
    By varying \(\sigma\) in \(A_n\), we get the permutations of type
    \(\sigma\kcycle{1,2}\), which are odd.
    Thus, the odd permutations are at least as many as the even ones.
    Likewise, we cans how that there are at least as many even permutations as the odd ones,
    and thus there are an equal amount of odd and even permutations.
\end{snippet}

\section{Length of k-cycle powers}

\begin{snippet}{permutation-powers-expl}
    Consider \(\sigma \in \permgrp_6\). The powers are given by
    \begin{align*}
        \sigma^0 &= \text{Id} \\
        \sigma^1 &= \kcycle{1,2,3,4,5,6} \\
        \sigma^2 &= \kcycle{1,3,5}\kcycle{2,4,6} \\
        \sigma^3 &= \kcycle{1,4}\kcycle{2,5}\kcycle{3,6} \\
        \sigma^4 &= \kcycle{1,5,3}\kcycle{2,6,4} \\
        \sigma^5 &= \kcycle{1,6,5,4,3,2}
    \end{align*}
\end{snippet}

\begin{snippetproposition}{k-cycle-power-length}{\(k\)-cycle power length}
    Let \(\sigma\) be a \(k\)-cycle.
    Then, \(\sigma^t\) is the product of \(\gcd(k,t)\)
    \disjointperm cycles of length
    \[
        \frac{k}{\gcd(k,t)}
    \]
\end{snippetproposition}

\begin{snippetproof}{k-cycle-power-length-proof}{k-cycle-power-length}{\(k\)-cycle power length}
    Consider \(\sigma = \kcycle{a_1, a_2, \cdots, a_k}\)
    and a power \(\sigma^t\).
    We want to express such power as a product of \disjointperm cycles.
    In such product, only one cycle with length \(l\) will contain \(a_1\).
    However, note that \(\sigma = \kcycle{a_2, a_3, \cdots, a_k, a_1}\),
    and the cycle containing \(a_2\) will also havde length \(l\).
    Thus, all cycles in the product have length \(l\).
    If \(l=1\), then \(\sigma^t = \text{Id}\).
    If \(l\neq 1\), \(\supp(\sigma) = \supp(\sigma^t) = \{a_1, a_2, \cdots, a_k\}\).
    In general, \(|\sigma^t|\) is given by the least common multiple of the lengths
    of the cycles, which is \(l\).
    On the other handle, the period
    \[
        |\sigma^t| = \frac{|\sigma|}{\gcd(|\sigma|, t)} = \frac{k}{\gcd(k, t)}
    \]
    meaning
    \[
        l = \frac{k}{\gcd(k, t)}
    \]
    Since every element of \(\{a_1, a_2, \cdots, a_k\}\) appears in exactly
    one cycle we have that, if \(s\) is the number of cycles, \(r=ls\)
    from which \(s = \gcd(r,t)\).
\end{snippetproof}

\end{document}