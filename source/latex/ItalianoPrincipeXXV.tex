\documentclass[preview]{standalone}

\usepackage{amsmath}
\usepackage{amssymb}
\usepackage{stellar}
\usepackage{definitions}

\begin{document}

\id{italiano-principe-capitolo-xxv}
\genpage

\section{Capitolo XXV}

\begin{snippet}{il-principe-capitolo-xxv}
    % prudenza è virtù
    Il capitolo 25 del Principe di Machiavelli affronta il tema della fortuna e del ruolo che essa gioca nella vita dei principi. Machiavelli discute come molte persone credano che il mondo sia governato dalla fortuna e da Dio, e che gli uomini non possano fare nulla per cambiarlo. Tuttavia, egli sostiene che mentre la fortuna possa influenzare la metà delle azioni umane, l'altra metà è ancora soggetta al controllo umano.
    \\\\
    Machivalli usa l'immagine di un fiume in piena per descrivere la fortuna: quando è arrabbiata, distrugge tutto ciò che trova sul suo cammino, ma durante i periodi di calma, gli uomini possono prendere precauzioni e proteggere se stessi con argini e ripari. Tuttavia, egli nota che l'Italia, che ha vissuto molte vicissitudini, manca di queste difese.
    Il principe che ha successo è colui che \textit{adatta} il proprio comportamento agli eventi della fortuna.
    \\\\
    Secondo Machiavelli, la fortuna dimostra la sua potenza dove non c'è virtù per resistere ad essa. Egli osserva che i principi possono essere felici o infelici a seconda di come il loro modo di agire si adatta ai tempi in cui vivono. Ad esempio, un principe che agisce con ferocia e impeto può avere successo se il momento è propizio, mentre un altro principe che agisce con prudenza e rispetto può fallire se non si adatta ai cambiamenti delle circostanze.
    \\\\
    Machiavelli conclude che è meglio essere impetuosi piuttosto che rispettivi, poiché la fortuna è simile a una donna e si lascia più facilmente vincere da coloro che la affrontano con audacia e ferocia. Infine, nota che la fortuna è amica dei giovani perché sono meno rispettosi e più audaci nel comandarla.
\end{snippet}

\end{document}