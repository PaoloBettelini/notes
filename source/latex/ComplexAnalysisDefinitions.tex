\documentclass[preview]{standalone}

\usepackage{amsmath}
\usepackage{amssymb}
\usepackage{parskip}
\usepackage{fullpage}
\usepackage{hyperref}
\usepackage{wrapfig}
\usepackage{pgfplots}
\usepackage{bettelini}
\usepackage{makecell}
\usepackage{stellar}
\usepackage{definitions}

\makeatletter 
\tikzset{ 
reuse path/.code={\pgfsyssoftpath@setcurrentpath{#1}} 
} 
\tikzset{even odd clip/.code={\pgfseteorule}, 
protect/.code={ 
\clip[overlay,even odd clip,reuse path=#1] 
(current bounding box.south west) rectangle (current bounding box.north east)
; 
}} 
\makeatother

\usetikzlibrary{3d,arrows.meta,decorations.markings,perspective}
\tikzset{->-/.style={decoration={% https://tex.stackexchange.com/a/39282/194703
  markings,
  mark=at position #1 with {\arrow{>}}},postaction={decorate}},
  ->-/.default=0.55}

\begin{document}
  
\id{complexanalysis-definitions}
\genpage

\section{Riemann Spheres}

\begin{snippetdefinition}{riemann-sphere-definition}{Riemann Sphere}
    A \textit{Riemann Sphere} is a unit sphere used to represent the complex plane
    using stereographic projection.
    The Riemann sphere lays on the complex plane. A complex number
    is represented by the intersection between the sphere
    and a ray starting from the topmost point of the sphere
    and intersecting with the given complex number on the complex plane.
\end{snippetdefinition}

\begin{snippet}{riemann-sphere-illustration}
\begin{center}
\pgfmathsetmacro{\myaz}{15}
\begin{tikzpicture}[
        declare function={%
            stereox(\x,\y)=2*\x/(1+\x*\x+\y*\y);
            stereoy(\x,\y)=2*(\y+1)/(1+\x*\x+(\y+1)*(\y+1)) + 1;
            stereoz(\x,\y)=(-1+\x*\x+\y*\y)/(1+\x*\x+\y*\y);
            Px=1.75;
            Py=-1.5;
            Qx=-1.0;
            Qy=-1.0;
            amax=2.5;
        },
        scale=2.5,
        line join=round,line cap=round,
        dot/.style={circle,fill,inner sep=1pt},>={Stealth[length=1.2ex]}]
    \pgfdeclarelayer{background}
    \pgfdeclarelayer{foreground} 
    \pgfsetlayers{background,main,foreground}
    \path[save path=\pathSphere,ball color=gray,fill opacity=0.6] 
        (0, 1, 0) circle[radius=1];
    \begin{scope}[3d view={\myaz}{15}]
        \draw (-amax,amax) -- (-amax,-amax) coordinate (bl) -- (amax,-amax) 
            coordinate (br)-- (amax,amax);
        \begin{scope}
            \tikzset{protect=\pathSphere}
            \draw (-amax,amax) -- (amax,amax);
        \end{scope}
        \begin{scope}
            \clip[reuse path=\pathSphere];
            \draw[dashed] (-amax,amax) -- (amax,amax);
        \end{scope}
        \begin{scope}[canvas is xy plane at z=0]
            \path[save path=\pathPlane] (\myaz:amax) -- (\myaz+180:amax) --(bl) -- (br) -- cycle;
            \begin{pgfonlayer}{background}
                \fill[blue!30,fill opacity=0.6]
                    (\myaz:1) arc[start angle=\myaz,end angle=\myaz-180,radius=1]
                    -- (-amax,0) -- (-amax,amax) -- (amax,amax) -- (amax,0) -- cycle;
            \end{pgfonlayer}
            \fill[blue!30,fill opacity=0.6]
                (\myaz:1) arc[start angle=\myaz,end angle=\myaz-180,radius=1]
                -- (-amax,0) -- (-amax,-amax) -- (amax,-amax) -- (amax,0) -- cycle;
        \end{scope}

        %\draw[-] (0,-2,0) node[dot,label=below:{$z$}](z){}
        %    -- node[auto,pos=0.5]{\(p\)} (0, -1, 1)
        %    node[dot,label=below right:{\(s\)}](z*){};

        \draw[-] (-2,0,0) node[dot,label=below:{$z$}](z){}
            -- node[auto,pos=0.5]{} (-1, 0, 1)
            node[dot,label=below right:{\(s\)}](z*){};
  
        \begin{pgfonlayer}{background} 
            \draw[dashed] (z*) -- (0,0,2) node[dot,label=above:{$\zeta$}](zeta){};
        \end{pgfonlayer}
    \end{scope}
\end{tikzpicture}
\end{center}
\end{snippet}

\section{Subsets of the complex plane}

\begin{snippetdefinition}{open-disk-definition}{Open Disk}
    An open disk \(D_\delta(z_0)\) is the set of points
    with distance less than \(\delta\) from \(z_0\)
    \[
        D_\delta(z_0)=\{z\in \complexnumbers \suchthat |z-z_0|<\delta\}
    \]
\end{snippetdefinition}

\begin{snippetdefinition}{closed-disk-definition}{Closed Disk}
    A closed open disk \(D_\delta(z_0)\) is the set of points
    with distance less than or equal to \(\delta\) from \(z_0\)
    \[
        \overline{D_\delta(z_0)}=\{z\in \complexnumbers \suchthat |z-z_0|\leq\delta\}
    \]
\end{snippetdefinition}

\begin{snippetdefinition}{circle-definition}{Circle}
    A circle \(C_\delta(z_0)\) is the set of points
    with distance equal to \(\delta\) from \(z_0\)
    \[
        C_\delta(z_0)=\{z\in \complexnumbers \suchthat |z-z_0|=\delta\}
    \]
\end{snippetdefinition}

\begin{snippetdefinition}{interior-point-definition}{Interior Point}
    A point \(z\) is an interior point of \(\Omega\) iff there is an open disk
    at \(z\) whose point are in \(\Omega\)
    \[
        \exists D_{r>0}(z) \subset \Omega
    \]
\end{snippetdefinition}

\begin{snippetdefinition}{boundary-point-definition}{Boundary Point}
    \(z\) is a boundary point of \(\Omega\) iff every open disk at \(z\)
    contains points both in \(\Omega\) and not in \(\Omega\).
\end{snippetdefinition}

\begin{snippetdefinition}{exterior-point-definition}{Exterior Point}
    The point \(z\) is an exterior point of \(\Omega\) iff it is not a boundary
    point of an interior point.
\end{snippetdefinition}

\begin{snippetdefinition}{accumulation-point-definition}{Accumulation Point}
    \(z\) is an \textit{accumulation point} or limit point of \(\Omega\) if any
    \(D_\delta(z)\difference \{z\}\) always contains points of \(\Omega\).
    \\
    In order to always contain points of \(\Omega\), \(\Omega\) must have an
    infinite amount of points, since \(\delta\) can be as little as we want.
\end{snippetdefinition}

\begin{snippetdefinition}{open-set-definition}{Open Set}
    A set \(\Omega\) is \textit{open} if all points in \(\Omega\)
    are interior points of \(\Omega\).
\end{snippetdefinition}

\begin{snippetdefinition}{closed-set-definition}{Closed Set}
    A set \(\Omega\) is \textit{closed} if every accumulation point of \(\Omega\)
    is in \(\Omega\). 
\end{snippetdefinition}

\begin{snippetdefinition}{bounded-set-definition}{Bunded Set}
    A set \(\Omega\) is \textit{bounded} if
    \[
        \exists M>0 \suchthat \Omega \subset D_M(0)
    \]
    In other words there must exist an \(M>0\) such that \(\forall z\in \Omega:|z|<M\)
\end{snippetdefinition}

\begin{snippetdefinition}{connected-set-definition}{Connected Set}
    An open set \(\Omega\) is connected if it cannot be written as
    \(\Omega=\Omega_1 \union \Omega_2\) where \(\Omega_1 \intersection \Omega_2 = \emptyset\).
    In other words any two points in \(\Omega\) must be connectable by a continuous
    curve where all the points of the curve are also in \(\Omega\).
\end{snippetdefinition}
% 1 2 3 4 5 6 7 8 del libro

\end{document}