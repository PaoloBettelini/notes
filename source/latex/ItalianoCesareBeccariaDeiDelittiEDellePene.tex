\documentclass[preview]{standalone}

\usepackage{amsmath}
\usepackage{amssymb}
\usepackage{stellar}

\hypersetup{
    colorlinks=true,
    linkcolor=black,
    urlcolor=blue,
    pdftitle={Stellar},
    pdfpagemode=FullScreen,
}

\begin{document}

\title{Stellar}
\id{cesare-beccaria-del-delitti-e-delle-pene}
\genpage

\section{Dei delitti e delle pene}

\begin{snippetdefinition}{dei-delitti-e-delle-pene-definition}{Dei delitti e delle pene}
    \textit{Dei delitti e delle pene} è un pamphlet del 1764 circa
    le modalità di accertamento dei delitti e circa le pene allora in uso.
    Cesare Beccaria affronta temi come la pena di morta senza trattarne la moralità, bensì puramente
    da un punto di vista utilitista.
\end{snippetdefinition}

\begin{snippet}{dei-delitti-e-delle-pene-introduzione}
    Questo testo rivoluzionario nasce da confronti e dibattiti circa la giurisprudenza criminale.
    Per sottrarre il libro alla censura, Pietri Verri manda il libro in Toscana
    ad essere stampato (stato più progressista in Italia),
    per far sì che il libro cominci a circolare prima fuori da Milano.
    Il libro viene pubblicato senza il nome dell'autore per tutela.
    \\\\
    Il libro vuole rovesciare la concezione tradizionale che lascia poca distinzione fra giustizia e vendetta,
    come la legge del taglione.
    \\\\
    Essenzialmente, Beccaria indica la presenza di leggi di Dio (peccati), leggi di natura e leggi dell'uomo (reati).
\end{snippet}

\end{document}