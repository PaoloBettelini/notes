\documentclass[preview]{standalone}

\usepackage{amsmath}
\usepackage{amssymb}
\usepackage{tikz}
\usepackage{stellar}
\usepackage{definitions}
\usepackage{bettelini}

\begin{document}

\id{geofisica-terremoti}
\genpage

% http://www.seismo.ethz.ch/it/home/

\section{Terremoti}

\plain{Le rocce si deformano elasticamente nel tempo e quando si rompono,
rimbalzano generando un terremoto.}

\begin{snippetdefinition}{faglia}{Faglia}
    Una \textit{faglia} è una frattura avvenuta entro
    un volume di roccia della crosta terrestre che mostra evidenze di
    movimento relativo tra le due masse rocciose da essa divise.
\end{snippetdefinition}

\begin{snippetdefinition}{ipocentro}{Ipocentro}
    Con \textit{ipocentro} si indica il cuore del terremoto, in profondità.
\end{snippetdefinition}

\begin{snippetdefinition}{epicentro}{Epicentro}
    Con \textit{epicentro} si indica il punto della superficie terrestre posto esattamente sopra l'ipocentro.
\end{snippetdefinition}

\plain{Le onde sismiche partono dall'epicentro, dove il terremoto ha generalmente un'intensità maggiore.}

\begin{snippetdefinition}{onde-di-volume}{Onde di volume}
    Le \textit{Onde di verificaolume} sono quelle onde che si propagano dalla sorgente sismica,
    attraverso il volume del mezzo interessato, in tutte le direzioni.
\end{snippetdefinition}

\begin{snippet}{onde-di-volume-tipi}
    Vi sono due tipi di onde:
    \begin{itemize}
        \item \textbf{onde P} (prime), onde compressionali o longitudinali;
        \item \textbf{onde S} (seconde), onde trasversali.
    \end{itemize}
\end{snippet}

\plain{Questi due tipi di onde hanno velocità diverse, per cui il sismografo comincierà a misurarne
prima una. Avendo molteplici sismografi è possibile localizzare l'epicentro di un terremoto.}

\plain{La scala Mercalli permette di misurare l'intensità, mentre la scala Richter misura l'energia/forza di un terremoto.
Chiaramente, la scala Richter è indipendente dal punto di misurazione, mentre la scala Mercalli cambia dal punto di misurazione.
}

\includesnpt[width=90\%|src=/snippet/static/onde-terremoto.png]{centered-img}

\plain{Quando le onde S colpiscono la superficie, vengono generate
le onde superficiali di Rayleigh e le onde di Love. Le onde di Love
vengono generate solo nei mezzi in cui la velocità delle
Onde S aumenta con la profondità (quindi siamo in presenza di un mezzo disomogeneo).}

\end{document}