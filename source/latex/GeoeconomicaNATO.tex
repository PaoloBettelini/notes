\documentclass[preview]{standalone}

\usepackage{amsmath}
\usepackage{amssymb}
\usepackage{stellar}

\hypersetup{
    colorlinks=true,
    linkcolor=black,
    urlcolor=blue,
    pdftitle={Stellar},
    pdfpagemode=FullScreen,
}

\begin{document}

\title{Stellar}
\id{geoeconomica-nato}
\genpage

\section{La NATO}

\begin{snippetdefinition}{nato-definition}{Nato}
    La \textit{NATO} (North Atlantic Treaty Organization) è un associazione di sicurezza composta da diverse nazioni.
\end{snippetdefinition}

\begin{snippet}{nato-expl}
    La NATO venne fondata nel 1949 come alleanza militare
    tra i paesi occidentali a scopo di contrastare le minacce sovietiche.
    I paesi che aderirono all'alleanza furono:
    \begin{itemize}
        \item Stati Uniti d'America;
        \item Italia;
        \item Portogallo;
        \item Germania federale;
        \item Grecia;
        \item Turchia;
        \item Spagna.
    \end{itemize}
    La NATO venne istituita con l'obiettivo di stabilire un comando militare unificato e di implementare politiche
    di difesa concertate a livello globale. La sua formazione rappresentò una risposta diretta al blocco orientale e
    divenne un pilastro fondamentale della strategia di contenimento del comunismo durante la Guerra fredda.
\end{snippet}

\end{document}