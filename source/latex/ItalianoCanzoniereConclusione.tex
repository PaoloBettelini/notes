\documentclass[preview]{standalone}

\usepackage{amsmath}
\usepackage{amssymb}
\usepackage{stellar}

\hypersetup{
    colorlinks=true,
    linkcolor=black,
    urlcolor=blue,
    pdftitle={Stellar},
    pdfpagemode=FullScreen,
}

\begin{document}

\title{Stellar}
\id{italiano-canzoniere-conclusione}
\genpage

\section{Conclusione}

\begin{snippet}{italiano-canzoniere-conclusione}
    Le poesie hanno i seguenti temi:
    \begin{itemize}
        \item \textbf{Rvf 1} prologo o epilogo;
        \item \textbf{Rvf 3} innamoramento;
        \item \textbf{Rvf 16} tema dell'allontanaza e della ricerca;
        \item \textbf{Rvf 22} sofferenza nel tempo;
        \item \textbf{Rvf 90} descrizione di Laura;
        \item \textit{foglio vuoto}
        \item \textbf{Rvf 264} (canzone, non sonetto) fine amore;
        \item \textbf{Rvf 267} annuncio implicito della morte di Laura;
        \item \textbf{Rvf 268} annuncio esplicito della morte.
    \end{itemize}
    
    In \textbf{Rvf 1}, l'amore inizia il 6 aprile (Venerdì Santo, morte di Cristo),
    mentre l'amore si chiude il 25 dicembre (Natale, nascita di Cristo).
    Al tempo di Petrarca questo giorno era il primo dell'anno. 
    
    La discrepanza fra la posizione del foglio bianco e la morte di Laura rappresenta il
    fatto che il distacco dell'autore dalla prima parte
    avviene autonomamente prima della morte di Laura.
\end{snippet}

\end{document}