\documentclass[preview]{standalone}

\usepackage{amsmath}
\usepackage{amssymb}
\usepackage{parskip}
\usepackage{fullpage}
\usepackage{hyperref}
\usepackage{stellar}
\usepackage{definitions}

\begin{document}

\id{taylor-series}
\genpage

\section{Prerequisites}

\begin{snippetproposition}{power-rule-nth-derivative}{\(n\)-th derivative of \(x^n\)}
    The \(n\)-the derivative of \(x^n\) for \(n \in {\naturalnumbers}^*\)
    is given by
    \[
        \derivativeD^n x^n = n\factorial
    \]
\end{snippetproposition}

\begin{snippetproof}{power-rule-nth-derivative-proof}{power-rule-nth-derivative}{\(n\)-th derivative of \(x^n\)}
    \begin{align*}
        \frac{d^n}{dx^n}\left(x^n\right)
        &=\frac{d^{(n-1)}}{dx^{(n-1)}}\left(nx^{(n-1)}\right)\\
        &=\frac{d^{(n-2)}}{dx^{(n-2)}}\left(n(n-1)x^{(n-2)}\right)\\
        &=\cdots\\
        &=\prod_i^n i=n\factorial
    \end{align*}
\end{snippetproof}

\begin{snippetlemma}{centered-polynomial-derivative}{Centered polynomial derivative}
    Let
    \[
        P(x) = \sum_{k=0}^n a_k{(x-x_0)}^k
    \]
    be a \polynomial where \(\polynomialdeg P(x) \leq n\) in \(x-x_0\).
    Then,
    \[
        \derivativeD^m P(x_0) = \sum_{k=0}^n a_k \derivativeD^m {(x-x_0)}^k
        = \begin{cases}
            0 & m \neq k \\
            m\factorial = k\factorial & m = k
        \end{cases}
    \]
\end{snippetlemma}

\section{Introduction}

\begin{snippet}{taylor-series-introduction-derivation}
    We want to construct a power series centered around the point \(x=a\),
    so the variable will be \(x-a\).
    \begin{align*}
        \sum_{n=0}^{\infty}c_n{(x-a)}^n
    \end{align*}
    The goal is to find the coefficients \(c_n\).
    We first notice that \(f(a)=c_0\), which is the only coefficient that does not multiply a variable.
    If we take the derivative, the coefficient \(c_1\) will lose its variable
    \begin{align*}
        c_1+2c_2(x-a)+3c_3{(x-a)}^2+\cdots
    \end{align*}
    Now we have \(f'(a)=c_1\).
    \\
    We take the derivative of the polynomial again
    \begin{align*}
        2c_2+6c_3(x-a)+\cdots
    \end{align*}
    This time we have
    \begin{align*}
        f''(a)&=2c_2\\
        c_2&=\frac{f''(a)}{2}
    \end{align*}
    And again
    \begin{align*}
        f'''(a)&=6c_3\\
        c_3&=\frac{f'''(a)}{6}
    \end{align*}

    More generally, to extract the coefficient \(c_n\) we take the n-th derivative of the function. \\
    This brings us to
    \begin{align*}
        c_n=\frac{f^{(n)}(a)}{n\factorial}
    \end{align*}
    and finally to
    \begin{align*}
        \sum_{n=0}^{\infty}\frac{{(x-a)}^n f^{(n)}(a)}{n\factorial}
    \end{align*}
    which might not converge and not coincide with the function.
\end{snippet}

\section{Taylor Polynomial}

\begin{snippetdefinition}{taylor-polynomial-definition}{Taylor polynomial}
    
\end{snippetdefinition}

\begin{snippetdefinition}{maclaurin-polynomial-definition}{MacLaurin polynomial}
    
\end{snippetdefinition}

\end{document}