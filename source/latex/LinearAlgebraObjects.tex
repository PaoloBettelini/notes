\documentclass[preview]{standalone}

\usepackage{amsmath}
\usepackage{amssymb}
\usepackage{stellar}
\usepackage{definitions}
\usepackage{pgfplots}
\pgfplotsset{compat=1.16,width=7cm,height=11cm}

\begin{document}

\id{linearalgebra-objects}
\genpage

\section{Plane}

\begin{snippetdefinition}{plane-definition}{Plane}
    A \textit{plane} is a flat two-dimensional surface that extends indefinitely.
\end{snippetdefinition}

\begin{snippet}{3d-plane-equations}
    A plane can be uniquely represented by its
    normal vector \(\vec{n}\)
    and a point on the plane \(P_0=(x_0, y_0, z_0)\).

    To describe the plane using an equation, we can
    consider an arbitrary point \(P=(x,y,z)\) on the plane.
    There is always a 90 degrees angle between the normal
    vector and the vector from \(P_0\) to \(P\) (i.e., their dot product is zero)
    \[
        \vec{n} \cdot \overrightarrow{P_0 P} = 0
    \]
    By plugging in the values for \(\vec{n}\) and \(P_0\)
    we get an equation in the form
    \[
        a(x-x_0) + b(y-y_0) + c(z-z_0) = 0
    \]
    or
    \[
        Ax+By+Cz+D=0
    \]
\end{snippet}

\section{Level curve}

% TODO level set

\begin{snippetdefinition}{level-curve-definition}{Level curve}
    A \textit{level curve} or \textit{contour curve}
    is the \set of values denoting the equation
    of a real-valued function with two independent variables
    when the function takes on a given constant.
\end{snippetdefinition}

\includesnpt{level-curve}

% TODO ellipsoid, cone, cylinder, Hyperboloid of One Sheet, Two, Elliptic Paraboloid
% Hyperbolic Paraboloid
% https://tutorial.math.lamar.edu/Classes/CalcIII/QuadricSurfaces.aspx

\end{document}
