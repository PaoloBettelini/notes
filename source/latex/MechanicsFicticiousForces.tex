\documentclass[preview]{standalone}

\usepackage{amsmath}
\usepackage{amssymb}
\usepackage{stellar}
\usepackage{definitions}

% https://tex.stackexchange.com/questions/120029/how-to-typeset-a-primed-vector
\newcommand{\pvec}[1]{\vec{#1}\mkern2mu\vphantom{#1}}

\begin{document}

\id{ficticious-forces}
\genpage

\section{Ficticious forces}

\begin{snippetdefinition}{ficticious-force-definition}{Ficticious force}
    A \emph{fictitious force} is a force that appears to act on a mass whose
    motion is described using a non-inertial frame of reference.
    The force is given by
    \[
        \vec{F}_\text{fic} = -m\vec{a} = -m \left[\vec{A} + \pvec{a}' + 2 \vec{\omega} \wedge \pvec{v}' + \frac{d\vec{\omega}}{dt} \wedge \pvec{r}'
        + \vec{\omega} \wedge (\vec{\omega} \wedge \pvec{r}') \right]
    \]
    such that
    \[
        \vec{F} + \vec{F}_\text{fic} = m \pvec{a'}
    \]
\end{snippetdefinition}

\begin{snippet}{ficticious-force-expl}
    In this way, the observer of the non-inertial system can use Newton's laws,
    provided that apparent forces are added in addition to the real forces.
    In this case, the forces are not responsible, but only because the reference system
    is not inertial.
    The Coriolis and centrifugal forces retain the same name.
\end{snippet}

\begin{snippetexample}{ficticious-force-example-1}{Ficticious force}
    We mount a pulley with a weight on the ceiling of a lift that is ascending with acceleration \(A\).
    We pull the rope downwards to make the weight rise with a constant velocity
    (relative to us).
    The weight is affected by the weight force \(-mg\) and the tension of the string \(T\).
    Since I am describing the weight in a non-inertial reference system, I will have
    an apparent force acting on it, i.e. \(-mA\).
    Hence
    \[
        F = -mg + T - mA = 0
    \]
    as the motion moves in uniform rectilinear motion in my system.
    Therefore the tension of the wire
    \[
        T = m(g+A)
    \]
    (it is subjected to a higher tension due to the accelerated motion of the lift).
    Therefore, due to the upward acceleration you have to exert more force to make the weight rise.
\end{snippetexample}

\begin{snippetexample}{ficticious-force-example-2}{Ficticious force}
    Let us consider an observer \(O'\) attached to a rod rotating with a certain angular velocity \(\omega\).
    On the rod, we place a sliding ring initially at a certain distance
    \(l_0\) from the origin.
    Why does the ring move away from the origin?
    Apart from the forces that constrain the ring to stay on the rod, the non-inertial observer
    in solidarity with the rod, sees that an apparent (centrifugal) force is acting on the body.
    This apparent centrifugal force is directed in the direction away from the origin
    while standing on the rod. Its modulus is given by
    \[
        m\omega^2 x
    \]
    where \(x\) is the distance.
    For the observer
    \[
        m \frac{d^2x}{dt^2} = m\omega^2 x
    \]
    which is reminiscent of the harmonic oscilator equation. The solution is
    \[
        x(t) = C_1e^{\omega t} + C_2e^{-\omega t}
    \]
    The initial conditions of the loop are \(x(0)=l_0\)
    and \(v(0) = 0\).
    Thus,
    \[
        \begin{cases}
            C_1 + C_2 = l_0 \\
            0 = \omega(C_1 - C_2)
        \end{cases}
        \implies a = b = \frac{l_0}{2}
    \]
    and then
    \[
        x(t) = l_0 \cosh(\omega t)
    \]
\end{snippetexample}

\end{document}