\documentclass[preview]{standalone}

\usepackage{amsmath}
\usepackage{amssymb}
\usepackage{stellar}
\usepackage{definitions}

\begin{document}

\id{mechanics-momentum}
\genpage

\section{Momentum of a vector}

\begin{snippetdefinition}{vector-momentum-definition}{Momentum of a vector}
    Let \(\vec{a}\) be a vector, \(P\) an \emph{application point}
    and \(O\) the \emph{pole}. Then, the \emph{momentum} is defined as
    \[
        \vec{M} \triangleq \vec{OP} \wedge \vec{a}
    \]
\end{snippetdefinition}

\begin{snippetdefinition}{angular-momentum-definition}{Angular momentum}
    \emph{Angular momentum} is defined as the momentum of the quantity of motion
    \[
        \vec{L} \triangleq \vec{r} \wedge \vec{Q} = \vec{r} \wedge m\vec{v}
    \]
\end{snippetdefinition}

\begin{snippetproposition}{derivative-of-angular-momentum}{Derivative of angular momentum}
    The derivative of angular momentum is the momentum of the force
    \begin{align*}
        \frac{d\vec{L}}{dt} = \vec{r} \wedge \vec{F} 
    \end{align*}
\end{snippetproposition}

\begin{snippetproof}{derivative-of-angular-momentum-proof}{derivative-of-angular-momentum}{Derivative of angular momentum}
    The derivative of angular momentum is given by
    \begin{align*}
        \frac{d\vec{L}}{dt} &= \vec{v} \wedge m\vec{v} + \vec{r} m\wedge{a} \\
        &= \vec{r} m\wedge{a} \\
        &= \vec{r} \wedge \vec{F} \\
        &= \vec{M}
    \end{align*}
    which is the momentum of the force \(\vec{F}\).
\end{snippetproof}

\begin{snippetproposition}{angular-momentum-of-central-force}{Angular momentum of central force}
    The angular momentum of a central force is null.
\end{snippetproposition}

\begin{snippetproof}{angular-momentum-of-central-force-proof}{angular-momentum-of-central-force}{Angular momentum of central force}
    By definition, the vector force and the direction are orthogonal, and thus the product is null.
\end{snippetproof}

\begin{snippetcorollary}{angular-momentum-conservation-central-force}{Angular momentum of central forces conservation}
    The angular momentum \(\vec{L}\) of a central force is conserved (both in module and direction).
    Since the direction is conserved, the orbit is always planar.
\end{snippetcorollary}

\begin{snippetproof}{angular-momentum-conservation-central-force-proof}{angular-momentum-conservation-central-force}{Angular momentum conservation}
    Since the angular momentum of a central force is null,
    \[
        \frac{d\vec{L}}{dt} = 0
    \]
    implies that \(\vec{L}\) is conserved.
\end{snippetproof}

% \plain{Il momento di una forza centrale è nulla}

\end{document}