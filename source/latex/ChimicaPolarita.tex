\documentclass[preview]{standalone}

\usepackage{amsmath}
\usepackage{amssymb}
\usepackage{bettelini}
\usepackage{stellar}
\usepackage[version=4]{mhchem}

\hypersetup{
    colorlinks=true,
    linkcolor=black,
    urlcolor=blue,
    pdftitle={Chimica},
    pdfpagemode=FullScreen,
}

\begin{document}

\title{Chimica}
\id{chimica-polarita}
\genpage

\begin{snippetdefinition}{polarita-definizione}{Polarità}
    Una molecola è \textit{polare} (non pura) se vi è una carica parziale.
\end{snippetdefinition}

\begin{snippet}{polarita-expl1}
    Il legale ionico è quello più polare perché strappa un elettrone. \\
    La differenza di elettronegatività deve essere da 0 a 0.45 per essere puro
    (il valore 0.45 è scelto per considerare il legame CH come apolare).

    Quando una molecola è fatta solo da 2 atomi, 
    se il legame è polare, la molecola è polare.
    Quando ci sono più legami, è necessario almeno un legame polare
    ma la molecola non deve essere simmetrica, altrimenti le cariche parziali si annullano.

    Le sostanze apolari si sciolgono in solventi apolari, e quelli polari in quelli polari.
    Di conseguenza, oer essere solubile in acqua una moecola deve essere polare.
\end{snippet}

\end{document}
