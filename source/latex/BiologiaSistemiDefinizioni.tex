\documentclass[preview]{standalone}

\usepackage{amsmath}
\usepackage{amssymb}
\usepackage{stellar}
\usepackage{bettelini}

\hypersetup{
    colorlinks=true,
    linkcolor=black,
    urlcolor=blue,
    pdftitle={Biologia},
    pdfpagemode=FullScreen,
}

\begin{document}

\title{Biologia}
\id{biologia-sistemi-definizioni}
\genpage

\section{Definizioni}

\begin{snippetdefinition}{sistema-definition}{Sistema}
    Un \textit{sistema} (vivente e non-vivente) è composto di parti differenti, specializzate e interdipendenti. 
    
    \begin{enumerate}
        \item Organizzazione della relazione fra le parti
        \item Struttura fisica, chimica etc. 
        \item Processo di riproduzione
    \end{enumerate}
\end{snippetdefinition}

\begin{snippetdefinition}{emergenza-sistemica-definition}{Emergenza Sistemica}
    Una \textit{emergenza sistemica} è lo scopo che le diverse parti riescono a raggiungere ed eseguire.
\end{snippetdefinition}

\plain{L'emergenza sistemica stabilisce anche delle relazioni materiali, energetiche e/o informazionali con l'esterno, cioè con l'ambiente.}

\begin{snippetdefinition}{molecola-organica-definition}{Molecola organica}
    Una \textit{molecola organica} è una molecola che contiene in generale il carbonio (e.g tranne CO\({}_2\)).
\end{snippetdefinition}

\end{document}