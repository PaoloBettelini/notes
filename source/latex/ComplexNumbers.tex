\documentclass[preview]{standalone}

\usepackage{amsmath}
\usepackage{amssymb}
\usepackage{parskip}
\usepackage{fullpage}
\usepackage{hyperref}
\usepackage{tikz}
\usepackage{stellar}
\usepackage{definitions}
\usepackage{bettelini}

\begin{document}

\id{complexnumbers}
\genpage

\section{Complex Numbers}

\subsection{Definition}

\begin{snippetdefinition}{complex-numbers-definition}{Complex numbers}
    The set of \textit{complex numbers} is defined as
    \[
    \i
        \mathbb{C} = \{ a+bi \suchthat a,b\in\realnumbers \}
    \]
    where \(i\) is defined as \(i^2=-1\), called the \textit{imaginary unit}.
\end{snippetdefinition}

\begin{snippet}{imaginary-unit-negative-version}
    The equation \(x^2 = -1\) has two solutions in \(\complexnumbers\): \(i\) and \(-i\), however,
    there is not any algebraic difference between these two solutions.
    You can choose \(i\) to arbitrarily be one of this solution as long as it is consistent.
\end{snippet}

\subsection{Properties of imaginary unit}

\begin{snippet}{imaginary-unit-exponentiations}
    \[
        \begin{cases}
            i^0=+1\\
            i^1=+i\\
            i^2=-1\\
            i^3=-i\\
        \end{cases}
        \quad
        \begin{cases}
            i^4=+1\\
            i^5=+i\\
            i^6=-1\\
            i^7=-i\\
        \end{cases}
        \quad
        \cdots
    \]
    \phantom{}
\end{snippet}

\begin{snippetproposition}{real-numbers-are-complex}{Real numbers are complex}
    \[\realnumbers \subset \complexnumbers\]
\end{snippetproposition}

\begin{snippetproof}{real-numbers-are-complex-proof}{real-numbers-are-complex}{Real numbers are complex}
    For any number \(r \in \realnumbers\), \(r\)
    is equivalent to \(r + 0i\) and thus \(\realnumbers \subseteq \complexnumbers\).
    However, there are numbers in \(\complexnumbers\) that are not in \(\realnumbers\),
    meaning \[\realnumbers \subset \complexnumbers\]
\end{snippetproof}

\subsubsection{Real part}

\begin{snippetdefinition}{complex-real-part-definition}{Real Part}
    The real part of a complex number \(s\) is denoted as \(\text{Re}(s)\) or \(\origRe(s)\)
    and is defined as.
    \[
        \origRe(a+bi) = a
    \]
\end{snippetdefinition}

\begin{snippetdefinition}{complex-imaginary-part-definition}{Imaginary Part}
    The imaginary part of a complex number \(s\) is denoted as \(\text{Im}(s)\) or \(\origIm(s)\)
    and is defined as.
    \[
        \origIm(a+bi) = b
    \]
\end{snippetdefinition}

\subsection{Complex plane}

\begin{snippet}{complex-plane}
We can represent each complex number on a plane (Argand plane), where the horizontal axis
represent the real numbers \(\realnumbers\) and the vertical axis represents
every scalar multiple of the imaginary unit \(i\).
\end{snippet}

%\begin{snippet}{complexplane-illustration}
%\begin{center}
%    \begin{tikzpicture}
%        \begin{scope}[thick,font=\scriptsize]
%
%            \draw [->] (-5,0) -- (5,0) node [above left]  {\(\Re(s)\)};
%            \draw [->] (0,-5) -- (0,5) node [below right] {\(\Im(s)\)};
%
%            \draw (0,0) -- (0,0)   node [above right] {\(0\)};
%            \foreach \n in {-4,...,-1,1,2,...,4}{
%                \draw (\n,-3pt) -- (\n,3pt) node [above] {\(\n\)};
%                \draw (-3pt,\n) -- (3pt,\n) node [right] {\(\n i\)};
%            }
%
%            \draw [color=black, fill=black] (3,2) circle (0.05) node [above] {\(3+2i\)};
%        \end{scope}
%    \end{tikzpicture}
%\end{center}
%\end{snippet}

\includesnpt{complexplane}

\subsection{Operations}

\begin{snippet}{complex-numbers-addition}
Let \(a, b \in \realnumbers\)
\[
    (a+bi)+(c+di)=a+bi+c+di=(a+c)+(b+d)i
\]
\end{snippet}

\begin{snippet}{complex-numbers-subtraction}
Let \(a, b \in \realnumbers\)
\[
    (a+bi)-(c+di)=a+bi-c-di=(a-c)+(b-d)i
\]
\end{snippet}

\begin{snippet}{complex-numbers-multiplication}
Let \(a, b \in \realnumbers\)
\[
    (a+bi)(c+di)=ac+adi+bci+bdi^2=(ac-db)+(ad+bc)i
\]
\end{snippet}

\begin{snippet}{complex-numbers-division}
Let \(a, b \in \realnumbers\)
\begin{align*}
    \frac{a+bi}{c+di} &= \frac{a+bi}{c+di} \cdot \frac{c-di}{c-di} = \frac{ac-adi+bci-bdi^2}{c^2-d^2i^2}
    \\ &= \frac{ac+bd+(bc-ad)i}{c^2+d^2} \\ &= \frac{ac+bd}{c^2+d^2} + \frac{bc-ad}{c^2 + d^2}i
\end{align*}
\end{snippet}

\subsubsection{Absolute value}

\begin{snippetdefinition}{complex-numbers-abs}{Absolute value of complex number}
    The \textit{absolute value} (or \textit{module}) of a complex number \(s = a+bi\) defined as its distance from the origin.
    \[
        |s| = |a+bi| = \sqrt{a^2 + b^2}
    \]
\end{snippetdefinition}

\subsubsection{Conjugate}

\begin{snippetdefinition}{complex-numbers-conj}{Complex Conjugate}
    The \textit{complex conjugate} of a complex number \(s=a+bi\) is denoted as \(s^*\) or \(\overline{s}\).
    It is defined as
    \[
        \overline{a+bi} = a-bi
    \]
\end{snippetdefinition}

\begin{snippet}{complex-numbers-conj-geometrical-expl}
Geometrically, \(s^*\) is the reflection about the real axis in the complex plane.
\end{snippet}

\begin{snippet}{complex-numbers-conj-properties}
The conjugate has the following trivial properties
\begin{align*}
    \overline{\overline{s}} &= s
    \\
    \text{Re}(\overline{s}) &= \text{Re}(s)
    \\
    \text{Im}(\overline{s}) &= -\text{Im}(s)
\end{align*}
\end{snippet}

\subsubsection{Argument}

\begin{snippetdefinition}{complex-numbers-arg}{Complex argument}
    The \textit{argument} of a complex number is the angle formed with the x-axis in
    the complex plane
    \[
        \arg(a+bi)= \arctan \left(\frac{b}{a}\right)
    \]
\end{snippetdefinition}

\subsubsection{Axiomatic definition}

\begin{snippetaxiom}{complex-number-axiomatic-def}{Complex Number}
    A complex number is a tuple \((a,b)\) where \(a,b\in \realnumbers\).
    
    \begin{enumerate}
        \item \textbf{Equality:} \( (a,b) = (c,d) \iff a=c \land b=d \)
        \item \textbf{Addition:} \((a,b) + (c,d) = (a+c, b+d)\)
        \item \textbf{Multiplication:} \((a,b) \cdot (c,d) = (ac-db, ad+bc)\)
        \item \textbf{Scalar Multiplication:} \(m(a,b) = (ma, mb), \quad m \in \realnumbers\)
    \end{enumerate}
    
    Let \(z_1, z_2, z_3 \in \complexnumbers\)
    \begin{enumerate}
        \item \(z_1+z_2\) and \(z_1z_2\) are also in \(\complexnumbers\)
        \item \(z_1+z_2=z_2+z_1\)
        \item \(z_1 + (z_2 + z_3) = (z_1 + z_2) + z_3\)
        \item \(z_1z_2=z_2z_1\)
        \item \(z_1(z_2z-3)=(z_1z_2)z_3\)
        \item \(z_1(z_2+z_3)=z_1z_2+z_1z_3\)
        \item \(z_1+0=z_1\)
        \item \(z_1\cdot 1=z_1\)
        \item \(\exists_{=1} z \,|\, z+z_1=0\)
        \item \(\exists_{=1} z \,|\, z\cdot z_1=1\)
    \end{enumerate}
\end{snippetaxiom}

\subsection{Trigonometric form}

\begin{snippettheorem}{complex-number-trig-form}{Complex Number Trigonometric Form}
    Any complex number \(s=a+bi\) can be represented in a trigonometric form
    \[
        a+bi=r(\cos\theta+i\sin\theta)
    \]
    where \(r=|s|\) and \(\theta = \arg(s)\).
\end{snippettheorem}

\subsection{Vector form}

\begin{snippet}{complex-number-vector-form}
    Any complex number \(a+bi\) can be represented by a vector \((a,b)\).
\end{snippet}

\begin{snippetdefinition}{complex-scalar-product}{Complex scalar product}
    The \textit{scalar product} between \(z_1=a+bi\) and \(z_2=c+di\) is given by
    
    \[
        z_1\circ z_2 = |z_1|\,|z_2|\cos\theta
        = ac+bd = \Re(z_1^*z_2) = \frac{1}{2}(z_1^*z_2+z_1z_2^*)
    \]
    where \(\theta\) is the angle formed by the two vectors.
\end{snippetdefinition}

\begin{snippetdefinition}{complex-vector-product}{Complex vector product}
    The vector product between \(z_1=a+bi\) and \(z_2=c+di\) is given by

    \[
        z_1\times z_2 = |z_1|\,|z_2|\sin\theta
        = ad-cb = \Im(z_1^*z_2) = \frac{1}{2i}(z_1^*z_2+z_1z_2^*)
    \]
    where \(\theta\) is the angle formed by the two vectors.

    We can see that
    \[
        z_1^*z_2 = (z_1\circ z_2) + i(z_1 \times z_2)
    \]
\end{snippetdefinition}

\subsection{Complex conjugate coordinates}

\begin{snippet}{complex-conj-coordinates}
Since for any complex number \(z=a+bi\)
\begin{align*}
    a&=\frac{1}{2}(z+z^*)
    \\
    b&=\frac{1}{2i}(z-z^*)
\end{align*}
\(z\) can also be represented by the conjugate coordinates \((z, z^*)\).
\end{snippet}

% TODO: exponentiation
% TODO: Cauchy - Schwarz inequality
% TODO: also polar form with Euler's formula

\end{document}
