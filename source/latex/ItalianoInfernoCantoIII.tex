\documentclass[preview]{standalone}

\usepackage{amsmath}
\usepackage{amssymb}
\usepackage{stellar}
\usepackage{bettelini}

\hypersetup{
    colorlinks=true,
    linkcolor=black,
    urlcolor=blue,
    pdftitle={Stellar},
    pdfpagemode=FullScreen,
}

\begin{document}

\title{Stellar}
\id{italiano-inferno-canto-iii}
\genpage

\section{Canto III}

\begin{snippet}{inferno-canto-iii-1}
    \StellarPoetry{1}{
        "Per me si va ne la città \textbf{dolente}, \\
        per me si va ne l'etterno \textbf{dolore}, \\
        per me si va tra la \textbf{perduta gente}.
    }{
        «Attraverso me si va nella città che soffre,\\
        attraverso me si va nel dolore senza fine,\\
        attraverso me si va tra i dannati.
    }

    Questa terzina è incisa sulla porta dell'inferno.
    La porta parla al singolare con un'anafora di \quotes{Per me si va}.
    La prima terzina esprime il dolore che è presente nel posto.
\end{snippet}

\begin{snippet}{inferno-canto-iii-4}
    \StellarPoetry{4}{
        Giustizia mosse il mio alto fattore; \\
        fecemi la divina podestate, \\
        la somma sapïenza e 'l primo amore.
    }{
        La Giustizia ha mosso il mio sommo Creatore;\\
        mi hanno creato il Padre,\\
        il Figlio e lo Spirito Santo.
    }
    Questi sono i tre attributi della trinità (Io, inferno, sono fatto da Dio).
    Tutto quello che si vedrà è quindi un luogo giusto, perché Dio è giusto per definizione.
    Nell'inferno manca quindi arbitrio.
\end{snippet}

\begin{snippet}{inferno-canto-iii-7}
    \StellarPoetry{7}{
        Dinanzi a me non fuor cose create \\
        se non etterne, e io etterno duro. \\
        Lasciate ogne speranza, voi ch'intrate".
    }{
        Prima di me non fu creato nulla se non\\
        le realtà eterne, e io stessa sono eterna.\\
        Lasciate ogni speranza, o voi che entrate».
    }
    
    L'inferno è sempiterno.
\end{snippet}

\begin{snippet}{inferno-canto-iii-10}
    \StellarPoetry{10}{
        Queste parole di colore oscuro \\
        vid'ïo scritte al sommo d'una porta; \\
        per ch'io: "Maestro, il senso lor m'è duro".
    }{
        Queste parole di senso difficile e minaccioso\\
        le vidi scritte sulla parte alta di una porta; perciò\\
        (dissi): «Maestro, il lor significato m'è oscuro».
    }
    Le scritte sono di colore nero, scure e hanno la funzione opposte delle scritte
    che c'erano nei portoni delle chiese medievali, le quali invitavano i fedeli ad entrare.
\end{snippet}

\begin{snippet}{inferno-canto-iii-13}
    \StellarPoetry{13}{
        Ed elli a me, come persona accorta: \\
        "Qui si convien lasciare ogne sospetto; \\
        ogne viltà convien che qui sia morta.
    }{
        Egli mi disse, come da esperto:\\
        «Qui è meglio abbandonare ogni paura;\\
        ogni pusillanimità dev'essere abbandonata.
    }
    Virgilio dice a Dante che deve lasciare ogni sospetto (esitazione).
    L'animità non ha più il tempo di esitare.

    Dante è un uomo con le sue debolezze ma che necessita umanamente di un uomo che lo aiuti con la sua paura.
\end{snippet}

\begin{snippet}{inferno-canto-iii-16}
    \StellarPoetry{16}{
        Noi siam venuti al loco ov'i' t' ho detto \\
        che tu vedrai le genti dolorose \\
        c' hanno perduto il ben de l'intelletto".
    }{
        Siamo giunti in quel posto dove t'ho detto\\
        che vedrai anime sofferenti che hanno\\
        smarrito la verità suprema, cioè Dio».
    }
    
\end{snippet}

\begin{snippet}{inferno-canto-iii-19}
    \StellarPoetry{19}{
        E poi che la sua mano a la mia puose \\
        con lieto volto, ond'io mi confortai, \\
        mi mise dentro a le segrete cose.
    }{
        E dopo che ebbe posto la sua mano sulla mia\\
        con uno sguardo sereno, così che mi confortai,\\
        e lui mi fece entrare a quei luoghi segreti.
    }
    Questo ultimo verso è l'ultimo alla luce del sole.
\end{snippet}

\begin{snippet}{inferno-canto-iii-22}
    \StellarPoetry{22}{
        Quivi \textbf{sospiri}, \textbf{pianti} e \textbf{alti guai} \\
        risonavan per l'aere sanza stelle, \\
        per ch'io al cominciar ne lagrimai.
    }{
        Qui sospiri, lamenti e alte grida\\
        risuonavano per la caverna senza stelle,\\
        e io, che li sentivo per la prima volta, piansi.
    }
    Le prime percezioni sono acustiche siccome Dante non vede quasi nulla (area senza stelle).
    Per tutto l'inferno Dante proverà compassione dei dolenti.

    Climax: \textbf{sospiri}, \textbf{pianti} e \textbf{alti guai}.
\end{snippet}

\begin{snippet}{inferno-canto-iii-25}
    \StellarPoetry{25}{
        Diverse \textbf{lingue}, orribili \textbf{favelle}, \\
        \textbf{parole} di dolore, \textbf{accenti} d'ira, \\
        voci alte e fioche, e suon di man con elle
    }{
        Lingue di varie provenienza, accenti sconosciuti,\\
        paroli di sofferenza, esclamazioni d'ira, voci\\
        alte e basse, con rumori di percosse mischiati
    }
    Anti-climax di dettaglio: \textbf{lingue}, \textbf{favelle}, \textbf{parole}, \textbf{accenti} (suoni).
\end{snippet}

\begin{snippet}{inferno-canto-iii-28}
    \StellarPoetry{28}{
        facevano un tumulto, il qual s'aggira \\
        sempre in quell'aura sanza tempo tinta, \\
        come la rena quando turbo spira.
    }{
        facevano un gran rumore, che si rimescola\\
        in eterno in quel mondo senza luce né tempo,\\
        come la sabbia quando soffia il vento.
    }
    Tutti i suoini giungono a Dante come se fossero un vortice (tromba d'aria) di sabbia.
\end{snippet}

\begin{snippet}{inferno-canto-iii-31}
    \StellarPoetry{31}{
        E io ch'avea d'error la testa cinta, \\
        dissi: "Maestro, che è quel ch'i' odo? \\
        e che gent'è che par nel duol sì vinta?".
    }{
        E io che era nel pieno dello smarrimento,\\
        dissi: «Maestro, cos'è ciò che sento? E chi sono\\
        questi che sembrano così schiacciati dal dolore?».
    }
    
\end{snippet}

\begin{snippet}{inferno-canto-iii-34}
    \StellarPoetry{34}{
        Ed elli a me: "Questo misero modo \\
        tegnon l'anime triste di coloro \\
        che visser \textbf{sanza 'nfamia e sanza lodo}.
    }{
        Ed egli a me: «Tengono questo vile \\
        atteggiamento le anime di quelli che vissero \\
        senza fare né il Male né il Bene.
    }
    Virgilio risponde che questo misero modo di lamentarsi appartiene
    alle anime dei pusillanimi (ignavi), ossia coloro che non hanno preso decisioni
    nella loro vita, non hanno commesso nè bene (lodo) nè male (infamia).
    
    La loro colpa è quella di non aver esercitato il libero arbitrio di Dio,
    per cui rinunciare alla identità più profonda di uomo e vivere come un animale.    
\end{snippet}

\begin{snippetdefinition}{pusillanimi-definition}{I pusillanimi}
    I \textit{pusillanimi} sono coloro che rifiutano la loro identità di uomo,
    sono timidi e non hanno coraggio e determinazione, per cui non esercitano il libero arbitrio.
\end{snippetdefinition}

\begin{snippet}{inferno-canto-iii-37}
    \StellarPoetry{37}{
        Mischiate sono a quel cattivo coro \\
        de li angeli che \textbf{non furon ribelli} \\
        \textbf{né fur fedeli} a Dio, ma per sé fuoro.
    }{
        Mischiate a questa schiera spregevole, ci sono\\
        quegli angeli che non furono né ribelli né fedeli\\
        a Dio, ma stettero da soli per se stessi.
    }
    
    Vi è un gruppo di angeli che sta a sè (non sta nè con Dio nè con il Demonio), angeli che non si schierarono.

    Questa categoria non è riconosciuta dalle Scrittura ma riconosciuta dalla cultura popolare.
\end{snippet}

\begin{snippet}{inferno-canto-iii-40}
    \StellarPoetry{40}{
        \textbf{Caccianli i ciel} per non esser men belli, \\
        \textbf{né lo profondo inferno li riceve}, \\
        ch'alcuna gloria i rei avrebber d'elli».
    }{
        Li hanno respinti i cieli per nomn rovinarsi,\\
        ma non li accetta nemmeno l'inferno profondo,\\
        ché i dannati non trarrebbero gloria da loro».
    }
    
    Virgilio smette di parlare dicendo che queste anime sono cacciate dai cieli
    ma nemmeno nell'inferno profondo. Ribadisce 3 volte che non sono nè da un lato nè dall'altro.
\end{snippet}

\begin{snippet}{inferno-canto-iii-43}
    \StellarPoetry{43}{
        E io: «Maestro, che è tanto greve \\
        a lor che lamentar li fa sì forte?». \\
        Rispuose: «Dicerolti molto breve.
    }{
        E io: «Maestro, cosa c'è di tanto doloroso\\
        che li fa lamentare così forte?».\\
        Rispose: Te lo dirò in poche parole.
    }
    
\end{snippet}

\begin{snippet}{inferno-canto-iii-46}
    \StellarPoetry{46}{
        Questi non hanno speranza di morte, \\
        e la lor cieca vita è tanto bassa, \\
        che 'nvidïosi son d'ogne altra sorte.
    }{
        Questi non hanno speranza di morire,\\
        e la loro vita qui è tanto spregevole e schifosa\\
        che sono invidiosi di ogni altro destino.
    }
    
    Le anime del profondo inferno, potrebbero vantarsi con i pusillani perché loro non hanno un peccato.
    Un'anima dannata che vede qualcuno che non ha commesso il male nella sua stessa posizione,
    potrebbe sminuire i propri peccati e fare meno importanza alla propria colpa.
    Questo è il motivo per cui i pusillani sono espulsi.

    Virgilio ritorna una spiegazione molto breve, come se loro non si meritassero nemmeno
    più parole. Tutte le anime sperano di essere annichilite, ma questi
    hanno una vita cieca tanto bassa e piccola che sono invidiosi di tutti.
\end{snippet}

\begin{snippet}{inferno-canto-iii-49}
    \StellarPoetry{49}{
        Fama di loro il mondo esser non lassa; \\
        misericordia e giustizia li sdegna: \\
        non ragioniam di lor, ma guarda e passa».
    }{
        Il mondo non ha lasciato testimonianza di loro;\\
        la giustizia e la misericordia li disprezzano:\\
        non occupiamoci di loro: guarda, e andiamo via».
    }
    
    Queste anime non mancheranno al mondo e non verranno ricordati.
    Dante e Virgilio hanno un grosso disprezzo verso questi spiriti.
    Non meritano nè parola nè tempo.
\end{snippet}

\begin{snippet}{inferno-canto-iii-52}
    \StellarPoetry{52}{
        E io, che riguardai, vidi una 'nsegna \\
        che girando correva tanto ratta, \\
        che d'ogne posa mi parea indegna;
    }{
        E io, che li osservai, vidi un'insegna, che\\
        correva girando così velocemente\\
        che mi sembrava incapace di fermarsi;
    }
    
    Dante vede una bandiera che corre molto rapida, la quale viene seguita
    da una massa di gente. È sorpreso che ci siano tanti pusillanimi.
\end{snippet}

\begin{snippet}{inferno-canto-iii-55}
    \StellarPoetry{55}{
        che girando correva tanto ratta, \\
        di gente, ch'i' non averei creduto \\
        che morte tanta n'avesse disfatta.
    }{
        e dietro di essa una fila di dannati\\
        così lunga, che io non avrei mai creduto che\\
        la morte ne avesse presi così tanti.
    }
    
    Viene quindi mostrata la prima relazione fra pena e colpa per contrappasso,
    in questo caso, per opposizione. Infatti, le anime devono sepre seguire il vessillo (simbolo dello schieramento).
    Così come in vita non si sono mai schierati, ora devono schierarsi per sempre.
\end{snippet}

\begin{snippet}{inferno-canto-iii-58}
    \StellarPoetry{58}{
        Poscia ch'io v'ebbi alcun riconosciuto, \\
        vidi e conobbi l'ombra di colui \\
        che fece per viltade il gran rifiuto.
    }{
        Dopo che io ebbi riconosciuto qualcuno,\\
        vidi e riconobbi l'anima di colui che\\
        per vigliaccheria fece la grande rinuncia.
    }
    
    Dante ne riconosce alcuni, in particolare l'anima di colui
    che per pusillanimità fece un gran rifiuto.

    Si riferisce probabilmente al papa Celestino V, il quale si è dimesso
    dopo qualche mese. Le sue dimissioni fecero eleggere Bonifacio VIII, il quale manderà Dante in esilio.
    A supporto di questa tesi vi è la frase "vidi e conobbi", il personaggio è quindi un contemporaneo
    di Dante. Inoltre, già tutti i commentatori antichi, i quali erano vinici a quest'epoca,
    hanno riferito questo papa.

    Dante avrebbe potuto scrivere il canto prima che il papa diventasse santo
    perché forse non avrebbe messo un santo all'inferno.
    D'altro canto non sarebbe un eresia farlo, in quanto è una scelta della chiesa.
\end{snippet}

\begin{snippet}{inferno-canto-iii-61}
    \StellarPoetry{61}{
        Incontanente intesi e certo fui \\
        che questa era la setta d'i cattivi, \\
        a Dio spiacenti e a' nemici sui.
    }{
        Subito capii e fui sicuro che questa\\
        era la schiera dei vili, che sono disprezzati\\
        sia da Dio che dalle forze infernali.
    }
    
\end{snippet}

\begin{snippet}{inferno-canto-iii-64}
    \StellarPoetry{64}{
        Questi sciaurati, che mai non fur vivi, \\
        erano ignudi e stimolati molto \\
        da mosconi e da vespe ch'eran ivi.
    }{
        Questi sciagurati, che non hanno mai vissuto,\\
        erano nudi e pungolati con forza\\
        dai mosconi e dalle vespe che si trovavano lì.
    }
    Questa è la seconda parte della pena: erano nudi (come tutte le anime)
    e punti da vespe e mosconi.
\end{snippet}

\begin{snippet}{inferno-canto-iii-67}
    \StellarPoetry{67}{
        Elle rigavan lor di sangue il volto, \\
        che, mischiato di lagrime, a' lor piedi \\
        da \textbf{fastidiosi} vermi era ricolto.
    }{
        Queste gli rigavano di sangue il volto,\\
        che, mischiato con le lacrime, veniva raccolto\\
        ai loro piedi da vermi luridi.
    }
    Questo è un contrappasso per analogia. Così come in vita sono stati insignificanti,
    ora sono tormentati piccoli insetti insegnificanti come loro (mosconi vespe e vermi).

    Lacrime e sangue indicano uno sforzo, quello sforzo che loro non hanno mai compiuto.
\end{snippet}

\begin{snippet}{inferno-canto-iii-70}
    \StellarPoetry{70}{
        E poi ch'a riguardar oltre mi diedi, \\
        vidi genti a la riva d'un gran fiume; \\
        per ch'io dissi: «Maestro, or mi concedi
    }{
        E quando mi volsi a guardare altrove,\\
        vidi una folla di gente presso un gran fiume;\\
        per cui dissi: «Maestro, concedimi
    }
    
\end{snippet}

\begin{snippet}{inferno-canto-iii-73}
    \StellarPoetry{73}{
        ch'i' sappia quali sono, e qual costume \\
        le fa di trapassar parer sì pronte, \\
        com' i' discerno per lo fioco lume».
    }{
        di sapere chi sono, e quale principio le fa\\
        sembrare così desiderose della traversata,\\
        come mi pare di capire nella poca luce che c'è».
    }
    Guardando in un'altra direzione, oltre un fiume, vede molte genti e chiede a Virgilio
    chi sono e perché sembrino (dalla poca luca) desiderose di attraversare la riva.
\end{snippet}

\begin{snippet}{inferno-canto-iii-76}
    \StellarPoetry{76}{
        Ed elli a me: «Le cose ti fier conte \\
        quando noi fermerem li nostri passi \\
        su la trista riviera d'Acheronte».
    }{
        Ed egli a me: «Tutto ti verrà spiegato\\
        quando noi ci fermeremo\\
        sulla triste riva del fiume Acheronte».
    }
    Virgilio indica che gli darà la risposta quando arriveranno là.
    Il primo fiume infernale si chiama Acheronte.
\end{snippet}

\begin{snippet}{inferno-canto-iii-79}
    \StellarPoetry{79}{
        Allor con li occhi vergognosi e bassi, \\
        temendo no 'l mio dir li fosse grave, \\
        infino al fiume del parlar mi trassi.
    }{
        Allora, con gli occhi bassi e pieni di vergogna,\\
        temendo che le mie parole fossero state\\
        sbagliate, rimasi in silenzio fino al fiume.
    }
    Dante, temendo di essere stato inopportuno, fa silenzio sino all'arrivo al fiume.
\end{snippet}

\begin{snippet}{inferno-canto-iii-82}
    \StellarPoetry{82}{
        Ed ecco verso noi venir per nave \\
        un vecchio, bianco per antico pelo, \\
        gridando: «Guai a voi, anime prave!
    }{
        Ed ecco giungere verso di noi su una nave\\
        un vecchio, bianco per la vecchiaia,\\
        che gridava: «Guai a voi, anime malvagie!
    }
    Giunge verso Dante un vecchio con barba e capelli bianchi su una nave che grida.
    Esso è Caronte ed è il traghettatore di anime.
\end{snippet}

\begin{snippet}{inferno-canto-iii-85}
    \StellarPoetry{85}{
        \textbf{Non isperate} mai veder lo cielo: \\
        i' vegno per menarvi a l'altra riva \\
        ne le tenebre \textbf{etterne}, in caldo e 'n gelo.
    }{
        Non sperate di veder mai più il cielo:\\
        vengo per condurvi all'altra sponda\\
        nel buio eterno, tra fiamme e ghiacci.
    }
    Le parole di Caronte replicano l'iscrizione della porta dell'inferno.
    Il personaggio è già presente con il medesimo ruolo nell'Eneide.
\end{snippet}

\begin{snippet}{inferno-canto-iii-88}
    \StellarPoetry{88}{
        E tu che se' costì, anima viva, \\
        pàrtiti da cotesti che son morti». \\
        Ma poi che vide ch'io non mi partiva,
    }{
        E tu, anima viva, che pure sei qua\\
        allontanati da questi, che sono già morti».\\
        Ma, poiché vide che non me ne andavo,
    }

    Caronte si ritrova davanti un vivo (Dante) e gli dice che lui non è un dannato, e che quindi se ne deve andare.
\end{snippet}

\begin{snippet}{inferno-canto-iii-91}
    \StellarPoetry{91}{
        disse: «Per altra via, per altri porti \\
        verrai a piaggia, non qui, per passare: \\
        più lieve legno convien che ti porti».
    }{
        disse: «Per un'altra strada, per altri porti giungerai\\
        alla spiaggia [del Purgatorio]; non da qui: è\\
        meglio che ti porti una nave più rapida».
    }
    
    Una volta visto che Dante non si muoveva, gli dice
    "[Quando sarà il tuo momento], farai un altro percorso su una barca più leggera".
    Per cui, non è un'anima dell'inferno.
\end{snippet}

\begin{snippet}{inferno-canto-iii-94}
    \StellarPoetry{94}{
        E 'l duca lui: «Caron, non ti crucciare: \\
        vuolsi così colà dove si puote \\
        ciò che si vuole, e più non dimandare».
    }{
        E Virgilio a lui: «Caron, non preoccuparti:\\
        si vuole così là dove si può realizzare ciò che\\
        si vuole; non chiedere altro».
    }
    Virgilio interviene e dice a Caronte di far passare Dante
    siccome il suo viaggio è voluto da Dio, e per cui Caronte non deve interferire.
\end{snippet}

\begin{snippet}{inferno-canto-iii-97}
    \StellarPoetry{97}{
        Quinci fuor quete le lanose gote \\
        al nocchier de la livida palude, \\
        che 'ntorno a li occhi avea di fiamme rote.
    }{
        Così si calmarono le guance barbute\\
        al nocchiero della plumbea palude, che attorno\\
        agli occhi aveva lingue di fiamme.
    }
    % nocchier = tragetatore
    Caronte ha delle fiamme attorno agli occhi. Il suo ritratto temrina qui, e lui si tranquillizza.
\end{snippet}

\begin{snippet}{inferno-canto-iii-100}
    \StellarPoetry{100}{
        Ma quell' anime, ch'eran lasse e nude, \\
        cangiar colore e dibattero i denti, \\
        ratto che 'nteser le parole crude.
    }{
        Ma quelle anime, che erano nude e stremate,\\
        impallidirono e cominciarono a battere i denti\\
        non appena compresero le parole crudeli.
    }
    A differenza di Dante, le anime dannate sanno che quelle parole erano rivolte a loro.
\end{snippet}

\begin{snippet}{inferno-canto-iii-103}
    \StellarPoetry{103}{
        Bestemmiavano \textbf{Dio} e lor \textbf{parenti}, \\
        \textbf{l'umana spezie} e 'l \textbf{loco} e 'l \textbf{tempo} e 'l \textbf{seme} \\
        di lor semenza e di lor nascimenti.
    }{
        Bestemmiavano il nome di Dio e dei parenti,\\
        il genere umano e il luogo e il tempo e la stirpe\\
        della loro genesi e della loro nascita.
    }
    
    Anti-climax: \textbf{Dio}, \textbf{parenti}, \textbf{l'umana spezie}, \textbf{tempo}
    e \textbf{seme}. \\
    Le anime maledicono il giorno della loro nascita.
\end{snippet}

\begin{snippet}{inferno-canto-iii-106}
    \StellarPoetry{106}{
        Poi si ritrasser tutte quante insieme, \\
        forte piangendo, a la riva malvagia \\
        ch'attende ciascun uom che Dio non teme.
    }{
        Poi, piangendo a gran voce, si ammassarono\\
        tutte quante insieme verso il fiume malvagio\\
        che aspetta chi non ha timore di Dio.
    }
    I pusillanimi si trovano nell'antinferno, uno spazio fra la porta e il fiume.
    L'inferno vero comincia infatti dopo il fiume.
\end{snippet}

\begin{snippet}{inferno-canto-iii-109}
    \StellarPoetry{109}{
        Caron dimonio, con occhi di bragia \\
        loro accennando, tutte le raccoglie; \\
        batte col remo qualunque s'adagia.
    }{
        Il demonio Caronte, con gli occhi come brace\\
        che accennava a loro, le aduna tutte;\\
        e colpisce con un remo chiunque si siede a terra.
    }
    Caronte viene anche definito come un demonio con occhi di brace.
    Con un cenno raduna tutte le anime e partono in barca.
    Caronte picchia con il remo chi anche solo minimanete provi ad adagiarsi.
\end{snippet}

\begin{snippet}{inferno-canto-iii-112}
    \StellarPoetry{112}{
        Come d'autunno si levan le foglie \\
        l'una appresso de l'altra, fin che 'l ramo \\
        vede a la terra tutte le sue spoglie,
    }{
        Come in autunno le foglie cadono scendendo\\
        l'una vicino all'altra, fin quando il ramo\\
        vede per terra tutte le sue vesti,
    }
    Come d'autunno un albero vede tutte le sue foglie cadere,
\end{snippet}

\begin{snippet}{inferno-canto-iii-115}
    \StellarPoetry{115}{
        similemente il mal seme d'Adamo \\
        gittansi di quel lito ad una ad una, \\
        per cenni come augel per suo richiamo.
    }{
        così la razza dannata di Adamo\\
        si getta dalla spiaggia sulla barca una ad una,\\
        come il falcone al richiamo del cenno [di Caronte].
    }
    
    similmente, ad una a una, le anime salgono sulla nave di Caronte.
    La similitudine presenta anche un segno di rassegnazione.
\end{snippet}

\begin{snippet}{inferno-canto-iii-118}
    \StellarPoetry{118}{
        Così sen vanno su per l'onda bruna, \\
        e avanti che sien di là discese, \\
        anche di qua nuova schiera s'auna.
    }{
        Così se ne vanno per il fiume cupo,\\
        e prima che siano scese sull'altra riva,\\
        già di qua si raduna una nuova schiera.
    }
    Caronte non fa a tempo a portare le anime dall'altra parte,
    che un nuovo gruppo di anime si raggruppa nuovamente.
\end{snippet}

\begin{snippet}{inferno-canto-iii-121}
    \StellarPoetry{121}{
        «Figliuol mio», disse 'l maestro cortese, \\
        «quelli che muoion ne l'ira di Dio \\
        tutti convegnon qui d'ogne paese;
    }{
        «Figliolo», disse Virgilio, «coloro che muoiono\\
        in condizione di peccato mortale, tutti\\
        convergono quida ogni paese del mondo;
    }
    Ecco finalmente le due risposte di Virgilio.
\end{snippet}

\begin{snippet}{inferno-canto-iii-124}
    \StellarPoetry{124}{
        e pronti sono a trapassar lo rio, \\
        ché la divina giustizia li sprona, \\
        sì che la tema si volve in disio.
    }{
        e sono desiderosi di attraversare l'Acheronte\\
        poiché li spinge la giustizia divina\\
        così che la paura si trasforma in desiderio.
    }
    
    Ogni uomo, pur dannato e peccatore (eccetto i pusillanimi),
    conserva la coscienza, e ritiene la pena giusta in quanto riesce a distinguere fra bene e male,
    nonostante la prorpia paura.
\end{snippet}

\begin{snippet}{inferno-canto-iii-127}
    \StellarPoetry{127}{
        Quinci non passa mai anima buona; \\
        e però, se Caron di te si lagna, \\
        ben puoi sapere omai che 'l suo dir suona».
    }{
        Da qui non transita mai un'anima buona e pura\\
        e quindi, se Caronte si lamenta della tua\\
        presenza, ora capisci cosa egli vuol dire».
    }
    
    Il fatto che Caronte non voglia che Dante sia lì, indica che Dante non ci passerà dopo la morte.
\end{snippet}

\begin{snippet}{inferno-canto-iii-130}
    \StellarPoetry{130}{
        Finito questo, la buia campagna \\
        tremò sì forte, che de lo spavento \\
        la mente di sudore ancor mi bagna.
    }{
        Detto questo, la campagna immersa nel buio\\
        tremò così forte, che il ricordo dello spavento\\
        mi ghiaccia di sudore ancor oggi.
    }
    
    Vi è un terremoto molto spaventoso, che spaventa ancora Dante scrittore.
\end{snippet}

\begin{snippet}{inferno-canto-iii-133}
    \StellarPoetry{133}{
        La terra lagrimosa diede vento, \\
        che balenò una luce vermiglia \\
        la qual mi vinse ciascun sentimento;
    }{
        Quella valle di lagrime fu colpita da terremoto,\\
        che fece lampeggiare una luce rosso vivo,\\
        che vinse tutte le mie facoltà sensibili;
    }
    
\end{snippet}

\begin{snippet}{inferno-canto-iii-136}
    \StellarPoetry{136}{
        e caddi come l'uom cui sonno piglia.
    }{
        e svenni come l'uomo che crolla nel sonno.
    }
    
    Dante sviene dalla grandissima luce intensa dopo il terremoto.
\end{snippet}

\end{document}