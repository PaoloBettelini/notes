\documentclass[preview]{standalone}

\usepackage{amsmath}
\usepackage{amssymb}
\usepackage{xcolor}
\usepackage{stellar}
\usepackage{definitions}

\begin{document}

\id{topology-continuity}
\genpage

\section{Continuity}

\begin{snippetdefinition}{topology-continuity-definition}{Topology continuity}
    Let \((X, {\mathcal{T}}_X)\) and \((Y, {\mathcal{T}}_Y)\) be \topologicalspace[topological spaces].
    A \function \(f\colon X \fromto Y\) is \textit{continuous} (with respect to \({\mathcal{T}}_X\)
    and \({\mathcal{T}}_Y\)) if
    \[
        u \in {\mathcal{T}}_Y \implies f^\inversefunction(u) \in {\mathcal{T}}_X
    \]
\end{snippetdefinition}

\plain{The definition of continuity needs to preserve the structure of the topology.
Continuity intuitively means that nearby points in the domain map to nearby points in the codomain.
This is achieved by ensuring that the preimage of every open set in the codomain is an open set in the domain.
If this preimage condition is met, it means that the function doesn't "break apart" or "tear" the
structure of open sets in the domain when mapping to the codomain.}

\begin{snippetdefinition}{homeomorphism-definition}{Homeomorphism}
    Let \((X, {\mathcal{T}}_X)\) and \((Y, {\mathcal{T}}_Y)\) be \topologicalspace[topological spaces].
    A \function \(f\colon X \fromto Y\) is a \textit{homeomorphism} if it is \topologycontinuous, \bijective
    and \(f^\inversefunction\) is \topologycontinuous.
\end{snippetdefinition}

\begin{snippetdefinition}{sequentially-continuous-definition}{Sequentially continuous}
    Let \((X, {\mathcal{T}}_X)\) and \((Y, {\mathcal{T}}_Y)\) be \topologicalspace[topological spaces].
    A \function \(f\colon X \fromto Y\) is \textit{sequentially continuous} if for every
    \(x\in X\), for every sequence \({\{x_n\}}_{n \in \naturalnumbers} \subseteq X\) with \(x_n \to x\)
    we have that \({\{f(x_n)\}}_{n \in \naturalnumbers} \subseteq Y\) converges with
    \(f(x_n) \topologyconverges f(x)\).
\end{snippetdefinition}

\end{document}