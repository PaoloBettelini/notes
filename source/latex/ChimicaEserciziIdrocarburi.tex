\documentclass[preview]{standalone}

\usepackage{amsmath}
\usepackage{amssymb}
\usepackage{stellar}
\usepackage{chemfig}

\hypersetup{
    colorlinks=true,
    linkcolor=black,
    urlcolor=blue,
    pdftitle={Stellar},
    pdfpagemode=FullScreen,
}

\begin{document}

\title{Stellar}
\id{chimica-esercizi-idrocarburi}
\genpage

\begin{snippetexercise}{idrocarburi-ex1}
    {Nomina la seguente molecola secondo la nomenclatura IUPAC}
    \begin{center}
        \chemfig[angle increment=30]{-[1]-[-1](-[:-90]-[-1])-[1]-[-1]-[1]-[-1]}
        \\\vspace{0.25cm}
        3-etileptano
    \end{center}
\end{snippetexercise}

\begin{snippetexercise}{idrocarburi-ex2}
    {Nomina la seguente molecola secondo la nomenclatura IUPAC}
    \begin{center}
        \chemfig[angle increment=30]{-[-1](-[:90])(-[:-150])-[-1]-[1]}
        \\\vspace{0.25cm}
        2,2-dimetilbutano
    \end{center}
\end{snippetexercise}

\begin{snippetexercise}{idrocarburi-ex3}
    {Nomina la seguente molecola secondo la nomenclatura IUPAC}
    \begin{center}
        \chemfig[angle increment=30]{-[1](-[:90])-[-1]-[1](-[:90]-[1]-[:90])-[-1]-[1]-[-1]-[1]}
        \\\vspace{0.25cm}
        2-metil-4-propilottano
    \end{center}
\end{snippetexercise}

\begin{snippetexercise}{idrocarburi-ex4}
    {Nomina la seguente molecola secondo la nomenclatura IUPAC}
    \begin{center}
        \chemfig{*5(---(-[1])-(-[:120]-[1])-)}
        \\\vspace{0.25cm}
        1-etil-2-metilciclopentano
    \end{center}
\end{snippetexercise}

\begin{snippetexercise}{idrocarburi-ex5}
    {Nomina la seguente molecola secondo la nomenclatura IUPAC}
    \begin{center}
        \chemfig[angle increment=30]{-[1](-[:90])(-[3])-[-1]-[1](-[:90]-[1]-[-1])-[-1]-[1]-[-1]-[1]-[-1]}
        \\\vspace{0.25cm}
        2,2-dimetil-4-propilnonano
    \end{center}
\end{snippetexercise}

\begin{snippetexercise}{idrocarburi-ex6}
    {Nomina la seguente molecola secondo la nomenclatura IUPAC}
    \begin{center}
        \chemfig[angle increment=30]{=[1]-[-1]=[1]}
        \\\vspace{0.25cm}
        1,3-butadiene
    \end{center}
\end{snippetexercise}

\begin{snippetexercise}{idrocarburi-ex7}
    {Nomina la seguente molecola secondo la nomenclatura IUPAC}
    \begin{center}
        \chemfig[angle increment=30]{-[-1](-[:-90])-[1]~[1]}
        \\\vspace{0.25cm}
        3-metil-1-butino
    \end{center}
\end{snippetexercise}

\begin{snippetexercise}{idrocarburi-ex8}
    {Nomina la seguente molecola secondo la nomenclatura IUPAC}
    \begin{center}
        \chemfig[angle increment=30]{-[1]-[-1](=[:-90])-[1]}
        \\\vspace{0.25cm}
        2-metil-1-butene
    \end{center}
\end{snippetexercise}

\begin{snippetexercise}{idrocarburi-ex9}
    {Nomina la seguente molecola secondo la nomenclatura IUPAC}
    \begin{center}
        \chemfig[angle increment=30]{-[1]=[-1]-[1](-[:90])-[-1]}
        \\\vspace{0.25cm}
        4-metil-2-pentene
    \end{center}
\end{snippetexercise}

\begin{snippetexercise}{idrocarburi-ex10}
    {Nomina la seguente molecola secondo la nomenclatura IUPAC}
    \begin{center}
        \chemfig[angle increment=30]{=[1](-[:90])-[-1]=[1]}
        \\\vspace{0.25cm}
        2-metil-1,3-butadiene
    \end{center}
\end{snippetexercise}

\begin{snippetexercise}{idrocarburi-ex11}
    {Nomina la seguente molecola secondo la nomenclatura IUPAC}
    \begin{center}
        \chemfig[angle increment=30]{-[-1](-[:-90])=[1]-[-1]-[1](-[:90])-[-1]-[1]}
        \\\vspace{0.25cm}
        2,5-dimetil-2-eptene
    \end{center}
\end{snippetexercise}

\begin{snippetexercise}{idrocarburi-ex12}
    {Nomina la seguente molecola secondo la nomenclatura IUPAC}
    \begin{center}
        \chemfig[angle increment=30]{~[1]-[1](-[:90])-[-1]-[1](-[:90])-[-1]-[1]}
        \\\vspace{0.25cm}
        3,5-dimetil-1-eptino
    \end{center}
\end{snippetexercise}

\begin{snippetexercise}{idrocarburi-ex13}
    {Nomina la seguente molecola secondo la nomenclatura IUPAC}
    \begin{center}
        \chemfig[angle increment=30]{-[-1]-[1](-[:90]-[1])-[-1]=[1]-[-1]-[1](-[:90])-[-1]-[1]}
        \\\vspace{0.25cm}
        3-etil-7-metil-4-nonene
    \end{center}
\end{snippetexercise}

% manca un O sopra il legame doppio?
%\begin{snippetexercise}{idrocarburi-ex14}
%    {Nomina la seguente molecola secondo la nomenclatura IUPAC}
%    \begin{center}
%        \chemfig[angle increment=30]{-[1](=[:90])-[-1](-[:-90]-[-1])-[1](-[:110])(-[:70])-[-1]-[1]}
%        \\\vspace{0.25cm}
%        3-etil-4,4-dimetil-2-esanone
%    \end{center}
%\end{snippetexercise}

\begin{snippetexercise}{idrocarburi-ex15}
    {Nomina la seguente molecola secondo la nomenclatura IUPAC}
    \begin{center}
        \chemfig[angle increment=30]{-[1](-[:90])-[-1]-[1](=[:90]O)-[-1]OH}
        \\\vspace{0.25cm}
        Acido 3-metilbutanoico
    \end{center}
\end{snippetexercise}

\begin{snippetexercise}{idrocarburi-ex16}
    {Nomina la seguente molecola secondo la nomenclatura IUPAC}
    \begin{center}
        \chemfig[angle increment=30]{-[1](=[:90]O)-[-1]}
        \\\vspace{0.25cm}
        Propanone (acetone)
    \end{center}
\end{snippetexercise}

\begin{snippetexercise}{idrocarburi-ex17}
    {Nomina la seguente molecola secondo la nomenclatura IUPAC}
    \begin{center}
        \chemfig[angle increment=30]{-[-1]-[1](=[:90]O)-[-1]-[1]}
        \\\vspace{0.25cm}
        3-pentanone
    \end{center}
\end{snippetexercise}

\begin{snippetexercise}{idrocarburi-ex18}
    {Nomina la seguente molecola secondo la nomenclatura IUPAC}
    \begin{center}
        \chemfig[angle increment=30]{-[1](-[:90])-[-1]-[1]OH}
        \\\vspace{0.25cm}
        2-metil-1-propanolo
    \end{center}
\end{snippetexercise}

\begin{snippetexercise}{idrocarburi-ex19}
    {Nomina la seguente molecola secondo la nomenclatura IUPAC}
    \begin{center}
        \chemfig[angle increment=30]{-[1](-[:90])-[-1]-[1](=[:90]O)-[-1]}
        \\\vspace{0.25cm}
        4-metil-2-pentanone
    \end{center}
\end{snippetexercise}

\begin{snippetexercise}{idrocarburi-ex20}
    {Nomina la seguente molecola secondo la nomenclatura IUPAC}
    \begin{center}
        \chemfig[angle increment=30]{-[-1]-[1]-[-1]-[1]-[-1](=[:-90]O)-[1]NH_2}
        \\\vspace{0.25cm}
        esammide
    \end{center}
\end{snippetexercise}

\begin{snippetexercise}{idrocarburi-ex21}
    {Nomina la seguente molecola secondo la nomenclatura IUPAC}
    \begin{center}
        \chemfig[angle increment=30]{*5(=-(-)---)}
        \\\vspace{0.25cm}
        3-metil-1-ciclopentene
    \end{center}
\end{snippetexercise}

\begin{snippetexercise}{idrocarburi-ex22}
    {Quanti isomeri del butene ci sono?}
    Il butene è un idrocarburo alchenico con la formula molecolare
    C\({}_4\)H\({}_8\). Essendo un alchene, contiene un doppio legame tra due atomi di carbonio. Ci sono quattro isomeri del butene:

    \vspace{0.1cm}
    \begin{minipage}{0.24\textwidth}
        \begin{center}
            \chemfig[angle increment=60]{-[1]=-[-1]}
        \end{center}
    \end{minipage}
    \begin{minipage}{0.24\textwidth}
        \begin{center}
            \chemfig[angle increment=60]{-[1]=-[1]}
        \end{center}
    \end{minipage}
    \begin{minipage}{0.24\textwidth}
        \begin{center}
            \chemfig[angle increment=45]{=[1]-[-1]-[1]}
        \end{center}
    \end{minipage}
    \begin{minipage}{0.24\textwidth}
        \begin{center}
            \chemfig[angle increment=45]{-[-1](-[:-90])=[1]}
        \end{center}
    \end{minipage}
    \begin{minipage}{0.24\textwidth}
        \begin{center}
            cis-2-butene
        \end{center}
    \end{minipage}
    \begin{minipage}{0.24\textwidth}
        \begin{center}
            trans-2-butene
        \end{center}
    \end{minipage}
    \begin{minipage}{0.24\textwidth}
        \begin{center}
            1-butene
        \end{center}
    \end{minipage}
    \begin{minipage}{0.24\textwidth}
        \begin{center}
            2-metil-1-propene
        \end{center}
    \end{minipage}
\end{snippetexercise}

% 5-metil-2-esanone
% 2-metilpropanale

\end{document}