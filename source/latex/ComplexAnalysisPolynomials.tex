\documentclass[preview]{standalone}

\usepackage{amsmath}
\usepackage{amssymb}
\usepackage{stellar}
\usepackage{definitions}

\begin{document}

\id{complex-analysis-polynomials}
\genpage

\section{Fundamental theorem of algebra}

\begin{snippettheorem}{fundamental-theorem-of-algebra}{Fundamental theorem of algebra}
    Let \(P(z) \in \complexnumbers[z]\) where \(\polynomialdeg P(z) = n\). Then,
    \(P(z)\) admits exactly \(n\) solutions up to multiplicity.
    \[
        P(z) = a\prod_{i=1}^n (z-z_i)^{m_i}
    \]
    for \(a\in\complexnumbers\) and \(n\in \naturalnumbers\) such that
    \[
        \sum_{i=1}^n m_i = n
    \]
\end{snippettheorem}

\begin{snippetproposition}{conjugate-of-root-is-root}{}
    Let \(P(z) \in \realnumbers[z]\).
    Then, if \(z_0\) is a root of \(P(z)\), so is \(z^\complexconj\) with the same multiplicity.
\end{snippetproposition}

\begin{snippetproof}{conjugate-of-root-is-root-proof}{conjugate-of-root-is-root}{}
    Let
    \[
        P(z) = a \sum a_k z^k
    \]
    for \(a, a_k \in \realnumbers\).
    If \(P(z_0) = 0\), we also have that
    \begin{align*}
        {P(z_0)}^\complexconj &= {\left(a \sum a_k z^k \right)}^\complexconj \\
        &= a^\complexconj \sum a_k^\complexconj {(z^k)}^\complexconj
    \end{align*}
    Since \(a, a_k\) are real, \(a^\complexconj = a\).
    \begin{align*}
        {P(z_0)}^\complexconj &= a \sum a_k {(z^k)}^\complexconj \\
        &= P(z_0^\complexconj) = P(z_0) \\
        &= P(0)
    \end{align*}
    Now we prove that \(z_0\) and \(z_0^\complexconj\) have the same multiplicity. \\
    
    % we find a root z_0 with multiplicity m
    % conj(m) has at least multiplicity 1.
    % Now (z-z_0)(z-z_0^*) is a polynomial with real coefficients.
    % If we do the division, P(z) = (z-z_0)(z-z_0^*) * Q(z).
    % Since P(z) has real coefficients, so does Q(z).
    % And thus if z_0 is a root of Q, then so is z_0^*
    % and so on. So the multiplicity is at least as much as z_0
    \todo
\end{snippetproof}

% Dimostrare che un polynomio complesso a coefficienti reali grado siapri ha semmpre almeno uno zero reale
% dicendo che se fosse complesso al posto che reale allora sarebbe a coeff. complessi

\begin{snippet}{decomposition-of-polynomial-with-real-coefficients}
    Let \(P(z) \in \realnumbers[z]\). Then, this polynomial has some real roots
    \(x_1, x_2, \cdots, x_r\) with multiplicity \(m_1, m_2, \cdots, m_r\)
    and complex roots \(z_1, z_1^\complexconj, z_2, z_2^\complexconj, \cdots, z^s, z_s^\complexconj\)
    with multiplicity \(n_1, n_2, \cdots, n_s\). Thus,
    \[
        \polynomialdeg P(z) = n = \sum_{k=1}^r x_k + 2 \sum_{k=1}^s n_k
    \]
    The polynomial can be written as
    \[
        P(z) = a \prod_{k=1}^r {(z-x_k)}^{m_k} \prod_{k=1}^s {\left[(z-z_k)(z-z_k^\complexconj)\right]}^{n_k}
    \]
    for \(a\in\complexnumbers\).
    We now note that
    \begin{align*}
        (z-z_k)(z-z_k^\complexconj) &= z^2 - z(z_k + z_k^\complexconj) + z_kz_k^\complexconj \\
        &= z^2 - 2 \Re z_k \cdot z + {|z_k|}^2
    \end{align*}
    These terms are polynomials of second degree, and their discriminant is clearly negative
    \[
        \frac{\Delta}{4} = {\left(\Re z_k\right)}^2 - {|z_k|}^2 < 0
    \]
    since \(z_k\) is non-real.
\end{snippet}

\end{document}