\documentclass[preview]{standalone}

\usepackage{amsmath}
\usepackage{amssymb}
\usepackage{parskip}
\usepackage{fullpage}
\usepackage{hyperref}
\usepackage{bettelini}
\usepackage{stellar}

\hypersetup{
    colorlinks=true,
    linkcolor=black,
    urlcolor=blue,
    pdftitle={Cyclic groups},
    pdfpagemode=FullScreen,
}

\begin{document}

\id{fourieranalysis-introduction}
\genpage

\section{What is Fourier Analysis?}

\begin{snippet}{fourier-analysis-introduction}
    Fourier analysis is the study of how a function can be represented as a sum of waves. Take a look at the animation playing at the side, a shape is being drawn using a chain of rotating circles of different sizes. You can even try drawing your own shape, it's interactive! This article will cover in detail how this animation works, and what math is behind it. The concepts that we'll discover are widely used in electronics, acoustics and communications. Operators such as the Fourier Transform are constantly used in the real world, without these discoveries the world would not be the same. Much software relies in Fourier Analysis, such as for instance Shazam, the famous service for identifying songs. Any audio spectrum visualized processes the signal using Fourier Transform, these are just a few of the many application of this analysis.
\end{snippet}

\includesnpt{fourier-lib}
\includesnpt{fourier-series-2d}

\section{Fourier Series vs Fourier Transform}

\begin{snippet}{fourier-series-vs-fourier-transform}
    The Fourier series is the representation of a periodic function
    with a summation of sine and cosine waves of discrete frequencies. Each wave is weighted
    according to "how important" it is to represent the original function.
    Fourier Series are often represented in two ways: trigonometric and exponential.
    They both work in the same way, but the exponential one is also defined on the
    complex plane and as we'll see, has a nicer, more elegant form.
    \\\\
    The Fourier transform is an operation that transforms a signal
    from time-domain to a continuous frequency-domain. The function can be a generic, not necessarily period function \(f(x)\).
    The output of the Fourier transform \(\mathcal{F}\) is a complex-valued function whose absolute value represents the magnitude of each frequency.</p>
    \[
        \mathcal{F}\{f(t)\}=\hat{f}(\xi)
    \]
\end{snippet}

\end{document}
