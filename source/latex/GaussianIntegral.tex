\documentclass[preview]{standalone}

\usepackage{amsmath}
\usepackage{amssymb}
\usepackage{stellar}
\usepackage{definitions}
\usepackage{bettelini}

\begin{document}

\id{gaussian-integral}
\genpage

\section{Gaussian integral}

\begin{snippettheorem}{gaussian-integral-theorem}{Gaussian integral}
    \[
        \integral[-\infty][+\infty][e^{-t^2}][t] = \sqrt{\picircle}
    \]
\end{snippettheorem}

\begin{snippetproof}{gaussian-integral-theorem-proof}{gaussian-integral-theorem}{Gaussian integral}
    Consider
    \[
        I = \integral[-\infty][+\infty][e^{-x^2}][x]
    \]
    and its square    
    \[
        I^2 = \left(\, \integral[-\infty][+\infty][e^{-x^2}][x] \right) \left(\, \integral[-\infty][+\infty][e^{-y^2}][y] \right)
    \]
    which becomes
    \[
        I^2 = \integral[-\infty][+\infty][\integral[-\infty][+\infty][e^{-(x^2 + y^2)}][x]][y]
    \]
    We now perform the conversion to polar coordinates    
    \[
        I^2 = \integral[0][2\picircle][\integral[0][\infty][e^{-r^2} r][r]][\theta]
    \]
    Since
    \[
        \integral[0][2\picircle][1][\theta] = 2\picircle
    \]
    we can just complete the integration  
    \begin{align*}
        I^2 &= 2\picircle \integral[0][\infty][e^{-r^2} r][r] \\
        &=\integral[0][\infty][e^{-r^2} r][r] = \frac{1}{2}\\
        &= 2\picircle \cdot \frac{1}{2} = \picircle
    \end{align*}
    and thus
    \[
        I = \sqrt{\picircle}
    \]
\end{snippetproof}

\end{document}