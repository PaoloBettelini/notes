\documentclass[preview]{standalone}

\usepackage{amsmath}
\usepackage{amssymb}
\usepackage{stellar}
\usepackage{definitions}

\begin{document}

\id{sylow-theorems-exercises}
\genpage

\section{Exercises}

\begin{snippetexercise}{sylow-ex1}{Group is not simple}
    Show that a \group of order \(105\) is not simple. % TODOURGENT \simplegrp
\end{snippetexercise}

\begin{snippetsolution}{sylow-ex1-sol}{Group is not simple}
    Let \(G\) be a \group where \(|G| = 3\cdot5\cdot 7\).
    The \sylowpsubgroup[Sylow \(p\)-subgroups] of order \(p=3,5,7\)
    have \primen order, thus are \cyclicgroup[cyclic].
    We want to find \normalsubgrptext[normal subgroups] that are not trivial.
    Let \(n_p\) be the amount of \sylowpsubgroup[Sylow \(p\)-subgroups] of \(G\).
    We have that \(n_3 \equiv 1 \pmod{3}\) and \(n_3 \divides 5\cdot 7\).
    The divisors of \(5\cdot 7\) are \(1,5,7,35\). Among them, both \(1\)
    and \(7\) are congruent to \(1\) modulo \(3\).
    Then, \(n_5 \equiv 1 \pmod{5}\) and \(n_5 \divides 3\cdot 7\).
    The divisors of \(3\cdot 7\) are \(1,3,7,21\).
    Among them, both \(1\) and \(21\) are congruent to \(1\) modulo \(5\).
    For \(n_7\), the divisors are \(1,3,5,15\) and \(1\) and \(15\)
    are congruent to \(1\) modulo \(7\).
    No luck thrice. If at least one among \(n_3, n_5, n_7 = 1\)
    we would be done as we would have found a \normalsubgrptext[normal] \sylowpsubgroup.
    If none of them is \(1\), we would have \(n_3 =7\), \(n_5 = 21\)
    and \(n_7 = 15\). Consider the \sylowpsubgroup[\(3\)-Sylows]: they have trivial intersection
    as their order is \primen. Each one of them contains \(3-1\) elements of period \(3\)
    that are not contained in the other \sylowpsubgroup[\(3\)-Sylows].
    In \(G\) we find \(7\cdot 2 = 14\) elements of period \(3\). Likewise,
    for the other \primen[primes] we find \(21 \cdot (5-1) = 84\) elements of period \(5\)
    and \(15\cdot (7-1) = 90\) elements of period \(7\).
    However, \(14+84+90>105\) and thus at least one among \(n_3, n_5, n_7\)
    is \(1\). We can also note that only \(n_5 + n_7 > 105\)
    and thus at least one between \(n_5\) and \(n_7\) is \(1\).
    Furthermore, if \(n_7 = 15\), meaning the \sylowpsubgroup[\(7\)-Sylows]
    are not \normalsubgrptext[normal], then \(n_5 \neq 21\) (which we had already noted)
    and thus \(n_5 = 1\) (with \(4\) elements of period \(5\)), and we would have
    \(14+4+90 > 105\), meaning \(n_3 = 1\) \lightning. In conclusion,
    if the \sylowpsubgroup[\(7\)-Sylows] are not \normalsubgrptext[normal],
    then the \sylowpsubgroup[\(3\)-Sylows] and the \sylowpsubgroup[\(5\)-Sylows]
    are not \normalsubgrptext[normal].
\end{snippetsolution}

\begin{snippetexercise}{sylow-ex2}{Group is not simple}
    Show that a \group of order \(56\) is not simple. % TODOURGENT \simplegrp
\end{snippetexercise}

\begin{snippetsolution}{sylow-ex2-sol}{Group is not simple}
    We have \(|G| = 56 = 2^3 \cdot 7\).
    Let \(n_p\) be the amount of \sylowpsubgroup[Sylow \(p\)-subgroups] of \(G\).
    We have \(n_2 \equiv 1 \pmod{2}\) and \(n_2 \divides 7\),
    \(n_7 \equiv 1 \pmod{7}\) and \(n_7 \divides 2^3\).
    The counting argument seems to not be effective.
    Tne \sylowpsubgroup[\(7\)-Sylows] are \cyclicgroup[cyclic] of \primen order,
    meaning they have trivial intersection.
    However, the \sylowpsubgroup[\(2\)-Sylows] have order \(8\),
    and we cannot say much about their intersection.
    If \(n_7 \neq 1\) we have \(8\) \sylowpsubgroup[\(7\)-Sylows] and each of them contains
    \(6\) elements of period \(7\) not contained in the other
    \sylowpsubgroup[\(7\)-Sylows]. Thus, we have \(8\cdot 6 = 48\) elements of period
    \(7\). Since the space left is \(56 - 48 = 8\) elements to form
    the \sylowpsubgroup[\(2\)-Sylows]. However, each \sylowpsubgroup[\(2\)-Sylow]
    has \(8\) elements and thus we have room only for \(1\) of them (which is thus \normalsubgrptext[normal]).
\end{snippetsolution}

\end{document}