\documentclass[preview]{standalone}

\usepackage{amsmath}
\usepackage{amssymb}
\usepackage{stellar}
\usepackage{definitions}

\begin{document}

\id{biologia-ciclo-cellulare-problemi}
\genpage

\section{Problemi della meiosi}

\begin{snippetdefinition}{trisomia-21-definition}{Trisomia 21}
    La \textit{trisomia 21} è un disturbo per quale vi è un cromosoma extra (il cromosoma 21).
\end{snippetdefinition}

\includesnpt[width=70\%|src=/snippet/static/trisomia21.png]{centered-img}

\begin{snippet}{e924d64e-f74d-41d0-8a44-01ea950d6324}
    % TODO spiegare lo schema
    La motivazione più comune è data dalla madre, dove l'indice di rischio aumenta
    esponenzialmente dopo i 35 anni.
    Usando ovuli vecchi (gli ovuli sono prodotti alla nascita e rimangono gli stessi),
    è più probabile che i cromosomi non si stacchino.
\end{snippet}

\includesnpt[width=75\%|src=/snippet/static/anom.png]{centered-img}

\begin{snippetdefinition}{sindrome-klinefelter-definition}{Sindrome di Klinefelter}
    La \textit{sindrome di Klinefelter}
    è una trisomia XXY. Si tratta di uomini sterili con problemi dello sviluppo
    dei caratteri sessuali secondari:
    sterili, testicoli piccoli, pattern peli pubici femminile,
    pattern grasso pancia femminile, ginecomastia.
\end{snippetdefinition}

\begin{snippetdefinition}{sindrome-turner-definition}{Sindrome di Turner}
    La \textit{sindrome di Turner} è
    una monosomia X0. Unicamente una delle due X funziona,
    mentre l'altra è in parte silenziata. Si tratta di femmine
    solitamente sterili con problemi dello sviluppo dei caratteri sessuali secondari.
\end{snippetdefinition}

\begin{snippet}{f982a995-039d-442d-8913-a5f95fad1e50}
    A volte possono capitare degli errori durante la meiosi, che comportano:
    \begin{itemize}
        \item Alterazioni della \textbf{numero} dei cromosomi (\(\rightarrow\) mutazioni genomiche)
        \item Alterazioni nella \textbf{struttura} dei cromosomi (\(\rightarrow\) mutazioni cromosomiche)
    \end{itemize}
\end{snippet}

\end{document}