\documentclass[preview]{standalone}

\usepackage{amsmath}
\usepackage{amssymb}
\usepackage{stellar}
\usepackage{bettelini}

\hypersetup{
    colorlinks=true,
    linkcolor=black,
    urlcolor=blue,
    pdftitle={Stellar},
    pdfpagemode=FullScreen,
}

\begin{document}

\id{orlando-furioso-xxiii-testo}
\genpage

\section{Testo}

\begin{snippet}{orlando-furioso-ottava-100-xxiii}
    \StellarPoetry{100}{
        Lo strano corso che tenne il cavallo\\
        del Saracin pel bosco senza via,\\
        fece ch'Orlando andò duo giorni in fallo,\\
        né lo trovò, né poté averne spia.\\
        Giunse ad un rivo che parea cristallo,\\
        ne le cui sponde un bel pratel fioria,\\
        di nativo color vago e dipinto,\\
        e di molti e belli arbori distinto.
    }{L'imprevedibile percorso che prese il cavallo
    di Mandricardo per il bosco privo di sentieri
    fece si che Orlando vagò per due giorni a vuoto,
    né lo trovò, né ne ebbe traccia.
    Arrivò a un ruscello che sembrava cristallo,
    sulle cui sponde fioriva un bel prato
    dei colori della natura dipinto,
    e variamente ornato da molti bei cespugli.}
    \\\\
    La prima parte dell'ottava è narrativa, mentre la seconda descrittiva.
    Nella prima parte viene ripresa una vicenda di un tale Saracin.
    Ad ogni verso vi è un'espressione che indica il movimento labirintico.
    Il posto dove Orlando giunse è un locus amoenus.
\end{snippet}

\begin{snippet}{orlando-furioso-ottava-101-xxiii}
    \StellarPoetry{101}{
        Il merigge facea grato l'orezzo\\
        al duro armento ed al pastore ignudo;\\
        sì che né Orlando sentia alcun ribrezzo,\\
        che la corazza avea, l'elmo e lo scudo.\\
        Quivi egli entrò per riposarvi in mezzo;\\
        e v'ebbe \textbf{travaglioso} albergo e \textbf{crudo},\\
        e più che dir si possa \textbf{empio} soggiorno,\\
        quell'\textbf{infelice} e \textbf{sfortunato} giorno.
    }{La calda ora del mezzogiorno rendeva gradita l'ombra
    agli animali e al pastore nudo;
    così che neppure Orlando ebbe alcun esitazione,
    avendo la corazza, l'elmo e lo scudo.
    Qui Orlando entrò per riposare in mezzo ai cespugli
    e vi trovò una dimora angosciosa, funesta
    e più di quanto si possa dire,
    quell'infelice e sfortunato giorno.}
    \\\\
    Il merigge sono le ore più calde della giornata.
    Vi sono quindi le condizioni perfette, l'aria fresca si bilancia perfettamente con il sole.
    Grazie a queste condizioni, Orlando non senti nessun brivido nonostante sia barbato.
    Da questo punto Ariosto comincia a infliggere la sventura a Orlando.
    La seconda parte dell'ottava rovescia tutti gli aggettivi positivi della prima parte.
    La prima metà termina la descrizione del locus amoenus, la seconda metà è una anticipazione
    sulla futura sventura di Orlando in quel luogo (prolessi).
    La costruzione aggettivo - nome - aggettivo (al posto di nome aggettivo oppure i due aggetivi
    e a seguire il nome), creano una frase più elaborata ed elegante.
\end{snippet}

\begin{snippet}{orlando-furioso-ottava-102-xxiii}
    \StellarPoetry{102}{
        Volgendosi ivi intorno, vide scritti\\
        molti arbuscelli in su l'ombrosa riva.\\
        Tosto che fermi v'ebbe gli occhi e fitti,\\
        fu certo esser di man de la sua \textbf{diva}.\\
        Questo era un di quei lochi già descritti,\\
        ove sovente con Medor veniva\\
        da casa del pastore indi vicina\\
        la bella donna del Catai regina.
    }{Girando intorno vide incisi con scritte
    molti arberelli sulla riva dell' ombroso fiume.
    Non appena ebbe gli occhi fermi e fissi con maggior attenzione
    fu sicuro che furono scritti dalla dea del suo cuore.
    Questo era uno di quei luoghi già descritti,
    dove spesso Medoro veniva
    dalla vicina casa del pastore
    con la bella Angelica.}
    \\\\
    Fra tutte le strade che avrebbe potuto prendere, Orlando prende proprio
    quella dove Medoro e Angelica hanno inciso il loro nome, di cui si parla quattro canti prima.
    Dopo aver visto le scritte, ferma lo sguardo e le fissa.
    Orlando definisce Angelica come \textbf{diva}, a questo punto del testo la sta quindi idealizzando.
    È ancora convinto che Angelica sia la donna raggiungibile che Orlando vuole.
    Vi è un climax con la tensione degli occhi di orlando (\textbf{vide}, \textbf{fermi} e \textbf{fitti}).
    Abbiamo anche un intervento diretto dal narratore che ci indica la familiarità del luogo in questione.
\end{snippet}

\begin{snippet}{orlando-furioso-ottava-103-xxiii}
    \StellarPoetry{103}{
        Angelica e Medor con \textbf{cento} nodi\\
        legati insieme, e in \textbf{cento} lochi vede.\\
        Quante lettere son, tanti son chiodi\\
        coi quali Amore il cor gli punge e fiede.\\
        Va col pensier cercando in mille modi\\
        non \colorbox{orange}{creder} quel ch'al suo dispetto \colorbox{orange}{crede}:\\
        ch'altra Angelica sia, \colorbox{orange}{creder} si sforza,\\
        ch'abbia scritto il suo nome in quella scorza.
    }{Vede Angelica e Medoro in diversi modi,
    intrecciarti insieme ed in diversi luoghi.
    Tante sono le lettere, tanti sono i chiodi
    con i quali Cupido gli ferisce e punge il cuore.
    Va a cercare in mille modi con il pensiero
    di non credere quello a cui, suo malgrado, crede:
    si sforza di credere che sia un' altra Angelica
    ad aver scritto il suo nome sul quella corteccia.}
    \\\\
    Orlando spera che questo sia un caso di omonimia, e quindi non la sua Angelica ad aver inciso il
    suo nome.
    In realtà, Orlando sa che si tratta della regina del Catai, ma si sforza di credere
    diversamente. La realtà è talmente dolorosa, per amore e gelosia, che non vuole accettarla.

    \fcolorbox{black}{orange}{\rule{0pt}{5pt}\rule{5pt}{0pt}}
    La ripetizione di questa parola indica il suo pensiero ossessivo.
    Più orlando legge, più si ferisce.
\end{snippet}

\begin{snippet}{orlando-furioso-ottava-104-xxiii}
    \StellarPoetry{104}{
        Poi dice: - Conosco io pur queste note:\\
        di tal'io n'ho tante vedute e lette.\\
        Finger questo Medoro ella si puote:\\
        forse ch'a me questo cognome mette. -\\
        Con tali opinion dal ver remote\\
        usando fraude a sé medesmo, stette\\
        ne la speranza il malcontento Orlando,\\
        che si seppe a se stesso ir procacciando.
    }{Poi dice: “Io conosco la grafia di queste lettere:
    di queste (lettere) ne ho viste e ne ho lette tante.
    Potrebbe essersi inventata questo Medoro:
    forse mi ha dato questo soprannome”.
    Con tali opinioni remote,
    continuò ad assillare se stesso, ponendo
    il suo malcontento nella speranza
    che seppe procurare a se stesso.}
    \\\\
    Orlando riconosce la scrittura di Angelica.
    Forse il nome Medoro è un riferimento ad Orlando, un soprannome.
    Questo è il secondo tentativo della delusione e autoinganno.
\end{snippet}

\begin{snippet}{orlando-furioso-ottava-105-xxiii}
    \StellarPoetry{105}{
        Ma sempre più raccende e più rinuova,\\
        quanto spenger più cerca, il rio sospetto:\\
        come l'incauto augel che si ritrova\\
        in ragna o in visco aver dato di petto,\\
        quanto più batte l'ale e più si prova\\
        di disbrigar, più vi si lega stretto.\\
        Orlando viene ove s'incurva il monte\\
        a guisa d'arco in su la chiara fonte.
    }{Ma più si riaccende e si rinnova
    il crudele sospetto più cerca di dimenticarlo:
    come il disattento uccello che finisce
    in una ragnatela o sui rami invischiati,
    quanto più batte le ali e più prova
    a liberarsi, più si lega stretto.
    Orlando giunge dove si incurva la montagna
    come un arco (formando una grotta) sulla fonte cristallina}
    \\\\
    Questa ottava è divisa in tre parti.
    I primi due versi consistono in una lotta psicologica di Orlando.
    Più Orlando cerca di smettere di pensarci, più il pensiero diventa forte.
    Dal verso 3 al 6 abbiamo la similitudine dell'uccello, e infine, ai versi 
    7 e 8 Orlando giunge alla grotta di Angelica e Medoro.
\end{snippet}

\begin{snippet}{orlando-furioso-ottava-106-xxiii}
    \StellarPoetry{106}{
        Aveano in su l'entrata il luogo adorno\\
        coi piedi storti edere e viti erranti.\\
        Quivi soleano al più cocente giorno\\
        stare abbracciati i duo felici amanti.\\
        V'aveano i nomi lor dentro e d'intorno,\\
        più che in altro dei luoghi circostanti,\\
        scritti, qual con carbone e qual con gesso,\\
        e qual con punte di coltelli impresso
    }{Avevano ornato l'ingresso (di quella grotta)
    edere e viti rampicanti con i loro fusti contorti.
    Nei giorni più caldi, qui erano soliti
    stare abbracciati i due felici amanti.
    C'erano i loro nomi dentro ed intorno (alla grotta)
    più che nei luoghi circostanti,
    scritti alcuni con il carbone ed altri con gesso
    e altri erano impressi con punte di coltelli.}
    \\\\
    Orlando trova la grotta dove i due si erano amati durante le ore più calde,
    dove vi sono molte incisioni rappresentati i loro nomi.
    Questo è il luogo con la concentrazione più alta di scritte circa i due amanti
    tra tutti quelli incisi.
\end{snippet}

\begin{snippet}{orlando-furioso-ottava-107-xxiii}
    \StellarPoetry{107}{
        Il mesto conte a piè quivi discese;\\
        e vide in su l'entrata de la grotta\\
        parole assai, che di sua man distese\\
        Medoro avea, che parean scritte allotta.\\
        Del gran piacer che ne la grotta prese,\\
        questa sentenza in versi avea ridotta.\\
        Che fosse culta in suo linguaggio io penso;\\
        ed era ne la nostra tale il senso:
    }{Qui scese il triste cavaliere;
    e vide sull' entrata della grotta
    tante parole, che erano state scritte dalla mano di
    Medoro, e sembravano esser state scritte proprio in quel momento.
    Per esprimere il grande piacere che provò (con Angelica) nella grotta,
    aveva composto questa iscrizione in versi.
    Io penso che fosse poeticamente elaborata in arabo (lingua di Medoro),
    ed era tale il senso nella nostra lingua:}
    \\\\
    L'informazione nuova è che Medoro ha lasciato una scritta, quasi fresca,
    in cui descrive il piacere che ha avuto in quella grotta.
    Evidentemente, essendo musulmano, Medoro ha scritto in arabo.
    Alle ottave successive, Orlando traduce le scritte dall'arabo.
\end{snippet}

\begin{snippet}{orlando-furioso-ottava-108-xxiii}
    \StellarPoetry{108}{
        - Liete piante, verdi erbe, limpide acque,\\
        spelunca opaca e di fredde ombre grata,\\
        dove la bella Angelica che nacque\\
        di Galafron, da molti invano amata,\\
        spesso ne le mie braccia nuda giacque;\\
        de la commodità che qui m'è data,\\
        io povero Medor ricompensarvi\\
        d'altro non posso, che d'ognor lodarvi:
    }{“Liete piante, verdi erbe, limpide acque,
    grotta gradevole per la fresca ombra,
    dove la bella Angelica nacque
    di Galafron, è stata amata vanamente da molti,
    spesso nelle mie braccia giacque nuda;
    dei piaceri che qui mi sono stati dati,
    io povero Medoro non posso
    ricompensarvi in altro modo, se non lodandovi in ogni momento:}
    \\\\
    Questi versi sono molto molto espliciti e si rivolgono direttamente alla natura (locus amoenus).
\end{snippet}

\begin{snippet}{orlando-furioso-ottava-109-xxiii}
    \StellarPoetry{109}{
        e di pregare ogni signore amante,\\
        e cavallieri e damigelle, e ognuna\\
        persona, o paesana o viandante,\\
        che qui sua volontà meni o Fortuna;\\
        ch'all'erbe, all'ombre, all'antro, al rio, alle piante\\
        dica: benigno abbiate e sole e luna,\\
        e de le ninfe il coro, che proveggia\\
        che non conduca a voi pastor mai greggia. -
    }{e di pregare ogni signore che vi ha amato,
    e cavalieri e damigelle ed ogni
    persona, del posto o forestiere,
    che capiti qui intenzionalmente o per caso;
    che all'erba, all'ombra, all'ingresso (delle grotta), al fiume e alle piante
    dica: che sole e luna vi siano favorevoli,
    e vi protegga il coro delle ninfe
    dai danni che potrebbero recare le greggi condotte lì da qualche pastore.”}
    \\\\
    Orlando dà la traduzione del testo scritto in arabo.
    Medoro, lodando il luogo, lascia un messaggio a chi verrà dopo, in maniera tale
    da non deturpare questi luoghi e farli rimanere puri e limpidi.
    Spera che questi luoghi rimangano immacolati così come quando lui è stato con Angelica
    quella volta.
    Medoro è straordinariamente lucido, consapevole della sua fortuna, omaggioso
    alla natura e di una grande altezza poetica.
\end{snippet}

\begin{snippet}{orlando-furioso-ottava-110-xxiii}
    \StellarPoetry{110}{
        Era scritto in arabico, che 'l conte\\
        intendea così ben come latino:\\
        fra molte lingue e molte ch'avea pronte,\\
        prontissima avea quella il paladino;\\
        e gli schivò più volte e danni ed onte,\\
        che si trovò tra il popul saracino:\\
        ma non si vanti, se già n'ebbe frutto;\\
        ch'un danno or n'ha, che può scontargli il tutto.
    }{Era scritto in arabo, che il cavaliere
    capiva bene come il latino:
    tra molte lingue che conosceva,
    il paladino sapeva benissimo quella;
    e gli fece evitare più volte danni e scontri,
    quando si trovò tra il popolo saraceno:
    ma non si rallegri, se altre volte (la conoscenza dell'arabo) gli fu propizia;
    perché ora gli arreca un danno tale da cancellare tutti i vantaggi ottenuti.}
    \\\\
    Orlando conosce varie lingue, fra cui l'arabo molto bene.
    Questa conoscenza gli è servita varie volte in territorio nemico.
    Questa nuova consapevolezza contrasta il valore positivo di sapere la lingua, perché
    Orlando capisce esattamente il testo di Medoro.
    Medoro, in maniera completamente innocua, ha citato nel testo il fatto che
    Angelica sia amata invano da molti. Questa espressione provoca molto dolore a
    Orlando.
\end{snippet}

\begin{snippet}{orlando-furioso-ottava-111-xxiii}
    \StellarPoetry{111}{
        \textbf{Tre} volte e \textbf{quattro} e \text{sei} lesse lo scritto\\
        quello infelice, e pur cercando invano\\
        che non vi fosse quel che v'era scritto;\\
        e sempre lo vedea più chiaro e piano:\\
        ed ogni volta in mezzo il petto afflitto\\
        stringersi il cor sentia con fredda mano.\\
        Rimase al fin con gli occhi e con la mente\\
        \textbf{fissi} nel sasso, al sasso indifferente.
    }{Lesse tre, quattro, sei volte la triste poesia
    l'infelice, ed anche cercando invano (di immaginare)
    che non ci fosse ciò che vi era scritta;
    ma gli risultava sempre più chiaro e facile da comprendere:
    ed ogni volta (che leggeva) si sentiva in mezzo al petto afflitto
    stringere il cuore con mano gelida.
    Rimase lì con gli occhi e con il pensiero
    rivolti al sasso, impietrito.}
    \\\\
    Orlando legge il testo svariate volte nella speranza di trovare un dettaglio
    per alimentare la propria delusione, ma la sua ossessione lo persuade
    e non riesce ad ingannarsi più.
    Anche l'aggettivo \textbf{fissi} incanala l'ossessione di Orlando, con i suoi occhi
    e con la sua mente, verso il testo inciso sul sasso.
\end{snippet}

\begin{snippet}{orlando-furioso-ottava-112-xxiii}
    \StellarPoetry{112}{
        Fu allora per uscir del sentimento\\
        sì tutto in preda del dolor si lassa.\\
        Credete a chi n'ha fatto esperimento,\\
        che questo è 'l duol che tutti gli altri passa.\\
        Caduto gli era sopra il petto il mento,\\
        la fronte priva di baldanza e bassa;\\
        né poté aver (che 'l duol l'occupò tanto)\\
        alle querele voce, o umore al pianto.
    }{Fu allora che inizio ad impazzire,
    così che in preda al dolore si abbandona completamente.
    Credete a chi lo ha provato su se stesso,
    che questa, d'amore, è la sofferenza che fa passare tutte le altre.
    Gli era caduto il mento sopra il petto (testa bassa),
    la fronte era priva di rughe ed era bassa;
    non poté aver e (che il dolore l'occupò tanto)
    voce per lamentarsi o lacrime per piangere.}
    \\\\
    Il narratore indica di aver provato per esperienza questo dolore, il dolore
    più forte di tutti, da cui nessuno è al riparo (versi 3-4, intervento diretto del narratore).
    Lo prova il narratore e lo prova pure il più grande dei cavalieri.
    Orlando comincia quindi a perdere il proprio senno.
\end{snippet}

\begin{snippet}{orlando-furioso-ottava-113-xxiii}
    \StellarPoetry{113}{
        L'impetuosa doglia entro rimase,\\
        che volea tutta uscir con troppa fretta.\\
        Così veggiàn restar l'acqua nel vase,\\
        che largo il ventre e la bocca abbia stretta;\\
        che nel voltar che si fa in su la base,\\
        l'umor che vorria uscir, tanto s'affretta,\\
        e ne l'angusta via tanto s'intrica,\\
        ch'a goccia a goccia fuore esce a fatica.
    }{Rimase dentro l'impetuoso dolore,
    che voleva uscire con troppa fretta.
    Così vediamo restare l'acqua nel vaso,
    che abbia largo il ventre e stretta la bocca;
    così ché, capovolgendo il vaso,
    l liquido che vorrebbe uscire, tanto velocemente si riversa,
    e si ingorga nella stretta apertura,
    uscendo così goccia a goccia, a fatica.}
    \\\\
    Il dolore viene descritto con la similitudine di questa ottava.
    La similitudine dell'anfora è una similitudine domestica e quotidiana,
    per essere comprensibile a tutti e quindi per far capire che questa
    storia possa capitare a chiunque.
\end{snippet}

\begin{snippet}{orlando-furioso-ottava-114-xxiii}
    \StellarPoetry{114}{
        Poi ritorna in sé alquanto, e pensa come\\
        possa esser che non sia la cosa vera:\\
        che voglia alcun così infamare il nome\\
        de la sua donna e \colorbox{orange}{crede} e \colorbox{orange}{brama} e \colorbox{orange}{spera},\\
        o gravar lui d'insopportabil some\\
        tanto di gelosia, che se ne pera;\\
        ed abbia quel, sia chi si voglia stato,\\
        molto la man di lei bene imitato.
    }{Poi ritorna abbastanza in sé, e pensa se
    la cosa potrebbe essere non vera:
    che qualcuno voglia così infamare il nome
    della sua donna, e crede e spera e brama,
    oppure (che qualcuno voglia) gravarlo di un così insopportabile peso
    di gelosia, da farlo morire;
    e abbia, chiunque sia stato,
    imitato molto bene la sua calligrafia (di Angelica).}
    \\\\
    Come ultimo tentativo di delusione, Orlando crede che qualcosa possa avere imitato
    la grafia di Angelica, o per infangarle il nome, o per farlo impazzire volontariamente.
    L'anticlimax mostra come Orlando non creda nemmeno lui a ciò che pensa,
    infatti inizialmente crede fermamente che ciò non sia vero, poi
    lo brama e infine lo spera.
\end{snippet}

\begin{snippet}{orlando-furioso-ottava-115-xxiii}
    \StellarPoetry{115}{
        In così poca, in così debol speme\\
        sveglia gli spiriti e gli rifranca un poco;\\
        indi al suo Brigliadoro il dosso preme,\\
        dando già il sole alla sorella loco.\\
        Non molto va, che da le vie supreme\\
        dei tetti uscir vede il vapor del fuoco,\\
        sente cani abbaiar, muggiare armento:\\
        viene alla villa, e piglia alloggiamento.
    }{Con una così debole speranza,
    gli si rianimarono gli spiriti vitali;
    quindi salì in groppa al suo Brigliadoro
    quando il sole stava già lasciando il posto a sua sorella luna (tramonto).
    Non va molto avanti, che dagli alti comignoli
    dei tetti vede uscire del fumo,
    sente cani abbaiare e una mandria muggire:
    va fino alla villa e prende posto.}
    \\\\
    Cala la sera e Orlando si sposta nella direzione dove intravvede la vita, e va a chiedere
    ospitalità per la notte a venire.
\end{snippet}

\begin{snippet}{orlando-furioso-ottava-116-xxiii}
    \StellarPoetry{116}{
        Languido smonta, e lascia Brigliadoro\\
        a un discreto garzon che n'abbia cura;\\
        altri il disarma, altri gli sproni d'oro\\
        gli leva, altri a forbir va l'armatura.\\
        Era questa la casa ove Medoro\\
        giacque ferito, e v'ebbe alta avventura.\\
        Corcarsi Orlando e non cenar domanda,\\
        di dolor sazio e non d'altra vivanda.
    }{Debole smonta (da cavallo), e lascia Brigliadoro
    a un abile garzone perché ne abbia cura:
    si fa disarmare da uno, gli sperono d'oro un altro
    gli leva, e si fa lucidare l'armatura da un altro ancora.
    Era questa la casa dove Medoro
    visse quando fu ferito, e dove ebbe grande fortuna.
    Orlando chiede solo da dormire e niente per cena,
    è sazio di dolore e non di altro cibo.}
    \\\\
    Orlando è distrutto e, venendo privato di cavallo, armatura, speroni d'oro (e di Angelica),
    torna ad essere raffigurato come uomo e non come cavaliere (simbolicamente);
    Il grande eroe, di fronte a questo dolore, regredisce.
\end{snippet}

\begin{snippet}{orlando-furioso-ottava-117-xxiii}
    \StellarPoetry{117}{
        Quanto più cerca ritrovar quiete,\\
        tanto ritrova più travaglio e pena;\\
        che de l'odiato scritto ogni parete,\\
        ogni uscio, ogni finestra vede piena.\\
        Chieder ne vuol: poi tien le labra chete;\\
        che teme non si far troppo serena,\\
        troppo chiara la cosa che di nebbia\\
        cerca offuscar, perché men nuocer debbia.
    }{Quanto più cerca di trovare tranquillità,
    tanto più prova travaglio e dolore;
    vede piena della odiata poesia (quella scritta da Medoro) ogni parete
    ogni finestra, ogni porta.
    Vorrebbe chiedere a riguardo ma poi tiene le labbra ferme (sta zitto);
    perché teme di rendere (a se stesso) troppo evidente,
    troppo chiara la cosa che
    cerca di dimenticare (offuscare), per provare meno dolore.}
    \\\\
    Orlando vorrebbe chiedere al pastore chi siano i due che hanno fatto le incisioni,
    ma si frena dal farlo in quanto teme che la nebbia che lo distacca dalla realtà svanisca.
    Ogni singola superficie della casa è incisa con la scritta \quotes{Angelica e Medoro}.
\end{snippet}

\begin{snippet}{orlando-furioso-ottava-118-xxiii}
    \StellarPoetry{118}{
        Poco gli giova usar fraude a se stesso;\\
        che senza domandarne, è chi ne parla.\\
        Il pastor che lo vede così oppresso\\
        da sua tristizia, e che voria levarla,\\
        l'istoria nota a sé, che dicea spesso\\
        di quei duo amanti a chi volea ascoltarla,\\
        ch'a molti dilettevole fu a udire,\\
        gl'incominciò senza rispetto a dire:
    }{Ingannare se stesso non gli giova;
    perché senza domandare (dell'accaduto) c'è chi ne parla.
    Il pastore, che lo vede così oppresso
    dalla sua tristezza, e vorrebbe alleviarla,
    iniziò a raccontargli la storia che conosceva bene; raccontava spesso
    dei due amanti a chi voleva ascoltare
    una storia molto dilettevole,
    e così, senza rispetto, cominciò a raccontare}
    \\\\
    Il pastore, vedendolo triste, prova a rincuorargli la bella storia di Angelica e Medoro.
\end{snippet}

\begin{snippet}{orlando-furioso-ottava-119-xxiii}
    \StellarPoetry{119}{
        come esso a prieghi d'Angelica bella\\
        portato avea Medoro alla sua villa,\\
        ch'era ferito gravemente; e ch'ella\\
        curò la piaga, e in pochi dì guarilla:\\
        ma che nel cor d'una maggior di quella\\
        lei ferì Amor; e di poca scintilla\\
        l'accese tanto e sì cocente fuoco,\\
        che n'ardea tutta, e non trovava loco:
    }{come egli, pregato dalla bella Angelica,
    aveva portato in casa sua Medoro,
    ferito gravemente; e che ella (Angelica)
    curò la ferita ed in pochi giorni la guarì:
    ma lei, con una piaga ancora maggiore di quella, nel cuore
    fu ferita da Amore (cupido); e da una piccola scintilla
    si accese tanto del così cocente fuoco,
    che la faceva ardere tutta, e non trovava pace:}
\end{snippet}

\begin{snippet}{orlando-furioso-ottava-120-xxiii}
    \StellarPoetry{120}{
        e sanza aver rispetto ch'ella fusse\\
        figlia del maggior re ch'abbia il Levante,\\
        da troppo amor costretta si condusse\\
        a farsi moglie d'un povero fante.\\
        All'ultimo l'istoria si ridusse,\\
        che 'l pastor fe' portar la gemma inante,\\
        ch'alla sua dipartenza, per mercede\\
        del buono albergo, Angelica gli diede.
    }{e senza aver riguardo che ella (Angelica) fosse
    figlia del più grande re che abbia mai avuto l'oriente,
    sospinta da un grandissimo amore fu portata
    a sposare Medoro, umile soldato.
    La conclusione della storia fu
    che il pastore mostrò ad Orlando il gioiello,
    che al momento della partenza, come ricompensa
    della buona ospitalità, gli diede Angelica.}
    \\\\
    Il pastore narra la storia di Angelica e Medoro a Orlando.
    In segno di dimostrazione che ciò che diceva fosse vero
    il pastore mostra il bracciale che Angelica gli mette in
    pegno in segno di gratitudine per il loro lungo soggiorno.

    C'è inoltre un'analogia tra il narratore che narra la storia principale
    al lettore e il pastore che narra la storia della coppia ad Orlando.
    Il cambio di narratore serve a sottolineare come una storia
    simile a quella vissuta da Orlando possa toccare chiunque.
\end{snippet}

\begin{snippet}{orlando-furioso-ottava-121-xxiii}
    \StellarPoetry{121}{
        Questa conclusion fu la secure\\
        che 'l capo a un colpo gli levò dal collo,\\
        poi che d'innumerabil battiture\\
        si vide il manigoldo Amor satollo.\\
        Celar si studia Orlando il duolo; e pure\\
        quel gli fa forza, e male asconder pòllo:\\
        per lacrime e suspir da bocca e d'occhi\\
        convien, voglia o non voglia, al fin che scocchi.
    }{Questa conclusione fu la scure
    che gli levò la testa dal collo in un colpo solo,
    una volta che delle innumerevoli bastonate
    fu sazio il carnefice Amore.
    Orlando si sforza di nascondere il dolore; e tuttavia
    quello è talmente violento che difficilmente lo può tenere nascosto:
    attraverso le lacrime degli occhi ed i sospiri della bocca
    è inevitabile che esploda.}
    \\\\
    Orlando, vedendo il bracciale che in passato lui aveva donato in
    segno d'amore ad Angelica nelle mani del pastore, perde definitivamente
    tutta la minima speranza che gli era rimasta in corpo
    dell'amore di Angelica per lui.

    Al verso 4 si fa riferimento all'immagine di Amore (personificato)
    che dopo questa delusione amorosa è sazio della sofferenza di Orlando.
\end{snippet}

\begin{snippet}{orlando-furioso-ottava-122-xxiii}
    \StellarPoetry{122}{
        Poi ch'allargare il freno al dolor puote\\
        (che resta solo e senza altrui rispetto),\\
        giù dagli occhi rigando per le gote\\
        sparge un fiume di lacrime sul petto:\\
        sospira e geme, e va con spesse ruote\\
        di qua di là tutto cercando il letto;\\
        e più duro ch'un sasso, e più pungente\\
        che se fosse d'urtica, se lo sente.
    }{Dopo che poté dar libero sfogo al dolore
    (perché resta solo senza doversi preoccupare di nessun altro),
    dagli occhi, rigando le guance
    sparge un fiume di lacrime sul petto:
    sospira e piange, e cammina, girandosi spesso,
    di qua e di là esplorando il letto:
    e più duro che un sasso, e più pungente
    dell'ortica se lo sente.}
    \\\\
    Orlando si trattiene le lacrime per non farsi vedere debole
    in compagnia del pastore, sfogandosi successivamente in
    stanza da solo. Il letto su cui dorme gli sembra orticante
    e duro, facendolo dunque girare e rigirare freneticamente
    e impedendogli di dormire.\\
    Al verso 4 è presente un'iperbole che indica la quantità
    abbondante di lacrime che Orlando versa una volta solo.
\end{snippet}

\begin{snippet}{orlando-furioso-ottava-123-xxiii}
    \StellarPoetry{123}{
        In tanto aspro travaglio gli soccorre\\
        che nel medesmo letto in che giaceva,\\
        l'ingrata donna venutasi a porre\\
        col suo drudo più volte esser doveva.\\
        Non altrimenti or quella piuma abborre,\\
        né con minor prestezza se ne leva,\\
        che de l'erba il villan che s'era messo\\
        per chiuder gli occhi, e vegga il serpe appresso.
    }{In tanto gli viene in mente l'atroce dubbio
    che nello stesso letto in cui egli (Orlando) giaceva,
    l'ingrata donna (Angelica) a coricare
    doveva essersi più volte venuta insieme al suo amante.
    Inevitabilmente ha in odio quel letto,
    né si alza dal letto meno velocemente
    del contadino che si leva dall'erba su cui si era steso
    per riposarsi, per aver visto vicino a sé un serpente.}
    \\\\
    Torna in mente a Orlando che nello stesso letto in cui giaceva,
    Angelica si era concessa a Medoro molte volte.
    Quando si rende conto di questo Orlando scatta fuori dal letto
    \quotes{come un contadino che si trova davanti un serpente e salta via} (vv. 7-8).

    Si nota come da questo evento Angelica venga descritta come ingrata e
    non più come una Dea, mentre a Medoro venga associato l'aggettivo di
    drudo, ossia un amante non legittimo (molto dispregiativo).
\end{snippet}

\begin{snippet}{orlando-furioso-ottava-124-xxiii}
    \StellarPoetry{124}{
        Quel letto, quella casa, quel pastore\\
        immantinente in tant'odio gli casca,\\
        che senza aspettar luna, o che l'albore\\
        che va dinanzi al nuovo giorno nasca,\\
        piglia l'arme e il destriero, ed esce fuore\\
        per mezzo il bosco alla più oscura frasca;\\
        e quando poi gli è aviso d'esser solo,\\
        con gridi ed urli apre le porte al duolo.
    }{Quel letto, quella casa, quel pastore
    immediatamente gli viene in tanto odio,
    che senza aspettare che sorga la luna o che l'alba,
    che precede il nuovo giorno, nasca,
    prende le armi e il destriero, ed esce fuori
    in mezzo al bosco, dove è più fitto e scuro l'intrico di rami;
    e poi quando si accorge di essere solo (che nessuno lo segue),
    con grida e urla apre le porte al dolore.}
    \\\\
    Orlando lascia la casa, si riappropria dei suoi averi e 
    al galoppo si reca nella zona più oscura del bosco per
    potersi aprire ai suoi sentimenti il più remoto possibile
    dalle altre persone.

    Al verso 1 si hanno pronomi ricorrenti \quotes{Quel [\dots] quella [\dots] quel [\dots]}
    che dimostrano ancora una volta l'ossessione di Orlando al
    pensiero della vicenda.
\end{snippet}

\begin{snippet}{orlando-furioso-ottava-125-xxiii}
    \StellarPoetry{125}{
        Di pianger mai, mai di gridar non resta;\\
        né la notte né 'l dì si dà mai pace.\\
        Fugge cittadi e borghi, e alla foresta\\
        sul terren duro al discoperto giace.\\
        Di sé si meraviglia ch'abbia in testa\\
        una fontana d'acqua sì vivace,\\
        e come sospirar possa mai tanto;\\
        e spesso dice a sé così nel pianto:
    }{Non smette mai di gridare e di urlare;
    non si dà mai pace né la notte né il giorno seguente.
    Fugge da città e da borghi, e nei luoghi inabitati
    sul terreno duro, all'aperto, giace.
    Si meraviglia che nella propria testa ci possa essere
    una sorgente così inesauribile di pianto,
    e come i sospiri possano essere mai così tanti;
    e spesso si dice nel pianto:}
    \\\\
    Ai versi 1 e 2 si ha la coordinata temporale dell'ottava,
    la quale indica che il dolore che prova non è momentaneo ma
    ciclico.
    Ai versi 3 e 4 si ha la coordinata spaziale, la quale mostra
    il percorso che fa Orlando mentre fugge. 
    Infine agli ultimi quattro versi si ha un'introspezione di Orlando
    e si interroga sul come sia possibile che lui pianga e sospiri
    così tanto. 
\end{snippet}

\begin{snippet}{orlando-furioso-ottava-126-xxiii}
    \StellarPoetry{126}{
        «Queste non son più lacrime, che fuore\\
        stillo dagli occhi con sì larga vena.\\
        Non suppliron le lacrime al dolore:\\
        finir, ch'a mezzo era il dolore a pena.\\
        Dal fuoco spinto ora il vitale umore\\
        fugge per quella via ch'agli occhi mena;\\
        ed è quel che si versa, e trarrà insieme\\
        e 'l dolore e la vita all'ore estreme.
    }{“Queste non sono più lacrime, che fuoriescono
    dagli occhi con flusso così abbondante.
    Non bastarono le lacrime al dolore:
    finirono quando il dolore si era manifestato solo per metà.
    Dal dolore della gelosia ora l'umor vitae
    fugge attraverso quella via a cui gli occhi conducono;
    ed e' quello che ne esce ora, quello che porterà via con sé insieme
    il dolore e la vita sul punto di morte.}
    \\\\
    Si ha un dialogo interiore di Orlando, il quale si risponde alla
    domanda precedente e si interroga sulle sue lacrime.
    La strofa è bipartita a metà, dove nella prima parte si ha
    la negazione della realtà, ossia che è impossibile che
    Orlando stia versando lacrime poiché non ne ha più da versare,
    mentre nella seconda parte si ha l'affermazione alla negazione
    precedente, ossia che sta piangendo il suo umore vitale.
    In antichità, nella medicina, si credeva che l'umore di una
    persona fosse un fluido interno al corpo.
\end{snippet}

\begin{snippet}{orlando-furioso-ottava-127-xxiii}
    \StellarPoetry{127}{
        Questi ch'indizio fan del mio tormento,\\
        sospir non sono, né i sospir sono tali.\\
        Quelli han triegua talora; io mai non sento\\
        che 'l petto mio men la sua pena esali.\\
        Amor che m'arde il cor, fa questo vento,\\
        mentre dibatte intorno al fuoco l'ali.\\
        Amor, con che miracolo lo fai,\\
        che 'n fuoco il tenghi, e nol consumi mai?
    }{Questi, che manifestano il mio tormeno,
    non sono sospiri, né i sospiri sono così.
    I sospiri ogni tanto si interrompono; io non sento mai
    il mio petto ridurre il sospirare per la pena.
    L'amore che mi arde il cuore crea questi sospiri
    mentre agita attorno al fuoco le proprie ali.
    Amore, con quale miracolo riesci
    a tenerlo (il cuore) nel fuoco senza mai consumarlo?}
    \\\\
    Si ha la struttura dell'ottava precedente e si interroga
    sul suo sospiro:
    nella prima metà si ha la negazione, ossia che
    è impossibile che lui stia sospirando poiché sospiro
    significa che ogni tanto hanno tregua, mentre i suoi
    tregua non hanno.
    Nella seconda metà si ha l'affermazione, ossia che
    quei \quotes{sospiri} fossero Amore (personificato)
    che, sbattendo le ali, alimenta il fuoco che fa
    ardere il suo cuore sempre di più.

    In questa ottava non è Orlando che parla, bensì il suo
    Amore, poiché una frase così artificiale non sarebbe mai
    potuta essere pronunciata da un Orlando in tale stato.
\end{snippet}

\begin{snippet}{orlando-furioso-ottava-128-xxiii}
    \StellarPoetry{128}{
        Non son, non sono io quel che paio in viso:\\
        quel ch'era Orlando è morto ed è sotterra;\\
        la sua donna ingratissima l'ha ucciso:\\
        sì, mancando di fé, gli ha fatto guerra.\\
        Io son lo spirto suo da lui diviso,\\
        ch'in questo inferno tormentandosi erra,\\
        acciò con l'ombra sia, che sola avanza,\\
        esempio a chi in Amor pone speranza.»
    }{Non sono io, non sono io quello che sembro in volto:
    quello che era Orlando è morto e sotterrato;
    la sua ingrata donna l'ha ucciso:
    si, mancandogli di fedeltà gli ha fatto la guerra.
    Io sono il suo spirito dal suo corpo diviso,
    che vaga tormentandosi in questo inferno,
    in modo che con il proprio fantasma, che e' tutto quello che gli resta, ammonisca con l'esempio colui che affida la sua speranza nell'Amore.”}
    \\\\
    In questa ottava si hanno i primi indizi concreti che
    Orlando sta impazzendo.
    Nella prima metà si ha la negazione sulla sua identità,
    ossia che non è più Orlando, il quale è morto e sepolto
    per mano di un'Angelica ingratissima e sleale.
    Nella seconda metà si ha l'affermazione, ossia
    che lui è ciò che rimane dell'anima di Orlando, la quale
    si trova all'Inferno terrestre per colpa di questo amore.

    I tre punti che indicano l'ascesa alla pazzia di Orlando sono:
    \begin{enumerate}
        \item parla in terza persona: \quotes{Orlando è morto};
        \item climax della visione di Angelica; da Diva a Ingrata a Ingratissima.
            Ciò rappresenta un distacco di Orlando sempre più grande dalla realtà;
        \item Orlando è fermamente convinto che Angelica lo abbia slealmente tradito,
            quando è evidente che questa relazione non sia mai esistita.
    \end{enumerate}
\end{snippet}

\begin{snippet}{orlando-furioso-ottava-129-xxiii}
    \StellarPoetry{129}{
        Pel bosco errò tutta la notte il conte;\\
        e allo spuntar de la diurna fiamma\\
        lo tornò il suo destin sopra la fonte\\
        dove Medoro isculse l'epigramma.\\
        Veder l'ingiuria sua scritta nel monte\\
        l'accese sì, ch'in lui non restò dramma\\
        che non fosse odio, rabbia, ira e furore;\\
        né più indugiò, che trasse il brando fuore.
    }{Tutta la notte il conte vagò per il bosco;
    ed al sorgere del sole
    il suo destino lo riportò vicino al fiume
    dove Medoro incise l'iscrizione.
    Vedere le parole che testimoniavano il suo disonore incise nel monte,
    lo accese, così che in lui non restò nulla
    che non fosse odio, rabbia, ira o furia;
    non resistette più e sguainò la spada.}
    \\\\
    Orlando ritorna casualmente alla grotta con le scritte incise da Medoro e,
    con la spada tratta, le distrugge.

    A verso 7 si ha un climax, il quale dà il titolo al libro:
    \quotes{odio, rabbia, ira e furore;}. Si ha dunque per la prima volta
    il nome \quotes{Furia} nell'opera.
\end{snippet}

\begin{snippet}{orlando-furioso-ottava-130-xxiii}
    \StellarPoetry{130}{
        Tagliò lo scritto e 'l sasso, e sin al cielo\\
        a volo alzar fe' le minute schegge.\\
        Infelice quell'antro, ed ogni stelo\\
        in cui Medoro e Angelica si legge!\\
        Così restar quel dì, ch'ombra né gielo\\
        a pastor mai non daran più, né a gregge:\\
        e quella fonte, già si chiara e pura,\\
        da cotanta ira fu poco sicura;
    }{Tagliò l'incisione e il sasso, e fino al cielo
    fece volare le piccole schegge.
    Infelice sia ogni grotta e ogni tronco
    in cui si leggono i nomi di Medoro ed Angelica!
    Furono così ridotte (le piante) quel giorno, che né ombra né refrigerio daranno più al pastore né al suo gregge:
    e il fiume, così chiaro e puro,
    non fu al riparo da un ira così grande;}
    \\\\
    Con la spada Orlando taglia tutte le scritte e
    deturpa il luogo, distruggendo la grotta, gli alberi
    e la limpidezza del fiume, inquinandone le acque per sempre.
    Ai versi 1 e 2 si ha una scena iperbolica violenta:
    \quotes{'l sasso, e sin al cielo \bfslash a volo alzar}.
\end{snippet}

\begin{snippet}{orlando-furioso-ottava-131-xxiii}
    \StellarPoetry{131}{
        che rami e ceppi e tronchi e sassi e zolle\\
        non cessò di gittar ne le bell'onde,\\
        fin che da sommo ad imo sì turbolle\\
        che non furo mai più chiare né monde.\\
        E stanco al fin, e al fin di sudor molle,\\
        poi che la lena vinta non risponde\\
        allo sdegno, al grave odio, all'ardente ira,\\
        cade sul prato, e verso il ciel sospira.
    }{poiché i rami, i tronchi, i sassi e le zolle di terra
    Orlando non smise di gettare nelle belle onde,
    fino a che dalla superficie fino al fondo, le rese torbide
    così tanto che non saranno mai più così limpide e pure.
    E alla fine, stanco e sudato,
    dal momento che la forza fisica, esaurita, non era più in grado di servire
    allo sdegno, al pesante odio e all'ardente ira,
    si abbandona sul prato e sospira al cielo.}
    \\\\
    Quest'ottava è bipartita a metà, dove la prima prima parte è
    strettamente legata all'ottava 130 e nella quale si ha un
    polisindeto che indica la rapidità e la furia di Orlando
    \quotes{che rami e ceppi e tronchi e sassi e zolle} (v.1),
    mentre la seconda parte è strettamente legata all'ottava 132 e
    nella quale si ha una pausa dovuta alla spossatezza e immobilità di
    Orlando. Infatti, la vicenda da dinamica diventa statica poiché il
    narratore utilizza i seguenti sintagmi: \quotes{stanco, sudor molle, lena, cadde}.
\end{snippet}

\begin{snippet}{orlando-furioso-ottava-132-xxiii}
    \StellarPoetry{132}{
        Afflitto e stanco al fin cade ne l'erba,\\
        e ficca gli occhi al cielo, e non fa motto.\\
        Senza cibo e dormir così si serba,\\
        che 'l sole esce tre volte e torna sotto.\\
        Di crescer non cessò la pena acerba,\\
        che fuor del senno al fin l'ebbe condotto.\\
        Il quarto dì, da gran furor commosso,\\
        e maglie e piastre si stracciò di dosso.
    }{Afflitto e stanco cadde infine nell'erba
    e fissò gli occhi al cielo senza dire parola alcuna.
    Rimane così, senza mangiare e senza dormire
    per tre giorni.
    Il suo dolore non smise di crescere,
    finché non l'ebbe fatto impazzire.
    Il quarto giorno, sconvolto dalla pazzia violenta,
    si tolse di dosso tutta l'armatura.}
    \\\\
    Fra le ottave 130-131 (bipartita) e le ottave 131 (bipartita)-132 non si hanno più
    i classici confini delle ottave, bensì analogamente alla storia è come se fossero
    scritte due strofe da 12 versi.

    Abbiamo una pausa di tre giorni in cui Orlando giace a terra guardando il cielo,
    non avendo nemmeno più la forza di parlare.
    Nonostante ciò, la sofferenza non smette di crescere e i versi sono ricchi di espressioni
    che indicano la sua stanchezza.
    \\
    Il quarto giorno la furia irrompe nuovamente, ancora più forte di prima, strappandosi
    tutti i vestiti di dosso.
\end{snippet}

\begin{snippet}{orlando-furioso-ottava-133-xxiii}
    \StellarPoetry{133}{
        Qui riman l'elmo, e là riman lo scudo,\\
        lontan gli arnesi, e più lontan l'usbergo:\\
        l'arme sue tutte, in somma vi concludo,\\
        avean pel bosco differente albergo.\\
        E poi si squarciò i panni, e mostrò ignudo\\
        l'ispido ventre e tutto 'l petto e 'l tergo;\\
        e cominciò la gran follia, sì orrenda,\\
        che de la più non sarà mai ch'intenda.
    }{Qui resta l'elmo e là resta lo scudo,
    lontano gli arnesi (corredo dell'armatura), e più lontano ancora la corazza: tutte le sue armi, concludendo,
    avevano ognuna diversa collocazione per il bosco.
    E poi si squarciò i vestiti, e rimasero nudi
    il peloso addome e la schiena;
    e iniziò la grande pazzia, così orrenda,
    che nessuno sentirà mai parlare di una (follia) più orrenda di questa.}
    \\\\
    Questa ottava può essere divisa in tre parti:
    Orlando butta tutte le sue armi in luoghi diversi, si strappa i vestiti
    e comincia la follia più grande mai vista.
    % ispido = peloso
    \\\\
    Come quando Orlando si era fatto aiutare a togliersi l'armatura, smette di essere un cavaliere,
    ma adesso è lui stesso che se la strappa di dosso.
    Il suo ventre e schiena sono pelosi come quella di una bestia,
    smettendo quasi di essere un uomo, puramente istintivo e senza capacità di ragionamento complesso (climax).
    Questo episodio (§§129-133) è quindi quello che dona il titolo all'intera opera.
    Gli ultimi due versi dell'ottava presentano un intervento dell'autore, il quale con un iperbole,
    annuncia che la follia di Orlando sia la più grande follia mai esistita.

    A differenza dell'ottava 116, nella quale erano altri uomini ad aiutare Orlando nello
    svestirsi, in questa ottava è Orlando stesso che, per pazzia e furia, se li leva di dosso
    e li lancia uno in un posto diverso dall'altro.
\end{snippet}

\begin{snippet}{orlando-furioso-ottava-134-xxiii}
    \StellarPoetry{134}{
        In tanta rabbia, in tanto furor venne,\\
        che rimase offuscato in ogni senso.\\
        Di tor la spada in man non gli sovenne;\\
        che fatte avria mirabil cose, penso.\\
        Ma né quella, né scure, né bipenne\\
        era bisogno al suo vigore immenso.\\
        Quivi fe' ben de le sue prove eccelse,\\
        ch'un alto pino al primo crollo \textbf{svelse}:
    }{Gli scaturì così tanta rabbia e così tanto furore
    che tutte le sue facoltà sensitive furono alterate.
    Non gli passò per la testa di prendere la spada,
    che tante incredibili avventure aveva passato, credo.
    Ma tanto né quella, né una scure, né una bipenne (scure a due lame)
    erano necessarie alla sua immensa forza.
    Qui fece davvero alcune tra le sue imprese più straordinarie,
    sradicò un grande pino con un solo scrollone:}
    \\\\
    La follia di Orlando irrompe e lui ne è completamente offuscato.
    Le sue armi non gli sarebbero servite, solo la sua forza bruta era necessaria a sfogare la sua ira.
    Vengono presentate molte iperbole come come quella di sdradicare un alto pino con le mani.
\end{snippet}

\begin{snippet}{orlando-furioso-ottava-135-xxiii}
    \StellarPoetry{135}{
        e \textbf{svelse} dopo il primo altri parecchi,\\
        come fosser finocchi, ebuli o aneti;\\
        e fe' il simil di querce e d'olmi vecchi,\\
        di faggi e d'orni e d'illici e d'abeti.\\
        Quel ch'un ucellator che s'apparecchi\\
        il campo mondo, fa, per por le reti,\\
        dei giunchi e de le stoppie e de l'urtiche,\\
        facea de cerri e d'altre piante antiche.
    }{e ne abbatté, dopo il primo, molti altri ancora
    come se fossero state piante dal fusto tenero;
    e fece la stessa cosa con querce, vecchi olmi,
    faggi e abeti.
    Come un uccellatore che per ripulire
    il campo, dove mettere le reti,
    estirpa le erbaccie, i ramoscelli e le ortiche,
    Orlando faceva con le querce e con le altre piante secolari del bosco.}
    \\\\
    Vi è una ripresa per anadiplosi (§§134-135).
    La prima metà dell'ottava è dedicata a una similitudine che vede
    Orlando deforestare con la stessa facilità con cui si colgono gli ortaggi.
    Anche la seconda metà è dedicata a una similitudine, ossia come il cacciatore di uccelli
    ripulisce il terreno per posare le reti e le resine, con la medesima facilità Orlando deforesta tutto
    attorno a lui.
\end{snippet}

\begin{snippet}{orlando-furioso-ottava-136-xxiii}
    \StellarPoetry{136}{
        I pastor che sentito hanno il fracasso,\\
        lasciando il gregge sparso alla foresta,\\
        chi di qua, chi di là, tutti a gran passo\\
        vi vengono a veder che cosa è questa.\\
        Ma son giunto a quel segno il qual s'io passo\\
        vi potria la mia istoria esser molesta;\\
        ed io la vo' più tosto diferire,\\
        che v'abbia per lunghezza a fastidire.
    }{I pastori che avevano sentito il gran chiasso,
    lasciando il gregge sparso per la foresta,
    da ogni luogo, di corsa
    vanno a vedere che cosa fosse quel rumore.
    Ma sono giunto a quel punto che se lo oltrepasso,
    la mia storia vi potrebbe essere dannosa;
    e io la voglio rinviare ad un altro canto
    prima che vi possa infastidire per la sua lunghezza.}
    \\\\
    Fino a metà dell'ottava si parla dei pastori che, in lontananza, sentono
    il casino dato dalla rabbia di Orlando e accorrono a controllare. 
    L'ottava si chiude con il narratore che interrompe prima di poter vedere
    che cosa farà successivamente Orlando con la tecnica dell'entrelacement.
\end{snippet}

\end{document}