\documentclass[preview]{standalone}

\usepackage{amsmath}
\usepackage{amssymb}
\usepackage{stellar}
\usepackage{bettelini}

\hypersetup{
    colorlinks=true,
    linkcolor=black,
    urlcolor=blue,
    pdftitle={Stellar},
    pdfpagemode=FullScreen,
}

\begin{document}

\title{Stellar}
\id{orlando-furioso-xxiii-testo}
\genpage

\section{Testo}

\begin{snippet}{orlando-furioso-ottava-100-xxiii}
    \StellarPoetry{100}{
        Lo strano corso che tenne il cavallo\\
        del Saracin pel bosco senza via,\\
        fece ch'Orlando andò duo giorni in fallo,\\
        né lo trovò, né poté averne spia.\\
        Giunse ad un rivo che parea cristallo,\\
        ne le cui sponde un bel pratel fioria,\\
        di nativo color vago e dipinto,\\
        e di molti e belli arbori distinto.
    }{\parbox[t]{0.4\textwidth}{L'imprevedibile percorso che prese il cavallo
    di Mandricardo per il bosco privo di sentieri
    fece si che Orlando vagò per due giorni a vuoto,
    né lo trovò, né ne ebbe traccia.
    Arrivò a un ruscello che sembrava cristallo,
    sulle cui sponde fioriva un bel prato
    dei colori della natura dipinto,
    e variamente ornato da molti bei cespugli.}}
    \\\\
    La prima parte dell'ottava è narrativa, mentre la seconda descrittiva.
    Nella prima parte viene ripresa una vicenda di un tale Saracin.
    Ad ogni verso vi è unêspressione che indica il movimento labirintico.
    Il posto dove Orlando giunse è un locus amoenus.
\end{snippet}

\begin{snippet}{orlando-furioso-ottava-101-xxiii}
    \StellarPoetry{101}{
        Il merigge facea grato l'orezzo\\
        al duro armento ed al pastore ignudo;\\
        sì che né Orlando sentia alcun ribrezzo,\\
        che la corazza avea, l'elmo e lo scudo.\\
        Quivi egli entrò per riposarvi in mezzo;\\
        e v'ebbe \textbf{travaglioso} albergo e \textbf{crudo},\\
        e più che dir si possa \textbf{empio} soggiorno,\\
        quell'\textbf{infelice} e \textbf{sfortunato} giorno.
    }{\parbox[t]{0.4\textwidth}{La calda ora del mezzogiorno rendeva gradita l'ombra
    agli animali e al pastore nudo;
    così che neppure Orlando ebbe alcun esitazione,
    avendo la corazza, l'elmo e lo scudo.
    Qui Orlando entrò per riposare in mezzo ai cespugli
    e vi trovò una dimora angosciosa, funesta
    e più di quanto si possa dire,
    quell'infelice e sfortunato giorno.}}
    \\\\
    Il merigge sono le ore più calde della giornata.
    Vi sono quindi le condizioni perfette, l'aria fresca si bilancia perfettamente con il sole.
    Grazie a queste condizioni, Orlando non senti nessun brivido nonostante sia barbato.
    Da questo punto Ariosto comincia a infliggere la sventura a Orlando.
    La seconda parte dell'ottava rovescia tutti gli aggettivi positivi della prima parte.
    % travaglioso albero crudo, mette Agg. Nome. Agg al posto che NAA oppure AAN
\end{snippet}

\begin{snippet}{orlando-furioso-ottava-102-xxiii}
    \StellarPoetry{102}{
        Volgendosi ivi intorno, vide scritti\\
        molti arbuscelli in su l'ombrosa riva.\\
        Tosto che fermi v'ebbe gli occhi e fitti,\\
        fu certo esser di man de la sua \textbf{diva}.\\
        Questo era un di quei lochi già descritti,\\
        ove sovente con Medor veniva\\
        da casa del pastore indi vicina\\
        la bella donna del Catai regina.
    }{\parbox[t]{0.4\textwidth}{Girando intorno vide incisi con scritte
    molti arberelli sulla riva dell' ombroso fiume.
    Non appena ebbe gli occhi fermi e fissi con maggior attenzione
    fu sicuro che furono scritti dalla dea del suo cuore.
    Questo era uno di quei luoghi già descritti,
    dove spesso Medoro veniva
    dalla vicina casa del pastore
    con la bella Angelica.}}
    \\\\
    Fra tutte le strade che avrebbe potuto prendere, Orlando prende proprio
    quella dove Medoro e Angelica hanno inciso il loro nome, di cui si parla quattro canti prima.
    Dopo aver visto le scritte, ferna lo sguardo e le fissa.
    Orlando defiinsce Angelica come diva, a questo punto del testo la sta quindi idealizzando.
    È ancora convinta che Angelicia sia la donna raggiungibile che Orlando vuole.
\end{snippet}

\begin{snippet}{orlando-furioso-ottava-103-xxiii}
    \StellarPoetry{103}{
        Angelica e Medor con cento nodi\\
        legati insieme, e in cento lochi vede.\\
        Quante lettere son, tanti son chiodi\\
        coi quali Amore il cor gli punge e fiede.\\
        Va col pensier cercando in mille modi\\
        non \colorbox{orange}{creder} quel ch'al suo dispetto \colorbox{orange}{crede}:\\
        ch'altra Angelica sia, \colorbox{orange}{creder} si sforza,\\
        ch'abbia scritto il suo nome in quella scorza.
    }{\parbox[t]{0.4\textwidth}{Vede Angelica e Medoro in diversi modi,
    intrecciarti insieme ed in diversi luoghi.
    Tante sono le lettere, tanti sono i chiodi
    con i quali Cupido gli ferisce e punge il cuore.
    Va a cercare in mille modi con il pensiero
    di non credere quello a cui, suo malgrado, crede:
    si sforza di credere che sia un' altra Angelica
    ad aver scritto il suo nome sul quella corteccia.}}
    \\\\
    Orlando spera che questo sia un caso di omonimia, e quindi non la sua Angelica ad aver inciso il
    suo nome.
    In realtà, Orlando sa che si tratta della regina del Catai, ma si sforza di credere
    diversamente. La realtà è talmente dolorosa, per amore e gelosia, che non vuole accettarla.

    \fcolorbox{black}{orange}{\rule{0pt}{5pt}\rule{5pt}{0pt}}
    La ripetizione di questa parola indica il suo pensiero ossessivo.
\end{snippet}

\begin{snippet}{orlando-furioso-ottava-104-xxiii}
    \StellarPoetry{104}{
        Poi dice: - Conosco io pur queste note:\\
        di tal'io n'ho tante vedute e lette.\\
        Finger questo Medoro ella si puote:\\
        forse ch'a me questo cognome mette. -\\
        Con tali opinion dal ver remote\\
        usando fraude a sé medesmo, stette\\
        ne la speranza il malcontento Orlando,\\
        che si seppe a se stesso ir procacciando.
    }{\parbox[t]{0.4\textwidth}{Poi dice: “Io conosco la grafia di queste lettere:
    di queste (lettere) ne ho viste e ne ho lette tante.
    Potrebbe essersi inventata questo Medoro:
    forse mi ha dato questo soprannome”.
    Con tali opinioni remote,
    continuò ad assillare se stesso, ponendo
    il suo malcontento nella speranza
    che seppe procurare a se stesso.}}
    \\\\
    Orlando riconosce la scrittura di Angelica.
    Forse il nome Medoro è un riferimento ad Orlando.
    Questo è il secondo tentantivo della delusione e autoinganno Orlando.
\end{snippet}

\begin{snippet}{orlando-furioso-ottava-105-xxiii}
    \StellarPoetry{105}{
        Ma sempre più raccende e più rinuova,\\
        quanto spenger più cerca, il rio sospetto:\\
        come l'incauto augel che si ritrova\\
        in ragna o in visco aver dato di petto,\\
        quanto più batte l'ale e più si prova\\
        di disbrigar, più vi si lega stretto.\\
        Orlando viene ove s'incurva il monte\\
        a guisa d'arco in su la chiara fonte.
    }{\parbox[t]{0.4\textwidth}{Ma più si riaccende e si rinnova
    il crudele sospetto più cerca di dimenticarlo:
    come il disattento uccello che finisce
    in una ragnatela o sui rami invischiati,
    quanto più batte le ali e più prova
    a liberarsi, più si lega stretto.
    Orlando giunge dove si incurva la montagna
    come un arco (formando una grotta) sulla fonte cristallina}}
    \\\\
    Questa ottava è divisa in tre parti.
    Più Orlando cerca di smettere di pensarci, più il pensiero diventa forte.
\end{snippet}

\begin{snippet}{orlando-furioso-ottava-106-xxiii}
    \StellarPoetry{106}{
        Aveano in su l'entrata il luogo adorno\\
        coi piedi storti edere e viti erranti.\\
        Quivi soleano al più cocente giorno\\
        stare abbracciati i duo felici amanti.\\
        V'aveano i nomi lor dentro e d'intorno,\\
        più che in altro dei luoghi circostanti,\\
        scritti, qual con carbone e qual con gesso,\\
        e qual con punte di coltelli impresso
    }{\parbox[t]{0.4\textwidth}{Avevano ornato l'ingresso (di quella grotta)
    edere e viti rampicanti con i loro fusti contorti.
    Nei giorni più caldi, qui erano soliti
    stare abbracciati i due felici amanti.
    C'erano i loro nomi dentro ed intorno (alla grotta)
    più che nei luoghi circostanti,
    scritti alcuni con il carbone ed altri con gesso
    e altri erano impressi con punte di coltelli.}}
    \\\\
    Orlando trova la grotta dove i due si erano amati durante le ore più calde,
    dove vi sono molte incisioni.
\end{snippet}

\begin{snippet}{orlando-furioso-ottava-107-xxiii}
    \StellarPoetry{107}{
        Il mesto conte a piè quivi discese;\\
        e vide in su l'entrata de la grotta\\
        parole assai, che di sua man distese\\
        Medoro avea, che parean scritte allotta.\\
        Del gran piacer che ne la grotta prese,\\
        questa sentenza in versi avea ridotta.\\
        Che fosse culta in suo linguaggio io penso;\\
        ed era ne la nostra tale il senso:
    }{\parbox[t]{0.4\textwidth}{Qui scese il triste cavaliere;
    e vide sull' entrata della grotta
    tante parole, che erano state scritte dalla mano di
    Medoro, e sembravano esser state scritte proprio in quel momento.
    Per esprimere il grande piacere che provò (con Angelica) nella grotta,
    aveva composto questa iscrizione in versi.
    Io penso che fosse poeticamente elaborata in arabo (lingua di Medoro),
    ed era tale il senso nella nostra lingua:}}
    \\\\
    L'informazione nuova è che Medoro ha lasciato una scritta, quasi fresca,
    in cui descrive il piacere che ha avuto in quella grotta.
    Evidentmeente, essendo musulmano, Medoro ha scritto in arabo.
\end{snippet}

\begin{snippet}{orlando-furioso-ottava-108-xxiii}
    \StellarPoetry{108}{
        - Liete piante, verdi erbe, limpide acque,\\
        spelunca opaca e di fredde ombre grata,\\
        dove la bella Angelica che nacque\\
        di Galafron, da molti invano amata,\\
        spesso ne le mie braccia nuda giacque;\\
        de la commodità che qui m'è data,\\
        io povero Medor ricompensarvi\\
        d'altro non posso, che d'ognor lodarvi:
    }{\parbox[t]{0.4\textwidth}{“Liete piante, verdi erbe, limpide acque,
    grotta gradevole per la fresca ombra,
    dove la bella Angelica nacque
    di Galafron, è stata amata vanamente da molti,
    spesso nelle mie braccia giacque nuda;
    dei piaceri che qui mi sono stati dati,
    io povero Medoro non posso
    ricompensarvi in altro modo, se non lodandovi in ogni momento:}}
\end{snippet}

\begin{snippet}{orlando-furioso-ottava-109-xxiii}
    \StellarPoetry{109}{
        e di pregare ogni signore amante,\\
        e cavallieri e damigelle, e ognuna\\
        persona, o paesana o viandante,\\
        che qui sua volontà meni o Fortuna;\\
        ch'all'erbe, all'ombre, all'antro, al rio, alle piante\\
        dica: benigno abbiate e sole e luna,\\
        e de le ninfe il coro, che proveggia\\
        che non conduca a voi pastor mai greggia. -
    }{\parbox[t]{0.4\textwidth}{e di pregare ogni signore che vi ha amato,
    e cavalieri e damigelle ed ogni
    persona, del posto o forestiere,
    che capiti qui intenzionalmente o per caso;
    che all'erba, all'ombra, all'ingresso (delle grotta), al fiume e alle piante
    dica: che sole e luna vi siano favorevoli,
    e vi protegga il coro delle ninfe
    dai danni che potrebbero recare le greggi condotte lì da qualche pastore.”}}
    Il narratore dà la traduzione del testo scritto in arabo.
    Medoro spera di lasciare un messaggio a chi viene dopo, in maniera tale
    da non deturbare questi luoghi.
    Spera che questi luoghi rimangano immacolati così come quando lui è stato con Angelica
    quella volta.
    Medoro è straordinariamente lucido, consapevole della sua fortuna e
    di una grande altezza poetica.
\end{snippet}

\begin{snippet}{orlando-furioso-ottava-110-xxiii}
    \StellarPoetry{110}{
        Era scritto in arabico, che 'l conte\\
        intendea così ben come latino:\\
        fra molte lingue e molte ch'avea pronte,\\
        prontissima avea quella il paladino;\\
        e gli schivò più volte e danni ed onte,\\
        che si trovò tra il popul saracino:\\
        ma non si vanti, se già n'ebbe frutto;\\
        ch'un danno or n'ha, che può scontargli il tutto.
    }{\parbox[t]{0.4\textwidth}{Era scritto in arabo, che il cavaliere
    capiva bene come il latino:
    tra molte lingue che conosceva,
    il paladino sapeva benissimo quella;
    e gli fece evitare più volte danni e scontri,
    quando si trovò tra il popolo saraceno:
    ma non si rallegri, se altre volte (la conoscenza dell'arabo) gli fu propizia;
    perché ora gli arreca un danno tale da cancellare tutti i vantaggi ottenuti.}}
    \\\\
    Orlando conosce varie lingue, fra cui l'arabo molto bene.
    Questa conoscenza gli è servita varie volte in territorio nemico.
    Questa nuova consapevolezza contrasta il valore positivo di sapere la lingua, perché
    Orlando capisce esattamente il testo di Medoro.
    Meodo, in maniera completamente innocua, ha citato nel testo il fatto che
    Angelica sia amata in vano da molti. Questa espressione provoca molto dolore a
    Orlando.
\end{snippet}

\begin{snippet}{orlando-furioso-ottava-111-xxiii}
    \StellarPoetry{111}{
        Tre volte e quattro e sei lesse lo scritto\\
        quello infelice, e pur cercando invano\\
        che non vi fosse quel che v'era scritto;\\
        e sempre lo vedea più chiaro e piano:\\
        ed ogni volta in mezzo il petto afflitto\\
        stringersi il cor sentia con fredda mano.\\
        Rimase al fin con gli occhi e con la mente\\
        fissi nel sasso, al sasso indifferente.
    }{\parbox[t]{0.4\textwidth}{Lesse tre, quattro, sei volte la triste poesia
    l'infelice, ed anche cercando invano (di immaginare)
    che non ci fosse ciò che vi era scritta;
    ma gli risultava sempre più chiaro e facile da comprendere:
    ed ogni volta (che leggeva) si sentiva in mezzo al petto afflitto
    stringere il cuore con mano gelida.
    Rimase lì con gli occhi e con il pensiero
    rivolti al sasso, impietrito.}}
    \\\\
    Orlando legge il testo svariate volte nella speranza di trovare un dettaglio
    per alimentare la propria delusione.
\end{snippet}

\begin{snippet}{orlando-furioso-ottava-112-xxiii}
    \StellarPoetry{112}{
        Fu allora per uscir del sentimento\\
        sì tutto in preda del dolor si lassa.\\
        Credete a chi n'ha fatto esperimento,\\
        che questo è 'l duol che tutti gli altri passa.\\
        Caduto gli era sopra il petto il mento,\\
        la fronte priva di baldanza e bassa;\\
        né poté aver (che 'l duol l'occupò tanto)\\
        alle querele voce, o umore al pianto.
    }{\parbox[t]{0.4\textwidth}{Fu allora che inizio ad impazzire,
    così che in preda al dolore si abbandona completamente.
    Credete a chi lo ha provato su se stesso,
    che questa, d'amore, è la sofferenza che fa passare tutte le altre.
    Gli era caduto il mento sopra il petto (testa bassa),
    la fronte era priva di rughe ed era bassa;
    non poté aver e (che il dolore l'occupò tanto)
    voce per lamentarsi o lacrime per piangere.}}
    \\\\
    Il narratore indica di aver provato per esperienza questo dolore, il dolore
    più forte di tutti, da cui nessuno è al riparo.
    Lo prova il narratore e lo prova pure il più grande dei cavallieri.
\end{snippet}

\begin{snippet}{orlando-furioso-ottava-113-xxiii}
    \StellarPoetry{113}{
        L'impetuosa doglia entro rimase,\\
        che volea tutta uscir con troppa fretta.\\
        Così veggiàn restar l'acqua nel vase,\\
        che largo il ventre e la bocca abbia stretta;\\
        che nel voltar che si fa in su la base,\\
        l'umor che vorria uscir, tanto s'affretta,\\
        e ne l'angusta via tanto s'intrica,\\
        ch'a goccia a goccia fuore esce a fatica.
    }{\parbox[t]{0.4\textwidth}{Rimase dentro l'impetuoso dolore,
    che voleva uscire con troppa fretta.
    Così vediamo restare l'acqua nel vaso,
    che abbia largo il ventre e stretta la bocca;
    così ché, capovolgendo il vaso,
    l liquido che vorrebbe uscire, tanto velocemente si riversa,
    e si ingorga nella stretta apertura,
    uscendo così goccia a goccia, a fatica.}}
    \\\\
    Il dolore viene descritto con la similitudine di questa ottava.
\end{snippet}

\begin{snippet}{orlando-furioso-ottava-114-xxiii}
    \StellarPoetry{114}{
        Poi ritorna in sé alquanto, e pensa come\\
        possa esser che non sia la cosa vera:\\
        che voglia alcun così infamare il nome\\
        de la sua donna e \colorbox{orange}{crede} e \colorbox{orange}{brama} e \colorbox{orange}{spera},\\
        o gravar lui d'insopportabil some\\
        tanto di gelosia, che se ne pera;\\
        ed abbia quel, sia chi si voglia stato,\\
        molto la man di lei bene imitato.
    }{\parbox[t]{0.4\textwidth}{Poi ritorna abbastanza in sé, e pensa se
    la cosa potrebbe essere non vera:
    che qualcuno voglia così infamare il nome
    della sua donna, e crede e spera e brama,
    oppure (che qualcuno voglia) gravarlo di un così insopportabile peso
    di gelosia, da farlo morire;
    e abbia, chiunque sia stato,
    imitato molto bene la sua calligrafia (di Angelica).}}
    \\\\
    Come ultimo tentativo di delusione, Orlando crede che qualcosa possa avere imitato
    la grafia di Angelica, o per infangarle il nome, o per farlo impazzire.
\end{snippet}

\begin{snippet}{orlando-furioso-ottava-115-xxiii}
    \StellarPoetry{115}{
        In così poca, in così debol speme\\
        sveglia gli spiriti e gli rifranca un poco;\\
        indi al suo Brigliadoro il dosso preme,\\
        dando già il sole alla sorella loco.\\
        Non molto va, che da le vie supreme\\
        dei tetti uscir vede il vapor del fuoco,\\
        sente cani abbaiar, muggiare armento:\\
        viene alla villa, e piglia alloggiamento.
    }{\parbox[t]{0.4\textwidth}{Con una così debole speranza,
    gli si rianimarono gli spiriti vitali;
    quindi salì in groppa al suo Brigliadoro
    quando il sole stava già lasciando il posto a sua sorella luna (tramonto).
    Non va molto avanti, che dagli alti comignoli
    dei tetti vede uscire del fumo,
    sente cani abbaiare e una mandria muggire:
    va fino alla villa e prende posto.}}
\end{snippet}

\begin{snippet}{orlando-furioso-ottava-116-xxiii}
    \StellarPoetry{116}{
        Languido smonta, e lascia Brigliadoro\\
        a un discreto garzon che n'abbia cura;\\
        altri il disarma, altri gli sproni d'oro\\
        gli leva, altri a forbir va l'armatura.\\
        Era questa la casa ove Medoro\\
        giacque ferito, e v'ebbe alta avventura.\\
        Corcarsi Orlando e non cenar domanda,\\
        di dolor sazio e non d'altra vivanda.
    }{\parbox[t]{0.4\textwidth}{Languido smonta (da cavallo), e lascia Brigliadoro
    a un abile garzone perché ne abbia cura:
    si fa disarmare da uno, gli sperono d'oro un altro
    gli leva, e si fa lucidare l'armatura da un altro ancora.
    Era questa la casa dove Medoro
    visse quando fu ferito, e dove ebbe grande fortuna.
    Orlando chiede solo da dormire e niente per cena,
    è sazio di dolore e non di altro cibo.}}
    Orlando è talmente distrutto, senza il cavallo, armatura, speroni d'oro, e senza Angelica,
    che non è più un cavaliere (simbolicamente); torna un uomo, tanto che qualcuno deve aiutarlo.
    Il grande eroe, di fronte a questo dolore, regredisce. 
\end{snippet}

\begin{snippet}{orlando-furioso-ottava-117-xxiii}
    \StellarPoetry{117}{
        Quanto più cerca ritrovar quiete,\\
        tanto ritrova più travaglio e pena;\\
        che de l'odiato scritto ogni parete,\\
        ogni uscio, ogni finestra vede piena.\\
        Chieder ne vuol: poi tien le labra chete;\\
        che teme non si far troppo serena,\\
        troppo chiara la cosa che di nebbia\\
        cerca offuscar, perché men nuocer debbia.
    }{\parbox[t]{0.4\textwidth}{Quanto più cerca di trovare tranquillità,
    tanto più prova travaglio e dolore;
    vede piena della odiata poesia (quella scritta da Medoro) ogni parete
    ogni finestra, ogni porta.
    Vorrebbe chiedere a riguardo ma poi tiene le labbra ferme (sta zitto);
    perché teme di rendere (a se stesso) troppo evidente,
    troppo chiara la cosa che
    cerca di dimenticare (offuscare), per provare meno dolore.}}
    \\\\
    Orlando vorrebbe chiedere al pastore chi siano i due che hanno fatto le incisioni,
    ma si frena dal farlo in quanto teme che la nebbia che lo distacca dalla realtà svanisca.
\end{snippet}

\begin{snippet}{orlando-furioso-ottava-118-xxiii}
    \StellarPoetry{118}{
        Poco gli giova usar fraude a se stesso;\\
        che senza domandarne, è chi ne parla.\\
        Il pastor che lo vede così oppresso\\
        da sua tristizia, e che voria levarla,\\
        l'istoria nota a sé, che dicea spesso\\
        di quei duo amanti a chi volea ascoltarla,\\
        ch'a molti dilettevole fu a udire,\\
        gl'incominciò senza rispetto a dire:
    }{\parbox[t]{0.4\textwidth}{Ingannare se stesso non gli giova;
    perché senza domandare (dell'accaduto) c'è chi ne parla.
    Il pastore, che lo vede così oppresso
    dalla sua tristezza, e vorrebbe alleviarla,
    iniziò a raccontargli la storia che conosceva bene; raccontava spesso
    dei due amanti a chi voleva ascoltare
    una storia molto dilettevole,
    e così, senza rispetto, cominciò a raccontare}}
    \\\\
    Il pastore, vedendolo triste, prova a rincuorargli la bella storia di Angelica e Medoro.
\end{snippet}

\begin{snippet}{orlando-furioso-ottava-119-xxiii}
    \StellarPoetry{119}{
        come esso a prieghi d'Angelica bella\\
        portato avea Medoro alla sua villa,\\
        ch'era ferito gravemente; e ch'ella\\
        curò la piaga, e in pochi dì guarilla:\\
        ma che nel cor d'una maggior di quella\\
        lei ferì Amor; e di poca scintilla\\
        l'accese tanto e sì cocente fuoco,\\
        che n'ardea tutta, e non trovava loco:
    }{\parbox[t]{0.4\textwidth}{come egli, pregato dalla bella Angelica,
    aveva portato in casa sua Medoro,
    ferito gravemente; e che ella (Angelica)
    curò la ferita ed in pochi giorni la guarì:
    ma lei, con una piaga ancora maggiore di quella, nel cuore
    fu ferita da Amore (cupido); e da una piccola scintilla
    si accese tanto del così cocente fuoco,
    che la faceva ardere tutta, e non trovava pace:}}
\end{snippet}

\begin{snippet}{orlando-furioso-ottava-120-xxiii}
    \StellarPoetry{120}{
        e sanza aver rispetto ch'ella fusse\\
        figlia del maggior re ch'abbia il Levante,\\
        da troppo amor costretta si condusse\\
        a farsi moglie d'un povero fante.\\
        All'ultimo l'istoria si ridusse,\\
        che 'l pastor fe' portar la gemma inante,\\
        ch'alla sua dipartenza, per mercede\\
        del buono albergo, Angelica gli diede.
    }{\parbox[t]{0.4\textwidth}{e senza aver riguardo che ella (Angelica) fosse
    figlia del più grande re che abbia mai avuto l'oriente,
    sospinta da un grandissimo amore fu portata
    a sposare Medoro, umile soldato.
    La conclusione della storia fu
    che il pastore mostrò ad Orlando il gioiello,
    che al momento della partenza, come ricompensa
    della buona ospitalità, gli diede Angelica.}}
    \\\\
    Il narratore riferisce cosa dice il pastore a Orlando.
    % il braccialetto di pegno al suo amore. 

    % orlando muore per il dolore, etc.
\end{snippet}

\begin{snippet}{orlando-furioso-ottava-132-xxiii}
    \StellarPoetry{132}{
        Afflitto e stanco al fin cade ne l'erba,\\
        e ficca gli occhi al cielo, e non fa motto.\\
        Senza cibo e dormir così si serba,\\
        che 'l sole esce tre volte e torna sotto.\\
        Di crescer non cessò la pena acerba,\\
        che fuor del senno al fin l'ebbe condotto.\\
        Il quarto dì, da gran furor commosso,\\
        e maglie e piastre si stracciò di dosso.
    }{\parbox[t]{0.4\textwidth}{Afflitto e stanco cadde infine nell'erba
    e fissò gli occhi al cielo senza dire parola alcuna.
    Rimane così, senza mangiare e senza dormire
    per tre giorni.
    Il suo dolore non smise di crescere,
    finché non l'ebbe fatto impazzire.
    Il quarto giorno, sconvolto dalla pazzia violenta,
    si tolse di dosso tutta l'armatura.}}
    \\\\
    Fra 131-132 è come quasi se non si rispettassero più i confini dell'ottava.
    Abbiamo una pausa di tre giorni in cui Orlando resta sdraiato guardando il cielo.
    Non avendo nemmeno più la forza di parlare giace a terra.
    Nonostante ciò, la sofferenza non smette di crescere e i versi sono ricchi di espressioni
    che indicano la sua stanchezza.
    \\
    Il quarto giorno la furia irrompe nuovamente, ancora più forte di prima.
\end{snippet}

\begin{snippet}{orlando-furioso-ottava-133-xxiii}
    \StellarPoetry{133}{
        Qui riman l'elmo, e là riman lo scudo,\\
        lontan gli arnesi, e più lontan l'usbergo:\\
        l'arme sue tutte, in somma vi concludo,\\
        avean pel bosco differente albergo.\\
        E poi si squarciò i panni, e mostrò ignudo\\
        l'ispido ventre e tutto 'l petto e 'l tergo;\\
        e cominciò la gran follia, sì orrenda,\\
        che de la più non sarà mai ch'intenda.
    }{\parbox[t]{0.4\textwidth}{Qui resta l'elmo e là resta lo scudo,
    lontano gli arnesi (corredo dell'armatura), e più lontano ancora la corazza: tutte le sue armi, concludendo,
    avevano ognuna diversa collocazione per il bosco.
    E poi si squarciò i vestiti, e rimasero nudi
    il peloso addome e la schiena;
    e iniziò la grande pazzia, così orrenda,
    che nessuno sentirà mai parlare di una (follia) più orrenda di questa.}}
    \\\\
    Questa ottava può essere divisa in tre parti:
    Orlando butta tutte le sue armi in luoghi diversi, si strappa i vestiti
    e comincia la follia più grande mai vista.
    % ispido = peloso
    \\\\
    Come quando Orlando si era fatto aiutare a togliersi l'armatura, smette di essere un cavaliere,
    ma adesso è lui stesso che se la strappa di dosso.
    Il suo ventro e schiena sono pelosi come quella di una bestia,
    smettendo quasi di essere un uomo, puramente istintivo e senza capacità di ragionamento complesso.
\end{snippet}

\begin{snippet}{orlando-furioso-ottava-134-xxiii}
    \StellarPoetry{134}{
        In tanta rabbia, in tanto furor venne,\\
        che rimase offuscato in ogni senso.\\
        Di tor la spada in man non gli sovenne;\\
        che fatte avria mirabil cose, penso.\\
        Ma né quella, né scure, né bipenne\\
        era bisogno al suo vigore immenso.\\
        Quivi fe' ben de le sue prove eccelse,\\
        ch'un alto pino al primo crollo svelse:
    }{\parbox[t]{0.4\textwidth}{Gli scaturì così tanta rabbia e così tanto furore
    che tutte le sue facoltà sensitive furono alterate.
    Non gli passo per la testa di prendere la spada,
    che tante incredibili avventure aveva passato, credo.
    Ma tanto né quella, né una scure, né una bipenne (scure a due lame)
    erano necessarie alla sua immensa forza.
    Qui fece davvero alcune tra le sue imprese più straordinarie,
    sradicò un grande pino con un solo scrollone:}}
    \\\\
    La follia di Orlando irrompe e lui ne è completamente offuscato.
    Le sue armi non gli sarebbero serviti, solo la sua forza bruta era necessaria a sfogare la sua ira.
    Vengono presentate molte iperboli come come quella di sdradicare un alto pino con le mani.
\end{snippet}

\begin{snippet}{orlando-furioso-ottava-135-xxiii}
    \StellarPoetry{135}{
        e svelse dopo il primo altri parecchi,\\
        come fosser finocchi, ebuli o aneti;\\
        e fe' il simil di querce e d'olmi vecchi,\\
        di faggi e d'orni e d'illici e d'abeti.\\
        Quel ch'un ucellator che s'apparecchi\\
        il campo mondo, fa, per por le reti,\\
        dei giunchi e de le stoppie e de l'urtiche,\\
        facea de cerri e d'altre piante antiche.
    }{\parbox[t]{0.4\textwidth}{e ne abbatté, dopo il primo, molti altri ancora
    come se fossero state piante dal fusto tenero;
    e fece la stessa cosa con querce, vecchi olmi,
    faggi e abeti.
    Come un uccellatore che per ripulire
    il campo, dove mettere le reti,
    estirpa le erbaccie, i ramoscelli e le ortiche,
    Orlando faceva con le querce e con le altre piante secolari del bosco.}}
    \\\\
    Vi è una ripresa per anadiplosi.
    Orlando deforesta come se fossero ortaggi.
    L'ottava si chiude con una similitudine: come il cacciatore di uccelli
    ripulisce il terreno per posare le reti, con la medesima facilità Orlando deforesta tutto
    attorno a lui.
\end{snippet}

\begin{snippet}{orlando-furioso-ottava-136-xxiii}
    \StellarPoetry{136}{
        I pastor che sentito hanno il fracasso,\\
        lasciando il gregge sparso alla foresta,\\
        chi di qua, chi di là, tutti a gran passo\\
        vi vengono a veder che cosa è questa.\\
        Ma son giunto a quel segno il qual s'io passo\\
        vi potria la mia istoria esser molesta;\\
        ed io la vo' più tosto diferire,\\
        che v'abbia per lunghezza a fastidire.
    }{\parbox[t]{0.4\textwidth}{I pastori che avevano sentito il gran chiasso,
    lasciando il gregge sparso per la foresta,
    da ogni luogo, di corsa
    vanno a vedere che cosa fosse quel rumore.
    Ma sono giunto a quel punto che se lo oltrepasso,
    la mia storia vi potrebbe essere dannosa;
    e io la voglio rinviare ad un altro canto
    prima che vi possa infastidire per la sua lunghezza.}}
    \\\\
    Il narratore interrompe prima di poter vedere che cosa farà Orlando a delle persone
    (tecnica dell'entrelacement).
\end{snippet}

\end{document}