\documentclass[preview]{standalone}

\usepackage{amsmath}
\usepackage{amssymb}
\usepackage{stellar}
\usepackage{bettelini}

\hypersetup{
    colorlinks=true,
    linkcolor=black,
    urlcolor=blue,
    pdftitle={Biologia},
    pdfpagemode=FullScreen,
}

\begin{document}

\title{Biologia}
\id{biologia-definizioni}
\genpage

\begin{snippetdefinition}{cellula-definizione}{Cellula}
    La \textit{cellula} è la più piccola unità funzionale degli esseri
    viventi. È delimitata da una membrana plasmatica
    che racchiude varie strutture, dette organuli cellulari.
\end{snippetdefinition}

\begin{snippetdefinition}{organulo-definizione}{Organulo}
    Un \textit{organulo} è una struttura racchiusa da una membrana che svolge
    una funzione specifica all'interno della cellula.
\end{snippetdefinition}

\begin{snippetdefinition}{tessuto-definizione}{Tessuto}
    Un \textit{tessuto} è costituito da gruppi
    di cellule simili che svolgono
    una specifica funzione.
\end{snippetdefinition}

\begin{snippet}{biologia-definizioni-expl1}
    % pag 14
Tutte le specie viventi sono suddivise in tre grandi gruppi:
\textbf{eubatteri} (Bacteria), \textbf{archebatteri} (Archaea)
ed \textbf{eucarioti} (Eukarya).
I primi due domini, eubatteri e archebatteri, identificano due gruppi di organismi
unicellulari molto diversi formati da cellule
procariote.
Il dominio degli eucarioti, invece, comprende or ganismi sia unicellulari
sia pluricellulari costituiti da cellule eucariote ed è,
a sua volta, suddiviso in più regni: \textbf{protisti},
\textbf{piante}, \textbf{funghi} e \textbf{animali}.
\end{snippet}

\end{document}