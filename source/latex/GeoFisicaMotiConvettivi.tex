\documentclass[preview]{standalone}

\usepackage{amsmath}
\usepackage{amssymb}
\usepackage{tikz}
\usepackage{stellar}
\usepackage{bettelini}

\hypersetup{
    colorlinks=true,
    linkcolor=black,
    urlcolor=blue,
    pdftitle={Assets},
    pdfpagemode=FullScreen,
}

\begin{document}

\title{Moti convettivi}
\id{geofisica-moticonvettivi}
\genpage

\section{Moti convettivi}

\begin{snippetdefinition}{moto-convettivo}{Moto convettivo}
    I \textit{moti convettivi} sono dei lenti movimenti di materiali solidi.
\end{snippetdefinition}

\plain{I moti convettivi possono scendere o salire.
Quando i moti salgono e colpiscono la litosfera intaccano le placche tettoniche.}

\plain{Le zolle litosferico sono trutture rigide che galleggiano sulla sottostante astenosfera,
la quale funge da strato plastico entro il quale si realizzano moti convettivi.}

\plain{Le zolle litosferiche possono essere composte da litosfera continentale o litosfera oceanica
(con crosta continentale e crosta oceanica) oppure entrambe.}

\end{document}