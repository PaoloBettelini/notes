\documentclass[preview]{standalone}

\usepackage{amsmath}
\usepackage{amssymb}
\usepackage{stellar}

\hypersetup{
    colorlinks=true,
    linkcolor=black,
    urlcolor=blue,
    pdftitle={Stellar},
    pdfpagemode=FullScreen,
}

\begin{document}

\id{storia-definizioni}
\genpage

\section{Definizioni}

\begin{snippetdefinition}{storiografia-definition}{Storiografia}
    La \textit{storiografia} è la disciplina scientifica che si occupa di studiare la storia.
\end{snippetdefinition}

\begin{snippetdefinition}{periodizzazione-definition}{Periodizzazione}
    La \textit{periodizzazione} è l'operazione culturale volta a suddividere la linea temporale in vari intervalli,
    ciascuno con caratteristiche comuni.
\end{snippetdefinition}

\plain{Le prime periodizzazioni derivano dalle prime religioni monoteiste (Es. nascità di Gesù, calendario islamico).
Le periodizzazioni sono delle convenzioni.}

\end{document}