\documentclass[preview]{standalone}

\usepackage{amsmath}
\usepackage{amssymb}
\usepackage{parskip}
\usepackage{fullpage}
\usepackage{hyperref}
\usepackage{stellar}

\hypersetup{
    colorlinks=true,
    linkcolor=black,
    urlcolor=blue,
    pdftitle={Stellar},
    pdfpagemode=FullScreen,
}

\begin{document}

\id{logical-connectives}
\genpage

\section{Logical connectives}

\begin{snippet}{logic-connectives-explanation}
    In propositional logic, any two propositional variables can be connected by some logical relation.
    Here is a list of the most commonly used logical connectives.
\end{snippet}

\begin{snippetnote}{logic-connectives-expl1}{}
    Some of the following operators are just abbrevietations.
    The composition of the operators \(\lnot\) and \(\implies\)
    would suffice to define every other operator.
\end{snippetnote}

\subsection{Negation}

\includesnpt{logic-logical-negation}

\begin{snippet}{logic-connectives-expl2}
This is equivalent to the linguistic word ``not''.
\end{snippet}

\subsection{Logical conjuction}

\includesnpt{logic-logical-conjunction}

\begin{snippet}{logic-connectives-expl2}
In order for the resulting proposition to be true, both of the propositions
must be true.
\end{snippet}

\subsection{Logical disjuction}

\includesnpt{logic-logical-disjunction}

\begin{snippet}{logic-connectives-expl3}
In order for the resulting proposition to be true, one proposition
or both must be true.
\end{snippet}

\subsection{Sufficiency}

\includesnpt{logic-sufficiency}

\begin{snippet}{logic-connectives-expl4}
In other words, if \(P\) is true, then \(Q\) must also be true.
This does \underline{not} necessarily mean that if \(Q\) is true, then \(P\) is also true.
\end{snippet}

\includesnpt{logic-sufficiency-example}

\subsection{Necessity}

\includesnpt{logic-necessity}

\begin{snippet}{logic-connectives-expl5}
In other words, if \(Q\) is true, then \(P\) must also be true.
This does \underline{not} necessarily mean that if \(P\) is true, then \(Q\) is also true.
\end{snippet}

\includesnpt{logic-necessity-example}

\subsection{Biconditional logical connective}

\begin{snippet}{logic-connectives-expl6}
Necessity and sufficiency can coincide. If they do coincide,
then the two propositional variables are always equivalent. \\
If one is true, then the other is also true. If one is false, then the other is also false.
\end{snippet}

\includesnpt{logic-biconditional-logical-connective}

\begin{snippet}{logic-connectives-expl7}
This is equivalent to saying \(P \implies Q \land Q \implies P\).
\end{snippet}

% principle of explosion

% first order logig
% higher order logic

\end{document}
