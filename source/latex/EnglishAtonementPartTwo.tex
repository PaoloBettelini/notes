\documentclass[preview]{standalone}

\usepackage{amsmath}
\usepackage{amssymb}
\usepackage{bettelini}
\usepackage{stellar}
\usepackage{definitions}

\begin{document}

\id{english-atonement-part-two}
\genpage

\section{Part Two}

\begin{snippetdefinition}{dunkirk-evacuation-definition}{Dunkirk evacuation}
    The \textit{Dunkirk evacuation} was the evacuation of more than 338,000
    Allied soldiers during the Second World War from the beaches
    and harbour of Dunkirk, in the north of France, between 26 May and 4 June 1940. 
\end{snippetdefinition}

\begin{snippetexercise}{atonement-ex-15}
    {Who is the narrator of Part Two? What is the setting (place and time)?}
    We can see Robbie being a soldier during WWII in France, 1940.
    He is evacuating during the Dunkirk evacuation.
    Robbie becomes the new narrator and point of view of the story.
\end{snippetexercise}

\begin{snippetexercise}{atonement-ex-16}
    {What images does the author use to make the reader feel close to the horrors of war witnessed
    by the soldiers?}
    Among various depictions of destruction, the author uses the image of a perfectly
    intact leg, smooth like the skin of a child.
    These vivid images depict a boundary between the first and second
    part, showing a contract between the slow and normal pace of life
    and the horrors of war.
    Robbie has mutated from the first section. Before, he was more naive,
    and now during the war his persona has substantially changed.
\end{snippetexercise}

\begin{snippetexercise}{atonement-ex-17}
    {What has happened to Robbie since the night of the dinner at the Tallis' house?}
    Robbie has been accused o rape and went to prison for a few years.
    The only person who would visit him was his mother.
    After this time, he becomes a soldier and stops being a prisoner.
\end{snippetexercise}

\begin{snippetexercise}{atonement-ex-19}
    {To what extent are Cecilia's letters important for Robbie as he makes his way towards Dunkirk?}
    Cecilia's letters are always important for Robbie as they are
    his only hope and reason to survive.
\end{snippetexercise}

\begin{snippetexercise}{atonement-ex-20}
    {What does the reader learn from Cecilia's last letter to Robbie?}
    Briony was to go to college but she went to nursing school instead
    and started working in an hospital which cures wounded soldiers.
    She would like to meet him and possibly reunite,
    but he isn't quite ready to do so.
    At this point, it's too late to save Robbie, but Briony could at least
    apologize rather than fully confessing the lie.
\end{snippetexercise}

\end{document}
