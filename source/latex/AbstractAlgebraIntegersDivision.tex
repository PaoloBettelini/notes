\documentclass[preview]{standalone}

\usepackage{amsmath}
\usepackage{amssymb}
\usepackage{parskip}
\usepackage{fullpage}
\usepackage{hyperref}
\usepackage{bettelini}
\usepackage{stellar}
\usepackage{definitions}

\begin{document}

\id{integers-division}
\genpage

\section{Divides operator}

\begin{snippetdefinition}{divide-operator-definition}{Divide Operator}
    Given \(a,b \in \integers\),
    we say that \(a \,|\, b\) \ifandonlyif \(a\) \textit{divides} \(b\),
    meaning that
    \[
        \exists x \in \integers \suchthat ax = b
    \]
\end{snippetdefinition}

\begin{snippetproposition}{divide-operator-properties}{Properties of the divide operator}
    Given the integers \(a\), \(b\) and \(c\)
    \begin{enumerate}
        \item \(a \divides b \land a \divides c \implies a \divides b+c\);
        \item \(a \divides b \implies a \divides bc\);
        % basta notare che se b=ac, allora -b=a(-c) e viceversa (per la prima)
        \item \(a \divides b \iff -a \divides b \iff a \divides -b\);
        \item \(|a| \leq |b|, \quad b \neq 0\);
        \item \(a \divides b \land b \divides c \implies a \divides c\);
    \end{enumerate}
\end{snippetproposition}

\begin{snippetlemma}{divide-operator-abs-property}{}
    If \(a \divides b \land b\neq 0\), then \(|a|\leq|b|\).
\end{snippetlemma}

\begin{snippetproof}{divide-operator-abs-property-proof}{divide-operator-abs-property}{}
    We are going to assume that \(b>0\) and \(a\leq0\) as the case where they are negative is analogous.
    There exists \(c \in \mathbb{Z} \suchthat b = ac\). Since \(b>0\) and \(a\geq 0\)
    we have \(a>0\) (otherwise \(b=ac=0\)) and \(c>0\).
    Thus, \(b= ac \geq a = a\), like we wanted.
\end{snippetproof}

\begin{snippet}{divide-operator-abs-property-consequence}
    The dividers of a non-zero numbers are of a finite number because their module
    is less than \(|b|\).
\end{snippet}

\section{Division with remainder}

\begin{snippetproposition}{division-with-remainder}{Division with remainder}
    Given two integers \(a\) and \(b\) with \(b > 0\),
    \[
        \exists_{=1} q,r \divides a=bq+r, \quad 0 \leq r < b
    \]
\end{snippetproposition}

\begin{snippetproof}{division-with-remainder-proof}{division-with-remainder}{Division with remainder}
    Consider \(a=bq+r\) with \(0 \leq r < b\).
    \begin{enumerate}
        \item \textbf{Existence:} consider \(R=\{a-bx \suchthat x\in\integers\} \intersection \naturalnumbers\)
        as the \set of potential remainders. We now show that \(R\neq\emptyset\) by
        letting \(x=-|a|\) and considering the number \(r'=a-b(-|a|) = a + b|a| \geq a+|a| \geq 0\)
        since \(b > 0\) and
        \[
            \begin{cases}
                a+|a| = 2a \geq 0 & a \geq 0 \\
                a+|a| = 0 \geq 0 & a < 0
            \end{cases}
        \]
        Thus, \(r'\in R \land R \neq\emptyset\),
        meaning that \(R\) is a non-empty subsets of \(\naturalnumbers\), and thus has a \leastelement,
        which is our remainder \(r\). There exists \(q\in\integers \suchthat r=a-bq\), or \(a=bq+r\).
        This values satisfies the necessary conditions as \(r\geq 0\). However,
        we still need to show that \(r < b\).
        If we had \(r \geq b\), then \(r-b > 0\), but we would arrive at
        \[
            r-b = a-bq - b = a - b(q+1) \in R
        \]
        contrary to the hypothesis that \(r\) is the \leastelement of \(R\) \lightning.
        Thus, \(r < b\).
        \item \textbf{Uniqueness:} Let \(a=bq' + r'\) with \(0\leq r' < b\).
        Let, for instance, \(q \geq q'\). We thus have \(b(q-q') = r' - r\).
        In the case where \(q-q' > 0\), the first term would be greater or equal than \(b\),
        However, \(r-r' \leq r' < b\) \lightning. 
        Thus, \(q-q' = 0\), which means that \(q=q'\) and \(r=r'\).
    \end{enumerate}
\end{snippetproof}

% Lemma 5.3.13
\begin{snippetlemma}{common-dividors-of-quotient-and-remainder}{Quotient and remainder common divisors}
    Let \(a \in \naturalnumbers\) and \(b \in {\naturalnumbers}^+\).
    Let \(q\) and \(r\) be the quotient and remainder of the division of \(b\)
    by \(a\).
    The common divisors of \(a\) and \(b\) are equivalent to the common divisors of \(r\) and \(q\).
\end{snippetlemma}

% TODO proof

\end{document}
