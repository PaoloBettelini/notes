\documentclass[preview]{standalone}

\usepackage{amsmath}
\usepackage{amssymb}
\usepackage{stellar}
\usepackage{definitions}

\begin{document}

\id{settheory-orders}
\genpage


\section{Types of orders}

\begin{snippetdefinition}{preorder-order-definition}{Preorder order}
    A \textit{preorder} is a \homrelation \(\leq\) on a set \(A\)
    with the following properties:
    \begin{enumerate}
        \item \textit{Reflexive}: \(\forall a \in A, a \leq a\)
        \item \textit{Transitive}: \(\forall a,b,c \in A, a \leq b \land b \leq c \implies a \leq c\)
    \end{enumerate}
\end{snippetdefinition}

\begin{snippetdefinition}{partial-order-definition}{Partial order}
    A \textit{partial order} is a \homrelation \(\leq\) on a set \(A\)
    with the following properties:
    \begin{enumerate}
        \item \textit{Reflexive}: \(\forall a \in A, a \leq a\)
        \item \textit{Transitive}: \(\forall a,b,c \in A, a \leq b \land b \leq c \implies a \leq c\)
        \item \textit{Antisymmetric}: \(\forall a,b \in A, a \leq b \land b \leq a \implies a=b\)
    \end{enumerate}
\end{snippetdefinition}

\begin{snippetdefinition}{total-orderv}{Total order}
    A \textit{total order} is a \homrelation \(\leq\) on a set \(A\)
    with the following properties:
    
    \begin{enumerate}
        \item \textit{Reflexive}: \(\forall a \in A, a \leq a\)
        \item \textit{Transitive}: \(\forall a,b,c \in A, a \leq b \land b \leq c \implies a \leq c\)
        \item \textit{Antisymmetric}: \(\forall a,b \in A, a \leq b \land b \leq a \implies a=b\)
        \item \textit{Strongly connected} (or \textit{total}): \(\forall a,b\in A, a \leq b \lor b\leq a\)
    \end{enumerate}
\end{snippetdefinition}

\begin{snippet}{settheory-3}
    A total order is a partial order where any two elements are comparable.
\end{snippet}

\subsection{Elements characterization}

\begin{snippetdefinition}{greatest-element-definition}{Greatest element}
    Given a partial order on a set \(A\), an element \(g\) is a \textit{greatest element}
    if \(\forall a\in A, a \leq g\).
\end{snippetdefinition}

\begin{snippetdefinition}{least-element-definition}{Least element}
    Given a partial order on a set \(A\), an element \(g\) is a \textit{least element}
    if \(\forall a\in A, g \leq a\).
\end{snippetdefinition}

\begin{snippetdefinition}{maximal-element-definition}{Maximal element}
    Given a partial order on a set \(A\), an element \(g\in A\) that is
    a greatest element is a \textit{maximal element}.
\end{snippetdefinition}

\begin{snippetdefinition}{minimal-element-definition}{Minimal element}
    Given a partial order on a set \(A\), an element \(g\in A\) that is
    a least element is a \textit{minimal element}.
\end{snippetdefinition}

\end{document}