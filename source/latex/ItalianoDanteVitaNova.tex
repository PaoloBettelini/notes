\documentclass[preview]{standalone}

\usepackage{amsmath}
\usepackage{amssymb}
\usepackage{stellar}

\hypersetup{
    colorlinks=true,
    linkcolor=black,
    urlcolor=blue,
    pdftitle={Stellar},
    pdfpagemode=FullScreen,
}

\begin{document}

\id{italiano-vita-nova}
\genpage

\section{Introduzione}

\begin{snippetdefinition}{sonetto-definition}{Sonetto}
    Il \textit{sonetto} è una poesia composto da 2 quartine e 2 terzine dove tutti i versi sono degli endecasillabi.
\end{snippetdefinition}

\begin{snippetdefinition}{prosimetro-definition}{Prosimetro}
    Il \textit{prosimetro} è un testo ibrido, composto da un
    racconto (prosa) intervallato e poesie (versi).
\end{snippetdefinition}

\begin{snippet}{vita-nova-expl1}
    La \textit{Vita nova} è il primo prosimetro di Dante. Esso racconta la storia d'amore da parte di Dante
    nei confronti di Beatrice.
    Questa vicenda diventa un modello per questa tipologia di narrativa.
    
    Il significato del titolo indica come Dante consideri l'inizio della sua vita (nuova vita, rinnovata)
    quando vide Beatrice per la prima volta.
    Il primo contatto amoroso nella poesie è spesso caratterizzato da un innamoramento a prima vista.
    
    % Il saluto nel medioevo da parte delle donne
    Quando la voce dell'interesse di Dante nei confronti di Beatrice le giunge, lei gli nega il saluto.
\end{snippet}

\begin{snippetnote}{saluto-medievale}{Il saluto nel medioevo}
    Il saluto nel medioevo ha un significato molto più profondo di quello odierno.
\end{snippetnote}

\begin{snippet}{vita-nova-expl2}
    Nonostante il rifiuto, Dante continua ad esprimere il suo amore verso Beatrice semplicemente
    lodandola (scrivendo di lei), completamente senza ricambio di interesse.
    Questa loda rappresenta la forma più pura di amore.
    
    Questo libro introduce la simbologia del numero 9 associato a Beatrice.
    Ciò è dato dal fatto che Dante l'abbia vista per la prima volta a 9 anni, rivista 9 anni dopo,
    e altri motivi che vengono descritti. Il numero 9 è anche un simbolo biblico (3 volte la trinità).
\end{snippet}

\end{document}