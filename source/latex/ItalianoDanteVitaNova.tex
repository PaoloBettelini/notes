\documentclass[preview]{standalone}

\usepackage{amsmath}
\usepackage{amssymb}
\usepackage{stellar}
\usepackage{tikz}
\usepackage{definitions}

\usetikzlibrary{cd}

\begin{document}

\id{italiano-vita-nova}
\genpage

\section{Introduzione}

\begin{snippetdefinition}{sonetto-definition}{Sonetto}
    Il \textit{sonetto} è una poesia composto da 2 quartine e 2 terzine dove tutti i versi sono degli endecasillabi.
\end{snippetdefinition}

\begin{snippetdefinition}{prosimetro-definition}{Prosimetro}
    Il \textit{prosimetro} è un testo ibrido, composto da un
    racconto (prosa) intervallato e poesie (versi).
\end{snippetdefinition}

\begin{snippet}{vita-nova-expl1}
    La \textit{Vita nova} è il primo prosimetro di Dante. Esso racconta la storia d'amore da parte di Dante
    nei confronti di Beatrice.
    Questa vicenda diventa un modello per questa tipologia di narrativa.
    
    Il significato del titolo indica come Dante consideri l'inizio della sua vita (nuova vita, rinnovata)
    quando vide Beatrice per la prima volta.
    Il primo contatto amoroso nella poesie è spesso caratterizzato da un innamoramento a prima vista.
    Infatti, nello stil novo il primo contatto parte sempre dagli occhi per poi scendere al cuore.
    
    % Il saluto nel medioevo da parte delle donne
    Quando la voce dell'interesse di Dante nei confronti di Beatrice le giunge, lei gli nega il saluto.
\end{snippet}

\begin{snippetnote}{saluto-medievale}{Il saluto nel medioevo}
    Il saluto nel medioevo ha un significato molto più profondo di quello odierno.
    Dal latino "salutem" deriva saluto ma anche salvezza.
\end{snippetnote}

\begin{snippet}{vita-nova-expl2}
    Nonostante il rifiuto, Dante continua ad esprimere il suo amore verso Beatrice semplicemente
    lodandola (scrivendo di lei), completamente senza ricambio di interesse.
    Questa lode rappresenta la forma più pura di amore, l'amore disinteressato, che si appaga di sé stesso.
    
    Questo libro introduce la simbologia del numero 9 associato a Beatrice.
    Ciò è dato dal fatto che Dante l'abbia vista per la prima volta a 9 anni, rivista 9 anni dopo,
    e altri motivi che vengono descritti. Il numero 9 è anche un simbolo biblico (3 volte la trinità).
\end{snippet}

\begin{snippet}{morte-beatrice-e-numero-nove}
    Quando Beatrice muore, Dante non ne parla per tre motivi piuttosto oscuri. 
    
    Dante afferma anche che non scriverà di lei finché lui non possa parlarne più degnamente.

    \vspace{0.25cm}

    Il numero nove viene invece menzionato durante la morte di Beatrice: viene infatti detto che
    perde la vita il giorno 08.06.1290, che rappresenta il nono giorno del calendario siriano, mentre
    in un altro calendario che comincia ad ottobre luglio è il nono mese.

    \vspace{0.25cm}

    Un'altra apparizione del numero nove viene detta alla nascita di Beatrice, quando i nove cieli si allineano.

    \vspace{0.25cm}

    Si può quindi fare uno schema di questo tipo: se il numero tre è legato alla trinità, a Dio,
    e il tre basta per fare sé stesso (\(3 \times 3\)), allora il numero 9 è il miracolo

    \vspace{0.25cm}

    % https://tikzcd.yichuanshen.de/#N4Igdg9gJgpgziAXAbVABwnAlgFyxMJZABgBpiBdUkANwEMAbAVxiRAGYQBfU9TXfIRQAmclVqMWbAJzdeIDNjwEiZAIzj6zVohAARfHL5LBRURupapugLJYATnQDGEBhG7iYUAObwioADN7CABbJDIQHHdENR5A4LDEUUjo9jiQINDw6iikYXTMxLUc1K4KLiA
    \begin{center}
        \begin{tikzcd}
            3 \arrow[rr] \arrow[d] &  & 9 \arrow[d] \\
            \text{Dio} \arrow[rr]         &  & \text{Miracolo}   
        \end{tikzcd}
    \end{center}

    \vspace{0.25cm}

    Siccome il numero nove è legato a Beatrice, allora Beatrice è un miracolo.

\end{snippet}

\end{document}