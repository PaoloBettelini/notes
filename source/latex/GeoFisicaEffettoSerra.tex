\documentclass[preview]{standalone}

\usepackage{amsmath}
\usepackage{amssymb}
\usepackage{tikz}
\usepackage{stellar}
\usepackage{bettelini}

\hypersetup{
    colorlinks=true,
    linkcolor=black,
    urlcolor=blue,
    pdftitle={Assets},
    pdfpagemode=FullScreen,
}

\begin{document}

\title{Effetto serra}
\id{geofisica-effetto-serra}
\genpage

\begin{snippetdefinition}{effetto-serra}{Effetto serra}
    L'\textit{effetto serra} è un fenomeno per il quale alcuni gas
    atmosferici, detti appunti \textit{gas serra},
    permettono l'ingresso della radiazione solare proveniente dalla stella, 
    mentre ostacolano l'uscita della radiazione infrarossa riemessa dalla
    superficie del corpo celeste.
\end{snippetdefinition}

%\plain{L'effetto serra è un fenomeno naturale, ma viene accentuato dall'essere umano.
%Nessun modello riescie a rappresentare la situazione attuale senza fenomeno artificiali.}

\includesnpt{nasa-temperature}

% https://it.wikipedia.org/wiki/Moti_millenari

\end{document}