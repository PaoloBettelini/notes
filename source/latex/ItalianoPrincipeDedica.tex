\documentclass[preview]{standalone}

\usepackage{amsmath}
\usepackage{amssymb}
\usepackage{stellar}
\usepackage{makecell}

\hypersetup{
    colorlinks=true,
    linkcolor=black,
    urlcolor=blue,
    pdftitle={Stellar},
    pdfpagemode=FullScreen,
}

\begin{document}

\title{Stellar}
\id{italiano-principe-dedica}
\genpage

\section{La dedica}

\begin{snippet}{machiavelli-dedica-analisi-1}
    Per entrare nelle grazie di un sovrano è necessario offrirgli qualcosa di prezioso o gradito.
    Ciò che Machiavelli può offrire è
    \begin{center}
        \textit{la cognizione delle azioni degli uomini grandi}
    \end{center}
    La parola grande non ha nessuna connotazione morale, bensì indica solamente
    persone di potere.
    Questa conoscenza è stata acquisita dalla sua lunga esperienza delle cose moderne
    e dalla lettura dei testi antichi.
    Il suo modo per offrire questa conoscenza è quello di includerle in un piccolo volume (\textit{Il Principe}).
    Il vantaggio di un dono simile è che un libro è una sintesi di migliaia di pagine
    lette e quindici anni di esperienza nella repubblica.
\end{snippet}

\begin{snippetnote}{topos-della-modestia-machiavelli}{Topos della modestia}
    Fra i testi di Machiavelli vi è sempre una opposizione fra modestia e orgoglio.
    L'autore dona una accezione di poco conto alle sue opere, mentre
    offre anche una connotazione della loro grandezza e importanza (topos della modestia).
\end{snippetnote}

\begin{snippet}{machiavelli-dedica-analisi-2}
    \begin{center}
        \begin{minipage}{0.75\textwidth}
            \itshape
            la quale opera io non ho ornata nè ripiena di clausule ampie, o di parole ampollose o magnifiche, o di qualunque altro lenocinio o ornamento estrinseco, con li quali molti sogliono le lor cose discrivere ed ornare; perchè io ho voluto o che veruna cosa la onori, o che solamente la verità della materia, e la gravità del soggetto la faccia grata.
        \end{minipage}
    \end{center}
    \vspace{0.25cm}

    La forma del libro viene descritta come esteticamente poco attraente,
    il motivo è quello che l'autore vuole che il suo testo venga onorato
    solo per il suo contenuto effettivo, oppure per null'altro (piuttosto prefesce che non venga apprezza).
    L'assenza di ornamenti è quindi quella di non far apprezzare il libro per qualcosa che non sia lo stretto contenuto.
    \vspace{0.25cm}
    \begin{center}
        \begin{minipage}{0.75\textwidth}
            \itshape
            Nè voglio sia riputata presunzione, se uno uomo di basso ed infimo stato ardisce discorrere e regolare i governi de' Principi; perchè così come coloro che disegnano i paesi, si pongono bassi nel piano a considerare la natura de' monti e de' luoghi alti, e per considerare quella de' bassi si pongono alti sopra i monti; similmente, a cognoscer bene la natura de' popoli bisogna esser Principe, ed a cognoscer bene quella de' Principi conviene essere popolare.
        \end{minipage}
    \end{center}
    \vspace{0.25cm}
    Come i pittori/cartografi devono essere in basso per disegnare le altitudini, e viceversa, per parlare
    del popolo bisogna essere principi, e per parlare di principi bisogna essere facenti parte del popolo.
    Questa è la motivazione di Machiavelli per poter scrivere il principato.
    \\\\
    Infine, l'autore si rivolge a Lorenzo de' Medici, augurandogli il meglio nel suo potere, e facendogli
    notare la sua posizione sfortunata.
\end{snippet}

\end{document}