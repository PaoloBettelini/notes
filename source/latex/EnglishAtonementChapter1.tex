\documentclass[preview]{standalone}

\usepackage{amsmath}
\usepackage{amssymb}
\usepackage{bettelini}
\usepackage{stellar}
\usepackage{makecell}

\hypersetup{
    colorlinks=true,
    linkcolor=black,
    urlcolor=blue,
    pdftitle={English},
    pdfpagemode=FullScreen,
}

\begin{document}

\title{English}
\id{english-atonement-chapter1}
\genpage

\section{Summary}

\begin{snippetsummary}{atonement-summary-chapter1}{Atonement, Ian McEwan - Chapter 1}
    In the upper-class Tallis family house in England in 1935,
    Briony Tallis writes a play to perform with her visiting
    cousins in honor of her adored brother Leon's return home.
    Her mother, Emily Tallis, indulges Briony by complimenting her work,
    but her cousins Lola and twins Jackson and Pierrot Quincey, are really
    not interested in it. The cousing are staying with the Tallis family
    because their parents are divorcing, and Lola ruins
    Briony's plan to take the role of the play's main character, Arabella, by
    claiming the part for herself.
\end{snippetsummary}

\section{Quotes}

\begin{snippet}{atonement-chapter1-quotes}
\begin{minipage}[l]{0.05\textwidth}
    \circled{1}
\end{minipage}
\begin{minipage}[r]{0.95\textwidth}
    \textbf{Quote:} \textit{
        THE PLAY […] was written by her in a two-day tempest of
        composition, causing her to miss a breakfast and a lunch.
        (p. 3)
    }
    \\
    \textbf{Meaning:}
    Briony is very creative and likes to write stories.
    At times she is so focused on writing that she
    forgets about everything else, including her
    basic needs, such as eating.
\end{minipage}
\hr
\begin{minipage}[l]{0.05\textwidth}
    \circled{2}
\end{minipage}
\begin{minipage}[r]{0.95\textwidth}
    \textbf{Quote:} \textit{
        Mrs. Tallis read the seven pages of The Trials of Arabella
        in her bedroom, at her dressing table, with the author's
        arm around her shoulder the whole while. Briony studied
        her mother's face for every trace of shifting emotion, and
        Emily Tallis obliged […]. (p. 4)
    }
    \\
    \textbf{Meaning:} Briony is seeking her mother's approval.
\end{minipage}
\hr
\begin{minipage}[l]{0.05\textwidth}
    \circled{3}
\end{minipage}
\begin{minipage}[r]{0.95\textwidth}
    \textbf{Quote:} \textit{
        Briony was hardly to know it then, but this was the
        project's highest point of fulfillment. (p. 4)
    }
    \\
    \textbf{Meaning:} She will not be able to achieve her goal because this is the highest point,
    and everything can only go down from here. She puts much emphasis on what she wants,
    needing lots of attention and being very self-centered.
\end{minipage}
\hr
\begin{minipage}[l]{0.05\textwidth}
    \circled{4}
\end{minipage}
\begin{minipage}[r]{0.95\textwidth}
    \textbf{Quote:} \textit{
        Her play was […] for her brother, to celebrate his return,
        provoke his admiration and guide him away from his
        careless succession of girlfriends, toward the right form
        of wife, the one who would persuade him to return to the
        countryside, the one who would sweetly request Briony's
        services as a bridesmaid. (p. 4)
    }
    \\
    \textbf{Meaning:} She is selfish and a little bit self-centered but at the same time naïve.
    She thinks she knows better than her brother, even is she is only thirteen years old.
\end{minipage}
\hr
\begin{minipage}[l]{0.05\textwidth}
    \circled{5}
\end{minipage}
\begin{minipage}[r]{0.95\textwidth}
    \textbf{Quote:} \textit{
        Nothing in her life was sufficiently interesting or
        shameful to merit hiding […]. (p. 5)
    }
    \\
    \textbf{Meaning:} We can see the difference between what she writes in her stories and her life.
    By writing, she's projecting the fact that she would like have a more interesting life,
    and have something to hide, which she currently hasn't.
\end{minipage}
\hr
\begin{minipage}[l]{0.05\textwidth}
    \circled{6}
\end{minipage}
\begin{minipage}[r]{0.95\textwidth}
    \textbf{Quote:} \textit{
        […] she was discovering, as had many writers before her,
        that not all recognition is helpful. (p. 7)
    }
    \\
    \textbf{Meaning:} She compares herself to other accomplishes (real) writers, even thought she is
    young and does it as a hobby. She would like to be older and an actual writer.
    On the contrary of her mother, which gives her compliments, her sister is too extreme
    in this behavior, making it blatantly fake.
    This shows that the family tends to overprotect her.
\end{minipage}
\hr
\begin{minipage}[l]{0.05\textwidth}
    \circled{7}
\end{minipage}
\begin{minipage}[r]{0.95\textwidth}
    \textbf{Quote:} \textit{
        Her sandals revealed an ankle bracelet and toenails
        painted vermilion. The sight of these nails gave Briony a
        constricting sensation around her sternum, and she knew
        at once that she could not ask Lola to play the prince. (p.
        11)
    }
    \\
    \textbf{Meaning:} She has shown a sign of being a mature and caring about being aesthetically pleasing.
\end{minipage}
\hr
\begin{minipage}[l]{0.05\textwidth}
    \circled{8}
\end{minipage}
\begin{minipage}[r]{0.95\textwidth}
    \textbf{Quote:} \textit{
        In a generally pleasant and well-protected life, she had
        never really confronted anyone before. (p. 15)
    }
    \\
    \textbf{Meaning:}
    Now, for the first time, she has a confrontation with somebody who is a little bit outside of her
    circle. He understands that this is outside of her comfort zone given that she has never had
    to be confronted with anyone and had a really easy life.
    This is basically her first time noticing what real life is: not everybody gives you what you want.
\end{minipage}
\end{snippet}

\end{document}
