\documentclass[preview]{standalone}

\usepackage{amsmath}
\usepackage{amssymb}
\usepackage{stellar}
\usepackage{definitions}
\usepackage{bettelini}

\begin{document}

\id{homomorphisms-from-cyclic-groups}
\genpage

\section{Homomorphism and generators}

\begin{snippetproposition}{equal-maps-subgroup}{}
    Let \(\varphi\) and \(\phi\) be \grouphomomorphism
    from \(G\) to \(H\). The subset \(S\) of \(G\)
    in which \(\varphi\) and \(\phi\) coincide is a \subgroup of \(G\)
\end{snippetproposition}

\begin{snippetproof}{equal-maps-subgroup-proof}{equal-maps-subgroup}{}
    \begin{enumerate}
        \item \(1_G \in S\): Indeed, \(\varphi(1_G) = 1_H = \psi(1_G) = 1_H\);
        \item if \(x\) and \(y\) are in \(S\), meaning \(\varphi(x) = \psi(x)\)
        and \(\varphi(y) = \psi(y)\), then
        \[
            \varphi(xy) = \varphi(x) \psi(y) = \psi(xy)    
        \]
        meaning \(xy \in S\);
        \item if \(x\in S\), meaning \(\varphi(x) = \psi(x)\), then
        \[
            \varphi(x^{-1}) = {(\varphi(x))}^{-1} = \psi(x^{-1})    
        \]
        meaning \(x^{-1} \in S\).
    \end{enumerate}
\end{snippetproof}

\begin{snippetcorollary}{equal-maps-on-generated-subgroup}{}
    Let \(\varphi\) and \(\phi\) be \grouphomomorphism
    from \(G\) to \(H\) and let \(X\) be a subset
    of \(G\) in which  \(\varphi\) and \(\phi\) coincide.
    Then, \(\varphi\) and \(\psi\) coincide in \(\gengrp{X}\).
    In particular, if \(G = \gengrp{X}\), then \(\varphi = \psi\).
\end{snippetcorollary}

\plain{If we know how the homomorphisms behave on the generators, we can reconstruct the entire homomorphism.
This does not mean that we can randomly associate images to the generators, but sometimes we can.}

\section{Homomorphism from cyclic groups}

\begin{snippet}{homomorphisms-from-cyclic-group-expl}
    Let \(G = \gengrp{g}\) a \cyclicgroup. We want to define a \grouphomomorphism
    from \(G\) to \(H\).
    We know that \(\varphi \colon G \to H\)
    is defined by the information of \(\varphi(g)\).
    Since the elements of \(G\) are the powers of \(g\),
    we can say that \(\varphi(g^n) = {(\varphi(g))}^n\).
    If \(h\in H\) and we define \(\varphi(g) = h\),
    we cannot just say that \(\varphi(g^n) = h^n\). We need to make sure that if \(|G| =d\),
    the other period \(|H| = f\) divides \(d\).\\
    If \(d\) is infinite, this \function is well-defined (\(g^n = g^m \iff m=n\)). \\
    If \(d\) is finite, this \function might not be (\(g^n = g^m \iff m \equiv n \pmod{d}\))
    and thus \(m-n\) must be a multiple of \(f\) (the period of \(H\) must diivde the period of \(G\)).
    If the \function is well-defined, it is also a \grouphomomorphism:
    two elements of \(G\) have form \(g^n\) and \(g^m\) for some \(n,m\in\integers\)
    and we have \(h^n = \varphi(g^n)\), \(h^m = \varphi(g^m)\)
    and \(\varphi(g^n g^m) = \varphi(g^{m+n})\).
\end{snippet}

\begin{snippettheorem}{infinite-ciclicgroup-homomorphism-theorem}{}
    Let \(G = \gengrp{g}\) an infinite \cyclicgroup
    and \(H\) be a \group.
    Given \(h\in H\), there exist a unique
    \grouphomomorphism \(\varphi \colon G \to H\)
    such that \(\varphi(g) = h\).
\end{snippettheorem}

\begin{snippettheorem}{finite-ciclicgroup-homomorphism-theorem}{}
    Let \(G = \gengrp{g}\) a finite \cyclicgroup
    and \(H\) be a \group.
    Given \(h\in H\), there exist a \grouphomomorphism
    \(\varphi\colon G \to H\) such that \(\varphi(g) = h\)
    \ifandonlyif \(|H|\) is finite and it divides \(|G|\).
    In such case, \(\varphi\) is unique.
\end{snippettheorem}

\begin{snippetcorollary}{cyclic-group-isomorphism}{Isomorphism between cyclic groups}
    Two \cyclicgroup[cyclic groups] are isomorphic
    \ifandonlyif \(|G| = |H|\).
\end{snippetcorollary}

\begin{snippetproof}{cyclic-group-isomorphism-proof}{cyclic-group-isomorphism}{isomorphism between cyclic groups}
    \iffproof{
        If \(G\) and \(H\) are isomorphic, then \(|G| = |H|\).
    }{
        If \(|G| = |H|\), let \(G = \gengrp{g}\)
        and \(H = \gengrp{h}\).
        If \(G\) and \(H\) are infinite we have a \grouphomomorphism \(\varphi \colon G \to H\)
        which sends \(g\) to \(h\) and a \grouphomomorphism \(\psi\colon H \to G\)
        that sends \(h\) to \(g\). Clearly, \(\varphi\) and \(\psi\)
        are inverses of eachtoher.
        If \(G\) and \(H\) are finite, we can do the same:
        there is a \group \(\varphi \colon G \to H\) which sends \(g\) in \(h\)
        (\(|h|\) divides \(|g|\)) and there exist a \grouphomomorphism
        \(\psi \colon H \to G\) which sends \(h\) in \(g\). The same goes for this case.
    }
\end{snippetproof}

\subsection{Cyclic endomorphisms and automorphisms}

\plain{For a finite cyclic group, we have as many endomorphisms as there are elements of the group.}

% C_n = gruppo ciclico arbitrario di ordine n
% e \varphi_n come endomorphismo

% La funzione da Z in End(G) come \varphi_n(g) = g^n
% è un omomorfismo dal monoide moltiplicativo degli interi, nel monoide degli endomorfismi di G.
% poiché ogni endomorfismo di G è un elevamento a una potenza fissata questo omomorfismo è suriettivo.
% Se il gruppo dell'ordine è finito, non è un isomorfismo in quanto non è iniettivo
% Se invece il gruppo è infinito, la funzione è anche iniettiva e quindi automorphism
% Possiamo notare ciò dicendo che se m \neq n, \varphi_n manda g in g^n e \varphi_m
% manda g in g_m. Tuttavia, g ha periodo infinito, quindi g^n \neq g^m, cioè \varphi_n \neq \varphi_m.

% Conslusione: Se G è ciclico infinito, End(G) è isomorfo come monoide a
% (Z, +). E Aut(G) sarà isomorfo al gruppo Inv(Z, *) = {1,-1}, cioè
% Aut(G) è formato dai due automorfismi: identità e inverso.

% Per studiare il caso infinito End(G) serve un po' di lavoro.
% G = \gengrp{g} con |g| = d, occorre stabilire quando \varphi_n = \varphi_m
% ciò avviene se e solo se g^n = g^m, quindi m \equiv n \pmod{d}
% Abbiamo (verifiche facili) un isomorfismo fra (\integers / d, *) e End(G).
% quindi {[m]}_d \to \varphi_m.
% Inoltre, Aut(G) è isomorfo al gruppo degli invertibili Inv(Z/d. *),
% che contiene \eulertotient(d) = |Aut(G)|.

\end{document}