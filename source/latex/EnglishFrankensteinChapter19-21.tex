\documentclass[preview]{standalone}

\usepackage{amsmath}
\usepackage{amssymb}
\usepackage{bettelini}
\usepackage{stellar}

\hypersetup{
    colorlinks=true,
    linkcolor=black,
    urlcolor=blue,
    pdftitle={English},
    pdfpagemode=FullScreen,
}

\begin{document}

\title{English}
\id{english-frankenstein-chapter-19-21}
\genpage

\subsection{Exercises}

\begin{snippetexercise}{frankenstein-ex-37}
    {Where does Victor go to complete the creation of the female monster? In what way(s) does the
    landscape reflect his thoughts as he carries out the task?}
    TODO 
\end{snippetexercise}

\begin{snippetexercise}{frankenstein-ex-38}
    {Why does Victor not go through with the experiment? In other words, what makes him change
    his mind?}
    TODO 
\end{snippetexercise}

\begin{snippetexercise}{frankenstein-ex-39}
    {Is Victor's destruction of the second creature justified? Why/why not?}
    Partially, yes because if we think of the consequences, it wouldn't be the best
    course of action to have two monsters. However, even destroying the new creature
    wasn't the best action to take because now the first monster is furious.
    It would definitely have been better to talk about these doubts with the monster
    at first. 
\end{snippetexercise}

\begin{snippetexercise}{frankenstein-ex-40}
    {Read the quote. What does the creature mean with the final sentence? What might this passage
    illustrate in the broader context of the novel?}
    \begin{center}
        \quotes{\textit{Slave, I before reasoned with you, but you have proved yourself}\\
        \textit{unworthy of my condescension. Remember that I have power; you believe}\\
        \textit{yourself miserable, but I can make you so wretched that the light of day}\\
        \textit{will be hateful to you. You are my creator, but I am your master; obey!}} (p. 202)
    \end{center}
    This passage shows us that now the monster won't be up to any compromises
    anymore and that he dedicates himself completely to destruction.
\end{snippetexercise}

\begin{snippetexercise}{frankenstein-ex-41}
    {Read the quote. What does the creature threaten to do? How does Victor interpret his words?}
    
    \begin{center}
        \quotes{\textit{It is well. I go; but remember, I shall be with you on your wedding-night.} I started forward and}\\
        \textit{exclaimed, \quotes{Villain! Before you sign my death-warrant, be sure that you are yourself safe. […]}}\\
        \textit{I shuddered to think who might be the next victim sacrificed to his insatiate revenge. And then I}\\
        \textit{thought again of his words—\quotes{I WILL BE WITH YOU ON YOUR WEDDING-NIGHT.} That, then,}\\
        \textit{was the period fixed for the fulfilment of my destiny. In that hour I should die and at once satisfy}\\
        \textit{and extinguish his malice. The prospect did not move me to fear; yet when I thought of my beloved}\\
        \textit{Elizabeth, of her tears and endless sorrow, when she should find her lover so barbarously snatched}\\
        \textit{from her, tears, the first I had shed for many months, streamed from my eyes, and I resolved not to}\\
        \textit{fall before my enemy without a bitter struggle.} (p. 203)
    \end{center}
    
    The monster threatens Victor and his family, making him fear for their safety.
    The monster makes clear that he will follow and stalk Victor whenever he goes,
    whatever he does. This sounds as the deadline for the final fight between the two,
    where one must die. 
\end{snippetexercise}

\begin{snippetexercise}{frankenstein-ex-42}
    {When Victor finds out who was murdered, what causes him to become a suspect? What happens
    at his trial?}
    
    He becomes a suspect because he is found on the shore a short while after the dead body
    was found. However, he's saved because there was proof that he was still in Scotland during the murder. 
\end{snippetexercise}

\end{document}
