\documentclass[preview]{standalone}

\usepackage{amsmath}
\usepackage{amssymb}
\usepackage{tikz}
\usepackage{stellar}
\usepackage{bettelini}

\hypersetup{
    colorlinks=true,
    linkcolor=black,
    urlcolor=blue,
    pdftitle={Assets},
    pdfpagemode=FullScreen,
}

\begin{document}

\title{Isostasia}
\id{geofisica-circolazione-planetaria}
\genpage

\section{Circolazione planetaria}

% % https://it.wikipedia.org/wiki/Modello_generale_della_circolazione

\begin{snippet}{circolazione-planetaria-expl1}
    Il modello generale della circolazione divide le corrento planetarie in tre diversi fattori:
    \begin{enumerate}
        \item cella polare;
        \item cella di Ferrel;
        \item cella di Hadley.
    \end{enumerate}
\end{snippet}

\plain{Il fronte polare è la fascia in cui si scontrano i venti dai poli e i venti dalle medie latitudini.
Fra i due tropici è presente una zona di convergenza tropicale data dalla bassa pressione.
Questa linea non è lineare e si sposta con la pioggia.}

\plain{All'equatore piove di più perché l'aria è più calda e quindi può contenere più acqua (punto di saturazione maggiore). }

\plain{
    In India vi è la stagione delle pioggie. Nell'india settentrionale,
la stagione monsonica si svolge tra giugno e
ottobre, mentre nell'India meridionale la maggior parte
delle precipitazioni cade a giugno, ottobre e novembre.
}

\begin{snippetdefinition}{zona-convergenza-intertropicale}{Zona di convergenza intertropicale}
    La \textit{zona di convergenza intertropicale} è un'area del pianeta Terra, mediamente situata in prossimità dell'equatore
    dove si ha la convergenza degli alisei, e la risalita di masse d'aria calda che determinano l'area di instabilità equatoriale, con piogge e temporali.
\end{snippetdefinition}

\begin{snippet}{d9bc89bb-32ae-4742-9a2f-82048d4199d7}
    La zona di convergenza intertropicale è molto calda in quanto è molto esposta ai raggi solari.
    Infatti, vi è poca differenza fra il giorno e la notte.
    \\
    In contrasto, vi sono le zone temperate, dove vi sono giornate corte in inverno e lunghe
    in estate, e le calotte polari (artica e antartica), dove si alterna fra un gran dì e una grande notte.
    \\
    La delineazione di questa zona non è lineare ed è dinamica, a dipendenza dall'inclinazione della terra
    e correnti oceaniche.
\end{snippet}

\begin{snippetdefinition}{tropico-del-capricorno}{Tropico del Capricorno}
    Il \textit{tropico del Capricorno} è il tropico terrestre
    situato nell'emisfero australe in cui il Sole
    culmina allo zenit un giorno all'anno (nel solstizio di dicembre).
\end{snippetdefinition}

\begin{snippetdefinition}{tropico-del-cancro}{Tropico del Cancro}
    Il \textit{tropico del Cancro} è il tropico terrestre
    situato nell'emisfero boreale in cui il Sole culmina
    allo zenit un giorno all'anno (nel solstizio di giugno).
\end{snippetdefinition}

\includesnpt[src=https://www.ventusky.com/|width=1280px|height=700px]{iframe}

\includesnpt[width=75\%|src=/snippet/static/emisferi.png]{centered-img}

\end{document}