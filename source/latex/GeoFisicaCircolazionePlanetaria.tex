\documentclass[preview]{standalone}

\usepackage{amsmath}
\usepackage{amssymb}
\usepackage{tikz}
\usepackage{stellar}
\usepackage{bettelini}

\hypersetup{
    colorlinks=true,
    linkcolor=black,
    urlcolor=blue,
    pdftitle={Assets},
    pdfpagemode=FullScreen,
}

\begin{document}

\title{Isostasia}
\id{geofisica-circolazione-planetaria}
\genpage

% % https://it.wikipedia.org/wiki/Modello_generale_della_circolazione

\begin{snippet}{circolazione-planetaria-expl1}
    Il modello generale della circolazione divide le corrento planetarie in tre diversi fattori:
    \begin{enumerate}
        \item cella polare;
        \item cella di Ferrel;
        \item cella di Hadley.
    \end{enumerate}
\end{snippet}

\plain{Il fronte polare è la fascia in cui si scontrano i venti dai poli e i venti dalle medie latitudini.
Fra i due tropici è presente una zona di convergenza tropicale data dalla bassa pressione.
Questa linea non è lineare e si sposta con la pioggia.}

\plain{All'equatore piove di più perché l'aria è più calda e quindi può contenere più acqua (punto di saturazione maggiore). }

\plain{
    In India vi è la stagione delle pioggie. Nell'india settentrionale,
la stagione monsonica si svolge tra giugno e
ottobre, mentre nell'India meridionale la maggior parte
delle precipitazioni cade a giugno, ottobre e novembre.
}

\end{document}