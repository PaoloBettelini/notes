\documentclass[preview]{standalone}

\usepackage{amsmath}
\usepackage{amssymb}
\usepackage{stellar}

\hypersetup{
    colorlinks=true,
    linkcolor=black,
    urlcolor=blue,
    pdftitle={Stellar},
    pdfpagemode=FullScreen,
}

\begin{document}

\title{Stellar}
\id{fisica-onde}
\genpage

\begin{snippetdefinition}{wave-definizione}{Wave}
    A wave is a propagation of a disturbance (energy) which oscillates
    repeatedly.
    \begin{itemize}
        \item the \textit{wavelength} \(\lambda\) of a wave describes how long the wave is;
        \item the \textit{period} \(T\) of a wave is the time it takes to complete
            a full oscillation;
        \item the \textit{frequency} \(f\) of a wave represents how many oscillation
        completed in one unit of time (seconds)
        \[
            f = \frac{1}{T}
        \]
        \item the \textit{phase velocity} \(v\) is the rate at which
        the wave propagates
        \[
            v = \frac{\lambda}{T} = f\lambda
        \]
        \item The \textit{amplitude} \(A\) of a mechanical wave is the
        measure of the maximum distance a point can reach
        from its equilibrium position.
    \end{itemize} 
\end{snippetdefinition}

\begin{snippet}{onde-expl}
    \vspace{-0.6cm}
    \paragraph{Waves in different dimensions:}
    Here are examples of each dimension
    \begin{itemize}
        \item 1 dimension: an oscillating rope
        \item 2 dimensions: surface of water oscillating
        \item 3 dimensions: sound propagating through the air
    \end{itemize}

    \paragraph{Direction of the wave:}
    A wave is \textit{transverse} when its oscillations are perpendicular
    to the direction of the wave propagation (e.g. slinky up and down).

    A wave is \textit{longitudinal} when its oscillations are parallel
    to the direction of the wave propagation (e.g. slinky left and right).

    \paragraph{Types of waves:}
    There are different types of waves, namely,
    \textit{mechanical} waves, \textit{electromagnetic} wave
    and \textit{gravitational} waves.

    Electromagnetic and gravitational waves are always longitudinal.
\end{snippet}

\section{Harmonic waves}

\begin{snippetdefinition}{harmonic-wave-definizione}{Harmonic wave}
    An \textit{harmonic wave} is a periodic wave where
    the points of the medium where it moves oscillate.

    \[
        s(t;x) = A \sin
        \left(
            \omega t - \frac{2\pi}{\lambda}x
        \right)
        \text{ where } \omega = \frac{2\pi}{T}
    \]
\end{snippetdefinition}

\end{document}