\documentclass[preview]{standalone}

\usepackage{amsmath}
\usepackage{amssymb}
\usepackage{stellar}
\usepackage{bettelini}

\hypersetup{
    colorlinks=true,
    linkcolor=black,
    urlcolor=blue,
    pdftitle={Biologia},
    pdfpagemode=FullScreen,
}

\begin{document}

\title{Biologia}
\id{biologia-genetica-classica}
\genpage

\section{Genetica classica}

\begin{snippetdefinition}{genetica-classica-definition}{Genetica classica}
    Ramo della genetica che studia le modalità di trasmissione
    dei geni basandosi unicamente sui risultati visibili di atti riproduttivi.
\end{snippetdefinition}

\begin{snippetdefinition}{genetica-molecolare-definition}{Genetica molecolare}
    Studia dei meccanismi molecolari dell'ereditarietà (struttura e funzione dei geni).
\end{snippetdefinition}

%\begin{snippetdefinition}{genotipo-definition}{Genotipo}
%    Il corredo genetico di un individuo (l'insieme dei suoi geni).
%\end{snippetdefinition}
%
%\begin{snippetdefinition}{fenotipo-definition}{Fenotipo}
%    La manifestazione fisica dei geni ereditati da un individuo
%    (l'aspetto esteriore).
%\end{snippetdefinition}

%% https://moodle.edu.ti.ch/libe/pluginfile.php/119651/mod_resource/content/1/2.2%20Genetica%20classica.pdf

\begin{snippet}{mendel-expl}
    Mendel cerca di capire se i caratteri dell'ereditarietà
    seguono delle regole studiando l'ereditarietà nella pianta di pisello.
    Dopo aver creare una linea sufficiente pura di piselli per ciò che concerno un gene,
    le incrocia impollinandole manualmente.
    I fiori sono ermafroditi e Mendel si accorge che, facendo fare il maschio ad uno, la femmina all'altro
    e viceversa, nulla cambia.
    Mendel analizza caratteri che sono dicotomi (ovviamente molti gene hanno più varianti).
\end{snippet}

\begin{snippetdefinition}{allele-definition}{Allele}
    Le forme divderse di uno stesso gene sono dette \textit{alleli}.
\end{snippetdefinition}

\begin{snippetdefinition}{omozigote-definition}{Omozigote}
    Organismo che ha due alleli uguali per un dato carattere.
\end{snippetdefinition}

\begin{snippetdefinition}{eterozigate-definition}{Eterozigate}
    Un organismo che ha due alleli diversi per un dato carattere.
\end{snippetdefinition}

\begin{snippetdefinition}{genotipo-definition}{Genotipo}
    Caratteristica genetica data dall'insieme degli alleli di un individuo.
\end{snippetdefinition}

\begin{snippetdefinition}{fenotipo-definition}{Fenotipo}
    Caratteristiche che viene espressa e quindi si manifesta in un individuo.
\end{snippetdefinition}

\begin{snippetdefinition}{allele-dominante-definition}{Allele dominante}
    Allele che viene espresso, sia quando è da solo (omozigote dominante) che
    quando è con l'allele recesivo (eterozigote). Si indica con una lettera maiuscola.
\end{snippetdefinition}

\begin{snippetdefinition}{alelle-recessivo-definition}{Allele recessivo}
    Allele che si manifesta unicamente quando è d solo (omozigote recessivo).
    Se è con un allele dominante (eterozigote), rimane nascosto. Si indica con una lettera minuscola.
\end{snippetdefinition}

% incrocio diibrido

\begin{snippetexercise}{mendel-ex1}{}
    Abbiamo i geni (U, u) e (L, l).
    Per quanto riguarda i lobi, la donna può essere o omozigote dominante o eterozigote
    (LL e Ll). Questa donna è in grado di arrotolare la lingua, per cui è
    omozigote dominante o eterozigote (UU o Uu).
    Sposa un uomo con i lobi omozigoti recessivi (ll) e che non può arrotolare la lingua
    (uu). Il genotipo del papà è quindi lluu. Il genitipo del suo primogenito 
    è anch'esso lluu.
    Il genotipo della madre è a questo punto per froza LlUu, perché
    per far si che il figlio ce li abbia entrambi minuscoli, entrambi
    i genitori devono averne almeno uno minuscolo.
    Il secondo figlio deve avere LL o Ll e uu. Chiaramente la possibilità
    LL non è possibile perché il padre ha solo ll.
    Il genotipo del secondo figlio è quindi Lluu.
    I gameti del figlio sono quindi l'incroncio di quelli del padre (l)
    e quelli della madre possibili (Lu, Lu, ll, l).
    La probabilità di ottenere questo genotipo dall'incrocio è 25\%.
\end{snippetexercise}

\end{document}