\documentclass[preview]{standalone}

\usepackage{amsmath}
\usepackage{amssymb}
\usepackage{stellar}
\usepackage{bettelini}

\hypersetup{
    colorlinks=true,
    linkcolor=black,
    urlcolor=blue,
    pdftitle={Biologia},
    pdfpagemode=FullScreen,
}

\begin{document}

\title{Biologia}
\id{biologia-biomolecole}
\genpage

\begin{snippetdefinition}{biomolecola-definition}{Biomolecola}
    Le \textit{biomolecole} sono le molecole dei processi biologici degli essere viventi.
\end{snippetdefinition}

\plain{Tutte le biomolecole contengono C, O e H.
Ci sono delle eccezioni, per esempio, gli idrocarburi contengono solamente C e H.}

\begin{snippet}{biomolecole-tipi}
    Le biomolecole sono di 4 tipi:
    \begin{itemize}
        \item Lipidi (grasso)
        \item Acidi nucleici (DNA e RNA)
        \item Carboidrati
        \item Proteine
    \end{itemize}
\end{snippet}

\begin{snippet}{macromolecole-expl1}
    Le macromolecole sono composte da \textit{monomeri} e \textit{polimeri}.
Nel corpo umano i polimeri sono creati dalle cellule mediante alle istruzioni nel DNA.
Le biomolecole fanno dei polimeri.
\end{snippet}

\begin{snippetdefinition}{isomero-definition}{Isomero}
    Gli \textit{isomeri} sono delle molecole distinte con il medesimo numero di atomi,
    ma con una struttura diversa. Diversi isomeri potrebbero avere proprietà diverse.
\end{snippetdefinition}

\section{Costruzione di polimeri}

\begin{snippet}{costruzione-polimeri}
    Tutti i monomeri posseggono, da una parte un gruppo di idrogeno \(H\),
e dall'altra un gruppo \(OH\).
Due monomeri si uniscono mediante una reazione chimica chiamata \textit{condensazioe} o \textit{disidratazione}, la quale consiste
nell'unire un'estremità \(H\) con una \(OH\) mediante un legame.
La condensazione libera una molecola d'acqua come scarto.
\end{snippet}

\section{Disintegrazione di polimeri}

\begin{snippet}{disintegrazione-polimeri}
    Per separare un legame fra due monomeri, viene utilizzata la reazione chimica di \textit{idrolisi} o \textit{idratazione}.
    Questa reazione necessita di una molecola di \(H_2O\).
\end{snippet}

\end{document}