\documentclass[preview]{standalone}

\usepackage{amsmath}
\usepackage{amssymb}
\usepackage{stellar}
\usepackage{definitions}

\begin{document}

\id{sylow-theorems}
\genpage

\section{Sylow theorems}

\begin{snippetdefinition}{sylow-subgroup-definition}{Sylow subgroup}
    Let \(G\) be a finite \group of order \(p^\alpha r\) where \(p\) is \primen and
    \(r \in \integers\) where \(r\) and \(p\) are \coprime.
    Then, a \subgroup of \(G\) of order \(p^\alpha\) is a \emph{Sylow \(p\)-subgroup} of \(G\).
\end{snippetdefinition}

\begin{snippettheorem}{sylow-theorem}{Sylow theorem}
    Let \(G\) be a finite \group of order \(p^\alpha r\) where \(p\) is \primen and
    \(r \in \integers\) where \(r\) and \(p\) are \coprime.
    Let \(n_p\) be the number of \sylowpsubgroup[Sylow \(p\)-subgroups] of \(G\). Then,
    \begin{enumerate}
        \item there exist \sylowpsubgroup[Sylow \(p\)-subgroups] in \(G\);
        \item every \(p\)-subgroup of \(G\) is contained in at least a \sylowpsubgroup;
        \item the \sylowpsubgroup[Sylow \(p\)-subgroups] are conjugated between eachother;
        \item \(n_p\) is a divisor of \(r\) and \(n_p \equiv 1 \pmod{p}\).
    \end{enumerate}
\end{snippettheorem}

\end{document}