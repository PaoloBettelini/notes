\documentclass[preview]{standalone}

\usepackage{amsmath}
\usepackage{amssymb}
\usepackage{stellar}
\usepackage{definitions}

\begin{document}

\id{sylow-theorems}
\genpage

\section{Sylow theorems}

\begin{snippetdefinition}{sylow-subgroup-definition}{Sylow subgroup}
    Let \(G\) be a finite \group of order \(p^\alpha r\) where \(p\) is \primen and
    \(r \in \integers\) where \(r\) and \(p\) are \coprime.
    Then, a \subgroup of \(G\) of order \(p^\alpha\) is a \emph{Sylow \(p\)-subgroup} of \(G\).
\end{snippetdefinition}

\begin{snippettheorem}{sylow-theorem}{Sylow theorem}
    Let \(G\) be a finite \group of order \(p^\alpha r\) where \(p\) is \primen and
    \(r \in \integers\) where \(r\) and \(p\) are \coprime.
    Let \(n_p\) be the number of \sylowpsubgroup[Sylow \(p\)-subgroups] of \(G\). Then,
    \begin{enumerate}
        \item there exist \sylowpsubgroup[Sylow \(p\)-subgroups] in \(G\);
        \item every \(p\)-subgroup of \(G\) is contained in at least a \sylowpsubgroup;
        \item the \sylowpsubgroup[Sylow \(p\)-subgroups] are conjugated between eachother;
        \item \(n_p\) is a divisor of \(r\) and \(n_p \equiv 1 \pmod{p}\).
    \end{enumerate}
\end{snippettheorem}

\begin{snippetproof}{sylow-theorem-proof}{sylow-theorem}{Sylow theorem}
    We are going to use the action of \(G\) on the \set \(X\)
    containing the subsets of cardinality \(p^\alpha\). % TODOURGENT introdotta come
    We know that the orbits containing \subgroup[subgroups]
    (exactly one each) are \(n_p\) and each one of them contains \(r\) elements.
    The other orbits contain each more than \(r\) elements
    and each of them contains a number of elements that divides \(n=p^\alpha r\).
    The divisors of \(n\) that are \coprime with \(p\)
    divide \(r\). Thus, the divisors of \(n\) that are greater of \(r\)
    are multiples of \(p\). This means that these orbits contain
    each a multiple of \(p\) of elements.
    Thus,
    \[
        |X| = n_p \cdot + p^\alpha \xi \equiv n_p r \pmod{p}
    \]
    the second term is the amount of elements
    of the orbits without subgroups.
    Remember that \(X\) contains all the subsets of cardinality \(p^\alpha\), meaning
    \[
        |X| = \binom{p^\alpha r}{p^\alpha}
    \]
    If \(G'\) is another \group of order \(p^\alpha r\),
    we get an analogous equation
    \[
        |X'| \equiv n_p' r \pmod{p}
    \]
    Clearly \(|X| = |X'|\). Thus,
    \[
        n_p r \equiv n_p' r \pmod{p}
    \]
    Since \(r\) is \coprime with \(p\), it is invertible modulo \(p\)
    \[
        n_p \equiv n_p' \pmod{p}
    \]
    We choose \(G'\) such that \(n_p\) is easy to find. If \(G'\) is a \cyclicgroup,
    we have a unique \sylowpsubgroup, meaning \(n_p' = 1\).
    Thus,
    \[
        n_p \equiv 1 \pmod{p}
    \]
    In particular \(n_p \neq 0\) (there exist \sylowpsubgroup[Sylow \(p\)-subgroups]).
    We now consider a \sylowpsubgroup \(P\) and \sylowpsubgroup[Sylow \(q\)-subgroup]
    \(Q\) of \(G\). Consider the \set \(Y\) of the conjugates of \(P\): those are all
    \sylowpsubgroup[Sylow \(p\)-subgroups]. We want to show that these are the only ones.
    Now,
    \[
        |Y| = |G \,:\, \groupnormalizer_G(P)| \divides |G \,:\, P| = r
    \]
    In particular \(p\) does not divide \(|Y|\).
    We consider the action of \(Q\) on \(Y\) by conjugation.
    The orbits of this action contain each an amount of elements
    which divides the order of \(|Q|\). Out of all the powers of \(p\),
    \(p^0\) is the only one that is not a multiple of \(p\).
    Since \(p\) does not divide \(|Y|\), meaning \(|Y| \not\equiv 0 \pmod{p}\),
    we need to have at least one orbit with only one element.
    This is an element \(P^x\) of \(P\) such that \({(P^x)}^q = p^x\)
    for every \(q\in Q\). In other words,
    \[
        Q \subgroupleq \groupnormalizer_G(P^x)
    \]
    Now, \(P^x \unlhd \groupnormalizer_G(P^x)\). Thus \(QP^x \subgroupleq \groupnormalizer_G(P^x)\)
    and thus is also a \subgroup of \(G\).
    Also,
    \[
        |QP^x| = \frac{|Q| \cdot |P^x|}{|Q \intersection P^x|}
        = \frac{|Q|}{|Q \intersection P^x|} p^\alpha
    \]
    which divides \(p^\alpha r\). Since \(p^\alpha\) is the maximal power of \(p\)
    which divides \(n = p^\alpha r\), it follows that
    \[
        \frac{|Q|}{|Q \intersection P^x|} = 1
    \]
    meaning that \(|Q| = |Q \intersection P^x|\) and \(Q = Q \intersection P^x\),
    meaning \(Q \subseteq P^x\).
    We thus showed that every \sylowpsubgroup \(Q\) is contained in a conjugate of \(P\).
    In particular, if \(Q\) is contained in a \sylowpsubgroup.
    We now consider the particular case where \(Q\) is a \sylowpsubgroup, we have
    \(Q \subseteq P^x\), meaning \(Q = P^x\) (3). Furthermore,
    \(n_p = |Y|\) and thus \(n_p \divides r\).
\end{snippetproof}

\begin{snippetcorollary}{cauchy-lemma}{Cauchy's lemma}
    Let \(G\) be a finite \group and let \(p^\beta\) be a power of a \primen \(p\)
    that divides \(|G|\). Then, there exist at least a \subgroup of order \(p^\beta\) in \(G\).
    In particular, if \(p \divides |G|\), there exist \subgroup[subgroups]
    of order \(p\) and thus elements of period \(p\).
\end{snippetcorollary}

\begin{snippetproof}{cauchy-lemma-proof}{cauchy-lemma}{Cauchy's lemma}
    We know that in \(G\) there exist at least a \sylowpsubgroup \(P\)
    of order \(p^\alpha\) where \(\beta \leq \alpha\). We also know that \(P\)
    contains \subgroup[subgroups] of every possible order, meaning some \subgroup[subgroups]
    of order \(p^\beta\) (which is also \subgroup of \(G\)).
\end{snippetproof}

\subsection{Helpful results}

\begin{snippetproposition}{product-normal-subgroups-order}{}
    Let \(G\) be a \group and \(H_1 \normalsubgrp G, H_2 \normalsubgrp G, \cdots, H_n \normalsubgrp G\)
    with orders \coprime between eachother. Then, the product
    \[
        \prod_i^n H_i
    \]
    is direct and has order
    \[
        \prod_i^n |H_i|
    \]
\end{snippetproposition}

\begin{snippetproof}{product-normal-subgroups-order-proof}{product-normal-subgroups-order}{}
    We proceed by \principleofinduction[induction] on \(n\).
    \begin{itemize}
        \item the base case is trivial;
        \item suppose the proposition is true for \(n-1\). We have that
        \[
            \prod_i^{n-1} H_i
        \]
        is a \normalsubgrptext[normal subgroup] of order
        \[
            \prod_i^{n-1} |H_i|
        \]
        Thus, \(H_1H_2\cdots H_{n-1}\) and \(H_n\) have \coprime orders.
        In particular, \(H_n \intersection H_1H_2\cdots H_{n-1} = 1\)
        by Lagrange's theorem.
    \end{itemize}
    On the other hand, this is true is we isolate any other \subgroup
    in \(H_1, H_2, \cdots, H_n\) and take its intersection with the others: the result is trivial.
    This is the condition to define the (internal) direct product of \normalsubgrptext[normal subgroups].
\end{snippetproof}

\begin{snippetproposition}{p-q-groups-distinct-primes-product-order-theorem}{}
    Let \(p,q\) be \primen[primes] where \(p<q\). Then,
    \begin{enumerate}
        \item if \(q\not\equiv 1 \pmod{p}\), there exist a unique \group of order \(pq\)
        up to isomorfism: the \cyclicgroup;
        \item if \(q\equiv 1 \pmod{p}\), there exist two \group[groups] of order \(pq\)
        up to isomorfism: the \cyclicgroup and a \abeliangroup[non-abelian group].
    \end{enumerate}
\end{snippetproposition}

\begin{snippetproof}{p-q-groups-distinct-primes-product-order-theorem-proof}{p-q-groups-distinct-primes-product-order-theorem}{}
    \todo
    % per dimostrare che è uno solo:
    % dallo studio dei campi AUt(Q) con Q ciclico ordine primo, è a sua volt ciclico,
    % ciò comporta che tutti i gruppi non abeliani di ordine pq sono isomorfi fra di loro
\end{snippetproof}

\end{document}