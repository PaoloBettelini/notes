\documentclass[preview]{standalone}

\usepackage{amsmath}
\usepackage{amssymb}
\usepackage{stellar}
\usepackage{mathrsfs}
\usepackage{bettelini}

\hypersetup{
    colorlinks=true,
    linkcolor=black,
    urlcolor=blue,
    pdftitle={Stellar},
    pdfpagemode=FullScreen,
}

\begin{document}

\title{Stellar}
\id{time-independent-quantum-mechanics}
\genpage

\plain{In this section the time is not considered, everything pertains a single instant.}

\subsection{Particle in One-Dimension}

\begin{snippet}{particle-in-one-dimension-expl}
    A particle of mass \(m\) can move freely in a one-dimensional bounded space.
    The energy of this particle is quantized, meaning it can possess only certain discrete energy values.
    Given the energy we can form a probability curve that predicts the likelihood of fnding the particle at various locations within out bounded space.
\end{snippet}

\subsection{The State Space}

\begin{snippet}{the-state-space-expl}
    For each physical system \(\mathscr{S}\), we associate a Hilbert space \(\mathcal{H}\).
    Each physical state in \(\mathscr{S}\) is represented by a point on the projective sphere (all unit vectors) of \(\mathcal{H}\).

    \[
        \text{physical state} \in \mathscr{S} \quad\longleftrightarrow\quad \hat{v} \in \mathcal{H}
    \]

    The Hilbert space \(\mathcal{H}\) is often referred to as \textit{the state space}.

    The state space for a system \(\mathscr{S}\) is a 2-dimensional complex Hilbert space

    \[
        \mathcal{H} \equiv
        \left\{
            \begin{pmatrix}
                \alpha \\
                \beta
            \end{pmatrix}
            ,\quad \alpha ,\beta \in \mathbb{C}
        \right\}
    \]

    The basis of this space is the orthogonal pair \(\{|+\rangle ,|-\rangle\}\)

    \begin{align*}
        |+\rangle &\equiv
        \begin{pmatrix}
            1 \\
            0
        \end{pmatrix} \\
        |-\rangle &\equiv
        \begin{pmatrix}
            0 \\
            1
        \end{pmatrix}
    \end{align*}

    % 158 utilizzando solo gli scalari, le rappresentazioni di +_y, +_z, +_x sarebbero uguali.
    % ma non possono esserlo
\end{snippet}

\subsection{The Operator for an Observable}

\begin{snippet}{operator-of-observable-expl}
    An observable quantity \(\mathcal{A} \in \mathscr{S}\) corresponds to a linear
    transformation \(T\). The operator \(T\) is defined by a Hermitian matrix, meaning that
    \(T_\mathcal{A}^{\dagger}=T_\mathcal{A}\).

    The linear transformations for \(S_x\), \(S_y\) and \(S_z\) are

    \[
        S_x = \frac{\hbar}{2}
        \begin{pmatrix}
            0 & 1 \\
            1 & 0
        \end{pmatrix}
        \quad\quad
        S_y = \frac{\hbar}{2}
        \begin{pmatrix}
            0 & -i \\
            i & 0
        \end{pmatrix}
        \quad\quad
        S_z = \frac{\hbar}{2}
        \begin{pmatrix}
            1 & 0 \\
            0 & -1
        \end{pmatrix}
    \]

    We can remove the \(\frac{\hbar}{2}\) term to get the \textit{Pauli spin matrices}

    \[
        \sigma_x =
        \begin{pmatrix}
            0 & 1 \\
            1 & 0
        \end{pmatrix}
        \quad\quad
        \sigma_y =
        \begin{pmatrix}
            0 & -i \\
            i & 0
        \end{pmatrix}
        \quad\quad
        \sigma_z =
        \begin{pmatrix}
            1 & 0 \\
            0 & -1
        \end{pmatrix}
    \]
\end{snippet}

\subsection{The Eigenvalues of an Observable}

\begin{snippet}{eigenvalues-of-an-observable}
    An observable quantity \(\mathcal{A}\) can be measure to be one of its \textit{eigenvalues}.
    Eigenvalues are a set of real scalars \(a_1, a_2, \cdots , a_n\) assosiates with
    an operator's matrix.
\end{snippet}

\subsection{Eigenvectors and Eigenvalues}

\begin{snippet}{eigenvectors-and-eivenvalues-qm}
    A matrix \(M\) might be assosiated with a set of \textit{eigenvector-eigenvalue} pairs.
    For each pair of eigenvalue \(a\) and eigenvector \(\vec{u}\):

    \[
        M\vec{u}=a\vec{u},
        \quad\vec{u}\neq 0
    \]

    There may be more than one eigenvector for a given eigenvalue. \\
    An eigenvalue is called
    \begin{itemize}
        \item \textbf{non-degenerate} if it corresponds to only one eigenvector
        \item \textbf{degenerate} if it works for multiple eigenvectors
    \end{itemize}

    A basis in which every vector is an eigenvector of \(M\) is called an \textit{eigenbasis}.
    If \(M\) is written in its eigenbasis, it is a diagonal matrix.
\end{snippet}

%%% PAG 163

\end{document}