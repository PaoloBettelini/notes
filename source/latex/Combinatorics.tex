\documentclass[preview]{standalone}

\usepackage{amsmath}
\usepackage{amssymb}
\usepackage{stellar}
\usepackage{definitions}
\usepackage{bettelini}

\begin{document}

\id{combinatorics}
\genpage

\section{Definitions}

\begin{snippetdefinition}{factorial-definition}{Factorial}
    Given an \(n \in \naturalnumbers\), the \textit{factorial} of \(n\) is defined as
    \[
        n\factorial = \begin{cases}
            1 & n=0 \\
            n(n-1)\factorial & \text{otherwise}
        \end{cases}
    \]
\end{snippetdefinition}

\section{Basic results}

\begin{snippettheorem}{fundamental-principle-counting-theorem}{Fundamental Principle of Counting}
    If a procedure can be executed in \(n_1\) different ways, and after that
    another procedure can be executed in \(n_2\) different ways
    up to \(n_k\), then the total amount of possible outcomes is given by
    \[
        n_1n_2\cdots n_k
    \]
\end{snippettheorem}

\begin{snippetcorollary}{fundamental-principle-counting-theorem-simple-permutations}{Simple permutations}
    The number of ways to arrange \(n\) objects is given by \(n\factorial\).
\end{snippetcorollary}

\begin{snippetcorollary}{fundamental-principle-counting-theorem-permutations}{Permutations with repetitions}
    The number of ways to arrange \(n\) objects where there are
    \(k\) types of objects which repeat and the amount of the repeating objects
    are \(n_1\), \(n_2\), \(\cdots\), \(n_k\), then the amount of possible permutations is
    given by
    \[
        \frac{n\factorial}{n_1\factorial n_2\factorial\cdots n_k\factorial}
    \]
\end{snippetcorollary}

% Disposizioni semplici

% Disposizioni semplici con ripetizione

\begin{snippetdefinition}{binomial-coefficient-definition}{Binomial coefficient}
    The \textit{binomial coefficient} is defined as
    \[
        \binom{n}{k} = \frac{n\factorial}{k\factorial(n-k)\factorial}
    \]
    It represents the number of ways to choose $k$ elements from a set of $n$ distinct
    elements without considering the order.
\end{snippetdefinition}

% Combinazioni con ripetizioni

\begin{snippettheorem}{binomial-theorem}{Binomial theorem}
    Let \(n \in \naturalnumbers\) and \(x,y \in \realnumbers\).
    \[
        {(x+y)}^n = \sum_{k=0}^n \binom{n}{k} x^{n-k}y^k = \sum_{k=0}^n \binom{n}{k} x^ky^{n-k}
    \]
\end{snippettheorem}

\begin{snippetproof}{binomial-theorem-proof}{binomial-theorem}{Binomial theorem}
    \begin{itemize}
        \item The base case is \((x+y)^0 = \binom{0}{0} = 1\).
        \item The induction case is given by
        \begin{align*}
            {(x+y)}^{n+1} &= {(x+y)}^n \cdot (x+y) \\
            &= (x+y) \cdot \sum_{k=0}^n \binom{n}{k} x^{n-k}y^k \\
            &= \sum_{k=0}^n \binom{n}{k} x^{n-k+1}y^k + \sum_{k=0}^n \binom{n}{k} x^{n-k}y^{k+1} \\
            &= \sum_{h=0}^n \binom{n}{h} x^{n+1-h}y^k + \sum_{k=0}^n \binom{n}{h-1} x^{n-k+1}y^{h} \\
            &= \sum_{h=0}^n \binom{n}{h} x^{n+1-h}y^k + \sum_{h=1}^n \binom{n}{h-1} x^{n-k+1}y^{h} \\
            &= \binom{n}{0}x^{n+1}y^0 + \sum_{h=1}^n \binom{n}{h} x^{n+1-h}y^k
                + \sum_{h=1}^n \binom{n}{h-1} x^{n-k+1}y^{h} + \binom{n}{n}x^0y^{n+1} \\
            &= \binom{n+1}{0}x^{n+1}y^0 + \sum_{k=1}^n \left[\binom{n}{h} + \binom{n}{h-1}\right]
            x^{n+1}y^h + \binom{n+1}{n+1}x^0y^{n+1} \\
            &= \sum_{h=0}^{n+1} \binom{n+1}{h}x^{n-k+1}y^h \\
            &= \sum_{k=0}^{n+1} \binom{n+1}{k} x^{n-k+1}y^k
        \end{align*}
    \end{itemize}
\end{snippetproof}

\section{Tartaglia's triangle}

\begin{snippet}{tartaglia-triangle}
    Tartaglia's triangle orders all the binomial coefficients.
    The value of each element is equal to the sum of the two adjacent elements above it,
    where all the invisible cells are considered to be \(0\).

    \[ \binom{0}{0} \]
    \[ \binom{1}{0} \binom{1}{1} \]
    \[ \binom{2}{0} \binom{2}{1} \binom{2}{2} \]
    \[ \binom{3}{0} \binom{3}{1} \binom{3}{2} \binom{3}{3} \]
    \[ \cdots \]

    Or, with computed values

    \[ 1 \]
    \[ 1\,\,1 \]
    \[ 1\,\,2\,\,1 \]
    \[ 1\,\,3\,\,3\,\,1 \]
    \[ 1\,\,4\,\,6\,\,4\,\,1 \]
    \[ \cdots \]
\end{snippet}

\begin{snippet}{tartaglia-triangle-usage-ex}
    By looking at the triangle we can easily expand values such as
    \begin{align}
        {(a+b)}^4 &= \binom{4}{0}a^4 + \binom{4}{1}a^3b + \binom{4}{2}a^2b^2+ \binom{4}{3}ab^3 + \binom{4}{4} b^4 \\
        &= a^4 + 4a^3b + 6a^2b^2 + 4ab^3a + b^4
    \end{align}
\end{snippet}

\end{document}