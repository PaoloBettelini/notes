\documentclass[preview]{standalone}

\usepackage{amsmath}
\usepackage{amssymb}
\usepackage{stellar}
\usepackage{bettelini}

\hypersetup{
    colorlinks=true,
    linkcolor=black,
    urlcolor=blue,
    pdftitle={Stellar},
    pdfpagemode=FullScreen,
}

\begin{document}

\title{Stellar}
\id{combinatorics}
\genpage

\begin{snippettheorem}{fundamental-principle-counting-theorem}{Fundamental Principle of Counting}
    If a procedure can be executed in \(n_1\) different ways, and after that
    another procedure can be executed in \(n_2\) different ways
    up to \(n_k\), then the total amount of possible outcomes is given by
    \[
        n_1n_2\cdots n_k
    \]
\end{snippettheorem}

\begin{snippetcorollary}{fundamental-principle-counting-theorem-simple-permutations}{Simple permutations}
    The number of ways to arrange \(n\) objects is given by \(n!\).
\end{snippetcorollary}

\begin{snippetcorollary}{fundamental-principle-counting-theorem-permutations}{Permutations with repetitions}
    The number of ways to arrange \(n\) objects where there are
    \(k\) types of objects which repeat and the amount of the repeating objects
    are \(n_1\), \(n_2\), \(\cdots\), \(n_k\), then the amount of possible permutations is
    given by
    \[
        \frac{n!}{n_1!n_2!\cdots n_k!}
    \]
\end{snippetcorollary}

% Disposizioni semplici

% Disposizioni semplici con ripetizione

\begin{snippetdefinition}{binomial-coefficient-definition}{Binomial coefficient}
    The \textit{binomial coefficient} is defined as
    \[
        \binom{n}{k} = \frac{n!}{k!(n-k)!}
    \]
    It represents the number of ways to choose $k$ elements from a set of $n$ distinct
    elements without considering the order.
\end{snippetdefinition}

% Combinazioni con ripetizioni

\end{document}