\documentclass[preview]{standalone}

\usepackage{amsmath}
\usepackage{amssymb}
\usepackage{stellar}
\usepackage{definitions}
\usepackage{bettelini}
\usepackage{tikz}

\usetikzlibrary{cd}

\begin{document}

\id{italiano-inferno-canto-ii}
\genpage

\section{Canto II}

\begin{snippet}{canto-ii-expl1}
    Il secondo canto serve da proemio al libro dell'Inferno.
    Tutto il secondo canto si svolge nel medesimo punto dove il primo canto termina.
    Infatti, Dante e Virgilio non si muovono.
\end{snippet}

\begin{snippetdefinition}{invocazione-muse-definition}{Invocazione alle Muse}
    Nel ricevere e far suo il tradizionale uso retorico d'invocare le Muse, quando più arduo si presenti l'impegno dell'arte.
\end{snippetdefinition}

\begin{snippet}{canto-ii-expl2}
    Dante invoca le Muse, ingegno e memoria per quello che sta per scrivere.

    A Dante giungono dei dubbi, ossia quale sia lo scopo del suo viaggio
    (la risposta verrà data in Paradiso XVII) e chi gli permetta di fare tale.
    Perché Dante ha il permesso di compiere questo viaggio?
    \\
    Lo stesso viaggio è stato compiuto solamente da San Paolo (Nuovo Testamento)
    e Enea (Eneide), uno ai cieli e uno agli inferi. I due avevano però uno scopo grande,
    cosa che Dante non possiede.
    
    Virgilio risponde alla seconda domanda mediante le seguenti affermazioni.
    È stata Beatrice ad avvisarlo che Dante fosse in difficoltà.
    La Madonna ha visto la difficoltà di Dante, l'ha detto a Santa Lucia, che va da Beatrice,
    la quale scende da Virgilio. Questa è detta una \textbf{staffetta relogiosa}.
    Per cui, il viaggio è permesso da 3 donne benedette.

    \vspace{0.25cm}

    \begin{center}
        \begin{tikzcd}
            \text{Madonna} \arrow[r] & \text{Santa Lucia} \arrow[r] & \text{Beatrice} \arrow[d] \\
            & \text{Dante}                 & \text{Virgilio} \arrow[l]
        \end{tikzcd}
    \end{center}

    \vspace{0.25cm}

    Il motivo per cui Beatrice ha scelto Virgilio è per la sua ragione e uso della parola.
    
    Beatrice si sente infatti in Debito con Dante per la sua gratitudine (\textit{Vita Nova}),
    in particolare è grata per tutto il suo amore.
\end{snippet}

\end{document}
