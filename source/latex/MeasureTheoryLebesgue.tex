\documentclass[preview]{standalone}

\usepackage{amsmath}
\usepackage{amssymb}
\usepackage{bettelini}
\usepackage{stellar}
\usepackage{definitions}

\begin{document}

\id{measuretheory-lebesgue}
\genpage

\section{Lebegue Measure}

\begin{snippet}{lebesgue-measure-meaning}
    The idea is to create a measure \(\mu\) which encapsulates
    and generalizes the classical meaning of a measure (i.e. length, area, volume, \(\cdots\)).
    However, such measure does not exist for the measurable space \((\realnumbers, \powerset(\realnumbers))\).
\end{snippet}

\begin{snippettheorem}{no-measure-on-r}{}
    Let \(M=(\realnumbers, \powerset(\realnumbers), \mu)\)
    be a measure space with
    \begin{enumerate}
        \item \(\mu([a;b]) = b-a, \quad b>1\);
        \item \textbf{traslation invariant:} \(\mu(A) = \mu(\{k + a \suchthat a \in A\}), \quad A \in \powerset(\realnumbers) \land x \in \realnumbers\).
    \end{enumerate}
    Then, \(M\) does not exist.
\end{snippettheorem}

\begin{snippetproposition}{condition-gives-only-zero-measure}{}
    Let \((\realnumbers, \powerset(\realnumbers), \mu)\) be a measure space where
    \begin{enumerate}
        \item \(\mu((0; 1]) < \infty\);
        \item \textbf{traslation invariant:} \(\mu(A) = \mu(\{k + a \suchthat a \in A\}), \quad A \in \powerset(\realnumbers) \land x \in \realnumbers\).
    \end{enumerate}
    Then, \(\mu = 0\).
\end{snippetproposition}

% proof: https://youtu.be/Ur3ofJ61bpk?list=PLBh2i93oe2qvMVqAzsX1Kuv6-4fjazZ8j

\section{Measurable maps}

\begin{snippetdefinition}{measurable-map-definition}{Measurable Map}
    Let \((\Omega_1, \Sigma_1)\) and \((\Omega_2, \Sigma_2)\) be measurable spaces. \\
    A map \(f \colon \Omega_1 \to \Omega_2\) is called \textit{measurable} (with respect to \(\Sigma_1\) and \(\Sigma_2\))
    if \(\forall \sigma_2 \in \Sigma_2, f^{-1}(\sigma_2) \in \Sigma_1\).
    That is, the preimage of any measurable set is still measurable.
\end{snippetdefinition}

\plain{if you consider a simple real function, we want the sets on the y axis to be measurable,}
\plain{as well as those on the x axis. Thus, the preimage of the x values must be measurable.}

\begin{snippetdefinition}{indicator-function-definition}{Indicator function}
    Let \(A\) be a set.
    The \textit{indicator function} of \(A\) is defined as
    \[
        1_{A}(x) = \begin{cases}
            1 & x \in A \\
            0 & x \notin A
        \end{cases}
    \]
\end{snippetdefinition}

% https://youtu.be/11heoNVavvM?list=PLBh2i93oe2qvMVqAzsX1Kuv6-4fjazZ8j&t=356

\begin{snippetproposition}{measurable-map-composition}{Composition of measurable maps}
    Let \((\Omega_1, \Sigma_1)\), \((\Omega_2, \Sigma_2)\) and \((\Omega_3, \Sigma_3)\)
    be measurable spaces.
    If the maps \(f\colon \Omega_1 \to \Omega_2\) and \(g\colon \Omega_2 \to \Omega_3\)
    are measurable, then \(f \circ g\) is also a measurable map.
\end{snippetproposition}

% also f+g, f-g, |f|

\section{Step functions and Lebesgue integration}

\begin{snippetdefinition}{step-function-definition}{Step function}
    A \textit{step function} or \textit{simple function} \(f\) is a \function that can be
    expressed as a combination of indicator functions.
    Let \(\{S\}_{i \in I}\) be a collection of disjoint measurable sets and \(a_{i \in I}\) a sequence of scalars
    \[
        f = \sum_{i \in I} 1_{S_i} a_i
    \]
\end{snippetdefinition}

% measure defined on
% mu in (-inf; inf)
% mu > 0

\end{document}
