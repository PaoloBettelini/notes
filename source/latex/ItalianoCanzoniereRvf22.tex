\documentclass[preview]{standalone}

\usepackage{amsmath}
\usepackage{amssymb}
\usepackage{stellar}
\usepackage{bettelini}

\hypersetup{
    colorlinks=true,
    linkcolor=black,
    urlcolor=blue,
    pdftitle={Stellar},
    pdfpagemode=FullScreen,
}

\begin{document}

\title{Stellar}
\id{italiano-canzoniere-rvf-22}
\genpage

\section{Rvf 22: A qualunque animale alberga in terra}

\begin{snippet}{canzoniere-rvf-22-parte1}
    \StellarPoetry{X}{
        A qualunque animale alberga in terra, \\
        se non se alquanti ch'anno in odio il sole, \\
        tempo da travagliare e quanto e 'l giorno; \\
        \textbf{ma} poi che 'l ciel accende le sue stelle, \\
        qual torna a casa et qual s'anida in selva \\
        per aver posa almeno infin a l'alba.
    }{XXX}
    \StellarPoetry{X}{
        \textbf{Et} io, da che comincia la bella alba \\
        a scuoter l'ombra intorno de la terra \\
        svegliando gli animali in ogni selva, \\
        non o mai triegua di sospir' col sole; \\
        pur quand'io veggio fiammeggiar le stelle \\
        vo lagrimando, et disiando il giorno.
    }{XXX}
    \StellarPoetry{X}{
        Quando la sera scaccia il chiaro giorno, \\
        et le tenebre nostre altrui fanno alba, \\
        miro pensoso le crudeli stelle, \\
        che m'anno facto di sensibil terra; \\
        et maledico il di ch'i' vidi 'l sole, \\
        e che mi fa in vista un huom nudrito in selva.
    }{XXX}
    \StellarPoetry{X}{
        Non credo che pascesse mai per selva \\
        si aspra fera, o di nocte o di giorno, \\
        come costei ch'i 'piango a l'ombra e al sole; \\
        et non mi stancha primo sonno od alba: \\
        che, bench'i' sia mortal corpo di terra, \\
        lo mi fermo \textbf{desir} vien da le stelle.
    }{XXX}
    \StellarPoetry{X}{
        Prima ch'i' torni a voi, lucenti stelle, \\
        o torni giu ne l'\textbf{amorosa selva}, \\
        lassando il corpo che fia trita terra, \\
        vedess'io in lei pieta, che 'n un sol giorno \\
        puo ristorar molt'anni, e 'nanzi l'alba \\
        puommi arichir dal tramontar del sole.
    }{XXX}
    \StellarPoetry{X}{
        Con lei foss'io da che si parte il sole, \\
        et non ci vedess'altri che le stelle, \\
        sol una nocte, et mai non fosse l'alba; \\
        et non se transformasse in verde selva \\
        per uscirmi di braccia, come il giorno \\
        ch'Apollo la seguia qua giu per terra.
    }{XXX}
    \StellarPoetry{X}{
        Ma io saro sotterra in secca selva \\
        e 'l giorno andra pien di minute stelle \\
        prima ch'a si dolce alba arrivi il sole. 
    }{XXX}
    
    La metrica è caratterizzata da sestine con parole-rime: \\
    \textbf{A:} \quotes{terra}, \textbf{B:} \quotes{sole},
    \textbf{C:} \quotes{giorno}, \textbf{D:} \quotes{stelle},
    \textbf{E:} \quotes{selva}, \textbf{F:} \quotes{alba}
    in retrogradatio cruciata e congedo (A)E(C)D(F)B. \\
    Fra queste parole vi è un rapporto di contenimento di sineddoche (alba-giorno, sole-stelle e selva-terra).
\end{snippet}

\begin{snippetdefinition}{retrogradatio-cruciata-definition}{Retrogradatio Cruciata}
    Con \textit{retrogradatio cruciata} (retrogradazione incrociata)
    si indica un principio di rotazione per il quale le sei parole-rima
    della prima stanza ritornano sempre e obbligatoriamente nelle cinque stanze successive.
\end{snippetdefinition}
    
\begin{snippet}{canzoniere-rvf-22-parte2}
    \circled{1} L'autore si rivolge inizialmente verso tutti gli animali della terra,
    i quali, eccetto quelli notturni, faticano e soffrono durante il giorno.
    Tuttavia, quando il cielo si oscura e le stelle appaiono,
    gli animali tornano alle loro tane o ai loro rifugi per trovare
    riposo almeno fino all'alba.
    Questa descrizione riflette il ciclo naturale di attività e riposo,
    e potrebbe essere interpretata anche come una metafora della condizione umana,
    con le fasi di lavoro e di riposo nella vita di ognuno di noi. \\
    Questa sestina può essere divisa da un \quotes{ma} avversativo che separa il giorno e la notte.
    L'opposizione non è solamente luce contro buio, ma anche un'opposizione fra occupazione/sofferenza e riposo.
    \\\\
    \circled{2} Questa introduzione è molto generale e l'autore non parla di sè stesso.
    Nella seconda sestina, invece, descrive i suoi sentimenti contrastanti nei confronti
    del giorno e della notte.
    Inizia dicendo che quando inizia l'alba e il sole inizia a dissipare
    le ombre intorno alla terra, facendo svegliare gli animali nelle foreste,
    lui non smette mai di sospirare finché c'è luce solare.
    Quando vede brillare le stelle di notte, piange e desidera che sia giorno. \\
    Questa sestina può essere divisa fra i primi quattro versi e gli ultimi due:
    la prima rivolta al giorno e la seconda verso la notte.
    A differenza della prima sestina, questa è interamente incentrata sull'io poetico.
    La parola \quotes{Et} del primo verso lo distingue da tutto il resto del mondo.
    Questa sua incessante sofferenza nel tempo, senza un momento di tregua,
    si confonde fra il giorno e la notte.
    \\\\
    \circled{3} Contrariamente a prima, la notte non scaccia più il sole, bensì
    le stesse gli portano il sentimento del dolore poiché uomo (\quotes{sensibil terra},
    come Dio creò l'uomo dal fango).
    Inoltre, Petrarca maledice il giorno in cui è nato, il cui Sole lo rende come un uomo selvaggio cresciuto nei boschi.
    \\
    Questi versi riflettono un senso di inquietudine,
    disgusto o disincanto nei confronti della vita umana e delle sue circostanze,
    e trasmettono una sensazione di scontento e disillusione verso il proprio destino.
    La maledizione del giorno della nascita è paragonata al giorno in cui Petrarca vide Laura (Rvf. 3).
    \\\\
    \circled{4} Alla quarta strofa troviamo, per la prima volta, la causa delle sofferenze ininterrotta dell'autore,
    ossia Laura. Ne esalta la bellezza e la purezza ed egli afferma che nemmeno una bestia selvaggia,
    né di notte né di giorno, si sarebbe mai nutrita in una foresta così severa
    e selvaggia come la donna di cui lui piange all'ombra e sotto il sole.
    La sua amata è così straordinaria che il sonno o
    l'alba non possono allontanare il suo desiderio per lei.
    Anche se egli è un essere mortale fatto di materia terrena,
    il suo desiderio per lei sembra provenire direttamente dalle stelle,
    elevandolo al di là dell'umano e conferendo alla sua passione un carattere
    quasi divino o trascendente. \\
    Questi versi si possono direttamente collegare alla filosofia di Platone, il quale
    diceva che tutte le idee erano nelle stesse, e le cose sulla terra ne erano le implementazioni
    imperfette.
    \\\\
    \circled{5} Questi versi esprimono il desiderio di Petrarca di poter vivere abbastanza
    per tornare a contemplare le stelle prima di morire.
    Egli spera di tornare a guardarle prima di essere ridotto in terra,
    cioè prima della sua morte, chiedendo che la sua amata possa mostrargli pietà.
    Petrarca crede che un solo giorno trascorso con lei possa rigenerare e rivitalizzare
    molti anni di sofferenza, sottolineando il potere e l'importanza del suo amore per lui.
    La speranza di arricchirsi dall'alba al tramonto del sole sottolinea quanto l'amore possa
    influenzare e arricchire la vita di una persona anche nel breve spazio di un giorno. \\
    Con \quotes{amorosa selva} non si intende l'inferno dantesco, bensì quello pagano in cui
    si trova chi muore da innamorato.
    \\\\
    \circled{6} Nell'ultima strofa viene descritta in maniera più approfondita l'ipotetica notte con Laura.
    Una notte da solo con lei, senza testimoni se non le stesse, senza l'alba che li sopraggiunga.
    Viene fatto un riferimento ad un mito dove Apollo, il dio del Sole, segue Dafne.
    Dafne, chiedendo aiuto agli Dei, viene trasformata in un'albero per sfuggire ad Apollo.
    Viene quindi espressa la paura che Laura, come Dafne, diventi una pianta e gli sfuggisca dalle braccia.
    La pianta di questo mito è l'alloro, il quale viene anche chiamato \textit{lauro}.
    \\\\
    \circled{7} Tuttavia, questo desiderio è destinato a infrangersi perché Petrarca sarà già morto
    e sepolto, e quel giorno sarà pieno di stelle, come per dire che non avverrà mai siccome
    il giorno non può essere pieno di stelle (adynaton).
    \\\\
    La zona dell'inferno dedicata agli spiriti amanti possiede solo piante sempreverdi.
    Tuttavia, la selva viene definita come \quotes{secca}.
    Questo indica che Petrarca smetterà quindi di amarla prima di morire.
\end{snippet}

\begin{snippetdefinition}{adynaton-definition}{Adynaton}
    L'\textit{adynaton} è una figura retorica per la quale qualcosa succederà all'avvenire
    di qualcosa di sicuramente irrealizzabile, e di conseguenza non succederà mai.
\end{snippetdefinition}

\end{document}