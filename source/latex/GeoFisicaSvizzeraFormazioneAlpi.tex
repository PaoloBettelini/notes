\documentclass[preview]{standalone}

\usepackage{amsmath}
\usepackage{amssymb}
\usepackage{tikz}
\usepackage{stellar}
\usepackage{bettelini}

\hypersetup{
    colorlinks=true,
    linkcolor=black,
    urlcolor=blue,
    pdftitle={Assets},
    pdfpagemode=FullScreen,
}

\begin{document}

\title{Geografia}
\id{geofisica-svizzera-formazione-alpi}
\genpage

\section{La formazione delle alpi}

\begin{snippet}{0b094466-05e1-4d2c-a226-aa4649fd63a7}
    L'inizio della struttura geologica della Svizzera è dato dal formarsi delle Alpi. Tale processo
    ha avuto molteplici fasi. Nel mesozoico, il medioevo geologico, la Svizzera era ancora ricoperta
    dalle acque. Era parte di un esteso mare che i geologi chiamano mare di Tethys (Tetide). Sul
    fondo di questo si depositò il materiale trasportato dai fiumi. Gli strati continuarono a
    depositarsi uno sull'altro fino a raggiungere uno spessore valutato in varie migliaia di metri.

    La formazione delle rocce sedimentarie dipendeva da vari fattori, principalmente dalla qualità
    del materiale sedimentario depositato e dalle caratteristiche delle zone di mare nelle quali
    queste sedimentazioni, questi consolidamenti e queste formazioni di rocce ebbero luogo. Si
    distinguono essenzialmente tre zone di sedimentazione, l'area elvetica, comprendente la costa
    settentrionale del Tethys, la penninica comprendente i fondali centrali, e la alpino-orientale
    comprendente la costa meridionale. La crosta terrestre è suddivisa in diverse zolle, che
    possono spingersi l'una verso l'altra: da ciò risulta il fatto che verso la fine del medioevo
    geologico - e cioè grosso modo 100 milioni di anni fa - tutto l'insieme della terraferma situata
    a sud del mare di Tethys si mise in movimento dirigendosi verso nord.
    In una prima fase, e fino a enormi profondità, la zona alla deriva spinse i sedimenti depositati
    in fondo al mare avanti a sé. La resistenza della massa di terraferma situata a settentrione
    provocò il piegamento in falde di questi sedimenti. Prime a piegarsi furono le sedimentazioni
    più molli del fondo marino situate nella zona penninica. Presumibilmente in seguito a tale
    movimento il fondale marino emerse dall'acqua formando lunghe catene di isole. In un
    crescendo di spinte titaniche seguì la seconda fase. A causa della pressione sempre crescente
    le masse sedimentarie alpino-orientali, che inizialmente si trovavano ancora molto a sud, si
    misero in moto fino a sovrapporsi alle pieghe degli strati penninici e invasero l'area di sedi
    mentazione elvetica. Queste pieghe, che si protendono molto più in senso orizzontale di
    quanto non si elevino verticalmente con le loro sinclinali ma che ricoprono anche superfici
    caratterizzate da una massa rocciosa di diverso tipo, vengono chiamate falde di ricoprimento
    o «couches». Tutta la regione alpina orientale si è formata dalla sovrapposizione di tali falde
    (est-alpine). Dove questo strato di copertura, che forma un unico insieme, venne perforato in
    seguito ad erosioni - per esempio nella Bassa Engadina - troviamo le cosiddette «finestre
    geologiche», che portano alla luce strati penninici più profondi.
    La parte occidentale e quella orientale delle Alpi si formarono in modo assai più complesso, a
    causa di fenomeni geologici verificatisi nella terza fase principale della formazione delle mon
    tagne. Non appena l'azione di movimenti interni determina differenze di altitudine, l'erosione
    comincia a demolire le alture in formazione. Gli agenti atmosferici disgregano, decompongono
    e frantumano la roccia. Le acque correnti trasportano i detriti dell'erosione. I fiumi
    approfondiscono sempre più il loro alveo, erodono le montagne e producono il loro
    abbassamento smantellandole.
\end{snippet}

\end{document}