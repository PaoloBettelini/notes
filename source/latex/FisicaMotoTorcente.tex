\documentclass[preview]{standalone}

\usepackage{amsmath}
\usepackage{amssymb}
\usepackage{stellar}

\hypersetup{
    colorlinks=true,
    linkcolor=black,
    urlcolor=blue,
    pdftitle={Stellar},
    pdfpagemode=FullScreen,
}

\begin{document}

\title{Stellar}
\id{fisica-moto-torcente}
\genpage

\section{Moto torcente}

\begin{snippet}{moto-torcente-expl}
    Il momento di una forza o momento torcente è una grandezza fisica che permette di descrivere l'azione torcente di una forza in relazione al punto di applicazione della forza rispetto a un asse di rotazione.
di una forza in relazione al punto di applicazione della forza rispetto a un asse di rotazione.
\\ L'unità di misura è [N \(\cdot\) m].
\end{snippet}

\section{Definizione}

\begin{snippetdefinition}{momento torcente-definition}{Momento torcente}
    Il \textit{momento torcente} è definito come
    \[ \tau = \vec{r} \times \vec{F} \]
    dove \(\vec{r}\) è il vettore della posizione dell'applicazione della forza
    e \(\vec{F}\) è il vettore della forza.
\end{snippetdefinition}

\begin{snippet}{moto-torcente-expl2}
    Using the \(3\)-dimensional definition of the scalar product

    \[
        \vec{M}=
        \begin{pmatrix}
            R_y \cdot F_z - R_z \cdot F_y \\
            R_z \cdot F_x - R_x \cdot F_z \\
            R_x \cdot F_y - R_y \cdot F_x
        \end{pmatrix}
    \]

    Since two vectors are always coplanar, it is possible to choose a coplanar reference system

    \[
        \vec{R}=
        \begin{pmatrix}
            R_1 \\
            R_2 \\
            0
        \end{pmatrix}
        ,\quad
        \vec{F}=
        \begin{pmatrix}
            F_1 \\
            F_2 \\
            0
        \end{pmatrix}
        ,\quad
        \vec{M}=
        \begin{pmatrix}
            0 \\
            0 \\
            R_1 \cdot F_2 - R_2 \cdot F_1
        \end{pmatrix}
    \]

    \(\vec{M}\) can also be computed by \(\vec{R}\cdot\vec{F}\cdot\sin(\alpha)\) where \(\alpha\)
    is the angle between \(\vec{R}\) and \(\vec{F}\).

    Sometimes we interpret the formula \(|\vec{M}|=\vec{R}\cdot\vec{F}_\perp\) as
    the twisting effect of the force component perpendicular to \(\vec{R}\), or
    \(|\vec{M}|=b\cdot\vec{F}\) where \(b\) is the lever arm.

    The sign convention for this measurement is

    \[
        \begin{cases}
            +1, \quad \text{if the rotation is counterclockwise} \\
            -1, \quad \text{if the rotation is clockwise}
        \end{cases}
    \]
\end{snippet}

\subsection{Condizioni di equilibrio}

\begin{snippet}{condizioni-equilibrio}
    Affinché un sistema sia in equilibrio, la forza e i momenti di torsione risultanti devono essere nulli.
    \[
        \begin{cases}
            \sum \vec{F}_j=0 \\
            \sum \vec{M}_j=0
        \end{cases}
    \]
    \phantom{}
\end{snippet}

\end{document}