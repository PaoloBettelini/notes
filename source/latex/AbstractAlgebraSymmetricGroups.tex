\documentclass[preview]{standalone}

\usepackage{amsmath}
\usepackage{amssymb}
\usepackage{stellar}
\usepackage{definitions}

\begin{document}

\id{symmetric-groups}
\genpage

\section{Definition}

\begin{snippetdefinition}{symmetric-group-definition}{Symmetric group}
    Consider a \set \(X\) and the \monoid \((X^X, \text{composition})\).
    Then, the \group induced by \((\text{Inv}(X^X), \text{composition})\)
    is called the \textit{symmetric group}
    \[
        \text{Sym}(X)
    \]
\end{snippetdefinition}

\begin{snippetdefinition}{permutation-group-definition}{Permutation group}
    Let \(X\) be a finite \set of symbols and \(n = \cardinality{X}\). Then, the \symmetricgroup
    over the permutations of \(X\) is called the \textit{permutation group}
    \[
        \text{Sym}_n
    \]
\end{snippetdefinition}

\begin{snippettheorem}{order-of-permutation-group-theorem}{Order of permutation group}
    Let \(X\) be a finite \set of symbols and \(n = \cardinality{X}\). Then,
    \[
        \cardinality{\permgrp_n} = n\factorial
    \]
\end{snippettheorem}

\begin{snippettheorem}{smallest-non-abelian-group-theorem}{Smallest non-abelian group}
    The smallest \abeliangroup[non abelian group] is \(\permgrp_3\), which has order \(6\).
\end{snippettheorem}

\plain{Every group of order less than 6 is abelian.}

\end{document}