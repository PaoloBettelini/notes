\documentclass[preview]{standalone}

\usepackage{amsmath}
\usepackage{amssymb}
\usepackage{stellar}
\usepackage{definitions}

\begin{document}

\id{two-body-problem}
\genpage

\section{Two-body problem reduction}

\begin{snippet}{two-body-problem-reduction}
    Consider two bodies (point particles) that interact only with one another.
    We have the following system
    \[
    \begin{cases}
        m_1\vec{a}_1 = \vec{F}_{1,2} \\
        m_2\vec{a}_2 = \vec{F}_{2,1}
    \end{cases}
    \]
    where \(\vec{F}_{1,2} = - \vec{F}_{2,1}\).
    We now write
    \begin{align*}
        \vec{a}_1 - \vec{a}_2 &= \frac{\vec{F}_{1,2}}{m_1} - \frac{\vec{F}_{2,1}}{m_2} \\
        &= \vec{F}_{1,2} \left(\frac{1}{m_1} + \frac{1}{m_2}\right)
    \end{align*}
    Let the reduced mass \(\mu\) be defined as
    \[
        \frac{1}{\mu} = \frac{1}{m_1} + \frac{1}{m_2}
    \]
    We now have
    \[
        \mu(\vec{a}_1 - \vec{a}_2) = \vec{F}_{1,2}
    \]
    This is the acceleration of particle 1 as seen by particle 2.
    If the observer is on particle 2, it is not in an inertial frame of reference.
    However, it can still use Newton's laws if the mass is replaced with \(\mu\).
    We now define the center of mass as
    \[
        R \triangleq \frac{m_1x_1 + m_2x_2}{m_1 + m_2}
    \]
    If we take the second derivative
    \[
        \frac{d^2R}{dt^2} = \frac{m_1\vec{a}_1 + m_2\vec{a}_2}{m_1 + m_2} = 0
    \]
    this shows that the velocity of the center of mass is constant (and so is the momentum).
    The center of mass can be determined by the initial conditions.
\end{snippet}

\end{document}