\documentclass[preview]{standalone}

\usepackage{amsmath}
\usepackage{amssymb}
\usepackage{stellar}
\usepackage{bettelini}

\hypersetup{
    colorlinks=true,
    linkcolor=black,
    urlcolor=blue,
    pdftitle={Stellar},
    pdfpagemode=FullScreen,
}

\begin{document}

\id{geoeconomica-nuova-via-seta}
\genpage

\section{Nuova via della seta}

\begin{snippetdefinition}{via-seta-definition}{Via della seta}
    La \textit{via della seta} consiste nel reticolo, che si sviluppava per circa 8 000 km,
    costituito da itinerari terrestri, marittimi e fluviali lungo i quali nell'antichità si
    erano snodati i commerci tra l'Impero cinese e l'Impero romano.
\end{snippetdefinition}

\begin{snippetdefinition}{nuova-via-seta-definition}{Nuova via della seta}
    La \textit{nuova via della seta} è un'iniziativa strategica della Repubblica
    Popolare Cinese del 2013 per il miglioramento dei suoi collegamenti commerciali con
    i paesi nell'Eurasia. Comprende le direttrici terrestri della \quotes{zona economica della via della seta}
    e la \quotes{via della seta marittima del XXI secolo}.
\end{snippetdefinition}

\begin{snippet}{ea43b2a9-2b72-4c19-bab2-50692ba26f2a}
    In inglese questa iniziativa viene chiamata \textbf{Belt and Road Initiative} (BRI),
e in Cina \textbf{One Belt One Road} (OBOR).
\end{snippet}

\begin{snippetexercise}{obor-ex-1}
    {Quali obiettivi (espliciti ed impliciti) persegue il progetto OBOR (One Belt One Road) / BRI (Belt and Road initiative)?}
    Gli obiettivi del progetto sono quelli di aumentare l'influenza della Cina
    con maggiori sbocchi commerciali (prodotti industriali cinesi, acciaio, cemento etc.),
    linee energetiche, gas e petrolio, quello di ridurre il tempo di trasporto di merci
    e creare legami di dipendenza nei confronti delle altre nazioni (con i prestiti).
\end{snippetexercise}

\begin{snippetexercise}{obor-ex-2}
    {Quali attori potrebbero trarre più vantaggi che svantaggi da questo progetto?}
    In primis abbiamo la Cina, i paesi contenenti punti che vengono attraversati
    e i punti di arrivo.
\end{snippetexercise}

\begin{snippetexercise}{obor-ex-3}
    {Quali attori potrebbero invece subire più svantaggi che vantaggi?}
    Gli aspetti negativi si verificano piuttosto nei confronti dell'ambiente o
    dei lavoratori che vengono sfruttati.
\end{snippetexercise}

\begin{snippetexercise}{obor-ex-4}
    {Che rilevanza assume il progetto OBOR/BRI nel processo di globalizzazione}
    XXX
\end{snippetexercise}

\end{document}