\documentclass[preview]{standalone}

\usepackage{amsmath}
\usepackage{amssymb}
\usepackage{stellar}
\usepackage{bettelini}

\hypersetup{
    colorlinks=true,
    linkcolor=black,
    urlcolor=blue,
    pdftitle={Stellar},
    pdfpagemode=FullScreen,
}

\begin{document}

\id{geoeconomica-piano-marshall}
\genpage

\section{Il piano Marshall}

\begin{snippetdefinition}{piano-marshall-definition}{Piano Marshall}
    Il \textit{Piano Marshall}, ufficialmente noto come Piano di Ricostruzione Europea,
    è stato un massiccio piano di aiuti economici e finanziari offerti dagli Stati Uniti
    all'Europa occidentale dopo la Seconda Guerra Mondiale.
    Proposto dal Segretario di Stato degli Stati Uniti George Marshall nel 1947,
    questo piano mirava a sostenere la ricostruzione economica e a prevenire
    il diffondersi del comunismo offrendo assistenza finanziaria, materiale e tecnica.
\end{snippetdefinition}

\begin{snippet}{piano-marshall-expl}
    George Marshall, allora segretato di stato americano, elaborò un piano che prevedeva prestiti e sostegno economico
    per la ricostruzione fisica e il recupero economico dell'Europa. L'obiettivo era reintegrare le amministrazioni
    europee nel sistema commerciale globale e rinvigorire gli scambi economici tra Europa e Stati Uniti d'America.
    \\\\
    Nell'anno 1947, gli Stati Uniti procedettero con l'avvio del Piano Marshall.
    Marshall concepì il suo piano non soltanto per ragioni economiche, ma anche per motivi politici volti a contrastare
    l'espansione del comunismo in Europa. L'idea mirava a delineare con maggiore chiarezza la divisione tra i paesi
    che aderivano alle ideologie capitaliste da quelli comunisti.
    \\\\
    Come previsto dagli Stati Uniti, l'Unione Sovietica proibì ai suoi stati satellite di accettare l'aiuto previsto dal
    Piano Marshall e ciò aiutò maggiormente a distinguere l'Est comunista dall'Ovest democratico.
\end{snippet}

\end{document}