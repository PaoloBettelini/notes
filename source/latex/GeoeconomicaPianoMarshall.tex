\documentclass[preview]{standalone}

\usepackage{amsmath}
\usepackage{amssymb}
\usepackage{stellar}
\usepackage{bettelini}

\hypersetup{
    colorlinks=true,
    linkcolor=black,
    urlcolor=blue,
    pdftitle={Stellar},
    pdfpagemode=FullScreen,
}

\begin{document}

\title{Geografia economica}
\id{geoeconomica-piano-marshall}
\genpage

\begin{snippetdefinition}{piano-marshall-definizione}{Piano Marshall}
    Il \textit{Piano Marshall}, ufficialmente noto come Piano di Ricostruzione Europea,
    è stato un massiccio piano di aiuti economici e finanziari offerti dagli Stati Uniti
    all'Europa occidentale dopo la Seconda Guerra Mondiale.
    Proposto dal Segretario di Stato degli Stati Uniti George Marshall nel 1947,
    questo piano mirava a sostenere la ricostruzione economica e a prevenire
    il diffondersi del comunismo offrendo assistenza finanziaria, materiale e tecnica.
\end{snippetdefinition}

\end{document}