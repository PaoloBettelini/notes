\documentclass[preview]{standalone}

\usepackage{amsmath}
\usepackage{amssymb}
\usepackage{stellar}
\usepackage{bettelini}

\hypersetup{
    colorlinks=true,
    linkcolor=black,
    urlcolor=blue,
    pdftitle={Stellar},
    pdfpagemode=FullScreen,
}

\begin{document}

\title{Geografia economica}
\id{geoeconomica-demografia}
\genpage

% https://moodle.edu.ti.ch/libe/pluginfile.php/119548/mod_resource/content/1/Pass%2004c%20Sviluppo-C%2023-24_schede.pdf

% grafico della popolazione
% fase i -1928
% punto di flesso
% fase ii 1928-2100

\begin{snippetdefinition}{tasso-natalita-definizione}{Tasso di natalità}
    Con \textit{tasso di natalità}
    si intende il numero di nascite sul totale della popolazione
    (generalmente indicato in \textperthousand).
\end{snippetdefinition}

\begin{snippetdefinition}{tasso-mortalita-definizione}{Tasso di mortalità}
    Con \textit{tasso di mortalità}
    si intende il numero di morti sul totale della popolazione
    (generalmente indicato in \textperthousand).
\end{snippetdefinition}

\begin{snippetdefinition}{tasso-crescita-naturale-definizione}{Tasso di crescita naturale}
    Con \textit{tasso di crescita naturale}
    si intende la differenza fra il tasso di natalità e quello di mortalità
    (generalmente indicato in \textperthousand).
\end{snippetdefinition}

\begin{snippetdefinition}{tasso-fecondità-definizione}{Tasso di fecondità}
    Con \textit{tasso di fecondità}
    si intende il numero medio di figli per donna.
\end{snippetdefinition}

\end{document}