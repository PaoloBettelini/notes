\documentclass[preview]{standalone}

\usepackage{amsmath}
\usepackage{amssymb}
\usepackage{bettelini}
\usepackage{stellar}
\usepackage[version=4]{mhchem}

\hypersetup{
    colorlinks=true,
    linkcolor=black,
    urlcolor=blue,
    pdftitle={Chimica},
    pdfpagemode=FullScreen,
}

\begin{document}

\id{chimica-redox-esercizi}
\genpage

\begin{snippetexercise}{redox-ex-1}{Scrivi le semireazioni di ossidoriduzione}
    Data la reazione
    \[
        2\text{Na} + \text{Cl}_2 \rightarrow 2\text{NaCl}
    \]
    Con i numeri di ossidazione
    \begin{align*}
        &\,\bullet \text{Na} & 0 &&&&&&\\
        &\,\bullet \text{Cl}_2 & 0 &&&&&&\\
        &\,\bullet \text{NaCl} & +1, -1 &&&&&&\\
    \end{align*}
    Le equazioni di semi reazione sono
    \begin{align*}
        2\text{Na} &\rightarrow 2\text{Na}^+ + 2\text{e}^- \\
        \text{Cl}_2 + 2\text{e}^- &\rightarrow 2\text{Cl}^-
    \end{align*}
\end{snippetexercise}

\begin{snippetexercise}{redox-ex-2}{Trova il numero di ossidazione}
    \begin{align*}
        &\,\bullet \text{NaCLO}_3 & +1, +5, -2 &&&&&&\\
        &\,\bullet \text{SnCLl}_4 & +4, -1 &&&&&&\\
        &\,\bullet \text{MnO}_4^{2-} & +6, -2 &&&&&&\\
        &\,\bullet \text{MnO}_2 & +4, -2 &&&&&&\\
        &\,\bullet \text{H}_2\text{O}_2 & 1, -1&&&&&&
    \end{align*}
    L'idrogeno ha solo un elettrone, per cui il numero di ossidazione dell'ossigeno deve essere
    ridotto.
\end{snippetexercise}

\begin{snippetexercise}{redox-ex-3}{Stechiometrica e reazione di ossi-riduzione}
    Una lamina metallica di zinco elementare della massa iniziale
    di 5,474 grammi viene immersa in una soluzione
    acquosa azzurra contenente solfato di rame
    (CuSO\({}_4\)).
    Si osserva che sulla superficie di contatto
    tra la lamina e la soluzione avviene una reazione
    chimica che porta alla formazione
    di rame elementare in polvere.
    Inoltre, la soluzione acquosa si decolora lentamente
    (diminuzione del contenuto di solfato di rame).
    Oltre al rame elementare,
    dalla reazione viene prodotto solfato di zinco
    (ZnSO\({}_4\)), disciolto in soluzione.

    Dopo circa 24 ore si nota che tutto il solfato di rame si è consumato.
    A questo punto, la rimanente lamina di zinco viene tolta dalla soluzione. In seguito,
    vien sciacquata, asciugata ed infine pesata.
    La massa finale della lamina ammonta a 4,928 grammi.
    Il rame elementare prodotto, una volta separato
    dalla soluzione, viene anch'esso pesato.

    \begin{enumerate}
        \item \textbf{Scrivi l'equazione chimica bilanciata relativa alla
        trasformazione chimica. Per ognuna delle sostanze esplicita
        il rispettivo stato di aggregazione.}

        Il numeri di ossidazione sono
        \begin{align*}
            &\,\bullet \text{Zn} & 0 &&&&&&\\
            &\,\bullet \text{Cu} & 0 &&&&&&\\
            &\,\bullet \text{CuSO}_4 & +2, +6, -2 &&&&&&\\
            &\,\bullet \text{ZnSO}_4 & +2, +6, -2 &&&&&&
        \end{align*}
        Il rame si riduce e lo zinco si ossida.

        \item \textbf{Quale elemento si è ossidato e quale invece si è ridotto?}

        \item \textbf{Scrivi le equazioni chimiche relative alla semi-reazione
        di riduzione rispettivamente di ossidazione.}

        \item \textbf{Determina la massa di rame elementare prodotta.}

        \item \textbf{Determina la massa del solfato di rame disciolto nella
        soluzione azzurra iniziale.}

        \item \textbf{Determina la massa di solfato di zinco presente nella soluzione finale incolore.}

        \item \textbf{Quale massa finale della lamina di zinco si sarebbe misurata se la massa di rame
        elementare prodotta fosse state 1,000 grammi?}

        \item \textbf{Quale quantità in elettroni (in moli) sono stati trasferiti in totale
        durante tutta la durata del processo di ossido-riduzione?}
    \end{enumerate}
\end{snippetexercise}

\end{document}
