\documentclass[preview]{standalone}

\usepackage{amsmath}
\usepackage{amssymb}
\usepackage{stellar}
\usepackage{definitions}
\usepackage{bettelini}

\begin{document}

\id{multivariablecalculus-exercises}
\genpage

\section{Limits}

\begin{snippetexercise}{multivariable-limit-ex-1}{}
Solve the limit \(\lim _{(x, y) \rightarrow(2,1)} \frac{x^2-3 y}{x+2 y^2}\)
\end{snippetexercise}

\begin{snippetsolution}{multivariable-limit-ex-1-sol}{}
By plugging the values \[
    \lim _{(x, y) \rightarrow(2,1)} \frac{x^2-3 y}{x+2 y^2}
    = \frac{2^2-3}{2+2} = \frac{1}{4}
\]
\end{snippetsolution}

\begin{snippetexercise}{multivariable-limit-ex-2}{}
Solve the limit \(\lim _{(x, y) \rightarrow(0,0)} \frac{\sin \left(x^2+y^2\right)}{x^2+y^2}\)
\end{snippetexercise}

\begin{snippetsolution}{multivariable-limit-ex-2-sol}{}
By using polar coordinate \(x=r\cos\theta\) and \(y=r\sin\theta\),
and thus \(x^2+y^2=r^2\), therefore
\[
    \lim_{(x, y) \rightarrow(0,0)} \frac{\sin \left(x^2+y^2\right)}{x^2+y^2}
    = \lim_{r \rightarrow 0} \frac{\sin \left(r\right)}{r}=1
\]
\end{snippetsolution}

\begin{snippetexercise}{multivariable-limit-ex-3}{}
Solve the limit \(\lim _{(x, y) \rightarrow(0,0)} \frac{x y^2}{x^2+y^4}\)
\end{snippetexercise}

\begin{snippetexercise}{multivariable-limit-ex-4}{}
Solve the limit \(\lim _{(x, y) \rightarrow(0,0)} \frac{x^2 y}{x^2+y^4}\)
\end{snippetexercise}

\section{Level curves}

\begin{snippetexercise}{multivariable-level-curve-ex-1}{}
    Find the equation of the level curve of the function \(f(x,y) = 16-x^2-y^2\)
    that passes through the point \(p=(2\sqrt{2}, \sqrt{2})\).
\end{snippetexercise}

\begin{snippetsolution}{multivariable-level-curve-ex-1-sol}{}
    \phantom{}\(f(2\sqrt{2}, \sqrt{2}) = 6\), so the height of the plane is 6.
    By plugging \(z=6\) in we get \(6=16-x^2-y^2\)
    and thus the level curve is given by \(10=x^2+y^2\).
\end{snippetsolution}

\begin{snippetexercise}{multivariable-level-curve-ex-2}{}
    Find the equation of the level curve of the function \(f(x,y) = \sqrt{x^2-1}\)
    that passes through the point \(p=(1,0)\).
\end{snippetexercise}

\begin{snippetsolution}{multivariable-level-curve-ex-2-sol}{}
    \phantom{}\(f(1, 0) = 0\), so the height of the plane is 0.
    By plugging \(z=0\) in we get \(0 = \sqrt{x^2-1}\)
    and thus the level curve is given by \(x^2=1\), that is
    the lines \(x=1\) and \(x=-1\).
\end{snippetsolution}

\begin{snippetexercise}{multivariable-level-curve-ex-3}{}
    Find the equation of the level curve of the function \(f(x,y) = \int_x^y \frac{d\theta}{\sqrt{1-\theta^2}}\)
    that passes through the point \(p=(0,1)\).
\end{snippetexercise}

\begin{snippetsolution}{multivariable-level-curve-ex-3-sol}{}
    We first solve the integral
    \[ \int_x^y \frac{d\theta}{\sqrt{1-\theta^2}} = \arcsin(y) - \arcsin(x) \]
    Now we can evaluate \(f(0,1) = \arcsin(1) - \arcsin(0) = \frac{\picircle}{2}\),
    so the height of the plane is \(\frac{\picircle}{2}\).
    By plugging \(z=\frac{\picircle}{2}\) in we get
    \begin{align*}
        &\frac{\picircle}{2} = \arcsin(y) - \arcsin(x) \\
        &y = \sin\left( \frac{\picircle}{2} + \arcsin(x) \right) \\
        &y = \sqrt{1-x^2}
    \end{align*}
    with \(x < 0\).
\end{snippetsolution}

\end{document}