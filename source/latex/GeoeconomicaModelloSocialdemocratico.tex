\documentclass[preview]{standalone}

\usepackage{amsmath}
\usepackage{amssymb}
\usepackage{stellar}
\usepackage{bettelini}

\hypersetup{
    colorlinks=true,
    linkcolor=black,
    urlcolor=blue,
    pdftitle={Stellar},
    pdfpagemode=FullScreen,
}

\begin{document}

\title{Geografia economica}
\id{geoeconomica-modello-socialdemocratico}
\genpage

\section{Modello socialdemocratico}

\begin{snippetdefinition}{welfare-state-definition}{Welfare State}
    Con \textit{Welfare State} si intende un insieme di politiche sociali
    che proteggono i cittadini dai rischi e li assistono nei bisogni circa le condizioni di vita e sociali.
\end{snippetdefinition}

\begin{snippetdefinition}{modello-socialdemocratico-definition}{Modello socialdemocratico}
    Il \textit{modello socialdemocratico} è un sistema politico ed economico che combina elementi del capitalismo con un forte intervento dello Stato per garantire il benessere sociale.
\end{snippetdefinition}

\begin{snippet}{welfare-state-expl}
    Il modello socialdemocratico è quindi un modello di Welfare State.
    Le sue principali caratteristiche sono:
    \begin{itemize}
        \item coerente politica estera a sostegno delle istituzioni europeiste e internazionali (come l'ONU);
        \item coinvolgimento dei sindacati e delle organizzazioni dei lavoratori nelle decisioni economiche e sociali, sostenendo la negoziazione collettiva per migliorare le condizioni lavorative;
        \item diritti civili, la libertà individuale e la partecipazione democratica, garantendo un sistema politico aperto e inclusivo.
        \item economia di mercato con un settore privato attivo, ma con un ruolo significativo dello Stato nell'economia.
            Equità fra sistema statale e privato (statalismo).
    \end{itemize}
\end{snippet}

\end{document}