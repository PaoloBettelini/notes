\documentclass[preview]{standalone}

\usepackage{amsmath}
\usepackage{amssymb}
\usepackage{stellar}
\usepackage{definitions}

\begin{document}

\id{integers-diophantine-equations-exercises}
\genpage

\section{Exercises}

\begin{snippetexercise}{diophantine-equation-2-unknowns-ex-1}{Diophantine equation with 2 unknowns}
    Solve the following diophantine equation
    \[
        6x + 8y = 25
    \]
\end{snippetexercise}

\begin{snippetsolution}{diophantine-equation-2-unknowns-ex-1-sol}{Diophantine equation with 2 unknowns}
    We note that \(\gcd(6,8) = 2\) does not divide \(25\), so the equation is not solvable.
\end{snippetsolution}

\begin{snippetexercise}{diophantine-equation-2-unknowns-ex-2}{Diophantine equation with 2 unknowns}
    Solve the following diophantine equation
    \[
        6x + 8y = 34
    \]
\end{snippetexercise}

\begin{snippetsolution}{diophantine-equation-2-unknowns-ex-2-sol}{Diophantine equation with 2 unknowns}
    The equation is solvable as \(\gcd(6,8) = 2 \divides 36\).
    The equation is equivalent to
    \[
        3x + 4y = 17
    \]
    (with \(3\) and \(4\) \coprime). We find a Bezout's identity
    between \(3\) and \(4\):
    \[
        3(-1) + 4(1) = 1
    \]
    By multiplying by \(17\) we get a solution
    \[
        3(-17) + 4(17) = 17
    \]
    The general solution is
    \(x = -17 + 4h\) and \(y = 17 -3h\) for \(h\in \integers\).
\end{snippetsolution}

\begin{snippetexercise}{diophantine-equation-3-unknowns-ex-1}{Diophantine equation with 3 unknowns}
    Solve the following diophantine equation
    \[
        6x + 8y + 12z = 10
    \]
\end{snippetexercise}

\begin{snippetsolution}{diophantine-equation-3-unknowns-ex-1-sol}{Diophantine equation with 3 unknowns}
    We first divide the equation to make it simplier
    \[
        3x + 4y + 6z = 5
    \]
    We introduce an auxiliary variable \(w = 4y + 6z\).
    We now have
    \[
        \begin{cases}
            3x + w = 5 \\
            4y + 6z = w
        \end{cases}
    \]
    The second equation has solution \ifandonlyif \(\gcd(4,6) = 2\) divides \(w\).
    If we choose \(w\) as a multiple of \(2\), then there exist \(y\) and \(x\)
    that satisfy the equation.
    Let \(w = 2k\) with \(k\in\mathbb{Z}\) and let s rewrite the system
    \[
        \begin{cases}
            3x + 2k = 5 \\
            2y + 3z = k
        \end{cases}
    \]
    We now solve the two diophantine equations (the second one is parametric over \(k\)).
    All the coefficients are irreducible as they are coprime.
    The first has solutions \(x=1+2h\) and \(k = 1 - 3h\) with \(h\in\integers\).
    The second equation has solutions \(y = -k + 3t\) and \(z = k -2t\) with \(t \in \integers\).
    We now delete the auxiliary variable \(k\) (by eliminating \(w\)), and we find
    \(x=1+2h\), \(y=-1+3h+3t\) and \(z = 1- 3h - 2t\) for \(h,t \in \integers\).
\end{snippetsolution}

\end{document}