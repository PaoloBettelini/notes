\documentclass[preview]{standalone}

\usepackage{amsmath}
\usepackage{amssymb}
\usepackage{stellar}
\usepackage{definitions}

\begin{document}

\id{construction-of-the-reals}
\genpage

\section{Missing supremum in the rationals}

\plain{There are subsets in the rationals without a supremum.}

\begin{snippetproposition}{no-supremum-in-subset-of-rationals}{No supremum in subsets of rationals}
    Consider the \set
    \[
        E = \{ r \in \rationalnumbers \suchthat r \geq 0 \land r^2 \leq 2 \}
    \]
    Then,
    \begin{enumerate}
        \item \(E \neq \emptyset\);
        \item \(E\) is \upperbound[bounded above];
        \item \(\sup E\) does not exist.
    \end{enumerate}
\end{snippetproposition}

\begin{snippetproof}{no-supremum-in-subset-of-rationals-proof}{no-supremum-in-subset-of-rationals}{No supremum in subsets of rationals}
    \begin{enumerate}
        \item To show that \(E \neq \emptyset\) we can simply explicitly give an element of \(E\).
            For example, \(1 \in E\) because \(1 \in \rationalnumbers\) and \(0 \leq 1^2 = 1\leq 2\).
        \item The \set is trivially bounded by all values \(x\geq 2\).
        \item Assume that \(\exists \mu = \sup E\). Note that \(\mu > 0\).
            Also, \(\mu^2 = 2\) is impossible.
            Thus, either \(\mu^2>2\) or \(\mu^2<2\).
            Suppose \(\mu^2<2\), we want to show that \(\exists x\in E\) such that
            \(x>\mu\) meaning that \(\mu\) is not an \upperbound\,\lightning. Consider the rationals of the form
            \[
                \mu + \frac{1}{n}
            \]
            We want to find an \(n\in \naturalnumbers\) sufficiently big such that
            \[
                2-{\left(\mu + \frac{1}{n}\right)}^2 > 0
            \]
            We have
            \begin{align*}
                2 - {\left(\mu + \frac{1}{n}\right)}^2 &= (2 - \mu^2) + \frac{2\mu}{n} + \frac{1}{n^2} \\
                &> (2-\mu^2) - \frac{2\mu}{n} - \frac{1}{n}
            \end{align*}
            since \(1/n > 1/n^2\). We can now solve the inequality:
            \begin{align*}
                2-{\left(\mu + \frac{1}{n}\right)}^2 &> 0 \\
                (2-\mu^2) - \frac{2\mu + 1}{n} &> 0 \\
                \frac{2\mu + 1}{2 - \mu^2} &< n
            \end{align*}
            Likewise, we can prove the same if \(\mu^2>2\).
    \end{enumerate}
\end{snippetproof}

\section{Construction of the reals}

\includesnpt{totally-ordered-field-definition}

\begin{snippetaxiom}{completness-axiom}{Completness axiom}
    For every \(E\subseteq \mathbb{R}\) where \(E\neq \emptyset\) and where \(E\) is \upperbound[bounded above],
    \(\sup E\) exists.
\end{snippetaxiom}

\begin{snippetdefinition}{real-numbers-definition}{Real numbers}
    The \textit{real numbers} are defined
    as a \field \((\mathbb{R}, +, \cdot)\) with a
    \snippetref[totally-ordered-field-definition|Total order on a field][total order] \(\leq\)
    and the \completnessaxiom.
\end{snippetdefinition}

\section{Existence and uniqueness of the reals}

\begin{snippettheorem}{uniqueness-reals-ordered-field-theorem}{Uniqueness of the reals as a complete ordered field}
    Let \((F_1, +_1, \circ_1)\) and \((F_2, +_2, \circ_2)\) be \field[fields] with
    \snippetref[totally-ordered-field-definition|Total order on a field][total orders] \(\leq_1\) and
    \(\leq_2\) on them and the \completnessaxiom.
    Then, there exists a unique \fieldisomorphism \(\varphi\colon F_1 \fromto F_2 \)
    such that
    \begin{enumerate}
        \item \(a \leq_1 b \iff \varphi(a) \leq_2 \varphi(b)\).
    \end{enumerate}
\end{snippettheorem}

\begin{snippetproof}{uniqueness-reals-ordered-field-theorem-proof}{uniqueness-reals-ordered-field-theorem}{Uniqueness of the reals as a complete ordered field (partial proof)}
    Given the properties of a \field, a \field can generate a \set of rationals \(\rationalnumbers\).
    Clearly, different \field[fields] generate different \set[sets] but with the same elements in some way.
    We can construct a map between \(\mathbb{Q}_1\) and \(\mathbb{Q}_2\) of two different \field[fields].
    We can easily define \(\phi_0\colon \mathbb{Q}_1 \subseteq F_1 \to \mathbb{Q}_2 \subseteq F_2\).
    We can achieve our mapping using the \completnessaxiom.
    Given \(x\in F_1\), we have \(x = \sup \{ r \in \mathbb{Q}_1 \suchthat r \leq x \} = \sup E_x\).
    We thus define \(\phi (x) = \sup \{ \phi_0(r) \suchthat r \in E_x \}\) to extend
    \(\phi_0\) to every element. We still need to prove that the classical properties are preserved.
\end{snippetproof}

\begin{snippettheorem}{real-numbers-existence-theorem}{Existence of the real numbers}
    The \field \((\realnumbers, +, \cdot)\) exists.
\end{snippettheorem}

\begin{snippetproof}{real-numbers-existence-theorem-proof}{real-numbers-existence-theorem}{Existence of the real numbers}
    We will prove a simple construction of such \field using decimal expansions (or any other base).
    We will construct an element of the reals as a decimal expansion \(x=n.a_1a_2\cdots a_k\cdots\)
    with \(n,a_i\in\integers\). Consider another element \(y=m.b_1b_2\cdots b_k\cdots\).
    Note that two distinct decimal expansions can represent the same element (E.g. \(0.\overline{9}=1\)).
    Thus, we let \(n,a_1, a_2, a_3, \cdots, a_k, \overline{9} = n, a_1, \cdots, a_{k-1}, (a_k+1)\).
    We will assume that \(x,y\) do not admit such a pattern.
    \begin{itemize}
        \item \emph{equality:} \[ x=y \iff m=n\land \forall i, a_i = b_i \]
        \item \emph{inequality:} \[ x<y \iff m<n \lor (m = n \land a_i = b_i \land a_k < b_k), \quad 1\leq i < k \]
        \item \emph{addition:} defined by aligning their decimal expansions and adding corresponding digits, adjusting for carries as necessary;
        \item \emph{multiplication:} similar;
        \item \emph{\completnessaxiom:} let \(E\subseteq \mathbb{R}\) where \(E\neq \emptyset\) and where \(E\) is \upperbound[bounded above].
            We want to show that \(\mu = \sup E\) exists. Clearly,
            \[
                \forall x \in E, \exists x = n_x.\alpha_1(x)\alpha_2(x)\cdots
            \]
            Likewise,
            \[
                \mu = n.\alpha_1\alpha_2\cdots\alpha_k\cdots
            \]
            Let \[ N = \max\{n_x \suchthat x \in E\} \]
            which exists since \(E\neq\emptyset\) and it is \upperbound[bounded above].
            We now consider the \set
            \[
                E_0 = \{x \in E \suchthat n_x = N\} \neq \emptyset
            \]
            Let \(a_1 = \max\{\alpha_1(x) \suchthat x\in E_0\}\). We then have
            \[
                E_1 = \{x \in E_0 \suchthat \alpha_1(x) = a_1\} \neq \emptyset
            \]
            We now let \(a_2 = \max\{\alpha_2(x) \suchthat x\in E_1\}\). By iterating, we find \(a_3, a_4, \cdots\) meaning
            \[
                a_k = \max\{\alpha_k(x) \suchthat x \in E_{k-1}\}
                \qquad
                a_{k+1} = \max\{\alpha_{k+1}(x) \suchthat x \in E_k\}
            \]
            We now need to show that \(\mu\) is
            an \upperbound. Let \(\overline{x} \in E = n_{\overline{x}}\alpha_1(\overline{x})\alpha_2(\overline{x})\cdots\).
            By the definition of \(\mu\), every digit of \(x\) is less than or equal to every corresponding digit of \(\mu\).
            By definition, this means that \(x\leq \mu\). In particular, if \(\mu \in E\), then \(\mu = \max E\).
            We now need to show that \(\mu\) is the smallest of the \upperbound[upper bounds].
            In order to some \(z\) to be less than \(\mu\) for some \(z\), there must be a digit in \(z\)
            that is less than the corresponding digit in \(\mu\). By the definition of \(\mu\), this is not possible.
    \end{itemize}
\end{snippetproof}

\plain{The reals can also be defined using Cauchy sequences or Dedekind cuts}

\subsection{Consequences of the completness axiom}

\begin{snippetproposition}{archimedean-property}{Archimedean property}
    \[
        \forall x \in \realnumbers, \forall a>0, \exists n \in \naturalnumbers
        \suchthat na > x
    \]
\end{snippetproposition}

\plain{Between every two reals, there exist (infinite) rationals. This means that the rationals are dense in the reals.}

\begin{snippettheorem}{real-nth-roots-existence-theorem}{Existence of real nth roots}
    \[
        \forall y>0, \forall n\geq 1 \in \naturalnumbers, \exists_{=1} x > 0 \suchthat x^n = y
    \]
\end{snippettheorem}

\begin{snippetproof}{real-nth-roots-existence-theorem-proof}{real-nth-roots-existence-theorem}{Existence of real nth roots}
    Let \[ E = \{ z \in \realnumbers \suchthat z>0 \land z^n \leq y \} \]
    We need to show that \(E\) is not empty and is bounded above.
    Define \(x = \sup E\) and show that \(x^n = y\).
    \begin{itemize}
        \item \textbf{Not empty:} if \(y \geq 1\), we can choose \(x=1\) since \(x^n=1 \leq y\).
        Otherwise, if \(y < 1\), set \(x=y\) and note that, because \(y < 1\), it follows that
        \(y^n < y\), and hence \(y \in E\).
        \item \textbf{Bounded above:} \(E\) is bounded above, as \(1+y\)
        is an \upperbound of \(E\). If \(z \geq (1+y)\), since the function \(t\to t^n\) is increasing for \(t > 0\),
        we have \(z^n \geq {(1+y)}^n > {(1+y)} > y \implies z \notin E\).
        Let \(x =\sup E\). I claim that \(x^n = y\). I prove this by contradiction: if we suppose \(x^n > y\),
        then for sufficiently large \(k\),
        \[
            {\left(x- \frac{1}{k}\right)}^n > y
        \]
        and therefore \(x-\frac{1}{k}\) is still an \upperbound of \(E\), contradicting the assumption that \(x\), which is the \(\sup E\), is the smallest upper bound.
        Conversely, if \(x^n < y\), then for sufficiently large \(k\),
        \[
            {\left(x + \frac{1}{k}\right)}^n < y
        \]
        so \(x+\frac{1}{k}\in E\) and is greater than \(x\), contradicting that \(x\) is an \upperbound\,\lightning.
        Since \(x\) can neither be greater nor smaller, \(x^n=y\).
        \item \textbf{Uniqueness:} Note that if \(0 < t_1 < t_2 \implies t_1^n < t_2^n\).
    \end{itemize}
    We can also set \(z \geq 0\) to make proving that \(E \neq \emptyset\) easier.
\end{snippetproof}

\section{Extended real line}

\begin{snippetdefinition}{extended-real-line-definition}{Extended real number line}
    The \textit{extended reals} are defined as
    \[
        \overline{\realnumbers} = \realnumbers \union \{+\infty, -\infty\}
    \] 
\end{snippetdefinition}

\begin{snippetproposition}{infimum-supremum-empty-set-extended-reals}{Infimum and supremum of empty set}
    Consider \((\extendedrealnumbers, \leq)\) where \(\leq\) is the classical ordering.
    Then, \(\inf\emptyset = +\infty\) and \(\sup\emptyset = -\infty\).
\end{snippetproposition}

\end{document}