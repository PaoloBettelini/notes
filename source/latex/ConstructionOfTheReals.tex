\documentclass[preview]{standalone}

\usepackage{amsmath}
\usepackage{amssymb}
\usepackage{stellar}
\usepackage{definitions}

\begin{document}

\id{construction-of-the-reals}
\genpage

\section{Missing supremum in the rationals}

\plain{There are subsets in the rationals without a supremum.}

\begin{snippetproposition}{no-supremum-in-subset-of-rationals}{No supremum in subsets of rationals}
    Consider the set
    \[
        E = \{ r \in \rationalnumbers \suchthat r \geq 0 \land r^2 \leq 2 \}
    \]
    Then,
    \begin{enumerate}
        \item \(E \neq \emptyset\);
        \item \(E\) è superiormente limitato;
        \item \(\sup E\) does not exist.
    \end{enumerate}
\end{snippetproposition}

\begin{snippetproof}{no-supremum-in-subset-of-rationals-proof}{no-supremum-in-subset-of-rationals}{No supremum in subsets of ratinoals}
    \begin{enumerate}
        \item To show that \(E \neq \emptyset\) we can simply explicitly give an element of \(E\).
            For example, \(1 \in E\) because \(1 \in \rationalnumbers\) and \(0 \leq 1^2 = 1\leq 2\).
        \item TODO
        \item TODO
    \end{enumerate}
\end{snippetproof}

\section{Uniqueness of the reals}

\begin{snippettheorem}{uniqueness-reals-ordered-field-theorem}{Uniqueness of the reals as a complete ordered field}
    TODO
    %\begin{enumerate}
    %    \item
    %    \item
    %    \item
    %\end{enumerate}
\end{snippettheorem}

\end{document}