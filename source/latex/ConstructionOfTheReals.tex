\documentclass[preview]{standalone}

\usepackage{amsmath}
\usepackage{amssymb}
\usepackage{stellar}
\usepackage{definitions}

\begin{document}

\id{construction-of-the-reals}
\genpage

\section{Missing supremum in the rationals}

\plain{There are subsets in the rationals without a supremum.}

\begin{snippetproposition}{no-supremum-in-subset-of-rationals}{No supremum in subsets of rationals}
    Consider the set
    \[
        E = \{ r \in \rationalnumbers \suchthat r \geq 0 \land r^2 \leq 2 \}
    \]
    Then,
    \begin{enumerate}
        \item \(E \neq \emptyset\);
        \item \(E\) is \upperbound[bounded above];
        \item \(\sup E\) does not exist.
    \end{enumerate}
\end{snippetproposition}

\begin{snippetproof}{no-supremum-in-subset-of-rationals-proof}{no-supremum-in-subset-of-rationals}{No supremum in subsets of rationals}
    \begin{enumerate}
        \item To show that \(E \neq \emptyset\) we can simply explicitly give an element of \(E\).
            For example, \(1 \in E\) because \(1 \in \rationalnumbers\) and \(0 \leq 1^2 = 1\leq 2\).
        \item The set is trivially bounded by all values \(x\geq 2\).
        \item \todo
    \end{enumerate}
\end{snippetproof}

\section{Construction of the reals}

\begin{snippetdefinition}{totally-ordered-field-definition}{Totally ordered field}
    A \field \((F, +, \cdot)\) together with a \totalorder \(\leq\) on \(R\) is a
    \textit{totally ordered field} if it satisfies the following properties:
    \begin{enumerate}
        \item \(\forall a,b,c \in F, a \leq b \implies a+c \leq b+c\);
        \item \(\forall a,b,c \in F, 0 \leq a \land 0 \leq b \implies 0 \leq a \cdot b\).
    \end{enumerate}
\end{snippetdefinition}

\begin{snippetaxiom}{completness-axiom}{Completness axiom}
    For every \(E\subseteq \mathbb{R}\) where \(E\neq \emptyset\) and where \(E\) is \upperbound[bounded above],
    \(\sup E\) exists.
\end{snippetaxiom}

\begin{snippetdefinition}{real-numbers-definition}{Real numbers}
    The \textit{real numbers} and their properties can be defined
    as a \field \((\mathbb{R}, +, \cdot)\) with a
    \snippetref[totally-ordered-field-definition|Total order on a field][total order] \(\leq\)
    and the \snippetref[completness-axiom|Completness axiom][completness axiom].
\end{snippetdefinition}

\section{Uniqueness of the reals}

\begin{snippettheorem}{uniqueness-reals-ordered-field-theorem}{Uniqueness of the reals as a complete ordered field}
    Let \((F_1, +_1, \circ_1)\) and \((F_2, +_2, \circ_2)\) be \field[fields] with
    \snippetref[totally-ordered-field-definition|Total order on a field][total orders] \(\leq_1\) and
    \(\leq_2\) on them and the \snippetref[completness-axiom|Completness axiom][completness axiom].
    Then, there exists a unique isomorfism \(\varphi\colon F_1 \fromto F_2 \)
    such that
    \begin{enumerate}
        \item \(\varphi(a+_1b) = \varphi(a)+_2\varphi(b)\);
        \item \(\varphi(a\circ_1b) = \varphi(a)\circ_2\varphi(b)\);
        \item \(a \leq_1 b \iff \varphi(a) \leq_2 \varphi(b)\).
    \end{enumerate}
\end{snippettheorem}

%\section{Continuity axiom}
%
%\begin{snippetaxiom}{continuity-axiom}{Continuity axiom}
%    Let \(\{I_n\}\) be a sequence of closed intervals of the form
%    \(I_n = [a_n; b_n]\) such that
%    \begin{enumerate}
%        \item \(I_{n+1} \subseteq I_n\);
%        \item \(l(I_{n+1}) = b_{n+1}-a_{n+1} = \frac{1}{2}l(I_n)\).
%    \end{enumerate}
%    e quindi \[l(I_n) = \frac{1}{2^{n-1}}l(I_i)\] allora esiste \(c\in \mathbb{R}\)
%    tale che
%    \[
%        \bigcup_{n\in\mathbb{N}} I_n = \{c\}
%    \]
%\end{snippetaxiom}

\section{Extended real line}

\begin{snippetdefinition}{extended-real-line-definition}{Extended real number line}
    The \textit{extended reals} are defined as
    \[
        \overline{\realnumbers} = \realnumbers \union \{+\infty, -\infty\}
    \] 
\end{snippetdefinition}

\end{document}