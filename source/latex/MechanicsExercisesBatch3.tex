\documentclass[preview]{standalone}

\usepackage{amsmath}
\usepackage{amssymb}
\usepackage{stellar}
\usepackage{definitions}
\usepackage{bettelini}

\begin{document}

\id{mechanics-ex-3}
\genpage

\section{Exercises - Batch 3}

\begin{snippetexercise}{mechanics-ex-3.1}{\underline{3.1}}
    A man is in an elevator moving upwards at constant velocity $V_0$. He throws a ball vertically upwards with a velocity $v_0$ relative to the elevator.
    \begin{itemize}
        \item Determine how much time it takes for the ball to return to the man's hand.
        \item Answer the previous question in the case where the elevator has an upward acceleration $A_{\text{asc}}$.
    \end{itemize}
    [Hint: Try solving the problem in two ways: (1) using the laws of relative motion, and (2) using the laws of motion for the two bodies viewed from the reference frame of the stationary ground. Verify that the results are the same.]
\end{snippetexercise}

\begin{snippetsolution}{mechanics-ex-3.1-sol}{\underline{3.1}}
    \todo
\end{snippetsolution}

\begin{snippetexercise}{mechanics-ex-3.2}{\underline{3.2} Rotating platform}
    A platform rotates with angular velocity
    \(\omega\) around a vertical central axis.
    At the instant \(t=0\), a ball is launched horizontally with velocity \(v_0\)
    from the center of the platform; the friction the ball encounters is negligible,
    so that it moves with respect to the earth in uniform rectilinear motion with velocity \(v_0\).
    Determine the acceleration of the ball, at a generic instant, with respect to a reference system
    integral with the platform.
\end{snippetexercise}

\begin{snippetsolution}{mechanics-ex-3.2-sol}{\underline{3.2} Rotating platform}
    The position is given by
    \[
        \vec{R}(t) = \begin{pmatrix}
            v_0t\cos(\omega t) \\
            v_0t\sin(\omega t)
        \end{pmatrix}
    \]
    The velocity is given by
    \[
        \vec{v}(t) = v_0 \begin{pmatrix}
            -\omega t \sin(\omega t) + \cos(\omega t) \\
            \omega t \cos(\omega t) + \sin(\omega t)
        \end{pmatrix}
    \]
    The acceleration is given by
    \[
        \vec{a}(t) = v_0 \omega \begin{pmatrix}
            -\omega t \cos(\omega t) - 2\sin(\omega t) \\
            -\omega t \sin(\omega t) + 2\cos(\omega t)
        \end{pmatrix}
    \]
    and its module
    \begin{align*}
        |\vec{a}(t)| &= v_0 \omega \sqrt{
            {(-\omega t \cos(\omega t) - 2\sin(\omega t))}^2 + {(-\omega t \sin(\omega t) + 2\cos(\omega t))}^2
        } \\
        &= v_0 \omega \sqrt{4 + \omega^2t^2}
    \end{align*}
\end{snippetsolution}

\begin{snippetexercise}{mechanics-ex-3.3}{\underline{3.3}}
    Let $g_0$ be the gravitational acceleration measured at a point $P$ on the Earth's surface if the Earth were not rotating. Determine the effective gravitational acceleration measured by an observer fixed to the Earth. Also calculate the deviation experienced by a body in free fall due to the Coriolis acceleration at the equator.
\end{snippetexercise}

\begin{snippetsolution}{mechanics-ex-3.3-sol}{\underline{3.3}}
    \todo
\end{snippetsolution}

\begin{snippetexercise}{mechanics-ex-3.4}{\underline{3.4}}
    A force that varies with time is acting on a body of mass $m$, given by:
    \[
        F = F_0 - \alpha t,
    \]
    where $F_0$ and $\alpha$ are constants. At the initial instant, the body passes through the origin with velocity $v_0$. Find the velocity and position of the body as functions of time.
\end{snippetexercise}

\begin{snippetsolution}{mechanics-ex-3.4-sol}{\underline{3.4}}
    \todo
\end{snippetsolution}

\begin{snippetexercise}{mechanics-ex-3.5}{\underline{3.5}}
    A particle moves under the action of a force $\vec{F} = \vec{v} \times \vec{c}$, where $\vec{c}$ is a constant vector. Find the trajectory and the law of motion.
\end{snippetexercise}

\begin{snippetsolution}{mechanics-ex-3.5-sol}{\underline{3.5}}
    \todo
\end{snippetsolution}

\begin{snippetexercise}{mechanics-ex-3.6}{\underline{3.6}}
    Two tugboats tow a boat using steel cables attached to the bow of the boat. The angle between the cables and the horizontal is $60^\circ$, and the tension in each cable is $2 \times 10^5$ N. Find the resistive force due to water such that the boat moves at constant speed.
\end{snippetexercise}

\begin{snippetsolution}{mechanics-ex-3.6-sol}{\underline{3.6}}
    \todo
\end{snippetsolution}

\end{document}