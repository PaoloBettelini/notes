\documentclass[preview]{standalone}

\usepackage{amsmath}
\usepackage{amssymb}
\usepackage{stellar}
\usepackage{definitions}

\begin{document}

\id{permutations-and-k-cycles-exercises}
\genpage

\section{Exercises}

\begin{snippetexercise}{permutation-ex1}{}
    Find a permutation \(\sigma\)
    such that
    \[
        \sigma^4 = \kcycle{1,2,3,4,5,6,7}
    \]
    in \(\permgrp_7\) and \(\permgrp_9\).
\end{snippetexercise}

\begin{snippetsolution}{permutation-ex1-sol}{}
    Assume that \(\sigma=\gamma_1\gamma_2\cdots \gamma_r\)
    where \(\gamma\) are \disjointperm permutations.
    Now, \(\sigma^4 = \gamma_1^4\gamma_2^4\cdots \gamma_r^4\)
    since \(\gamma\) commute.
    Clearly, they are \disjointperm but might not by cycles.
    Furthermore, the cycles that appear in every \(\gamma_i^4\)
    are not longer than those which appear in \(\gamma_i\).
    Note that \(\sigma\) cannot be a product of cycles of length less than \(7\).
    Also, \(\sigma\) should have a period of \(7\), just like \(\sigma^4\).
    The period of \(\sigma^4\) is \(\frac{7}{\gcd(4,7)} = 7\). We know that \(\sigma^7 = \text{Id}\).
    Now, \(4\) has \(2\) as inverse modulo \(7\).
    \[
        \kcycle{1,2,3,4,5,6,7}^2 = {\sigma^4}^2 = \sigma^8 = \sigma
    \]
    Thus,
    \[
        \sigma = \kcycle{1,2,3,4,5,6,7}^2 = \kcycle{1,3,5,7,2,4,6}
    \]
    % provare a completare l'esercizio per sym9
    % problema: come sono fatte le potenza di un ciclo?
    % considerare prima il caso sigma = (1,2,3,4,5,6)
    % chiaramente per un primo le potenza hanno tutte periodo 7/gcd, quindi o 7 o 1
\end{snippetsolution}

\end{document}