\documentclass[preview]{standalone}

\usepackage{amsmath}
\usepackage{amssymb}
\usepackage{stellar}
\usepackage{definitions}

\begin{document}

\id{permutations-and-k-cycles-exercises}
\genpage

\section{Exercises}

\begin{snippetexercise}{permutation-ex1}{}
    Find a permutation \(\sigma\)
    such that
    \[
        \sigma^4 = \kcycle{1,2,3,4,5,6,7}
    \]
    in \(\permgrp_7\) and \(\permgrp_9\).
\end{snippetexercise}

\begin{snippetsolution}{permutation-ex1-sol}{}
    Assume that \(\sigma=\gamma_1\gamma_2\cdots \gamma_r\)
    where \(\gamma\) are \disjointperm permutations.
    Now, \(\sigma^4 = \gamma_1^4\gamma_2^4\cdots \gamma_r^4\)
    since \(\gamma\) commute.
    Clearly, they are \disjointperm but might not by cycles.
    Furthermore, the cycles that appear in every \(\gamma_i^4\)
    are not longer than those which appear in \(\gamma_i\).
    Note that \(\sigma\) cannot be a product of cycles of length less than \(7\).
    Also, \(\sigma\) should have a period of \(7\), just like \(\sigma^4\).
    The period of \(\sigma^4\) is \(\frac{7}{\gcd(4,7)} = 7\). We know that \(\sigma^7 = \identityfunc\).
    Now, \(4\) has \(2\) as inverse modulo \(7\).
    \[
        \kcycle{1,2,3,4,5,6,7}^2 = {\sigma^4}^2 = \sigma^8 = \sigma
    \]
    Thus,
    \[
        \sigma = \kcycle{1,2,3,4,5,6,7}^2 = \kcycle{1,3,5,7,2,4,6}
    \]
    In the case of \(\permgrp_9\), there could be a \disjointperm cycles
    with the one we found in \(\permgrp_7\). Indeed,
    \[
        \sigma = \kcycle{1,3,5,7,2,4,6}\kcycle{8,9}^4 = \kcycle{1,3,5,7,2,4,6}
    \]
\end{snippetsolution}

\begin{snippetexercise}{permutation-ex2}{}
    Find a permutation \(\tau\) such that
    \[
        \tau^4 = \kcycle{1,2,3,4,5,6}\kcycle{6,7}
    \]
    in \(\permgrp_7\).
\end{snippetexercise}

\begin{snippetsolution}{permutation-ex2-sol}{}
    Since the permutation is a product of cycles of different lengths, it must originate
    from a product of cycles with different lengths.
    Note that there must be at least a cycle of a length that is multiple of \(2\)
    and one is multiple of \(5\).
    The only possibility is that there is exactly a cycle of length \(5\)
    and one of length \(2\). We have
    \[
        (\gamma_1 \gamma_2)^4= \gamma_1^4 \gamma_2^4 = \gamma_1^4
    \]
    whch is only a \(5\)-cycle.
    Thus, such \(\tau\) does not exist.
\end{snippetsolution}

\end{document}