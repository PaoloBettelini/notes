\documentclass[preview]{standalone}

\usepackage{amsmath}
\usepackage{amssymb}
\usepackage{bettelini}
\usepackage{stellar}

\hypersetup{
    colorlinks=true,
    linkcolor=black,
    urlcolor=blue,
    pdftitle={Chimica},
    pdfpagemode=FullScreen,
}

\begin{document}

\title{Chimica}
\id{chimica-soluzioni}
\genpage

\plain{Ogni soluzione è caratterizzata da un <i>soluto</i> ed un <i>solvente</i>.}

\begin{snippetdefinition}{solubilita-definizioe}{Solubilità}
    La \textit{solubilità} è la quantità massima che una sostanza
    può essere sciolta da una determinata quantità di solvente.
\end{snippetdefinition}

\plain{La solubilità dipende dalle proprietà chimica e altri fattori come la temperatura. \\
La solubilità dei gas diminuisce con l'aumento della temperatura.}

\begin{snippetdefinition}{saturazione-definizioe}{Saturazione}
    Una soluzione è detta \textit{satura} o \textit{insatura}
    se ha raggiunto il suo quantitativo massimo o meno.
\end{snippetdefinition}

\plain{Quando un soluto viene sciolto in un solvente, il volume della soluzione aumenta,
ma meno della somma dei due volumi. Questo è dato dal fatto che il soluto prende spazio fra le molecole del solvente.}

\end{document}
