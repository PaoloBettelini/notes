\documentclass[preview]{standalone}

\usepackage{amsmath}
\usepackage{amssymb}
\usepackage{stellar}
\usepackage{definitions}
\usepackage{bettelini}

\begin{document}

\id{groups-direct-product}
\genpage

\section{External direct product}

\plain{The goal of the following definition is to construct a single group from various groups.}

\begin{snippetdefinition}{group-external-direct-product-definition}{External direct product of groups}
    Let \((G_1, \circ_1), (G_2, \circ_2), \cdots, (G_n, \circ_n)\)
    be \group[groups].
    Then, their \emph{external direct product} is defined as the \group
    \((G, \circ)\) where
    \[
        G = G_1 \cartesianprod G_2 \cartesianprod \cdots \cartesianprod G_n
    \]
    and
    \[
        (x_1, x_2, \cdots, x_n) \circ 
        (y_1, y_2, \cdots, y_n) \triangleq
        (x_1 \circ_1 y_1, x_2 \circ_2 y_2, \cdots, x_n \circ_n y_n)
    \]
\end{snippetdefinition}

\begin{snippettheorem}{external-direct-product-is-group-theorem}{External direct product is group}
    The external direct product of the \group[groups] \(G_1, G_2, \cdots, G_n\),
    is a \group \(G\). Furthermore, for \(1 \leq i \in \naturalnumbers \leq n\), the subset
    \[
        H_i = \left\{
            \left(
                1_{G_1}, 1_{G_2}, \cdots,
                x_i, 1_{G_{i+1}}, \cdots, 1_{G_n}
            \right) \suchthat x_i \in G_i
        \right\}
    \]
    is a \normalsubgrptext[normal subgroup] of \(G\) where \(H_i \groupisomorphic G_i\) and
    \begin{enumerate}
        \item \[G = \prod_{i=1}^n H_i \]
        \item \[
            H_i \intersection \prod_{\substack{k=1\\k \neq i}}^n H_k = 1, \quad 1 \leq i \in \naturalnumbers \leq n
        \]
    \end{enumerate}
\end{snippettheorem}

\begin{snippetproof}{external-direct-product-is-group-theorem-proof}{external-direct-product-is-group-theorem}{External direct product is group}
    We first show that \(G\) is a \group:
    \begin{enumerate}
        \item \emph{associativity}: trivial;
        \item \emph{identity element}: \((1_{G_1}, 1_{G_2}, \cdots, 1_{G_n})\);
        \item \emph{inverse element}: tuple of the inverses.
    \end{enumerate}
    Define \(\varphi_i \colon G_1 \to G\) as
    \[
        x\varphi_i \triangleq
        (1_{G_1}, 1_{G_2}, \cdots, x, 1_{G_{i+1}}, \cdots, 1_{G_n})
    \]
    The image is \(H_i\) and it is easy to check that \(\varphi\)
    is a \groupisomorphism between \(G_i\) and \(H_i\).
    We now show that \(H_1 \unlhd G\). Let
    \[
        (1_{G_1}, 1_{G_2}, \cdots, x, 1_{G_{i+1}}, \cdots, 1_{G_n}) \in H_i
    \]
    and \((g_1, g_2, \cdots, g_n) \in G\).
    We have
    \begin{align*}
        &\phantom{=}
        {(g_1, g_2, \cdots, g_n)}^{-1} (1_{G_1}, 1_{G_2}, \cdots, x, 1_{G_{i+1}}, \cdots, 1_{G_n})
        (g_1, g_2, \cdots, g_n) \\
        &= (1, 1, \cdots, g_i^{-1} x g_i, 1, \cdots, 1) \in H_i
    \end{align*}
    We will now show that \(G = H_1 H_2 \cdots H_n\):
    if \(g = (g_1, g_2, \cdots, g_n) \in G\) then
    \begin{align*}
        g = (g_1, 1, 1, \cdots, 1) (1, g_2, 1, \cdots, 1) \cdots (1,1,\cdots, g_n)
    \end{align*}
    Note that this product is also unique.
    We will now show that \(H_i \intersection H_1H_2\cdots H_{i-1} H_{i+1} \cdots H_n = 1\).
    Without loss of generality let \(i=1\).
    We have \(g = H_1 \intersection H_2 H_3 \cdots H_n\) which means that both
    \[
        g = (g_1, 1, 1, \cdots, 1) \in H_1
    \]
    and
    \[
        g = (1, g_2, 1, \cdots, 1)(1,1,g_3, \cdots, 1) \cdots (1,1,\cdots 1, g_n) \in H_2H_3\cdots H_n
    \]
    need to be true. Thus, \(g_1 = 1, g_2 = 1, \cdots, g_n = 1\), meaning \(g = (1,1,\cdots, 1) = 1_G\).
\end{snippetproof}

\begin{snippetproposition}{external-direct-product-abelian-iff-groups-are}{}
    The external direct product of the \group[groups]
    \(G_1, G_2, \cdots, G_n\)
    is an \abeliangroup \ifandonlyif \(G_i\) is \abeliangroup[abelian].
\end{snippetproposition}

\begin{snippetproof}{external-direct-product-abelian-iff-groups-are-proof}{external-direct-product-abelian-iff-groups-are}{}
    Consider two elements \[(g_1, g_2, \cdots, g_n), (h_1, h_2, \cdots, h_n) \in G_1 \cartesianprod G_2 \cartesianprod \cdots \cartesianprod G_n \]
    The product is \abeliangroup[abelian] \ifandonlyif
    \begin{align*}
        (g_1, g_2, \cdots, g_n)(h_1, h_2, \cdots, h_n)
        &=
        (h_1, h_2, \cdots, h_n)(g_1, g_2, \cdots, g_n) \\
        (g_1h_1, g_2h_2, \cdots, g_nh_n)
        &=
        (h_1g_1, h_2g_2, \cdots, h_2g_n)
    \end{align*}
    which only happens \ifandonlyif \(h_ig_i = g_ih_i\) and thus
    \ifandonlyif \(G_i\) is \abeliangroup[abelian].
\end{snippetproof}

\begin{snippetproposition}{external-direct-product-permutation-isomorphism}{}
    Let \(G_1, G_2, \cdots, G_n\) be \group[groups]
    and let \(\sigma \in \permgrp_n\). Then,
    \[
        G_1 \cartesianprod
        G_2 \cartesianprod
        \cdots \cartesianprod
        G_n
        \groupisomorphic
        G_{\sigma(1)} \cartesianprod
        G_{\sigma(2)} \cartesianprod
        \cdots \cartesianprod
        G_{\sigma(n)}
    \]
\end{snippetproposition}

\subsection{Periods and ciclicity}

\begin{snippetproposition}{external-direct-product-element-period}{}
    Let \(G_1, G_2, \cdots, G_n\) be \group[groups]
    and \(G = G_1 \cartesianprod G_2 \cartesianprod \cdots \cartesianprod G_n\).
    Consider an element \((g_1, g_2, \cdots, g_n) \in G\).
    Then,
    \[
        \elementperiod{(g_1, g_2, \cdots, g_n)}
        = \lcm(
            \elementperiod{g_1},
            \elementperiod{g_2},
            \cdots,
            \elementperiod{g_n}
        )
    \]
    If at least one of \(g_1, g_2, \cdots, g_n\) has infinite \elementperiodtext,
    then \(\elementperiod{(g_1, g_2, \cdots, g_n)} = \infty\).
\end{snippetproposition}

\begin{snippetproof}{external-direct-product-element-period-proof}{external-direct-product-element-period}{}
    Even if only one of them has infinite \elementperiodtext
    (meaning \(x_i^r = 1\) \ifandonlyif \(r=0\)), we have that
    \[
        {(x_1, x_2, \cdots)}^r = (x_1^r, x_2^r, \cdots) = (1,1,\cdots)
    \]
    \ifandonlyif \(r=0\), meaning that is also has infinite \elementperiodtext.
    If instead they have finite \elementperiodtext (\(d_1\cdots d_n\) respectively)
    we have that the period is \(\lcm(d_1, d_2, \cdots)\).
\end{snippetproof}

\begin{snippettheorem}{external-direct-product-cyclicity-theorem}{}
    Let \(G_1, G_2, \cdots, G_n\) be \group[groups]
    and \(G = G_1 \cartesianprod G_2 \cartesianprod \cdots \cartesianprod G_n\).
    Then,
    \begin{enumerate}
        \item if \(G\) is \cyclicgroup[cyclic], then \(G_i\) is \cyclicgroup[cyclic];
        \item if at least one of \(G_i\) is an infinite \cyclicgroup, then \(G\) is not cyclic;
        \item suppose \(n>1\) and \(G_i\) are non-trivial \cyclicgroup[cyclic groups]. Then,
            \(G\) is \cyclicgroup[cyclic] \ifandonlyif \(G_i\) are finite \cyclicgroup[cyclic groups]
            with \coprime orders among eachother.
    \end{enumerate}
\end{snippettheorem}

\begin{snippetproof}{external-direct-product-cyclicity-theorem-proof}{external-direct-product-cyclicity-theorem}{}
    \begin{enumerate}
        \item Sappiamo che ciascun \(G_i\) è isomorfo a un sottogruppo \(H_i\) di \(G\).
        Se almeno uno degli \(G_i\) non è ciclico, allora \(G\)
        non è ciclico.
        \item Supponiamo allora che siamo tutti ciclici.
        Se \(n=1\), chiaramente \(G=G_1\) e quindi è ciclico.
        Questo è un caso banale, come quello in cui alcuni dei \(G_i\)
        siano banali, cioè non danno contributo.
        Considerimo allora il caso non banale \(n>1\) e con \(G_i\) non banale.
        Sappiamo che \(G\) contiene dei sottogruppi
        \(H_1, H_2, \cdots, H_n\) isomorfi a \(G_1, G_2, \cdots, G_n\),
        dunque gli \(H_i\) sono non banali e inoltre
        \[
            H_1 \cap H_2H_3 \cdots H_r = 1
        \]
        cioè in \(G\) esistono due sottogruppi non banali
        \(H_1\) e \(H_2H_3 \cdots H_r\) con intersezione banale.
        Ma questo in un sottogruppo ciclico infinito \(C = \langle g \rangle\) non può succedere.
        Infatti, i sottogruppi non banali di \(C\) sono del tipo \({\langle g^t \rangle}\)
        con \(t\neq 0\) e \(\langle g^t \rangle \cap \langle g^s \rangle\)
        con \(t, s \neq 0\) è non banale.
        Infatti, l'intersezione è \(\langle g^u\rangle\) con \(u=\lcm(t, u)\).
        Dunque, \(G\) non è ciclico.
        \item Nel caso in cui sono finiti con ordini \(d_1, d_2, \cdots, d_n\).
        Ora \[
            |G| = \prod_i |G_i| = \prod_i d_i
        \]
        Un elemento di \(G\) ha periodo \(\prod_i d_i\)
        se \(d_i\) sono coprimi.
        Se non sono coprimi, vi è un divisore in comune dalla tabella dei ciclici della fattorizzazione,
        e quindi ho più sottogruppi dello stesso ordine, e quindi il gruppo non è ciclico. \\
        \iffproof{
            Supponiamo che almeno due tra i \(d_i\) non siano coprimi, ad esempio
            \(d_1\) e \(d_2\) e sia \(d = \text{mcd}(d_1, d_2) \neq 1\).
            Ora, \(G_1\) e \(G_2\) contengono
            sottogruppi di ordine \(d\) e quindi lo stesso è vero per
            \(H_1 \cong G_1\) e \(H_2 \cong G_2\).
            Ma allora, questi due sottogruppi di \(G\)
            di ordine \(d\) sono diversi perché la loro intersezione è contenuta
            in \(H_1 \cap H_2 = 1\). Un gruppo ciclico finito non contiene
            due sottogruppi distinti dello stesso ordine. Quindi \(G\)
            non è ciclico.
        }{
            Se \(d_1, d_2, \cdots, d_n\) sono a due a due coprimi
            e \(g_1, g_2, \cdots, g_n\) sono elementi di \(G_1, G_2, \cdots, G_n\)
            di periodo \(d_1, d_2, \cdots, d_n\), allora
            \[
                |(g_1, g_2, \cdots, g_n)| = \lcm(d_1, d_2, \cdots, d_n)
            \]
            cioè \((g_1, g_2, \cdots, g_n)\) genera un sottogruppo di ordine
            \(d_1d_2\cdots d_n\) che coincide quindi con \(G\).
        }
    \end{enumerate}
\end{snippetproof}

\section{Internal direct product}

\plain{The goal of the following definition is to decompose a single group into various groups.}

\begin{snippetdefinition}{group-internal-direct-product-definition}{Internal direct product of group}
    Let \(G\) be a \group where \(H_1, H_2, \cdots, H_n\)
    are \normalsubgrptext[normal subgroups] of \(G\).
    Then, \(G\) is the \emph{internal direct product} of
    the \(H_i\) \group[groups] if:
    \begin{enumerate}
        \item \[G = \prod_{i=1}^n H_i \]
        \item \[
            H_i \intersection \prod_{\substack{k=1\\k \neq i}}^n H_k = 1, \quad 1 \leq i \in \naturalnumbers \leq n
        \]
    \end{enumerate}
\end{snippetdefinition}

\plain{The internal direct product and the external direct product are fundamentally the same concept.}

\begin{snippettheorem}{internal-direct-external-product-isomorphism-theorem}{Isomorphism between internal and external product}
    Let \(G\) be an internal direct product of
    \(H_1, H_2, \cdots, H_n\). Then,
    \[
        G \groupisomorphic H_1 \times H_2 \times \cdots \times H_n
    \] % TODOURGENT: link times to external direct product
\end{snippettheorem}

\begin{snippetlemma}{normal-subgroups-commutativity-special-case}{Normal subgroup commutativity}
    Let \(H\) and \(K\) be \normalsubgrptext[normal subgroups] of \(G\)
    such that \(H \intersection K = 1\). Then, \(hk=kh\)
    for every \(h\in H\) and \(k\in K\).
\end{snippetlemma}

\begin{snippetproof}{normal-subgroups-commutativity-special-case-proof}{normal-subgroups-commutativity-special-case}{Normal subgroup commutativity}
    Consider the commutator of
    \(h\) and \(k\), meaning the element \(x\)
    such that \(hk = khx\). We need to show that \(x=1\).
    Now, \(x = h^{-1} k^{-1} h k\), and thus \(x = h^{-1} h^k\).
    The first member is in \(H\), and the second member is too as \(H\) as \normalsubgrptext.
    Likewise, \({(k^{-1})}^h \in K\).
    Thus, \(x \in H \intersection K\) and the commutator \(x = 1\).
\end{snippetproof}

\begin{snippetproof}{internal-direct-external-product-isomorphism-theorem-proof}{internal-direct-external-product-isomorphism-theorem}{Isomorphism between internal and external product}
    \todo
\end{snippetproof}

\end{document}