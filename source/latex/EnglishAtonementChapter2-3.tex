\documentclass[preview]{standalone}

\usepackage{amsmath}
\usepackage{amssymb}
\usepackage{bettelini}
\usepackage{stellar}
\usepackage{makecell}

\hypersetup{
    colorlinks=true,
    linkcolor=black,
    urlcolor=blue,
    pdftitle={English},
    pdfpagemode=FullScreen,
}

\begin{document}

\id{english-atonement-chapter2-3}
\genpage

\section{Summary}

\section{Exercises}

\begin{snippetexercise}{atonement-ex-1}%
{What do we learn about Cecilia Tallis in chapter 2? Focus on her character and her relationship
with Robbie.}
    She's the oldest of the sisters, she's educated and went to Cambridge.
    She was able to afford to go to college because she is rich, even thought many women
    did not get an education back then, and thus is anti conventional.
    Cecilia is shown to have a strange dynamic with Robbie, she wants to go against convention
    by doing what is not expected of her (jumping into the fountain).
    She doesn't behave like a women from the upper-class in the 1930s should behave.
    She prefers to stay home instead of going out with her college friends.
    We can infer that Cecilia and Robbie have a lingering attraction, even thought she
    pretends to be annoyed by it because her emotions are forbidden since Robbie is from a lower-class.
\end{snippetexercise}

\begin{snippetexercise}{atonement-ex-2}%
{Reread the incident at the fountain from Cecilia (ch. 2) and Briony's (ch. 3) point of view.}
    \begin{itemize}
        \item \textit{How should the reader interpret the scene from Cecilia's point of view?}
            The reader should interpret sexual tension, that they are attracted but cannot
            oblige their feelings. She is enjoying the fact
            that this incident might make Robbie feel at fault.
            She wants to punish him, not negatively, but rather in a way to
            make him closer to her.
        \item \textit{How does Briony's perspective during this scene influence her interpretation of events?}
            She does not perceives any sexual attraction between them,
            but rather perceives it as if Robbie is attacking her in some sort of way.
            Briony is too young and inexperienced to actually understand this kind of
            interactions. Briony is left confused by the event and uses
            her imagination to describe it using surreal adjectives, but is surprised
            when she realizes that the pieces of her made up story do not have a logical sequence.
    \end{itemize}
\end{snippetexercise}

\begin{snippetexercise}{atonement-ex-3}%
{What is hinted at in the following lines at the end of chapter 3?}
    \hspace{2cm}\makecell[l]{
        \textit{Six decades later she would describe how at the age of thirteen she had} \\
        \textit{written her way through a whole history of literature, beginning with stories} \\
        \textit{derived from the European tradition of folktales, through drama with simple moral} \\
        \textit{intent, to arrive at an impartial psychological realism which she had discovered} \\
        \textit{for herself, one special morning during a heat wave in 1935. (p. 38)}
    }
    This is a flashf orward sixty years into the future where Briony talks about
    what she learned six decades prior.
\end{snippetexercise}

\end{document}
