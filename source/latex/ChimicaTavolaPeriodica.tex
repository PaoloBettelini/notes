\documentclass[preview]{standalone}

\usepackage{amsmath}
\usepackage{amssymb}
\usepackage{stellar}
\usepackage{enumitem}

\hypersetup{
    colorlinks=true,
    linkcolor=black,
    urlcolor=blue,
    pdftitle={Stellar},
    pdfpagemode=FullScreen,
}

\begin{document}

\title{Stellar}
\id{chimica-tavola-periodica}
\genpage

\includesnpt[src=https://ptable.com/?lang=en\#Properties/Series|width=90\%|height=1000px]{iframe}

\section{Struttura}

\begin{snippet}{tavola-periodica-struttura}
    I \textit{gruppi} della tavola periodica vengono oraganizzati nella seguente maniera:
    \begin{enumerate}[label=\textbf{\Roman*}]
        \item $\rightarrow$ \textbf{metalli alcalini}, formano con l'ossigeno composti alcalini solubili in acqua;
        \item $\rightarrow$ \textbf{metalli alcalino-terrosi,} formano con l'ossigeno composti alcalini non
            sempre solubili in acqua;
        \item $\rightarrow$ \textbf{alogeni}, reagiscono facilmente con i metalli dando composti solidi chiamati sali;
        \item $\rightarrow$ \textbf{gas nobili}, gas poco reattivi.
    \end{enumerate}
    \vspace{.5cm}
\end{snippet}

\end{document}