\documentclass[preview]{standalone}

\usepackage{amsmath}
\usepackage{amssymb}
\usepackage{stellar}
\usepackage{definitions}

\def\setX{\blue X \clear}
\def\setT{\teal \mathcal{T} \clear}
\def\ts{(\setX, \setT)}

\begin{document}

\id{quotient-topology}
\genpage

\section{Quotient spaces}

\plain{Given a topological space and an equivalence relation, we can naturally
create another topological space where the points of each equivalence classes are "glued together".
The topology of the new space is defined such that a set is open if the "glued points" formed an open set
in the original space.}

\begin{snippetdefinition}{quotient-topological-space-definition}{Quotient topological space}
    \def\setTHat{\scolor[orange!50!black] \hat{\mathcal{T}} \clear}
    \def\uEl{\scolor[orange!50!black] u \clear}
    \def\xEl{\blue x \clear}
    \def\rel{{\violet \sim \clear}}
    Let \(\ts\) be a \topologicalspace and \(\rel\) be an \equivrelation.
    We define a canonical \surjective[surjection] \(q\colon \setX \to \setX /_\rel\)
    such that \(q(\xEl) = {[\xEl]}_\rel\).
    The \textit{quotient space} of \(\ts\) under \(\rel\) is the \topologicalspace
    \((\setX /_\rel, \setTHat)\) where
    \[
        \uEl \in \setTHat \iff q^{-1}(\uEl) \in \setT
    \]
\end{snippetdefinition}

\end{document}