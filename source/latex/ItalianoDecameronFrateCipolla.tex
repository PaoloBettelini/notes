\documentclass[preview]{standalone}

\usepackage{amsmath}
\usepackage{amssymb}
\usepackage{stellar}

\hypersetup{
    colorlinks=true,
    linkcolor=black,
    urlcolor=blue,
    pdftitle={Stellar},
    pdfpagemode=FullScreen,
}

\begin{document}

\title{Stellar}
\id{italiano-decameron-frate-cipolla}
\genpage

\section{Analisi}

\begin{snippetnote}{e9e278e6-8ef7-4bb6-a4f2-924b43155f08}{}
    Le novelle sono solitamente corte siccome la morale è centrata attorno una singola battuta.
    La novella di Frate Cipolla ne fa eccezione, compensando la brevità delle prime nove.
\end{snippetnote}

\begin{snippet}{frate-cipolla-analisi}
    \textbf{Rubrica:} Frate Cipolla promette a certi contadini di mostrar loro la penna dell'agnolo Gabriello, in luogo della quale trovando carboni, quegli dice esser di quegli che arostiron san Lorenzo.

    Frate Cipolla è un frate di santo Antonio conosciuto per essere un bravo e scaltro oratore.
    A messa terminata parla di una reliquia che avrebbe presentato, una piuma che fu una delle penne dell'Arcangelo Gabriele.
    Due suoi amici nella chiesa si decidono di fargli una beffa e di sottrargliela.
    Per eludere il fante ingaggiano una donna molto formosa per sedurlo, il quale viene distratto ma fallisce nel sedurre lei.
    I due rubano la penna e la rimpiazzano con del carbone.
    Il giorno dopo, Frate Cipolla si trova molti fedeli lì per vedere la piuma, ma quando ne apre la scatola
    ci trova il carbone. Allora comincia a parlare del viaggio che ha compiuto per trovarla,
    alla fine se la cava dicendo di aver preso la scatoletta per sbaglio contenente il carbone di San Lorenzo,
    e quindi era un miracolo che Dio gli avesse fatto prendere il carbone proprio due giorni prima del giorno santo.
    
    % Personaggi
    % Gli elementi comici
    %  - Il discordo di Guccio Imbrato
    %  - Il discorso di Frate Cipolla
    % Il pubblico 
    % La polemica religiosa
    
    La novella presente delle descrizioni eccezionalmente lunghe, in particolare addirittura tre.
    Frate cipolla è piccolo, lievo nel viso, con capelli rossi (tratto da imbroglione) ma
    soprattutto è bravo a parlare nonostante non abbia studiato (§7).
    Guccio Imbratta, il fante di Frate Cipolla, viene presentato da Frate Cipolla,
    mentre Frate Cipolla viene presentato dal narratore Dioneo.
    Guccio viene presentato (§17) con nove aggettivi in rime da tre:
    sugliardo e bugiardo; negligente, disubidente e maldicente; trascutato, smemorato e scostumato.
    Il terzo personaggio (§21) è la serva Nuta, piccolo, brutta, sudata, unta, con un paio di poppe che sembravano due
    cestoni di letame.
    Le descrizioni di personaggi fatte da altri personaggi danno informazioni circa i loro gusti.
    
    Il discorso di Guccio Imbratta, per cercare di conquistare la Nuta, è più corto del discorso di Frate Cipolla.
    Questo discorso è come un'anticipazione del discorso che farà il maestro Frate.
    La strategia di Frate Cipolla è un fiume di parole che in realtà hanno l'obiettivo di confondere il suo interlocutore.
    Anche Guccio è abbastanza scaltro, ma lui fallisce nel sedurre la serva.
    Frate cipolla dice di aver (§37) compiuto un viaggio dove sorge il sole (verso Oriente)
    che in realtà è una frase ambigua perché il sole sorge quasi ovunque sul pianeta terra.
    Oltre alla grande tecnica di spacciare per fantastiche delle cose che in realtà sono normalissime,
    si inventa anche delle parole.
    
    Il luogo della novella è la città di Boccaccio, Certaldo, per cui è come se l'autore prendesse in giro i suoi compaesani.
    La novella è narrata da Dioneo.
    All'interno della novella, Frate Cipolla racconta agli abitanti di Certaldo e ai suoi due amici (pubblico nascosto).
    Il fatto che gli abitanti siano di Certaldo è rivolto agli altri nove novellatori.
    
    Anche in questa novella il giudizio morale verso il protagonista è assente.
    Nonostante Frate Cipolla sia un ingannatore, ne esce positivamente come scaltro.
\end{snippet}

\begin{snote}{afefb98a-ca1d-48ce-a287-3d70a8c679bb}{Dante, Paradiso, XXIV, 124-126 (Beatrice)}
    L'ordine di sant'Antonio era particolarmente avido e scaltro con gli imbrogli,
    come per esempio le indulgenze.
    % TODO metti i versi qua
\end{snote}

\end{document}