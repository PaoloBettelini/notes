\documentclass[preview]{standalone}

\usepackage{amsmath}
\usepackage{amssymb}
\usepackage{stellar}
\usepackage{bettelini}

\hypersetup{
    colorlinks=true,
    linkcolor=black,
    urlcolor=blue,
    pdftitle={Biologia},
    pdfpagemode=FullScreen,
}

\begin{document}

\title{Biologia}
\id{biologia-esercizi}
\genpage

\begin{snippetexercise}{biologia-ex-1}{Qual'è il polimero di glucosio che permette agli esseri umani una riserva energetica?}
    Il glicogeno.
\end{snippetexercise}

\begin{snippetexercise}{biologia-ex-2}{In che modo l'ossigeno gassoso riesce ad entrare nel sistema cardiocircolatorio e raggiunge i mitocondri dei muscoli degli arti inferiori? Spiega sul piano anatomico e fisiologico}
    L'ossigeno entra nell'apparato respiratorio.
    L'ossigeno viene inserito nel globuli rossi passando per l'alveolo.
    Un po' di ossigeno viene disciolto nel sangue piuttosto che nel globulo rosso.
    I globuli rossi vengono trasportati dal sangue fino alle cellule di tutto il corpo.
    L'ossigeno, la quale è una molecole piccola, raggiunge il mitocondrio mediante una diffusione semplice (per gradiente) per cui trasporto passivo semplice.
\end{snippetexercise}

\begin{snippetexercise}{biologia-ex-3}{Spiega come avviene la condensazione di due amminoacidi in una cellula umana, specificando i dettagli molecolari}
    Due monomeri di proteine si uniscono mediante la condensazione, la quale libera una molecola di acqua.
    All'interno della cellula, ciò avviene quando un gruppo \(O\) si incontra con il gruppo \(OH\) dell'altro monomero.
\end{snippetexercise}

\begin{snippetexercise}{biologia-ex-4}{È possibile definire un mitocondrio all'interno di una cellula un sistema vivente? Spiega e argomenta}
    Sì, perché il mitoconfrio è autopoietico e può riprodursi da solo.
\end{snippetexercise}

\begin{snippetexercise}{biologia-ex-5}{Degli anticorpi (proteine) sono stati prodotti dal sistema immunitario (cellule bianche) per contrastare un'infezione virale nel lume intestinale. In che modo queste proteine possono raggiungere tale luogo?}
    Una volta prodotti, gli anticorpi vengono rilasciati nel flusso sanguigno e circolano in tutto il corpo. Essi sono in grado di raggiungere qualsiasi parte del corpo attraverso il sistema circolatorio.
    Per raggiungere il lume intestinale e combattere un'infezione virale in quella zona, gli anticorpi devono attraversare la barriera mucosale dell'intestino. Questa barriera mucosale è costituita da uno strato di muco e cellule epiteliali. Gli anticorpi non possono semplicemente passare attraverso questa barriera a meno che non siano specifici per l'infezione virale nell'intestino.
    per esocitosi esce dalla cellula della parete del vaso, per endocitosi entra nella parete dell'intestino, e per esocitori viene buttato fuori nel lume intestinale.
\end{snippetexercise}

\begin{snippetexercise}{biologia-ex-6}{Dove vengono interpretati gli impusli elettrici provenienti dall'occhio?}
    Zona logocipitale, nella parte posteriore della corteccia celebrare.
\end{snippetexercise}

\begin{snippetexercise}{biologia-ex-7}{Come fa l'occhio a mettere a fuoco un oggetto vicino?}
    Mediante il processo di accomodamento/accomodazione.
\end{snippetexercise}

\begin{snippetexercise}{biologia-ex-8}{Come si chiamano i due tipi principali di fotorecettori?}
    Coni e bastoncelli.
\end{snippetexercise}

\begin{snippetexercise}{biologia-ex-9}{Come mai la pupilla è nera?}
    La pupilla ha il colore dato dall'iride,
    grazie al nero tutti i colori vengono assorbiti.
\end{snippetexercise}

\end{document}