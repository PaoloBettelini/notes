\documentclass[preview]{standalone}

\usepackage{amsmath}
\usepackage{amssymb}
\usepackage{stellar}
\usepackage{bettelini}

\hypersetup{
    colorlinks=true,
    linkcolor=black,
    urlcolor=blue,
    pdftitle={Biologia},
    pdfpagemode=FullScreen,
}

\begin{document}

\title{Biologia}
\id{biologia-esercizi}
\genpage

\section{Esercizi}

\begin{snippetexercise}{biologia-ex-1}{Qual'è il polimero di glucosio che permette agli esseri umani una riserva energetica?}
    Il glicogeno.
\end{snippetexercise}

\begin{snippetexercise}{biologia-ex-2}{In che modo l'ossigeno gassoso riesce ad entrare nel sistema cardiocircolatorio e raggiunge i mitocondri dei muscoli degli arti inferiori? Spiega sul piano anatomico e fisiologico}
    L'ossigeno entra nell'apparato respiratorio.
    L'ossigeno viene inserito nel globuli rossi passando per l'alveolo.
    Un po' di ossigeno viene disciolto nel sangue piuttosto che nel globulo rosso.
    I globuli rossi vengono trasportati dal sangue fino alle cellule di tutto il corpo.
    L'ossigeno, la quale è una molecole piccola, raggiunge il mitocondrio mediante una diffusione semplice (per gradiente) per cui trasporto passivo semplice.
\end{snippetexercise}

\begin{snippetexercise}{biologia-ex-3}{Spiega come avviene la condensazione di due amminoacidi in una cellula umana, specificando i dettagli molecolari}
    Due monomeri di proteine si uniscono mediante la condensazione, la quale libera una molecola di acqua.
    All'interno della cellula, ciò avviene quando un gruppo \(O\) si incontra con il gruppo \(OH\) dell'altro monomero.
\end{snippetexercise}

\begin{snippetexercise}{biologia-ex-4}{È possibile definire un mitocondrio all'interno di una cellula un sistema vivente? Spiega e argomenta}
    Sì, perché il mitoconfrio è autopoietico e può riprodursi da solo.
\end{snippetexercise}

\begin{snippetexercise}{biologia-ex-5}{Degli anticorpi (proteine) sono stati prodotti dal sistema immunitario (cellule bianche) per contrastare un'infezione virale nel lume intestinale. In che modo queste proteine possono raggiungere tale luogo?}
    Una volta prodotti, gli anticorpi vengono rilasciati nel flusso sanguigno e circolano in tutto il corpo. Essi sono in grado di raggiungere qualsiasi parte del corpo attraverso il sistema circolatorio.
    Per raggiungere il lume intestinale e combattere un'infezione virale in quella zona, gli anticorpi devono attraversare la barriera mucosale dell'intestino. Questa barriera mucosale è costituita da uno strato di muco e cellule epiteliali. Gli anticorpi non possono semplicemente passare attraverso questa barriera a meno che non siano specifici per l'infezione virale nell'intestino.
    per esocitosi esce dalla cellula della parete del vaso, per endocitosi entra nella parete dell'intestino, e per esocitori viene buttato fuori nel lume intestinale.
\end{snippetexercise}

\begin{snippetexercise}{biologia-ex-6}{Dove vengono interpretati gli impusli elettrici provenienti dall'occhio?}
    Zona logocipitale, nella parte posteriore della corteccia celebrare.
\end{snippetexercise}

\begin{snippetexercise}{biologia-ex-7}{Come fa l'occhio a mettere a fuoco un oggetto vicino?}
    Mediante il processo di accomodamento/accomodazione.
\end{snippetexercise}

\begin{snippetexercise}{biologia-ex-8}{Come si chiamano i due tipi principali di fotorecettori?}
    Coni e bastoncelli.
\end{snippetexercise}

\begin{snippetexercise}{biologia-ex-9}{Come mai la pupilla è nera?}
    La pupilla ha il colore dato dall'iride,
    grazie al nero tutti i colori vengono assorbiti.
\end{snippetexercise}

\begin{snippetexercise}{biologia-ex-10}{Spiegare in modo dettagliato come viene attivata e mantenuta la produzione di succhi gastrici a
    livello dello stomaco}
    La primissima attivazione della digestione parte da impulsi nervosi che scaturiscono dai sensi
    (es. odore o vista etc.).
    Subito dopo la masticazione stimola principalmente la produzione.
    Una volta giunti nello stomaco le cellule della parate dello stomaco vengono stimolate e producono la gastrina.
    La gastrina a sua volta stimolerà il resto della digestione.
\end{snippetexercise}

\begin{snippetexercise}{biologia-ex-11}{Spiegare come fa lo stomaco a sapere che deve smettere di produrre succhi gastrici}
    La gastrina viene continuamente prodotta ed essa stimola la produzione di succhi gastrici.
    Inoltre, vi è un meccanismo di feedback positivo che stimola la produzione di succhi gastrici.
    Più pepsina c'è, più si attiva il pepsinogeno, per cui in breve tempo la concentrazione di pepsina aumenta esponenzialmente.
\end{snippetexercise}

\begin{snippetexercise}{biologia-ex-12}{Spiegare come fa lo stomaco a sapere che deve smettere di produrre succhi gastrici}
    Il chimo acido esce dallo stomaco ed entra nel duodeno.
    Quando non c'è il chimo lo stomaco continua a produrre succhi gastrini,
    ma vi è un sistema di feedback negativo che ne blocca la produzione.
    Questo feedback negativo è attivato dal fatto che l'acidità dello stomaco aumenta.
    Inoltre, il chimo viene percepito dalle cellule della parete del duodeno, le quali iniziano a produrre
    due ormoni (che vanno nel sangue come tutti gli ormoni), ossia il CCK (colecistochemica) e la secretina.
    Gli ormoni raggiungono il pancreas ed il fegato.
    Il fegato rilascia la bile che viene depositata nella cistifellea.
    Il CCK stimola la cistifellea a rilasciare la bile.
    Il CCK e la secretina bloccano la produzione di gastrina.
\end{snippetexercise}

% sono nella 3
%\sexercise{Spiegare cosa è il CCK, dove viene rilasciato e quando}{
%\sexercise{Elencare i passaggi che portano alla produzione di succhi pancreatici. Partire dall'arrivo del chimo}{

\begin{snippetexercise}{biologia-ex-13}{Da dove arrivano gli amminoacidi che vengono
    assorbiti dall'intestino? Descrivere il loro percorso.}
    Vengono principalmente ingeriti dopo essere masticati e lubrificati, dopodiché passano attraverso
    l'esofago ed entrano nello stomaco. Una volta che il bolo si trova all'interno dello stomaco,
    viene scomposto dalla pepsina e dai succhi gastrini (HCL), la cui produzione viene stimolata dalla gastrina.
    Il bolo diventa quindi chimo acido, il quale entrando nel duodeno ne stimola le pareti producendo
    CCK e secretina. La secretina stimola il pancreas a rilasciare una soluzione di bicarbonato di sodio
    nel duodeno che neutralizza l'acidi del chimo.
    Il CCK, invece, stimola il rilascio dal fegato dei succhi epatici e del pancreas i succhi pancreatici.
    Dopo aver attraversato il duodeno il prodotto (chilo, chimo neutralizzato) attraversa il digiuno e l'ileo dove
    viene completata l'assunzione delle sostanze nutritive.
    Gli amminoacidi (proteine idrolizzate in precedenza nel duodeno grazie agli enzimi) vengono assorbiti dalla parete intestinale (grazie ai vili e ai microvilli).
\end{snippetexercise}

\begin{snippetexercise}{biologia-ex-14}{Cosa succede agli amminoacidi una volta assorbiti a
    livello intestinale? Per rispondere alla domanda
    considerare i seguenti punti:}
    \begin{enumerate}
        \item L'urea è un prodotto del metabolismo delle proteine che
            viene eliminato attraverso le urine
        \item Nel fegato gli amminoacidi possono essere trasformati in
        zuccheri, grassi o precursori dell'ATP
        \item Gli enzimi digestivi (come per esempio la pepsina o la
        lipasi) sono delle proteine.
    \end{enumerate}

    % una volta finiti nel sangue vengono utilizzati per l'anabolismo o eliminati se in eccesso.
\end{snippetexercise}

\begin{snippetexercise}{biologia-ex-15}{Come mai i vertebrati terrestri una circolazione unica non è sufficiente?}
    Gli uccelli e mammiferi possiedono un ventricolo in più rispetto agli anfibi.
    Il problema degli anfibi è che il sangue viene mischiato in un unico punto, rendendo un'efficienza minore (contiene sempre un po' di CO2).
    I pesci necessitano di una sola circolazione: il sangue decellera nei capillari (perché sì passa da un tubo grande a tanti piccoli, aumenta l'attrito),
    dai capillare alle branchi il sangue riparte e va ovunque nel pesce. Dai capillari sistemici esso torna al cuore.
    Siccome il sangue è fermo, non vi è una spinta per portarlo avanti.
    Questa spinta viene dalla compressione del tubo, ossia i muscoli che permettono al pesce di nuotare
    spingono il sangue. Ne consegue che i pesci non possono stare fermi.
    Per i pesci è facile nuotare perché la gravità non è così forte, data la presenza della forza di Archimede.
    Negli anfibi, il cuore contraendosi spinge sia la circolazione polmonare che sistemica.
\end{snippetexercise}

\begin{snippetexercise}{biologia-ex-16}{Circolazione sistemica e polmonare}
    La circolazione polmonare prevede la circolazione del sangue fra il cuore, i polmoni e viceversa,
    mentre la circolazione sistemica prevede la circolazione del sangue fra il cuore e il resto del corpo e viceversa.
\end{snippetexercise}

\end{document}