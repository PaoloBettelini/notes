\documentclass[preview]{standalone}

\usepackage{amsmath}
\usepackage{amssymb}
\usepackage{stellar}
\usepackage{definitions}

\begin{document}

\id{topology-basic-results}
\genpage

\section{Basic results}

\newcommand\ts{{(X, \mathcal{T})}}

\begin{snippettheorem}{closure-in-terms-of-closed-supsets}{Closure in terms of closed supsets}
    Let \(\ts\) be a \topologicalspace and \(S\in X\).
    \[
        \closure[\ts][S] = \bigcap_{C \supseteq S} C
    \]
    where \(C\) is closed.
\end{snippettheorem}

\begin{snippetproposition}{properties-of-the-closure}{Properties of the closure}
    Let \(\ts\) be a \topologicalspace and \(S\in X\).
    \begin{enumerate}
        \item \(S \subseteq \closure[\ts][S]\);
        \item If \(S\) is \closedset[closed] and \(C \supseteq S\), \(C \subseteq \closure[\ts][S]\);
        \item \(S\) is \closedset[closed] \ifandonlyif \(\closure[\ts][S] = S\);
        \item \(\closure[\ts][S] = S \union S'\) where \(S'\)
        is the set of all the \accumulationpoint of \(S\).
        Since a finite \set cannot have \accumulationpoint[accumulation points],
        finite \set[sets] are always \closedset[closed].
    \end{enumerate}
\end{snippetproposition}

\begin{snippetproof}{properties-of-the-closure-proof}{properties-of-the-closure}{Properties of the closure}
    \begin{enumerate}
        \item Since every \set \(C\) in the intersection contains \(S\)
        it followes that \[S \subseteq \bigcap_{C \supseteq S} C\]
        where \(C\) is \closedset[closed] and \(S \subseteq {\text{cl}}_{\ts} S\)
        and \( {\text{cl}}_{\ts} S\) is is \closedset[closed] since it is an intersection of
        \closedset[closed sets].
        \item  By the definition of intersection \({\{B_\alpha\}}_{\alpha \in A}\), we have that
        \[
            \forall \closure[\ts][A] \in A, \bigcap_{\alpha \in A} B_\alpha
        \]
        whose intersection defines \(S\) and thus \(C \supseteq \closure[\ts][S]\).
        \item If \(\closure[\ts][S] = S\), then \(S\) is \closedset[closed] because \(\closure[\ts][S]\) is \closedset[closed].
        Viceversa, if \(S\) is \closedset[closed], since \(S \supseteq \closure[\ts][S]\).
        \item We show that the following inclusions are true:
        \begin{enumerate}
            \item \(\closure[\ts][S] \subseteq S \union S'\):
                by showing that if \(x \notin S \union S'\), we have
                \(x \notin \closure[\ts][S]\). That is, \(\exists C\)
                such that \(C \supseteq E\) and \(x \notin C\).
                Let \(x \notin S \union S'\). \[x\notin S' \iff \exists r > 0 \suchthat (x+r,x-r) \difference \{x\} \intersection E \neq 0\]
                XXX % TODOURGENT
            \item \(S \union S' \subseteq \closure[\ts][S]\):
        \end{enumerate}
    \end{enumerate}
\end{snippetproof}

\begin{snippetproposition}{interior-point-is-accumulation-point}{Interior point is accumulation point}
    Every \interiorpoint of a \set is an \accumulationpoint of that \set.
\end{snippetproposition}

\section{Open and closed sets}

\begin{snippetproposition}{open-sets-union-and-intersection}{Open sets union and intersection}
    \begin{enumerate}
        \item given a family of \set[sets] \({\{A_i\}}_{i\in I}\) where \(\forall i \in I, A_i\)
        is \topologicalspace[open][Open set],
        \[
            \bigcup_{i \in I} A_i
        \]
        is \topologicalspace[open][Open set];
        \item given the  \set[sets] \(A_0, A_1, \cdots, A_n\) where \(A_i\)
        is \topologicalspace[open][Open set],
        \[
            \bigcap_{k=0}^n A_k
        \]
        is \topologicalspace[open][Open set].
    \end{enumerate}
\end{snippetproposition}

\begin{snippetproof}{open-sets-union-and-intersection-proof}{open-sets-union-and-intersection}{Open sets union and intersection}
    \begin{enumerate}
        \item Every element of the union is in at least one \topologicalspace[open set][Open set], quindi se è interno
        ad un insieme aperto la loro union è anch'essa aperta.
        \item In the case where
        \[
            \bigcup_{i \in I} A_i = \emptyset
        \]
        the proposition is obviously true.
        Otherwise, \(\forall k, x \in \mathbb{R}\) che è aperto quindi
        esiste un raggio \(r > 0\) tale che la bolla di raggio attorno all'elemento è un sottinsieme di
        tutti gli elementi di \(A_k\). Per trovare questo raggio, prendiamo il minimo di tutti
        \[
            r = \min\{ r_1, r_2, \cdots, r_n \}
        \]
        dove \(r_n\) è un raggio tale che la bolla di raggio \(r_n\) attorno all'elemento
        è contenuta nell'insieme.
        Se il numero degli insiemi fosse infinitio, dovremmo prendere
        \[
            r = \inf\{ r_1, r_2, \cdots, r_n \}
        \]
        che potrebbe essere anche \(0\) o non esistere.
    \end{enumerate}
\end{snippetproof}

\begin{snippetproposition}{closed-sets-union-and-intersection}{Closed sets union and intersection}
    \begin{enumerate}
        \item given the \set[sets] \(A_0, A_1, \cdots, A_n\) where \(A_i\)
        is \closedset[closed],
        \[
            \bigcup_{k=0}^n A_k
        \]
        is \closedset[closed];
        \item given a family of \set[sets] \({\{A_i\}}_{i\in I}\) where \(\forall i \in I, A_i\)
        is \closedset[closed],
        \[
            \bigcap_{i \in I} A_i
        \]
        is \closedset[closed].
    \end{enumerate}
\end{snippetproposition}

\begin{snippetproposition}{set-is-open-iff-it-is-union-of-open-sets}{}
    A \set \(A\) is \topologicalspace[open][Open set] \ifandonlyif it is a union of \topologicalspace[open sets][Open set].
\end{snippetproposition}

\begin{snippetproposition}{both-open-and-closed-set-empty-reals}{Both open and closed set}
    The \set[sets] \(\emptyset\) and \(\realnumbers\) are both \topologicalspace[open][Open set]
    and \closedset[closed].
\end{snippetproposition}

\end{document}