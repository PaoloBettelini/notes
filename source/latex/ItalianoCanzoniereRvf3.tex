\documentclass[preview]{standalone}

\usepackage{amsmath}
\usepackage{amssymb}
\usepackage{stellar}
\usepackage{bettelini}

\hypersetup{
    colorlinks=true,
    linkcolor=black,
    urlcolor=blue,
    pdftitle={Stellar},
    pdfpagemode=FullScreen,
}

\begin{document}

\title{Stellar}
\id{italiano-canzoniere-rvf-3}
\genpage

\section{Rvf 3: Era il giorno ch'al sol si scoloraro}

\begin{snippet}{canzoniere-rvf-3}
    La terza poesia è un sonetto con schema delle rime ABBA ABBA CDE CDE.
    \\\\
    \StellarPoetry{1}{
        Era il giorno ch'al sol si scoloraro \\
        per la pietà del suo factore i rai, \\
        quando i' \textbf{fui preso}, \textbf{et non me ne guardai}, \\
        ché i be' vostr'occhi, donna, \textbf{mi legaro}.
    }{
        \parbox[t]{0.4\textwidth}{Era Venerdì Santo, giorno della passione e morte di Cristo, durante il quale la luce dei raggi del sole sbiadì
        \textbf{/}per la compassione che essi hanno nei confronti del loro Creatore [incarnato e crocifisso];
        \textbf{/}quando io fui conquistato dall'amore, e non pensai a difendermi [da quell'incanto],
        \textbf{/}perché i tuoi begli occhi, Laura, mi legarono indissolubilmente a te.}
    }
    \StellarPoetry{5}{
        Tempo non mi parea da far riparo \\
        contra colpi d'Amor: però m'andai \\
        secur, senza sospetto; onde i miei guai \\
        nel commune dolor s'incominciaro.
    }{
        \parbox[t]{0.4\textwidth}{(Essendo un giorno di lutto e meditazione religiosa) non mi pareva quello il momento di difendermi
        \textbf{/}dall'attacco improvviso di Amore; perciò andavo
        \textbf{/}fiducioso e senza timori: ragion per cui il mio dolore interiore e personale
        \textbf{/}ebbero inizio in mezzo al dolore generale per la Passione di Cristo (\quotes{commune dolor}).}
    }
    \StellarPoetry{9}{
        Trovommi Amor del tutto \textbf{disarmato} \\
        et aperta la via per gli occhi al core, \\
        che di lagrime son fatti uscio et varco:
    }{
        \parbox[t]{0.4\textwidth}{Amore mi colse del tutto disarmato,
        \textbf{/}e trovò libero il cammino per entrare nel cuore attraverso gli occhi,
        \textbf{/}che da quel momento sono diventati una sorgente da cui sgorgano lacrime.}
    }
    \StellarPoetry{12}{
        però al mio parer non li fu honore \\
        ferir me de \textbf{saetta} in quello stato, \\
        a voi \textbf{armata} non mostrar pur l'\textbf{arco}.
    }{
        \parbox[t]{0.4\textwidth}{Però, a mio parere, da parte di Amore fu comportamento da vigliacco
        \textbf{/}colpire con la freccia me, che ero in quello stato inerme [di pietà e contemplazione religiosa],
        \textbf{/}e non mostrare neppure l'arco a te, Laura, che eri ben difesa (dalla tua virtù e castità).}
    }

    La prima quartina è bipartita siccome i primi due versi danno l'indicazione del tempo,
    mentre gli altri due indicano cosa accade al poeta proprio in quel giorno.
    Era il giorno a cui al sole, per pietà nei confronti di Chi l'ha creato, si scolorirono i raggi.
    Questa è una parafrasi riguardante l'eclissi, e per il cui il giorno è il Venerdì santo, quando
    Cristo morì prima di risuscitare (il giorno della Passione di Cristo).
    Più nello specifico, la data di questo giorno di innamoramento è il 6 aprile 1327.
    % data presa da una nota abituaria | nella chiesa di santa chiara la vide per la prima volta
    Questa data è tuttavia forzata e non era un venerdì santo.
    Una peculiarità di Petrarca è quella di far coincidere le date della propria biografia
    con una storia superiore. Infatti, il 6 aprile 1348 è il giorno della morte di Laura.
    Per Petrarca il numero 6 è un numero centrale.
    L'innamoramento viene sempre attraverso lo sguardo, come il tipico innamoramento cortese,
    un legame che imprigiona le due anime assieme.
    Petrarca, con i secondi due versi, parla si come si sia innamorato durante il giorno della morte di Cristo.
    Infatti, i due verbi usati, presero e legarono, sono gli stessi due verbi utilizzati nella descrizione
    del Venerdì santo nel Vangelo per indicare l'arresto di Cristo.
    Il suo innamoramento è quindi fortemente legato alla passione di Cristo.
    I pensieri dell'autore erano orientati altrove, quando avvenne l'innamoramento, e non si aspettava
    di innamorarsi, proprio quando i suoi pensieri erano orientati verso la morte di Cristo.
    Questo concetto sarà ripreso varie volte.
    Inoltre, la prima prima e l'ultima della quartina sono in contrasto, perché il verbo scolorare
    indica una forza che si affievolisce, mentre legare indica una forza che aumenta.
    In oltre parole, come il sole scompare, Laura appare. \\
    % per cui i miei guai cominciavano nel comune dolore
    % non pensavo che in quel momento dovessi ripararmi/proteggermi
    Nella terza strofa l'autore si trova totalmente disarmato e trovò aperta la via
    che dagli occhi scende al cuore, occhi che sono uscio e varco elle lacrime.
    Il medesimo concetto di impreparatezza ritorna con il verbo disarmato, come d'altronde viene
    nuovamente espresso dalla seconda strofa.
    Nell'ultima terzina viene espresso come
    l'innamoramento colpisca soltando lui piuttosto che anche Laura.
    Non fu molto onorevole essere capito in un momento di preparazione, mentre
    Laura non vide nemmeno l'arco.
    \\\\
    Alcuni elementi della poesie cortese sono l'amore personificato come un guerriero
    senza onore e il passaggio dagli occhi al cuore.
    Vi sono anche degli elementi caratteristici della prospettiva religiosa e cristiana.
    Questi sono i primi due versi che si riferiscono al Vangelo, il Tempo (del Venerdì santo),
    e il comun dolore sempre riferitosi alla medesima vicenda.
    Tutto il canzoniere è incentrato su questo oscillamento fra amore sacro e prospettiva religiosa.
    Nel momento di massimo dolore per la cristianità non c'è una solo verso in cui Petrarca si accusi di
    aver orientato i propri pensieri altrove.
\end{snippet}

\end{document}