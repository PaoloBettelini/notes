\documentclass[preview]{standalone}

\usepackage{amsmath}
\usepackage{amssymb}
\usepackage{parskip}
\usepackage{fullpage}
\usepackage{hyperref}
\usepackage{wrapfig}
\usepackage{bettelini}
\usepackage{makecell}
\usepackage{stellar}
\usepackage{definitions}

\begin{document}

\id{complexanalysis-integration}
\genpage

\section{Complex integration}

\subsection{Complex integrals}

\begin{snippet}{complex-integral}
    Let \(f(t)\) be a complex-valued \function of a real parameter \(t\). Then
    we can decompose \(f\) into its real and imaginary parts
    \[
        \integral[a][b][f(t)][t]=\integral[a][b][u(t)][t]+i\integral[a][b][v(t)][t]
    \]
\end{snippet}

\subsection{Contour integrals}

\begin{snippetdefinition}{complex-path-definition}{Complex Path}
    Let \(z_0, z_1 \in \complexnumbers\) and \(a,b\in \realnumbers\).
    A \textit{complex path} from \(z_0\) to \(z_1\) is a continuous \function
    \(\gamma\colon [a,b] \to \complexnumbers\) where \(\gamma(a)=z_0\) and \(\gamma(b)=z_1\).
\end{snippetdefinition}

\begin{snippetdefinition}{complex-path-differentiable-definition}{Differentiable Complex Path}
    A complex path \(\gamma\) is differentiable if its real and imaginary parts are differentiable.
\end{snippetdefinition}

\begin{snippetdefinition}{closed-path-definition}{Closed Complex Path}
    A complex path \(\gamma\colon [a,b] \to \complexnumbers\)
    is \textit{closed} if \(\gamma(a)=\gamma(b)\).
\end{snippetdefinition}

\begin{snippetdefinition}{contour-integral-definition}{Contour Integral}
    Let \(f(z)\) be a complex-valued \function.
    When computing a definite integral we need a way to go from \(z_0\) to \(z_1\).
    \[
        \integral[z_0][z_1][f(z)][z]
    \]
    In order to compute this we need a complex path from \(z_0\) to \(z_1\).
\end{snippetdefinition}


\begin{snippettheorem}{contour-integral-parametrizes-curve}{Parametrized contour integral}
    Let \(f(z)\) be a complex-valued \function and
    let \(\Gamma\) be a complex path from \(z_0\) to \(z_1\), then
    \[
        \int_\Gamma f(z)\,dz = \integral[t_0][t_1][f(z(t))z'(t)][t]
    \]
\end{snippettheorem}

% anti holomorphic derivative

\begin{snippetdefinition}{cauchys-theorem}{Cauchy's Theorem}
    Let \(f(z)\) be a holomorphic \function on a simply-connected region \(\Omega\). \\
    Then, for any closed curve \(\gamma\) in \(\Omega\)
    \[
        \oint_\gamma f(z) \,dz = 0
    \]
\end{snippetdefinition}

%% https://www.math.ucdavis.edu/~romik/data/uploads/notes/complex-analysis.pdf
% https://courses.maths.ox.ac.uk/pluginfile.php/94006/mod_resource/content/4/complex.pdf
% https://www.maths.ed.ac.uk/~jmf/Teaching/MT3/ComplexAnalysis.pdf
% https://dzackgarza.com/rawnotes/Class_Notes/2020/Spring/Complex%20Analysis/ComplexAnalysis.html

\end{document}