\documentclass[preview]{standalone}

\usepackage{amsmath}
\usepackage{amssymb}
\usepackage{stellar}
\usepackage{definitions}

\begin{document}

\id{monoids-introduction}
\genpage

\section{Monoids}

\begin{snippetdefinition}{monoid-definition}{Monoid}
    A \textit{monoid} \((G,\circ)\) is a tuple containing a set \(G\) and
    a \binoperation \(\circ\) on \(G\).
    
    The relation must satisfy the following properties
    
    \begin{enumerate}
        \item \textbf{Associativity}: \(\forall a,b,c\in G, a \circ (b \circ c) = (a \circ b) \circ c\)
        \item \textbf{Identity}: \(\exists e \suchthat \forall a \in G, ea=ae=a\) 
    \end{enumerate}
\end{snippetdefinition}

\subsection{Expoonentiation}

\begin{snippetdefinition}{monoid-exponents-definition}{Exponents on monoids}
    Let \((G, \circ)\) be a \monoid. We define exponentiation
    for \(n \in \naturalnumbers\) and \(a\in G\) as follows:
    \[
        a^n \triangleq
        \begin{cases}
            1 & n=0 \\
            a^n \circ a & n \neq 0
        \end{cases}
    \]
    where \(1\) is the identity element.
    If \(a\) has inverse \(a^{-1}\), we also define exponentiation for \(n\in{\mathbb{Z}}^-\)
    as \({(a^{-1})}^{-n}\).
\end{snippetdefinition}

\begin{snippetproposition}{monoid-inverse-is-exponent-negative-1}{}
    Let \((G, \circ)\) be a \monoid and \(g\in G\).
    The operation of exponentiation by \(-1\) on \(g\) gives
    the inverse \(g^{-1}\) if \(g\) is invertible.
\end{snippetproposition}

\begin{snippetproof}{monoid-inverse-is-exponent-negative-1-proof}{monoid-inverse-is-exponent-negative-1}{}
    By definition, \(g^{-1} = {(g^{-1})}^1 = g^{-1}\), which is the inverse.
\end{snippetproof}

\begin{snippetproposition}{properties-of-monoid-exponents}{Properties of exponents on a monoid}
    Let \((M, \circ)\) be a \monoid. Then, for all \(g\in M\)
    and \(m,n \in\naturalnumbers\):
    \begin{enumerate}
        \item \(a^m \circ a^n = a^{m+n}\);
        \item \({(a^m)}^n = a^{mn}\).
    \end{enumerate}
    If \(g\) is invertible, the properties are satisfied for \(n \in \integers\).
\end{snippetproposition}

\begin{snippetlemma}{monoid-opposite-powers-are-inverses}{Opposites powers are inverses of eachother}
    Let \((M, \circ)\) be a \monoid.
    If \(a\in M\) is invertible and \(n \in \naturalnumbers\), then \(a^n\)
    and \(a^{-n}\) are inverses of eachother.
\end{snippetlemma}

\begin{snippetproof}{monoid-opposite-powers-are-inverses-proof}{monoid-opposite-powers-are-inverses}{Opposites powers are inverses of eachother}
    Without loss of generality, assume \(n\in\naturalnumbers\).
    We proceed by \principleofinduction[induction] on \(n\):
    \begin{itemize}
        \item the base case is \(a^0 = 1\) and \(a^{-0} = a^0 = 1\);
        \item for the inductive case let \(k\in\naturalnumbers\) such that \(a^k\)
        and \(a^{-k}\) are inverses of eachother.
        For simplity, we denote the inverse of \(a\) as \(\overline{a}\).
        Now,
        \begin{align*}
            a^{k+1} = a^k \circ a 
        \end{align*}
        which has \(\overline{a}\overline{a^k}\) as an inverse.
        By the inductive hypothesis, \(\overline{a^k}\) is
        \(a^{-k}\). That is, by definition, \({\overline{a}}^k\).
        Thus, \[\overline{a}\overline{a^k} = \overline{a}{\overline{a}}^k = {\overline{a}}^{k+1}\]
        Finally, by definition \({\overline{a}}^{k+1} = a^{-(k+1)}\).
    \end{itemize}
\end{snippetproof}

\begin{snippetproof}{properties-of-monoid-exponents-proof}{properties-of-monoid-exponents}{Properties of exponents on a monoid}
    \begin{enumerate}
        \item We proceed by \principleofinduction[induction] on \(n\):
        \begin{itemize}
            \item the base case is \(a^m \circ a^0 = a^m \circ 1 = a^m\);
            \item for the inductive case let \(k \in \naturalnumbers\)
            where the properties are satisfied:
            \begin{align*}
                a^m \circ a^{k+1} &= a^m \circ (a^k \circ a) \\
                &= (a^m \circ a^k) \circ a \\
                &= a^{m+k} \circ a \\
                &= a^{m+k+1}
            \end{align*}
        \end{itemize}
        \item We proceed by \principleofinduction[induction] on \(n\):
        \begin{itemize}
            \item the base case is \({(a^m)}^0 = {(a^{m\cdot 0})} = a^0 = 1\);
            \item for the inductive case let \(k \in \naturalnumbers\)
            where the properties are satisfied:
            \begin{align*}
                {(a^m)}^{k+1} &= {(a^m)}^k \circ a^m\\
                &= a^{mk} \circ a^m \\
                &= a^{mk + m} \\
                &= a^{m(k+1)}
            \end{align*}
        \end{itemize}
    \end{enumerate}

    To extend these properties to \(n\in\integers\) (for invertible elements),
    we need to distinguish various cases:
    \begin{enumerate}
        \item if \(m+n \geq 0\), we consider various sub-cases:
        \begin{enumerate}
            \item if \(m \geq 0\) and \(n \geq 0\): this case has already been tackled;
            \item if \(m \geq 0\) and \(n < 0\): we have \(a^m\circ a^n = a^m \circ {(a^{-1})}^{-n}\).
            Now, \(m+n \geq 0\) so \(m \neq -n\). Thus, \(a^{m+n} \circ a^{-n} = a^m\)
            because \(m+n \geq 0\). We thus have \[a^m\circ a^n = {\left(a^{m+n} \circ (a^{-n})\right)}\circ a^n\]
            By \snippetref[monoid-opposite-powers-are-inverses][this lemma],
            \(a^{-n} \circ a^n = 1\). Finally,
            \[
                a^m \circ a^n = a^{m+n} (a^{-n} \circ a^n) = a^{m+n} \circ 1 = a^{m+n}
            \]
            \item if \(m < 0\) and \(n \geq 0\): analogous to the last case;
        \end{enumerate}
        \item if \(m+n < 0\), we use the last cases.
        We have that \begin{align*}
            a^{m+n} &= {(a^{-1})}^{-m-n} \\
            &= {(a^{-1})}^{-m} \circ {(a^{-1})}^{-n}
        \end{align*}
        By \snippetref[monoid-opposite-powers-are-inverses][this lemma],
        \({(a^{-1})}^{-m}\) is the inverse of \({(a^{-1})}^{m}\)
        and \({(a^{-1})}^{-n}\) is the inverse of \({(a^{-1})}^{-n}\).
        Thus, \({(a^{-1})}^{-m} \circ {(a^{-1})}^{-n}\) is the inverse of
        \({(a^{-1})}^{n}\). Thus, \({(a^{-1})}^{-n}\circ{(a^{-1})}^{-m}\)
        is the inverse of \({(a^{-1})}^{m}\circ {(a^{-1})}^{n}\), thus the inverse of
        \(a^{-n} \circ a^{-m}\). But the inverse of \(a^{-n} \circ a^{-m}\)
        is \({(a^{-m})}^{-1} \circ {(a^{-n})}^{-1}\), which by \snippetref[monoid-opposite-powers-are-inverses][this lemma],
        is equal to \(a^m \circ a^n\).
    \end{enumerate}
    The second property is proved analogously.
\end{snippetproof}

\begin{snippetproposition}{monoid-exponent-distribution}{}
    Let \((M, \circ)\) be a \monoid.
    If for \(a,b\in M\), \(a\circ b = b\circ a\),
    then for \(n\in\naturalnumbers\)
    \[
        {(a\circ b)}^n = a^n \circ b^n
    \]
    If \(a\) and \(b\) are invertible, then the property holds for \(n\in\integers\).
\end{snippetproposition}

\begin{snippetproof}{monoid-exponent-distribution-proof}{monoid-exponent-distribution}{}
    We proceed by \principleofinduction[induction]:
    \begin{itemize}
        \item the base case is \({(a\circ b)}^0 = 1 = a^0 \circ b^0 = 1\);
        \item for the induction case let \(k\in\naturalnumbers\)
        such that the property holds. Then,
        \begin{align*}
            {(a\circ b)}^{k+1} &= {(a\circ b)}^k \circ a \circ b \\
            &= a^k \circ b^k \circ a \circ b
        \end{align*}
        If we had shown that \(b^k \circ a = a \circ b^k\),
        we could have concluded by saying that
        \[
            {(a\circ b)}^{k+1} = a^k \circ b^k \circ a \circ b
            = a^k \circ a \circ b^k \circ b = a^{k+1} \circ b^{k+1}
        \]
        Indeed, we are going to show that \(a\circ b^n = b^n \circ a\)
        by \principleofinduction[induction] on \(n\):
        \begin{itemize}
            \item the base case is \(a\circ b^0 = a \circ 1 = a\)
            and \(b^0 \circ a = 1 \circ a = a\);
            \item the induction case is given by
            \begin{align*}
                a\circ b^{k+1} &= a \circ b^k \circ b \\
                &= b^k \circ a \circ b \\
                &= b^k \circ b \circ a \\
                &= b^{k+1} \circ a
            \end{align*}
        \end{itemize}
    \end{itemize}
    To extend this property to \(n\in\integers\) (for invertible elements), we have
    \begin{align*}
        {(a\circ b)}^n &= {\left({(a\circ b)}^{-n}\right)}^{-1} \\
        &= {\left((a^{-n}) \circ (b^{-n})\right)}^{-1}
    \end{align*}
    We showed that if \(a\) and \(b\) commute,
    then \(a\) commutes with the powers of \(b\), the powers of \(b\)
    commute with the powers of \(a\) (for natural exponent).
    Thus, this is
    \begin{align*}
        &\phantom{=} {\left((b^{-n}) \circ (a^{-n})\right)}^{-1} \\
        &= (a^{-n})^{-1} \circ (b^{-n})^{-1} \\
        &= a^n \circ b^n
    \end{align*}
\end{snippetproof}

\begin{snippetproposition}{monoid-inverse-commutativity}{}
    Let \((G, \circ)\) be a \monoid.
    Then, if \(a\circ b = b\circ a\), \(a\circ b^k = b^k \circ a\).
\end{snippetproposition}

\begin{snippetproof}{monoid-inverse-commutativity-proof}{monoid-inverse-commutativity}{}
    \todo
    Since \(a\) commuted with every power of \(b\), every power of \(b\)
    commuted with \(a\).
\end{snippetproof}

\begin{snippetcorollary}{monoid-powers-of-same-element-commut}{Powers of same element commut}
    Let \((G, \circ)\) be a \monoid.
    Then, \(a^m \circ a^n = a^n \circ a^m\).
\end{snippetcorollary}

\subsection{Inverse element}

\begin{snippettheorem}{uniqueness-of-the-inverse-element}{Uniqueness of the inverse element}
    If \(a^{-1}\) is an inverse of \(a\) in a \monoid, then it is unique.
\end{snippettheorem}

\begin{snippetproof}{uniqueness-of-the-inverse-element-proof}{uniqueness-of-the-inverse-element}{Uniqueness of the inverse element}
    Suppose we have \(a\in G\) with inverses \(b\) and \(c\).
    \begin{align*}
        b = b \circ e &= b \circ (a \circ c) \\
        (b \circ a) \circ c &= e \circ c \\
        &= c
    \end{align*}
    Thus, \(b\) and \(c\) must be the same.
\end{snippetproof}

\begin{snippetproposition}{inverse-element-of-inverse-element}{Inverse element of inverse element}
    Let \((M, \circ)\) be a \monoid. If \(a\) is invertible by \(a^{-1}\),
    then \(a^{-1}\) is also invertible and its inverse is \(a\).
\end{snippetproposition}

\begin{snippetproposition}{inverse-of-inverse-product}{Inverse of product}
    Let \((M, \circ)\) be a \monoid. If \(a\) and \(b\) are invertible by \(a^{-1}\) and \(b^{-1}\),
    then \(a\circ b\) is the inverse of\(a^{-1} \circ b^{-1}\).
\end{snippetproposition}

\end{document}