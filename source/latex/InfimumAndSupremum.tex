\documentclass[preview]{standalone}

\usepackage{amsmath}
\usepackage{amssymb}
\usepackage{stellar}
\usepackage{definitions}

\begin{document}

\id{infimum-and-supremum}
\genpage

\section{Definitions}

\begin{snippetdefinition}{supremum-definition}{Supremum}
    Let \(E\) and \(X\) be \set[sets] where \(E\subseteq X \land E \neq \emptyset\)
    and \(\leq\) be a \partialorder on \(E\).
    Then, the \textit{supremum} of \(E\) is the \upperbound of \((E,\leq)\) that is less than every other 
    \upperbound[upper bounds] of \((E,\leq)\), if it exists.
    \[
        \mu = \origsup E
    \]
\end{snippetdefinition}

\begin{snippetdefinition}{infimum-definition}{Infimum}
    Let \(E\) and \(X\) be \set[sets] where \(E\subseteq X \land E \neq \emptyset\)
    and \(\leq\) be a \partialorder on \(E\).
    Then, the \textit{infimum} of \(E\) is the \lowerbound of \((E,\leq)\) that is greater than every other 
    \lowerbound[lower bounds] of \((E,\leq)\), if it exists.
    \[
        \mu = \originf E
    \]
\end{snippetdefinition}

\section{Basic results}

%\begin{snippetproposition}{infimum-and-supremum-empty-set}{Infimum and supremum of \(\emptyset\)}
% !!! questo però nell'ordinamento classico.  
%\end{snippetproposition}

% Mettere in una proposizione che P(n) vale per tutti gli elementi di \emptyset
% se non c'è già, e citarlo nella dimostrazione

\end{document}