\documentclass[preview]{standalone}

\usepackage{amsmath}
\usepackage{amssymb}
\usepackage{stellar}

\hypersetup{
    colorlinks=true,
    linkcolor=black,
    urlcolor=blue,
    pdftitle={Stellar},
    pdfpagemode=FullScreen,
}

\begin{document}

\id{italiano-principe-capitolo-vi}
\genpage

\section{Capitolo VI}

\begin{snippet}{il-principe-capitolo-vi-parte1}
    \begin{center}
        \begin{minipage}{0.75\textwidth}
            \itshape
            De' Principati nuovi, che con le proprie armi e virtù si acquistano. 
        \end{minipage}
    \end{center}
    \vspace{0.25cm}
    Il testo parte con il concetto di imitazione, per il quale uno scrittore
    debba necessariamente ispirarsi a qualcuno.
    In quanto, Machiavelli indica il suo volere di fornire dei modelli altissimi
    ai quali i principi dovrebbero ispirarsi.
    Questi modelli sono irraggiungibili ma, secondo Machiavelli, è necessario ispirarsi a loro.
    Per motivare ciò, viene data l'analogia dell'arciere che mira più in alto per raggiungere il suo bersaglio
    (mira più in alto così sei sicuro di raggiungere i tuoi obiettivi, che dovrebbero essere chiaramente
    più modesti rispetto a quelli degli imperi antichi, che sono appunti, irraggiungibili, come l'Impero Romano).
\end{snippet}

\begin{snippetnote}{630b4e00-9bb3-4f9f-9733-88ab2488795e}{}
    Chiaramente, nell'ambito \underline{politico}, il fatto di avere avuto successo in passato
    non garantisce quello di avere successo nel presente.
    Questo motivo denota un inevitabile fallimento in questa ideologia.
\end{snippetnote}

\plain{I modelli da imitare indicati sono i seguenti:}

\begin{snippetcharacter}{moise}{Moisè}
    \textit{Moisè} è un personaggio biblico, legislatore e fondatore del popolo ebraico di Israele.
\end{snippetcharacter}

\begin{snippetcharacter}{ciro}{Ciro}
    \textit{Ciro II} è il fondatore della monarchia di Persia.
\end{snippetcharacter}

\begin{snippetcharacter}{romolo}{Romolo}
    \textit{Romolo} è un personaggio legato al mito della fondazione di Roma.
\end{snippetcharacter}

\begin{snippetcharacter}{teseo}{Teseo}
    \textit{Teseo} è un personaggio mitologico, re di Atene.
\end{snippetcharacter}

\begin{snippet}{il-principe-capitolo-vi-parte2}
    Tutti e quattro sono dei fondatori di regno o di repubbliche.
    Tutti e quattro sono inoltre riusciti nei loro intenti grazie
    alle loro capacità, e non certo grazia alla fortuna.
    Inoltre, il loro tratto comune è quello di avere avuto un difficile riconoscimento iniziale,
    e nessuna agevolazione.
    \\\\
    Tuttavia, l'unico personaggio realmente esistito è Ciro.
    Nonostante ciò, tutti sono scritti sulla stessa riga: all'autore non importa
    questa distinzione, bensì solo l'importanza dell'esempio.
    L'unica distinzione fatta è quella su Moisè, ossia l'ironia circa il fatto che lui
    avesse il privilegio di parlare direttamente con Dio. 
    \\\\
    Andare al potere con virtù significa fatica per raggiungere la carica,
    ma successivamente mantenerla in maniera semplice.
    Mentre raggiungere il potere con la fortuna, significa guadagnare la carica semplicemente,
    ma fare fatica a mantenerla.
    \\\\
    Un elemento intermedio fra la virtù e la fortuna è l'occasione.
    Se il principe non ha virtù, l'occasione non è né riconosciuta né sfruttata.
    Un minimo di fortuna è necessaria, ma l'assenza di fortuna viene compensata dalla grande virtù.
    Infatti, questi elementi di fortuna sono descritti per ognuno dei quattro modelli:
    \begin{itemize}
        \item per Moisè, trovare il popolo di Israele in Egitto schiavo e oppressi;
        \item per Ciro, trovare i persi malcontenti;
        \item per Romolo, essere allontano dal suo paese natale;
        \item per Teseo, trovare una popolazione dispersa per poi riunirla.
    \end{itemize}

    Quando un principe domina un nuovo popolo, spodestando il vecchio regime,
    il gruppo che sostiene il vecchio principe è più agguerrito, perché lotta per delle \textbf{certezze},
    mentre quello che sostiene quello nuovo è più tiepido, siccome lotta per un'\textbf{idea}.
    Gli uomini lottano debolmente per un ideale se non hanno certezza di un risultato.
    \\\\
    Un altro fattore chiave è quello dell'autosufficienza:
    nel caso in cui il principe debba pregare per il sostegno da altri, cade inevitabilmente,
    ma quando il principe è autonomo, è raro che incorra in pericoli.
    Da questo deriva il fatto che solo chi è armato può vincere, mentre chi
    non possiede forza è destinato ad andare in rovina.
    Infatti, Per quanto i quattro sopracitati fossero virtuosi, ad un certo punto hanno avuto la necessità
    di usare le armi.
    \\\\
    Inoltre, le convinzioni del popolo sono importanti: è facile convincere il popolo di qualcosa,
    ma è difficile fargli cambiare idea e fermarli.
    Nel caso in cui la popolazione sia convinta fermamente di qualcosa, è necessario
    fargli cambiare idea con le armi.
    \\
    % ovvero = oppure

    Viene infine dato un altro esempio (minore), ossia quello di Ierone (Gerone) Siracusano.
    Questo esempio riassume tutto il capitolo: diventò principe di Siracusa
    senza nessuna esperienza di potere e avendo dalla fortuna \textit{solo} l'occasione
    (essendo I siracusani oppressi, lo elessero come principe).
    Egli eliminò il vecchio esercito per averne uno proprio e fedele,
    cambia completamente le alleanze, e su tale edificio (fondamenta faticate) poté
    governare facilmente.
\end{snippet}

\end{document}