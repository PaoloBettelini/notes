\documentclass[preview]{standalone}

\usepackage{amsmath}
\usepackage{amssymb}
\usepackage{stellar}
\usepackage{definitions}

\begin{document}

\id{biologia-regolazione-insulina}
\genpage

\section{Glucotrasportatori di membrana (Glut4)}

\begin{snippetdefinition}{enterociti-definition}{Enterociti}
    Gli \textit{enterociti} sono cellule epiteliali che rivestono l'intestino tenue.
    Queste cellule svolgono un ruolo fondamentale nell'assorbimento dei nutrienti durante la digestione.
\end{snippetdefinition}

\begin{snippet}{glut4-expl}
    Gli enterociti prendono il glucosio e il Na\({}^+\) dal lume intestinale
    e li trasportano fino al sangue grazie a un simporto Na\({}^+\)/glucosio, una glucosio permeasi
    (una proteina per la diffusione facilitata del glucosio), e la Na\({}^+\)/K\({}^+\)ATPasi. 
\end{snippet}


\includesnpt[width=80\%|src=/snippet/static/glut4.png]{centered-img}

\begin{snippet}{glut4-expl2}
    I Glut4 verranno inseriti nella cellula quando il sangue è pieno di glucosio (quando la \textit{glicemia} è alta).
È necessario che la cellula risponda alla alta glicemia nel sangue per produrre i Glut4,
questo messaggio viene spedito attraverso il sangue con l'\textit{insulina}.
L'insulina lega il ricettore, cambiandone la struttura. \\
L'insulina viene prodotta dal pancreas, infatti, le cellule del pancreas sono in grado di misurare la glicemia.
\end{snippet}

\includesnpt[width=80\%|src=/snippet/static/insulin.png]{centered-img}

\plain{Il diabete è un problema nella ricezione dell'insulina.}

\begin{snippetdefinition}{omeostasi-definition}{Omeostasi}
    L'\textit{omeostasi} è il processo di autoregolazione dei vari valori.
\end{snippetdefinition}

\plain{Il seguente diagramma illustra un esempio di omeostasi per la regolazione della glicemia nel sangue.}

\includesnpt[width=80\%|src=/snippet/static/omeostasi.png]{centered-img}

\end{document}