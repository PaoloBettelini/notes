\documentclass[preview]{standalone}

\usepackage{amsmath}
\usepackage{amssymb}
\usepackage{stellar}

\hypersetup{
    colorlinks=true,
    linkcolor=black,
    urlcolor=blue,
    pdftitle={Stellar},
    pdfpagemode=FullScreen,
}

\begin{document}

\title{Stellar}
\id{biologia-evoluzione-uomo}
\genpage

\section{Classificazione di Homo sapiens}

\begin{snippet}{classificazione-homo-sapiens}
    \vspace{0.25cm}
    \begin{center}
        \begin{tabular}{|l|l|}
            \hline Dominio & Eukarya \\
            \hline Regno & Animalia \\
            \hline Phylum & Chordata \\
            \hline Subphylum & Vertebrata \\
            \hline Classe & Mammalia \\
            \hline Ordine & Primates \\
            \hline Superfamiglia & Hominoidea \\
            \hline Famiglia & Hominidae \\
            \hline Sottofamiglia & Homininae \\
            \hline Genere & Homo \\
            \hline Specie & H. sapiens \\
            \hline
        \end{tabular}
    \end{center}
    \vspace{0.25cm}
\end{snippet}

\begin{snippet}{a893df50-d271-49ba-bdfb-b437fe390a6a}
    7 milioni di anni fa, il genere homo e scimpanzè si separano.
    Gli ominidi comprendono uomini e scimpanzè.
    Ci sono tre generi di homo.
    %% RIFARE.
    % Fare l'albero ominidi ominini 

    Gli adattamenti che portarono all'evoluzione dei primati sono legati alla
    vita arboricola, ovvero:
    \begin{itemize}
        \item estremità prensili;
        \item dominanza della vista;
        \item verticalizzazione del corpo;
        \item aumento delle dimensioni cerebrali;
        \item incremento delle cure parentali.
    \end{itemize}

    L'ardipithecus è il primo genere di ominini che tendenzialmente inizia a verticalizzarsi.
    L'australopithecus ha il bacino ancora più dritto e non sta più sugli alberi.
    Successivamente arriva l'homo di neanderthal.
\end{snippet}

% TODO: usare ciclo_amminoacidi.png

\end{document}