\documentclass[preview]{standalone}

\usepackage{amsmath}
\usepackage{amssymb}
\usepackage{stellar}
\usepackage{bettelini}

\hypersetup{
    colorlinks=true,
    linkcolor=black,
    urlcolor=blue,
    pdftitle={Stellar},
    pdfpagemode=FullScreen,
}

\begin{document}

\title{Stellar}
\id{orlando-furioso-proemio-testo}
\genpage

\section{Testo}

% https://letteritaliana.weebly.com/la-fuga-di-angelica1.html

\begin{snippet}{orlando-furioso-ottava-1-proemio}
    \StellarPoetry{1}{
        Le donne, i cavallier, l'arme, gli amori,\\
        le cortesie, l'audaci imprese io canto,\\
        che furo al tempo che passaro i Mori\\
        d'Africa il mare, e in Francia nocquer tanto,\\
        seguendo l'ire e i giovenil furori\\
        d'Agramante lor re, che si diè vanto\\
        di vendicar la morte di Troiano\\
        sopra re Carlo imperator romano.
    }{\parbox[t]{0.4\textwidth}{Delle donne, dei cavalieri, delle battaglie, degli amori,
    degli atti di cortesia, delle audaci imprese io canto,
    che ci furono nel tempo in cui gli Arabi
    attraversarono il mare d'Africa, e arrecarono tanto danno in Francia, seguendo le ire e i furori giovanili
    del loro re Agramante, il quale si vantò
    di poter vendicare la morte di Traiano
    contro il re Carlo, imperatore romano.}}
    \\\\
    I due grandi temi del libro sono l'amore e la guerra.
    I modelli letterali al quale si ispira sono al letteratura bretone (per l'amore),
    mentre letteratura Carolingia (per l'amore).
    Nella poesia è presente un doppio chiasmo, indicante.
    Tradizionalmente il proemio è suddiviso in tre parti: protasi, invocazione e dedica.
    La prima parte è la protasi cioè la dichiarazione della materia di
    cui parlerà il libro, in questo caso va dall'ottava 1 all'ottava 2 verso 4.
    Poi c'è l'invocazione che occupa soltanto la seconda parte della seconda ottava
    (2.4 - 2.8). Infine, c'è la dedica che va dall'ottava 3 alla 4.
    All'inizio c'è la spiegazione del tema, cioè amore e guerra.
    Ariosto si basa su due cicli, quello carolingio e quello bretone.
    Il ciclo carolingio, insieme di racconti francesi e ispirati a Carlo Magno,
    che si impronta sulla guerra e la virtù religiosa.
    Il ciclo bretone o arturiana si basa sulla corte di re Artù,
    è improntato sulla fantasia e l'avventura. Le prime parole sono legate
    tra loro tra un chiasmo di “donne, amori, cortesie” e “cavallier, arme,
    audaci imprese”. Il chiasmo è presente per presagire la struttura del libro,
    i due cicli saranno uniti lungo il libro, intrecciandosi come il chiasmo. 
    La struttura dei primi due versi segue un ordine non “corretto”,
    il soggetto che parla è all'ultimo.
    Dunque, apparentemente l'Orlando viene scritto in maniera oggettiva,
    il poeta si mette in secondo piano.
    L'oggettività sta nel mettere la materia prima del narratore.
\end{snippet}

\begin{snippet}{orlando-furioso-ottava-2-proemio}
    \StellarPoetry{2}{
        Dirò d'Orlando in un medesmo tratto\\
        cosa non detta in prosa mai, né in rima:\\
        che per amor venne in furore e matto,\\
        d'uom che sì saggio era stimato prima;\\
        se da colei che tal quasi m'ha fatto,\\
        che 'l poco ingegno ad or ad or mi lima,\\
        me ne sarà però tanto concesso,\\
        che mi basti a finir quanto ho promesso.
    }{\parbox[t]{0.4\textwidth}{Nello stesso tempo, racconterò di Orlando
    cose che non sono state mai dette né in prosa né in rima:
    che per amore, divenne completamente folle,
    lui che prima era considerato uomo così saggio;
    dirò queste cose se da parte di colei che mi ha quasi reso tale
    e che a poco a poco consuma il mio piccolo ingegno,
    me ne sarà concesso a sufficienza (di ingegno)
    che mi basti a finire l'opera che ho promesso.}}
    \\\\
    Vi dirò meglio una cosa che non ha mai scritto nessuno,
    ossia l'impazzimento di Orlando a causa dell'amore,
    lui che era così saggio.
    Orlando è un personaggio storico, paladino di un nobile, sotto Carlo Magno,
    ma le storie di pazzia dell'innamoramento sono inventate.
    Questo innamoramento è straordinario perché Orlando era particolarmente saggio.
    Anche se una persona così saggia come lui può impazzire per amore come lui,
    da questa sorte non è al riparo nessuno.
    %Vi è una autoironia dicendo: io l'ingegno ce l'ho già, speriamo
    %non mi venga tolto.
    Una lima, col suo agire costante, erode nel tempo.
\end{snippet}

\begin{snippet}{orlando-furioso-ottava-3-proemio}
    \StellarPoetry{3}{
        Piacciavi, generosa Erculea prole,\\
        ornamento e splendor del secol nostro,\\
        Ippolito, aggradir questo che vuole\\
        e darvi sol può l'umil servo vostro.\\
        Quel ch'io vi debbo, posso di parole\\
        pagare in parte e d'opera d'inchiostro;\\
        pagare in parte e d'opera d'inchiostro;\\
        che quanto io posso dar, tutto vi dono.
    }{\parbox[t]{0.4\textwidth}{Vi piaccia, generosa e nobile prole del [duca] Ercole I,
    che siete ornamento e splendore del nostro tempo,
    Ippolito, di gradire questo poema che vuole
    e darvi solo può il vostro umile servitore.
    Il mio debito nei vostri confronti, lo posso solo
    pagare in parte con le mie parole ed opere scritte;
    non mi si potrà accusare di darvi poco,
    perché io vi dono tutto quanto posso donarvi, non ho altro.}}
    \\\\
    Questi versi rappresentano un esempio di cortesia e
    ammirazione nei confronti del destinatario del canto,
    Ippolito d'Este, figlio di Ercole I d'Este.
    Il poeta esprime la sua gratitudine e il suo rispetto
    per Ippolito, riconoscendolo come un ornament
    e uno splendore del suo tempo.

    Il tono del canto è deferente e rispettoso,
    e Ariosto si colloca in una posizione di umiltà di
    fronte al destinatario.
    Utilizza l'immagine di sé stesso come "umile servo" di Ippolito,
    sottolineando il suo desiderio di compiacerlo con
    il suo lavoro poetico.

    Il verso finale, "che quanto io posso dar, tutto vi dono",
    sottolinea l'impegno totale del poeta nel rendere omaggio
    e onore a Ippolito, promettendo di dedicargli tutto
    ciò che può offrire, sia in parole che in azione.
\end{snippet}

\begin{snippet}{orlando-furioso-ottava-4-proemio}
    \StellarPoetry{4}{
        Voi sentirete fra i più degni eroi,\\
        che nominar con laude m'apparecchio,\\
        ricordar quel Ruggier, che fu di voi\\
        e de' vostri avi illustri il ceppo vecchio.\\
        L'alto valore e' chiari gesti suoi\\
        vi farò udir, se voi mi date orecchio,\\
        e vostri alti pensieri cedino un poco,\\
        sì che tra lor miei versi abbiano loco.
    }{\parbox[t]{0.4\textwidth}{Voi mi sentirete ricordare fra i più valorosi eroi,
    che mi appresto a citare lodandoli,
    di quel Ruggiero che fu il vostro
    e dei vostri nobili avi il capostipite.
    Il suo grande valore e le sue imprese
    vi farò udire se mi presterete ascolto;
    e ile vostre profonde preoccupazioni cedano un poco,
    in modo che tra loro i miei versi possano trovare spazio.}}
    \\\\
    TODO
\end{snippet}

\begin{snippet}{orlando-furioso-ottava-5-proemio}
    \StellarPoetry{5}{
        Orlando, che gran tempo innamorato\\
        fu de la bella Angelica, e per lei\\
        in India, in Media, in Tartaria lasciato\\
        avea infiniti ed immortal trofei,\\
        in Ponente con essa era tornato,\\
        dove sotto i gran monti Pirenei\\
        con la gente di Francia e de Lamagna\\
        re Carlo era attendato alla campagna,
    }{\parbox[t]{0.4\textwidth}{Orlando, che per tanto tempo era stato innamorato
    della bella Angelica e per lei
    in India, in Oriente, aveva lasciato
    trofei immortali ed in numero infinito,
    era tornato infine con la donna amata in Occidente
    dove, sotto gli alti monti Pirenei,
    con i Francesi ed i Tedeschi,
    il re Carlo si era insediato in campo aperto}}
    \\\\
    Orlando, che per molto tempo è stato innamorato de la bella Angelica (principessa musulmana del Catai, oggi Cina).
    Riesce a portarla a casa sua in Ponente, dove trova una situazione di guerra,
    con re Carlo si era accampato preparandosi per la guerra.
\end{snippet}

\begin{snippet}{orlando-furioso-ottava-6-proemio}
    \StellarPoetry{6}{
        per far al re Marsilio e al re Agramante\\
        battersi ancor del folle ardir la guancia,\\
        d'aver condotto, l'un, d'Africa quante\\
        genti erano atte a portar spada e lancia;\\
        l'altro, d'aver spinta la Spagna inante\\
        a destruzion del bel regno di Francia.\\
        E così Orlando arrivò quivi a punto:\\
        ma tosto si pentì d'esservi giunto:
    }{\parbox[t]{0.4\textwidth}{perché il re Marsilio ed il re Agramante
    si pentissero ancora una volte delle loro folli azioni;
    Agramante per avere condotto dall'Africa tante
    persone quanto erano in grado di portare spada e lancia,
    Marsilio per avere condotto la Spagna
    nella distruzione del bel regno di Francia.
    E così Orlando arrivò sul posto al momento giusto,
    ma subito si pentì di esservi giunto.}}
    \\\\
    Re Carlo si accampa per fare pentire al re Marsilio (re della spagna musulmana)
    di aver condotto il proprio esercito contro la Francia e
    al re Agramante (re dell'Africa, musulmano)
    di aver portato le sue truppe in Europa.
    Quando Orlando torna con la donna amata, si pente di essere tornato proprio in quel momento.
\end{snippet}

\begin{snippet}{orlando-furioso-ottava-7-proemio}
    \StellarPoetry{7}{
        Che vi fu tolta la sua donna poi:\\
        ecco il giudicio uman come spesso erra!\\
        Quella che dagli esperi ai liti eoi\\
        avea difesa con sì lunga guerra,\\
        or tolta gli è fra tanti amici suoi,\\
        senza spada adoprar, ne la sua terra.\\
        Il savio imperator, ch'estinguer volse\\
        un grave incendio, fu che gli la tolse.
    }{\parbox[t]{0.4\textwidth}{Gli anche fu tolta la donna che amava:
    ecco come il giudizio umano spesso sbaglia!
    La donna che dalle coste Orientali a quelle Occidentali
    aveva difeso con una tanto lunga guerra,
    ora gli viene tolta tra tanti suoi amici,
    senza che sia adoperata spada alcuna, sulla sua terra.
    Il saggio imperatore, con la volontà di estinguere
    un grave incendio (pericolosa contesa d'amore), fu a togliergliela.}}
    \\\\
    Carlo Magno stesso, sottrae Angelica da Orlando.
    Quella che aveva difeso gli viene portata via così dagli amici,
    viene detto che Carlo Magno volesse \quotes{estinguere un incendio} (metaforicamente).
\end{snippet}

\begin{snippet}{orlando-furioso-ottava-8-proemio}
    \StellarPoetry{8}{
        Nata pochi dì inanzi era una gara\\
        tra il conte Orlando e il suo cugin Rinaldo,\\
        che entrambi avean per la bellezza rara\\
        d'amoroso disio l'animo caldo.\\
        Carlo, che non avea tal lite cara,\\
        che gli rendea l'aiuto lor men saldo,\\
        questa donzella, che la causa n'era,\\
        tolse, e diè in mano al duca di Bavera;
    }{\parbox[t]{0.4\textwidth}{Pochi giorni prima era infatti iniziato un conflitto
    tra il conte Orlando e suo cugino Rinaldo,
    poiché entrambi, per la rara bellezza di Angelica,
    avevano l'animo infiammato dal desiderio amoroso.
    Carlo non vedeva di buon occhio tale lite,
    che poteva mettere in dubbio il loro aiuto,
    questa fanciulla (Angelica), che ne era la causa,
    prese e consegno nelle mani del duca Namo di Baviera;}}
    \\\\
    In questa ottava viene spiegato la metafora dell'incendio.
    Il motivo è che il conte Orlando e suo cugino Rinaldo sono erano
    innamorati di Angelica.
    Allora, re Carlo, temendo una riduzione dell'efficienza dei due guerrieri più bravi,
    sottrae la donna per non farli distrarre.
    La donna viene data a Namo, il duca di Bavera, per custodirla.
\end{snippet}

\begin{snippet}{orlando-furioso-ottava-9-proemio}
    \StellarPoetry{9}{
        in premio promettendola a quel d'essi,\\
        ch'in quel conflitto, in quella gran giornata,\\
        degl'infideli più copia uccidessi,\\
        e di sua man prestasse opra più grata.\\
        Contrari ai voti poi furo i successi;\\
        ch'in fuga andò la gente battezzata,\\
        e con molti altri fu 'l duca prigione,\\
        e restò abbandonato il padiglione.
    }{\parbox[t]{0.4\textwidth}{promettendola in premio a chi dei due,
    nell'imminente conflitto, in quella battaglia campale,
    avesse ucciso il maggior numero di infedeli,
    e con la sua mano avesse quindi reso maggior servizio.
    Gli eventi fecero però venire meno le promesse;
    perché i cristiani dovettero ritirarsi,
    insieme a molti altri, il duca Namo fu fatto prigioniero
    e la sua tenda rimase vuota (Angelica rimase incustodita).}}
    \\\\
    La donna viene promessa a chi fra Orlando e il cugino farà più morti in questa battaglia
    - premio per chi fa più morte.
    Questa ottava è bipartita perché, nella sua seconda parte,
    in modo tutto imprevedibile, i cristiani (gente battezzata)
    perdono la battaglia e si devono ritirare.
    Namo viene imprigionato e non può più custodire Angelica, per cui rimane sola.

    Qui termina l'Orlando innamorato.
    Dalla prossima ottava la storia è tutta un'invenzione.
\end{snippet}

\begin{snippet}{orlando-furioso-ottava-10-proemio}
    \StellarPoetry{10}{
        Dove, poi che rimase la donzella\\
        ch'esser dovea del vincitor mercede,\\
        inanzi al caso era salita in sella,\\
        e quando bisognò le spalle diede,\\
        presaga che quel giorno esser rubella\\
        dovea Fortuna alla cristiana fede:\\
        entrò in un bosco, e ne la stretta via\\
        rincontrò un cavallier ch'a piè venìa.
    }{\parbox[t]{0.4\textwidth}{Rimasta sola nella tenda, la donzella,
    che avrebbe dovuto essere la ricompensa del vincitore,
    visto l'andamento degli eventi, salì in sella ad un cavallo
    e ad momento opportuno scappò,
    avuto presagio che, quel giorno, avversa
    alla fede cristiana sarebbe stata la fortuna.
    Entrò in un bosco e per lo stretto sentiero
    incontrò un cavaliere che avanzava a piedi.}}
    \\\\
    Angelica, ancora prima dell'esito della battaglia, aveva intuito che sarebbe andata male per i cristiani,
    e si era preparata a scappare.
    Non perde tempo e scappa a cavallo. Incontra un cavaliere a piedi.
\end{snippet}

\begin{snippetnote}{labirinti-ariosteschi}{Labirinti ariosteschi}
    Un tipico elemento di Ariosto
    è il luogo di una selva, una stretta via labirintica
    dove, \textit{per caso}, incontra un cavaliere.
    Questo caso è il tema fondamentale di Ariosto. 
\end{snippetnote}

\begin{snippet}{orlando-furioso-ottava-11-proemio}
    \StellarPoetry{11}{
        Indosso la corazza, l'elmo in testa,\\
        la spada al fianco, e in braccio avea lo scudo;\\
        e più leggier correa per la foresta,\\
        ch'al pallio rosso il villan mezzo ignudo.\\
        Timida pastorella mai sì presta\\
        non volse piede inanzi a serpe crudo,\\
        come Angelica tosto il freno torse,\\
        che del guerrier, ch'a piè venìa, s'accorse.
    }{\parbox[t]{0.4\textwidth}{Con addosso la corazza, in testa l'elmo,
    al fianco la spada ed al braccio lo scudo,
    correva per la foresta più rapidamente
    di un contadino poco vestito in una gara di corsa.
    Una timida pastorella mai così rapidamente
    sottrasse il piede dal morso di un serpente letale,
    quanto rapidamente Angelica tirò le redini per cambiare direzione
    non appena si accorse del guerriero che sopraggiungeva a piedi.}}
    \\\\
    L'ottava è bipartita perfettamente a metà.
    Nella prima parte abbiamo la descrizione del cavaliere,
    mentre la seconda è la reazione di Angelica.
    La prima parte può ancora essere suddivisa a metà, perché i primi
    due descrviono l'aspetto fisico, mentre gli altri due parlano di 
    come il cavaliere si muovesse: più rapido di
    chi un contadino che sta partecipando ad una gara
    dove bisognasse inseguire un panno e prenderlo.
    Nei primi due versioni abbiamo un chiasmo doppio fra la parte del corpo
    e l'arma/oggetto.
    \\ Appena Angelica vede il cavaliero, si ferma con una rapidità
    maggiore di una timida pastorella che si scansa quando si trova un
    serpente in mezzo ai piedi.
    Angelica ha quindi una reazione terrorizzata.
\end{snippet}

\begin{snippet}{orlando-furioso-ottava-12-proemio}
    \StellarPoetry{12}{
        Era costui quel paladin gagliardo,\\
        figliuol d'Amon, signor di Montalbano,\\
        a cui pur dianzi il suo destrier Baiardo\\
        per strano caso uscito era di mano.\\
        Come alla donna egli drizzò lo sguardo,\\
        riconobbe, quantunque di lontano,\\
        l'\textbf{angelico sembiante} e quel bel volto\\
        ch'all'amorose reti il tenea involto.
    }{\parbox[t]{0.4\textwidth}{Era questo guerriero (Rinaldo) quel paladino,
    figlio di Amone, signore di Montauban,
    al quale poco prima il proprio destriero
    per uno strano caso era fuggito di mano.
    Non appena posò lo sguardo sulla donna,
    riconobbe, nonostante fosse lontana,
    l'angelica figura ed il bel volto
    che lo avevano fatto prigioniero delle reti dell'amore.}}
    \\\\
    Il cavaliere era il figlio del duca Amone, Rinaldo, uno dei cugini,
    solo adesso viene svelato la sua identità.
    Rinaldo sta inseguendo il cavallo Baiardo che era scappato per sbaglio.
    Questa storia è stata raccontata nell'Orlando innamorato.
    Abbiamo un'allusione al nome Angela mediante la sembianza angelica, e la visione
    delle reti come catturato dall'amore.
\end{snippet}

\begin{snippet}{orlando-furioso-ottava-13-proemio}
    \StellarPoetry{13}{
        La donna il palafreno a dietro volta,\\
        e per la selva a tutta briglia il caccia;\\
        né per la rara più che per la folta,\\
        la più sicura e miglior via procaccia:\\
        ma pallida, tremando, e di sé tolta,\\
        uderline{lascia cura al destrier che la via faccia.}\\
        \underline{Di sù di giù}, ne l'alta selva fiera\\
        \underline{tanto girò}, che venne a una riviera.
    }{\parbox[t]{0.4\textwidth}{La donna volta indietro il cavallo
    e per il bosco lo lancia in corsa a briglia sciolta;
    più per la rada (sgombra) che per la fitta boscaglia
    non va cercando la via migliore e più sicura,
    perché pallida, tremante, e fuori di sé,
    lascia che sia il cavallo a frasi strada da solo.
    L'animale da ogni parte, nell'inospitale foresta,
    tanto vagò che infine giunse alla riva di un fiume.}}
    \\\\
    Angelica non sceglie la strada, ma lascia che il cavallo la scelga al posto suo,
    fino ad arrivare ad un fiume.
    Il movimento casuale viene indicato da \underline{diverse espressioni}.
\end{snippet}

\begin{snippet}{orlando-furioso-ottava-14-proemio}
    \StellarPoetry{14}{
        Su la riviera Ferraù trovosse\\
        di sudor pieno e tutto polveroso.\\
        Da la battaglia dianzi lo rimosse\\
        un gran disio di bere e di riposo;\\
        e poi, mal grado suo, quivi fermosse,\\
        perché, de l'acqua ingordo e frettoloso,\\
        l'elmo nel fiume si lasciò cadere,\\
        né l'avea potuto anco riavere
    }{\parbox[t]{0.4\textwidth}{In riva al fiume trovò Ferraù
    tutto impolverato e sudato.
    Poco prima lo aveva tolto dalla battaglia
    una grande desiderio di bere di riposarsi;
    e poi, contro la sua volontà, lì si dovette fermare ,
    perché, nella fretta di bere,
    lasciò cadere nel fiume il proprio elmo
    ed ancora non era riuscito a ritrovarlo.}}
    \\\\
    Nell'Orlando innamorato, Angelica e Rinaldo avevano bevuto da delle fontane magiche:
    Angelica da quella che fa innamorare, per cui si era innamorata di Rinaldo,
    mentre Rinaldo da quella che fa odiare, per cui odiava Angelica.
    Successivamente, i due bevettero dalle fontane opposte. Il motivo per cui
    Angelica ha questa reazione è quindi perché odia e prova ribrezzo
    per Rinaldo.

    Sulla riviera trovò Ferraù, un guerriero musulmano.
    Questo guerriera, pensava che si sarebbe fermato a bere dell'acqua e a riposare,
    ma dalla sua sete il suo elmo era caduto nel fiume, e non l'aveva ancora
    recuperato.
\end{snippet}

\begin{snippetnote}{ariosto-personaggi-movimento}{Movimento personaggi}
    Tutti i personaggi stanno o scappando o sono alla ricerca di qualcosa.
\end{snippetnote}

\begin{snippet}{orlando-furioso-ottava-15-proemio}
    \StellarPoetry{15}{
        Quanto potea più forte, ne veniva\\
        gridando la donzella ispaventata.\\
        A quella voce salta in su la riva\\
        il Saracino, e nel viso la guata;\\
        e la conosce subito ch'arriva,\\
        ben che di timor pallida e turbata,\\
        e sien più dì che non n'udì novella,\\
        che senza dubbio ell'è Angelica bella.
    }{\parbox[t]{0.4\textwidth}{Sopraggiunse, gridando quanto più poteva
    la donzella spaventata.
    Udita la voce, il Saracino salta sulla riva
    la guarda attentamente in viso
    e subito riconosce che chi sta arrivando arriva al fiume,
    nonostante fosse pallida e turbata dalla paura
    e fossero passati più giorni dall'ultima volta che ne ebbe notizia,
    era senza dubbio la bella Angelica.}}
    \\\\
    Il guerrieri riconosce Angelica nonostante fosse pallida 
    e terrorizzata e non avesse avuto notizie su di lei.
    Anche il guerriero è innamorato di lei.
\end{snippet}

\begin{snippet}{orlando-furioso-ottava-16-proemio}
    \StellarPoetry{16}{
        E perché era cortese, e n'avea forse\\
        non men de' dui cugini il petto caldo,\\
        l'aiuto che potea tutto le porse,\\
        pur come avesse l'elmo, ardito e baldo:\\
        trasse la spada, e minacciando corse\\
        dove poco di lui temea Rinaldo.\\
        Più volte s'eran già non pur veduti,\\
        m'al paragon de l'arme conosciuti.
    }{\parbox[t]{0.4\textwidth}{Essendo di indole gentile e forse avendo
    anche l'animo infiammato non meno dei due cugini,
    porse a lei tutto l'aiuto che era in grado di dare,
    come se avesse riavuto l'elmo, temerario e spavaldo:
    sguainò la spada e corse minaccioso
    verso Rinaldo, che in realtà non era per niente intimorito da lui.
    Più volte si era già non solo visti
    ma anche scontrati con le armi.}}
    \\\\
    Il guerriere le porge tutto il suo aiuto, nonostante non avesse l'elmo.
    Prende la spada per difenderla da chiunque stesse arrivando.
    Più volte i due si erano già visti e si erano anche combattuti (Orlando innamorato).
    Ferraù ha questa reazione istintiva per la sua cortesia (possiede i valori cavallereschi,
    come difenderela donzella perseguitata).
\end{snippet}

\begin{snippet}{orlando-furioso-ottava-17-proemio}
    \StellarPoetry{17}{
        Cominciar quivi una crudel battaglia,\\
        come a piè si trovar, coi brandi ignudi:\\
        non che le piastre e la minuta maglia,\\
        ma ai colpi lor non reggerian gl'incudi.\\
        Or, mentre l'un con l'altro si travaglia,\\
        bisogna al palafren che 'l passo studi;\\
        che quanto può menar de le calcagna,\\
        colei lo caccia al bosco e alla campagna.
    }{\parbox[t]{0.4\textwidth}{Cominciò lì una battaglia crudele,
    a piedi, come si trovavano entrambi, con le spade sguainate,
    Non solo le piastre della corazza e la maglia di ferro
    ma neanche gli scudi reggevano ai loro colpi.
    Ora, mentre l'uno si occupa affannosamente dell'altro,
    il destriero di Angelica è costretto ad affrettare il passo,
    perché con quanta forza riesce a spronarlo,
    la donna lo spinge a correre per il bosco e l'aperta campagna.}}
    \\\\
    L'ottava è bipartita. La prima parte è dedicata al duello (ad armi pari).
    i colpi che si davano, non solo sfondavano le piastre,
    ma avrebbero spezzato anche le incudini (iperbole).
    La seconda parte parla dell'esito del duello.
    Mentre i due si stanno ammazzando, Angelica scappa con il suo cavallo.
\end{snippet}

\begin{snippet}{orlando-furioso-ottava-18-proemio}
    \StellarPoetry{18}{
        Poi che s'affaticar gran pezzo invano\\
        i dui guerrier per por l'un l'altro sotto,\\
        quando non meno era con l'arme in mano\\
        questo di quel, né quel di questo dotto;\\
        fu primiero il signor di Montalbano,\\
        ch'al cavallier di Spagna fece motto,\\
        sì come quel ch'ha nel cuor tanto fuoco,\\
        che tutto n'arde e non ritrova loco.
    }{\parbox[t]{0.4\textwidth}{Dopo che si furono affaticati invano
    i due cavalieri nel tentativo ognuno di fare soccombere l'altro,
    in quanto, con la spada in mano, non
    meno istruito, capace, era l'uno dell'altro;
    fu per primo il signore di Montauban
    a rivolgersi al cavaliere spagnolo,
    così come colui ha in petto, nel cuore, tanto fuoco
    che lo fa ardere tutto senza trovare pace.}}
    \\\\
    Entrambi sono abilissimi guerrieri, e nessuno dei due riesce a sopraffare l'altro.
\end{snippet}

\begin{snippet}{orlando-furioso-ottava-19-proemio}
    \StellarPoetry{19}{
        Disse al pagan: - Me sol creduto avrai,\\
        e pur avrai te meco ancora offeso:\\
        se questo avvien perché i fulgenti rai\\
        del nuovo \textbf{sol} t'abbino il petto acceso,\\
        di farmi qui tardar che guadagno hai?\\
        che quando ancor tu m'abbi morto o preso,\\
        non però tua la bella donna fia;\\
        che, mentre noi tardiam, se ne va via.
    }{\parbox[t]{0.4\textwidth}{Disse al pagano: “Avrai creduto me solo
    di ferire quando invece ferisci anche te stesso,
    se questo accade perché la sfavillante bellezza
    di Angelica ha acceso d'amore anche il tuo petto,
    che cosa guadagni facendomi perdere tempo qui?
    Che anche se tu mi catturi o mi uccidi
    non riuscirai a fare tua la bella donna,
    da momento che, mentre noi ci attardiamo, lei scappa via.}}
    \\\\
    Rinaldo parla al guerriero musulmano alludendo all'inutilità del duello: entrambi si stanno danneggiando.
    Angelica è scappata via, e anche se Ferraù uccide Rinaldo, la donna non sarà sua.
    Angelica non viene nominata direttamente ma viene citata con
    una descrizione stillnovistica, ossia quella della donna come raggi solari.
\end{snippet}

\begin{snippet}{orlando-furioso-ottava-20-proemio}
    \StellarPoetry{20}{
        Quanto fia meglio, amandola tu ancora,\\
        che tu le venga a traversar la strada,\\
        a ritenerla e farle far dimora,\\
        prima che più lontana se ne vada!\\
        Come l'avremo in potestate, allora\\
        di chi esser de' si provi con la spada:\\
        non so altrimenti, dopo un lungo affanno,\\
        che possa riuscirci altro che danno. -
    }{\parbox[t]{0.4\textwidth}{Quanto sarebbe meglio, poiché ancora la ami,
    che tu le vada invece ad incrociarne la strada
    a trattenerla e farla fermare,
    prima che ancora più lontano scappi!
    Appena ne avremo il possesso, allora
    a chi dei due avrà appartenere verrà poi deciso con la spada:
    non so altrimenti, dopo una lungo e faticoso combattimento,
    cosa riusciamo ad ottenere se non un danno.”}}
    \\\\
    Per terminare il discorso viene fatta una proposta pragmatica:
    quella di risparmiare energia, raggiungere Angelica, bloccandola (farle far dimora),
    e poi risumendo il duello.
\end{snippet}

\begin{snippet}{orlando-furioso-ottava-21-proemio}
    \StellarPoetry{21}{
        Al pagan la proposta non dispiacque:\\
        così fu differita la tenzone;\\
        e tal tregua tra lor subito nacque,\\
        sì l'odio e l'ira va in oblivione,\\
        che 'l pagano al partir da le fresche acque\\
        non lasciò a piedi il buon figliuol d'Amone:\\
        con preghi invita, ed al fin toglie in groppa,\\
        e per l'orme d'Angelica galoppa.
    }{\parbox[t]{0.4\textwidth}{Al pagano (Ferraù) la proposta piacque:
    così il duello fu rimandato
    e la tregua proposta fu subito fra loro attuata;
    tanto l'odio e l'ira vengono dimenticati,
    che il pagano nel partire dalle fresche acque del fiume
    non lasciò a piedi il buon figlio di Amone:
    lo preghiere lo invita ed alla fine lo fa montare a cavallo
    ed all'inseguimento di Angelica galoppa.}}
    \\\\
    I due si dimenticano completamente dell'odio e dell'ira di pochi
    minuti prima, e Ferraù, dopo aver accetato, offre un passaggio a cavallo.
    Questo passaggio è dato dal fatto che entrambi posseggano i valori cavallereschi,
    in particolare quello di avere armi pari. In questo caso,
    il passaggio viene offerto affinché uno dei due non sia svantaggiato
    rimandendo a piedi.
\end{snippet}

\begin{snippet}{orlando-furioso-ottava-22-proemio}
    \StellarPoetry{22}{
        Oh gran bontà de' cavallieri antiqui!\\
        Eran rivali, eran di fé diversi,\\
        e si sentian degli aspri colpi iniqui\\
        per tutta la persona anco dolersi;\\
        e pur per selve oscure e calli obliqui\\
        insieme van senza sospetto aversi.\\
        Da quattro sproni il destrier punto arriva\\
        ove una strada in due si dipartiva.
    }{\parbox[t]{0.4\textwidth}{Oh bontà dei cavalieri antichi!
    Erano rivali, parlavano una diversa lingua,
    si sentivano dei duri colpi crudeli
    ancora dolere tutto il corpo;
    eppure per boschi oscuri e sentieri tortuosi
    vanno insieme senza temersi tra loro.
    Da quattro speroni punto, il destriero arriva
    ad un bivio.}}
    \\\\
    I due cavallieri sono rivali in amore, di diversa fede, e sono ancora
    feriti dai colpi appena subiti. Nonostante ciò, i due si trovano sul medesimo cavallo.
    I valori cavallereschi sono quindi più forti dei motivi per i quali
    potrebbero continuare a duellare. Nessuno dei due teme di essere colpito alle spalle.
\end{snippet}

\begin{snippet}{orlando-furioso-ottava-23-proemio}
    \StellarPoetry{23}{
        E come quei che non sapean se l'una\\
        o l'altra via facesse la donzella\\
        (però che senza differenza alcuna\\
        apparia in amendue l'orma novella),\\
        si messero ad arbitrio di fortuna,\\
        Rinaldo a questa, il Saracino a quella.\\
        Pel bosco Ferraù molto s'avvolse,\\
        e ritrovossi al fine onde si tolse.
    }{\parbox[t]{0.4\textwidth}{E come quelli che non sapevano se l'una
    l'altra via avesse imboccato la donzella
    (poiché senza alcuna differenza,
    su entrambi i sentieri l'impronta appariva fresca, recente)
    misero la propria sorte nelle mani della fortuna.
    Rinaldo per questo sentiero, il saracino per quello.
    Per il bosco Ferraù molto s'aggirò
    ad alla fine si ritrovò al punto di partenza.}}
    \\\\
    I due trovano un bivio con orme fresche da ambo el parti,
    di conseguenza i due si separano.
    Ferraù si avvolge nel bosco fino a ritorna al fiume di partenza. 
\end{snippet}

\begin{snippetnote}{ricerca-casuale-orlando-furioso}{Ricerca casuale}
    La ricerca che i personaggi affrontano è casuale,
    priva di indizi, senza tracce, basata sulle fortuna,
    senza fine e quindi senza una direzione precisa.
\end{snippetnote}

\begin{snippet}{orlando-furioso-ottava-24-proemio}
    \StellarPoetry{24}{
        Pur si ritrova ancor su la rivera,\\
        là dove l'elmo gli cascò ne l'onde.\\
        Poi che la donna ritrovar non spera,\\
        per aver l'elmo che 'l fiume gli asconde,\\
        in quella parte onde caduto gli era\\
        discende ne l'estreme umide sponde:\\
        ma quello era sì fitto ne la sabbia,\\
        che molto avrà da far prima che l'abbia.
    }{\parbox[t]{0.4\textwidth}{Viene a ritrovarsi infine ancora sulla riva del fiume,
    là dove l'elmo gli cascò dalla testa tra le onde.
    Poiché non ha più speranze di ritrovare la donna,
    per riavere l'elmo che il fiume gli nasconde,
    dalla parte dove gli era caduto
    scende fino alle estreme umide sponde:
    ma l'elmo era così ben nascosto nella sabbia
    che dovrà operare molto prima di poterlo riavere.}}
    \\\\
    Ferraù, immediatamente, torna a ricercare il suo elmo, come se tutta la vicenda
    centrale di Angelica fosse sparita.
\end{snippet}

\begin{snippet}{orlando-furioso-ottava-25-proemio}
    \StellarPoetry{25}{
        Con un gran ramo d'albero rimondo,\\
        di ch'avea fatto una pertica lunga,\\
        tenta il fiume e ricerca sino al fondo,\\
        né loco lascia ove non batta e punga.\\
        Mentre con la maggior stizza del mondo\\
        tanto l'indugio suo quivi prolunga,\\
        vede di mezzo il fiume un cavalliero\\
        insino al petto uscir, d'aspetto fiero.
    }{\parbox[t]{0.4\textwidth}{Con un lungo ramo d'albero ripulito da rami e foglie,
    con il quale si era costruito una lunga pertica,
    sonda il fiume e cerca fino sul fondo,
    battendo e pungendo con la punto in tutti i punti del fiume.
    Mentre con un enorme risentimento, stizza,
    prolunga oltre la sua permanenza in quel luogo,
    vede in mezzo il fiume un cavaliere
    uscire dall'acqua fino al petto, di aspetto fiero.}}
    \\\\
    Prendendo un ramo, crea un bastone e lo usa per tastare il fondo del fiume.
    Mentre svolge questo lavoro in maniera metodica e ossessiva,
    sbuca un cavaliere che emerge fino al petto dal fiume, con un aspetto arrabiato.
\end{snippet}

\begin{snippetdefinition}{stizza-definition}{Stizza}
    Viva irritazione, per lo più momentanea, provocata da un senso di fastidio o di molestia.
\end{snippetdefinition}

\begin{snippet}{orlando-furioso-ottava-26-proemio}
    \StellarPoetry{26}{
        Era, fuor che la testa, tutto armato,\\
        ed avea un elmo ne la destra mano:\\
        avea il medesimo elmo che cercato\\
        da Ferraù fu lungamente invano.\\
        A Ferraù parlò come adirato,\\
        e disse: - Ah \textbf{mancator di fé}, marano!\\
        perché di lasciar l'elmo anche t'aggrevi,\\
        che render già gran tempo mi dovevi?
    }{\parbox[t]{0.4\textwidth}{Era, ad eccezione della testa, completamente armato,
    ed aveva una elmo nella mano destra:
    aveva in particolare lo stesso elmo che aveva cercato
    Ferraù invano per così tanto tempo.
    Il cavaliere si rivolse a Ferraù in tono adirato,
    disse: “Ah traditore che non mantiene la parola data!
    Perché ti dispiace anche di abbandonare l'elmo,
    che invece mi avresti dovuto rendere già da tanto tempo?}}
    \\\\
    L'ottava è bipartita in quattro versi di descrizione e quattro di dialogo.
    Dalle parole del cavaliere si intuisce che i due si conoscono, e Ferraù
    viene disprezzato per essere stato sleato.
\end{snippet}

\begin{snippet}{orlando-furioso-ottava-27-proemio}
    \StellarPoetry{27}{
        Ricordati, pagan, quando uccidesti\\
        d'Angelica il fratel (che son quell'io),\\
        dietro all'altr'arme tu mi promettesti\\
        gittar fra pochi dì l'elmo nel rio.\\
        Or se Fortuna (quel che non volesti\\
        far tu) pone ad effetto il voler mio,\\
        non ti \textbf{turbare}; e se \textbf{turbar} ti déi,\\
        \textbf{turbati} che di fé mancato sei.
    }{\parbox[t]{0.4\textwidth}{Ricordati, pagano, di quanto hai ucciso
    il fratello di Angelico, sono io quello (Argalia),
    insieme alle altre armi tu mi promettesti
    di gettare entro pochi giorni anche il mio elmo.
    Ora, se la fortuna (quello che non hai voluto
    fare tu) ha poi voluto che si realizzasse il mio volere,
    non ti devi dispiacere; e se anzi ti devi dispiacere,
    devi solo dispiacerti di non avere mantenuto la parola data.}}
    \\\\
    Il cavaliere che sta parlando è un fantasma, ossia il fratello morte di Angelica,
    ucciso da Ferraù.
\end{snippet}

\begin{snippet}{orlando-furioso-ottava-28-proemio}
    \StellarPoetry{28}{
        Ma se desir pur hai d'un elmo fino,\\
        trovane un altro, ed abbil con più onore;\\
        un tal ne porta Orlando paladino,\\
        un tal Rinaldo, e forse anco migliore:\\
        l'un fu d'Almonte, e l'altro di Mambrino:\\
        acquista un di quei dui col tuo valore;\\
        e questo, ch'hai già di lasciarmi detto,\\
        farai bene a lasciarmi con effetto. -
    }{\parbox[t]{0.4\textwidth}{Ma se desideri ancora un buon elmo,
    trovane un altro e portalo con te con più onore;
    uno di buona fattura lo porta il paladino Orlando,
    un altro Rinaldo, forse anche migliore di quello d'Orlando:
    prima uno apparteneva ad Almonte e l'altro a Mambrino:
    conquistane uno dei due con il tuo valore,
    questo invece, che avevi già promesso di lasciarmi,
    farai bene a lasciarmelo effettivamente.”}}
    \\\\
    Il cavaliere continua suggerendo di trovarsi un altro elmo, con più orgoglio,
    piuttosto che fare il vigliacco, e suggerisce anche alcune persone dalle quali prenderlo.
\end{snippet}

\begin{snippet}{orlando-furioso-ottava-29-proemio}
    \StellarPoetry{29}{
        All'apparir che fece all'improvviso\\
        de l'acqua l'ombra, ogni pelo arricciossi,\\
        e scolorossi al Saracino il viso;\\
        la voce, ch'era per uscir, fermossi.\\
        Udendo poi da l'Argalia, ch'ucciso\\
        quivi avea già (che l'Argalia nomossi)\\
        la rotta fede così improverarse,\\
        di scorno e d'ira dentro e di fuor arse.
    }{\parbox[t]{0.4\textwidth}{Non appena, all'improvviso, appare
    dall'acqua il fantasma, si rizzo ogni pelo
    del Saracino ed il viso gli si fece scolorito;
    la voce gli si strozzò in gola.
    Udendo poi da Argalia, che ucciso
    lui aveva (perché Argalia si chiamava),
    rimproverare a sé stesso di non aver mantenuto la parola data,
    di scocciatura e di ira si accese tutto, dentro e fuori.}}
    \\\\
    Dopo le parole, viene data la reazione di Ferraù, il quale rimane pietrificato di paura.
    Ferraù sa di essere stato sleale, e se ne vergogna.
\end{snippet}

\begin{snippet}{orlando-furioso-ottava-30-proemio}
    \StellarPoetry{30}{
        Né tempo avendo a pensar altra scusa,\\
        e conoscendo ben che 'l ver gli disse,\\
        restò senza risposta a bocca chiusa;\\
        ma la vergogna il cor sì gli trafisse,\\
        che giurò per la vita di Lanfusa\\
        non voler mai ch'altro elmo lo coprisse,\\
        se non quel buono che già in Aspramonte\\
        trasse dal capo Orlando al fiero Almonte.
    }{\parbox[t]{0.4\textwidth}{Non avendo tempo per cercare una altra scusa,
    sapendo benissimo che Argalia diceva il vero,
    Ferraù rimase a bozza chiusa, senza controbattere;
    ma il suo cuore fu talmente trafitto dalla vergogna,
    che giurò sulla vita di sua madre (Lanfusa)
    non volere indossare più nessun altro elmo
    se non quello di buona fattura che nell'Aspromonte
    Orlando levò dal capo di Almonte (dopo averlo ucciso).}}
    \\\\
    Ferraù giura sulla propria madre di non volere altro elmo oltre quello di Orlando,
    e parte per la ricerca del suo elmo.
\end{snippet}

\begin{snippet}{orlando-furioso-ottava-31-proemio}
    \StellarPoetry{31}{
        E servò meglio questo giuramento,\\
        che non avea quell'altro fatto prima.\\
        Quindi si parte tanto malcontento,\\
        che molti giorni poi si rode e lima.\\
        Sol di cercare è il paladino intento\\
        di qua di là, dove trovarlo stima.\\
        Altra ventura al buon Rinaldo accade,\\
        che da costui tenea diverse strade.
    }{\parbox[t]{0.4\textwidth}{E mantenne questo giuramento meglio
    di quanto non aveva fatto con quell'altro prima.
    Ripartì dal fiume con tanto malcontento
    che per molti successivi giorni si tormentò e consumò.
    Ha voglia solo di cercare il Paladino (Orlando)
    in ogni luogo dove ritiene possa trovarlo.
    Avventura diversa accadde al valoroso Rinaldo
    che si incamminò su sentieri diversi da quelli percorsi da costui.}}
    \\\\
    La \textit{ricerca}, casuale e consumante, va avanti per molti giorni.
    La vicenda di Ferraù si chiude nei primi sei versi.
    Attraverso la tecnica dell'entrelacement si torna alla vicenda di Ronaldo.
\end{snippet}

\begin{snippet}{orlando-furioso-ottava-32-proemio}
    \StellarPoetry{32}{
        Non molto va Rinaldo, che si vede\\
        saltare inanzi il suo destrier feroce:\\
        - Ferma, Baiardo mio, deh, ferma il piede!\\
        che l'esser senza te troppo mi nuoce. -\\
        Per questo il destrier sordo, a lui non riede\\
        anzi più se ne va sempre veloce.\\
        Segue Rinaldo, e d'ira si distrugge:\\
        ma seguitiamo Angelica che fugge.
    }{\parbox[t]{0.4\textwidth}{Rinaldo non fa molta strada che vede
    comparire davanti a sé il proprio focoso destriero:
    “Fermati, Boiardo mio, dai, arresta il galoppo!
    Perché stare senza di te è per me troppo pericoloso.”
    Non per questo il cavallo, sordo ai richiami, torna da lui,
    anzi si allontana veloce sempre di più.
    Rinaldo lo segue, tormentandosi d'ira:
    ma seguiamo ora Angelica in fuga.}}
    \\\\
    Anche qui, con la tecnica dell'entrelacement, l'ottava viene suddivisa
    cambiando la storia.
    Rinaldo, per caso, ritrova il suo cavallo che stava cercando all'inizio
    della sua prima apparizione.
    Rinaldo ricomincia ad inseguirlo pieno di rabbia.
    I verbi sono tutti legati all'idea di movimento.
\end{snippet}

\begin{snippet}{orlando-furioso-ottava-33-proemio}
    \StellarPoetry{33}{
        Fugge tra selve spaventose e scure,\\
        per lochi inabitati, ermi e selvaggi.\\
        Il mover de le frondi e di verzure,\\
        che di cerri sentia, d'olmi e di faggi,\\
        fatto le avea con subite paure\\
        trovar di qua di là strani viaggi;\\
        ch'ad ogni ombra veduta o in monte o in valle,\\
        temea Rinaldo aver sempre alle spalle.
    }{\parbox[t]{0.4\textwidth}{Fugge tra spaventosi ed oscuri boschi,
    per luoghi inabitati, selvaggi e solitari.
    Il rumore provocato dal movimento dei rami e dalla vegetazione
    di querce, olmi e faggi, che Angelica sentiva,
    causa le improvvise paure, le avevano
    fatto intraprendere insoliti sentieri da ogni parte;
    perché ogni ombra che vedeva sui monti o nelle valli,
    le facevano temere di avere ancora alle spalle Rinaldo.}}
    \\\\
    La descrizione della selva è come quella Dantesca.
    Ogni votla che Angelica sente un rumore, cambia strada data la sua paranoia.
\end{snippet}

\begin{snippet}{orlando-furioso-ottava-34-proemio}
    \StellarPoetry{34}{
        Qual pargoletta o damma o capriuola,\\
        che tra le fronde del natio boschetto\\
        alla madre veduta abbia la gola\\
        stringer dal pardo, o aprirle 'l fianco o 'l petto,\\
        di selva in selva dal crudel s'invola,\\
        e di paura trema e di sospetto:\\
        ad ogni sterpo che passando tocca,\\
        esser si crede all'empia fera in bocca.
    }{\parbox[t]{0.4\textwidth}{Come un cucciolo di daino o capriolo,
    che tra i rami del boschetto nel quale è nato
    abbia visto la gola della madre dal morso
    del leopardo stretta, o che le squarcia il petto od il fianco,
    scappa dall'animale crudele di bosco in bosco
    e trema per la paura e per il sospetto della sua presenza:
    per ogni cespuglio che tocca al proprio passaggio
    crede di essere già già in bocca alla belva crudele.}}
    \\\\
    Il sentimento di Angelica è esattamente come quello di una damma (femmina del daino) o una capriola,
    che si ritrova da sola, perché ha assistito all'omicidio della madre,
    e che quindi fugge temendo di far la stessa fine.
\end{snippet}

\begin{snippet}{orlando-furioso-ottava-35-proemio}
    \StellarPoetry{35}{
        Quel dì e la notte a mezzo l'altro giorno\\
        s'andò aggirando, e non sapeva dove.\\
        Trovossi al fin in un boschetto adorno,\\
        che lievemente la fresca aura muove.\\
        Duo chiari rivi, mormorando intorno,\\
        sempre l'erbe vi fan tenere e nuove;\\
        e rendea ad ascoltar dolce concento,\\
        rotto tra picciol sassi, il correr lento.
    }{\parbox[t]{0.4\textwidth}{Quel giorno, la stessa notte e per metà del giorno seguente
    vagò senza sapere dove stesse andando.
    Venne a trovarsi infine in un boschetto leggiadro,
    mosso delicatamente da un vento fresco.
    Due ruscelli trasparenti, riempiendo l'aria del loro gorgoglio,
    consentono la presenza sempre dell'erba e la sua crescita;
    e rendevano piacevole da ascoltare il concerto,
    interrotto solo tra piccoli sassi, del loro scorrere lento.}}
    \\\\
    I primi due versi indicano la fuga frenetica, mentre i restanti sei indica il luogo.
    Il luogo descritto è un locus amoenus. 
\end{snippet}

\begin{snippet}{orlando-furioso-ottava-36-proemio}
    \StellarPoetry{36}{
        Quivi parendo a lei d'esser sicura\\
        e lontana a Rinaldo mille miglia,\\
        da la via stanca e da l'estiva arsura,\\
        di riposare alquanto si consiglia:\\
        tra' fiori smonta, e lascia alla pastura\\
        andare il palafren senza la briglia;\\
        e quel va errando intorno alle chiare onde,\\
        che di fresca erba avean piene le sponde.
    }{\parbox[t]{0.4\textwidth}{Qui, credendo di essere al sicuro
    e lontana mille miglia da Rinaldo,
    per lo stancante tragitto ed il caldo estivo
    decide di riposare per un po' tempo:
    scende da cavallo tra i fiori e lascia andare a nutrirsi,
    senza briglia, libero, il proprio destriero;
    l'animale vaga quindi nei dintorni dei ruscelli,
    che avevano piene le rive di fresca erba.}}
    \\\\
    Angelica si calma, pensando di aver seminato Rinaldo, si riposa e lascia pascolare
    il proprio cavallo.
\end{snippet}

\begin{snippet}{orlando-furioso-ottava-37-proemio}
    \StellarPoetry{37}{
        Ecco non lungi un bel cespuglio vede\\
        di prun fioriti e di vermiglie rose,\\
        che de le liquide onde al specchio siede,\\
        chiuso dal sol fra l'alte querce ombrose;\\
        così voto nel mezzo, che concede\\
        fresca stanza fra l'ombre più nascose:\\
        e la foglia coi rami in modo è mista,\\
        che 'l sol non v'entra, non che minor vista.
    }{\parbox[t]{0.4\textwidth}{Non lontano da sé Angelica scorge un bel cespuglio,
    fiorito di susine e di rose rosse,
    che si specchia nelle onde limpide dei ruscelli
    ed è riparato dal sole dalle alte querce ombrose;
    vuoto nel mezzo, così da concedere
    fresco giaciglio tra le ombre più nascoste:
    le sue foglie ed i suoi rami sono talmente intrecciati che non
    passa il sole, e nemmeno la vista dell'uomo, meno penetrante.}}
    \\\\
    La natura ha creato come un cespuglio vuoto al suo interno, al suo quale 
    si può entrare ma dove non entra quasi del tutto la luce, per cui un rifugio naturale.
\end{snippet}

\begin{snippet}{orlando-furioso-ottava-38-proemio}
    \StellarPoetry{38}{
        Dentro letto vi fan tenere erbette,\\
        ch'invitano a posar chi s'appresenta.\\
        La bella donna in mezzo a quel si mette,\\
        ivi si corca ed ivi s'addormenta.\\
        Ma non per lungo spazio così stette,\\
        che un calpestio le par che venir senta:\\
        cheta si leva e appresso alla riviera\\
        vede ch'armato un cavallier giunt'era.
    }{\parbox[t]{0.4\textwidth}{L'erbetta morbida crea un letto all'interno del cespuglio,
    invitando a stendersi sopra chi vi giunge.
    La bella donna si mette in mezzo al cespuglio,
    lì si corica e quindi si addormenta.
    Ma non rimane lì addormentata molto tempo,
    che le sembra di sentire avvicinarsi un rumore di calpestio:
    si solleva piano piano e presso la riva di un ruscello
    vede essere giunto un cavaliere armato.}}
    \\\\
    Angelica si addormenta in mezzo alla natura.
    Possiamo misurare una specie di anti-climax dall'ottava 33 (terrore assoluto),
    36 (la calma del locus amoenus) e 38 (si addormenta).
\end{snippet}

\begin{snippet}{orlando-furioso-ottava-39-proemio}
    \StellarPoetry{39}{
        Se gli è \textbf{amico} o \textbf{nemico} non comprende:\\
        \textbf{tema} e \textbf{speranza} il dubbio cor le scuote;\\
        e di quella aventura il fine attende,\\
        né pur d'un sol sospir l'aria percuote.\\
        Il cavalliero in riva al fiume scende\\
        sopra l'un braccio a riposar le gote;\\
        e in un suo gran pensier tanto penètra,\\
        che par cangiato in insensibil pietra.
    }{\parbox[t]{0.4\textwidth}{Angelica non riesce a capire se gli è amico o nemico:
    il timore e la speranza le scuotono il suo cuore dubbioso;
    attende che quella avventura giunga ad un termine
    senza emettere neanche un solo sospiro.
    Il cavaliere si siede in riva al ruscello
    reggendosi la testa con un braccio;
    e viene tanto rapito dai propri pensieri, al punto che,
    immobile, sembra essersi mutato in insensibile pietra.}}
    \\\\
    I primi due versi sono pervase da un chiasmo.
    Angelica non viene vista dal cavaliere, ma riesce ad intravvederlo.
    Il cavaliere si blocca nel suo pensiero.
\end{snippet}

\begin{snippet}{orlando-furioso-ottava-40-proemio}
    \StellarPoetry{40}{
        Pensoso più d'un'ora a capo basso\\
        stette, Signore, il cavallier dolente;\\
        poi cominciò con suono afflitto e lasso\\
        a lamentarsi sì soavemente,\\
        ch'avrebbe di pietà spezzato un sasso,\\
        una tigre crudel fatta clemente.\\
        Sospirante piangea, tal ch'un ruscello\\
        parean le guance, e 'l petto un Mongibello.
    }{\parbox[t]{0.4\textwidth}{Assorto dai propri pensieri, con il capo basso, per più di un'ora
    stette, cardinale Ippolito, il cavaliere abbattuto;
    dopo di ché cominciò con un lamento afflitto e dolente
    a lamentarsi in modo tanto struggente,
    che avrebbe infranto un sasso per pietà,
    una crudele tigre fatta misericordiosa.
    Piangeva tra i sospiri, tanto che un ruscello
    sembrava scorrergli sulle guance ed il petto un vulcano infuocato.}}
    \\\\
    Il cavalliere si lamenta così soavemente che, dalla compassione, avrebbe
    pure spaccato un sasso, o reso una tigre clemente (quattro iperboloi).
    Il Mongibello è un altro nome dell'Etna.
    Anche qui è presente un chiasmo (sospiri, pianto, ruscello, Mongibello).
\end{snippet}

\begin{snippet}{orlando-furioso-ottava-41-proemio}
    \StellarPoetry{41}{
        - Pensier (dicea) che 'l cor m'agghiacci ed ardi,\\
        e causi il duol che sempre il rode e lima,\\
        che debbo far, poi ch'io son giunto tardi,\\
        e ch'altri a corre il frutto è andato prima?\\
        a pena avuto io n'ho parole e sguardi,\\
        ed altri n'ha tutta la spoglia opima.\\
        Se non ne tocca a me frutto né fiore,\\
        perché affligger per lei mi vuo' più il core?
    }{\parbox[t]{0.4\textwidth}{Diceva: “Pensiero che mi ghiaccia ed arde il cuore,
    e causa il dolore che sempre lo consuma,
    che ci posso fare se sono giunto tardi
    ed altri, arrivati prima, avevano già colto il frutto (Angelica)?
    Ho ricevuto a stento suoi sguardi e parole,
    altri hanno invece ricevuto tutto il ricco bottino.
    Se a me non spettano né il frutto né il fiore,
    perché per lei voglio ancora tormentare il mio cuore?}}
    \\\\
    Il cavaliere parla, all'insaputa della presenza di Angelica.
    Quello del verso primo è un ossimoro. Il cuore ghiacciato è una tipica immagine di Petrarca.
    La disperazione del cavaliere è quello di amare una donna sapendo che si sia
    precedentemente concessa ad un altro uomo (significato erotico sessuale esplicito).
\end{snippet}

\begin{snippet}{orlando-furioso-ottava-42-proemio}
    \StellarPoetry{42}{
        La verginella è simile alla rosa,\\
        ch'in bel giardin su la nativa spina\\
        mentre sola e sicura si riposa,\\
        né gregge né pastor se le avvicina;\\
        l'aura soave e l'alba rugiadosa,\\
        l'acqua, la terra al suo favor s'inchina:\\
        gioveni vaghi e donne inamorate\\
        amano averne e seni e tempie ornate.
    }{\parbox[t]{0.4\textwidth}{La vergine è simile ad una rosa,
    che in un bel giardino, sul rovo che l'ha generata,
    si riposa finché è sola ed al sicuro,
    e né gregge né pastore le si avvicinano;
    la brezza delicata e la rugiada del mattino,
    l'acqua e la terra si inchinano davanti al suo fascino:
    giovani amanti e donne innamorate
    amano ornarsi il collo e la testa lei, la rosa.}}
    \\\\
\end{snippet}

\begin{snippet}{orlando-furioso-ottava-43-proemio}
    \StellarPoetry{43}{
        Ma non sì tosto dal materno stelo\\
        rimossa viene e dal suo ceppo verde,\\
        che quanto avea dagli uomini e dal cielo\\
        favor, grazia e bellezza, tutto perde.\\
        La vergine che 'l fior, di che più zelo\\
        che de' begli occhi e de la vita aver de',\\
        lascia altrui corre, il pregio ch'avea inanti\\
        perde nel cor di tutti gli altri amanti.
    }{\parbox[t]{0.4\textwidth}{Ma non appena dallo stelo materno
    e dal ceppo verde del cespuglio viene staccata,
    quanto aveva per gli uomini e per il cielo
    fascino, grazia e bellezza, tutto perde.
    La vergine che il proprio fiore, del quale deve avere cura più
    che dei propri begli occhi e della propria vita,
    lascia cogliere ad altra persona, perde l'ammirazione che poco
    prima aveva nel cuore di tutti i propri amanti.}}
    \\\\
    Ma non appena la rosa viene colta, staccata dal suo stelo materno,
    perde tutta la sua bellezza poiché sfiorisce.
    Quando una donna vergine lascia cogliere a qualcuno il fiore di cui deve avere più cura,
    il pregio che aveva prima perde nel cuore di tutti gli altri amanti.
    Nonostante dovrebbe perdere interesse, il cavaliere continua ad essere
    tormentato dal pensiero di questa donna.
\end{snippet}

\begin{snippet}{orlando-furioso-ottava-44-proemio}
    \StellarPoetry{44}{
        Sia Vile agli altri, e da quel solo amata\\
        a cui di sé fece sì larga copia.\\
        Ah, Fortuna crudel, Fortuna ingrata!\\
        trionfan gli altri, e ne moro io d'inopia.\\
        Dunque esser può che non mi sia più grata?\\
        dunque io posso lasciar mia vita propia?\\
        Ah più tosto oggi manchino i dì miei,\\
        ch'io viva più, s'amar non debbo lei! -
    }{\parbox[t]{0.4\textwidth}{Diviene di scarso valore agli occhi degli altri, ed amata solo
    da colui al quale fece così grande dono di sé.
    Ah, fortuna crudele, fortuna ingiusta!
    Gli altri godono mentre io muoio di stenti.
    Non potrebbe allora essermi lei meno cara?
    Non potrei forse abbandonare la mia propria vita?
    Ah, che io muoia oggi stesso piuttosto
    che vivere più a lungo, se non dovessi amare lei!”}}
    \\\\
    Il paradosso è quello di non riuscire a smettere di amare
    perché si desidera qualcosa che non è ottenibile.
    Lui è convinto che Angelica abbia concesso la sua verginità a qualcun'altro.
\end{snippet}

%\textbf{Dal 49:}
% Il resto della storia:
%Angelica lo sdegna ma lo necessita per aiuto, e decise di cercare di farsi aiutare ma senza concedersi.
%Angelica esce dal cespuglio e gli dice di non avere opinioni false.
%Tuttavia, solo nel canto 19 verrà svelato la veridicità di questa proposizione.
%Il cavaliere comincia a spogliarsi, ma proprio in quel momento arriva un altro cavaliere,
%tutto vestito di bianco, con il quale comincia il duello.
%Sacripante viene sconfitto con tutto il peso del suo cavallo addosso,
%ma il cavaliere bianco non lo uccide e lo lascia così.

\end{document}