\documentclass[preview]{standalone}

\usepackage{amsmath}
\usepackage{amssymb}
\usepackage{bettelini}
\usepackage{stellar}
\usepackage{definitions}

\begin{document}

\id{chimica-grandezza-sistemi}
\genpage

\section{Grandezze e sistemi}

\begin{snippetdefinition}{chimica-sistema}{Sistema}
    Con \textit{sistema}
    si intende un oggetto o insieme di oggetti isolati
    di cui di studiano le proprietà termodinamiche.
\end{snippetdefinition}

\begin{snippetdefinition}{chimica-ambiente}{Ambiente}
    Con \textit{ambiente} si intende tutto ciò che si
    trova al di fuori del sistema e che è in grado
    di provocare in esso una modifica delle proprietà
    termodinamiche.
\end{snippetdefinition}

\begin{snippet}{sistema-ambiente-spiegazione}
Sistema \(\subseteq\) Ambiente \(\subseteq\) Universo.
\\\\
Un sistema può essere:
\begin{itemize}
    \item \textbf{aperto:} se scambia materia/energia con l'ambiente;
    \item \textbf{chiuso:} se scambia solo energia con l'ambiente;
    \item \textbf{osolato:} se non scambia né energia né materiale con l'ambiente.
\end{itemize}
\phantom{}\\
Studiare un sistema significa descrivere le sue proprietà
\begin{itemize}
    \item \textbf{Qualitative:} possono essere definite senza avvalersi
    di misure.
    \item \textbf{Quantitative:} richiedono delle misure.
\end{itemize}
Le priorità misurabili sono delle \textit{grandezze}.
\\\\
TODO: intensive / estensive
\end{snippet}

\end{document}
