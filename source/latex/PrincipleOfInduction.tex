\documentclass[preview]{standalone}

\usepackage{amsmath}
\usepackage{amssymb}
\usepackage{stellar}
\usepackage{definitions}
\usepackage{bettelini}

\begin{document}

\id{principle-of-induction}
\genpage

\section{Axioms of induction}

\begin{snippetaxiom}{axiom-of-induction}{Axiom of induction}
    For any \set \(E \subseteq \naturalnumbers\), let
    \(S\) be a \textit{successor} \function that is \injective.
    If:
    \begin{enumerate}
        \item an initial element is in the set \(0 \in S\);
        \item \(n\in E \implies S(n) \in E\),
    \end{enumerate}
    then \(E = \naturalnumbers\).
\end{snippetaxiom}

% Could also do axiom of strong induction

\section{Principle of induction}

\begin{snippet}{principle-of-induction-expl}
    Induction can be used to prove a statement in the form \(P(n)\)
    for all \(n \in \mathbb{N}\).
    For induction to work one must need to prove the base base, which is usually \(P(0)\),
    but other starting points can be used to prove the statement from that point on.
\end{snippet}

\begin{snippettheorem}{weak-induction-theorem}{Weak induction}
    Proving that the \textit{base case} \(P(0)\) is true,
    along with proving the \textit{induction step} \(P(n) \implies P(n+1)\), implies \(P(n)\)
    for all \(n \geq 0\). In second-order logic
    \[
        \forall P \left(
            P(0) \land \forall n \left( P(n) \implies P(n+1) \right)
            \implies \forall n \left( P(n) \right)
        \right)
    \]
\end{snippettheorem}

\begin{snippettheorem}{strong-induction-theorem}{Strong induction}
    Proving that the base case \(P(k)\) for \(k < m\) is true, along with
    proving the induction step \(P(m)\), implies \(P(n)\)
    for all \(n\).
\end{snippettheorem}

\plain{The base case for the strong version seems to disappear,
but it is actually still the first proposition.}

\begin{snippettheorem}{strong-weak-induction-equivalence-theorem}{Weak and strong induction equivalence}
    Strong and weak induction are equivalent.
\end{snippettheorem}

\begin{snippetproof}{strong-weak-induction-equivalence-proof}{strong-weak-induction-equivalence-theorem}{Weak and strong induction equivalence}
    \iffproof{
        Let \(B \subseteq \mathbb{N}\) be the \set formed by all the \(m\)
        such that a proposition \(P(m)\) is true.
        We know that \(B\) has a \leastelement \(n\) if it is non-empty.
        Thus, for every \(k < n\) we have that \(k\notin B\) and this \(P(k)\)
        is true for every \(k < n\).
        However, by hypothesis, if \(P(k)\) is true for every \(k<n\), then \(P(n)\)
        is true. That is, \(n\notin B\) \lightning.
    }{
    }
\end{snippetproof}

\section{Equivalence of axiom and principle of induction}

\begin{snippettheorem}{axiom-and-principle-of-indunction-equivalence-theorem}{Equivalence of axiom and principle of induction}
    The \axiomofinduction and \principleofinduction are \ifandonlyif[equivalent].
\end{snippettheorem}

\begin{snippetproof}{axiom-and-principle-of-indunction-equivalence-theorem-proof}{axiom-and-principle-of-indunction-equivalence-theorem}{Equivalence of axiom and principle of induction}
    \newcommand{\successor}{\labelref["Successor function"][\scolorweak[black]S]}
    Given a proposition \(P(n)\), let \[
        E= \{ n\in\naturalnumbers \,|\, P(n) \}
    \]
    \iffproof{
        If \(0\in E\), then \(P(0)\) is true. \\
        If \(n \in E \implies \successor(n) \in E\), then \(P(n) \implies P(\successor(n))\).\\
        If the latter conditions are satisfied, then by the \axiomofinduction,
        \(E=\naturalnumbers\), and thus \[\forall n\in \naturalnumbers, P(n)\]
    }{
        If \(P(0)\) is true, then \(0\in E\).\\
        If \(P(n) \implies P(\successor(n)) \), then if \(n\in E \implies \successor(n) \in E\).\\
        If the latter conditions are satisfied, then by the \principleofinduction
        \[\forall n\in\naturalnumbers, n \in E\] and thus \(E=\naturalnumbers\).
    }
\end{snippetproof}

\end{document}