\documentclass[preview]{standalone}

\usepackage{amsmath}
\usepackage{amssymb}
\usepackage{stellar}
\usepackage{definitions}
\usepackage{bettelini}

\begin{document}

\id{principle-of-induction}
\genpage

\section{Axioms of induction}

\begin{snippetaxiom}{axiom-of-induction}{Axiom of induction}
    For any \set \(E \subseteq \naturalnumbers\), let
    \(S\) be a \textit{successor} \function that is \injective.
    If:
    \begin{enumerate}
        \item an initial element is in the set \(0 \in S\);
        \item \(n\in E \implies S(n) \in E\),
    \end{enumerate}
    then \(E = \naturalnumbers\).
\end{snippetaxiom}

% Could also do axiom of strong induction

\section{Principle of induction}

\begin{snippet}{principle-of-induction-expl}
    Induction can be used to prove a statement in the form \(P(n)\)
    for all \(n \in \mathbb{N}\).
    For induction to work one must need to prove the base base, which is usually \(P(0)\),
    but other starting points can be used to prove the statement from that point on.
\end{snippet}

\begin{snippettheorem}{weak-induction-theorem}{Weak induction}
    Proving that the \textit{base case} \(P(0)\) is true,
    along with proving the \textit{induction step} \(P(n) \implies P(n+1)\), implies \(P(n)\)
    for all \(n \geq 0\). In second-order logic
    \[
        \forall P \left(
            P(0) \land \forall n \left( P(n) \implies P(n+1) \right)
            \implies \forall n \left( P(n) \right)
        \right)
    \]
\end{snippettheorem}

\begin{snippettheorem}{strong-induction-theorem}{Strong induction}
    Proving that the base case \(P(k)\) for \(k < m\) is true, along with
    proving the induction step \(P(m)\), implies \(P(n)\)
    for all \(n\).
\end{snippettheorem}

%The base case here is when we need to prove \(P(m)\) for \(m=0\).
\plain{Weak and strong induction are equivalent.}

\section{Equivalence of axiom and principle of induction}

\begin{snippettheorem}{axiom-and-principle-of-indunction-equivalence-theorem}{Equivalence of axiom and principle of induction}
    The \axiomofinduction and \principleofinduction are \ifandonlyif[equivalent].
\end{snippettheorem}

\begin{snippetproof}{axiom-and-principle-of-indunction-equivalence-theorem-proof}{axiom-and-principle-of-indunction-equivalence-theorem}{Equivalence of axiom and principle of induction}
    \newcommand{\successor}{\labelref["Successor function"][\scolorweak[black]S]}
    Given a proposition \(P(n)\), let \[
        E= \{ n\in\naturalnumbers \,|\, P(n) \}
    \]
    \iffproof{
        If \(0\in E\), then \(P(0)\) is true. \\
        If \(n \in E \implies \successor(n) \in E\), then \(P(n) \implies P(\successor(n))\).\\
        If the latter conditions are satisfied, then by the \axiomofinduction,
        \(E=\naturalnumbers\), and thus \[\forall n\in \naturalnumbers, P(n)\]
    }{
        If \(P(0)\) is true, then \(0\in E\).\\
        If \(P(n) \implies P(\successor(n)) \), then if \(n\in E \implies \successor(n) \in E\).\\
        If the latter conditions are satisfied, then by the \principleofinduction
        \[\forall n\in\naturalnumbers, n \in E\] and thus \(E=\naturalnumbers\).
    }
\end{snippetproof}

\section{Exercises}

\begin{snippetexercise}{induction-ex-1}{Weak induction}
    Prove that for each \(n \geq 1\), the sum \[ \sum_{k=1}^n k = \frac{n(n+1)}{2} \].
    \begin{itemize}
        \item The base case is given by \(n=1\) where \(1 = \frac{2}{2} = 1\).
        \item The inductive case is given by \(\xi = n+1\)
        \begin{align*}
            \frac{n(n+1)}{2} + \xi &= \frac{n(n+1)}{2} + \frac{2n}{2} + \frac{2}{2} \\
            &= \frac{n^2 + 3n + 2}{2} \\
            &= \frac{(n+1)(n+2)}{2} \\
            &= \frac{\xi(\xi+1)}{2}
        \end{align*}
    \end{itemize}
\end{snippetexercise}

\begin{snippetexercise}{induction-ex-2}{Weak induction}
    Prove \(n! > n^2\) for \(n \geq 4\).
    The base case is \(4!=24 > 4^2 = 16\).

    The induction step is to prove \(n! > n^2 \implies (n +1)! > {(n+1)}^2\).
    Note that \((n+1)!=(n+1)n!\).
    Since \(n! > n^2\), then
    \begin{align*}
        n!(n+1) &> n^2(n+1) \\
        n!(n+1) &> n^3 + n^2
    \end{align*}
    Since \(n \geq 4\), \(n^3 + n^2 > {(n+1)}^2=n^2+2n+1\).
    Thus, by the transitive property, \((n+1)! > {(n+1)}^2\).
\end{snippetexercise}

\begin{snippetexercise}{induction-ex-3}{Weak induction}
    Prove that
    \[
        \sum_{k=1}^n k^2 = \frac{n(n+1)(2n+2)}{6}
    \]
    \begin{itemize}
        \item The base case is \(\frac{6}{6}=1\)
        \item The induction case is given by
        \begin{align*}
            \frac{n(n+1)(2n+2)}{6} + {(n+1)}^2  &= \frac{n(n+1)(2n+2)}{6} + n^2 + 2b + 1 \\
            &= \frac{(n+1)(n+2)(2(n+1)+1)}{6}
        \end{align*}
    \end{itemize}
\end{snippetexercise}

\begin{snippetexercise}{induction-ex-4}{Weak induction}
    Per ogni \(n \geq 0\) e per ogni \(h > -1\),
    \[
        {(1+h)}^n \geq 1 + nh
    \]
    TODO:
\end{snippetexercise}

\begin{snippetexercise}{induction-ex-5}{Strong induction}
    Prove that every integer is written as a product of primes.
    \begin{itemize}
        \item The base case is \(n=2\), which is a prime number.
        \item The induction case is given by assuming \(\forall k \leq m\),
        the number \(k\) is a product of primes. The next number \(n+1\)
        is either prime or not prime. In the first case it is prime and we can thus
        write it as a product of prime. In the latter case it is not prime, and thus can be written
        as \(n+1 = kh\) for \(2\leq k,h< n+1\).
        By the inductive hypothesis \(h\) and \(k\) can be written as a product of primes.
        The same goes for \(n+1=hk\). 
    \end{itemize}
\end{snippetexercise}

\end{document}