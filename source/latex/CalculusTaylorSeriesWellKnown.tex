\documentclass[preview]{standalone}

\usepackage{amsmath}
\usepackage{amssymb}
\usepackage{stellar}
\usepackage{definitions}

\begin{document}

\id{taylor-series-well-known}
\genpage

\section{Well-known Taylor series}

\begin{snippetproposition}{exponentia-function-maclaurin-series}{Exponential function MacLaurin series}
    \[
        \eulernumber^x = \sum_{k=0}^\infty \frac{x^k}{k\factorial}
        = 1 + x + \frac{x^2}{2!} + \frac{x^3}{3!} + \cdots
    \]
\end{snippetproposition}

\begin{snippetproposition}{sine-maclaurin-series}{Sine MacLaurin series}
    \[
        \sin(x) = \sum_{k=0}^\infty \frac{{(-1)}^k}{(2k+1)\factorial}x^{2k+1}
        = x - \frac{x^3}{3!} + \frac{x^5}{5!} + \cdots
    \]
\end{snippetproposition}

\begin{snippetproposition}{cosine-maclaurin-series}{Cosine MacLaurin series}
    \[
        \cos(x) = \sum_{k=0}^\infty \frac{{(-1)}^k}{(2k)\factorial}x^{2k}
        = 1 - \frac{x^2}{2!} + \frac{x^4}{4!} + \cdots
    \]
\end{snippetproposition}

\begin{snippetproposition}{tangent-maclaurin-series}{Tangent MacLaurin series}
    \[
        \tan(x) = x + \frac{x^3}{3} + \frac{2x^5}{15} + \bigO(x^7)
    \]
\end{snippetproposition}

\begin{snippetproposition}{hyperbolic-sine-maclaurin-series}{Hyperbolic sine MacLaurin series}
    \[
        \sinh(x) = \sum_{k=0}^\infty \frac{x^{2k+1}}{(2k+1)\factorial}
        = = x + \frac{x^3}{3!} + \frac{x^5}{5!} + \cdots
    \]
\end{snippetproposition}

\begin{snippetproposition}{hyperbolic-cosine-maclaurin-series}{Hyperbolic cosine MacLaurin series}
    \[
        \cosh(x) = \sum_{k=0}^\infty \frac{x^{2k}}{(2k)\factorial}
        = 1 - \frac{x^2}{2!} + \frac{x^4}{4!} + \cdots
    \]
\end{snippetproposition}

\begin{snippetproposition}{hyperbolic-tangent-maclaurin-series}{Hyperbolic tangent MacLaurin series}
    \[
        \tanh(x) = x - \frac{x^3}{3} + \frac{2x^5}{15} + \bigO(x^7)
    \]
\end{snippetproposition}

\begin{snippetproposition}{log-maclaurin-series}{Logarithm MacLaurin series}
    \[
        \ln(1 + x) = \sum_{n=1}^\infty {(-1)}^{n+1}\frac{x^n}{n}
        = x - \frac{x^2}{2} + \frac{x^3}{3} + \cdots, \quad x\in(-1,1)
    \]
\end{snippetproposition}

\begin{snippetproposition}{binomial-maclaurin-series}{Binomial MacLaurin series}
    \[
        {(1 + x)}^\alpha = \sum_{n=0}^\infty \binom{\alpha}{n}x^n, \quad x\in(-1,1)
    \]
\end{snippetproposition}

\end{document}