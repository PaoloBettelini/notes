\documentclass[preview]{standalone}

\usepackage{amsmath}
\usepackage{amssymb}
\usepackage{stellar}

\hypersetup{
    colorlinks=true,
    linkcolor=black,
    urlcolor=blue,
    pdftitle={Stellar},
    pdfpagemode=FullScreen,
}

\begin{document}

\title{Stellar}
\id{orlando-furioso-xxiii-analisi}
\genpage

\section{Analisi}

\begin{snippet}{orlando-furioso-xxiii-analisi}
    Questo è l'episodio centrale del proemio, e lo è per tre motivi.
    \begin{enumerate}
        \item L'episodio della follia di Orlanndo occupa un terzo di tutto il libro;
        \item l'episodio dona il titolo al libro, fra tutti le vicende che avvengono;
        \item l'episodio contiene una chiave di .
    \end{enumerate}

    % parlare della tabella con 3 colonne

    % TODO: correlazione dell'amore per petrarca e per ariosto
    % Amore porta alla pazzia. Tutti quelli che indugiano nell'amore danno qualche segnaled di pazzi
\end{snippet}

\end{document}