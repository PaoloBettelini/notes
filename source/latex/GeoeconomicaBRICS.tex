\documentclass[preview]{standalone}

\usepackage{amsmath}
\usepackage{amssymb}
\usepackage{stellar}
\usepackage{bettelini}

\hypersetup{
    colorlinks=true,
    linkcolor=black,
    urlcolor=blue,
    pdftitle={Stellar},
    pdfpagemode=FullScreen,
}

\begin{document}

\title{Geografia economica}
\id{geoeconomica-brics}
\genpage

\section{BRICS}

\begin{snippetdefinition}{bric-definizione}{BRIC}
    I \textit{BRIC} (Brasile, Russia, India e Cina)
    sono un gruppo di nazioni povere (dagli anni 2000) con una economia
    molto promettente per l'imminente futuro, data la loro forte crescita. 
\end{snippetdefinition}

\begin{snippet}{brics-expl}
    Dal 2010 circa i BRIC hanno un incontro informale (forum) dei capi di Stato.
    I BRIC non sono quindi un ente giuridico.
    Durante questo incontro viene deciso di incontrarci ogni anno, e nel 2011 viene aggiunto il Sudafrica,
    facendo divenire l'accronico BRICS.
    
    I BRICS sono caratterizzati per la loro popolosità, forte crescita economica.
    Ognuno di esso ha un punto di forza specifico.
    Essi vogliono presentarsi come visione/posizione alternativa all'Occidente.
    
    \textbf{crescita eterogenea:} \\
    Ad oggi il Sudafrica è il membro che ha avuto la crescita minore.
    La Cina ha invece avuto invece la crescita maggiore, guadagnando il ruolo dominante.
    
    Globalmente vi è anche una assenza di una visione comune/possibili tensioni interne.
    
    \textbf{Brics+:} TODO.
\end{snippet}

\end{document}