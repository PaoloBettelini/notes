\documentclass[preview]{standalone}

\usepackage{amsmath}
\usepackage{amssymb}
\usepackage{tikz}
\usepackage{stellar}
\usepackage{bettelini}
\usepackage{tabularx}
\usepackage{makecell}

\hypersetup{
    colorlinks=true,
    linkcolor=black,
    urlcolor=blue,
    pdftitle={Assets},
    pdfpagemode=FullScreen,
}

\begin{document}

\title{Isostasia}
\id{geofisica-stagioni}
\genpage

\plain{La terra percorre un'orbita elissoidale attorno al sole. Il sole si trova in uno dei due fuochi dell'ellisse.
È importante notare che l'asse di rotazione della terra è leggermente inclinato, ma la sua
inclinazione è sempre la stessa.}

\begin{snippetdefinition}{afelio}{Afelio}
    L'\textit{afelio} è il punto di massima distanza dal sole sull'orbita
    della terra.
\end{snippetdefinition}

\begin{snippetdefinition}{perielio}{Perielio}
    Il \textit{perielio} è il punto di minima distanza dal sole sull'orbita
    della terra.
\end{snippetdefinition}

\plain{Afelio e perielio sono i due punti di massima distanza fra di loro sull'orbita terrestre.}

\begin{snippetdefinition}{equinozio}{Equinozio}
    L'\textit{equinozio} è il momento temporale in
    cui il piano contenente il disco dell'equatore
    passa per il centro geometrico del sole.
\end{snippetdefinition}

\begin{snippet}{equinozio-giorno-e-notte}
    L'equinozio avviene due volte all'anno, e ovunque nel pianeta i giorni dell'equinozio hanno lo stesso quantitativo di ore notturne e di luce solare.
    In realtà, durante l'equinozio il giorno e la notte non hanno la stessa durata.
    Questo è dovuto da due motivi:
    \begin{enumerate}
        \item la rifrazione atmosferica fa giungere la luce solare all'osservatore qualche minuto prime dell'alba;
        \item i momenti di alba e tramonto, che delineano le ore notturne da quelle diurne, sono definiti come il momento in cui il punto visibile più alto del sole è allineato con l'orizzonte. Questa definizione non è omogenea e sfasa il la durata del giorno da quella della notte.
    \end{enumerate}
    Di conseguenza, il giorno avrebbe la stessa durata della notte se la terra non avesse atmosfera e il sole fosse un singolo punto.
\end{snippet}

\begin{snippetdefinition}{solstizio}{Solstizio}
    Il \textit{solstizio} è il momento temporale in
    cui il Sole raggiunge, nel suo moto apparente lungo l'eclittica,
    il punto di declinazione massima o minima.
\end{snippetdefinition}

\includesnpt{earthsun}

%\begin{snippet}{equinozio-solstizio-tabella}
%    \begin{center}
%        \begin{tabular}{|c|c|c|c|c|}
%            \hline
%            & \makecell{Solstizio \\ d'inverno} & \makecell{Equinozio di\\ primavera} & \makecell{Solstizio \\d'estate} & \makecell{Equinozio \\d'autunno} \\
%            \hline
%            Data & 23 dicembre & 21 marzo & 21 giugno & 23 settembre \\
%            \hline
%            \makecell{Parallelo in cui il sole \\ è allo Zenit (a \\ mezzogiorno)} & \makecell{Tropico del \\Capricorno} & Equatore & Tropico del Cancro & Equatore \\
%            \hline
%            \makecell{Luogo con il maggior \\ numero di ore di luce} & Polo Sud & \makecell{Stessa durata dì e \\ della notte su tutto \\ il pianeta} & Polo Nord & \makecell{Stessa durata del dì e\\ della notte su tutto il pianeta} \\
%            \hline
%            \makecell{Luogo con il minor \\ numero di ore di luce} & Polo Nord & \makecell{Stessa durata dì e \\ della notte su tutto \\ il pianeta} & Polo Sud & \makecell{Stessa durata del dì e\\ della notte su tutto il pianeta} \\
%            \hline
%            \makecell{Che stagione inizia \\ nell'emisfero nord} & Inverno & Primavera & Estate & Autunno \\
%            \hline
%            \makecell{Che stagione inizia \\ nell'emisfero sud} & Estate & Autunno & Inverno & Primavera \\
%            \hline
%        \end{tabular}
%    \end{center}
%\end{snippet}

\begin{snippetdefinition}{zona-convergenza-intertropicale}{Zona di convergenza intertropicale}
    La \textit{zona di convergenza intertropicale} è un'area del pianeta Terra, mediamente situata in prossimità dell'equatore
    dove si ha la convergenza degli alisei, e la risalita di masse d'aria calda che determinano l'area di instabilità equatoriale, con piogge e temporali.
\end{snippetdefinition}

\plain{La zona di convergenza intertropicale è molto calda in quanto è molto esposta ai raggi solari.
Infatti, vi è poca differenza fra il giorno e la notte.}

\plain{In contrasto, vi sono le zone temperate, dove vi sono giornate corte in inverno e lunghe
in estate, e le calotte polari (artica e antartica), dove si alterna fra un gran dì e una grande notte.}

\plain{La delineazione di questa zona non è lineare ed è dinamica, a dipendenza dall'inclinazione della terra
e correnti oceaniche.
}

\begin{snippetdefinition}{tropico-del-capricorno}{Tropico del Capricorno}
    Il \textit{tropico del Capricorno} è il tropico terrestre
    situato nell'emisfero australe in cui il Sole
    culmina allo zenit un giorno all'anno (nel solstizio di dicembre).
\end{snippetdefinition}

\begin{snippetdefinition}{tropico-del-cancro}{Tropico del Cancro}
    Il \textit{tropico del Cancro} è il tropico terrestre
    situato nell'emisfero boreale in cui il Sole culmina
    allo zenit un giorno all'anno (nel solstizio di giugno).
\end{snippetdefinition}

\includesnpt{ventusky}

\end{document}