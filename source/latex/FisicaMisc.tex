\documentclass[preview]{standalone}

\usepackage{amsmath}
\usepackage{amssymb}
\usepackage{stellar}
\usepackage{definitions}
\usepackage{bettelini}
\usepackage{wrapfig}

\begin{document}

\id{fisica-misc}
\genpage

\section{Forze}

\begin{snippetdefinition}{costante-coulomb-definition}{Costante di Coulomb}
    La \textit{costante di Coulomb} è data da
    \[
        k = 9 \cdot 10^9 \frac{N \cdot m^2}{C^2} 
    \]
    dove \(C\) è l'unità di misura della carica elettricità.
\end{snippetdefinition}

\begin{snippetdefinition}{forza-coulomb-definition}{Forza di Coulomb}
    La \textit{forza di Coulomb} è la forze con la quale due cariche elettriche ferme,
    \(q_1\) e \(q_2\), a distanza \(r\), si attraggono
    \[
        F_Q = k \frac{q_1 q_2}{r^2}
    \]
    dove \(k\) è la costante di Coulomb.
\end{snippetdefinition}

\section{Molle}

\plain{Due molle in parallelo hanno il medesimo allungamento,
mentre due molle in serie hanno la stessa forza.}

% Spinta di Archimede
% Volume immerso = V liquido spostato
% Massa liquido spostato = Massa totale oggetto

\section{Scontri fra oggetti}

\begin{snippetdefinition}{urto-elastico-anelastico-definition-definition}{Urto elastico e anelastico}
    Quando due oggetti si scontrano, se essi rimangono
    attaccati viene detto \textit{anelastico}, mentre se i due oggetti
    si dividono l'urto viene detto \textit{elastico}.
\end{snippetdefinition}

\begin{snippetdefinition}{quantita-moto-definition}{Quantità di moto}
    La \textit{quantità di moto} è una grandezza fisica definita come il prodotto fra
    massa e velocità
    \[
        p = mv
    \]
    dove \(p\) è la quantità di moto, \(v\) la velocità e \(m\) la massa.
\end{snippetdefinition}

\plain{In un urto la quantità di moto viene conservata.}

\begin{snippet}{mass-given-momentum}
    \begin{align*}
        p_1^i + p_2^i &= p_1^f + p_2^f \\
        m_1^i v_1^i + mp_2^i  v_2^i &= m_1^f v_1^f + m_2^f v_2^f \\
        m_1 (v_1^i - v_1^f) &= m_2 (v_2^i - v_2^f) \\
        m_1 &= m_2 \left(
            \frac{v_2^i - v_2^f}{v_1^i - v_1^f}
        \right)
    \end{align*}
\end{snippet}

\begin{snippetdefinition}{impulso-definition}{Impulso}
    Il \textit{teorema dell'impulso}
    dice che il cambiamento della quantità di modo in un impulso
    è pari alla forza applicata per il tempo passato
    \[
        \Delta \vec{p} \triangleq \int_{t_0}^{t_1} \vec{F} dt    
    \]
\end{snippetdefinition}

\begin{snippet}{impulso-forza-costante}
    Nel caso in cui la forza è costante abbiamo
    \[
        \Delta \vec{p} = \vec{F}\Delta t
    \]
\end{snippet}

%Resistenze in serie si sommano, mentre in parallelo usano 1/(1/R1 + 1/R2).
%Lo stesso principio ma in casi opposti avviene per le molle F/k = F/k1 + F/k2 in serie e
%k = k1 + k2 in parallelo. Come le molle, anche i condensatori lo fanno

\end{document}