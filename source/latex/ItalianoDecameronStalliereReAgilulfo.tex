\documentclass[preview]{standalone}

\usepackage{amsmath}
\usepackage{amssymb}
\usepackage{stellar}

\hypersetup{
    colorlinks=true,
    linkcolor=black,
    urlcolor=blue,
    pdftitle={Stellar},
    pdfpagemode=FullScreen,
}

\begin{document}

\id{italiano-decameron-stalliere-re-agilulfo}
\genpage

\section{Analisi}

\begin{snippetdefinition}{industria-definition}{Industria}
    Con \textit{industria} si intende la capacità di chi acquista o recupera una cosa desiderata.
    Molto spesso la cosa desiderata è di tipo erotico e sessuale.
\end{snippetdefinition}

\begin{snippet}{stalliere-re-agilulfo-analisi}
    \textbf{Rubrica:} Un pallafreniere giace con la moglie d'Agilulf re, di che Agilulf tacitamente s'accorge; truovalo e tondalo; il tonduto tutti gli altri tonde, e così scampa della mala ventura.

    Le rr. 4-8 della novella dichiarano la morale ideologica del quale la novella sarà un esempio.
    Questo concetto è quello che non sempre è auspicabile pretendere di sapere tutto,
    perché a volte sapere certe cose porta allo svergognarsi e si ritorcono contro di noi.
    La novella è quindi una dimostrazione di questo fenomeno.
    
    La regina è vedova e sfortunata in amore (rr. 11),
    mentre il re è bravo, saggio e capace governare.
    Lo stalliere appartiene al ceto più basso, ma è intelligente e fisicamente simile al re.
    La dinamica dei personaggi crea un triangolo amoroso, ma nonostante
    i vertici del triangolo siano molto distanti da un punto di vista sociale, possono comunque battersi ad armi pari.
    Gli ambiti dove si possono battere ad armi pari sono l'ingegno e la prestazione amorosa (rr. 58-59).
    La donna cantata è sempre posta ad un livello più alto e l'amore da parte dello stalliere
    è di tipo platonico (rr. X).
    Questo amore rimane segreto e arde come il fuoco (rr. 19), si innamora in maniera nobile e
    cortese. Nonostante esso sia nobile d'animo, vuole anche avere delle effusioni carnali con la regina.
    
    Il testo può essere principalmente diviso in due parti: nella prima lo stalliere
    ha uno scopo erotico, e vuole andare a letto con la regina. Nella seconda,
    vuole eludere la vendetta del re non facendosi scoprire.
    In ambo le parti l'obiettivo è raggiunto mediante la furbizia, con una doppia beffa.
    Per due volte vi è uno scambio di persona. Inizialmente, si scambia l'identità con il re, mentre
    nella seconda diluisce la sua identità fra quelle dei suoi compagni.
    Possiamo notare come tutti guadagnino qualcosa: la regina guadagna una notte più appagante,
    il re guadagna reputazione con la regina e lo stalliere ci va a letto.
    
    Lo stalliere fa finta di essere arrabbiato per non parlare con la regina, e
    possiede un senso del limite, siccome si ferma prima per non rischiare di essere scoperto (rr. 54) (e non sfiderà più la fortuna).
    Possiede la capacità di autocontrollo poiché sta fermo quando il re gli posa la mano
    sul petto. Inoltre, compie la seconda beffa tagliando i capelli a tutti gli altri (rr. 95-96).
    Le sue azioni rimarranno per sempre un segreto.
    Questi sono i punti dove esso dimostra una grande furbizia e scaltrezza.
    Similarmente, il re presenta anche una grande intelligenza
    perché riesce a contenersi quando scope che sua moglie è stata a letto con un altro,
    riesce ad auto controllarsi, quando un altro invece avrebbe fatto una scenata (rr. X).
    Il re presume che il colpevole sia qualcuno di vicino (rr. 75) e ascolta i battiti
    cardiaci degli stallieri. Anche quando scopre il colpevole, riesce a contenersi e gli taglia
    i capelli per riconoscerlo il giorno successivo.
    Il re riconosce inoltre il merito dell'avversario, dopo aver combattuto sullo stesso
    piano nonostante essendo di ceti sociali completamente opposti (rr. 102-103).
    
    Ci sono quindi degli ambiti, in questo caso, prestazione erotica e ingegno,
    dove vengono completamente rimosse le condizioni sociali.
\end{snippet}

\end{document}