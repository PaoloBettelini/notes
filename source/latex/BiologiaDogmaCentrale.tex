\documentclass[preview]{standalone}

\usepackage{amsmath}
\usepackage{amssymb}
\usepackage{stellar}
\usepackage{bettelini}

\hypersetup{
    colorlinks=true,
    linkcolor=black,
    urlcolor=blue,
    pdftitle={Biologia},
    pdfpagemode=FullScreen,
}

\begin{document}

\title{Biologia}
\id{biologia-dogma-centrale}
\genpage

\section{Il dogma centrale della biologia}

\begin{snippet}{2cf38e3a-75c8-420c-8f40-645aaa99b55f}
    Qualsiasi processo vitale, sia nella singola cellula che nell'organismo pluricellulare nel suo
    insieme, dipende dalla presenza di proteine.

    Le proteine hanno compiti molteplici, sia nella \textbf{struttura}
    che nel \textbf{funzionamento} del sistema
    vivente.

    A livello \textbf{cellulare}, i diversi tipi di proteine legati alle membrane o liberi negli spazi
    intracellulari

    \begin{itemize}
        \item servono a trasportare molecole e ioni da un lato all'altro della membrana;
        \item consentono di ricevere informazioni dall'ambiente extracellulare;
        \item permettono di convertire forme di energia;
        \item sono responsabili dei movimenti cellulari con ciglia e flagelli;
        \item costituiscono l'impalcatura della cellula conferendo ad essa la sua forma specifica;
        \item sono all'origine degli spostamenti di organelli e agglomerati molecolari all'interno della
        cellula come pure dei cromosomi durante la riproduzione cellulare;
        \item fungono da catalizzatori specifici (enzimi) delle migliaia di reazioni chimiche diverse del
        metabolismo cellulare.
    \end{itemize}

    A livello di \textbf{organismo pluricellulare}, le proteine hanno un ruolo
    fondamentale.

    \begin{itemize}
        \item nel movimento, essendo le componenti delle fibre muscolari contrattili
        (actina e miosina);
        \item nel mantenere forma e consistenza degli organi e della pelle (collagene e
        cheratina);
        \item nel trasporto di sostanze nel sangue (e.g. emoglobina);
        \item nella difesa dell'organismo da agenti estranei (anticorpi).
    \end{itemize}
    
    Nelle cellule si trovano (e vengono costruite) migliaia di proteine diverse, con
    funzioni differenti. La specializzazione delle cellule in compiti particolari
    all'interno dell'organismo pluricellulare dipende dalla presenza di proteine
    specifiche, tipiche di quel tipo cellulare. Nel corpo umano, sono decine di
    migliaia le proteine diverse presenti.
\end{snippet}

\end{document}