\documentclass[preview]{standalone}

\usepackage{amsmath}
\usepackage{amssymb}
\usepackage{stellar}
\usepackage{definitions}

\begin{document}

\id{sequences-exercises}
\genpage

\section{Exercises}

\begin{snippetexample}{sequences-ex-1}{}
    Consider the \sequence
    \[
        x_n = \frac{n-8}{n^2 + 1}
    \]
    The \sequence is eventually decreasing.
    Let us show that \(\exists n \suchthat \forall n \geq N, x_n > x_{n+1}\). That is, \(\forall n \geq N\)
    \begin{align*}
        \frac{n-8}{n^2 + 1} - \frac{(n+1)-8}{{n+1}^2 + 1} &> 0, \\
        n^2 - 15n - 9 &> 0
    \end{align*}
    which gives the solution
    \[
        n > \frac{15-\sqrt{261}}{2} < 0
    \]
    Note that the inequality \(n^2 - 15n - 9\) could be reduced to
    \begin{align*}
        n^2 - 15n - 9 &>  n^2 - 15n - 9n \\
        n(n - 24) &> 0
    \end{align*}
    which gives the solution \(n>24\). This inequality is different but it still shows the
    existence of \(n\).
\end{snippetexample}

\begin{snippetexample}{sequence-limit-1}{}
    Consider the \sequence
    \[
        x_n = \frac{1}{n}
    \]
    We can show that
    \[
        x_n \to 0^+
    \]
    Since \(\varepsilon > 0\), we show that \(\exists N\) such that \(\forall n \geq N\),
    \[
        0 < x_n = \frac{1}{n} < \varepsilon
    \]
    Indeed, it suffices that \(n > \frac{1}{\varepsilon}\)
    and we can choose \(N\) as any integer such that \(N > \frac{1}{\varepsilon}\).
    For example,
    \[
        N = \left[\frac{1}{\varepsilon}\right] + 1
    \]
    where \([x]\) is the greatest integer less than or equal to \(x\).
    We have that \([x] \in \integers\) defined as
    \[
        [x] \leq x < [x] + 1
    \]
\end{snippetexample}

\begin{snippetexample}{sequence-limit-2}{}
    Consider the \sequence
    \[
        x_n = \sqrt{\frac{4n+1}{n+2}}
    \]
    Prove \(\lim x_n = 2\).

    Let \(\varepsilon > 0\). We want to verity whether
    \[
        2 - \varepsilon < \sqrt{\frac{4n+1}{n+2}} < 2 + \varepsilon
    \]
    is true \(\forall n > N\) for some \(N\).
    Without loss of generality, we can suppose \(\varepsilon < 2\)
    such that all the terms are positive and by squaring
    \[
        4-4\varepsilon + \varepsilon^2 < \frac{4n + 1}{n+2} < {(2 + \varepsilon)}^2 = 4 + 4\varepsilon + \varepsilon^2
    \]
    We now have the inequalities
    \[
        \begin{cases}
            \frac{4n + 1}{n+2} < 4 + 4\varepsilon + \varepsilon^2 \\
            \frac{4n + 1}{n+2} > 4 - 4\varepsilon + \varepsilon^2 \\
        \end{cases}
    \]
    We note that
    \[
        \frac{4n + 1}{n+2} - 4 = \frac{-7}{n+2} < 0
    \]
    and thus is always less than \(4\varepsilon + \varepsilon^2\).
    We now get
    \[
        \frac{(4n + 1) - 4n - 8}{n+2} = \frac{7}{n + 2} > -4\varepsilon + \varepsilon^2
    \]
    which is equivalent to
    \begin{align*}
        \frac{7}{n+2} < 4\varepsilon - \varepsilon^2 \quad \implies \quad
        \frac{n+2}{7} < \frac{1}{4\varepsilon - \varepsilon^2}
    \end{align*}
    and thus
    \[
        n > \frac{7}{4\varepsilon - \varepsilon^2} - 2
    \]
    Since \[
        \frac{-7}{n+2} < 0
    \]
    the limit is \(x_n \to 2^-\).
\end{snippetexample}


\begin{snippetexample}{sequence-limit-3}{}
    Consider the \sequence
    \[
        x_n = \sqrt{n^2 - n}
    \]
    Show that \(x_n \to +\infty \).
    Let \(M > 0\). We now study the inequality
    \begin{align*}
        \sqrt{n^2 - n} &> M \\
        n^2 - n &> M^2
    \end{align*}
    Since \(n \leq \frac{n^2}{2}\) for \(n \geq 2\), we can rewrite it as
    \begin{align*}
        n^2 - n \geq n^2 - \frac{n^2}{2} = \frac{n^2}{2} > \frac{n^2}{4} > M
    \end{align*}
    We thus get \(n > 2M\).
\end{snippetexample}

\begin{snippetexample}{sequence-limit-4}{}
    Compute the limit of the \sequence
    \[
        \lim \sin n
    \]
\end{snippetexample}

\begin{snippetsolution}{sequence-limit-4-sol}{}
    The limit does not exist. Indeed, the set
    \[
        \{ x = \sin n \suchthat n\in \naturalnumbers \}
    \]
    is dense in \([-1,1]\). That is, \(\forall t \in [-1, 1]\)
    and \(\forall \epsilon > 0, \exists n \in \naturalnumbers \suchthat |\sin n - t| < \epsilon\).
    This implies that the limit does not exist.
    Indeed, \(-1 < \sin n < 1\) implies the fact that if the limit exists, then the limit is in \([-1, 1]\).
    Let \(l \in [-1, 1]\), \(l' \in [-1, 1]\) with \(l'\notin \{\sin n \suchthat n\in\integers\}\)
    and \(\epsilon_1 = \frac{l+l'}{2}\).
    This implies that \(\exists n_1\) such that \(|l' - \sin n_1| < \epsilon_1\).
    Now, let \[\varepsilon_2 = \frac{1}{2} \min\{|l' - \sin n| \suchthat n = 1, 2, \cdots, n_1\}\]
    Esiste \(n_2 > n_1\) tale che \(|l' - \sin n_2 < \varepsilon_2|\).
    We continue the process by iterating.
    \todo
\end{snippetsolution}

\begin{snippetexample}{sequence-limit-5}{}
    Show that
    \[
        \lim \frac{\sin n}{n} \to 0
    \]
\end{snippetexample}

\begin{snippetsolution}{sequence-limit-5-sol}{}
    We consider
    \[
        -\frac{1}{n} < \frac{\sin n}{n} < \frac{1}{n}
    \]
    and this both \(-\frac{1}{n}\) and \(\frac{1}{n}\) tend to \(0\), \(\frac{\sin n}{n} \to 0\).
\end{snippetsolution}

\end{document}