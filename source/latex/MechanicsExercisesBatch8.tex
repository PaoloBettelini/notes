\documentclass[preview]{standalone}

\usepackage{amsmath}
\usepackage{amssymb}
\usepackage{stellar}
\usepackage{definitions}
\usepackage{bettelini}

\begin{document}

\id{mechanics-ex-8}
\genpage

\section{Exercises - Batch 8}

\begin{snippetexercise}{mechanics-ex-8.1}{\underline{8.1}}
    A projectile of mass \(m\) is launched from the ground at time \(t=0\)
    with an initial velocity \(v_0\) in a direction forming an angle of \(\alpha\)
    with the horizontal.
    During its flight, the projective explodes into two fragments of masses
    \(\frac{2}{3}m\) and \(\frac{1}{3}m\). The two fragments
    land on the ground simultaneously, and the distance of the lighter fragment from the launch point is
    \(x_2\). Determine at what distance from the launch point the heavier fragment lands (neglecting air resistance).
\end{snippetexercise}

\begin{snippetsolution}{mechanics-ex-8.1-sol}{\underline{8.1}}
    \todo
\end{snippetsolution}

\begin{snippetexercise}{mechanics-ex-8.2}{\underline{8.2}}
    A body of mass \(m\) is placed on a perch at a height \(h\) above the ground.
    At a certain instant, the body explodes and splits into two fragments of masses \(m_1\) and \(m_2 = m_1 / 2\),
    which reach the ground respectively to the left and to the right of the perch.
    Given that the initial velocities of the fragments are directed horizontally and that th
    distance between the points where they hit the ground is \(d\), calculate the initial velocities of
    the two fragments.
\end{snippetexercise}

\begin{snippetsolution}{mechanics-ex-8.2-sol}{\underline{8.2}}
    \todo
\end{snippetsolution}

\begin{snippetexercise}{mechanics-ex-8.3}{\underline{8.3}}
    Two bodies of masses \(m_1\) and \(m_2 = 2m_1\) are on a smooth horizontal plane,
    connected by a spring of negligible mass, elastic constant\(k\), and natural length \(L_0\).
    Initially, the spring is compressed by a length \(\Delta L\) and held in this state by a thin thread
    placed between the masses. Then, the thread is cut.
    Calculate the maximum velocities reached by the masses in their motion.
\end{snippetexercise}

\begin{snippetsolution}{mechanics-ex-8.3-sol}{\underline{8.3}}
    \todo
\end{snippetsolution}

\begin{snippetexercise}{mechanics-ex-8.4}{\underline{8.4}}
    A body of mass \(M\) in the shape of a wedge rests on a horizontal plane.
    A block of mass \(m\) is placed on top of the wedge at a height \(h\) relative to the horizontal plane.
    All surfaces are frictionless.
    The system is initially at rest, so the block slides along the wedge.
    Calculate the velocity of the wedge when the block reaches the horizontal plane
\end{snippetexercise}

\begin{snippetsolution}{mechanics-ex-8.4-sol}{\underline{8.4}}
    \todo
\end{snippetsolution}

\begin{snippetexercise}{mechanics-ex-8.5}{\underline{8.5}}
    A marble of mass \(m_1\) moves with velocity \(v_0\) on a smooth horizontal plane and undergoes
    a head-on elastic collision with another marble of mass \(m_2 = 3m_1\), initially at rest.
    Subsequently, the second marble falls down a step of height \(h\).
    Determine the velocity \(v\) with which the marble reaches the ground and the distance \(d\)
    from the edge of the step where it lands.
\end{snippetexercise}

\begin{snippetsolution}{mechanics-ex-8.5-sol}{\underline{8.5}}
    \todo
\end{snippetsolution}

\begin{snippetexercise}{mechanics-ex-8.6}{\underline{8.6}}
    A ball of mass \(m_1\) is attached to a spring with a natural length \(L_0\),
    with the other end fixed at a point \(O\).
    Initially, the ball compresses the spring by a length \(x\); it is then released and elastically
    collides with another ball of mass \(m_2 = 3m_1\) placed at a distance \(d\) from \(O\).
    Calculate the maximum compression of the spring in the motion of mass \(m_1\) following the collision.
\end{snippetexercise}

\begin{snippetsolution}{mechanics-ex-8.6-sol}{\underline{8.6}}
    \todo
\end{snippetsolution}

\begin{snippetexercise}{mechanics-ex-8.7}{\underline{8.7}}
    A billiard player, aiming to pocket a ball of mass \(m\), strikes it with their cue ball,
    also of mass \(m\), so that it deflects at an angle \(\beta\) relative to the line connecting the
    two balls. Given the initial velocity \(v_0\) of the player's ball and assuming an elastic collision,
    determine the velocities of the two balls after the impact.
\end{snippetexercise}

\begin{snippetsolution}{mechanics-ex-8.7-sol}{\underline{8.7}}
    \todo
\end{snippetsolution}

\end{document}