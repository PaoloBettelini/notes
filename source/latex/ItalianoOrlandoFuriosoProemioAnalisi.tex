\documentclass[preview]{standalone}

\usepackage{amsmath}
\usepackage{amssymb}
\usepackage{stellar}

\hypersetup{
    colorlinks=true,
    linkcolor=black,
    urlcolor=blue,
    pdftitle={Stellar},
    pdfpagemode=FullScreen,
}

\begin{document}

\title{Stellar}
\id{orlando-furioso-proemio-analisi}
\genpage

\section{Analisi}

\begin{snippet}{orlando-furioso-proemio-analisi}
    L'ottava 22 è emblematica del fatto che Ariosto mostri, da una parte una
    certa nostalgia di un vecchio mondo (di sette secondi prima) con dei valori cavallereschi,
    dall'altra una grossa carica ironica.
    La ricerca è una ricerca casuale, senza punti di riferimento, senza senso e senza indici
    ma soprattutto senza risultato.
    I personaggi si muovono senza sapere di per certo dove stanno andando
    e alla fine vi è sempre la frustazione del non ottenere ciò che si desidera (stizza).
    \\\\
    La visione medievale è quella del peccato rappresentato dalla selva, una volta caduti
    nel peccato è difficile uscirne. La selva non è nè positiva nè negativa, è il posto
    in cui capitano le cose più splendide e meravigliose e quelle più terribili.
    \\\\
    Il percorso di Dante nella Commedia è una linea dritta attraverso il pianeta terra.
    Invece, i peronsaggi di Ariosto si muovono in un modo completamente casuale (mondo Rinascimentale).
    Questa differenza rappresenta il lento cambiamento che porta ad un mondo meno assoluto e più relativo,
    con sempre più radici nell'immanenza piuttosto che nel trascendente (verso il Rinascimento).
    \\\\
    Vi sono una serie di interventi da parte del narratore il quale
    esprime il proprio giudizio (7.2, 22.2, 29.6, 32.8, 81.7-8).
    Gli interventi diretti al lettore sono per far sì che il lettore mantenga vivo il suo senso critico, ricordandogli
    della natura fittizia della storia.
\end{snippet}

\end{document}