\documentclass[preview]{standalone}

\usepackage{amsmath}
\usepackage{amssymb}
\usepackage{stellar}
\usepackage{definitions}
\usepackage{bettelini}

\begin{document}

\id{orlando-furioso-proemio-analisi}
\genpage

\section{Analisi}

\begin{snippet}{orlando-furioso-proemio-analisi-parte1}
    \textbf{Ariosto e la materia cavalleresca:} l'ottava 22 è emblematica del fatto che Ariosto mostri, da una parte una
    certa nostalgia di un vecchio mondo (di sette secondi prima) con dei valori cavallereschi,
    dall'altra una grossa carica ironica.
    La ricerca è una ricerca casuale, senza punti di riferimento, senza senso e senza indici
    ma soprattutto senza risultato.
    \\\\
    \textbf{La ricerca senza fine:} i personaggi si muovono senza sapere di per certo dove stanno andando
    e alla fine vi è sempre la frustrazione del non ottenere ciò che si desidera (stizza).
    Il verso 1 dell'ottava \quotes{Oh gran bontà dei cavalieri antichi} fa capire quanto il narratore
    vedesse con amarezza e nostalgia la lealtà cavalleresca.
    Aggiunge però note ironiche che mostra al lettore le debolezze dei cavalieri, vengono
    ridicolizzati tra di loro e vengono aggiunte note di non vera lealtà, ad esempio le ottave del
    cavaliere nel fiume (23-31)
    \\\\
    \textbf{Nella selva della vita:} la visione medievale è quella del peccato rappresentato dalla selva, una volta caduti
    nel peccato è difficile uscirne. La selva non è né positiva né negativa, è il posto
    in cui capitano le cose più splendide e meravigliose e quelle più terribili.
    Il percorso di Dante nella Commedia è una linea dritta attraverso il pianeta terra.
    Invece, i personaggi di Ariosto si muovono in un modo completamente casuale (mondo Rinascimentale).
    Questa differenza rappresenta il lento cambiamento che porta ad un mondo meno assoluto e più relativo,
    con sempre più radici nell'immanenza piuttosto che nel trascendente (verso il Rinascimento).
    Per Dante la selva è connotata ed è basata sulla trascendenza.
    \\\\
    \textbf{Gli interventi del narratore:} vi sono una serie di interventi da parte del narratore il quale
    esprime il proprio giudizio (7.2, 22.2, 29.6, 32.8, 81.7-8).
    Gli interventi diretti al lettore sono per far sì che il lettore mantenga vivo il suo senso critico, ricordandogli
    della natura fittizia della storia.
    \\\\
    \textbf{Angelica e gli altri personaggi: funzioni e simboli:}
    i personaggi in generale sono poco descritti e privi di una solida identità, rendendoli piatti e
    poco memorabili. Nessuno si ferma mai a pensare.
    Angelica viene vista come un camaleonte che si colora sempre con lo stesso colore dello
    sfondo; si adatta a qualsiasi circostanza, non fa niente di sua volontà in contrapposizione con
    la storia.
\end{snippet}

\section{Microcosmo armonico}

\begin{snippet}{orlando-furioso-proemio-analisi-parte2}
    L'Orlando Furioso è scritto in assenza totale di enjambement.
    La misura metrica spesso corrisponde alla misura sintattica.
    [11] è perfettamente equilibrata, così come la [42], senza nessuna suddivisione, chiara e
    diretta.

    Esempio di cambiamenti stilistici tra [1] e [2]:
    \begin{enumerate}
        \item è molto alta stilisticamente per i seguenti motivi: \begin{itemize}
            \item Doppio chiasmo iniziale
            \item Un lungo periodo complesso di otto endecasillabi
            \item Inversione del predicato ai primi due versi (latineggiante)
            \item Inversione del predicato "passaro i Mori" (v.3)
            \item Complemento di specificazione invertito "d'Africa il mare" (v.4)
            \item Molti enjambement (vv.2-3, 5-6, 6-7)
        \end{itemize}
        \item è molto semplice stilisticamente: \begin{itemize}
            \item Pausa alla fine di ogni verso che rende la lettura molto più semplice
            \item Periodi molto più spezzati
            \item Quasi totale assenza di enjambement
            \item Lessico meno ricercato (es. \quotes{matto}, parola italiana molto colloquiale, oltretutto in rima)
        \end{itemize}
    \end{enumerate}

    Ariosto è eccelle nel variare i registri linguistici anche a poca distanza l'uno dall'altro:
    \begin{itemize}
        \item Registro encomiastico, ottave in omaggio al dedicatario (es. [3])
        \item Registro epico, ottave della battaglia (es. [17])
        \item Registro comico, ottave sarcastiche (es. [22])
        \item Registro lirico, ottave intimistico e/o tragico (es. [44])
    \end{itemize}
\end{snippet}

\end{document}