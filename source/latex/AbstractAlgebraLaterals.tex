\documentclass[preview]{standalone}

\usepackage{amsmath}
\usepackage{amssymb}
\usepackage{stellar}
\usepackage{definitions}
\usepackage{bettelini}

\begin{document}

\id{laterals}
\genpage


\section{Products of sets}

\begin{snippetdefinition}{subset-product-definition}{Subset product}
    Let \((G, \circ)\) be a \group and \(A_1,A_2,\cdots, A_n \subseteq G\),
    the \emph{product} \(A_1A_2\cdots A_n\) is defined as
    \[
        A_1A_2\cdots A_n = \{ a_1\circ a_2 \cdots a_n \suchthat a_i\in A_i \}
    \]
\end{snippetdefinition}

\begin{snippetproposition}{subset-product-group-cardinality}{}
    Let \((G, \circ)\) be a \group and \(A_1,A_2,\cdots, A_n \subseteq G\)
    where \(A_i \neq \emptyset\) and \(\cardinality{A_i} < \infty\).
    We have
    \[
        \max\{\cardinality{A_i}\} \leq \cardinality{A_1A_2\cdots A_n}
        \leq \prod_i \cardinality{A_i}
    \]
\end{snippetproposition}

\begin{snippetproof}{subset-product-group-cardinality-proof}{subset-product-group-cardinality}{}
    Consider \(A\) and \(B\).
    The first equality follows from the cancellation law.
    For thr second, if \(\cardinality{A} \leq \cardinality{B}\),
    fixed \(a\in A\), the products \(a\circ b\) by varying \(x \in B\)
    are all distinct (of magnitude \(\cardinality{B}\)).
\end{snippetproof}

\plain{This does not work in a monoid.}

\section{Laterals}

\begin{snippetdefinition}{right-lateral-definition}{Right lateral}
    Let \((G, \circ)\) be a \group, \((H, \circ) \subgroupleq (G, \circ)\)
    and \(g\in G\).
    The \emph{right lateral} of \((H, \circ)\) in \((G,\circ)\) containing \(g\)
    is defined as
    \[
        Hg \triangleq \{h\circ g \suchthat h\in H\}
    \]
\end{snippetdefinition}

\begin{snippetdefinition}{left-lateral-definition}{Left lateral}
    Let \((G, \circ)\) be a \group, \((H, \circ) \subgroupleq (G, \circ)\)
    and \(g\in G\).
    The \emph{left lateral} of \((H, \circ)\) in \((G,\circ)\) containing \(g\)
    is defined as
    \[
        gH \triangleq \{g\circ h \suchthat h\in H\}
    \]
\end{snippetdefinition}

%\begin{snippet}{properties-of-laterals}
%    Tra i laterali destri di \(H\) c'è \(H\) stesso.
%    Infatti, \(H1 = H\). Analogamente \(1H=H\) cioè \(H\)
%    compare anche tra i laterali sinistri.
%
%    Consideriamo un gruppo non-abeliano \((G, \circ) = \permgrp_3\) and let \((H,\circ) \subgroupleq (G, \circ)\)
%    where \[H = \gengrp{\kcycle{1,2}} = \{\text{Id}, \kcycle{1,2}\}\]
%    I laterali sono dati da
%    \begin{align*}
%        H\text{Id} &= H \\
%        H\kcycle{1,2} &= H \\
%        H\kcycle{2,3} &= \{\kcycle{2,3}, \kcycle{1,3,2}\} \\
%        H\kcycle{1,3} &= \{\kcycle{1,3}, \kcycle{1,3,2}\} \\
%        H\kcycle{1,2,3} &= \{\kcycle{1,2,3}, \kcycle{1,3}\} \\
%        H\kcycle{1,3,2} &= \{\kcycle{1,3,2}, \kcycle{2,3}\}
%    \end{align*}
%    Notiamo che \(H\text{Id} = H\kcycle{1,2}\),
%    \(H\kcycle{2,3} = H\kcycle{1,3,2}\) e \(H\kcycle{1,3} = H\kcycle{1,2,3}\).
%    Tutti i laterali sono disgiunti (salvo le ripetizioni).
%    Ciascuno dei laterali contiene il suo elemento che lo identifica,
%    quindi l'unione di tutti i laterali è l'insieme intero.
%    Ogni laterale ha numero di elementi quanto il sottogruppo.
%\end{snippet}

\begin{snippetproposition}{laterals-cardinality}{Cardinality of laterals}
    Let \((G, \circ)\) be a \group, \((H, \circ) \subgroupleq (G, \circ)\).
    Then, for every \(g\in G\),
    \[
        \cardinality{H} = \cardinality{Hg} = \cardinality{gH}
    \]
\end{snippetproposition}

\begin{snippetproof}{laterals-cardinality-proof}{laterals-cardinality}{Cardinality of laterals}
    Without loss of generality, we are going to use the right lateral.
    Consider a \function
    \[
        f\colon H \fromto Hg
    \]
    where \(f(h) = h\circ g\).
    By definition, it is \surjective, and by the cancellation laws it is \injective.
    Since the \function is \bijective, every lateral has cardinality \(\cardinality{H}\).
\end{snippetproof}

\begin{snippetproposition}{laterals-partition}{Laterals form a partition}
    Let \((G, \circ)\) be a \group, \((H, \circ) \subgroupleq (G, \circ)\).
    The laterals of \(H\) (either right or left) form a \partition
    of \(G\) with subsets of cardinality \(\cardinality{H}\).
    In particular, every element \(g\in G\) is contained in exactly one lateral.
\end{snippetproposition}

\plain{If we had defined the laterals on a monoid, a partition would not necessarily
form a partition and maintain the same cardinality within the laterals.}

\begin{snippetproof}{laterals-partitionunion-proof}{laterals-partition}{Laterals form a partition}
    Without loss of generality, we are going to use the right lateral.
    We first show that \(g\) is contained in at least one lateral.
    Indeed, \(g\in Hg\).
    We now need to show that \(g\) is in only one lateral of \(H\).
    This is equivalent to saying that, given \(Hg\) and \(Hk\) for \(g,k\in G\),
    either \(Hg = Hk\) or \(Hg \intersection Hk = \emptyset\).
    Assume there exist \(x\in Hg \intersection Hk\). We want to show that \(Hg = Hk\).
    We know that \(x=hg\) for some \(h\in H\)
    and \(x = \overline{h}k\) for some \(\overline{h}\in H\).
    We thus have \(hg = \overline{h}k\) and by using the inverse we get
    \(g = h^{-1} \overline{h} k \in H\).
    The elements of \(Hg\) have the form \(\hat{h}g\) for \(\hat{h} \in H\),
    and since \(\hat{h}h^{-1}\overline{h} \in H\), \(\hat{h}h^{-1}\overline{h} k \in Hk\).
    Thus, \(Hg \subseteq Hk\). By symmetry, \(Hk \subseteq Hg\) and thus \(Hk=Hg\).
\end{snippetproof}

\begin{snippetdefinition}{subgroup-index-definition}{Subgroup index}
    \todo
\end{snippetdefinition}

\begin{snippettheorem}{lagrange-theorem}{Lagrange theorem}
    Let \((H, \circ) \subgroupleq (G, \circ)\)
    where \(\cardinality{G} < \infty\).
    Then, \[\cardinality{G} = |(G, \circ)\,:\,(H,\circ)| \cdot \cardinality{H}\]
    In particular, \(\cardinality{H}\) divides \(\cardinality{G}\).
\end{snippettheorem}

\plain{In the infinite case we can still say tha that the cardinality of the right laterals
is the same as the cardinality of the left cardinals, but without the algebraic simplification.}

\begin{snippetproof}{lagrange-theorem-proof}{lagrange-theorem}{Lagrange theorem}
    Obvious by \snippetref[laterals-partition][Laterals form a partition].
\end{snippetproof}

\begin{snippetproposition}{amount-of-right-and-left-laterals}{Amount of right and left laterals}
    Let \((H, \circ) \subgroupleq (G, \circ)\).
    Then, the amount of right laterals is the same as the amount of left laterals.
\end{snippetproposition}

\begin{snippetproof}{amount-of-right-and-left-laterals-proof}{amount-of-right-and-left-laterals}{Amount of right and left laterals}
    We want to find a \bijective \function
    \[
        f\colon \{Hg\}_{g\in G} \fromto \{gH\}_{g\in G}
    \]
    It is not possible to send \(Hg\) to \(gH\).
    We need to send \(Hg\) in \(g^{-1}H\). We need to verify that this function is well-defined.
    Let \(Hg_1 = Hg_2\), meaning \(g_1 \in Hg_2\) or \(g_1 = hg_2\) for some \(h\in H\).
    Now, \(g^{-1}_1 = {(hg_2)}^{-1} = g_2^{-1} h^{-1}\).
    Thus, \(g_1^{-1} \in g_2^{-1}H\). If two elements are in the left lateral, their inverse
    are in the right lateral (and viceversa).
    It follows that \(g_1^{-1}H = g_2^{-1}H\), and the \function is well-defined since
    we can easily construct the inverse \function with the same argument.
    By composing the function with its inverse we get the identity in both direction, so it is \bijective.
\end{snippetproof}

% TODOURGENT dovrebbe essere la notazione g\circ H e H \circ g

\begin{snippet}{laterals-form-congruence-class-1}{}
    Consider the \group \((\integers, +)\)
    and the \subgroup formed by the multiples of a positive integer \(n\integers\).
    Consider the right laterals (or left laterals since the group is \abeliangroup[abelian])
    \begin{align*}
        n\integers + a = \{h + a \suchthat h\in n\integers\}
        = \{nk + a \suchthat k \in \integers\} = {[a]}_n
    \end{align*}
\end{snippet}

\plain{This also means that an infinite group can have finite laterals.}

\begin{snippetcorollary}{generating-group-order-divides-cardinality}{}
    Let \((G, \circ)\)
    be a \group where \(\cardinality{G} < \infty\) and let \(g\in G\).
    Then, the order of \(\gengrp{g}\) divides \(\cardinality{G}\).
\end{snippetcorollary}

\begin{snippetproof}{generating-group-order-divides-cardinality-proof}{generating-group-order-divides-cardinality}{}
    Since \(|\gengrp{g}| = |g|\), we also have that the \cyclicperiod
    of \(g\) divides \(|G|\).
\end{snippetproof}

\begin{snippetcorollary}{prime-order-group-is-cyclic}{Group with prime order is cyclic}
    Let \((G, \circ)\) be a \group with order \(p\) where \(p\) is prime.
    Then, \((G, \circ)\) is \cyclicgroup[cyclic].
\end{snippetcorollary}

\begin{snippetproof}{prime-order-group-is-cyclic-proof}{prime-order-group-is-cyclic}{Group with prime order is cyclic}
    Let \(g\in G\) be a non trivial element. Its period divides \(\cardinality{G}\),
    but \(|g|\) is prime, and thus \(|g| = 1\) or \(|g| = |G|\), but
    \(|g|\neq 1\) by the choice of \(g\), and thus \(g\) generates \((G, \circ)\).
\end{snippetproof}

\begin{snippettheorem}{index-multiplication-theorem}{Index multiplication theorem}
    Let \((K, \circ) \subgroupleq (H, \circ) \subgroupleq (G, \circ)\)
    be \group[groups]. Then, \(|G\,:\,K|\) is finite \ifandonlyif
    \(|G\,:\,H|\) is finite. In particular,
    \[
        |G\,:\,K| = |G\,:\,H| |H\,:\,K|
    \]
\end{snippettheorem}

\begin{snippetproof}{index-multiplication-proof}{index-multiplication}{Index multiplication theorem}
    We first consider thr case where \(|G\,:\,K|\)
    is finite, meaning \((K, \circ)\) has a finite amount of right laterals in \((G, \circ)\).
    Consider the laters of \((H, \circ)\) in \((G, \circ)\). We now show that their amounts
    does not exceed the amount of laterals \(|G\,:\,K|=n\). It suffices to show that
    if we take \(Hg_1, Hg_2 \cdots Hg_{n+1}\), at least two of those
    are the same. Consider \(Kg_1, Kg_2 \cdots Kg_{n+1}\) which are \(n+1\).
    Assume, for instance, that \(Kg_1 = Kg_2\) are the two laterals that are the same,
    meaning \(g_1 \in Kg_2 \subseteq Hg_2\), which implies \(Hg_1 = Hg_2\).
    We now show that \(|H\,:\,K| \leq n\). Since the laterals
    of \(K\) in \(H\) have form \(hk\) for some \(h\in (H, \circ)\), and they are
    some of the laterals of \((K, \circ)\) in \((G, \circ)\).
    Viceversa, let \(|G\,:\,H| = r\) and \(|H\,:\,K| = s\) be finite.
    We want to show that \(|G\,:\,K| = rs\).
    Let \(Hg_1, Hg_2 \cdots Hg_r\) the distinct laterals of \((H, \circ)\) in \((G, \circ)\)
    and \(Kh_1,Kh_2,\cdots Kh_s\) the distinct laterals of \((K, \circ)\) in \((H, \circ)\).
    Consider the laterals \(Kh_ig_j\), which are laterals of \((K, \circ)\) in \(K, \circ\).
    We want to show that they are all distinct and that they are all the laterals of \((K, \circ)\)
    in \((G, \circ)\):
    \begin{enumerate}
        \item let \(KH_ig_j = Kh_ug_v\).
        This means that \(h_ig_j = kh_ug_b\) for some \(k\in K\).
        By multiplying by the inverse we get
        \(g_j = h_i^{-1}kh_ug_v \in Hg_v\) meaning \(Hg_j = Hg_v\) and, thus,
        \(g_j = g_v\). From \(h_ig_j = kh_ug_v\) we get
        \(h_i = Kh_u \in Kh_u\) meaning \(Kh_i = Kh_u\) and thus \(h_i = h_u\);
        \item let \(Kg\) such a lateral. We know that \(g \in Hg_i\)
        for some \(1\leq i \leq r\), meaning \(g=hg_i\) with \(h\in H\).
        We also know that \(h \in Kh_j\) for some \(j \leq i \leq s\)
        and thus \(h=kh_j\) with \(k\in K\).
        Finally, \(g = hg_i = kh_j g_i \in Kh_j g_i\).
    \end{enumerate}
\end{snippetproof}

\begin{snippetproposition}{laterals-two-elements-equivalences}{}
    Let \((K, \circ) \subgroupleq (G, \circ)\)
    be \group[groups] and let \(g_1,g_2\in G\). Then:
    \begin{enumerate}
        \item \(Hg_1 = Hg_2 \iff g_1g_2^{-1} \in H\);
        \item \(g_1 H = g_2 H \iff g_1^{-1} g_2 \in H\);
    \end{enumerate}
\end{snippetproposition}

\begin{snippetproof}{laterals-two-elements-equivalences-proof}{laterals-two-elements-equivalences}{}
    \begin{enumerate}
        \item Suppose that \(Hg_1 = Hg_2\), meaning \(g_1 = hg_2\) for some \(h\in H\).
        This menas that \(g_1g_2^{-1} = h \in H\).
        Likewise, if \(g_1g_2^{-1} \in H\), then \(g_1 = (g_1g_2^{-1})g_2 \in Hg_2\)
        meaning \(Hg_1 = Hg_2\);
        \item The proof is similar.
    \end{enumerate}
\end{snippetproof}

\section{Subgroups as product of sets}

\begin{snippetproposition}{union-of-subgroups-is-in-product}{}
    Let \((H, \circ), (K, \circ) \subgroupleq (G, \circ)\) be \group[groups].
    Then, \(HK \supseteq H \union K\)
    and \(HK \subseteq \gengrp{H \union K}\).
    % TODOURGENT write this better and only the general case
    In general, if \(H_1, H_2, \cdots, H_r\) are \subgroup[subgroups]
    of \((G, \circ)\), then
    \[
        \bigcup H_i \subseteq \prod H_i \subseteq \gengrp{H_1 \union H_2 \cdots H_r}
    \]
\end{snippetproposition}

\begin{snippetproof}{union-of-subgroups-is-in-product-proof}{union-of-subgroups-is-in-product}{}
    Since \(h\in H\), we have \(h=h \circ 1 \in HK\)
    and the same goes ofr \(K\).
    For the second proposition if \(h\in H\),
    if \(k \in K\) we have \(k\in \gengrp{H \union K}\)
    and thus \(hk \in \gengrp{H \union K}\).
\end{snippetproof}

\begin{snippettheorem}{set-product-subgroup-conditions}{}
    Let \((H, \circ), (K, \circ) \subgroupleq (G, \circ)\) be \group[groups].
    The following are equivalent:
    \begin{enumerate}
        \item \(HK \subgroupleq (G, \circ)\);
        \item \(HK = KH\);
        \item \(HK \supseteq KH\).
    \end{enumerate}
\end{snippettheorem}

\begin{snippetproof}{set-product-subgroup-conditions-proof}{set-product-subgroup-conditions}{}
    \begin{enumerate}
        \item \((1) \implies (3)\): we know that \(HK \supseteq H\)
        and \(HK \supseteq K\).
        Since \(HK\) is a \subgroup of \((G, \circ)\), it contains the products of its elements.
        In particular, if \(k \in K\) and \(h\in H\), we have that
        \(hk \in HK\);
        \item \((3) \implies (2)\): we need to show that if \(HK \supseteq KH\),
        then \(HK \subseteq KH\).
        Let \(h \in H\) and \(k\in K\). Now,
        \[
            hk = {({(hk)}^{-1})}^{-1} = {(k^{-1} h^{-1})}^{-1}
        \]
        However,
        \[
            k^{-1}h^{-1} \in KH \subseteq HK
        \]
        This means that \(k^{-1}h^{-1} = \overline{h}\overline{k}\)
        for some \(\overline{h} \in H\) and \(\overline{k} \in K\).
        Thus, \[ hk = {({(hk)}^{-1})}^{-1} = {\overline{h} \overline{k}}^{-1}
        = \overline{k}^{-1} \overline{h}^{-1} \in K \]
        \item \((2) \implies (1)\): we know that \(HK = KH\)
        and we need to show that \(HK \subgroupleq (G, \circ)\):
        \begin{enumerate}
            \item \(HK \neq \emptyset\): this is trivial;
            \item \(HK\) is closed under \(\circ\): let \(g_1, g_2 \in HK\),
            meaning \(g_1 = h_1k_1\) and \(g_2 = h_1k_1\)
            with \(h_1, h_2 \in H\) and \(k_1k_2 \in K\). Thus,
            \[
                g_1g_2 = h_1k_1h_2k_2
            \]
            by the hypothesis \(HK=KH\).
            However, \(k_1h_2 \in KH = HK\)
            from which we get \(k_1h_2 = h_3k_3\)
            for some \(h_3 \in H\) and \(k_3 \in K\).
            Thus,
            \[
                g_1g_2 = h_1h_3k_3k_2 \in HK
            \]
            \item \(HK\) is closed with respect to the inverse operation:
            let \(g\in HK\), meaning \(g = hk\) for osme \(h\in H\)
            and \(k\in K\).
            Thus,
            \[
                g^{-1} = {(hk)}^{-1} = k^{-1}h^{-1} \in KH = HK
            \]
        \end{enumerate}
    \end{enumerate}
\end{snippetproof}

\begin{snippetnote}{sets-products-commutativity}{}
    If \(\forall h \in H, \forall k\in K, hk = kh\), then \(KH=HK\),
    but the inverse is not guaranteed, unless the \group is \abeliangroup[abelian],
    where every product of subsets is necessarily a subgroup.
\end{snippetnote}

\begin{snippetproposition}{subgroup-center-product}{}
    Let \(H \subgroupleq G\) and \(K \subgroupleq \groupcenter(G)\) be \group[groups].
    Then,
    \[
        HK \subgroupleq G
    \]
\end{snippetproposition}

\begin{snippetproposition}{subgroup-product-cardinality-theorem}{}
    Let \((H, \circ), (K, \circ) \subgroupleq (G, \circ)\) be \group[groups]
    and let \(h,\overline{h}\in H\), \(k,\overline{k}\in K\).
    Then,
    \[
        h\circ k = \overline{h} \circ \overline{k}
        \iff \exists x \in H \intersection K \suchthat \overline{h} = hx
        \land \overline{k} = x^{-1}k
    \]
    In particular, if \(K\) and \(H\) are finite,
    \[
        \cardinality{HK} = \frac{\cardinality{H} \cdot \cardinality{K}}{\cardinality{H \intersection K}}
    \]
\end{snippetproposition}

\begin{snippetproof}{subgroup-product-cardinality-theorem-proof}{subgroup-product-cardinality-theorem}{}
    Let \(x \in H \intersection K\) and let \(\overline{h} = hx\)
    and \(\overline{k} = x^{-1}x\). \\
    \iffproof{
        This means that \(\overline{h} \circ \overline{k} = hxx^{-1}k = hk\).
    }{
        Suppose that \(hk = \overline{h} \circ \overline{k}\). Thus,
        \(\overline{h}^{-1}hk = \overline{k}\) and
        \[
            \overline{h}^{-1}h = \overline{k}k^{-1} \in H \intersection K
        \]
        But then \(\overline{k} = (\overline{k}k^{-1})k\)
        and \(\overline{h} = h{(\overline{h}^{-1} h)}^{-1}\).
        Denote \(x^{-1} = \overline{k}k^{-1}\)
        and \(x = {(\overline{h}^{-1} h)}^{-1}\).
    }
    Assume that \(H\) and \(K\) are finite.
    In order to compute \(HK\), we need to compute \(\cardinality{H} \cdot \cardinality{K}\)
    multiplications.
    However, each result appears \(\cardinality{H \intersection K}\) times.
\end{snippetproof}

\end{document}