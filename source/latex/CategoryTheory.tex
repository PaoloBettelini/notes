\documentclass[preview]{standalone}

\usepackage{amsmath}
\usepackage{amssymb}
\usepackage{parskip}
\usepackage{fullpage}
\usepackage{hyperref}
\usepackage{tikz}
\usepackage{makecell}
\usepackage{adjustbox}
\usepackage{stellar}

\usetikzlibrary{ % tikz packages
	cd % tikz-cd communitative diagrams
}

\hypersetup{
    colorlinks=true,
    linkcolor=black,
    urlcolor=blue,
    pdftitle={CategoryTheory},
    pdfpagemode=FullScreen,
}

\newcommand{\opp}[1]{{#1}^{\text{op}}}

\begin{document}

\id{categorytheory}
\genpage

% after abstraction we notice that everything is the same. Every theory is the same thing.

\section{Category}

\begin{snippetdefinition}{category-definition}{Category Definition}
    A \textit{category} is a tuple \(C=(\text{ob}(C), \text{hom}(C), \circ)\) where:
    \begin{itemize}
        \item \(\text{ob}(C)\) is a class of \textit{objects};
        \item \(\text{hom}(C)\) is a class of \textit{morphisms} or \textit{arrows};
        \item \(\circ\) a binary operation called \textit{composition}.
    \end{itemize}
    where \(\circ\) is transitive, reflexive and associative.
\end{snippetdefinition}

\begin{snippet}{category-example-illustration}
    \[
        \begin{tikzcd}
            a \arrow[r, bend left] \arrow[r, bend left=49, shift left] \arrow[loop, distance=2em, in=215, out=145] & b \arrow[loop, distance=2em, in=35, out=325] \arrow[l, bend left]
        \end{tikzcd}
        \\
        \makecell[l] {
            \text{An example of}
            \\
            \text{objects and morphisms}
        }
    \]
\end{snippet}

%The set of morphisms of a category \(C\) is denoted \(\text{hom}(C)\)
%and the set of objects \(\text{ob}(C)\).

%\subsection{Composition or Transitivity}

Composition is a property that says that if there is an arrow from
\(a\) to \(b\), and an arrow from \(b\) to \(c\), there must exist an arrow
from \(a\) to \(c\).

\begin{center}
    \begin{tikzcd}
        a \arrow[r, "f"]
        \arrow[rr, "f \circ g", bend left=49] & b
        \arrow[r, "g"] & c
    \end{tikzcd}
\end{center}

%\subsection{Identity or Reflexivity}

For every object there is an identity arrow.

\begin{center}
    \begin{tikzcd}
        a \arrow["\text{id}_a"', loop, distance=2em, in=35, out=325]
    \end{tikzcd}
\end{center}

The composition of an arrow with an identity is the arrow itself

\begin{center}
    \begin{tikzcd}
        a \arrow[r, "f"] &
        b \arrow["\text{id}_b"', loop, distance=2em, in=35, out=325]
    \end{tikzcd}
\end{center}

\[
    f \circ \text{id}_b = f
\]

and also vice versa

\[
    \text{id}_b \circ f = f
\]

%\subsection{Associativity}

Compositions have the associative property

\begin{center}
    \begin{tikzcd}
        a \arrow[r, "f"]
        \arrow[rr, "g \circ f", bend left, shift left=2]
        \arrow[rrr, "h \circ (g \circ f)", bend left=49, shift left=2]
        \arrow[rrr, "(h \circ g) \circ f", bend right=49, shift right=2] & b
        \arrow[r, "g"] \arrow[rr, "h \circ g", bend right, shift right=2] & c
        \arrow[r, "h"] & d
    \end{tikzcd}
\end{center}

\[
    h \circ (g \circ f) = (h \circ g) \circ f
\]

\section{Homomorphisms}

\begin{snippetdefinition}{homomorphism-definition}{Homomorphism}
    A homomorphism is a map between two structures of the same type.
\end{snippetdefinition}

\subsection{Epimorphisms}

\begin{snippetdefinition}{epimorphism-definition}{Epimorphism}
    An epimorphism is a \textit{surjective} morphism.
    An epimorphism is labelled with \(\twoheadrightarrow\).
\end{snippetdefinition}

\begin{snippet}{surjectivity-from-morphisms-expl}
    Consider a morphism \(f: a \rightarrow b\) which maps elements of \(a\) onto \(b\).
    Let's also define the morphisms \(g_1\) and \(g_2\) which map elements from \(b\) to \(c\).
    The domain of \(g_1\) and \(g_2\) is the codomain of \(f\). These two functions act
    as \(f\) for object in the image of \(f\), but may map objects differently
    for objects in the codomain of \(f\) but outside the image of \(f\).
    If the morphism is surjective, hence if the codomain and the image of \(f\) are the same,
    then \(g_1\) and \(g_2\) will always act as \(f\).
\end{snippet}

\begin{snippetdefinition}{surjectivity-from-morphisms}{Surjectivity from morphisms}
    A morphism \(f: a \rightarrow b\) is a epimorphism if

    \[
        \forall c\, \forall g_1, g_2 : b \rightarrow c, g_1 \circ f = g_2 \circ f \Rightarrow g_1 = g_2
    \]

    \begin{center}
        \begin{tikzcd}
            a \arrow[r, "f"] &
            b \arrow[r, "g_1", shift left]
            \arrow[r, "g_2"', shift right] & c
        \end{tikzcd}
    \end{center}
\end{snippetdefinition}

\subsection{Monomorphisms}

\begin{snippetdefinition}{monomorphism-definition}{Monomorphism}
    An monomorphism is an \textit{injective} morphism.
    A monomorphism is labelled with \(\hookrightarrow\).
\end{snippetdefinition}

\begin{snippetdefinition}{injectivity-from-morphisms}{Injectivity from morphisms}
    A morphism \(f: a \rightarrow b\) is a monomorphism if
    
    \[
        \forall c\, \forall g_1, g_2: c \rightarrow a, 
        f \circ g_1 = f \circ g_2 \Rightarrow g_1 = g_2
    \]

    \begin{center}
        \begin{tikzcd}
            c \arrow[r, "g_1", shift left]
            \arrow[r, "g_2"', shift right] &
            a \arrow[r, "f"] & b
        \end{tikzcd}
    \end{center}
\end{snippetdefinition}

\subsection{Isomorphisms}

\begin{snippetdefinition}{cat-theory-invertible-morphism-definition}{Invertible morphism}
    A morphism \(f: a \rightarrow b\) is \textit{invertible} if there
    is a function \(g\) that goes from \(b\) to \(a\)
    
    \[
        b:\quad b \rightarrow a
    \]
    
    such that
    
    \begin{align*}
        g \circ f &= \text{id}_b
        \\
        f \circ g &= \text{id}_a
    \end{align*}
    
    \begin{center}
        \begin{tikzcd}
            a \arrow[r, "f", bend left] & b \arrow[l, "g", bend left]
        \end{tikzcd}
    \end{center}
\end{snippetdefinition}

\begin{snippetdefinition}{cat-theory-isomorphism-definition}{Isomorphism}
    An \textit{isomorphism} is a bijective morphism.
    An isomorphism is labelled with \(\xrightarrow{\sim}\).
\end{snippetdefinition}

\plain{(mono and epic, but not every mono and epic
is an isomorphism).}

\section{Types of elements}

\subsection{Initial objects}

\begin{snippetdefinition}{initial-object-definition}{Initial object}
    An \textit{initial object} \(I\) is an object if
    for every object \(X\) in a category \(C\)
    there exists exactly one morphism \(I\to X\).
    \[
        \forall X\in \text{ob}(C) \,\exists_{=1}f\colon I\to X
    \]
\end{snippetdefinition}

\subsection{Terminal objects}

\begin{snippetdefinition}{terminal-object-definition}{Terminal object}
    A \textit{terminal object} \(T\) is an object if
    for every object \(X\) in a category \(C\)
    there exists exactly one morphism \(X\to T\).
    \[
        \forall X\in \text{ob}(C) \,\exists_{=1}f\colon X\to T
    \]
\end{snippetdefinition}

\subsection{Void}

\begin{snippetdefinition}{cat-theory-void-definition}{Void}
    The \textit{void} element is equivalent to the logical \textbf{false}.
    It is impossible to construct. Thus, functions that take void
    as an argument are impossible to call.
    It is impossible to map a set of the empty set, because
    any object canot have an image.
\end{snippetdefinition}

\plain{In the category of sets, the singleton is the only terminal initial object, as there exists}
\plain{a unique function (the empty function) from the empty set into any other given set.}

\subsection{Singleton}

\begin{snippetdefinition}{cat-theory-singleton-definition}{Singleton}
    A singleton is a set with one element (e.g. \(\{0\}\)) and
    is equivalent to the logical \textbf{true}.
    It can be constructed from nothing.
\end{snippetdefinition}

\plain{In the category of sets, any singleton is a terminal object, as there exists a unique}
\plain{function (the constant function) from any given set into the singleton.}

% \subsection{Bool}

\section{Types of categories}

\subsection{Thin categories}

\begin{snippetdefinition}{thin-category-definition}{Thin Category}
    A \textit{thin category} is a category in which each pair of objects
    has either \(0\) or \(1\) morphism.
    Every hom-set of a thin category has either \(1\) or \(0\) elements.
\end{snippetdefinition}

\subsubsection{Order categories}

\begin{snippetdefinition}{thin-category-definition}{Order category}
    An \textit{order category} is a thin category where morphisms represent relationships.
\end{snippetdefinition}

\begin{snippetexample}{order-category-example}{Order category}
    Given an equality relationship

    \begin{center}
        \begin{tikzcd}
            a \arrow[r, "\leq"] & b
        \end{tikzcd}
    \end{center}

    The relationship must be reflexive since there must exist an identity morphsim.

    \begin{center}
        \begin{tikzcd}
            a \arrow["\leq"', loop, distance=2em, in=215, out=145]
        \end{tikzcd}
    \end{center}

    It must also be composable and associative.
    The relations may for example be preorders, partial orders
    or total orders.
\end{snippetexample}

\subsection{Monoid}

\begin{snippetdefinition}{cat-theory-monoid-defintion}{Monoid}
    A \textit{monoid} is a category with a single object.
    It is equivalent to a set closed under an associative binary operation.
\end{snippetdefinition}

\subsection{Kleisli category}

% Given a category \(C\) the Kleisli category has an arrow
% from a to b (a and b are objects in C) which represent
% a morphism from a to (b, T)
% in the category C.

\subsection{Monad}

% A monad is this mapping from a to (b, T)

\subsection{Opposite Category}

\begin{snippetdefinition}{opposite-category}{Opposite Category}
    Given a category \(C\) we can defined an \textit{opposite category} \(\opp{C}\)
    where the objects are the same and the morphisms are inverted.
    
    \begin{center}
    % https://tikzcd.yichuanshen.de/#N4Igdg9gJgpgziAXAbVABwnAlgFyxMJZABgBpiBdUkANwEMAbAVxiRDpAF9T1Nd9CKMgEYqtRizYAjLjxAZseAkWGlR1es1aIQAY1m9FAogGZyYzZJ0duh-spRn14rdIPy+SwcgAsaixLaelxiMFAA5vBEoABmAE4QALZIZCA4EEiqLlYgMSDUDHRSMAwACp7GOnFY4QAWOO7xSZnU6UgATBqBbOH5IIXFZRUOINV1Dba5CcmIqW2IndlB4QAEADprulhxuit5k00zfmkZiGZLbDEAesAbODAAHjjAEGicnI3TSACsrafnliC11ua3uTxeb046022124Rud0ez1e7wAvMAABSrDZbHZ7ACUnARoKREPefQGJXKRhGY3qfTgtSwMQaswOX0QvxOSGOgJ6xLByMhn2anL+SBM7NFXPmPilR3FZ3lKUV7WViCy8zVckOKu5Gs4FE4QA
    \begin{tikzcd}
        a \arrow[d, "f"'] \arrow[rd, "g \circ f"] \arrow[rd] \arrow[d] &   &  & a                                      &                                                                                                                              \\
        b \arrow[r, "g"'] \arrow[r]                                    & c &  & b \arrow[u, "f^{\text{op}}"] \arrow[u] & c \arrow[lu, "f^{\text{op}} \circ g^{\text{op}}={(g \circ f)}^{\text{op}}"'] \arrow[l, "g^{\text{op}}"] \arrow[lu] \arrow[l]
    \end{tikzcd}
    \end{center}
    % REDO this picture because there are two arrows overlapping
\end{snippetdefinition}


\section{Universal Construction}

\begin{snippetdefinition}{universal-construction-defintion}{Universal Construction}
    Universal construction is used to define objects in terms of their
    relationships up to a unique isomorphism.
    This means that the objects that satisfy the same universal property
    are isomorphic.
    Its purpose is to define an object without knowledge about the object itself,
    but rather just the morphisms.
\end{snippetdefinition}

\section{Product}

\begin{snippet}{categorial-product-universal-construction}
The categorical product represents many operations
such as the cartesian product of sets.
A product of two objects \(a\) and \(b\) has the following
morphisms:
\begin{enumerate}
    \item \(p:a\times b \to a\), which returns the first value of the ordered pair
    \item \(q:a\times b \to b\), which returns the second value of the ordered pair
\end{enumerate}

In the following diagram \(c\) is the actual product and \(c'\) is a candidate object
for the cardinal product.
We rank a candidate \(c_i\) higher than \(c_j\) if there is a morphism
\(m:c_i \to c_j\).

\begin{minipage}{0.5\textwidth}
    \adjustbox{scale=1.5,center}{%
    \begin{tikzcd}
        & c' \arrow[d, "m"] \arrow[rdd, "q'", bend left] \arrow[ldd, "p'"', bend right] &   \\
        & c \arrow[ld, "p"] \arrow[rd, "q"']                                         &   \\
      a &                                                                                  & b
    \end{tikzcd}
    }
\end{minipage}
\begin{minipage}{0.5\textwidth}
    \begin{align*}
        p' &= p \circ m \\
        q' &= q \circ m 
    \end{align*}
\end{minipage}

Whenever \(m\) is \textit{flawed}, such as when it loses information
or does not preserve structure, we discard \(c_i\).

The universal property is that for any other product
\(c'\) with morphisms \(p'\) and \(q'\),
s a unique morphism
\[
    m:c'\to c
\]
such that
\begin{align*}
    p \circ m &= p' \\
    q \circ m &= q'
\end{align*}
\end{snippet}

\subsection{Coproduct}

\begin{snippet}{categorial-coproduct-universal-construction}
The coproduct represents operations
such as the disjoint union of sets.
The coproduct is the product when the morphisms are inverted.
The coproduct of two objects \(a\) and \(b\) has the following
morphisms:
\begin{enumerate}
    \item \(p:a \to a\sqcup b\)
    \item \(q:b \to a\sqcup b\)
\end{enumerate}

Let \(c=a\sqcup b\).

\begin{minipage}{0.5\textwidth}
    \adjustbox{scale=1.5,center}{%
    \begin{tikzcd}
        & c'                &                                                  \\
        & c \arrow[u, "m"'] &                                                  \\
        a \arrow[ruu, "p'", bend left] \arrow[ru, "p"'] &                   & b \arrow[luu, "q'"', bend right] \arrow[lu, "q"]
    \end{tikzcd}
    }
\end{minipage}
\begin{minipage}{0.5\textwidth}
    \begin{align*}
        p' &= m \circ p \\
        q' &= m \circ q
    \end{align*}
\end{minipage}

The universal property is that for any other object \(c'\)
with morphisms \(p':a \to c'\) and \(q':b \to c'\),
there exists a unique morphism \(m:c \to c'\)
such that
\begin{align*}
    p' &= m \circ p \\
    q' &= m \circ q
\end{align*}
\end{snippet}

\section{Functor}

\plain{A functor is a mapping from a category ...}

\end{document}

% this is so sick
% https://categorytheory.gitlab.io/functor_and_natural_transformation.html#index-0

% https://tikzcd.yichuanshen.de/
% https://youtu.be/LkIRsNj9T-8?list=PLbgaMIhjbmEnaH_LTkxLI7FMa2HsnawM_&t=772