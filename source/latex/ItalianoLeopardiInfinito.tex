\documentclass[preview]{standalone}

\usepackage{amsmath}
\usepackage{amssymb}
\usepackage{stellar}
\usepackage{bettelini}

\hypersetup{
    colorlinks=true,
    linkcolor=black,
    urlcolor=blue,
    pdftitle={Stellar},
    pdfpagemode=FullScreen,
}

\begin{document}

\title{Stellar}
\id{italiano-leopardi-infinito}
\genpage

\section{L'infinito}

% leggere articolo di Gilberto Lonardi

\begin{snippet}{leopardi-infinito-analisi}
    La poesia è un componimento di 15 endecasillabi sciolti. 

    \StellarPoetry{1}{
        Sempre caro mi fu quest'ermo colle, \\
        e questa siepe, che da tanta parte \\
        dell'ultimo orizzonte il guardo esclude. \\
        \textbf{Ma} sedendo e mirando, interminati \\
        spazi di là da \textbf{quella}, e sovrumani \\
        silenzi, e profondissima quiete \\
        io nel pensier mi fingo; ove per poco \\
        il cor non si spaura. E come il vento \\
        odo stormir tra queste piante, io quello \\
        infinito silenzio a questa voce \\
        vo comparando: e mi sovvien l'eterno, \\
        e le morte stagioni, e la presente \\
        e viva, e il suon di lei. Così tra questa \\
        immensità s'annega il pensier mio: \\
        e il naufragar m'è dolce in questo mare.
    }{XXX}

    La parola ermo significa solitario.
    Il colle si riferisce al Monte Tabor (Recanati).
    Il colle e la siepe sono due oggetti di cui l'autore può fare esperienza diretta
    (entrambi definiti come \quotes{questo}).
    Lo spazio viene diviso nello spazio fra l'occhio e la siepe
    e fra la siepe e l'infinito.
    La siepe è quindi un oggetto limitante, che impedisce la vista.

    Leopardi costruire nel pensiero (atto volontario e voluto),
    arrivando a concepire l'infinito.
    L'avversione del \quotes{ma} significa \quotes{nonostante}.
    L'autore si immagina l'infinito precisamente per la presenza della siepe.
    L'ostacolo alla vista è ciò che induce la costruzione dell'infinito, il desiderio
    di immaginare, nei pensieri.

    Questo ragionamento sposta l'autore dallo spazio fisico a quello immaginario,
    mediante l'esperienza sensibile della siepe.
    L'esperienza sensibile della siete è ormai lontana (\quotes{quella}).

    Noi tendiamo ad immaginarsi l'inifito perché l'inifito non ha confini, e questo ci da un brivido.
    L'autore sente il vento fra le piante, e questa esperienza lo riporta
    alla realtà sensibile, riportando l'infinito come concetto lontano.
    Il testo delinea quindi complessivamente un doppio movimento,
    verso l'infinito e indietro.

    Non vi è mai confusione fra i due piani (fra realtà sensibile e facoltà immaginativa), in quanto
    vengono sempre confrontate.
    Il primo inifnito ha più una connotazione spaziale, mentre il secondo
    è temporale.
    % sovvenire = ricordare. Mi si presenta alla mente il passato e il  presente
    % nella sua eternità.
    Il dolce naufragio è ossimero un ossimero in quanto
    il bridivo dell'infinito è piacevole, pur essendo pauroso.
    Il processo di transizione è reversibile e potenzialmente ripetibile.
    Un atto di volontà piacevolmente ripetibile.

    L'onda viene manifestata su tre livelli diversi:
    \begin{enumerate}
        \item \textbf{livello metrico-sintattico:} quattro enjambement particolamente forti;
            % interminati | sovrumani | quello |  questa
            Le parole come \textit{interminati}, \textit{sovrumani}, \textit{infinito} e \textit{immenso}
            cominciano con un prefisso che nega la finitezza.
            I due versi che rimangono autonomi, sena enjambement,
            sono il primo e l'ultimo.
            In questi due versi non vi è movimento, come se fossero due antipodi.
        \item \textbf{livello morfologico:};
        \item \textbf{livello lessicale:}.
    \end{enumerate}

    Questo è l'unico testo leopardiano privo di dolore.
\end{snippet}

\end{document}