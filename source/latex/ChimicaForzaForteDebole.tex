\documentclass[preview]{standalone}

\usepackage{amsmath}
\usepackage{amssymb}
\usepackage{bettelini}
\usepackage{stellar}
\usepackage[version=4]{mhchem}

\hypersetup{
    colorlinks=true,
    linkcolor=black,
    urlcolor=blue,
    pdftitle={Chimica},
    pdfpagemode=FullScreen,
}

\begin{document}

\title{Chimica}
\id{chimica-forza-forte-debole}
\genpage

\section{Lgami secondari (forze intermolecolari)}

\begin{snippetdefinition}{forza-forte-definition}{Forza forte}
    Legame covalente, metallico o ionico.
\end{snippetdefinition}

\begin{snippetdefinition}{forza-debole-definition}{Forza debole}
    Forze di Van der Walls, forze di Londom, ponte a idrogeno.
\end{snippetdefinition}

\begin{snippet}{forza-forte-debole-expl1}
    I legami secondari (deboli, intermolecolari) sono responsabili delle interazioni fra molecole uguali o diverse tra loro,
    o anche fra parti diverse della stessa molecola.

    Se il legame non è un ponte idrogeno ma è lo stesso principio, di dice dipolo-dipolo.
    Infatti, il legame ponte idrogeno è dipolo-dipolo ma ha un nome speicfico.
    Le forze di Van der Walls sono i legami dipolo-dipolo.
    Quando le interazioni non sono polari si parla di forze di London.
\end{snippet}

\section{Dissoluzione del sale nell'acqua}

\begin{snippet}{dissoluzione-sale-in-acqua}
L'acqua ed il sale Na\(^{+}\)Cl\(^{-}\) inducono un polo.
Le cariche positive dell'acqua (idrogeno) vengono attratte da quelle negative
del cloruro, mentre quelle negative dell'acqua (ossigeno)
vengono attratti da quelle negative del sale (Na).
Il cristallo del sale viene quindi separato dalle forze
esercitate dai dipoli dell'acqua.

Il motivo per cui il cristallo si spacca e non le molecole di acqua
è dato dal fatto che l'energia delle interazioni deboli è più che sufficiente
per compensare l'energia necessaria per rompere le interazioni ione-ione
nel cristallo e alcuni legami idrogeno acqua-acqua.
\end{snippet}

\section{Forze deboli nell'H2O}

\begin{snippet}{forze-deboli-acqua}
    I ponti a idrogeno creano una struttura esagonale con le molecole d'acqua,
    formando il ghiaccio.
    Questo è il motivo per cui la struttura dei giocchi di neve è esagonale.
    Quando l'acqua è gassosa non ci sono queste forze debole, e quando sono
    liquide ce ne sono poche e casuali.
    Il motivo è che l'energia aumuenta con l'aumentare della temperatura,
    e per cui con temperature troppe alte, questa energia spacca i legami deboli.

    Il sale nell'acqua salata rende più difficile la creazione di ponti a idrogeno.

    % anche il motivo per cui l'acqua ghiaccia solo sopra
    % foto esagono e fiocchi di neve
\end{snippet}

\end{document}
