\documentclass[preview]{standalone}

\usepackage{amsmath}
\usepackage{amssymb}
\usepackage{stellar}
\usepackage{definitions}

\begin{document}

\id{complex-analysis-equations-exercises}
\genpage

\section{Exercises}

\begin{snippetexercise}{complex-analysis-equation-ex-1}{}
    Solve the following equation in \(\complexnumbers\):
    \[
        {|z|}^2 + z + z^\complexconj + 1 = 0
    \]
\end{snippetexercise}

\begin{snippetsolution}{complex-analysis-equation-ex-1-sol}{}
    Let \(z=a+bi\). Then,
    \begin{align*}
        a^2 + b^2 + a + bi + a - bi 1 &= 0 \\
        a^2 + b^2 + 2a + 1 &= 0 \\
        {(a+1)}^2 + b^2 &= 0
    \end{align*}
    whose only solution is \(-1\).
\end{snippetsolution}

\begin{snippetexercise}{complex-analysis-equation-ex-2}{}
    Solve the following equation in \(\complexnumbers\):
    \[
        {(z^\complexconj)}^2 \cdot |z| + \frac{i}{z} = 0
    \]
\end{snippetexercise}

\begin{snippetsolution}{complex-analysis-equation-ex-2-sol}{}
    We first note that \(z\neq 0\).
    \begin{align*}
        {(z^\complexconj)}^2 z |z| &= -i \\
        z^\complexconj (z^\complexconj z) |z| &= -i \\
        z^\complexconj {|z|}^3 &= -i
    \end{align*}
    We now write \(z\) as \(z = re^{i\varphi}\)
    and thus \(re^{-i\varphi}r^3 = e^{-i\frac{\pi}{2}}\).
    The radius is given by \(r=1\) and \(\varphi = \frac{\pi}{2}\), so the solution
    is \(z=i\).
\end{snippetsolution}

\begin{snippetexercise}{complex-analysis-equation-ex-3}{}
    Solve the following equation in \(\complexnumbers\):
    \[
        {(z-1)}^5 \cdot |z-1| = \frac{8i}{{(z^\complexconj - 1)}^2}
    \]
\end{snippetexercise}

\begin{snippetsolution}{complex-analysis-equation-ex-3-sol}{}
    Let \(t = z-1\), so \(t^\complexconj = z^\complexconj-1\)
    \begin{align*}
       t^5 |t| &= \frac{8i}{{(t^\complexconj)}^2} \\
       t^5t^{-2}|t| &= 8i \\
       t^3t^2{(t^\complexconj)}^2|t| &= 8i \\
       t^3{|t|}^5&=8i
    \end{align*}
    Now let \(t = re^{i\varphi}\).
    \begin{align*}
        r^3 e^{3i\varphi} r^5 = 8 e^{i\frac{\pi}{2}}
    \end{align*}
    from which \(r = 2^{\frac{3}{8}}\)
    and \(3\varphi = \frac{\pi}{2} + 2k\pi\)
    and thus \(\varphi = \frac{\pi}{6} + \frac{2k\pi}{3}\)
    with \(k = 0,1,2\).
    We now substitute and get \(z = 2^\frac{3}{8}e^{i\frac{\pi}{6} + \frac{2k\pi}{3}} + 1\).
\end{snippetsolution}

\begin{snippetexercise}{complex-analysis-equation-ex-4}{}
    Solve the following equation in \(\complexnumbers\):
    \[
        |2z-1| = |2z + i|
    \]
\end{snippetexercise}

\begin{snippetsolution}{complex-analysis-equation-ex-4-sol}{}
    Let \(2z = a+bi\). Geometrically, the equation is asking for points where
    moving 1 unit to the left or 1 unit to the up, keeps the distance to the origin the same.
    We thus have
    \begin{align*}
        {(a-1)}^2 + b^2 &= a^2 + {(b+1)}^2 \\
        b = -a
    \end{align*}
    So, \(2z = a-ai \iff z = r-ri\) for some \(r \in \realnumbers\).
    This is the \set of points of the bisector of the second and fourth quadrant.
\end{snippetsolution}

\begin{snippetexercise}{complex-analysis-equation-ex-5}{}
    Solve the following equation in \(\complexnumbers\):
    \[
        z^3 z^\complexconj + \frac{{\left(\sqrt{3} + i\right)}^6}{{(1-i)}^2 z} = 0
    \]
\end{snippetexercise}

\begin{snippetsolution}{complex-analysis-equation-ex-5-sol}{}
    We not that \(z\neq 0\).
    \begin{align*}
        z^3 z^\complexconj z &= -\frac{{\left(\sqrt{3} + i\right)}^6}{{(1-i)}^2 z} \\
        z^3 {|z|}^2 &= -\frac{{\left(\sqrt{3} + i\right)}^6}{{(1-i)}^2 z} \\
    \end{align*}
    We need to write the right-hand member in exponential form:
    \(\sqrt{3} + 1 = 2e^{\frac{i\pi}{6}}\) and \(i-i = \sqrt{2} e^{\frac{-i\pi}{4}}\).
    Thus, by letting \(z=e^{i\varphi}\)
    \begin{align*}
        z^3 {|z|}^2 &= 2^5 i \\
        r^5 e^{3i\varphi} &= 2^5 i
    \end{align*}
    and finally we get \(r = 2\) and \(3\varphi = \frac{\pi}{2} + 2k\pi \implies \varphi = \frac{\pi}{6}+\frac{2}{3}k\pi\)
    with \(k=0,1,2\).
\end{snippetsolution}

\begin{snippetexercise}{complex-analysis-equation-ex-6}{}
    Solve the following equation in \(\complexnumbers\):
    \[
        z^6 z^\complexconj - z {\left(\frac{1 + i}{1-i}\right)}^2 = 0
    \]
\end{snippetexercise}

\begin{snippetsolution}{complex-analysis-equation-ex-6-sol}{}
    The solution is \(z=re^{i\theta}\) with
    \begin{align*}
        r \in \{0, 1\}
    \end{align*}
    and
    \begin{align*}
        \theta = \frac{\pi}{4} +k\frac{\pi}{2}
    \end{align*}
    where \(k\in\integers\), which yields \(5\) distinct solutions.
\end{snippetsolution}

\begin{snippetexercise}{complex-analysis-equation-ex-7}{}
    Solve the following equation in \(\complexnumbers\):
    \[
        z^2 + {|z|}^2 + z = 0
    \]
\end{snippetexercise}

\begin{snippetsolution}{complex-analysis-equation-ex-7-sol}{}
    \begin{align*}
        z^2 + {|z|}^2 + z &= 0 \\
        z^2 + zz^\complexconj + z &= 0 \\
        z(z + z^\complexconj + 1) &= 0
    \end{align*}
    yields the solution \(z=0\)
    and with \(z=a+bi\)
    \begin{align*}
        z + z^\complexconj + 1 &= 0 \\
        a+bi + a-bi + 1 &= 0 \\
        2\Re(z) &= - 1 \\
        \Re(z) &= -\frac{1}{2}
    \end{align*}
    which yields \(z = -\frac{1}{2} + ti\) for \(t\in\realnumbers\).
\end{snippetsolution}

\end{document}