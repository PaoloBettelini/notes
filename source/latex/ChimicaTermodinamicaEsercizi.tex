\documentclass[preview]{standalone}

\usepackage{amsmath}
\usepackage{amssymb}
\usepackage{bettelini}
\usepackage{stellar}
\usepackage[version=4]{mhchem}

\hypersetup{
    colorlinks=true,
    linkcolor=black,
    urlcolor=blue,
    pdftitle={Chimica},
    pdfpagemode=FullScreen,
}

\begin{document}

\title{Chimica}
\id{chimica-termodinamica-chimica-esercizi}
\genpage

\begin{snippetexercise}{effetto-equilibrio-ex}{Spiega l'effetto ottenuto sull'equilibrio da una reazione}
    \[ O_{2(\text{g})} + N_{2(\text{g})} \rightleftharpoons 2NO_{2(\text{g})} \]
    \begin{itemize}
        \item \textbf{aumento della temperatura:}
            se la reazione diretta è endotermica, scaldare favorirà la produzione di prodotti,
            mentre se la reazione è esotermica, raffreddare favorirà la formazione di prodotti;
        \item \textbf{diminuzione della pressione:}
            ci sono due moli di gas sia a destra che a sinistra, per cui non abbiamo nessun influsso.
        \item \textbf{aumento della concentrazione di ossigeno:}
            viene favorita la produzione di prodotti (spostamento verso destra);
        \item \textbf{diminuzione pressione parziale di N\({}_2\):}
            viene diminuita la concentrazione di azoto, e per cui avremo meno prodotti;
        \item \textbf{aumento della concentrazione NO:}
            aumentano i reagenti (viene favorita la reazione inversa);
        \item \textbf{presenza di un catalizzatore:}
            l'equilibrio non cambia.
    \end{itemize}
\end{snippetexercise}

\end{document}
