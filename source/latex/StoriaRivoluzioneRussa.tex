\documentclass[preview]{standalone}

\usepackage{amsmath}
\usepackage{amssymb}
\usepackage{stellar}
\usepackage{bettelini}

\hypersetup{
    colorlinks=true,
    linkcolor=black,
    urlcolor=blue,
    pdftitle={Stellar},
    pdfpagemode=FullScreen,
}

\begin{document}

\title{Stellar}
\id{storia-rivoluzione-russa}
\genpage

\section{La rivoluzione russa}

\begin{snippetdefinition}{bolscevismo-definition}{Bolscevismo}
    Il \textit{bolscevismo} è una corrente del pensiero
    politico marxista, sviluppatasi all'inizio del XX secolo all'interno
    del Partito Operaio Socialdemocratico Russo (POSDR) e
    concretizzatasi nella formazione del Partito bolscevico,
    poi Partito Comunista dell'Unione Sovietica (PCUS).
\end{snippetdefinition}

\begin{snippetdefinition}{menscevismo-definition}{Menscevismo}
    I \textit{menscevichi} furono una fazione del movimento rivoluzionario
    russo che emerse nel 1903 dopo una disputa tra Lenin e Julij Martov,
    entrambi membri del Partito Operaio Socialdemocratico Russo.
\end{snippetdefinition}

\begin{snippet}{fbdfd352-6014-45e2-933c-37927ad087dc}
    Le etimologie di questi due termini sono la minoranza (menscevichi)
    e maggioranza (bolscevichi).
    
    Al tempo delle rivoluzioni russe, il paese è molto arretrato su vari piani.
    
    \textbf{Piano economico:}
    \begin{itemize}
        \item paese enorme, arretrato sul piano politico ed economico;
        \item 170 Mio di abitanti, paese agricolo, campagne arretrate;
        \item rivoluzione industriale 1880-90 limitata a poche città (ma con produttività enorme). 
    \end{itemize}
    
    \textbf{Piano sociale:}
    \begin{itemize}
        \item nobiltà dominante, conserva il possesso della terra;
        \item borghesia quasi inesistente
        Grande massa di popolazione, composta prevalentemente da contadini (80\%),
        in maggioranza povera e senza terra o con piccoli apprezzamenti terrieri, dopo l'abolizione della servitù
        della gleba nel 1861;
        \item proletariato 2\% della popolazione nelle città industrializzate.
    \end{itemize}
    
    \textbf{Piano politoco:}
    \begin{itemize}
        \item sistema politico russo è uno dei più arcaici in Europa;
        \item Russia governata da uno zar in modo assolutistico e autoritario;
        \item libertà inesistenti o limitate.
    \end{itemize}
\end{snippet}

\begin{snippetdefinition}{domenica-sangue-definition}{Domenica di sangue}
    La \textit{domenica di sangue} (1905) fa riferimento all'eccidio compiuto a San Pietroburgo
    il 22 gennaio 1905 da reparti dell'esercito e della Guardia imperiale russa
    che aprirono il fuoco contro una manifestazione pacifica di dimostranti disarmati
    diretti al Palazzo d'Inverno per presentare una supplica allo zar Nicola II. 
\end{snippetdefinition}

\begin{snippetdefinition}{duma-definition}{Duma}
    Una \textit{Duma} è ogni diversa assemblea rappresentativa della Russia moderna e della Russia storica.
\end{snippetdefinition}

\begin{snippet}{55a6f028-d9ab-47ac-9d5b-52911d1c20da}
    Tendenzialmente i bolscevichi si rifanno al marxismo ma,
    se dovessero aspettare il collasso del capitalismo, ci vorrebbe probabilmente troppo
    tempo - il loro scopo è quindi quello di accelerare questo processo.
    La Russia è infatti il paese meno predisposto ad una rivoluzione marxista che ne ha ricevuta una,
    in quanto il capitalismo non era così sviluppato.
\end{snippet}

\section{La rivoluzione (liberale) di febbraio}

\begin{snippet}{bb270615-ef27-48e9-a7f1-3087a206abb7}
    \begin{itemize}
        \item Abdicazione dello zar;
        \item costituzione di un governo provissorio con elementi della media e alta borghesia, qualche socialista moderato, ma non i bolscevichi;
        \item governo provvisorio concede le libertà fondamentali d'opinione, espressione e riunione;
    \end{itemize}
    
    Nonostante ciò, non si pronuncia sulla distribuzione della terra ai contadini.
    Continua la guerra contro gli imperi centrali.
    
    % Definizione di Soviet (consigli). Vorrebbero una repubblica dei soviet.
    
    % Nazionalizzare le terre ? sotrarle ai contadini
    Bisogna convincere i cittadini che il governo dei Soviet è superiore
    con la rivoluzione.
    Assunzione da parte dei Soviet del pieno controllo sui mezzi di produzione: nazionalizzazione delle terre, delle banche e delle fabbriche.
\end{snippet}

\section{La rivoluzione di ottobre}

\begin{snippet}{833a502b-9ea4-4ade-a5b9-aa3137000ca1}
    La rivoluzionari bolscevica ha lo scopo di prendere il potere.
    \begin{itemize}
        \item Distribuzione della terra ai contadini;
        \item Abolizione della proprietà terriera;
        \item pace con austria e germania (Trattato di Brest -Litovsk).
    \end{itemize}
    
    % assemblea
    Nonostante le votazioni non portino i bolscevichi a vincere,
    spodesteranno gli altri partiti con la forza e si giungerà ad una guerra civile.
\end{snippet}

\section{Guerra civile}

\begin{snippetdefinition}{guerra-civile-russa-definition}{Guerra civile russa}
    La \textit{guerra civile russa} fu un sanguinoso conflitto che scoppiò in Russia in
    seguito alla Rivoluzione d'ottobre e alla presa del potere da parte dei bolscevichi,
    combattuta tra questi, detti "rossi", e vari gruppi controrivoluzionari,
    detti "bianchi", ufficiali fedeli allo zar nel tentativo di restaurare l'antico regime,
    appoggiati da una coalizione di paesi quali Regno Unito, Stati Uniti d'America e Francia.
    I rossi ottennero la vittoria finale nel conflitto che per tre anni aveva
    travagliato il paese (dal 1918 al 1920), liquidando le forze
    controrivoluzionarie e instaurando il loro potere su tutto il
    territorio della nascente Unione Sovietica.
\end{snippetdefinition}

\begin{snippet}{f953695c-6cb8-4c59-8720-12fcefe076d6}
    Le misure dei bolscevichi sono:
    \begin{itemize}
        \item l'istituzione della Ceka (polizia segreta politica);
        \item tribunale rivoluzionario (tribunale per condannare chi agisce contro il regime bolscevico);
        \item partiti politici di opposizione messi fuori legge;
        \item reintrodotta la pena di morte;
        \item censura sulla stampa.
    \end{itemize}
    Questi elementi portano alla riduzione della libertà di operai, contadini, intellettuali
    nei soviet (per via del monopartitismo).
    
    Dopo aver preso il potere con la rivoluzione di ottobre, i bolscevichi iniziano la trasformazione
    dello stato. Questa trasformazione è un percorso che punta ad eliminare la concorrenza
    degli altri partiti (anche fisicamente, come nei processi di Stalin),
    a mettere le mani sulle istituzioni dello Stato.
    Il risultato di questo è che \textbf{il partito e Stato coincidono},
    e quindi convergono nel PCUS.
\end{snippet}

\begin{snippetdefinition}{sistema-monopartitico-definition}{Sistema monopartitico}
    Un \textit{sistema monopartitico} è un sistema politico in cui la legge
    permette ad un solo partito di governare.
\end{snippetdefinition}

\begin{snippet}{28b691f2-40df-4476-a3d5-2a2069c6a546}
    % ANALISI dipinto quello che sembra una roba di arte moderna
    % le freccie rosse indicano il controllo dal partito VERSO lo stato
    La più alta carica amministrativa del partito, che coprirà Stalin,
    è quella del segretario generale (segretariato).
    Esso controlla l'applicazione delle decisioni del partito e soprattutto controlla
    la selezione dei quadri dirigenti.
\end{snippet}

\section{Da Lenin a Stalin}

\begin{snippet}{77973904-67db-4bc4-8cbb-cf7c37546cfd}
    Nel dicembre del 1922 venne proclamata l'URSS, Unione delle Repubbliche
    Socialiste Sovietiche, nel 1924 moriva Lenin. All'interno del partito bolscevico si aprì
    la lotta per la successione che si affiancò alle diverse strategie relative allo sviluppo
    politico-economico dell'Urss. I protagonisti principali dello scontro furono Stalin e
    Trotzkij. Il primo sosteneva la necessità, dopo i fallimenti delle rivoluzioni
    comuniste nei paesi dell'Europa occidentale, di consolidare i risultati della
    rivoluzione secondo la formula del "socialismo in un solo paese". Il secondo la
    necessità di allargare il processo rivoluzionario a tutti i paesi capitalisti secondo la
    formula della "rivoluzione permanente". Forti erano anche le divergenze sulla
    relazione tra agricoltura e industria, anche se ambedue concordavano
    sull'importanza di un intervento deciso da parte dello stato per accelerare lo
    sviluppo economico. Stalin ottenne la maggioranza del partito contro Trotzkij, che
    venne espulso e deportato.

    Nel 1928 venne deciso il primo piano quinquennale di sviluppo il cui obiettivo era
    una rapida industrializzazione della Russia. Le campagne avrebbero dovuto fornire
    uomini e capitali senza ridurre la produttività. Per raggiungere questo scopo fu
    decisa la collettivizzazione delle terre creando grandi aziende gestite direttamente
    dallo stato, i sovchoz, o da cooperative di produzione, i kolchoz.

    Pur tra difficoltà il primo piano si concluse con successo e fu seguito da altri con
    obiettivi sempre più elevati e l'URSS, proprio mentre il mondo occidentale era
    colpito da una grave crisi di recessione, divenne la seconda potenza industriale del
    mondo, sotto il rigido controllo di un apparato burocratico sempre più diffuso e
    opprimente.

    Dal testamento di Lenin è possibile notare che,
    secondo lui, Stalin non è adatto come Segretario Generale, in quanto troppo grossolano.
    Non mancano le doti da politico, ma mancano le doti da leader politico.
\end{snippet}

% cos'è una rivoluzione permamente nel lesso storiografico?

\section{Schema rivoluzioni}

\begin{snippet}{rivoluzioni-russe-schema}
    \begin{center}
    \begin{tabular}{|l|l|l|}
        \hline & \multicolumn{1}{|c|}{ Febbraio 1917 } & \multicolumn{1}{c|}{ Ottobre 1917 } \\
        \hline Quando? & Dal 22 febbraio al 2 marzo & Notte dal 24 al 25 ottobre \\
        \hline Dove? & Pietrogrado & Pietrogrado \\
        \hline Chi? & Il popolo, soldati, operai, contadini... & I bolscevichi \\
        \hline Come? & Manifestazione spontanea & Piano preparato \\
        \hline Perché? & \begin{tabular}{l} 
        Malcontento \\
        Disordini dovuti alla fame
        \end{tabular} & Prendere il potere \\
        \hline Risultato & \begin{tabular}{l} 
        Caduta dello zarismo. \\
        Instaurazione di un Governo \\
        provvisorio (composto da elementi \\
        appartenenti alla borghesia liberale e \\
        da qualche socialista moderato).
        \end{tabular} & \begin{tabular}{l} 
        Rovesciamento del governo \\
        provvisorio \\
        Instaurazione di un \\
        governo bolscevico
        \end{tabular} \\
        \hline
    \end{tabular}
    \end{center}
    \phantom{}
\end{snippet}

\section{Lo Stalinismo}

\begin{snippetdefinition}{piano-quinquennale-definition}{Piano quinquennale}
    Il \textit{piano quinquennale} è uno strumento di politica economica utilizzato
    nei paesi socialisti dove l'iniziativa economica è in larga parte gestita da enti pubblici.
    Un piano quinquennale individua determinati obiettivi da raggiungere in un periodo di
    cinque anni nei vari ambiti dell'economia,
    per uno sviluppo anche nel settore industriale.
    Gli obiettivi consistono in una definita quantità fisica di beni che dovranno essere prodotti. 
\end{snippetdefinition}

\begin{snippet}{9e4b8b7e-f295-494c-a329-519b787be249}
    La modernizzazione della Russia
    \begin{itemize}
        \item presupposto: avere il controllo completo su tutta l'economia e fare una pianificazione;
        \item pianificazione centralizzata delle scelte economiche (piani quinquennali);
    \end{itemize}

    Viene depotenziata la produzione di beni personali ma vengono aumentata l'industria pesante.
    La qualità di vita del sovietici rimane quindi ben al di sotto degli altri paesi industrializzati.

    Ma dove si trovano le risorse da immettere nella pianificazione?
    Viene lanciata una campagna propagandistica che, prendendo spunto dalle eccezionali prestazioni
    lavorative di Alexei Stakhanov, esaltava le imprese degli eroi del lavoro.
\end{snippet}

\begin{snippetdefinition}{stacanovismo-definition}{Stacanovismo}
    Lo \textit{stacanovismo} è stato un movimento di massa e di propaganda
    sovietica, iniziato durante il secondo piano quinquennale
    dell'URSS durante il secondo quarto del XX secolo. 
\end{snippetdefinition}

\begin{snippet}{stacanovismo-expl}
    Deriva ideologicamente da un particolare sistema di utilizzo
    dell'attrezzatura di scavo e divisione ed organizzazione del lavoro,
    ideato da un operaio russo dell'Unione Sovietica del bacino del Don,
    Aleksej Grigor'evič Stachanov.
    Quest'ultimo, infatti, estrasse con una tecnica di divisione dei
    ruoli lavorativi di sua ideazione, la notte del 31 agosto 1935,
    102 tonnellate di carbone, pari a quattordici volte la quota prevista,
    in meno di sei ore.

    La sua immagine fu utilizzata allo scopo di aumentare la produttività di
    tutti i lavoratori, con l'obiettivo non secondario di
    \quotes{dimostrare al mondo} l'efficacia del sistema del lavoro socialista;
    il suo esempio diede vita al cosiddetto stacanovismo,
    l'aumento della produttività individuale unita all'ideazione
    di nuove tecniche di lavoro. Culmine dell'ideologia produttivistica
    dell'industrializzazione forzata, l'emulazione stacanovista divenne un
    fenomeno di massa. Essa comportava premi di produzione in benefici e
    in denaro e nello stesso tempo costringeva gli operai a raggiungere indici di
    produttività migliori.
\end{snippet}

\begin{snippet}{statalizzazione-agricoltura}
    I contadini vengono spronati ad abbandonare volontariamente le loro terre
    (collettivizzazione delle campagne).
    Lo scopo è quello di creare delle fattorie collettive, più grandi,
    ove i mezzi si produzione vengono messi assieme e lavorate da delle famiglie
    di agricoltori.
\end{snippet}

\begin{snippetdefinition}{kulaki-definition}{Kulaki}
    I \textit{kulaki} erano una categoria di contadini
    presente negli ultimi anni dell'Impero russo, e nei primi
    della neonata Unione Sovietica, finché nel 1924,
    con la morte di Lenin, prese il potere Stalin, che diede il
    via alla collettivizzazione e i kulaki divennero a tutti gli effetti
    nemici dello stato. Iniziò così un vero e proprio rastrellamento nelle
    campagne, e moltissimi finirono nei gulag.
    La parola kulaki inizialmente si riferiva a contadini indipendenti
    della Russia, che possedevano grandi appezzamenti di terreno ed
    utilizzavano mezzadri; successivamente il termine fu utilizzato
    spregiativamente dai bolscevichi per indicare i contadini agiati. 
\end{snippetdefinition}

\begin{snippetdefinition}{holodomor-definition}{Holodomor}
    \textit{Holodomor} è il nome attribuito alla carestia nel territorio dell'Ucraina dal 1932 al 1933, che causò alcuni milioni di morti.
\end{snippetdefinition}

\begin{snippet}{holodomor-expl}
    La carestia nasce dai contadini che si rifiutano si aggregarsi al piano statale,
    rifiutandosi di seminare le terre, macellando gli animali prima
    di essere forzati a metterli in comune nelle fattorie collettive.
    Questa carestia ha causato circa 7 milioni di morti, tra cui molti bambini.
    L'Ucraina non è l'unico posto che è stato affetto da questa carestia,
    ma è stato il posto che ha subito il maggior danno.
\end{snippet}

% Questione delle penne per i presunti terroristi e terroristi, che non si possono difendere
% confessavano per tortura psicologica
% Grigory Zinoviev
% e Lev Kamenev

% Bukharin
% NKVD

\end{document}





%%%%%%%%%%


%%%%%%%%%%


%%%%%%%%%%


%%%%%%%%%%


\section{Correzione Verifica 1}

\section{Prima domanda}

\textbf{Contesto storico (periodo):}
\begin{enumerate}
    \item Restaurazione (cos'è la Restaurazione), periodizzazione e caratteristiche;
    \item potenze della restaurazione vs Impero napoleonico;
    \item congresso di Vienna (1815).
\end{enumerate}

\textbf{Idee de Maistre:}
\begin{enumerate}
    \item Contrario alla Rivoluzione francese e ai suoi principi (principi Rivoluzione francese: quali sono in generale?);
    \item contrario al liberalismo (aspetti principali come uguaglianza giuridica, monarchia costituzionale/governo rappresentativo...);
    \item unione tra Chiesa e sovranità;
    \item sovranità è tale per diritto divino (sovranità dall'alto e no per volere della nazione, da cittadini si torna ad essere sudditi).
\end{enumerate}

\textbf{Corrente ideologica dell'autore:} \\
L'autore è un conservatore (reazionario). \\
Caratteristiche del movimento conservatore:
\begin{enumerate}
    \item Sua genesi (vs Illuminismo, Rivoluzione francese e liberalismo);
    \item anti egalitario, difesa della gerarchie;
    \item mutamenti graduali e non violenti della società;
    \item timore verso il futuro; opposizione al progresso e alle nnovazioni, difesa delle tradizioni;
    \item difesa di una società corporativa contro le libertà individuali.
\end{enumerate}

\section{Seconda domanda}

\textbf{Successo nella classe operaia dovuto a:}
\begin{enumerate}
    \item Volontà di risolvere concretamente la questione sociale apparsa con l'industrializzazione; volontà di trovare soluzioni per la drammatica situazione della classi lavoratrici, ponendo fine allo sfruttamento borghese dei proletari;
    \item aspirazione all'uguaglianza formale e sostanziale tramite l'abolizione della proprietà privata e della società divisa in classi (obiettivi del socialismo) di fronte a delle condizioni disumane;
    \item l'idea che la classe operaia attraverso la sua lotta spontanea e sempre più organizzata sia il veicolo di una trasformazione radicale che possa eliminare lo sfruttamento e creare una società di uomini liberi e uguali (messianismo: la classe operaia come portatrice della speranza di una vita migliore sulla terra).
\end{enumerate}

\textbf{Ideologia dominante rispetto alle altre di ispirazione socialista perché:}
\begin{enumerate}
    \item È concreto, non utopico, mira  a trovare delle soluzioni pratiche, scientifiche alla questione sociale.
    \item Si fonda su un'attenta e scientifica analisi della realtà: non si limita a criticare la società capitalistica e a sognarne una migliore, piuttosto ad abolire la prima e costruire materialmente la seconda:
        Definizione di "materialismo storico dialettico", o di "socialismo scientifico". La combinazione di tre ambiti di studio: l'economia politica inglese, la filosofia del diritto tedesca e le scienze storiche e politiche francesi.
    \item Prevede che le contraddizioni del capitalismo lo porteranno inevitabilmente alla sua autodistruzione attraverso una lotta di classe e un antagonismo iscritto nelle dinamiche stesse del capitalismo, così come dimostrato dalla legge del valore lavoro: il profitto capitalista è inversamente proporzionale al salario distribuito alla classe operaia.
\end{enumerate}