\documentclass[preview]{standalone}

\usepackage{amsmath}
\usepackage{amssymb}
\usepackage{stellar}
\usepackage{tabularx}
\usepackage{bettelini}

\hypersetup{
    colorlinks=true,
    linkcolor=black,
    urlcolor=blue,
    pdftitle={Stellar},
    pdfpagemode=FullScreen,
}

\begin{document}

\id{storia-imperialismo}
\genpage

\section{L'età dell'imperialismo (1870-1914)} % alcuni fino al 45

\begin{snippetdefinition}{colonialismo-definition}{Colonialismo}
    Con \textit{colonialismo} si intende la tendenza di uno stato o di un popolo ad acquisire
    il dominio e il controllo politico
    o economico, diretto oppure indiretto, su un altro stato o su un altro popolo.
\end{snippetdefinition}

\begin{snippetdefinition}{colonia-definition}{Colonia}
    Una \textit{colonia} è il possedimento di uno Stato in un territorio
    lontano e abitato da popolazioni che non godono
    degli stessi diritti civili dei gruppi di persone che vengono dallo stato dominante
    che pratica il colonialismo nei loro confronti.
\end{snippetdefinition}

\begin{snippet}{motivi-colonialismo}
    I motivi per il colonialismo sono i seguenti:
    \begin{itemize}
        \item impossessarsi di materie prime per le proprie industrie;
        \item aprire nuovi mercati dove esportare i propri prodotti;
        \item mostrare la grandezza e la superiorità dello stato;
        \item civilizzare i popoli extraeuropei (obbligo morale).
    \end{itemize}
\end{snippet}

\begin{snippetdefinition}{imperialismo-definition}{Imperialismo}
    Con \textit{imperialismo} si intende un'estensione del colonialismo
    in uno stato con grande espansione del capitalismo.
    Esso considera fattori economici ed è volto alla costituzione di Imperi coloniali da parte
    delle potenze industriali europee, con lo scopo di procurarsi materie prime necessarie all'industria ed esportarvi prodotti finiti.
\end{snippetdefinition}

\begin{snippetdefinition}{eta-imperialismo-definition}{Età dell'imperialismo}
    Con \textit{età dell'imperialismo} si intende un periodo
    caratterizzato da una grande ricerca senza precedenti di acquisizioni territoriali.
    In particolare, Francia, GB, Germiania, Belgio, Olanda, Italia, e all'estero
    USA e Giappone sono le nazioni esponenti di questo fenomeno
    fra circa il 1870 e il 1914/15, alcuni fino al 1945.
    Infatti, in questo periodo nascono l'\textbf{Impero tedesco} e l'\textbf{Impero Austro-Ungarico}.
    L'acquisizione di nuove colonie fa sì che questi stati si definiscano quindi imperi.
\end{snippetdefinition}

\begin{snippetdefinition}{impero-tedesco-definition}{Impero tedesco}
    L'\textit{impero tedesco}, noto anche come \textit{impero germanico}
    o \textit{Secondo Reich} è lo stato monarchico che governò i
    territori della Germania nel periodo che va dal conseguimento
    dell'unificazione tedesca il 18 gennaio 1871 fino all'abdicazione del
    Kaiser Guglielmo II il 9 novembre 1918. 
\end{snippetdefinition}

\begin{snippetnote}{kaiser-significato}{Kaiser}
    Con \textit{kaiser} si intende un titolo imperiale per indicare il governante.
\end{snippetnote}

\begin{snippetdefinition}{impero-austro-ungarico-definition}{Impero Austro-Ungarico}
    L'Impero austro-ungarico fu uno Stato dell'Europa centrale nato nel 1867
    inteso a riformare l'Impero austriaco nato nel 1804.
\end{snippetdefinition}

\begin{snippet}{ef890e64-0850-4600-bf9f-39a6ebc24499}
    Vi è una sovrapproduzione e i mercati europei non sono più in grado di assorbire ciò che le industrie producono.
    Queta crisi ha protato alla Grande Depressione, la quale ha portato a investire capitali 
    e a vendere prodotti che non si collocano in Europa.
\end{snippet}

\begin{snippetdefinition}{grande-depressione-definition}{Grande depressione}
    La \textit{Grande depressione} (detta anche \textit{Grande crisi} o
    \textit{Crollo di Wall Street}) fu una grave crisi economica e finanziaria che
    sconvolse l'economia mondiale alla fine degli anni venti,
    con forti ripercussioni anche durante i primi anni del decennio successivo.
\end{snippetdefinition}

\begin{snippet}{3fae6716-0fdd-4503-8d7c-a0cae0eac04e}
    La crisi inizia negli Stati Uniti d'America e porta al definitivo crollo
    (crack) della Borsa di New York del 24 ottobre 1929 (giovedì nero) dopo anni di boom azionario. 
    
    Con la seconda rivoluzione industriale, la catena di montaggio nelle fabbriche
    permise una produzione pressoché illimitata di merci a un costo più basso.
    
    In trent'anni si passa da un 10\% di Africa colonizzata (1884, poche colonie sulle coste)
    a più del 90\% di
    colonizzazione del continente (uniche eccezioni: Etiopia e Liberia) nel 1910.
    Questa volta alla colonizzazione hanno partecipato anche Giappone, Russia e Stati Uniti, unici stati
    extra-europei che si sono concentrati soprattutto sull'Asia.
    Prime colonie: Egitto (1882 \(\rightarrow\) Gran Bretagna) e Tunisia (1881 \(\rightarrow\) Francia), che appartenevano
    all'Impero Ottomano da tempo in declino, quindi troppo debole per avere il controllo effettivo di
    queste terre.
\end{snippet}

\section{La questione del Congo}

\begin{snippetdefinition}{conferenza-berlino-definition}{Conferenza di Berlino}
    La \textit{Conferenza di Berlino} del 1884-1885,
    detta anche Conferenza dell'Africa Occidentale o Conferenza sul Congo,
    regolò il commercio europeo in Africa
    centro-occidentale nelle aree dei fiumi Congo e Niger e sancì la
    nascita dello Stato Libero del Congo.
    L'obiettivo è quello di spartire su carta i territori tra le diverse potenze europee. 
    I confini vengono tracciati senza criteri antropologici,
    ma solo tenendo conto dell'interesse economico
    delle nazioni europee.
\end{snippetdefinition}

\begin{snippet}{congo-expl}
    Il Congo è il territorio che inizialmente scatenò i conflitti più duri.
    Dal 1876 il Belgio aveva forti interessi economici nella regione, nella quale
    erano stati scoperti ricchi giacimenti minerari che spinsero re Leopoldo II a cercare
    uno sbocco sull'Atlantico. Il Portogallo controllava la vicina Angola riteneva al zona di propria competenza.
\end{snippet}

\begin{snippet}{caucciu-congo-expl}
    Durante la Conferenza di Berlino viene deciso che il Congo verrà assegnato
    direttamente a Leopoldo II (1885), quello che lui chiamerà lo Stato indipendente del Congo.
    Leopoldo II è interessato alla zona del bacino del Congo, ricca di
    minerali. Incarica un esploratore di perlustrare il territorio percorrendo il fiume e di stipulare
    contratti ingannevoli con i capitribù locali. Leopoldo II è interessato al caucciù, una resina
    ricavata incidendo la corteccia di alberi particolari, considerata il precursore della plastica.
    Durante la conferenza di Berlino viene deciso che il Congo diventerà un possedimento
    privato del sovrano belga a partire dal 29 maggio 1985, che lo proclama ufficialmente come
    stato indipendente del Congo. Gli ideali pubblicizzati dal sovrano sono la promozione delle
    ricerche geografiche e scientifiche, la lotta ai mercanti di schiavi arabi e la diffusione della
    civiltà e del progresso. Leopoldo II recluta le persone locali per raccogliere il caucciù e
    trasportarlo lungo i sentieri fino al mare. Ogni villaggio doveva consegnare una certa
    quantità di caucciù: chi si rifiutava, o consegnava quantità minori di quelle richieste, veniva
    punito dalla Forze Publique. La Forze Publique è una sorta di polizia dello stato, composta
    da persone locali, che assicurano lo svolgimento del compito effettuando punizioni e
    mutilazioni. Contro i ribelli, invece, si ricorreva all'assassinio o alla presa in ostaggio delle
    famiglie.
\end{snippet}

\section{Jules Ferry, Discorso al parlamento francese (1885)}

\begin{snippet}{discorso-jules-ferry-parlamento}
    Jules Ferry, sindaco di Parigi, si rivolge al parlamento e
    discute della colonizzazione, collegandola all'espansione economica e politica.
    Sostiene che la colonizzazione offra sbocchi commerciali e afferma la superiorità delle razze
    nell'obbligo di civilizzare quelle considerate inferiori.
    Viene contestato riguardo ai diritti umani e alla giustificazione della forza nella
    gestione delle colonie.
    Ferry insiste sull'importanza della colonizzazione per la grandezza nazionale e
    avverte che il Paese rischia la decadenza senza un'adeguata politica coloniale.
    
    La Francia è destinata ad una forte espansione ed è alla ricerca di sbocchi commerciali,
    e non voglio perdere la loro posizione di stato coloniale come la Spagna o il Portogallo.
    Il testo è molto imperialista, nazionalista, razzista e militarista.
    
    Come fattori economici abbiamo:
    \begin{itemize}
        \item ricerca di materie prime per la seconda Rivoluzione Industriale;
        \item ricerca di mercati (sblocchi commerciali, espansione del capitalismo, investimenti, \(\cdots\)).
    \end{itemize}
    Oltre ai fattori economici vi sono tuttavia i fattori culturali e ideologici:
    \begin{itemize}
        \item aspetto umanitario e civilizzatore;
        \item (\(\implies\) esistono razze superiori con il dovere morale di civilizzare quelle inferiori)
    \end{itemize}
    e i fattori politici (nazionalismo e militarismo):
    \begin{itemize}
        \item gli Stati fra loro sono rivali, e non possono soccombere come la Spagna o il Portogallo;
        \item la politica coloniale è un mezzo per dimostrare la propria potenza e supremazia.
    \end{itemize}
\end{snippet}

\section{Politica imperialista}

\begin{snippet}{politica-imperialistica-expl}
    Vi è un rafforzamento dell'esercito e una politica estera aggressiva, che porta a un
    rafforzamento del nazionalismo. Il rafforzamento del nazionalismo si manifesta attraverso la
    superiorità nazionale e l'esclusivismo nazionale. L'esclusivismo nazionale è la tendenza
    politico-economica di uno Stato ad accordare determinati privilegi a società private e a
    favorire il monopolio. I sistemi politici hanno la necessità di “insegnare la nazione” perché
    vogliono fornire una legittimazione all'operato dei governi. L'insegnamento è possibile
    tramite la scuola, l'esercito e i rituali pubblici.
\end{snippet}

\subsection{Scolarizzazione}

\begin{snippet}{scolarizzazione-expl}
    Con \textit{insegnare la nazione} si indica la necessità di educare circa il nazionalismo
    chi era analfabeta da parte dei sistemi politici.
    Questo è necessario e possiede lo scopo di fornire una legittimazione e
    giustificazione allo Stato per le èlite politiche.
    Nei casi più estremi, come nel nazionalsocialismo, tutte le ideologia
    contrarie a quello che vuole lo Stato (come alcuni libri) devono sparire.
    L'intento della \textit{scolarizzazione} è quindi quello,
    con la scuola elementare obbligatoria, di accendere i sentimenti patriottici nei bambini,
    con particolare attenzione alla storia e letteratura nazionale (esaltazione degli eroi nazionali).
\end{snippet}

\subsection{Esercito}

\begin{snippet}{politica-imperialistica-esercito-expl}
    È l'elemento che forse più di tutti ha contribuito alla nazionalizzazione delle masse,
    ovvero il fenomeno di pedagogia della nazione (insegnamento).
    Tutti i maschi devono compiere il servizio militare e dunque fare un servizio per la patria.
    Il saluto alla bandiera rappresenta il simbolo della nazione, è un rituale
    interno che serva a rafforzare l'appartenenza alla nazione.
\end{snippet}

\subsection{Rituali pubblici}

\begin{snippet}{rituali-pubblici-expl}
    In questo periodo si stabiliscono una serie di simboli e rituali che servono ad onorare
    l'appartenenza a una nazione. Molti stati definiscono le bandiere nazionali, gli inni nazionali
    e le feste nazionali. L'inno nazionale svizzero è stato scritto nel 1841, viene dichiarato tale
    in ambito militare diplomatico nel 1961 e ufficialmente nel 1981.
\end{snippet}

\section{Differenza fra nazionalismo e principio di nazionalità}

\begin{snippet}{differenza-nazionalismo-e-principio-nazionalita}
    \begin{table}[htbp]
        \begin{tabularx}{\textwidth}{p{2cm}|p{6cm}|p{6cm}}
            \hline
            & \textbf{Principio di nazionalità} & \textbf{Nazionalismo} \\
            \hline
            \textbf{Periodo} & Prima metà del XIX secolo & Fine del XIX secolo - metà del XX secolo \\
            \hline
            \textbf{Definizione} & Consapevolezza dell'identità
            culturale e storica del proprio popolo & Consapevolezza della superiorità
            culturale e razziale del proprio popolo \\
            \hline
            \textbf{Idea di nazione} & La nazione si fonda sulla volontà dei
            cittadini e sull'autonoma decisione
            (libertà e democrazia) & La nazione si fonda sull'espressione
            della naturale diversità (affermazione
            e superiorità di un popolo) \\
            \hline
            \textbf{Rapporti con le nazioni} & Le nazioni non sono rivali tra di loro e
            i loro rapporti sono regolati dal diritto
            internazionale & Le nazioni sono rivali tra di loro e i
            rapporti si basano sulla legge del più
            forte \\
            \hline
            \textbf{Fondamenti culturali} & Ricerca delle radici storiche del
            popolo, valorizzazione del principio
            della libertà e della fratellanza & Esaltazione dell'idea di superiorità
            razziale, della forza e della guerra \\
            \hline
            \textbf{Finalità politiche} & Lotta per la libertà e l'indipendenza
            di tutti i popoli & Conquista di nuovi territori e
            affermazione del proprio dominio \\
            \hline
        \end{tabularx}
    \end{table}
    \phantom{}
\end{snippet}

\section{Il razzismo scientifico}

\begin{snippetdefinition}{razzismo-scientifico-definition}{Razzismo scientifico}
    Il \textit{razzismo scientifico} appare per la prima volta in un vocabolario scientifico nel 1694,
    durante la rivoluzione scientifica e illuminismo.
\end{snippetdefinition}

\begin{snippet}{razzismo-scientifico-expl}
    Il collegamento fra l'illuminismo e il razzismo scientifico, è dato dal fatto che più
    la bellezza (oggettiva) esteriore è misurabile, più è misurabile la moralità e intellettualità della persona.
    Più siamo moralmente superiori dentro di noi, più siamo oggettivamente belli.
    La perfezione esteriore indica una bellezza interiore, un'essere moralmente alto.
    
    La concezione di razzismo è quindi inizialmente culturale e non genetica,
    ed essa nasce dalla razionalità illuminista che cerca di posizionare l'uomo nell'universo (antropocentrismo).
    Questi criteri di virtù e bellezza derivano dall'osservazione della natura e dai classici,
    portando ad una connessione fra scienza ed estetica, che fa classificare le razze
    in base al loro posto nella natura (gerarchizzazione misurando le misure estetiche, che
    sono direttamente correlate alla mente).
    
    Un esempio pratico ne è la geometrizzazione facciale di Peter Camper (1722-1989),
    che misurò l'angolo facciale. Più il cranio è sviluppato in una certa maniera (angolo tendente ad una linea verticale),
    più le facoltà intellettuali sono alte. In realtà, non si
    tratta solo di bellezza estetica, ma anche morale: l'appartenenza esteriore rispecchia
    la grazia interiore.
\end{snippet}

\begin{snippetdefinition}{darwinismo-sociale-definition}{Darwinismo sociale}
    Con \textit{darwinismo sociale} si intende una teoria nata negli anni 1870-80, secondo la quale ogni comunità
    funziona in base alle leggi naturali descritte da Charles Darwin nella sua teoria
    dell'evoluzione: anche nella società umana, i più capaci avrebbero la meglio sui meno
    capaci, così come nella lotta per la sopravvivenza. Il fatto che nella lotta per la sopravvivenza
    delle nazioni alla fine vincessero quelle più potenti fu sfruttato dall'imperialismo come una
    sorta di legittimazione biologistica.
\end{snippetdefinition}

\begin{snippet}{c74b2c3b-ca9c-4f16-8501-f4b8d9bfe03c}
    Le \textit{teorie razziste} distinguono una superiorità biologica culturale.
    Fra il diciottesimo e diciannovesimo secolo nasce l'ideologia razzista come percepita nel mondo odierno.

    Secondo Joseph-Arthur Gobineau, la decadenza delle civiltà è data dall'innata
    diversità delle razze. La razza Aria (dell'elemento germanico),
    va primatizzata, attuando la discriminazione delle razze superiori.\\
    Secondo l'autore, in cima alla scala gerarchica vi è la razza bianca.
    Un sottoinsieme della razza bianca è quella ariana.
    La razza bianca deve essere preservata (non si deve mischiare con quella gialla o nera).
\end{snippet}

\begin{snippetdefinition}{eugenetica-definition}{Eugenetica}
    Disciplina nata verso la fine dell'Ottocento che, basandosi su considerazioni genetiche e applicando i metodi di selezione usati per animali e piante, si poneva l'obiettivo del miglioramento della specie umana; la difficoltà nell'individuazione dei caratteri ereditari e l'indeterminatezza del concetto di miglioramento genetico, soggetto a interpretazioni preconcette come dimostrato storicamente, ne hanno determinato il declino; attualmente un diverso approccio eugenetico è ravvisabile nella possibilità di trattamento delle malattie ereditarie attraverso l'ingegneria genetica.
\end{snippetdefinition}

\begin{snippetdefinition}{ghettizzazione-definition}{Ghettizzazione}
    Estromissione o isolamento di una minoranza da una comunità.
\end{snippetdefinition}

\begin{snippetdefinition}{proselitismo-definition}{Proselitismo}
    La tendenza a fare proseliti, e l'attività svolta per cercarli e formarli: p. di una religione, di un partito, o dei seguaci di una religione, di un partito, di un'idea.
\end{snippetdefinition}

\begin{snippet}{79182efc-ad03-4581-b0b4-d14131595ff4}
    Secondo Kipling, il compito dell'uomo bianco è quello di far progredire
    altri popoli e civilizzarli. È un \quotes{fardello}
    perché questi popoli non accettano il \quotes{dono} che la razza bianca vuole fare loro.
\end{snippet}

\begin{snippetdefinition}{ariano-definition}{Ariano}
    Con \textit{ariano} si intende un individuo appartenente ad un popolo
    che secondo alcune teorie, avrebbe diffuso la lingua indoeuropea in Europa
    e in India. Questo termine è stato poi utilizzato dai nazisti per definire
    una presunta razza pura e superiore, in opposizione a ogni altra razza,
    nello specifico a quella ebraica.
\end{snippetdefinition}

\section{Decolonizzazione}

\begin{snippetdefinition}{decolonizzazione-definition}{Decolonizzazione}
    La \textit{decolonizzazione} è il processo attraverso cui
    un territorio sottoposto a un dominio coloniale ottiene l'indipendenza
    politica, economica e tecnologica dal paese ex-colonizzatore.
\end{snippetdefinition}

\begin{snippet}{decolonizzazione-expl}
    Le cause della decolonizzazione sono:
    \begin{itemize}
        \item costi di mantenimento dei possedimenti coloniali;
        \item aumenta l'ingovernabilità di tali territori;
        \item aspirazione di libertà e indipendenza dei popoli colonizzati;
        \item emergere delle due superpotenze, USA e URSS, che vogliono
        allargare le loro sfere d'influenza.
    \end{itemize}
    
    La decolonizzazione avviene principalmente in tre fasi:
    
    \begin{enumerate}
        \item \textbf{1945-1956}: Asia e maggior parte del mondo arabo (Marocco, Egitto, Arabia Sautida..) in modo pacifico;
        \item \textbf{1957-1965}: Africa Nera e Algeria a volte in seguito a conflitti;
        \item \textbf{1966-1990}: America centrale e l'Africa meridionale, cioè le colonie africane del Portogallo e i
        paesi dominati da minoranze razziste bianche (Rodesia, Zimbabwe, Sudafrica).
    \end{enumerate}
    
    Ancora oggi vi è una grande instabilità politica, come in molti paesi africani,
    perché il tradizionalismo e modernismo vanno in conflitto.
    A livello economico, i paesi decolonizzati
    devono far diventare la propria economia una indipendente.
    Di conseguenza, è più facile che la loro economia rimanga
    strettamente legata a quella dei paesi occidentali.
    A volte manca addirittura l'autosufficienza alimentare.
    \begin{itemize}
        \item \textbf{scaristà risorse materiali:}
            è assente l'autosufficienza alimentare perché non c'è
            una cultura abbastanza variegata da
            poter sopperire alle esigenze della popolazione;
        \item \textbf{dipendenza economica:}
            nonostante il percorso verso l'indipendenza, le ex colonie dipendono ancora dai paesi più
            industrializzati, soprattutto per la gestione e lo sfruttamento delle risorse locali;
        \item \textbf{fragilità del consenso popolare:}
            i popoli sono stati governati con forza e sono stati tracciati confini netti, portando tribù rivali
            tra di loro a convivere all'interno di uno stesso stato. Ciò che impediva lo scontro era un
            forte controllo da parte delle potenze europee e una volta che è venuto a mancare iniziano
            importanti scontri etnici;
        \item \textbf{permanenza culture tradizionaliste:}
            vi è il mantenimento degli usi tradizionali che non è al passo con il mutamento sociale;
        \item \textbf{esplosioone rivalità etniche:}
            sono presenti rivalità all'interno di stati o tra popolazioni diverse che rivendicano il possesso
            di determinati territori.
    \end{itemize}
\end{snippet}

\subsection{La decolonizzazione dell'India}

\begin{snippet}{2e9f01cf-70e6-4090-8a58-444adacee8e1}
    In India alcuni movimenti nazionalisti vogliono
    rendere la nazione indipendente. 
    
    Il leader del partito del \textbf{Congresso Nazionale Indiano} Mahatma Gandhi,
    è a favore di uno Stato unico come soluzione.
    La tecnica di questo movimento è la non-violenza, il boycottaggio
    delle infrastrutture e la disobbedienza cittadina.
    Questo partito è l'esponente dell'indipendenza indiana e della lotta contro l'imperialismo britannico.
    
    La lega musulmana proponeva invece una soluzione a due Stati,
    separando l'india secondo il fattore religioso (induisti - musulmani).
    Nel 1947 quest'ultima soluzione viene approvata, e viene emanato
    l'\textbf{Indian-Indipendent Act}. 
    L'India viene quindi separata in India (induisti) e Pakistan (musulmani).
    
    Questa divisione denota uno spostamento della popolazione,
    la quale provoca diversi scontri sanguinosi.
    Anche Gandhi sarà infatti assassinato da un estremista indù.
\end{snippet}

\begin{snippetdefinition}{indu-definition}{Indù}
    Proprio o indigeno dell'India non musulmana; abitante non musulmano dell'India.
\end{snippetdefinition}

\begin{snippet}{2b2d993c-2484-4738-b7c4-6587e81a6e3a}
    Il partito del congresso era a favore di India + Pakistan. 
    Nel 1950 viene accettata la costituzione, istituendo una sorta di democrazia parlamentare.
    Tuttavia, L'India fatica a costruirsi una propria identità nazionale (es. ci sono 15 lingue ufficiali),
    e il paese è molto arretrato rispetto a ai paesi occidentali.

    Nel 1971 la parte orientale del Pakistan decide di staccarsi dallo stato e viene fondato
    il Bangladesh.
\end{snippet}

\begin{snippetdefinition}{sikhismo-definition}{Sikhismo}
    Il \textit{sikhismo} è una religione monoteistica che crede nel karma e nella reincarnazione.
\end{snippetdefinition}

\begin{snippet}{1d632cea-eb39-43af-867b-c553c17a4fc0}
    La regione del Punjab, con forte presenza dei Sikh, è stata molto contesa fra gli
    indiani e i Pakistani.
    Il Kashmir è la regione fra Pakistan, India e Cina.
\end{snippet}

\subsection{Decolonizzazione dell'Africa}

\plain{Il 1960 viene considerato l'anno dell'africa in ambito decolonizzazione.
Molte nazioni riconoscono l'indipendenza delle propria colonie.}

\begin{snippetdefinition}{fronte-liberazione-nazionale-algerino-definition}{Fronte di Liberazione Nazionale algerino}
    Il \textit{Fronte di Liberazione Nazionale algerino}
    nacque nel 1954 dalla fusione di altri gruppi più piccoli per conseguire l'indipendenza
    dell'Algeria dalla Francia. 
\end{snippetdefinition}

\plain{La Francia non vuole rinunciare all'Algeria e gli algerini non vogliono diventare francesi.}

\begin{snippetdefinition}{guerra-civile-angola-definition}{Guerra civile in Angola}
    La \textit{Guerra civile in Angola} è stata una guerra civile del 1975.
\end{snippetdefinition}

\begin{snippet}{7f3c0ac8-8bae-409e-9a47-4716eb6a28c1}
    Come conseguenze abbiamo che gli Stati nati da queste decolonizzazioni
    erano debolissimi e spesso non riuscirono a
    contrastare guerre interne ed etniche (es: Nigeria, Biafra)
    \(\rightarrow\) il Ruanda ne è un tragico esempio con il genocidio del 1994.
    \\
    Il loro governi
    \begin{itemize}
        \item Si allineavano sulle volontà politiche dei loro ex coloni;
        \item Cercavano partner e ealleanze in area socialista, ossia nella ex URSS.
    \end{itemize}
    Raramente si realizzò la democrazia in Africa.

    I governo più comuni erano quelli a guida Militare (es: Egitto, Eritrea, Libia),
    spesso nascostamente appoggiati da forze economiche occidentali.

    Ancora oggi sono in corso sanguinosi conflitti, ad esempio in Darfur (Sudan)
    e in Somalia, ai quali non sono estranei gli interessi dei paesi ricchi del mongo
    occidentale, della Cina e del fondamentalismo islamico.
\end{snippet}

\subsection{Decolonizzazione del Congo}

\plain{La repubblica democratica del Congo viene lierata dal governo belga.
Una parte del Congo si divide dalla repubblica, formando nel Katanga.}

\begin{snippetdefinition}{secessionismo-definition}{Secessionismo}
    Con \textit{secessionista} si intende un gruppo che vuole distaccarsi
    dalla propria nazione o gruppo che lo contiene per la propria religione o ideologia.
\end{snippetdefinition}

\begin{snippet}{774e59e6-ef33-4579-a798-e250c37b05bc}
    Patrice Lumumba fu il primo ministro del Congo nel 1960 (che si chiamava repubblica del Congo).
    Egli fu un attivista per l'indipendenza e venne ucciso nel 1961.
    Lumumba voleva, contrariamente al Belgio, che la nazione non si separasse.
    Ad oggi non si è formata una repubblica del Katanga.
    Patrice Lumumba si trova a dover attuare un nuovo Stato democratico,
    e far fonte alle forze secessioniste armate.
    
    % Discorso indipendenza del Congo (da un discordo di P. Lumumba 30 giugno 1960)
    
    % aiut
\end{snippet}

\section{Genocidio del Ruanda e la sua indipendenza}

\begin{snippet}{38241c1b-f597-4cc6-85a3-a5be4d97aaf9}
    Il Ruanda era composto dall'1\% twa, 14\% tutsi e 75\% hutu.
    Nel 1885 arrivarono i tedeschi, mentre nel 1919 i belgi (dividi et impera, chi vuole governare
    il paese giova dalla suddivisioni della popolazione).
    
    I tutsi erano generalmente più alti, sbilanciati, magri e benestanti.
    Se si possedeva \(\geq 10\) capi di bestiame eri un tutsi, altrimenti tutu (etno genesi,
    creazione di \textbf{rigidi} identità etniche).
    
    Di conseguenza, i tutu e altri erano gerarchicamente sotto i tutsi, i quali a loro volta
    erano sotto gli uomini bianchi.
    
    La decolonizzazione avvenne fra il 1858 e 1862.
    
    % TODO le freccie dello schema di frongi
\end{snippet}

\begin{snippetdefinition}{neocolonialismo-definition}{Neocolonialismo}
    Per neocolonialismo si intendono tutte le forme di dipendenza nelle quali alcuni paesi,
    pur essendo passati attraverso un processo di conquista dell'indipendenza,
    si trovano nei confronti di altri stati più potenti e in uno
    sviluppo economico-industriale più avanzato.
    In senso opposto è il fenomeno in cui ex potenze coloniali controllano paesi
    economicamente sottosviluppati, utilizzando strumenti economici e
    culturali anziché la forza militare.
    Si tratta di un colonialismo \quotes{informale},
    rispetto a quello \quotes{formale} che temporalmente lo precede.
\end{snippetdefinition}

\plain{Le colonie furono obbligate a importare prodotti industriali
pagandoli con l'esportazione di materie prime.}

% TODO

\section{Palestina e Israele}

\plain{Durante la WWI, la Palestina diventa un protettorato delle forze inglesi.
Possiamo dire che la Gran Bretagna controllasse la Palestina.}

\begin{snippetdefinition}{dichiarazione-balfour-definition}{Dichiarazione di Balfour}
    La \textit{Dichiarazione di Balfour} (1917) è un documento ufficiale della politica del
    governo britannico in merito alla spartizione dell'Impero
    ottomano all'indomani della prima guerra mondiale.
\end{snippetdefinition}

\begin{snippet}{fad47c82-ae35-4194-87fc-358d3882bb60}
    Questa dichiarazione è intesa come principale rappresentante della
    comunità ebraica inglese, e referente del movimento
    sionista, con la quale il governo britannico affermava di guardare
    con favore alla creazione di una \quotes{dimora nazionale per il popolo ebraico} in Palestina.
\end{snippet}

\begin{snippetdefinition}{sionismo-definition}{Sionismo}
    Il \textit{sionismo} è un'ideologia politica il cui fine è l'affermazione del
    diritto alla autodeterminazione del popolo ebraico e il supporto a uno
    Stato ebraico in quella che è definita \quotes{Terra di Israele}.
\end{snippetdefinition}

\begin{snippet}{de603a54-22e3-4a48-ac0d-4464ead3dacd}
    Vi è quindi la necessità di creare un luogo dove la comunità ebraica possa vivere in sicurezza.
    Nel 1917 vi è la migrazione ebraica.
    
    L'ONU prevede la risoluzione della cartina:
    Gerusalemme sarebbe stata una città sotto controllo ONU.
    Per ribadire il concetto viene fondata la Lega araba
    (Egitto, Siria, Libano, Iraq, Transgiordania, Arabia Saudita).
    Nel 1948 nasce lo stato di Israele. 
    
    % TODO UN SACCO DI ROBA CHE HO SALTATO
\end{snippet}

\begin{snippetdefinition}{accordi-di-oslo-definition}{Accordi di Oslo}
    Gli \textit{accordi di Oslo}, ufficialmente chiamati Dichiarazione dei
    Principi riguardanti progetti di auto-governo ad interim o
    Dichiarazione di Principi (DOP), sono una serie di accordi
    politici conclusi ad Oslo (Norvegia) il 20 agosto 1993.
\end{snippetdefinition}

\begin{snippet}{8c8546ca-d8ac-4823-ac42-329cff2eac63}
    \begin{itemize}
        \item Riconoscimento reciproco
        \item Graduale ritiro di Israele dalla Striscia di gaza e dalla Cisgiordania
        \item Diritto Palestinese all'autogoverno attraverso la nascita dell'Autorità nazionale palestinese;
        \item Divisione della Cisgiordania in tre zone;
        \item \begin{enumerate}
            \item Zone sotto controllo dell'ANP;
            \item Zona sotto controllo civile palestinese e israeliano per la sicurezza;
            \item Zona a forte presenza di insediamenti ebraici, sotto controllo israeliano.
        \end{enumerate}
        \item Le questioni annose come Gerusalemme e i rifugiati palestinese negli insediamenti israeliani vengono tralasciate.
    \end{itemize}
\end{snippet}

\begin{snippetdefinition}{hamas-definition}{Hamas}
    \textit{Hamas} (1987) è un movimento religioso di resistenza Islamica.
\end{snippetdefinition}

\begin{snippet}{0a7df511-a254-4730-87b5-96b831f5f755}
    Hamas acquisirà un ruolo politico-militare importante.
    \\
    I governi che si susseguono in Israele portano ad una chiusura sempre maggiore
    nei confronti dei palestinese, fino al 2002, quando sotto il Governo
    Sharon, Israele decide di costruire la Barriera di separazione israeliana.
\end{snippet}

\begin{snippetdefinition}{barriera-israeliana-definition}{Barriera di separazione israeliana}
    La \textit{Barriera di separazione israeliana} è un muro lungo 730 km costruito nel 2002.
    Il muro ha lo scopo di dividere le zone della Cisgiordania per proteggersi
    dai possibili attacchi terroristici.
\end{snippetdefinition}

\begin{snippet}{ec2d830e-0d85-4022-b872-c821fa567c7c}
    Se lo Stato di Israele lo considera un mezzo di difesa dal terrorismo,
    i palestinesi lo ritengono uno strumento di segregazione razziale,
    tanto ché, mentre il primo si riferisce ufficialmente ad esso come
    \quotes{chiusura di sicurezza israeliana} o \quotes{barriera antiterrorista} o
    \quotes{muraglia di protezione} o \quotes{muro salvavita},
    i secondi lo chiamano \quotes{muro della vergogna}.
    
    Nel 2004 muore Yasser Arafat, che aveva coperto la carica di
    di presidente dell'Autorità Nazionale Palestinese (ANP).
\end{snippet}

\begin{snippetdefinition}{autorita-nazionale-palestinese-definition}{Autorità Nazionale Palestinese}
    L'\textit{Autorità Nazionale Palestinese} (ANP) è
    l'organismo politico di autogoverno palestinese ad interim,
    formato nel 1994 in conseguenza degli accordi di Oslo
    per governare la striscia di Gaza e le aree A e B della Cisgiordania. 
\end{snippetdefinition}

\plain{Nel 2005 le truppe israeliane si ritirano dalla Striscia di Gaza.}

\end{document}