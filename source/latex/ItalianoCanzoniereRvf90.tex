\documentclass[preview]{standalone}

\usepackage{amsmath}
\usepackage{amssymb}
\usepackage{stellar}
\usepackage{definitions}
\usepackage{bettelini}

\begin{document}

\id{italiano-canzoniere-rvf-90}
\genpage

\section{Rvf 90: Erano i capei d'oro a l'aura sparsi}

\begin{snippet}{canzoniere-rvf-90-parte1}
    La metrica è un sonetto con schema di rima CDE DCE.
    \\\\
    \StellarPoetry{1}{
        \textbf{Erano} i capei d'oro a l'aura sparsi \\
        che 'n mille dolci nodi gli avolgea, \\
        e 'l vago lume oltra misura ardea \\
        di quei begli occhi, ch'or ne son sì scarsi;
    }{
        I biondi capelli come l'oro (di Laura) erano sparsi al vento,
        \bfslash che li avvolgeva in tanti dolci giri,
        \bfslash e la bella luce di quegli occhi, che ora sono così privi di luminosità, splendeva straordinariamente;
    }
    \StellarPoetry{5}{
        e 'l viso di pietosi color' farsi, \\
        non so se vero o falso, mi parea: \\
        i' che l'esca amorosa al petto avea, \\
        qual meraviglia se di sùbito \textbf{arsi}? 
    }{
        e il viso mi sembrava (v. 6: \quotes{mi parea}) assumere un'espressione di benevolenza nei miei confronti
        \bfslash Ma non posso dire con certezza se ciò fosse vero o falso:
        \bfslash io che avevo confitta nel petto il dardo dell'amore,
        \bfslash cosa c'è da stupirsi se subito arsi d'amore?
    }
    \StellarPoetry{9}{
        \textbf{Non era} l'andar suo cosa mortale, \\
        ma d'angelica forma; e le parole \\
        sonavan altro, che pur voce humana.
    }{
        Il suo incedere non era quello di un corpo mortale,
        \bfslash ma di uno spirito angelico, e la sua voce
        \bfslash aveva un suono diverso da una soltanto (pur) umana;
    }
    \StellarPoetry{12}{
        Uno spirto celeste, un vivo sole \\
        fu quel ch'i' vidi: e se non fosse or tale, \\
        piagha per allentar d'arco non sana.
    }{
        una creatura del cielo, un sole vivente
        \bfslash fu quello che vidi; e anche nel caso in cui non avesse più quell'aspetto,
        \bfslash di certo la ferita procurata da una freccia non si risana solo perché la corda dell'arco, dopo il colpo sferrato, si allenta.
    }
\end{snippet}

\begin{snippetdefinition}{distico-definition}{Distico}
    Il \textit{distico} è una strofa formata da una coppia di versi, solitamente legati da una rima.
\end{snippetdefinition}

\begin{snippetdefinition}{senhal-definition}{Senhal}
    Il \textit{senhal} è una figura retorica e consiste in un
    appellativo riservato generalmente alla donna amata ma anche ad amici o
    altri personaggi.
\end{snippetdefinition}

\begin{snippet}{canzoniere-rvf-90-parte2}
    \circled{1-4} Il primo distico è dedicato ai capelli di Laura, mentre il secondo è
    dedicato ai suoi occhi. Il nome Laura non è scritto letteralmente,
    ma \quotes{l'aura} è un allusione fonetica al suo nome (non vi erano ancora gli apostrofi,
    e quindi viene scritto nell'originale come \quotes{laura}).
    Questa strategia di illudere ad un nome in modo nascosto è un senhal.
    I capelli biondi rappresentano la bellezza nella poesia antica perché:

    \begin{enumerate}
        \item è il colore della divinità:
        \item è il colore degli Dei;
        \item gli angeli sono biondi;
        \item i capelli biondi sono più rari, e quindi più preziosi (nel Mediterrano).
    \end{enumerate}
    Nelle fiabe popolari i capelli neri rappresentano il cattivo,
    mentre i capelli rossi rappresentano il diavolo.
    Invece, nelle poesie nordiche i capelli neri sono belli.

    La prima parola del primo verso, \quotes{Erano}, indica che queste sono caratteristiche
    passate di Laura. Infatti, si sta riferendo al momento del suo innamoramento.
    Nel secondo distico viene imposta una differenza fra la luminosità dei suoi occhi
    fra il momento dell'innamoramento ed il momento della scrittura.
    La grande novità di questo sonetto è che Laura è invecchiata, e ora i suoi occhi sono scarsi di luce.
    \\\\

    % Si pensava che il mondo fosse costruire attorno al numero quattro. Quattro cardinali, quattro stagioni, quattro colori puri in natura
    % Siccome la donna è quasi l'emblema della perfezione, essa deve portare i quattro colori.
    % bianco (pelle) nero (capelli, ciglia, occhi) rosso (gote, labbra) giallo (deve per forza essere nei capelli)
    \circled{5-8} Nella seconda strofa abbiamo la descrizione del viso.
    Durante l'innamoramento gli sembravano che il suo viso si formasse dei colori della (pietà) compassione,
    cioè che anche lei ricambiasse il sentimento, ma l'autore non sa se fosse vero o meno.
    L'esca è un materiale infiammabile, e l'autore che la possedeva (ossia una predisposizione ad un potenziale innamoramento),
    qual'è la meraviglia se lui subito si innamorò? Il suo sguardo fu sufficiente per innescarne la scintilla.
    \\\\
    \circled{9-11} Questi versi descrivono la straordinaria grazia e l'elevatezza dell'amata Laura.
    Egli sottolinea come il suo modo di muoversi non sembrasse terreno o umano, ma piuttosto celestiale,
    quasi angelico. Le parole che usava non sembravano essere solamente suoni umani,
    ma avevano un tono o una qualità che andavano oltre la semplice voce umana, forse evocando
    un senso di purezza o trascendenza.
    Questa descrizione enfatizza la visione ideale e quasi divina che Petrarca aveva dell'amata Laura.
    \\\\
    \circled{12-14} E se anche ora non fosse più così, la ferita non si rimargina nonostante l'arco sia allenato (cioè la frecca è già stata scoccata tempo fa).
    Anche se Laura non è più splendente come in quel momento, e anche se l'arco non è più teso, la ferita dell'autore non si rimargina.
    L'amore non è quindi finito, ed esso dura nel tempo. Questa è una dichiarazione di fedeltà dell'amore nonostante il tempo passi.
    \\\\
    Il protagonista di questo sonetto è Petrarca che ricorda Laura.
    Gli elementi tipicamente cortesi del testo sono:
    \begin{enumerate}
        \item l'innamoramento con gli occhi;
        \item la freccia scagliata da Dio che crea un incendio;
        \item il potenziale di innamoramento dell'uomo che la donna accende;
        \item tratti tipici della donna: voce, occhi, viso, capelli.
        \item la visione della donna come angelica.
    \end{enumerate}

    I verbi al passato remoto \quotes{arsi}, \quotes{vidi} e \quotes{fu},
    esprimono il fatto che il momento dell'innamoramento è irripetibile.
    I verbi al presente sono \quotes{son}, \quotes{se fosse} (ora) e \quotes{sana},
    con quella bellezza ormai sfiorita.
    Tutti gli altri sono all'imperfetto, ossia che quelle azioni che hanno una durata fino al presente.

    Questo sonetto potrebbe essere diviso a metà siccome i tratti di Laura
    della sua bellezza terreno (viso, occhi etc.) sono descritti nella quartine,
    mentre nelle terzine danno dei tratti più angelicati.
    Infatti, a differenza delle quartine, le terzine descrivono alcuni tratti con
    negazioni. Sopra, la bellezza terrena è esplicita, mentre quella angelica è indicibile,
    e viene quindi descritta con cosa non è.

    Il modello dietro a questo tipo di descrizione deriva da Enea nell'\textit{Eneide}:
    prima i tratti terreni e poi quelli divini. Enea pensa che una donna sia una cacciatrice
    prima di una dea.

    % passato remoto = momento chiuso nel passato, non tanto temppo fa, ma chiuso nel passato

    % tratti stilnovistici
    % gli elementi cortesi possono essere dimostrati con i riferimenti:
    % - a Guido guinizzelli "io voglio del ver la mia donna laudare",
    % - Dante (vita nova, XXVI) "tanto gentile e tanto onesta pare
    % più liminosa della stella diana (la srella più luminosa)
\end{snippet}

\end{document}