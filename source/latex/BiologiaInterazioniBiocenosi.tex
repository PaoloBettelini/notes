\documentclass[preview]{standalone}

\usepackage{amsmath}
\usepackage{amssymb}
\usepackage{stellar}
\usepackage{bettelini}

\hypersetup{
    colorlinks=true,
    linkcolor=black,
    urlcolor=blue,
    pdftitle={Stellar},
    pdfpagemode=FullScreen,
}

\begin{document}

\id{biologia-interazioni-biocenosi}
\genpage

\section{Interazioni biocenosi}

\begin{snippetdefinition}{simbiosi-definition}{Simbiosi}
    Una \textit{simbiosi} è un qualsiasi tipo di interazione biologica stretta
    e a lungo termine tra due diversi organismi biologici.
    Una simbiosi può essere:
    \begin{itemize}
        \item \textit{mutualistica};
        \item \textit{parassitistica};
        \item \textit{commensalistica}.
    \end{itemize}
\end{snippetdefinition}

\begin{snippetexample}{simbiosi-mutualistica-example}{Simbiosi mutualistica}
    I giardini del diavolo sono grandi gruppi di alberi nella foresta amazzonica
    costituiti quasi interamente da una sola specie, Duroia
    irsuta, e coltivati, secondo le leggende locali, da uno spirito
    malvagio della foresta.
    In realtà, si scoprì che il motivo di questa morte era dato dalla presenza
    di una particolare specie di formiche, le quali producevano una sostanza
    chimica velenosa che uccideva gli alberi di altre specie.
    La relazione simbiotica mutualistica fra formiche e pianta è data dal fatto che
    la pianta non ha competizione, mentre la formica guadagna l'habitat.  
\end{snippetexample}

\begin{snippetdefinition}{Interazioni-interspecifiche-definition}{Interazioni interspecifiche}
    Le \textit{interazioni interspecifiche} influenzano la struttura e la
    dinamica delle popolazioni di una comunità e vengono
    descritte in base al loro effetto sugli organismi che
    interagiscono.
    \begin{itemize}
        \item \textbf{Competizione interspecifica:} popolazioni di specie diverse
            competono per la stessa risorsa limitata (cibo e spazio).
            L'effetto della competizione interspecifica è negativo per
            entrambe le popolazioni coinvolte (-/-);
        \item \textbf{Mutualismo:} è un tipo di simbiosi in cui entrambe le
            popolazioni traggono vantaggio dalla relazione (+/+);
        \item \textbf{Predazione:} è un'interazione in cui una specie carnivora
            (il predatore) cattura e uccide un'altra specie (la preda)
            per mangiarla (+/-);
        \item \textbf{Alimentazione erbivora:} è l'equivalente della predazione
            per gli animali (detti erbivori) che si nutrono di vegetali,
            anche se il loro consumo non implica la cattura e, solo
            raramente, comporta l'uccisione della pianta (+/-);
        \item Altre interazioni sono quelle tra ospite e parassita o agente patogeno (+/-).
    \end{itemize}
\end{snippetdefinition}

\begin{snippet}{9789c35f-43ad-485d-a3d7-51401192408f}
    Ragionando in termini di interspecifiche: tolta la competizione abbiamo la predazione.
    Se un predatore viene rimosso da un sistema, è possibile che una popolazione diventi
    troppo grande, a scapito di altre specie che potrebbero estinguersi.
    Quindi, le popolazioni predate traggono anche vantaggi da questa dinamica di predazione.
    In generale, è possibile che l'equilibrio si sbilanci.
    Alcune interazioni interspecifiche sono uno svantaggio nel breve termine, ma un vantaggio sul lungo termine.
\end{snippet}

\begin{snippetdefinition}{struttura-trofica-definition}{Struttura trofica}
    La \textit{struttura trofica} è l'intreccio di relazioni alimentari,
    strutturate su più livelli, che interessano una comunità.
\end{snippetdefinition}

\begin{snippetdefinition}{catena-alimentare-definition}{Catena-alimentare}
    La \textit{catena alimentare} è la sequenza di passaggi
    attraverso cui la materia e virtualmente l'energia
    vengono trasferite dai \textit{livelli trofici} più bassi ai più alti.
\end{snippetdefinition}

\begin{snippet}{rete-alimentare}
    L'insieme delle catene alimentari genera una \textit{rete alimentare}, ossia un grafo relazionale.
\end{snippet}

\begin{snippetdefinition}{produttore-definition}{Produttore}
    Un \textit{produttore} è un organismo che fa la fotosintesi, e quindi produce materia organica.
\end{snippetdefinition}

\begin{snippet}{57073b2c-075f-4f90-adde-8335f7d5f9c0}
    I livelli \textit{trofici} sono gli scalini della catena alimentare.
    Nel livello trofico più passo vi è sempre un produttore, ossia un organismo autotrofo.
    I livelli trofici rappresentano la popolazione. La rete trofica rappresenta le relazioni
    nella popolazione.

    % I vegetariani e vegani non sono una specie e non possono essere catalogiati
    % in un altro punto della catena alimentare.

    Nei concimi industriali è presente principalmente l'azoto.

    Tutto ciò che muore in una rete alimentare diviene un detrito. A questo punto diventa
    facente parte della catena dei detriti, cambiando la sua rete di appartenenza.
    Altri organismi (consumatori detriviti) si nutrono di detriti e batteri e funghi (decompositori) che trasformano la materia organica in inorganica.
\end{snippet}

\begin{snippetdefinition}{consumatori-detritivori-definition}{Consumatori detritivori}
    I \textit{consumatori detritivi} sono degli organismi che si cibano di detriti
    nella catena dei resti.
\end{snippetdefinition}

\begin{snippetdefinition}{decompositori-definition}{Decompositori}
    I \textit{decompositori} trasformano la materia organica in inorganica.
\end{snippetdefinition}

\plain{decompositori sono generalmente unicellulari, e si nutrono di grandi detriti per farli diventare
detriti più piccoli.}

\begin{snippetexample}{terra-in-ampolla-example}{}
    Considera un sistema chiuso dove vengono aggiunti della terra e dei funghi compositori.
    Dopo un certo tempo, i funghi muoiono e tutto il contenitore sembrerà vuoto.
    Questo è dato dal fatto che tutta la materia è diventata \(CO_2\),
    prodotta dalla respirazione cellulare dei funghi.
\end{snippetexample}

\plain{La luce del sole è la fonte di energia primaria per far funzionare l'intero sistema.}

\begin{snippetdefinition}{biomassa-definition}{Biomassa}
    La \textit{biomassa} è la massa di un organismo (massa senza acqua)
    oppure materia organica (sia vivente che detriti) presente in un ecosistema.
\end{snippetdefinition}

\begin{snippet}{05477a6f-9d01-47a6-a652-cb93fd415266}
    La fotosintesi converte energia solare in energia chimica, con un rendimento del 1\%.
    Ogni livello trofico ha un rendimento di circa il 10\%, per cui la catena alimentare
    è limitata a quattro o cinque zone trofiche data la bassa quantità di energia
    che giunge a quello più in alto.

    Per misurare la biodiversità si può misurare il numero di specie presenti in un ecosistema
    e la loro concentrazione.
\end{snippet}

\begin{snippetdefinition}{habitat-definition}{Habitat}
    Con \textit{habitat} si intende lo spazio fisico all'interno del quale vivono gli organismi
    appartenenti a una determinata specie e può essere suddiviso in \textit{microhabitat}.
\end{snippetdefinition}

\begin{snippetdefinition}{nicchia-ecologica-definition}{Nicchia ecologica}
    Con \textit{nicchia ecologica} si intende la funzione che una specie svolge
    in relazione alle componenti del proprio ecosistema.
\end{snippetdefinition}

\plain{Due specie diverse non possono condividere la stessa nicchia.}

\begin{snippet}{nicchia-effettiva}
    La \textit{nicchia effettiva} è la nicchia occupata da una specie in una comunità
    dove la specie potrebbe però usufruire di una nicchia più grande se
    non fosse già occupata da un'altra specie possibilmente più forte.
    Due specie in competizione hanno un conflitto circa la loro nicchia.
    Alternativamente, è possibile spartirsi delle risorse in maniera complementare.
\end{snippet}

\begin{snippetdefinition}{principio-esclusione-competitiva-definition}{Principio di esclusione competitiva}
    Il \textit{principio di esclusione competitiva} indica che quando due specie coesistono in un medesimo ambiente e le due specie
    presentano nicchie sovrapposte, allora una delle due specie prenderà il sopravvento sull'altro
    fino ad eliminarla.
\end{snippetdefinition}

% crescita esponenziale e logistica

\end{document}