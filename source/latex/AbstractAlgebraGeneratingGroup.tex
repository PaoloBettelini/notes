\documentclass[preview]{standalone}

\usepackage{amsmath}
\usepackage{amssymb}
\usepackage{stellar}
\usepackage{definitions}
\usepackage{makecell}

\begin{document}

\id{generating-group}
\genpage

\section{Generating subgroup}

\begin{snippet}{group-generator-expl1}
    The union of two \set[sets] is the biggest subset that contains both \set[sets].
    Given a \group \((G, \circ)\) and \(X \subseteq G\), we want to find a
    \subgroup of \(G\) that contains \(X\). Trivially, \((G, \circ)\) does.
    However, we focus on the smallest possible \subgroup containing \(X\)
    (i.e. contained in every \subgroup[subgroups] containing \(X\)).
    We consider the intersection of all \subgroup[subgroups]
    of \((G, \circ)\) containing \(X\) (which is non-empty, as we pointed out).
    This construction is a \subgroup containing \(X\) and it is the smallest doing so.
\end{snippet}

\begin{snippetdefinition}{generating-set-of-group-definition}{Generating set of a group}
    Let \((G, \circ)\) be a \group and \(X\subseteq G\).
    The \textit{subgroup generated by} \(X\) is the smallest \subgroup of \((G, \circ)\)
    such that \(X\subseteq G\).
    \[
        \langle X \rangle = \left(\, \bigcap_{H \in \Xi} H, \circ\right),
        \quad \Xi = \{ H \suchthat (H, \circ) \subgroupleq (G,\circ) \land X \subseteq H \}
    \]
    \textit{Syntax:} \(\langle \{x_1, x_2, \cdots, x_n\} \rangle \triangleq \langle x_1, x_2, \cdots, x_n \rangle\).
\end{snippetdefinition}

\begin{snippet}{group-generator-expl2}
    Note that \(X \subgroupleq G \implies \gengrp{X} = X\).
    We now want to describe the elements of \(\gengrp{X}\).
    If \(X = \{x,y,z\} \neq \emptyset\), then \(\gengrp{X}\) must contain the products and the inverses
    of the elements of \(X\) (as well as the other elements in \(X\)).
    These are \(x\circ y, y\circ z, \cdots, x^{-1}, z^{-1}, \cdots\) but also
    the products and the inverses of the elements found
    \(x\circ y \circ y \circ z, x \circ y \circ z^{-1}, \cdots\).
    All of these elements are sufficient.
\end{snippet}

\begin{snippettheorem}{generated-subgroups-contains-powers-theorem}{}
    Let \((G, \circ)\) and \(X \subseteq G\).
    Then, if \(X = \emptyset\), \[
        \gengrp{X} = (\{1\}, \circ)
    \]
    Otherwise, \(\gengrp{X}\)
    is formed by the elements
    \[
        x_1 \circ x_2 \circ \cdots \circ x_n
    \]
    with \(x_i \in X\) or \(x_i^{-1} \in X\) for \(1 \leq i \leq n\).
\end{snippettheorem}

\begin{snippetproof}{generated-subgroups-contains-powers-proof}{generated-subgroups-contains-powers}{}
    The case \(X=\emptyset\) is trivial.
    Let \(X \neq \emptyset\) and \(W\) be the \set formed by elements of the form
    \[
        x_1 \circ x_2 \circ \cdots \circ x_n
    \]
    with \(x_i \in X\) or \(x_i^{-1} \in X\) for \(1 \leq i \leq n\).
    Clarly, the elements of \(W\) must be in every \subgroup[subgroups] containing \(X\).
    Thus, \(W \subseteq \gengrp{x}\).
    On the other hand, \(X \subseteq W\).
    We need to show that \((W, \circ)\) is a \subgroup.
    If \(x_1 \circ x_2 \circ \cdots \circ x_n \in W\) and
    \(y_1 \circ y_2 \circ \cdots \circ y_m \in W\) (that is, every \(x_i\) and \(y_i\)
    is in \(X\) o has inverse in \(X\)), then clearly
    \[
        x_1 \circ x_2 \circ \cdots \circ x_n
        \circ
        y_1 \circ y_2 \circ \cdots \circ y_m
        \in W
    \]
    If \(x_1 \circ x_2 \circ \cdots \circ x_n \in W\), then\[
        {(x_1 \circ x_2 \circ \cdots \circ x_n)}^{-1}
        = x_n^{-1} \circ x_{n-1}^{-1} \circ x_1^{-1}
    \]
    and \(x_i^{-1}\) is an element of \(X\)
    or has inverse in \(X\).
\end{snippetproof}

\end{document}