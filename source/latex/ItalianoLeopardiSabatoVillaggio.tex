\documentclass[preview]{standalone}

\usepackage{amsmath}
\usepackage{amssymb}
\usepackage{stellar}
\usepackage{definitions}

\begin{document}

\id{italiano-leopardi-sabato-villaggio}
\genpage

\section{Il sabato del villaggio}

\includesnpt[text=Testo:|display=Il sabato nel villaggio|href=http://homes.di.unimi.it/\~{}pasteris/progettoMM/avvisi/temi/Sabato\_villaggio.pdf]{file-url}

\begin{snippet}{leopardi-sabato-del-villaggio-analisi}
    La metrica consiste in quattro strofe libere di settenari.
    \\
    Si apre il canto con questa figura (non personaggio) che torna a casa dalla campagna.
    ogni verso contiene una coordinata (spazio vv. 1, tempo vv. 2, complemento di unione vv.3).
    Il tempo è la fine del sabato (il quale è ancora un giorno lavorativo)
    e la donzella si appresta al giorno successivo.
    \\
    L'accostamento di rose e viole è un accostamento letterario tradizionale,
    non inventato da Leopardi, piuttosto che reale in quanto
    questi due fiori non sbocciano durante gli stessi periodi.
    \\
    Vicino, vi è una vecchietta seduta sulle scale, rivolta al tramonto,
    che sta filando mentre racconta della sua giovinezza, di quanto si ornata
    ed era solita andare a ballare con i suoi compari.
    \\
    Le due figure sono due donne, condividono lo stesso villaggio,
    vengono chiamate con dei vezzeggiativi, entrambi si preparano ai giorni festivi ornandosi
    e sono solite fare ciò prima di ogni giorno festivo.
    La differenza fra le due è l'età.
    In altre parole, potremmo dire che la vecchierella è stata come la donzelletta
    e viceversa.
    \\
    Il tramonto è un momento di luce violenta e stravagante. Poi, vi è un momento in cui
    il cielo torna azzurro e arriva al luna. Le ombre tornano, ma sono le ombre date dalla
    luce lunare.
    È sera, e le campane annunciano le festa del giorno dopo.
    \\
    Abbiamo la descrizione di un villaggio contadini, della vita dei suoi abitanti,
    nelle ore comprese fra il tramonto e la sera, in una vigilia di un giorno festivo
    e un gruppo di ragazzi gioca facendo rumore.
    \\
    Il XX vuole andare a letto non sono libero dal lavoro, ma anche dal pensiero del lavoro.
    Il sabato è il giorno più gradito (per l'attesa della gioia) e la domenica quello più triste in quanto
    vi sarà il pensiero della fatica del lunedì.
    Il contorno sfumato dell'aspettativa è solitamente migliore dell'evento stesso.
    Il piacere dell'attesa è migliore del piacere effettivo stesso.
    \\
    Questo è il primo paradosso.
    Il secondo paradosso è quello per il quale la persona non si rende conto di ciò
    e continua a provare la medesima gioia, nonostante le delusioni precedenti.
\end{snippet}

\end{document}