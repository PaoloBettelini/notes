\documentclass[preview]{standalone}

\usepackage{amsmath}
\usepackage{amssymb}
\usepackage{stellar}
\usepackage{bettelini}

\hypersetup{
    colorlinks=true,
    linkcolor=black,
    urlcolor=blue,
    pdftitle={Stellar},
    pdfpagemode=FullScreen,
}

\begin{document}

\id{geoeconomica-polarismo}
\genpage

\section{Polarismo}

\begin{snippetdefinition}{bipolarismo-definition}{Bipolarismo}
    Con \textit{bipolarismo} si intende la contrapposizione di due blocchi distinti;
    a livello nazionale essi sono rappresentati, di solito, da due coalizioni o
    raggruppamenti di partiti e/o movimenti, che si contendono la conquista del potere.
\end{snippetdefinition}

\begin{snippet}{e14527ac-01be-4b83-b425-22c55995c242}
    Bipolarismo e unipolarismo sono una proprietà distinta dal multilateralismo.
    Il multilateralismo consiste in una cooperazione nazionale, come per esempio creare organizzazioni
    come l'ONU, ma esso è indipendente dal polarismo nazionale.

    Dopo la caduta del muro di Berlino, il mondo si frammenta e passa da bipolare ad unipolare.

    Oltre alla poteza economica e militare (gendarme del mondo), l'America
    guadagna anche un potenza sulla propria cultura, specialmente dopo la diffusione di internet
    e del cinema odierna (soft power).

    Prima dell'unipolarismo, lo stato autocratico con economia pianificata e
    lo stato democratico con il libero mercato sono valutabili alla pari.
    Data la caduta dell'URSS, il modello Stati Uniti viene preso come apice
    di modello socioeconomico.
    Ciò rimane indubbio fino all'attacco dell'11 settembre 2001, dove l'attacco mette
    in discussione l'aspetto culturale.
    Il mondo rimane quindi unipolare fino al 2001, dove lentamente questo status si sgretola
    portando ad un multipolarismo.

    La presidenza Obama diminuisce la gendarmeria.

    La presidenza Trump mette in pausa questo multilateralismo per mettere l'America
    in primo piano a scapito degli affari esteri.
    Questa presidenza aumenta anche gli interessi (export and import) con la Cina, portandone
    una grande crescita economica. Questo mette in dubbio pure
    il dominio economico degli Stati Uniti.
\end{snippet}

\end{document}