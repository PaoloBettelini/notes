\documentclass[preview]{standalone}

\usepackage{amsmath}
\usepackage{amssymb}
\usepackage{bettelini}
\usepackage{stellar}
\usepackage{definitions}

\begin{document}

\id{english-atonement-epilogue}
\genpage

\section{Exercises}

\begin{snippetexercise}{atonement-ex-31}
    {Who is the narrator of the last chapter of the novel, i.e. the epilogue?}
    The narrator of the last chapter is Briony herself (first person).
    Indeed, we learn that she is the author of the whole book.
    This is the first time that we get this type of narrator.
    She is a character in the story and now she is both telling the story and seeing it
    (both the narrator and focalizer, but not omniscent).
\end{snippetexercise}

\begin{snippetexercise}{atonement-ex-32}
    {What is the significance of Briony's diagnosis of vascular dementia?}
    The diagnosis is important because Briony will start to lose memory, the ability to
    think critically and brain function in general. Her life,
    especially as a writer, is slipping through her fingers,
    as she is also inevitably bound to die within a few years.
    The meaning of the illness is precisely its effects on the person,
    effectively losing her identity.
\end{snippetexercise}

\begin{snippetexercise}{atonement-ex-33}
    {What do we learn about the other characters of the novel from Briony's trips around the city and
    the scene in front of the museum?}
    Paul and Lola have not change; they are both powerful.
    Lola is basically the only one who doesn't show signs of aging.
    This is related to the story because Briony mentions something about outliving her.
    The fact that she seems to be outliving Briony is an issue for her because
    the editor told her that she should only publish the book when everyone else has died,
    as the names are not censored, and Paul Marshall might take her to court for it.
\end{snippetexercise}

\begin{snippetexercise}{atonement-ex-34}
    {\quotes{I love these little things, this pointillist approach to verisimilitude, the correction of detail that
    cumulatively gives such satisfaction.} (p. 339, my emphasis) How can the concept of verisimilitude
    be applied to the whole novel?}
    We don't know too what extent it was true.
    The entire section about the war is real in the sense that she researched it;
    went to the museum, read the letters from the soldiers and thus is historically accurate.
    However, at the same time, she cannot know everything, so not everything is necessarily true.
\end{snippetexercise}

\begin{snippetexercise}{atonement-ex-35}
    {What is happening at the Tilney's Hotel? What did it use to be?}
    The hotel was the previous home of the Tallis family
    and Briony connects thing to her childhood; notices differences and such.
\end{snippetexercise}

\begin{snippetexercise}{atonement-ex-36}
    {How does the story end?}
    She has finished what she wanted to write and now can die in peace.
    She might feel that she finally redeemed (atoned) herself.
    There is both a real ending (of the internal, fictional story)
    and the fantasy one which Briony made up.
    In order to atone for the real ending, she has to wait for the book to be published.
\end{snippetexercise}

\begin{snippetdefinition}{redemption-arc-definition}{Redemption arc}
    A \textit{redemption arc} is when a character with significantly flawed attributes and behavior learns to overcome their flaws and be a better person.
    It requires that a character atones for their actions, that the atonement outweighs whatever wrong they've done, and that they receive forgiveness for said actions.
\end{snippetdefinition}

\end{document}