\documentclass[preview]{standalone}

\usepackage{amsmath}
\usepackage{amssymb}
\usepackage{stellar}
\usepackage{definitions}
\usepackage{bettelini}

\begin{document}

\id{dedekind-axiom}
\genpage

\section{Axiom of continuity}

\begin{snippetaxiom}{dedekind-axiom}{Axiom of Dedekind}
    Let \(\{I_n\}\) be a \sequence of closed intervals of the form
    \(I_n = [a_n; b_n]\) such that \(I_{n+1} \subseteq I_n\) and
    \[
        l(I_{n+1}) = b_{n+1}-a_{n+1} = \frac{1}{2}l(I_n)
    \]
    and thus \[l(I_n) = \frac{1}{2^{n-1}}l(I_1)\] Then, there exists \(c\in \mathbb{R}\)
    such that
    \[
        \bigcap_{n\in\naturalnumbers} I_n = \{c\}
    \]
\end{snippetaxiom}

\begin{snippettheorem}{dedekind-completness-axiom-equivalence-theorem}{Dedekind and completness axioms equivalence}
    The \snippetref[dedekind-axiom|Dedekind axiom][Dedekind axiom]
    is equivalent to the \snippetref[completness-axiom|Completness axiom][completness axiom].
\end{snippettheorem}

\begin{snippetproof}{dedekind-completness-axiom-equivalence-theorem-proof}{dedekind-completness-axiom-equivalence-theorem}{Dedekind and completness axioms equivalence}
    \iffproof{
        Let \(E\) be a non-empty \set that is \upperbound[bounded above]. We will prove that there exists \(c=\sup E\).
        Since \(E\) is \upperbound[bounded above], there exists a \(b_1\) that is an \upperbound of \(E\)
        where \(b_1 \geq e, \forall e \in E\).

        Since \(E\) is non-empty, there exists \(\overline{e}\in E\),
        and we set \(a_1 = \overline{e} - 1\) so that \(a_1 < \overline{e}\)
        and \(a_1\) is not an \upperbound. Let \(I_1 = [a_1, b_1]\) and \(m_1=\frac{a_1 + b_1}{2}\).
        Then, there are two cases:
        \begin{itemize}
            \item If \(m_1\) is an \upperbound, we set \(a_2 = a_1\) e \(b_2 = m_1\);
            \item If \(m_1\) is not an \upperbound, we set \(a_2 = m_1\) e \(b_2 = b_1\).
        \end{itemize}
        Let \(I_2 = [a_2, b_2]\). We iterate this process, obtaining a \sequence of intervals
        \[
            I_n = [a_n, b_n]
        \]
        such that \(I_{n+1} \subseteq I_n\) and \(l(I_{n+1}) = \frac{1}{2}l(I_n)\).
        For each \(n\), \(a_n\) is not an \upperbound of \(E\), and \(b_n\) is an \upperbound of \(E\).
        By the axiom of continuity, there exists \(\exists c\in \mathbb{R}\) such that
        \[
            \bigcap_{n\in\mathbb{N}} I_n = \{c\}
        \]
        Our thesis is therefore \(c = \sup E\).
        Suppose for contradiction that it is not an \upperbound, then there exists an element
        \(e\in E\) such that \(e > c\).
        By definition \(c\) is in at least one \(I_n\) so \(a_n \leq c \leq b_n\)
        and since \(e > c\) we have \(a_1 \leq c < e \leq b_n\) because \(e\in E\)
        and \(b_n\) is an \upperbound of \(E\) for every \(n\).
        Thus, \[
            [c, e] \subseteq [a_n, b_n]
        \]
        and so
        \[
            [c, e] \in \bigcap_{n\in\mathbb{N}} I_n = \{c\}
        \]
        which is a contradiction \lightning.
        We now need to show that \(c\) is the least of the upper bounds.
        Suppose that there exists another \upperbound
        \(x<c\).
        Since \(b_n \geq c\) for every \(n\) we have
        \(x < c \leq b\). By hypothesis, \(x\) is an \upperbound of \(E\)
        while \(a_n\) is not an \upperbound of \(E\)
        for every \(n\). Thus, for all \(n\) there exists an element \(e_n\) such that
        \[ a_n < e_n \leq x < c \leq b_n \]
        and therefore, it follows that the interval \([x,c] \subseteq [a_n, b_n] = I_n\)
        which leads to
        \[
            [c, e] \in \bigcap_{n\in\mathbb{N}} I_n = \{c\}
        \]
        which is absurd \lightning.
    }{
        \todo
    }
\end{snippetproof}

\end{document}