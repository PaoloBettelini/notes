\documentclass[preview]{standalone}

\usepackage{amsmath}
\usepackage{amssymb}
\usepackage{stellar}
\usepackage{definitions}

\begin{document}

\id{dedekind-axiom}
\genpage

\section{Axiom of continuity}

%\saxiom{Assioma di Dedekind}{
%    Se \(\{I_n\}\) è una successione di intervalli chiusi della forma
%    \(I_n = [a_n; b_n]\) tali che \(I_{n+1} \subseteq I_n\) e
%    \[
%        l(I_{n+1}) = b_{n+1}-a_{n+1} = \frac{1}{2}l(I_n)
%    \]
%    e quindi \[l(I_n) = \frac{1}{2^{n-1}}l(I_1)\] allora esiste \(c\in \mathbb{R}\)
%    tale che
%    \[
%        \bigcap_{n\in\mathbb{N}} I_n = \{c\}
%    \]
%}
%
%Questo assioma non vale nei razionali.
%
%\stheorem{Assioma di Dedekind equivalenza assioma di completezza} {
%    L'assioma di Dedekind è equivalente all'assioma della completezza.
%}
%
%\sproof{Assioma di continuità equivalenza assioma di Dedeking}{
%    \iffproof{
%        Sia \(E\) un insieme non vuoto e limitato superiormente, dimostriamo che
%        esiste \(c=\sup E\).
%        Poiché \(E\) è limitato superiormente esiste un \(b_1\) maggiorante di \(E\)
%        dove \(b_1 \geq e, \forall e \in E\).
%        Poiché \(E\) è non vuoto esiste \(\overline{e}\in E\) e
%        poniamo \(a_1 = \overline{e} - 1\) cosicché \(a_1 < \overline{e}\)
%        e \(a_1\) non è maggiornate. Sia \(I_1 = [a_1, b_1]\) e sia \(m_1=\frac{a_1 + b_1}{2}\),
%        allora vi sono due casi:
%        \begin{itemize}
%            \item \(m_1\) è un maggiorante e allora poniamo \(a_2 = a_1\) e \(b_2 = m_1\);
%            \item \(m_1\) non è un maggiorante e allora poniamo \(a_2 = m_1\) e \(b_2 = b_1\).
%        \end{itemize}
%        Sia \(I_2 = [a_2, b_2]\). Iteriamo allora il procedimento
%        otteniamo una successione di intervalli
%        \[
%            I_n = [a_n, b_n]
%        \]
%        tali che \(I_{n+1} \subseteq I_n\) e \(l(I_{n+1}) = \frac{1}{2}l(I_n)\).
%        Per ogni \(n\), \(a_n\) non è maggiorante di \(E\), \(b_n\) è maggiorante di \(E\).
%        Per l'assioma di continuità \(\exists c\in \mathbb{R}\) tale che
%        \[
%            \bigcap_{n\in\mathbb{N}} I_n = \{c\}
%        \]
%
%        La nostra tesi è quindi \(c = \sup E\).
%        Supponiamo quindi per assurdo che non sia un maggiorante, allora che esiste un elemento
%        \(e\in E\) dove \(e > c\).
%        Per definizione \(c\) è in almeno un \(I_n\) quindi \(a_n \leq c \leq b_n\)
%        e poiché \(e > c\) abbiamo \(a_1 \leq c < e \leq b_n\) perché \(e\in E\)
%        e \(b_n\) è maggiorante di \(E\) per ogni \(n\).
%        Quindi \[
%            [c, e] \subseteq [a_n, b_n]
%        \]
%        e allora
%        \[
%            [c, e] \in \bigcap_{n\in\mathbb{N}} I_n = \{c\}
%        \]
%        che è quindi una contraddizione.
%        Dobbiamo ora mostrare che \(c\) è il più piccolo dei maggioranti.
%        Supponiamo quindi per assurdo che ci sia un altro maggiorante
%        \(x<c\).
%        Poiché \(b_n \geq c\) per ogni \(n\) abbiamo che
%        \(x < c \leq b\). Per ipotesi, \(x\) è maggiorante di \(E\)
%        mentre \(a_n\) non è maggiorante di \(E\)
%        per ogni \(n\). Quindi per tutte le \(n\) esiste un elemento \(e_n\) tale che
%        \[ a_n < e_n \leq x < c \leq b_n \]
%        e quindi si deduce che l'intervallo \([x,c] \subseteq [a_n, b_n] = I_n\)
%        da cui
%        \[
%            [c, e] \in \bigcap_{n\in\mathbb{N}} I_n = \{c\}
%        \]
%        che quindi è assurdo.
%    }{
%        TODO (in the future)
%    }
%}

\end{document}