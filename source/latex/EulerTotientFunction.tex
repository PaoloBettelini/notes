\documentclass[preview]{standalone}

\usepackage{amsmath}
\usepackage{amssymb}
\usepackage{stellar}
\usepackage{definitions}
\usepackage{boldline}

\begin{document}

\id{integers-euler-totient-function}
\genpage

\section{Euler's totient function}

\plain{We are now interested in counting how many congruence classes are invertible.}

\begin{snippetdefinition}{euler-totient-function-definition}{Euler's totient function}
    The \textit{Euler's totient function} counts the positive integers up
    to a given integer \(n\) that are \coprime to \(n\).
    \[
        \varphi(n)
    \]
\end{snippetdefinition}

\begin{snippet}{euler-totient-function-table-transposed}
    \begin{center}
        \bgroup{}
        \def\arraystretch{1.25}
        \begin{tabular}{|c| c| c| c| c| c| c| c| c|}
            \hline
            \(n\) & 1 & 2 & 3 & 4 & 5 & 6 & 7 & 8 \\
            \hline
            \(\eulertotient(n)\) & 1 & 1 & 2 & 2 & 4 & 2 & 6 & 4 \\
            \hline
        \end{tabular}
        \egroup{}
    \end{center}
    \phantom{}
\end{snippet}

\begin{snippetcorollary}{euler-totient-of-prime}{Euler's totient function of prime}
    Let \(p\) be a \primen. Then,
    \[ \eulertotient(n) = n-1 \]
\end{snippetcorollary}

\begin{snippetproposition}{amount-of-invertible-classes}{Amount of invertible modulo classes}
    The amount of \invertiblecongclass[invertible] \congruenceclass[classes] in \(\integers / n\) is \(\eulertotient(n)\).
\end{snippetproposition}

\begin{snippetproof}{amount-of-invertible-classes-proof}{amount-of-invertible-classes}{Amount of invertible modulo classes}
    The distinct \congruenceclass[classes] in \(\integers / n\) are
    \[
        {[0]}_n, {[1]}_n, \cdots, {[n-1]}_n
    \]
    Since \({[a]}_n\) is \invertiblecongclass[invertible] \ifandonlyif \(\gcd(a,n)=1\),
    the amount of \invertiblecongclass[invertible] \congruenceclass[classes] is given by \(\eulertotient(n)\).
\end{snippetproof}

\section{Theorems}

\begin{snippettheorem}{euler-theorem}{Euler's theorem}
    Let \(a,n\in\integers\) such that \(a\) and \(n\) are \coprime.
    Then,
    \[
        a^{\eulertotient(n)} \equiv 1 \pmod{n}
    \]
\end{snippettheorem}

\begin{snippetproof}{euler-theorem-proof}{euler-theorem}{Euler's theorem}
    Consider the \invertiblecongclass[invertible] \congruenceclass[classes] in \(\integers / n\)
    \[ {[b_1]}_n, {[b_2]}_n, \cdots, {[b_{\eulertotient(n)}]}_n \]
    Now, the \congruenceclass[classes] \[
        {[a]}_n{[b_1]}_n, {[a]}_n{[b_2]}_n, \cdots, {[a]}_n{[b_{\eulertotient(n)}]}_n 
    \]
    Since \({[a]}_n\) and \({[b_k]}_n\) are \invertiblecongclass[invertible], then
    \({[a]}_n{[b_k]}_n\) is also \invertiblecongclass[invertible]. They are also distinct
    by the cancellation law. Thus, the former and latter list coincide up to ordering.
    Given this information, we consider the product of said \congruenceclass[classes]:
    \begin{align*}
        \prod_{k=1}^{\eulertotient(n)} {[b_k]}_n &= \prod_{k=1}^{\eulertotient(n)} {[a]}_n{[b_k]}_n \\
        &= \left( \prod_{k=1}^{\eulertotient(n)} {[b_k]}_n \right)
        \left( \prod_{k=1}^{\eulertotient(n)} {[a]}_n \right)
    \end{align*}
    Since the \congruenceclass[classes] \({[b_k]}_n\) are \invertiblecongclass[invertible],
    we can simplify them
    \begin{align*}
        {[1]}_n &= \left( \prod_{k=1}^{\eulertotient(n)} {[a]}_n \right) \\
        &= {[a^{\eulertotient(n)}]_n}
    \end{align*}
\end{snippetproof}

\begin{snippettheorem}{fermat-little-theorem}{Fermat's little theorem}
    Let \(a,p\in\integers\) such that \(p\) is \primen. Then,
    \[ a^p \equiv a \pmod{p} \]
\end{snippettheorem}

\begin{snippetproof}{fermat-little-theorem-proof}{fermat-little-theorem}{Fermat's little theorem}
    If \(a\) and \(p\) are \coprime, by \eulertheorem we have
    \[
        a^{\eulertotient(p)} \equiv 1 \pmod{p}
    \]
    However, \(\eulertotient(p) = p-1\) since \(p\) is \primen, and thus
    \[
        a^{p-1} \equiv 1 \pmod{p}
    \]
    from which we get
    \[
        a^p \equiv p \pmod{p}
    \]
\end{snippetproof}

\begin{snippet}{fermat-little-theorem-but-not-coprime}
    If \(a\) is not \coprime with \(p\), we have that \(p\divides a\). That is,
    \(a \equiv 0 \pmod{p}\), and, thus, every power with positive exponent of \(a\) is
    congruent to \(0 \pmod{p}\). In particular, \(a^p \equiv 0 \pmod{p}\).
\end{snippet}

\section{Computing Euler's totient function}

\begin{snippetlemma}{euler-totient-function-lemma1}{}
    Let \(m,n \in \integers^+\) where \(m\) and \(n\) are \coprime. Then,
    \[ \eulertotient(mn) = \eulertotient(m)\eulertotient(n) \]
\end{snippetlemma}

\begin{snippetproof}{euler-totient-function-lemma1-proof}{euler-totient-function-lemma1}{}
    Consider the \function \(f\colon \integers / mn \fromto \integers / m \cartesianprod \integers / n\)
    defined as
    \[
        {[a]}_{mn} \to ({[a]}_m, {[a]}_n)
    \]
    This function is well-defined. Indeed,
    if \({[a]}_{mn} = {[b]}_{mn}\),
    which means that if \(a-b\) is a multiple of \(mn\), then \(a-b\)
    is a multiple of \(m\) (i.e. \({[a]}_m = {[b]}_m\)) and \(a-b\)
    is a multiple of \(n\) (i.e. \({[a]}_n = {[b]}_n\)).

    The function is \injective. Indeed, if we consider
    two classes \({[a]}_{mn}\) and \({[b]}_{mn}\) with the same image, that is
    \(({[a]}_{m}, {[a]}_{n}) = ({[b]}_{m}, {[b]}_{n})\), we have
    \(a-b\) is a multiple of \(m\) and \(a-b\) is a multiple of \(n\).
    
    Since \(m\) and \(n\) are \coprime, we have that \(a-b\) is a multiple of \(mn\),
    which means \({[a]}_{mn} = {[b]}_{mn}\).
    % If m and n are not coprime, the function is not injective but
    % it is still well defined.
    Now, \(\cardinality{\integers / mn} = mn\) and \(\cardinality{\integers / m \cartesianprod \integers / m} = mn\).
    The function is thus a an \injective \function between two \set[sets] of the same size,
    so it is \bijective.

    We now consider the subsets of \invertiblecongclass[invertible] elements of
    \(\integers / mn\), which we will denote \(\text{Inv}(\integers / mn)\).
    Thus,
    \[
        \cardinality{\text{Inv}(\integers / k)} = \eulertotient(k)
    \]
    If we prove that the \bijective between \(\integers / mn\)
    and \(\integers / m \cartesianprod \integers / mn\) induces a \bijective[bijection]
    between \(\text{Inv}(\integers / mn)\) and \(\text{Inv}(\integers / m \cartesianprod \integers / n)\),
    we will prove out thesis.
    Let \({[a]}_{mn} \in \text{Inv}(\integers / mn)\), which means that \(a\)
    is \coprime with \(mn\), and, thus, \coprime with \(m\) and \(n\).
    Thus, \(({[a]}_m, {[a]}_n) \in \text{Inv}(\integers / m) \cartesianprod \text{Inv}(\integers / n)\).
    The \function \(f\) sends elements of \(\text{Inv}(\integers / mm)\)
    into \(\text{Inv}(\integers / m) \cartesianprod \text{Inv}(\integers / m)\).
    Likewise, let  \(({[b]}_m, {[c]}_n) \in \text{Inv}(\integers / m) \cartesianprod \text{Inv}(\integers / n)\)
    and tkae che element \({[a]}_{mn}\) such that \(f({[a]}_{mn}) = ({[b]}_m, {[c]}_n)\).
    We thus have
    \[
        ({[a]}_m, {[a]}_n) = ({[b]}_m, {[c]}_n)
    \]
    which implies that \(a-b\) is a multiple of \(m\) (since \(b\) is \coprime with \(n\), so is \(a\)).
    Also, \(a-c\) is a multiple of \(n\) (since \(c\) is \coprime with \(n\), so is \(a\)).
    In conclusion, \(a\) is \coprime with \(mn\), and, thus,
    \({[a]}_{mn}\) is \invertiblecongclass[invertible] like we wanted.
\end{snippetproof}

\begin{snippetcorollary}{euler-totient-of-prime-factorization}{Euler's totient function of prime factorization}
    Let
    \[
        n = \prod_{i=1}^r p_i^{\alpha_i}
    \]
    where \(i\neq j \implies p_i \neq p_j\)
    and \(\alpha_i \in \integers^+\). Then,
    \[
        \eulertotient(n) = \prod_{i=1}^r \eulertotient(p_i^{\alpha_i})
    \]
\end{snippetcorollary}

\begin{snippet}{temp}
    Now, \[\eulertotient(p^\alpha) = \left|\left\{n \in \integers \suchthat 0 \leq n < p^\alpha, \gcd(n, p^\alpha) = 1 \right\}\right| \]
    Between \(0\) and \(p^\alpha-1\), I need to cout the amount of \coprime[coprimes]
    with \(p^\alpha\), that is the \coprime[coprimes] with \(p\), meaning those
    that are not multiples of \(p\).
    However, the multiples of \(p\) are \(1\) every \(p\).
    That is, \(p\alpha / p = p^{\alpha - 1}\).
    We thus have \(p^\alpha - p^{\alpha - 1}\) numbers left that are \coprime with \(p^\alpha\).
    Thus, \(\eulertotient(p^\alpha) = p^\alpha - p^{\alpha - 1}\).

    In conclusion,
    \begin{align*}
        \eulertotient(n) &= \prod_{i=1}^r \eulertotient(p_i^{\alpha_i}) \\
        &= \prod_{i=1}^r (p_i^{\alpha_i} - p_i^{\alpha_i - 1}) \\
        &= n \prod_{i=1}^r \left(1 - \frac{1}{p_i}\right)
    \end{align*}
\end{snippet}

\end{document}