\documentclass[preview]{standalone}

\usepackage{amsmath}
\usepackage{amssymb}
\usepackage{stellar}
\usepackage{definitions}
\usepackage{bettelini}

\begin{document}

\id{integers-diophantine-equations}
\genpage

\section{Linear diophantine equations}

\begin{snippetdefinition}{linear-diophantine-equation-definition}{Linear diophantine equation}
    A \textit{linear diophantine equations} is an equation with 2 or more integer unknowns of the following form.
    \[
        a_1x_1 + a_2x_2 + \cdots + a_nx_n = b
    \]
    where \(a_i,x_i,b \in \naturalnumbers\) and \(x_i\) are unknowns.
\end{snippetdefinition}

\begin{snippetproposition}{linear-diophantine-equation-solvability}{Linear diophantine equation solvability}
    A linear diophantine equation
    \[
        a_1x_1 + a_2+x_2 + \cdots + a_nx_n = b
    \]
    is solvable \ifandonlyif \(\gcd(a_1,a_2,\cdots,a_n) \divides b\)
    (i.e. \(\gcd(a_1,a_2,\cdots,a_n)c = b\) for some \(c\))
    and given a Bezout's identity of the form
    \[
        a_1u_1 + a_2u_2 + \cdots + a_nu_n = \gcd(a_1, a_2, \cdots, a_n)
    \]
    a solution is given by \(x_1 = u_1 c, x_2 = u_2 c \cdots\).
\end{snippetproposition}

\begin{snippetproof}{linear-diophantine-equation-solvability-proof}{linear-diophantine-equation-solvability}{Linear diophantine equation solvability}
    \iffproof{
        For any solution \(u_1, u_2, \cdots, u_n\) to the equation that we plug in,
        we get that the value \(u_1a_1, u_2a_2, \cdots, u_na_n\) is a multiple of \(gcd(a_1, a_2, \cdots, a_n)\)
        and thus \(gcd(a_1, a_2, \cdots, a_n)\) must divide \(b\).
    }{
        If \(\gcd(x_1, x_2, \cdots, x_n) \divides b\), that is \(b = c\gcd(x_1, x_2, \cdots, x_n)\)
        for some \(c\in \integers\), we can determine a Bezout's identity of the form
        \[
            a_1u_1 + a_2u_2 + \cdots + a_nu_n = \gcd(a_1, a_2, \cdots, a_n)
        \]
        By multiplying this by \(c\), we find
        \[
            ca_1u_1 + ca_2u_2 + \cdots + ca_nu_n = c\gcd(a_1, a_2, \cdots, a_n) = b
        \]
        which means that a solution is given by \(x_1 = u_1 c, x_2 = u_2 c, \cdots\).
    }
\end{snippetproof}

\plain{If there exist solutions, we can show that there actually exist an infinite amount
and how to determine them.}

\subsection{Two unkowns}

\begin{snippetproposition}{linear-diophantine-equation-two-unknown}{Linear Diophantine equation of two unknown}
    Let \[ax+by=c\] be a solvable diophantine equation
    and let \(d = \gcd(a,b)\).
    If \(d=0\), this means that \(a=b=0\) and \(c=0\) since \(d \divides c\) (identity).
    Otherwise, let \(a=da'\), \(b=db'\) and \(c=dc'\), then the equation is equivalent to
    \[
        a'x + b'y = c'
    \]
    Consider any solution to the equation \(x=\overline{x}\) and \(y=\overline{y}\).
    Then, the equation has infinitely many other solutions given by
    \begin{align*}
        x &= \overline{x} + b'h \\
        y &= \overline{y} - a'h
    \end{align*}
    for any \(h \in \naturalnumbers\).
\end{snippetproposition}

\begin{snippetproof}{linear-diophantine-equation-two-unknown-proof}{linear-diophantine-equation-two-unknown}{Linear Diophantine equation of two unknowns}
    We know that \(d = \gcd(a,b) = 0\) \ifandonlyif \(a=b\). Sice \(d\divides c\), we have that \(c=0\)
    and the equation is an identity.
    Otherwise, we have that \(d = \gcd(a,b) \neq 0\).
    In this case, the equation is equivalent to
    \[
        a'x + b'y = c'
    \]
    with \(a=da'\), \(b=db'\) and \(c=dc'\).
    Consider any two solution to the equation \(x=\overline{x}\) and \(y=\overline{y}\)
    and also \(x=\hat{x}\) and \(y=\hat{y}\). Then,
    \(a'\overline{x} + b'\overline{y} = c'\) and \(a'\hat{x} + b'\hat{y} = c'\).
    From these two equations, we get \(a'(\hat{x} - \overline{x}) = b'(\overline{y} - \hat{y})\).
    By \snippetref[coprimes-given-by-gcd][this proposition], \(a'\) and \(b'\) are \coprime,
    and by \snippetref[gcd-coprime-division][this lemma] we have that \(b' \divides \hat{x} - \overline{x}\)
    and \(a' \divides \overline{y} - \hat{y}\). This means that there exist \(h,k\in\integers\)
    such that \(\overline{y}-\hat{y} = ka'\) and \(\hat{x}-\overline{x} = hb'\).
    \begin{align*}
        \hat{x} = \overline{x} + b'h \\
        \hat{y} = \overline{y} - a'k
    \end{align*}
    We thus have \(a'b'h = a'b'k\). If \(a'b' \neq 0\), then \(h=k\). Otherwise,
    \(a'=0\lor b'=0\). Without loss of generality, assume \(a'=0\).
    Then, \(\hat{y} = \overline{y} - 0 \cdot k= \overline{y} - 0 \cdot h\). This means that
    we can always replace \(k=h\).
\end{snippetproof}

\end{document}
