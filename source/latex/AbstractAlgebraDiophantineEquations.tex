\documentclass[preview]{standalone}

\usepackage{amsmath}
\usepackage{amssymb}
\usepackage{parskip}
\usepackage{fullpage}
\usepackage{hyperref}
\usepackage{bettelini}
\usepackage{stellar}

\hypersetup{
    colorlinks=true,
    linkcolor=black,
    urlcolor=blue,
    pdftitle={Stellar},
    pdfpagemode=FullScreen,
}

\newcommand{\divides}{\,|\,}

\begin{document}

\id{integers-diophantine-equations}
\genpage

\section{Linear diophantine equations}

\subsection{Definition}

\begin{snippetdefinition}{linear-diophantine-equation-definition}{Linear diophantine equation}
    A linear diophantine equations is an equation with 2 or more integer unknowns of the following form.
    \[
        a_1x_1 + a_2+x_2 + \cdots + a_nx_n = b
    \]
    where \(a_i,x_i,b \in \mathbb{Z}\) and \(x_i\) are unknowns.
\end{snippetdefinition}

\begin{snippetproposition}{linear-diophantine-equation-solvability}{Linear diophantine equation solvability}
    A linear diophantine equation
    \[
        a_1x_1 + a_2+x_2 + \cdots + a_nx_n = b
    \]
    is solvable iff \(\gcd(a_1,a_2,\cdots,a_n) \divides b\).
\end{snippetproposition}

\begin{snippetproof}{linear-diophantine-equation-solvability-proof}{Linear diophantine equation solvability}
    This is because the left-hand side will always be a value that is a multiple
    of \(\gcd(a_1,a_2,\cdots,a_n)\).
\end{snippetproof}

\begin{snippetproposition}{linear-diophantine-equation-solution}{Linear diophantine equation solution}
    Let
    \[
        a_1x_1 + a_2+x_2 + \cdots + a_nx_n = b
    \]
    be a linear diophantine equation.
    If \(\gcd(a_1,a_2,\cdots,a_n) \divides b\),
    then \(b = \gcd(a_1,a_2,\cdots,a_n)e\) for some \(e\). \\
    By the Bezout identity, which can be find using Euclid's algorithm, we have
    \[a_1v_1+a_2v_2+\cdots+a_nv_n=\gcd(a_1,a_2,\cdots,a_n)\]
    meaning that an integer solution is given by \(x_n=ev_n\).
\end{snippetproposition}

\subsection{Two unkowns}

\begin{snippetproposition}{linear-diophantine-equation-two-unknown}{Linear Diophantine equation of two unknown}
    Let \[ax+by=c\] be a solvable diophantine equation
    and let \(d = \gcd(a,b)\).
    If \(d=0\), this means that \(a=b=0\) and \(c=0\) since \(d \divides c\) (identity).
    Otherwise, let \(a=da'\), \(b=db'\) and \(c=dc'\), then the equation is equivalent to
    \[
        a'x + b'y = c'
    \]
    Consider any solution to the equation \(x=\overline{x}\) and \(y=\overline{y}\),
    then the equation has infinitely many other solutions given by
    \begin{align*}
        x &= \overline{x} + b'h
        y &= \overline{y} + a'h
    \end{align*}
    for any \(h \in \mathbb{Z}\).
\end{snippetproposition}

% PROOF of d != 0
% e da lì in poi

\end{document}
