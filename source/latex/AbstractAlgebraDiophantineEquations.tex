\documentclass[preview]{standalone}

\usepackage{amsmath}
\usepackage{amssymb}
\usepackage{stellar}
\usepackage{definitions}
\usepackage{bettelini}

\begin{document}

\id{integers-diophantine-equations}
\genpage

\section{Linear diophantine equations}

\begin{snippetdefinition}{linear-diophantine-equation-definition}{Linear diophantine equation}
    A \textit{linear diophantine equations} is an equation with 2 or more integer unknowns of the following form.
    \[
        a_1x_1 + a_2x_2 + \cdots + a_nx_n = b
    \]
    where \(a_i,x_i,b \in \naturalnumbers\) and \(x_i\) are unknowns.
\end{snippetdefinition}

% TODOURGENT move this lemma
\begin{snippetlemma}{gcd-coprime-division}{}
    If \(a \divides bc\) and \(\gcd(a,b) = 1\), then \(a\divides c\).
\end{snippetlemma}

\begin{snippetproof}{gcd-coprime-division-proof}{gcd-coprime-division}{}
    We can assume that \(a,b,c\) are positive integers and thus there is an integer \(d\)
    such that \(ad = bc\). If \(a=1\), we have the trivial case \(a\divides c\).
    Otherwise, let \(p\) be a \primen which divides \(a\). This \primen \(p\)
    divdes then \(bc\), and thus either divides \(b\) or \(c\).
    However, \(\lnot(p \divides b)\) (otherwise \(p\divides \gcd(a,b)\) against the hypothesis, which
    is that \(a\) and \(b\) are \coprime). Thus, \(p\divides c\).
    We can then simplify by \(p\) and obtain \(a'd = bc'\) with \(a=pa'\)
    and \(c = pc'\) and proceed by induction; since \(a'<a\) and \(\gcd(a', b) = 1\),
    we have that \(a'\divides c'\), from which \(a\divides c\).
\end{snippetproof}

\begin{snippetproposition}{linear-diophantine-equation-solvability}{Linear diophantine equation solvability}
    A linear diophantine equation
    \[
        a_1x_1 + a_2+x_2 + \cdots + a_nx_n = b
    \]
    is solvable \ifandonlyif \(\gcd(a_1,a_2,\cdots,a_n) \divides b\)
    (i.e. \(\gcd(a_1,a_2,\cdots,a_n)c = b\) for some \(c\))
    and given a Bezout's identity of the form
    \[
        a_1u_1 + a_2u_2 + \cdots + a_nu_n = \gcd(a_1, a_2, \cdots, a_n)
    \]
    a solution is given by \(x_1 = u_1 c, x_2 = u_2 c \cdots\).
\end{snippetproposition}

\begin{snippetproof}{linear-diophantine-equation-solvability-proof}{linear-diophantine-equation-solvability}{Linear diophantine equation solvability}
    \iffproof{
        For any integer value that we assign to \(x_1, x_2, \cdots, x_n\), the first member
        is always a multiple of \(\gcd(x_1, x_2, \cdots, x_n)\).
    }{
        If \(\gcd(x_1, x_2, \cdots, x_n) \divides b\), that is \(b = c\gcd(x_1, x_2, \cdots, x_n)\)
        for some \(c\in \integers\), we can determine a Bezout's identity of the form
        \[
            a_1u_1 + a_2u_2 + \cdots + a_nu_n = \gcd(a_1, a_2, \cdots, a_n)
        \]
        By multiplying this by \(c\), we find
        \[
            ca_1u_1 + ca_2u_2 + \cdots + ca_nu_n = c\gcd(a_1, a_2, \cdots, a_n) = b
        \]
        which means that a solution is given by \(x_1 = u_1 c, x_2 = u_2 c, \cdots\).
        % TODOURGENT coprimi finish proof
    }
\end{snippetproof}

\plain{If there exist solutions, we can show that there actually exist an infinite amount
and how to determine them.}

\subsection{Two unkowns}

\begin{snippetproposition}{linear-diophantine-equation-two-unknown}{Linear Diophantine equation of two unknown}
    Let \[ax+by=c\] be a solvable diophantine equation
    and let \(d = \gcd(a,b)\).
    If \(d=0\), this means that \(a=b=0\) and \(c=0\) since \(d \divides c\) (identity).
    Otherwise, let \(a=da'\), \(b=db'\) and \(c=dc'\), then the equation is equivalent to
    \[
        a'x + b'y = c'
    \]
    Consider any solution to the equation \(x=\overline{x}\) and \(y=\overline{y}\).
    Then, the equation has infinitely many other solutions given by
    \begin{align*}
        x &= \overline{x} + b'h \\
        y &= \overline{y} + a'h % TODOURGENT è -?
    \end{align*}
    for any \(h \in \naturalnumbers\).
\end{snippetproposition}

\end{document}
