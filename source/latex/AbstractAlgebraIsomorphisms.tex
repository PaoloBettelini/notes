\documentclass[preview]{standalone}

\usepackage{amsmath}
\usepackage{amssymb}
\usepackage{stellar}
\usepackage{definitions}

\begin{document}

\id{isomorphisms}
\genpage

\section{Homomorphisms}

% TODOURGENT: define morphism property https://proofwiki.org/wiki/Definition:Morphism_Property

\begin{snippetdefinition}{monoid-homomorphism-definition}{Monoid homomorphism}
    Let \((G, \circ)\) and \((H, \ast)\) be \monoid[monoid].
    A \function \(\varphi\colon G \fromto H\)
    such that:
    \begin{enumerate}
        \item \(\varphi(1_G) = 1_H\);
        \item \emph{morphism property:} \(\forall a,b \in G, \varphi(a\circ b) = \varphi(a) \ast \varphi(b)\)
    \end{enumerate}
    is said to be a \emph{monoid homomorphism}.
\end{snippetdefinition}

\begin{snippetdefinition}{group-homomorphism-definition}{Group homomorphism}
    Let \((G, \circ)\) and \((H, \ast)\) be \group[groups].
    A \function \(\varphi\colon G \fromto H\)
    such that:
    \begin{enumerate}
        \item \emph{morphism property:} \(\forall a,b \in G, \varphi(a\circ b) = \varphi(a) \ast \varphi(b)\)
    \end{enumerate}
    is said to be a \emph{group homomorphism}.
\end{snippetdefinition}

\plain{In the case of a group, the morphism property with the cancellation law implies that property of the identity element.}

\begin{snippetproposition}{group-homomorphism-preserves-identity}{Group homomorphism preserves identity}
    Let \((G, \circ)\) and \((H, \ast)\) be \group[groups] and let
    \(\varphi\colon G \fromto H\) be a \grouphomomorphism.
    Then,
    \[
        \varphi(1_G) = 1_H
    \]
\end{snippetproposition}

\begin{snippetproof}{group-homomorphism-preserves-identity-proof}{group-homomorphism-preserves-identity}{Group homomorphism preserves identity}
    We start by the trivial property
    \begin{align*}
        1_G \circ 1_G &= 1_G \\
        \varphi(1_G \circ 1_G) &= \varphi(1_G)
    \end{align*}
    And since \(\varphi\) is a \grouphomomorphism,
    \(\varphi(1_G) \circ \varphi(1_G) = \varphi(1_G)\). \\
    This last value is an identity in \((H, \ast)\), meaning
    \begin{align*}
        (\varphi(1_G))\ast (\varphi(1_G)) &= (\varphi(1_G)) \circ 1_H \\
        \varphi(1_G) &= 1_H
    \end{align*}
    by the cancellation law.
\end{snippetproof}

\begin{snippetproposition}{exist-homomorphism-between-groups}{There always exist homomorphism between groups}
    Let \((G, \circ)\) and \((H, \ast)\) be \group[groups].
    Then, there always exist a \grouphomomorphism between them.
\end{snippetproposition}

\begin{snippetproof}{exist-homomorphism-between-groups-proof}{exist-homomorphism-between-groups}{There always exist homomorphism between groups}
    Consider the simple construction \(\varphi \colon G \fromto H\)
    such that \(\varphi(x) = 1_H\) for every \(x\in G\).
\end{snippetproof}

\begin{snippetproposition}{group-homomorphism-composition}{}
    Let \(\varphi\colon G \fromto H\) and \(\psi\colon H \fromto K\)
    be \grouphomomorphism[group homomorphisms].
    Then, \(\psi(\varphi)\) is a \grouphomomorphism.
\end{snippetproposition}

\begin{snippetproof}{group-homomorphism-composition-proof}{group-homomorphism-composition}{}
    Let \(x,y \in G\). Then,
    \[
        \psi(\varphi(xy)) = \psi(\varphi(x)\varphi(y)) = \psi(\varphi(x)) \psi(\varphi(y))
    \]
\end{snippetproof}

\plain{The same goes for a monoid, with the addition need to check that the identity is preserved.}

\section{Isomorphisms}

\begin{snippetdefinition}{group-isomorphism-definition}{Group isomorphism}
    Let \((G, \circ)\) and \((H, \ast)\) be \group[groups] and let
    \(\varphi\colon G \fromto H\) be a \grouphomomorphism.
    Then, \(\varphi\) is said to be a \emph{group isomorphism} if it is \bijective.
    \[
        (G, \circ) \cong (H, \circ)
    \]
\end{snippetdefinition}

\begin{snippetproposition}{group-isomorphism-inverse}{}
    The inverse of a \groupisomorphism is a \groupisomorphism.
\end{snippetproposition}

\begin{snippetproof}{group-homomorphism-inverse-proof}{group-homomorphism-inverse}{}
    Let \(\varphi \colon G \fromto H\) be a \groupisomorphism.
    Since the \function is \bijective, \(\varphi^\inversefunction\) exists
    and it is also \bijective.
    Let \(x,y\in H\). By definition, \(\varphi^\inversefunction(xy)\),
    is the only \(z\in G\) such that \(\varphi(z) = xy\).
    Likewise, \(\varphi^\inversefunction(x)\) is the only \(w\in G\)
    such that \(\varphi(w) = x\) and \(\varphi^\inversefunction(y)\)
    is the only \(u \in G\) such that \(\varphi(w) = y\).
    Thus, \(xy = \varphi(w)\varphi(u) = \varphi(wu)\)
    which implies
    \[
        \varphi^\inversefunction(xy) = \varphi^\inversefunction(\varphi(wu))
        = wu = \varphi^\inversefunction(x) \varphi^\inversefunction(y)
    \]
\end{snippetproof}

% TODOURGENT
\begin{snippetdefinition}{endomorphism-definition}{Endomorphism}
    \todo
\end{snippetdefinition}

\begin{snippetdefinition}{automorphism-definition}{Automorphism}
    \todo
\end{snippetdefinition}

\begin{snippettheorem}{endomorphism-automorphism-monoid-theorem}{}
    Let \(G\) be a \group.
    The \set of endomorphisms of \(G\)
    induces a \monoid with respect to composition \(\text{End}(G)\).
    The \group of the invertibles of \(\text{End}(G)\), denoted \(\text{Aut}(G)\),
    is formed by the automorphisms of \(G\).
\end{snippettheorem}

\begin{snippetproof}{endomorphism-automorphism-monoid-theorem-proof}{endomorphism-automorphism-monoid-theorem}{}
    We know that the set of \function[functions]
    from a \set to itself is a \monoid with respect to composition.
    In particular, the \function[functions] from the \group \(G\)
    in themselves form a \monoid. Within them, we consider
    those who are endomorphisms \(\text{End}(G)\).
    We have:
    \begin{enumerate}
        \item \(\text{Id}_G \in \text{End}(G)\);
        \item if \(\varphi\in \text{End}(G)\) and \(\psi \in \text{End}(G)\),
        then \(\psi(\varphi) \in \text{End}(G)\)
    \end{enumerate}
    meaning \(\text{End}(G)\) is a \submonoid of \(G^G\).
    The invertible elements of \(\text{End}(G)\) are precisely the automorphisms
    of \(G\).
\end{snippetproof}

\begin{snippetproposition}{group-homomorphism-preserved-exponentiation}{}
    Let \(\varphi \colon G \fromto H\) be a \grouphomomorphism.
    For every \(n\in\integers\) and every \(x\in G\) we have
    \[
        \varphi(x^n) = {(\varphi(x))}^n
    \]
\end{snippetproposition}

\begin{snippetproof}{group-homomorphism-preserved-exponentiation-proof}{group-homomorphism-preserved-exponentiation}{}
    We first proceed by \principleofinduction[induction] on \(n \geq 0\):
    \begin{itemize}
        \item the base case is trivial;
        \item assume that the property works for the value \(k\).
        \begin{align*}
            \varphi(x^{k+1}) = \varphi(x^k \circ x) = \varphi(x^k) \varphi(x) = {\left(\varphi(x)\right)}^{k+1}
        \end{align*}
    \end{itemize}
    We now consider the case \(n < 0\).
    We have \(-n > 0\) and, thus, \(\varphi(x^{-n}) = {(\varphi(x))}^{-n}\).
    We multiply both members by \(\varphi(x^n)\)
    and we find
    \[
        \varphi(x^n) \varphi(x^{-n}) = \varphi(x^n) {(\varphi(x))}^{-n}
    \]
    from which we get
    \[
        \varphi(x^n x^{-n}) = \varphi(x^n) {(\varphi(x))}^{-n}
    \]
    and
    \[
        \varphi(1_G) = \varphi(x^n) {(\varphi(x))}^{-n}
    \]
    meaning that \(1_H = \varphi(x^n) {(\varphi(x))}^{-n}\).
    Finally, \(\varphi(x^n)\) and \({(\varphi(x))}^{-n}\) are inverses of eachother
    \[
        \varphi(x^n) = {\left({\left(\varphi(x)\right)}^{-h}\right)}^{-1} =  {(\varphi(x))}^{n}
    \]
\end{snippetproof}

\begin{snippetcorollary}{group-homomorphism-period}{}
    Let \(\varphi \colon G \fromto H\) be a \grouphomomorphism.
    If \(x\in G\) has finite period \(n\), then \(\varphi(x)\)
    has finite period \(m\) where \(m \divides n\).
\end{snippetcorollary}

\begin{snippetproof}{group-homomorphism-period-proof}{group-homomorphism-period}{}
    If \(x^n = 1_G\), then \(\varphi(x^n) = \varphi(1_G) = 1_H\), but \(\varphi(x^n) = {(\varphi(x))}^n\),
    meaning \({(\varphi(x))}^n = 1_H\) and, consequently, \(\varphi(x)\) has a period which divides \(n\).
\end{snippetproof}

\plain{An element of finite period cannot be sent into one of infinite period. However, the opposite is possible.}

\begin{snippetproposition}{group-homomorphism-preserves-subgroup}{Group homomorphism preserves subgroup}
    Let \(\varphi \colon G \fromto H\) be a \grouphomomorphism.
    Then:
    \begin{enumerate}
        \item \(K \subgroupleq G \implies \varphi(K) \subgroupleq H\);
        \item \(L \subgroupleq H \implies \varphi^{\inversefunction}(L) \subgroupleq G\).
    \end{enumerate}
\end{snippetproposition}

\begin{snippetproof}{group-homomorphism-preserves-subgroup-proof}{group-homomorphism-preserves-subgroup}{}
    \begin{enumerate}
        \item \begin{enumerate}
            \item \(\varphi(K) \neq \emptyset\) because \(K \neq \emptyset\);
            \item let \(x,y \in \varphi(K)\), meaning that there exist \(u,v \in K\)
            such that \(x=\varphi(u)\) and \(y = \varphi(v)\).
            But then \begin{align*}
                xy^{-1} &= (\varphi(u)){(\varphi(v))}^{-1} = (\varphi(u)) \varphi(v^{-1}) \\
                &= \varphi(uv^{-1}) \in \varphi(K)
            \end{align*}
        \end{enumerate}
        \item \begin{enumerate}
            \item we have \(\varphi^\inversefunction(L) \neq \emptyset\)
                because \(\varphi(1_G) = 1_H\)
                and \(1_H \in L\) and thus \(1_G \in \varphi^\inversefunction(L)\);
            \item let \(x,y \in \varphi^\inversefunction(L)\).
            We have
            \[
                \varphi(xy^{-1}) = \varphi(x) \varphi(y^{-1}) = \varphi(x) {(\varphi(y))}^{-1} \in L
            \]
            meaning \(xy^{-1} \in \varphi^\inversefunction(L)\).
        \end{enumerate}
    \end{enumerate}
\end{snippetproof}

\plain{This also works in a monoid as the definition requires the identity to be preserved.}

\end{document}