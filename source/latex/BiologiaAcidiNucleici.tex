\documentclass[preview]{standalone}

\usepackage{amsmath}
\usepackage{amssymb}
\usepackage{stellar}
\usepackage{bettelini}

\hypersetup{
    colorlinks=true,
    linkcolor=black,
    urlcolor=blue,
    pdftitle={Biologia},
    pdfpagemode=FullScreen,
}

\begin{document}

\title{Biologia}
\id{biologia-acidi-nucleici}
\genpage

\section{Acidi nucleici}

\plain{I monomeri degli acidi nucleici si chiamano <b>nucleotidi</b>. Assieme compongono i polimeri di DNA e RNA}

\begin{snippetdefinition}{nucleotide-definition}{Nulceotide}
    I \textit{nucleotidi} sono composti da un gruppo fosfato, zucchero e base azotata.
\end{snippetdefinition}

\begin{snippetdefinition}{dna-definition}{DNA}
    Il \textit{DNA} è composto da due filamenti di nucleotidi.
\end{snippetdefinition}

\subsection{Composizione}

\includesnpt[width=50\%|src=/snippet/static/acidi-nucleici-composizione.png]{centered-img}

\begin{snippet}{acido-nucleico-composizione-expl}
    \begin{itemize}
        \item \textbf{gruppo fosfato:} possiede proprietà acide;
        \item \textbf{zucchero:} può essere il monosaccaride \textit{ribosio}, o il \textit{deossiribodio},
            che contiene un atomo in meno di ossigeno rispetto al ribosio;
        \item \textbf{base azotata:} adenina, timina, citosima, guanina, uracile.
    \end{itemize}
\end{snippet}

\end{document}