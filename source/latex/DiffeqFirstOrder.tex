\documentclass[preview]{standalone}

\usepackage{amsmath}
\usepackage{amssymb}
\usepackage{fullpage}
\usepackage{bettelini}
\usepackage{stellar}
%\usepackage{pgfplots}
%\usepackage{wrapfig}

%\usetikzlibrary{calc}

\hypersetup{
    colorlinks=true,
    linkcolor=black,
    urlcolor=blue,
    pdftitle={Differential Equations},
    pdfpagemode=FullScreen,
}

\begin{document}

\title{First Order Differential Equations}
\id{diffeq-first-order}
\genpage

\section{First-Order Differential Equations}

\begin{snippet}{diffeq-first-order-defintion}
A first-order differential equation is a differential equation in the form
\[
    y'(t)=f(t,y(t))
\]
where \(f\) is given.

The equation is said to be \textit{linear}
if \(f\) is linear on the second argument.
\[
    y'(t)=a(t)y(t) + b(t)
\]
The equation is also said to be \textit{constant}
if \(a\) and \(b\) are also constant.
\end{snippet}

\section{Constant Coefficient Differential Equation}

\begin{snippettheorem}{diffeq-constant-coeff-sol}{Constant Coefficients Differential Equations}{
    The general solution to the constant differential equation
    \[
        y' = ay+b,
        \quad a \neq 0
    \]
    is given by
    \[
        y(t)=Ce^{at}-\frac{b}{a}, \quad C \in \mathbb{R}
    \]
}
\end{snippettheorem}

\begin{snippetproof}{diffeq-constant-coeff-sol-proof-1}{Constant Coefficients Differential Equations}{
    Let's first consider the case when \(b=0\),
    \[
        y'=ay
    \]
    We divide both sides by \(y\) and simplify
    \[
        \frac{y'}{y}=a
        \implies
        \ln|y|'=a
        \implies
        \ln|y|=at+c_0
    \]
    concluding that
    \[
        y = \pm e^{at+c_0}=\pm e^{c_0} \cdot e^{at} = Ce^{at}
    \]
    Now let's consider \(b \in \mathbb{R}\)
    \[
        y'=a\left(y+\frac{b}{a}\right)
        \implies
        \left(y + \frac{b}{a}\right)' = a\left(y + \frac{b}{a}\right)
    \]
    Note that \(\frac{d}{dx}\left(\frac{b}{a}\right)=0\) \\
    Denoting \(\tilde{y}=y+\frac{b}{a}\), we have
    \[
        \tilde{y}=a\tilde{y}
    \]
    which has solution \(Ce^{at}\), hence
    \begin{align*}
        y+\frac{b}{a}&=Ce^{at} \\
        y &= Ce^{at} - \frac{b}{a}
    \end{align*}
}
\end{snippetproof}

\begin{snippet}{diffeq-potential-function}
It is important to note that we solved the equation by turning
it into a total derivative, which is simple to integrate \((\ln|y|'=a)\).
This function is called a \textit{potential function} \((\psi)\) and it's how
the equation is transformed into a total derivative
\[
    y'=ay+b \rightarrow \psi(t, y(t))'=0
\]
In this case
\[
    \psi = \ln|y| - at
\]

\paragraph{The Integrating Factor Method}
The integrating factor method is a method for solving linear
differential equations.
\\
We will prove the theorem again using this method.
\end{snippet}

\begin{snippetproof}{diffeq-constant-coeff-sol-proof-2}{Constant Coefficients Differential Equations}{
    We choose an integrating factor to be a function \(\mu\) such that
    \[
        \mu'=-a\mu
    \]
    By solving this differential equation we get
    \[
        \frac{\mu'}{\mu} = -a
        \implies
        \ln|\mu| = -at+C
        \implies
        \mu(t) = Ce^{-at}
    \]
    Now we multiply the equation by \(\mu\)
    \begin{align*}
        y'-ay&=b \\
        y'\mu - \mu ay &= b\mu \\
        y'\mu + \mu' y &= b \mu \\
        (\mu y)' &= \mu b
    \end{align*}
    Now choosing \(C=1\)
    \begin{align*}
        \left(e^{-at}y\right)'&=be^{-at} \\
        \left(e^{-at}y\right)'&=\left(-\frac{b}{a}e^{-at}\right)' \\
        \left(e^{-at}y+\frac{b}{a}e^{-at}\right)' &= 0 \\
        \left[\left(\frac{b}{a} + y\right)e^{-at}\right]' &= 0
    \end{align*}
    Now the differential equation is a total derivative of the potential function,
    which in this case in
    \[
        \psi(t, y) = \left(\frac{b}{a} + y\right)e^{-at}
    \]
    This is easy to integrate
    \[
        \left(\frac{b}{a} + y\right)e^{-at} = C
        \implies
        y = Ce^{at} - \frac{b}{a}
    \]
}
\end{snippetproof}

\subsection{Constant Coefficients Differential Equations with Initial Point}

\begin{snippet}{diffeq-ivp-motivation}
We want to constraint the equation such that it has an unique solution
rather than infinite solutions.
\[
    y' = ay + b, \quad y(t_0) = y_0
\]
\end{snippet}

\begin{snippettheorem}{diffeq-constant-coeff-ivp-sol}{Ordinary Constant Differential Equation with IV}{
    The general solution to the ordinary constant differential equation
    \[
        y' = ay + b,
    \]
    with a given point
    \[
        y(t_0) = y_0
    \]
    is given by
    \[
        y(t)=\left(y_0 + \frac{b}{a}\right) e^{a(t-t_0)}- \frac{b}{a},
        \quad a \neq 0
    \]
}
\end{snippettheorem}

\begin{snippetproof}{diffeq-constant-coeff-ivp-sol}{Ordinary Constant Differential Equation with IV}{
    Starting from the general solution of a constant ordinary differential equation
    \[
        y(t_0) = y_0 = Ce^{at_0} - \frac{b}{a}
    \]
    meaning that
    \[
        C = \left(y_0 + \frac{b}{a}\right) e^{-at_0}
    \]
    this constraints out result to
    \begin{align*}
        y(t) &= Ce^{at} - \frac{b}{a} \\
             &= \left(y_0 + \frac{b}{a}\right) e^{-at_0} e^{at} - \frac{b}{a} \\
             &= \left(y_0 + \frac{b}{a}\right) e^{a(t-t_0)} - \frac{b}{a}
    \end{align*}
}
\end{snippetproof}

\subsection{Linear Coefficients Differential Equations}

\begin{snippetdefinition}{diffeq-linear-diffeq}{Linear Differential Equation}{
    An ordinary linear differential equation with variable coefficients is defined as
    \[
        y' = a(t)y(t) + b(t)
    \]
    Where \(a\) and \(b\) are continuous functions.
    The solution to the linear equation with constant coefficients still applies
    if \(\frac{b}{a}\) is constant.
}
\end{snippetdefinition}

\begin{snippettheorem}{diffeq-linear-coeff-sol}{Theorem}{
    The general solution to the differential equation
    \[
        y' = a(t)y(t) + b(t)
    \]
    is given by
    \[
        y(t)=Ce^{A(t)} + e^{A(t)} \int e^{-A(t)} b(t)\, dt
    \]
    where \(A(t)=\int a\,dt\), \(c\in\mathbb{R}\) and \(a\) and \(b\) are
    continuous.
}
\end{snippettheorem}
    
\begin{snippetproof}{diffeq-linear-coeff-sol-proof}{Theorem}{
    Let's start by letting \(b(t)=0\)
    \[
        y' = ay
        \implies
        \frac{y'}{y}=a
        \implies
        \ln|y|'=a
        \implies
        \ln|y|=\int a\,dt
    \]
    concluding that
    \[
        y = \pm e^{A+c_0}=\pm e^{c_0} \cdot e^{A} = Ce^A
    \]
    where \(A=\int a\, dt\).
    \\
    As previosuly, we choose an integrating factor \(\mu\) such that
    \[
        -a\mu = \mu'
    \]
    By solving this differential equation we get
    \[
        \frac{\mu'}{\mu} = -a
        \implies
        \ln|\mu| = -A+C
        \implies
        \mu(t) = Ce^{-A}
    \]
    And by choosing \(C=1\) we have
    \[
        \mu(t) = e^{-A(t)}
    \]
    Now we multiply our equation by the integrating factor
    \begin{align*}
        y' - ay &= b \\
        y'\mu - a\mu y &= \mu b \\
        y' \mu + \mu' y &= \mu b \\
        \left(y \mu\right)' &= \mu b \\
        \left(e^{-A(t)}y\right)' &= e^{-A(t)}b \\
        e^{-A(t)}y &= \int e^{-A(t)}b \,dt + C \\
        y(t) &= Ce^{A(t)} + e^{A(t)} \int e^{-A(t)}b \,dt
    \end{align*}
}
\end{snippetproof}

\subsection{Linear Coefficients Differential Equations with Initial Point}

\begin{snippet}{diffeq-linear-var-coeff-ivp}
A linear differential equation with variable coefficients and an initial point is given by
\[
    y'(t) = a(t)y(t) + b(t), \quad
    y(t_0) = y_0
\]
\end{snippet}

\begin{snippettheorem}{diffeq-linear-var-coeff-sol}{Theorem}{
    The general solution to the differential equation
    \[
        y'=a(t)y(t)+b(t)
    \]
    where \(a\) and \(b\) are continuous functions, with a given point
    \[
        y(t_0)=y_0
    \]
    is given by
    \[
        y(t)=y_0 e^{A(t)}+e^{A(t)} \integral[t_0][t][e^{-A(s)}b(s)][s]
    \]
}
\end{snippettheorem}
\begin{snippetproof}{diffeq-linear-var-coeff-sol-proof}{Theorem}{
    TODO
}
\end{snippetproof}

\subsection{Bernoulli Equation}

\begin{snippetdefinition}{bernoulli-equation-definition}{Bernoulli Equation}{
    The Bernoulli equation has the form
    \[
        y' = p(t)y + q(t)y^n
    \]
}
\end{snippetdefinition}

\begin{snippettheorem}{bernoulli-equation-sol}{Theorem}{
    The general solution of the Bernoulli equation is the general
    solution of the linear equation
    \[
        v' = -(n-1)p(t)v-(n-1)q(t)
    \]
    where
    \[
        v = \frac{1}{y^{n-1}}
    \]
}
\end{snippettheorem}

\begin{snippetproof}{bernoulli-equation-sol-proof}{Theorem}{
    They idea is to transform this equation into a simplier
    linear first-order equation. \\
    Start by dividing both sides by \(y^n\)
    \[
        \frac{y'}{y^n} = \frac{p(t)}{y^{n-1}} + q(t)
    \]
    Let
    \[
        v = y^{-(n-1)}, 
        \quad
        v' = -(n-1)y^{-n}y'
    \]
    Thus
    \[
        -\frac{v'}{n-1} = \frac{y'(t)}{y^n(t)}
    \]
    By substituting we get
    \begin{align*}
        -\frac{v'}{n-1} &= p(t)v + q(t) \\
        v' &= -(n-1)p(t)v-(n-1)g(t)
    \end{align*}
}
\end{snippetproof}

\subsection{Separable equations}

\begin{snippetdefinition}{separale-equation-definition}{Separale Equation}{
    Separable equations are equations that can be solved by integrating both sides.
    This doesn't generally work with first-order linear equations.
    A separable equation has the form
    \[
        h(y)y'=g(t)
    \]
}
\end{snippetdefinition}

\begin{snippettheorem}{separale-equation-sol}{Theorem}{
    A separable differential equation has an implicit solution
    \[
        H(y(t)) = G(t) + C
    \]
    where
    \begin{align*}
        H(y) = \int h(s)\,ds
        ,\quad
        G(t) = \int g(t)\,dt
    \end{align*}
}
\end{snippettheorem}

\begin{snippetproof}{separale-equation-sol-proof}{Theorem}{
    Start by integrating both sides of the equation
    \[
        \int h(y(t))y'(t)\,dt =
        \int g(t)\,dt + C
    \]
    Now substitute for
    \[
        s=y(t),
        \quad
        ds=y'(t)\,dt
    \]
    meaning
    \[
        \int h(s)\, ds = 
        \int g(t)\,dt
    \]
    which could be written as
    \[
        H(y) = G(t) + C
    \]
}
\end{snippetproof}

\subsection{Exact equations}

\begin{snippetdefinition}{exact-equation-definition}{Exact Equation}{
    Consider a differential equation with the form
    \[
        M(x,y)+N(x,y)\frac{dy}{dx}=0
    \]
    Then, the equation is \textit{exact} if there exist a continuously
    differentiable function \(\Psi(x,y)\) such that
    \[
        \frac{\partial \Psi}{\partial x} = M(x,y)
        \quad \text{and} \quad
        \frac{\partial \Psi}{\partial y} = N(x,y)
    \]
}
\end{snippetdefinition}

\begin{snippet}{exact-equation-sol}
We can then rewrite the differential equation as
\[
    \frac{\partial \Psi}{\partial x}+\frac{\partial \Psi}{\partial y} \frac{dy}{dx}=0
\]
Using the multi variable chain rule it can be reduced to
\[
    \frac{d}{dx}\left( \Psi(x,y(x)) \right) = 0
\]
We can clearly see that here the derivative is equal to \(0\), meaning that
the function must be a constant. This gives us an implicit solution
\[
    \Psi(x,y)=C
\]
\end{snippet}

\subsection{Homogeneous equations}

\begin{snippetdefinition}{homogenous-equation-definition}{Homogeneous Equation}{
    A \textit{homogenous} equation has the form
    \[
        \frac{dy}{dx} = F(\frac{y}{x})
    \]
}
\end{snippetdefinition}

\begin{snippet}{homogenous-equation-sol}
We use the substitution
\[
    v=\frac{y}{x}
\]
Note that
\begin{align*}
    y' &= (xv)' = v+yv' \\
    &= F(v)
\end{align*}
We then have
\begin{align*}
    v+xv'&=F(v) \\
    xv'&=F(v)-v \\
    \frac{v'}{F(v)-v}&=\frac{1}{x}
\end{align*}

This is a separable equation. Thus, an implicit solution is given by
\[
    \integral[\frac{1}{F(v)-v}][v]=\ln|x|+C
\]
\end{snippet}

\subsection{Slope Field}

\begin{snippetdefinition}{slope-field-definition}{Slope Field}{
    A slope field or directional field is a field to visualize
    solutions to a first-order differential equation.
}
\end{snippetdefinition}

% TODO
%\begin{snippet}{slope-field-example}
%\begin{wrapfigure}{l}{7.5cm}
%    \begin{tikzpicture}[declare function={f(\x,\y)=\x+\y;}]
%        \def\xmax{3} \def\xmin{-3}
%        \def\ymax{3} \def\ymin{-3}
%        \def\nx{15}
%        \def\ny{15}
%        
%        \pgfmathsetmacro{\hx}{(\xmax-\xmin)/\nx}
%        \pgfmathsetmacro{\hy}{(\ymax-\ymin)/\ny}
%        \foreach \i in {0,...,\nx}
%            \foreach \j in {0,...,\ny}{
%                \pgfmathsetmacro{\yprime}{f({\xmin+\i*\hx},{\ymin+\j*\hy})}
%                \draw[blue,shift={({\xmin+\i*\hx},{\ymin+\j*\hy})}] 
%                (0,0)--($(0,0)!2mm!(.1,.1*\yprime)$);
%            }
%        
%        % a solution y=(yo+1)e^x-x-1
%        \def\yo{1}
%        \draw[magenta] plot[domain=\xmin:.9] (\x,{(\yo+1)*exp(\x)-\x-1});
%        
%        \draw[->] (\xmin-.5,0)--(\xmax+.5,0) node[below right] {\(x\)};
%        \draw[->] (0,\ymin-.5)--(0,\ymax+.5) node[above left] {\(y\)};
%        
%        \draw (current bounding box.south) node[below]
%        {Slope field of \quad \(\frac{dy}{dx}=x+y\).};
%    \end{tikzpicture}
%\end{wrapfigure}
%
%This field is obtained by picking points on the plane. \\
%For each point \((x,y)\) we know that the slope \((\frac{dy}{dx})\)
%is \(x + y\). \\
%This means that if a solution passes through \((x,y)\), then its slope is \(x+y\). \\
%The red curve shows a solution.
%
%\wrapfill
%\end{snippet}

\subsection{Euler's Method}

\begin{snippetdefinition}{euler-method}{Euler's Method}{
    Euler's method is a technique for solving a 
    first-order differential equation numerically given a point of the solution.
    \\
    Starting at the known solution point \(A_0\), we take small steps the direction
    of the slope field. As the length of the steps \(s \to 0\)
    we approach the solution to the equation. \\
}
\end{snippetdefinition}

\end{document}
’