\documentclass[preview]{standalone}

\usepackage{amsmath}
\usepackage{amssymb}
\usepackage{stellar}

\hypersetup{
    colorlinks=true,
    linkcolor=black,
    urlcolor=blue,
    pdftitle={Stellar},
    pdfpagemode=FullScreen,
}

\begin{document}

\id{storia-fake-news}
\genpage

\section{Fake news storiche}

% Da: Fascismo e fake news

\begin{snippet}{fake-news-storiche-expl}
    Le fake news sono in genere effimere, ma quelle storiche sono persistenti e
    profonde nelle persone.

    \begin{itemize}
        \item Più una bugia viene ripetuta, più la si può scambiare per verità.
        \item Notizie di oggi viaggiano velocemente, è difficile bloccarle e smentirle.
        \item Comprendere il passato è un modo per comprendere il presente.
        \item Esistono fake news storiche, ancorate ad un argomento preciso.
        \item Bufale storiche vanno contrastate perché falsificano il passato (così come il ricordo e la memoria).
        \item Bufale storiche nascono da osservazioni o testimonianze inesatte, che poi si diffondono in una società pronta ad accoglierle.
        \item Bufale storiche servono ad alimentare emozioni e a rassicurare: credere in un passato positivo può portare la speranza e rischia di creare una prospettiva a cui tendere.
    \end{itemize}

    Effetti di scardinare le bufale:

    \begin{itemize}
        \item Correggere le informazioni sul passato.
        \item Distruggere sicurezze, e ciò può creare incomunicabilità.
        \item Permette di limitare l'ambito di diffusione di queste notizie, che mistificano la memoria e la percezione del presente.
    \end{itemize}
\end{snippet}

\end{document}