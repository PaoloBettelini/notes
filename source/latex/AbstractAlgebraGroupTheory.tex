\documentclass[preview]{standalone}

\usepackage{amsmath}
\usepackage{amssymb}
\usepackage{amsthm}
\usepackage{parskip}
\usepackage{fullpage}
\usepackage{hyperref}
\usepackage{makecell}
\usepackage{stellar}
\usepackage{definitions}

\begin{document}

\id{groups-introduction}
\genpage

\section{Groups}

\subsection{Cayley tables}

\begin{snippet}{cayley-table-illustration}
A binary operation \(\circ\) on a finite set \(G\) can be
visualized using a \textit{Cayley table}.

Example: \(G=\{0,1\}\) and \(\circ \equiv \text{multiplication}\).
\makecell{
    \begin{tabular}{|c|c|c|}
        \hline
        \(\circ\) & 0 & 1 \\
        \hline
        0 & 0 & 0 \\
        \hline
        1 & 0 & 1 \\
        \hline
    \end{tabular}
}
\\
\phantom{ }
\end{snippet}

\subsection{Definition}

\begin{snippetdefinition}{monoid-definition}{Monoid}
    A \textit{monoid} \((G,\circ)\) is a tuple containing a set \(G\) and
    a binary operation \(\circ \colon G \times G \to G\).
    
    The relation must satisfy the following properties
    
    \begin{enumerate}
        \item \textbf{Associativity}: \(\forall a,b,c\in G, a \circ (b \circ c) = (a \circ b) \circ c\)
        \item \textbf{Identity}: \(\exists e \divides \forall a \in G, ea=ae=a\) 
    \end{enumerate}
    
    The operation \(\circ\) between \(a\) and \(b\) may be written as
    \(a\circ b\) or just \(ab\).
\end{snippetdefinition}

\begin{snippetdefinition}{group-definition}{Group}
    A \textit{group} is a monoid \((G,\circ)\) where each element has an inverse.
    
    \begin{enumerate}
        \item \textbf{Inverse}: \(\forall a \in G, \exists a^{-1} \in G \divides a^{-1}a = aa^{-1} = e\)
    \end{enumerate}
\end{snippetdefinition}

\subsection{Proof of uniqueness of the identity element}

\begin{snippettheorem}{uniqueness-of-the-identity-element}{Uniqueness of the identity element}
    If \(e\) is an identity element of a group, then it is unique.
\end{snippettheorem}

\begin{snippetproof}{uniqueness-of-the-identity-element-proof}{Uniqueness of the identity element}
    Suppose there is more than one identity element, \(e_1\) and \(e_2\).
    \begin{align*}
        e_1 &= e_1 \circ e_2 &\text { since \(e_2\) is an identity} \\
        &= e_2 &\text { since \(e_1\) is an identity}
    \end{align*}
    Thus, \(e_1\) and \(e_2\) must be the same. This reasoning can be extended
    to when we may suppose to have \(n\) identity elements.
\end{snippetproof}

\subsection{Proof of uniqueness of the inverse element}

\begin{snippettheorem}{uniqueness-of-the-inverse-element}{Uniqueness of the inverse element}
    If \(a^{-1}\) is an inverse of \(a\) in a group, then it is unique.
\end{snippettheorem}

\begin{snippetproof}{uniqueness-of-the-inverse-element-proof}{Uniqueness of the inverse element}
    Suppose we have \(a\in G\) with inverses \(b\) and \(c\).
    \begin{align*}
        b = b \circ e &= b \circ (a \circ c)\\
        (b \circ a) c &= e \circ c \\
        &= c
    \end{align*}
    Thus, \(b\) and \(c\) must be the same. This reasoning can be extended
    to when we may suppose to have \(n\) inverses of \(a\).
\end{snippetproof}

\subsection{Cancellation laws}

\begin{snippettheorem}{right-cancellation-law}{Right cancellation law}
    \[
        ba = ca \implies b = c
    \]
\end{snippettheorem}

\begin{snippettheorem}{left-cancellation-law}{Left cancellation law}
    \[
        ab = ac \implies b = c
    \]
\end{snippettheorem}

\subsection{Inverse of Product}

\begin{snippettheorem}{inverse-of-product}{Inverse of Product}
    Consider a group \((G, \circ)\). For any two elements \(a,b \in G\)
    \[{(a \circ b)}^{-1} = b^{-1} \circ a^{-1}\]
\end{snippettheorem}

\begin{snippetproof}{inverse-of-product-proof}{Inverse of Product}
    We start by noticing that by associativity we have
    \begin{align*}
        (a \circ b) \circ (b^{-1} \circ a^{-1}) &= a \circ (b \circ b^-1) \circ a^{-1} \\
        &= a \circ e \circ a^{-1} \\
        &= a \circ a^{-1} \\
        &= e
    \end{align*}
    This implies that \((a \circ b)\) is the inverse of \((b^{-1} \circ a^{-1})\).
    Since \((a\circ b) \circ {(a \circ b)}^{-1} =e\) we have
    \begin{align*}
        (a \circ b) \circ (b^{-1} \circ a^{-1}) = e = (a\circ b) \circ {(a \circ b)}^{-1}
    \end{align*}
    We can clearly see that \((b^{-1} \circ a^{-1}) = {(a \circ b)}^{-1}\).
\end{snippetproof}

\begin{snippettheorem}{generalized-inverse-of-product}{Generalized Inverse of Product}
    Consider a group \((G, \circ)\). For any elements \(a_1,a_2,\cdots,a_n \in G\)
    \[
        (a_1 \circ a_2 \circ \dots a_n)^{-1}
        = a_n^{-1} \circ \dots \circ a_2^{-1} \circ a_1^{-1}
    \]
\end{snippettheorem}

\section{Subgroups}

\subsection{Definition}

\begin{snippetdefinition}{subgroup-definition}{Subgroups}
    Given an algebraic structure \(g=(G, \circ)\) and a group \(h=(H, \circ)\), \(h\)
    is a subgroup of \(g\) (\(g \leq h\)) if \(H \subseteq G\).
\end{snippetdefinition}

\subsection{One-Step Subgroup Test}

\begin{snippettheorem}{one-step-subgroup-test}{One-Step Subgroup Test}
    Let \((G, \circ)\) be a group and let \(H \subseteq G\) where \(\emptyset \neq H\).\\
    Then \((H, \circ)\) is a subgroup of \((G, \circ) \iff
    \forall a,b \in H, a \circ b^{-1} \in H\).
\end{snippettheorem}

\begin{snippetproof}{one-step-subgroup-test-proof}{One-Step Subgroup Test}
    (\(\implies\)): Assume \((H, \circ) \leq (G, \circ)\).
    The properties of a group directly infer \(\forall a,b \in H, a \circ b^{-1} \in H\) \\
    (\(\impliedby\)): Assume \(\forall a,b \in H, a \circ b^{-1} \in H\)
    \begin{itemize}
        \item \textbf{Identity}: let \(a=b\), then \(a\circ a^{-1} H \implies e \in H\).
        \item \textbf{Inverse}: Let \(k\in H\), \(a=e\) and \(b=k\).
        \(a\circ b^{-1} = e \circ k^{-1} \implies k^{-1} \in H\).
        \item \textbf{Closure}: Let \(m, n \in H \implies n^{-1} \in H\) and let \(a=m\) and \(b=n^{-1}\).
        \(a\circ b^{-1} = a \circ (b^{-1})^{-1}=a\circ b\). This implies \(a, b \in H\).
    \end{itemize}
\end{snippetproof}

\subsection{The centralizer subgroup}

\begin{snippetdefinition}{centralizer-group}{The centralizer subgroup}
    Let \(H \leq G\) be groups and define
    \[
        \text{C}_G(H) = \{
            g \in G \divides \forall h \in H, gh=hg
        \}
    \]
    as the centralizer of \(H\).
    This is the set of all elements of \(G\) such that they commute with every element of \(H\).
\end{snippetdefinition}

\begin{snippettheorem}{centralizer-of-subgroup-is-subgroup}{}
    Let \(H \leq G\), then \(\text{C}_G(H) \leq G\).
\end{snippettheorem}

\begin{snippetproof}{centralizer-of-subgroup-is-subgroup-proof}{Centralizer of subgroup is subgroup}
    Suppose \(a,b \in \text{C}_G(H)\).
    We want to show \(ab^{-1} \in \text{C}_G(H)\).\\
    Note that the condition \(gh=hg \iff hg^{-1}=g^{-1}h\).\\
    Consider the expression \((ab^{-1})h = a(b^{-1}h) = ahb^{-1} = h(ab^{-1})\).
    This means that \(ab^{-1} \in \text{C}_G(H)\) and thus in \(H\).
\end{snippetproof}

\subsection{The conjugate subgroup}

\begin{snippetdefinition}{conjugate-subgroup}{The conjugate subgroup}
    Let \(H \leq G\) be groups and define
    \[
        g^{-1}Hg = \{
            g^{-1}hg \divides h \in H    
        \}
    \]
    as the conjugate subgroup.
\end{snippetdefinition}

\begin{snippettheorem}{conjugate-subgroup-is-subgroup}{}
    Let \(H \leq G\), then \(g^{-1}Hg \leq G\).
\end{snippettheorem}

\begin{snippetproof}{conjugate-subgroup-is-subgroup-proof}{}
    Suppose \(a,b \in g^{-1}Hg\).
    We want to show \(ab^{-1} \in g^{-1}Hg\).\\
    Note that \(a = g^{-1}h_1g\) and \(b = g^{-1}h_2g\)
    for some \(h_1, h_2 \in H\). \\
    This means that \(ab^{-1}=a{(g^{-1}h_2g)}^{-1} = a(g^{-1}h_2^{-1}g)
    =g^{-1}h_1gg^{-1}h_2^{-1}g = g^{-1} (h_1h_2) g \in g^{-1}Hg \)
    because \(h_1h_2 \in H\).
\end{snippetproof}

\section{Center of a group}

\begin{snippetdefinition}{center-of-group}{Center of a group}
    Let \(G\) be a group. The center of the group \(G\) is defined as
    \[
        \text{Z}(G) = \{
            g \in G \divides \forall x \in G, gx = xg
        \}
    \]

    This is the set of all elements that commute with every other element.
    The condition \(gx=xg\) is also sometimes expressed as \(gxg^{-1} = x\).
\end{snippetdefinition}

\begin{snippettheorem}{center-of-group-is-subgroup}{}
    Let \(G\) be a group, then \(\text{Z}(G) \leq G\).
\end{snippettheorem}

\begin{snippetproof}{center-of-group-is-subgroup-proof}{}
    Assume \(a, b \in \text{Z}(G)\) meaning \(a = gag^{-1}\) and \(b = gag^{-1}\) for any \(g \in G\). \\
    We want to show \(ab^{-1} \in \text{Z}(G)\).
    \(ab^{-1} = (gag^{-1}){(gbg^{-1})}^{-1} = gag^{-1}gb^{-1}g^{-1}
    = g ab^{-1} g^{-1}\) which is precisely the requirement to be in \(\text{Z}(G)\).
\end{snippetproof}

% https://www.youtube.com/watch?v=m4yYeTGe-ic&list=PL22w63XsKjqwN7sHsEiy0yqkcjQfXAuVb

\end{document}
