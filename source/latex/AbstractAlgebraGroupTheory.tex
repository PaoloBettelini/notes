\documentclass[preview]{standalone}

\usepackage{amsmath}
\usepackage{amssymb}
\usepackage{amsthm}
\usepackage{makecell}
\usepackage{stellar}
\usepackage{bettelini}
\usepackage{definitions}

\begin{document}

\id{groups-introduction}
\genpage

\section{Groups}

\subsection{Definition}

\begin{snippetdefinition}{group-definition}{Group}
    A \textit{group} is a \monoid \((G,\circ)\) where each element has an inverse.
    
    \begin{enumerate}
        \item \textbf{Inverse}: \(\forall a \in G, \exists a^{-1} \in G \suchthat a^{-1}a = aa^{-1} = e\)
    \end{enumerate}
\end{snippetdefinition}

\begin{snippetdefinition}{abelian-group-definition}{Abelian group}
    A \group \((G,\circ)\) is said to be \textit{abelian} if it is commutative under addition.
    \begin{enumerate}
        \item \textbf{Commutativity}: \(\forall a,b \in G, a+b=b+a\)
    \end{enumerate}
\end{snippetdefinition}

\begin{snippettheorem}{group-from-monoid-theorem}{Group from monoid}
    Let \((S, \circ)\) be a \monoid.
    Then, \((\text{Inv}(S), \circ)\) is a \group.
\end{snippettheorem}

\begin{snippetproof}{group-from-monoid-proof}{group-from-monoid}{Group from monoid}
    We know that:
    \begin{enumerate}
        \item \(1 \in \text{Inv}(M)\); %TODOURGENT define Inv
        \item se \(x\in\text{Inv}(M) \land y \in \text{Inv}(M)\), allora anche \(xy\in \text{Inv}(M)\);
        \item se \(x \in \text{Inv}(M)\), allora \(x^{-1} \in \text{Inv}(M)\).
    \end{enumerate}
    By (1), the operation has an identity.
    By (2), the operation is closed. By (3),
    the inverse elements are in the set. Thus, it is a group.
\end{snippetproof}

\subsection{Cancellation laws}

\begin{snippettheorem}{right-cancellation-law}{Right cancellation law}
    Let \((G, \circ)\) be a \group. Then,
    \[
        \forall a,b,c \in G, b \circ a = c \circ a \implies b = c
    \]
\end{snippettheorem}

\begin{snippetproof}{right-cancellation-law-proof}{right-cancellation-law}{Right cancellation law}
    Let \(a,b,c \in G\).
    \begin{align*}
        b \circ a &= c \circ a \\
    \end{align*}
    \todo
\end{snippetproof}

\begin{snippettheorem}{left-cancellation-law}{Left cancellation law}
    Let \((G, \circ)\) be a \group
    \[
        \forall a,b,c \in G, a \circ b = a \circ c \implies b = c
    \]
\end{snippettheorem}

\begin{snippetproof}{left-cancellation-law-proof}{right-cancellation-law}{Left cancellation law}
    Let \(a,b,c \in G\).
    \begin{align*}
        a \circ b &= a \circ c \\
        a \circ b \circ a^{-1} &= a \circ c \circ a^{-1} \\
        a^{-1} \circ (a \circ b) &= a^{-1} \circ (a \circ c) \\
        1 \circ b &= 1 \circ c \\
        b &= c
    \end{align*}
\end{snippetproof}

\begin{snippetproposition}{smallest-equation-in-group}{Smallest equation in a group}
    Let \((G, \circ)\) be a \group and let \(a,b \in G\).
    Then, the equation
    \[
        a \circ x = b
    \]
    has a unique solution in \(G\) given by \(a^{-1} \circ b\).
    Likewise, the equation
    \[
        x \circ a = b
    \]
    has a unique solution in \(G\) given by \(b \circ a^{-1}\).
\end{snippetproposition}

\begin{snippetproof}{smallest-equation-in-group-proof}{smallest-equation-in-group}{Smallest equation in a group}
    We directly show that \(a(a^{-1} b) = (aa^{-1})b = b\)
    is a solution. We need to show that it is unique.
    Let \(\overline{x}\) and \(\hat{x}\) be two elements such that
    \(a \circ \overline{x} = b\) and \(a \circ \hat{x} = b\).
    This means that \(x\overline{x} = a\hat{x}\). However, by the cancellation law
    we get that \(\overline{x} = \hat{x}\).
    The other proof is analogous.
\end{snippetproof}

\begin{snippetcorollary}{smallest-equation-in-group-corollary}{Smallest equation in a group}
    Let \((G, \circ)\) be a \group and fix \(x\in G\).
    Then, the multiplication \(x \circ a = b\) for \(x\in G\) will give every \(b \in G\)
    exactly once.
\end{snippetcorollary}

\subsection{Inverse of Product}

\begin{snippettheorem}{inverse-of-product}{Inverse of Product}
    Consider a \group \((G, \circ)\). For any two elements \(a,b \in G\)
    \[{(a \circ b)}^{-1} = b^{-1} \circ a^{-1}\]
\end{snippettheorem}

\begin{snippetproof}{inverse-of-product-proof}{inverse-of-product}{Inverse of Product}
    We start by noticing that by associativity we have
    \begin{align*}
        (a \circ b) \circ (b^{-1} \circ a^{-1}) &= a \circ (b \circ b^-1) \circ a^{-1} \\
        &= a \circ e \circ a^{-1} \\
        &= a \circ a^{-1} \\
        &= e
    \end{align*}
    This implies that \((a \circ b)\) is the inverse of \((b^{-1} \circ a^{-1})\).
    Since \((a\circ b) \circ {(a \circ b)}^{-1} =e\) we have
    \begin{align*}
        (a \circ b) \circ (b^{-1} \circ a^{-1}) = e = (a\circ b) \circ {(a \circ b)}^{-1}
    \end{align*}
    We can clearly see that \((b^{-1} \circ a^{-1}) = {(a \circ b)}^{-1}\).
\end{snippetproof}

\begin{snippettheorem}{generalized-inverse-of-product}{Generalized Inverse of Product}
    Consider a \group \((G, \circ)\). For any elements \(a_1,a_2,\cdots,a_n \in G\)
    \[
        (a_1 \circ a_2 \circ \dots a_n)^{-1}
        = a_n^{-1} \circ \dots \circ a_2^{-1} \circ a_1^{-1}
    \]
\end{snippettheorem}

\end{document}
