\documentclass[preview]{standalone}

\usepackage{amsmath}
\usepackage{amssymb}
\usepackage{stellar}
\usepackage{definitions}

\begin{document}

\id{chinese-remainder-theorem-exercises}
\genpage

\section{Exercises}

\begin{snippetexercise}{chinese-remainder-theorem-ex-1}{Chinese remainder theorem}
    Solve
    \[
        \begin{cases}
            x \equiv 3 \pmod{12} \\
            x \equiv 4 \pmod{5} \\
            x \equiv 7 \pmod{77} 
        \end{cases}
    \]
\end{snippetexercise}

\begin{snippetsolution}{chinese-remainder-theorem-ex-1-sol}{Chinese remainder theorem}
    We note that \(12, 5, 57\) are \coprime as pairs.
    We can thus use the chinese remainder theorem. %TODOURGENT link
    The system is equivalent to \(x \equiv a \pmod{12\cdot5\cdot77}\).
    Consider the first two congruences.
    \[
        \begin{cases}
            x \equiv 3 \pmod{12} \\
            x \equiv 4 \pmod{5}
        \end{cases}
    \]
    We need to find \(x\) such that \(x = 3+12h = 4+5k\) for some \(h,k \in\integers\).
    We thus have the diophantine equation \(12h - 5k = 1\).
    A Bezout's identity is for example \(12 \cdot (3) + 5(-7) = 1\),
    which yields a particular solution \(h=3\) and \(k=7\)
    which gives \(x=3+12\cdot3 = 39\).
    Thus, the first two congruences are equivalent to \(x\equiv 39 \pmod{12\cdot 5}\),
    reducing the system to
    \[
        \begin{cases}
            x \equiv 39 \pmod{12\cdot 5} \\
            x \equiv 7 \pmod{77}
        \end{cases}
    \]
    We need to find \(x=39 + 60h = 7+77k\) for some \(h,k\in\integers\).
    Consider the diophantine equation
    \(77k-60h = 32\). To find the Bezout's identity we use the Euclidean algorithm
    and we get \(60\cdot9 - 77\cdot 7\).
    We thus have \(77(-7 \cdot 32) + 60(9 \cdot 32) = 1\)
    from which \(h=-7\cdot 32\) and \(k=-9(32)\) which is a particular solution
    \(x = 39 + 60(-7\cdot 32)\).
    The system is thus equivalent to
    \[
        x \equiv 39 + 60(-7\cdot 32) \pmod{77 \cdot 12 \cdot 5}
    \]
\end{snippetsolution}

\begin{snippetexercise}{chinese-remainder-theorem-ex-2}{Chinese remainder theorem}
    Find the last two digits in base \(10\)
    of \(6^{6666}\).
\end{snippetexercise}

\begin{snippetsolution}{chinese-remainder-theorem-ex-2-sol}{Chinese remainder theorem}
    Since \(6\) is not coprime with \(10\), we cannot use \eulertheorem directly.
    We want to find a value \(0 \leq x \leq 99\) which is congruent to
    \(6^{6666} \pmod{100}\). The congruence is
    \[
        x \equiv 6^{6666} \pmod{100}
    \]
    We use the chiense remainder theorem in reverse.
    \[
        \begin{cases}
            x \equiv 6^{6666} \pmod{4} \\
            x \equiv 6^{6666} \pmod{25}
        \end{cases}
    \]
    Now, \(6\) is a multiple of \(2\). Its exponents of at least \(2\)
    are multiples of \(4\). Thus, \(6^{6666} \equiv 0 \pmod{4}\).
    Furthermore, \(6\) and \(25\) are \coprime. By \eulertheorem,
    \[
        6^{\eulertotient(25)} \equiv 1 \pmod{25}
    \]
    Now, \(\eulertotient(25) = 5^2 - 5 = 20\) and thus
    \[
        6^{20} \equiv 1 \pmod{25}
    \]
    However, \(6^{6666} \equiv 6 \pmod{20}\), and thus
    \(6^{6666} \equiv 6^6 \pmod{25}\).
    The system is reduced to
    \[
        \begin{cases}
            x \equiv 0 \pmod{4} \\
            x \equiv 6^6 \pmod{25}
        \end{cases}
    \]
    We compute \(6^6 \pmod{25}\): 
    \begin{align*}
        6 &\equiv 6 \pmod{25} \\
        6^2 &= 36 \equiv 11 \pmod{25} \\
        6^3 &\equiv 11\cdot 66 \equiv 11 \pmod{25} \\
        6^6 &\equiv 16\cdot 16 \pmod{25} \\ equiv 6 \pmod{25}
    \end{align*}
    The system is then reduced to
    \[
        \begin{cases}
            x \equiv 0 \pmod{4} \\
            x \equiv 6 \pmod{25}
        \end{cases}
    \]
    %  56
\end{snippetsolution}

\end{document}