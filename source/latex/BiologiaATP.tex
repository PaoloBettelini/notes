\documentclass[preview]{standalone}

\usepackage{amsmath}
\usepackage{amssymb}
\usepackage{stellar}

\hypersetup{
    colorlinks=true,
    linkcolor=black,
    urlcolor=blue,
    pdftitle={Stellar},
    pdfpagemode=FullScreen,
}

\begin{document}

\title{Stellar}
\id{biologia-atp}
\genpage

\section{ATP}

\includesnpt{atp-definition}

\begin{snippet}{atp-composizione}
    L'ATP è costituita dall'adenina legata a una molecola di ribosio, alla quale è
    attaccato un gruppo di tre fosfati. I legami tra i gruppi fosfato sono instabili e vengono
    facilmente idrolizzati, cioè scissi mediante aggiunta di acqua. Quando il legame tra il
    secondo e il terzo gruppo fosfato si spezza, l'ATP diventa ADP (adenosina difosfato), simile
    all'ATP ma con soli due gruppi fosfato; in questo processo viene liberata energia sufficiente
    per attivare gran parte delle reazioni cellulari.
\end{snippet}

\includesnpt[width=65\%|src=/snippet/static/atp-adp-composizione.png]{centered-img}

\section{Il ciclo dell'ATP}

\begin{snippet}{ciclo-atp-expl}
    Nel ciclo dell'ATP, l'energia liberata dalle reazioni esoergoniche viene utilizzata per
    rigenerare ATP a partire da ADP e da un gruppo fosfato, tramite una reazione di
    condensazione.
\end{snippet}

\includesnpt[width=65\%|src=/snippet/static/ciclo-atp.png]{centered-img}

\end{document}