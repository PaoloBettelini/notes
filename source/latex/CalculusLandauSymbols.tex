\documentclass[preview]{standalone}

\usepackage{amsmath}
\usepackage{amssymb}
\usepackage{stellar}
\usepackage{definitions}
\usepackage{mathtools}

\begin{document}

\id{landau-symbols}
\genpage

\section{Definitions}

% https://en.wikipedia.org/wiki/Big_O_notation#Little-o_notation
% could use the formal definition and then limit definition as a theorem

\begin{snippetdefinition}{big-o-definition}{Big-O Notation}
    Let \(f(x)\) and \(g(x)\) be \function[functions].
    The \textit{Big-O notation} is defined as
    \[
        f(x) = \mathcal{O}(g(x)) \iff
        \exists M>0, x_0 \suchthat \forall x > x_0, |f(x)| \leq M|g(x)|
    \]
\end{snippetdefinition}

\begin{snippet}{big-o-expl}
    This means that \(f(x)\) grows at most as fast as \(g(x)\) for large \(x\).
\end{snippet}

\begin{snippetdefinition}{big-omega-definition}{Big-\(\Omega\) Notation}
    Let \(f(x)\) and \(g(x)\) be \function[functions].
    The \textit{Big-\(\Omega\) notation} is defined as
    \[
        f(x) = \Omega(g(x)) \iff
        \exists M, x_0 \suchthat \forall x > x_0, |f(x)| \geq M|g(x)|
    \]
\end{snippetdefinition}

\begin{snippet}{big-omega-expl}
    This means that \(f(x)\) grows at least as fast as \(g(x)\) for large \(x\).
\end{snippet}

\begin{snippetdefinition}{little-o-definition}{Little-O Notation}
    Let \(f(x)\) and \(g(x)\) be \function[functions].
    The \textit{Little-O notation} is defined as
    \[
        f(x) = o(g(x)) \iff
        \lim_{x\to\infty}\frac{f(x)}{g(x)} = 0
    \]
\end{snippetdefinition}

\begin{snippet}{little-o-expl}
    This means that \(f(x)\) grows strictly slower than \(g(x)\) for large \(x\).
    The function becomes negligible compared to \(g(x)\) for large \(x\).
\end{snippet}

\begin{snippetdefinition}{little-omega-definition}{Little-\(\omega\) Notation}
    Let \(f(x)\) and \(g(x)\) be \function[functions].
    The \textit{Little-\(\omega\) notation} is defined as
    \[
        f(x) = o(g(x)) \iff
        \lim_{x\to\infty}\frac{f(x)}{g(x)} = \infty
    \]
\end{snippetdefinition}

\begin{snippet}{little-omega-expl}
    This means that \(f(x)\) grows strictly faster than \(g(x)\) for large \(x\).
    The function eventually exceeds any constant multiple of  \(g(x)\) for large \(x\).
\end{snippet}

\begin{snippetdefinition}{big-theta-definition}{Big-\(\Theta\) Notation}
    Let \(f(x)\) and \(g(x)\) be \function[functions].
    The \textit{Big-\(\Theta\) notation} is defined as
    \[
        f(x) = \Theta(g(x)) \iff
        \exists M_1>0, M_2>0, x_0 \suchthat \forall x > x_0, M_1g(x) \leq f(x) \leq M_2g(x)
    \]
\end{snippetdefinition}

\begin{snippet}{big-theta-expl}
    This means that \(f(x)\) and \(g(x)\) grow at the same rate asymptotically.
\end{snippet}

\begin{snippetdefinition}{asymptotic-equivalence-definition}{Asymptotic equivalence}
    Let \(f(x)\) and \(g(x)\) be \function[functions].
    The \textit{asymptotic equivalence} is a \binrelation \(\sim\)
    defined as
    \[
        f(x) \sim g(x) \iff \lim_{x\to\infty}\frac{f(x)}{g(x)} = 1
    \]
\end{snippetdefinition}

\begin{snippet}{asymptotic-equivalence-expl}
    This means that \(f(x)\) and \(g(x)\) grow at the precisely the asme rate, with their ratio
    approching \(1\) as \(x\to\infty\).
\end{snippet}

\section{Basic results}

\begin{snippetproposition}{big-o-big-omega-relation}{Big-O and Big-\(\Omega\) relation}
    Let \(f(x)\) and \(g(x)\) be \function[functions]. Then,
    \[ f(x) = \bigO(g(x)) \iff g(x) = \bigomega(f(x)) \]
\end{snippetproposition}

\begin{snippetproposition}{little-o-little-omega-relation}{Little-O and Little-\(\omega\) relation}
    Let \(f(x)\) and \(g(x)\) be \function[functions]. Then,
    \[ f(x) = \littleO(g(x)) \iff g(x) = \littleomega(f(x)) \]
\end{snippetproposition}

\begin{snippetproposition}{big-o-big-omega-big-theta-relation}{Big-\(\Theta\)}
    Let \(f(x)\) and \(g(x)\) be \function[functions]. Then,
    \[
        f(x) = \bigtheta(g(x)) \iff
        f(x) = \bigO(g(x))
        \land
        f(x) = \bigomega(g(x))
    \]
\end{snippetproposition}

\subsection{Properties of the asymptotic equivalence}

\begin{snippetproposition}{asymptotic-equivalence-big-theta-relation}{Asymptotic equivalence and Big-\(\Theta\) relation}
    Let \(f(x)\) and \(g(x)\) be \function[functions]. Then,
    \[ f(x) \asymptotic g(x) \implies f(x) = \bigtheta(g(x)) \]
\end{snippetproposition}

\begin{snippetproposition}{asymptotic-equivalence-limit}{Asymptotic equivalence limit}
    Let \(f(x)\) and \(g(x)\) be \function[functions] such that \(f(x) \asymptotic g(x)\). Then,
    \[
        \lim_{x\to\infty} f(x) = \xi \in \realnumbers
        \iff
        \lim_{x\to\infty} g(x) = \xi\in \realnumbers
    \]
\end{snippetproposition}

\begin{snippetproposition}{asymptotic-equivalence-third-function}{Asymptotic third function}
    Let \(f(x)\) and \(g(x)\) be \function[functions]. Then,
    \[
        f(x) \asymptotic g(x) \iff
        \exists h(x) \xrightarrow{x \to\infty} 0 \suchthat f(x) = h(x)g(x)
    \]
\end{snippetproposition}

\begin{snippetproposition}{asymptotic-equivalence-third-function-2}{Asymptotic third function}
    Let \(f(x)\) and \(g(x)\) be \function[functions]. Then,
    \[
        f(x) \asymptotic g(x) \iff
        f(x) = (1+h(x))g(x)
    \]
    for some \function \(h(x)\) such that
    \[
        \lim_{x\to\infty} h(x) = 0
    \]
\end{snippetproposition}

\begin{snippetproposition}{asymptotic-equivalence-operations}{Asymptotic equivalence operations}
    Let \(a(x)\), \(b(x)\), \(c(x)\) and \(d(x)\) be \function[functions]
    such that \(a(x) \asymptotic b(x)\) and \(c(x) \asymptotic d(x)\).
    Then,
    \begin{enumerate}
        \item \( \forall \alpha \in\realnumbers, f^\alpha(x) \asymptotic g^\alpha(x) \) when it exists;
        \item \(a(x)c(x) \asymptotic b(x)d(x)\);
        \item \[ \frac{a(x)}{c(x)} \asymptotic \frac{b(x)}{d(x)} \]
        when it exists.
    \end{enumerate}
\end{snippetproposition}

\plain{The same does not work for sum, logarithm and the exponential function.}

\subsection{Properties of Little-O}

\begin{snippetproposition}{little-o-big-o-relation}{Big-O and Little-O}
    Let \(f(x)\) and \(g(x)\) be \function[functions]. Then,
    \[
        f(x) = \littleO(g(x)) \implies f(x) = \bigO(g(x))
    \]
\end{snippetproposition}

\begin{snippetproposition}{little-o-product}{Little-O product}
    Let \(f(x)\), \(g(x)\), \(h(x)\) and \(k(x)\) be \function[functions]
    such that \(f(x) = \littleO(g(x))\) and \(h(x)=\littleO(k(x))\). Then,
    \[
        f(x)h(x) = o(g(x)k(x))
    \]
\end{snippetproposition}

\begin{snippetproof}{little-o-product-proof}{little-o-product}{Little-O product}
    We have
    \[
        \left|\frac{f(x)h(x)}{g(x)k(x)}\right|
        = \left|\frac{f(x)}{g(x)}\right|
        \cdot \left|\frac{h(x)}{k(x)}\right|
    \]
    which both tend to \(0\) as \(x\to\infty\).
\end{snippetproof}

\end{document}