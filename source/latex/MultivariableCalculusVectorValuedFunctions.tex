\documentclass[preview]{standalone}

\usepackage{amsmath}
\usepackage{amssymb}
\usepackage{stellar}
\usepackage{bettelini}

\hypersetup{
    colorlinks=true,
    linkcolor=black,
    urlcolor=blue,
    pdftitle={Functions},
    pdfpagemode=FullScreen,
}

\begin{document}

\title{Multivariable calculus}
\id{multivariablecalculus-vector-valued-functions}
\genpage

\begin{snippetdefinition}{parametric-curve-definition}{Parametric Curve}
    Let \(n \geq 1\) be an integer.
    Given a non-empty interval \(I \subseteq \mathbb{R}\)
    (\textit{parametric interval}), a vector-valued function \(\vec{f}: I\to{\mathbb{R}}^n\)
    is called a \textit{parametric curve}.
    \[
        \vec{f}(t) = \begin{pmatrix}
                f_1(t) \\
                f_2(t) \\
                \cdots \\
                f_n(t)
        \end{pmatrix}, \quad t \in I
    \]
\end{snippetdefinition}

\begin{snippetdefinition}{parametric-curve-continuity-definition}{Parametric Curve Continuity}
    A parametric curve \(\vec{f}: I \to {\mathbb{R}}^n\)
    is \textit{continuous} at \(t_0 \in I\)
    if
    \[
        \forall \epsilon > 0, 
        \exists \, \delta > 0 \,|\, \delta > |t-t_0|
        \implies ||\vec{f}(t) - \vec{f}(t_0)|| \leq \epsilon
    \]
    for \(t \in I\).
\end{snippetdefinition}

\begin{snippetdefinition}{parametric-path-definition}{Path}
    A \textit{path} is a parametric curve
    that is continuous.
\end{snippetdefinition}

\begin{snippetproposition}{parametric-curve-continuity-components}{Continuous components of curve}
    Let \(\vec{f}: I \to {\mathbb{R}}^n\) be a parametric curve
    \[
        \vec{f}(t) = \begin{pmatrix}
                f_1(t) \\
                f_2(t) \\
                \cdots \\
                f_n(t)
        \end{pmatrix}
    \]
    Then \(\vec{f}\) is continuous at a point \(t_0\) if and only if
    \(f_1(t), f_2(t), \cdots, f_n(t)\) are continuous at \(t_0\).
\end{snippetproposition}

\begin{snippetdefinition}{curve-arc-length-definition}{Curve arc length}
    Let \(\vec{f}:I\to{\mathbb{R}}^n\) be a differentiable curve
    and \(a < b \in I\).
    The \textit{arc length} of the curve \(\vec{f}:[a,b] \to {\mathbb{R}}^n\)
    is defined as
    \[
        \integral[a][b][\left\|\mathbf{f}^{\prime}(t)\right\|_2][t]
    \]
\end{snippetdefinition}

\section{TODO: move}

\begin{snippettheorem}{multivariable-chainrule}{Multivariable chainrule}
    Let \(x=g(t)\) and \(y=h(t)\) be differentiable functions of \(t\)
    and \(z=f(x,y)\) be a differentiable function of \(x\) and \(y\).
    Then \(z=f(x(t), y(t))\) is a differentiable function of \(t\) and
    \[
        \frac{dz}{dt} =
        \frac{\partial z}{\partial x} \cdot \frac{dx}{dt} +
        \frac{\partial z}{\partial y} \cdot \frac{dy}{dt}
    \]
    where the ordinary derivatives are evaluated at \(t\) and the partial
    derivatives are evaluated at \((x,y)\).
\end{snippettheorem}

\begin{snippetdefinition}{gradient-definition}{Gradient}
    Let \(E \subseteq {\mathbb{R}}^n\) be an open set and
    let \(f\to E \to \mathbb{R}\) be a function where
    all partial derivatives \(\frac{\partial f}{\partial x_1}\),
    \(\cdots\), \(\frac{\partial f}{\partial x_n}\) of \(f\)
    at a point \(a\in E\) exist.
    The \textit{gradient} of \(f\) at the point \(a\)
    is defined as
    \[
        \nabla f(a) \triangleq
        \left(\begin{array}{c}
        \frac{\partial f}{\partial x_1}(\mathbf{a}) \\
        \vdots \\
        \frac{\partial f}{\partial x_n}(\mathbf{a})
        \end{array}\right)
    \]
\end{snippetdefinition}

\end{document}