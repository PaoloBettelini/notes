\documentclass[preview]{standalone}

\usepackage{amsmath}
\usepackage{amssymb}
\usepackage{stellar}
\usepackage{bettelini}

\hypersetup{
    colorlinks=true,
    linkcolor=black,
    urlcolor=blue,
    pdftitle={Stellar},
    pdfpagemode=FullScreen,
}

\begin{document}

\title{Stellar}
\id{biologia-sintesi}
\genpage

\section{Fotosintesi}

\begin{snippetdefinition}{fotosintesi-definition}{Fotosintesi}
    La \textit{fotosintesi clorofilliana} è un processo chimico per mezzo del quale
    le piante verdi e altri organismi producono sostanze organiche -
    principalmente carboidrati - a partire dal primo reagente,
    l'anidride carbonica atmosferica e l'acqua metabolica,
    in presenza di luce solare, rientrando tra i processi di anabolismo dei carboidrati,
    del tutto opposta ai processi inversi di catabolismo.
\end{snippetdefinition}

\begin{snippet}{fotosintesi-expl}
    La fotosintesi è il processo metabolico che consente di catturare l'energia luminosa del Sole
    e utilizzarla per convertire l'acqua (H\({}_2\)O) e il diossido di carbonio (CO\({}_2\)) in zuccheri e
    ossigeno (O\({}_2\)). La fotosintesi si svolge all'interno di organuli specializzati, i cloroplasti. La
    fotosintesi è un processo biochimico complesso che avviene in due fasi, ciascuna costituita
    da diversi passaggi. Il termine \quotes{fotosintesi} fa riferimento a entrambe le fasi del processo:
    una prima fase che richiede la presenza di luce, detta fase luminosa, e una seconda fase
    che non necessita di luce, detta fase oscura o ciclo di Calvin, nella quale vengono sintetizzati
    gli zuccheri.
\end{snippet}

\end{document}