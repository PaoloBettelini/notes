\documentclass[preview]{standalone}

\usepackage{amsmath}
\usepackage{amssymb}
\usepackage{stellar}
\usepackage{definitions}

\begin{document}

\id{integers-euler-totient-function-exercises}
\genpage

\section{Exercises}

\begin{snippetexercise}{euler-totient-function-ex-1}{}
    Compute the remainder of \(33^{2541} + 18^{7211}\) by \(7\).
\end{snippetexercise}

\begin{snippetsolution}{euler-totient-function-ex-1-sol}{}
    We must find a number \(r\) with \(0 \leq r < 7\)
    such that \(33^{2541} + 18^{7211} \equiv r \pmod{7}\).
    We note that \(33 \equiv 5 \pmod{7}\) and \(18 \equiv 4 \pmod{7}\),
    so \(33^{2541} + 18^{7211} \equiv 5^{2541} + 4^{7211 \pmod{7}}\).
    Now, \(5\) and \(7\) are \coprime and \(\eulertotient(7) = 6\).
    Thus, \(5^6 \equiv 1 \pmod{7}\). It follows that \(5^{6k} = {(5^6)}^k \equiv 1 \pmod{7}\)
    for every \(k\in\naturalnumbers\). Since \(2541 \equiv 3 \pmod{6}\),
    we have that \(5^{2541} \equiv 5^3 \pmod{6}\).
    Likewise, \(4\) and \(7\) are \coprime and thus \(4^6 \equiv 1 \pmod{1}\).
    Since \(7211 \equiv 5 \pmod{6}\) it follows that \(4^{7211} \equiv 4^5 \pmod{7}\).
    Finally, \(33^{2541} + 18^{7211} \equiv 5^3 + 4^5 \pmod{7}\).
    We now compute \(5^3 \pmod{7}\):
    \begin{enumerate}
        \item \(5\equiv 5 \pmod{7}\);
        \item \(5^2\equiv 4 \pmod{7}\);
        \item \(5^3\equiv 4\cdot 5 \pmod{7} \equiv 7 \pmod{7}\);
    \end{enumerate}
    We now compute \(4^5 \pmod{7}\):
    \begin{enumerate}
        \item \(4\equiv 4 \pmod{7}\);
        \item \(4^2\equiv 2 \pmod{7}\);
        \item \(4^4 = 4^2 \cdot 4^2 \equiv 2\cdot 2 \pmod{7} \equiv 4 \pmod{7}\);
        \item \(4^5 = 4^4\cdot 4 \pmod{7} \equiv 4 \cdot 4 \pmod{7} \equiv 2 \pmod{7}\);
    \end{enumerate}
    We now have that \(33^{2541} + 18^{7211} \equiv 5^3 + 4^5 \pmod{7} \equiv 6+2 \pmod{7} \equiv 1 \pmod{7} \).
    The remainder is thus \(1\).
\end{snippetsolution}

\begin{snippetexercise}{euler-totient-function-ex-2}{}
    Compute the last digit base 10 of \(57^{7777}\).
\end{snippetexercise}

\begin{snippetsolution}{euler-totient-function-ex-2-sol}{}
    We need to compute \(57^{7777} \pmod{10}\).
    We have that \(57 \equiv 7 \pmod{10}\).
    Thus, \(57^{7777} \equiv 7^{7777} \pmod{10}\).
    Since \(7\) and \(10\) are \coprime, we have that \(7^{\eulertotient(10)} \equiv 1 \pmod{10}\). % TODOURGENT link euler's theroem
    Now, \(\eulertotient(10) = 4\) and thus \(7^4 \equiv 1 \pmod{10}\)
    and \(7777 \equiv 1 \pmod{4}\). It follows that \(7^{7777} \equiv 7^1 \pmod{10}\).
    The last digit is thus \(7\).
\end{snippetsolution}

\end{document}