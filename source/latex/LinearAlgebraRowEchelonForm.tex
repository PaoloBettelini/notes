\documentclass[preview]{standalone}

\usepackage{amsmath}
\usepackage{amssymb}
\usepackage{stellar}

\hypersetup{
    colorlinks=true,
    linkcolor=black,
    urlcolor=blue,
    pdftitle={Stellar},
    pdfpagemode=FullScreen,
}

\begin{document}

\title{Stellar}
\id{row-echelon-form}
\genpage

\section{Row Echelon Forms}

\begin{snippetdefinition}{row-echelon-form-definition}{Row echelon form}
    A \snippetref[matrix-definition][matrix] is said to be in \textit{row echelon form}
    if it fulfils the following properties:
    \begin{enumerate}
        \item if a row does not consist entirely of zeroes, then the first nonzero number in the row is a 1;
        \item all zero rows are at the bottom of the matrix;
        \item in any two successive rows that do not consist entirely of zeroes, the leading 1 in the lower
            row occurs farther to the right than the leading 1 in the higher row.
    \end{enumerate}
\end{snippetdefinition}

\begin{snippetproposition}{row-echelon-form-non-uniqueness}{Non uniqueness of row echelon form}
    \snippetref[row-echelon-form-definition][Row echelon form] is not unique.
\end{snippetproposition}

\begin{snippetdefinition}{reduced-row-echelon-form-definition}{Reduced row echelon form}
    A \snippetref[matrix-definition][matrix] is said to be in \textit{reduced row echelon form}
    if it is in \snippetref[row-echelon-form-definition][row echelon form] and
    each column containing a leading 1 has zeroes in all its other entries.
\end{snippetdefinition}

\begin{snippetproposition}{reduced-row-echelon-form-uniqueness}{Uniqueness of reduced row echelon form}
    \snippetref[reduced-row-echelon-form-definition][Reduced row echelon form] is unique.
\end{snippetproposition}

\end{document}