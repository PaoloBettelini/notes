\documentclass[a4paper]{article}

\usepackage{amsmath}
\usepackage{amssymb}
\usepackage{parskip}
\usepackage{fullpage}
\usepackage{hyperref}

\hypersetup{
    colorlinks=true,
    linkcolor=black,
    urlcolor=blue,
    pdftitle={FluidStatics},
    pdfpagemode=FullScreen,
}

\title{Fluid Statics}
\author{Paolo Bettelini}
\date{}

\begin{document}

\maketitle
\tableofcontents
\pagebreak

\section{Hydrostatic Pressure}

Pressure is defined as 

\[
    P = \frac{F}{A}
\]

Where \(F\) is a force acting on an area \(A\).

The hydrostatic pressure is given by

\begin{align*}
    P &= \frac{F_g}{A} = \frac{A \cdot h \cdot \rho \cdot g}{A} \\
    &= \rho \cdot h \cdot g
\end{align*}

This means that the pressure of a fluid doesn't depend
on the area.

\section{Stevin's law}

Stevin's law states that the pressure at any point
in a resting fluid is only proportional to the depth of that point.

\[
    P = P_0 + \rho \cdot g \cdot h
\]

where \(P_0\) is the external pressure (pressure at the surface).

\section{Pascal's Principle}

Pascal's Principle states that a pressure change in one part
is transmitted without loss to every portion of the fluid and walls of the container.

\section{Archimede's Law}

\subsection{Definition}

Archimedes' principle states that
«every body partially or completely immersed in a fluid receives a
vertical thrust from the bottom upwards,
equal in intensity to the weight of the displaced fluid».

The intensity of the force is given by

\[
    F = \rho_\text{fluid} \cdot g \cdot V_\text{body}
\]

The force is the same no matter the shape of the body.

\subsection{Accelleration}

An object dropped in a fluid will sink or float up
depending on its density and the density of the fluid.
The body will move with the following accelleration

\begin{align*}
    F &= F_g - F_\text{Archimede} \\
    &= gm - \rho_\text{fluid}V_\text{body}g \\
    &= gm - \rho_\text{fluid}\frac{m}{\rho_\text{body}}g \\
    ma &= m \left( g- \frac{\rho_\text{fluid}g}{\rho_\text{body}} \right) \\
    a &= \frac{g\rho_\text{body}}{\rho_\text{body}} - \frac{g\rho_\text{fluid}}{\rho_\text{body}} \\
    &= \frac{\rho_\text{body} - \rho_\text{fluid}}{\rho_\text{body}}g
\end{align*}

where \(m\) is the mass of the body.

A submerged body will

\begin{align*}
	\begin{cases}
        \text{if } \rho_\text{body} < \rho_\text{fluid},\quad \text{float up} \\
        \text{if } \rho_\text{body} = \rho_\text{fluid},\quad \text{float} \\
        \text{if } \rho_\text{body} > \rho_\text{fluid},\quad \text{sink}
	\end{cases}
\end{align*}

\pagebreak

\end{document}
