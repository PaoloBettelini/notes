\documentclass[a4paper]{article}

\usepackage{amsmath}
\usepackage{amssymb}
\usepackage{parskip}
\usepackage{fullpage}
\usepackage{hyperref}

\hypersetup{
    colorlinks=true,
    linkcolor=black,
    urlcolor=blue,
    pdftitle={Waves},
    pdfpagemode=FullScreen,
}

\title{Waves}

\author{Paolo Bettelini}
\date{}

\begin{document}

\maketitle

\tableofcontents

\pagebreak

\section{Wave}

A wave is a propagation of a disturbance (energy) which oscillates
repeatedly.

\subsection{Waves in different dimensions}

Waves can expand in different dimensions. Here are examples of each dimension
\begin{enumerate}
    \item 1 dimension: an oscillating rope
    \item 2 dimensions: surface of water oscillating
    \item 3 dimensions: sound propagating through the air
\end{enumerate}

\subsection{Direction of the wave}

A wave is \textit{transverse} when its oscillations are perpendicular
to the direction of the wave propagation (e.g. slinky up and down).

A wave is \textit{longitudinal} when its oscillations are parallel
to the direction of the wave propagation (e.g. slinky left and right).

\subsection{Types of waves}

There are different types of waves, namely,
\textit{mechanical} waves, \textit{electromagnetic} wave
and \textit{gravitational} waves.

Electromagnetic and gravitational waves are always longitudinal.

\section{Mechanical waves}

\subsection{Wave length}

The \textit{wavelength} \(\lambda\) of a wave describes how long the wave is.

\subsection{Period}

The \textit{period} \(T\) of a wave is the time it takes to complete
a full oscillation.

\subsection{Frequency}

The \textit{frequency} \(f\) of a wave represents how many oscillation
completed in one unit of time (seconds).
\[
    f = \frac{1}{T}
\]

\subsection{Phase velocity}

The \textit{phase velocity} \(v\) is the rate at which
the wave propagates.
\begin{align*}
    v &= \frac{\lambda}{T} \\
    &= f\lambda
\end{align*}

\subsection{Amplitude}

The \textit{amplitude} \(A\) of a mechanical wave is the
measure of the maximum distance a point can reach
from its equilibrium position.

\pagebreak

\section{Harmonic waves}

An armonic wave is a periodic wave where
the points of the medium where it moves oscillate.

\[
    s(t;x) = A \sin
    \left(
        \omega t - \frac{2\pi}{\lambda}x
    \right)
    \text{ where } \omega = \frac{2\pi}{T}
\]

\end{document}