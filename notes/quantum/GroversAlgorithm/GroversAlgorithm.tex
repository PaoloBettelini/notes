\documentclass{article}
\usepackage{amsmath}
\usepackage{parskip}
\usepackage{fullpage}

\title{Grover's Algorithm}
\author{Paolo Bettelini}
\date{}

\begin{document}

\maketitle
\tableofcontents
\pagebreak

\section{Introduction}

Given a list of \(N\) element, an item \(\omega\) with a unique properties, on average we will need to check \(\frac{N}{2}\) elements before finding \(\omega\).
This classical computation is \(O(n)\) in time complexity.

Grover's algorithm reduces this time complexity to \(O(\sqrt{n})\), meaning that if we have a list of size \(100\) it will take \(10\) steps to find \(\omega\) instead of \(50\) on average.

This quantum algorithm uses amplitude amplification of a superposition to have a near perfect probability of finding \(\omega\).

\pagebreak

\section{The oracle}

The oracle is a matrix that will be applied to a superposition of qubits in the algorithm, since quantum computers can apply matrices to qubits in parallel.

The list of elements is comprised of all the possible computational basis states the qubits can be in.\\
For example \((|0\rangle\ \Rightarrow |255\rangle)\) for 8 qubits.

The oracle \(U_\omega\) negates the phase of the state if it is not \(\omega\).

\begin{align*}
    U_\omega|x\rangle=
    \begin{cases}
        -|x\rangle,\quad \text{if } x=\omega \\
        +|x\rangle,\quad \text{if } x\neq\omega
    \end{cases}
\end{align*}

We define a function \(f(x)\) such that the output is \(1\) if (\(x=\omega\)), \(0\) otherwise.

\begin{align*}
    f(x)=
    \begin{cases}
        1,\quad \text{if } x=\omega \\
        0,\quad \text{if } x\neq\omega
    \end{cases}
\end{align*}

The oracle applied to a state is given by \(U_\omega|x\rangle =(-1)^{f(x)}|x\rangle\)

The oracle can be represented with a diagonal matrix, where only the position of \(\omega\) has a negative phase.
The diagonal is made of \(1s\) except for the entry of \(\omega\), \(-1\).

\begin{align*}
    U_\omega=
    \begin{bmatrix}
        (-1)^{f(0)} & 0 & \cdots & 0 \\
        0 & (-1)^{f(1)} & \cdots & 0 \\
        \vdots & 0 & \ddots & \vdots \\
        0 & 0 & \cdots & (-1)^{f(2^n-1)}
    \end{bmatrix}
\end{align*}

\section{The algorithm}

We start by creating a uniform superposition of all the item \(|s\rangle\)

\[
    |s\rangle=\frac{1}{\sqrt{N}}\sum_{n}|\Psi_n\rangle
\]

Collapsing this superposition has a chance of \(\frac{1}{N}\) of finding \(w\), which is basically a random guess.

\end{document}