\documentclass{article}
\usepackage[utf8]{inputenc}
\usepackage{amsmath}
\usepackage{amssymb}
\usepackage{parskip}
\usepackage{fullpage}

\title{Fundamentals of Quantum Physics }
\author{Paolo Bettelini}
\date{}

% 0.4.1 (?)

\newcommand{\quotes}[1]{``#1''}

\begin{document}

\maketitle
\tableofcontents
\pagebreak

\section{The Probabilistic Nature of Qubits}

A qubit is comprable to a bit in the \quotes{classical} world, but it exists on a sub-atomic level. \\
When a qubit is measured, its state will either be a \quotes{1} or a \quotes{0}. \\
The crucial aspect is that before the measurement, a qubit is in a \textit{superposition} of both states.

For example, a given qubit \(|\Psi\rangle\) can be represented as
\[
    |\Psi\rangle=\alpha |0\rangle+\beta |1\rangle
\]
which means a linear combination of the two states \(|0\rangle\) and \(|1\rangle\).

The coefficients \(\alpha\) and \(\beta\) represent the probability of the qubit collapsing into one of the two states when measured.
The probability of the qubit collapsing into \(|0\rangle\) is \(|\alpha|^2\),
while the probability of collapsing into \(|1\rangle\) is \(|\beta|^2\). \\
Since there is \(100\%\) chance of the qubits collapsing into one of the two states, \(\alpha\) and \(\beta\) must satisfy the following requirement:
\[
    |\alpha|^2+|\beta|^2=1
\]

A uniform superposition of the two states looks like this:
\[
    |\Psi\rangle=\frac{|0\rangle+|1\rangle}{\sqrt{2}}
\]
which means that we have \(50\%\) probability of the state collapsing into a \(|0\rangle\) or \(|1\rangle\)
since \({\left(\frac{1}{\sqrt{2}}\right)}^2=\frac{1}{2}\).

When the qubit is measured, the superposition is destroyed, leaving it in a \quotes{classical} binary state.

You mght have noticed that I used the absolute value in \(|\alpha|^2\) or \(|\beta|^2\), which doesn't usually make sense since squaring the value is already going to give us a positive result.
\\
This is because the coefficients \(\alpha\) and \(\beta\) can also be complex numbers. The absolute value of a complex number is defined as its distance from the origin.

\section{Types of Algorithms}

\subsection{Deterministic}

Example: the quantum circuit is run and we measure the output.

\subsection{Probabilistic}

Example: The more we run the circuit, the higher the chance of getting the right or a more precise result.

\end{document}