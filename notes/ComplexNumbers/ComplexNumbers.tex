\documentclass{article}

\usepackage{amsmath}
\usepackage{amssymb}
\usepackage{parskip}
\usepackage{fullpage}
\usepackage{hyperref}
\usepackage{tikz}

\hypersetup{
    colorlinks=true,
    linkcolor=black,
    urlcolor=blue,
    pdftitle={ComplexNumbers},
    pdfpagemode=FullScreen,
}

\title{Complex Numbers}
\author{Paolo Bettelini}
\date{}

\begin{document}

\maketitle
\tableofcontents
\pagebreak

\section{Imaginary unit}

\subsection{Definition}

The imaginary unit or imaginary number \(i\) is a solution to the quadratic
equation \(x^2=-1\) and is defined as

\[
    i^2 = -1
\]

The equation \(x^2 = -1\) has two solutions: \(i\) and \(-i\), however,
there is not any algebraic difference between these two solutions.

\subsection{Properties}

The imaginary number \(i\) has some amazing properties when it comes to exponentiation.

\[
	\begin{cases}
		i^0=+1\\
		i^1=+i\\
		i^2=-1\\
		i^3=-i\\
	\end{cases}
	\quad
	\begin{cases}
		i^4=+1\\
		i^5=+i\\
		i^6=-1\\
		i^7=-i\\
	\end{cases}
	\quad
	\cdots
\]

The multiplicative inverse of \(i\) is \(-i\).

\[
    \frac{1}{i} = \frac{1}{i} \cdot \frac{i}{i}
    = \frac{i}{i^2} = -i
\]

\pagebreak

\section{Complex Numbers}

\subsection{Definition}

Complex numbers are numbers in the form \(a + bi\),
where \(a,b\in\mathbb{R}\) and \(i\) is the imaginary unit.
\\
This set of numbers is called \(\mathbb{C}\).

Since every number \(n\in\mathbb{R}\) can be reprsented as
a complex number in the form \(n+0i\), \(\mathbb{R}\subset\mathbb{C}\).

\subsection{Complex plane}

We can represent each complex number on a plane, where the horizontal axis
represent the real numbers \(\mathbb{R}\) and the vertical axis represents
every scalar multiple of the imaginary unit \(i\).

\begin{center}
    \begin{tikzpicture}
        \begin{scope}[thick,font=\scriptsize]

            \draw [->] (-5,0) -- (5,0) node [above left]  {\(\Re(s)\)};
            \draw [->] (0,-5) -- (0,5) node [below right] {\(\Im(s)\)};

            \draw (0,0) -- (0,0)   node [above right] {\(0\)};
            \foreach \n in {-4,...,-1,1,2,...,4}{
                \draw (\n,-3pt) -- (\n,3pt) node [above] {\(\n\)};
                \draw (-3pt,\n) -- (3pt,\n) node [right] {\(\n i\)};
            }

            \draw [color=black, fill=black] (3,2) circle (0.05) node [above] {\(3+2i\)};
        \end{scope}
    \end{tikzpicture}
\end{center}

\subsection{Operations}

\subsubsection{Real part}

The real part of a complex number \(s\) is denoted by \(\text{Re}(s)\) or \(\Re(s)\).

\[
    \text{Re}(a+bi) = a
\]

\subsubsection{Imaginary part}

The imaginary part of a complex number \(s\) is denoted by \(\text{Im}(s)\) or \(\Im(s)\).

\[
    \text{Im}(a+bi) = b
\]

\subsubsection{Absolute value}

The absolute value of a complex number is its distance from the origin.

\[
    |a+bi| = \sqrt{a^2 + b^2}
\]

\subsubsection{Conjugate}

The complex conjugate of a number \(s=a+bi\) is denoted as \(s^*\) or \(\overline{s}\).
It is defined as

\[
    \overline{a+bi} = a-bi
\]

Geometrically, \(s^*\) is the reflection about the real axis in the complex plane.

We also have the following trivial properties.

\begin{align*}
    \overline{\overline{s}} &= s
    \\
    \text{Re}(\overline{s}) &= \text{Re}(s)
    \\
    \text{Im}(\overline{s}) &= -\text{Im}(s)
\end{align*}

\pagebreak

\end{document}
