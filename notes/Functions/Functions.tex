\documentclass{article}

\usepackage{amsmath}
\usepackage{amssymb}
\usepackage{parskip}
\usepackage{fullpage}
\usepackage{hyperref}

\hypersetup{
    colorlinks=true,
    linkcolor=black,
    urlcolor=blue,
    pdftitle={Functions},
    pdfpagemode=FullScreen,
}

\title{Functions}
\author{Paolo Bettelini}
\date{}

\begin{document}

\maketitle
\tableofcontents
\pagebreak

\section{Surjectivity}

A function \(f:D_f \to I_f\) is said to be \textbf{surjective} if

\[
    \forall y \in I_f, \exists x \in D_f | f(x) = y
\]

\section{Injectivity}

A function \(f:D_f \to I_f\) is said to be \textbf{injective} if

\[
    \forall x_1, x_2 \in D_f, f(x_1) = f(x_2) \Rightarrow x_1 = x_2 
\]

\section{Bijectivity}

A function is said to be \textbf{bijective} iff it's both injective
and surjective.

\section{Continuity}

A function \(f\) is continuous at a point \(c\) iff
\[
    \lim_{c_0 \to c^+} f(c_0) = \lim_{c_0 \to c^-} f(c_0) = f(c)
\]
A function \(f\) is continuous on an interval \([a;b]\) iff it is continuous at each point \(c \in [a;b]\)
\[
    \forall c \in [a;b],
    \lim_{c_0 \to c^+} f(c_0) = \lim_{c_0 \to c^-} f(c_0) = f(c)
\]

\section{Periodic functions}

A function \(f\) is periodic with a period \(T\) iff
\[
    f(x) = f(x + kT), \quad k \in \mathbb{Z}
\]

\section{Odd functions}

A function \(f\) is odd iff
\[
    f(-x) = -f(x)
\]

\section{Even functions}

A function \(f\) is even iff
\[
    f(-x) = f(x)
\]

\end{document}