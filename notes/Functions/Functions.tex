\documentclass[a4paper]{article}

\usepackage{amsmath}
\usepackage{amssymb}
\usepackage{parskip}
\usepackage{fullpage}
\usepackage{hyperref}
\usepackage{stellar}

\hypersetup{
    colorlinks=true,
    linkcolor=black,
    urlcolor=blue,
    pdftitle={Functions},
    pdfpagemode=FullScreen,
}

\title{Functions}
\author{Paolo Bettelini}
\date{}

\begin{document}

\maketitle
\tableofcontents
\pagebreak

\section{Definition}

A \textit{function} is a relation \(f\)
from a set \(A\) (domain) to a set \(B\) (codomain)
\[
    f: A \to B
\]
% TODO REDO more rigorous

\section{Properties}

\subsection{Injectivity}

\sdefinition{Injectivity}{
    A function \(f:A\to B\) is \textit{injective} if
    \[
        \forall a,b \in A, f(a) = f(b) \implies a = b
    \]
}

\subsection{Surjectivity}

\sdefinition{Surjectivity}{
    A function \(f:A\to B\) is \textit{surjectiv} if
    \[
        \forall b \in B \exists a \,|\, f(a)=b
    \]
}

\subsection{Bijectivity}

\sdefinition{Bijectivity}{
    A function \(f:A\to B\) is \textit{bijective} if
    it has a one-to-one correspondence between each element of \(A\) and  \(B\).
}

\scorollary{Bijectivity properties}{
    A function \(f:A\to B\) is bijective iff it is both injective and surjective.
}

\subsection{Invertibility}

\sdefinition{Invertibility}{
    A function \(f\) is invertible iff it is a bijection.
}

\subsection{Continuity}

\sdefinition{Continuity}{
    A function \(f\) is continuous at a point \(c\) if
    \[
        \lim_{c_0 \to c^+} f(c_0) = \lim_{c_0 \to c^-} f(c_0) = f(c)
    \]
}
A function \(f\) is continuous on an interval \([a;b]\) if it is continuous at each point \(c \in [a;b]\)
\[
    \forall c \in [a;b],
    \lim_{c_0 \to c^+} f(c_0) = \lim_{c_0 \to c^-} f(c_0) = f(c)
\]

\subsection{Periodic functions}

A function \(f\) is periodic with a period \(T\) if
\[
    f(x) = f(x + kT), \quad k \in \mathbb{Z}
\]

\subsection{Odd functions}

A function \(f\) is odd if
\[
    f(-x) = -f(x)
\]

\subsection{Even functions}

A function \(f\) is even if
\[
    f(-x) = f(x)
\]

\end{document}