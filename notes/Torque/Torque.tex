\documentclass{article}
\usepackage{amsmath}
\usepackage{amssymb}
\usepackage{parskip}
\usepackage{fullpage}
\usepackage{hyperref}

\hypersetup{
    colorlinks=true,
    linkcolor=black,
    urlcolor=blue,
    pdftitle={Torque},
    pdfpagemode=FullScreen,
}

\title{Torque}
\author{Paolo Bettelini}
\date{}

\begin{document}

\maketitle
\tableofcontents
\pagebreak

\section{Torque}

The moment of a force or twisting moment is a physical quantity
that allows to describe the twisting action of a force in relation
to the point of application of the force with respect to an axis of rotation.

The unit of measure is the \(N\cdot m\).

\section{Definition}

A rotational vector with respect to the plane formed by the vectors
\(\vec{R}\) and \(\vec{F}\) is defined as \((\vec{R}\times\vec{F})\), so \(\vec{M}\) is normal to the plane.

\[
    \vec{M}=\vec{R}\times\vec{F}
\]

Using the \(3\)-dimensional definition of the scalar product

\[
    \vec{M}=
    \begin{pmatrix}
        R_y \cdot F_z - R_z \cdot F_y \\
        R_z \cdot F_x - R_x \cdot F_z \\
        R_x \cdot F_y - R_y \cdot F_x
    \end{pmatrix}
\]

Since two vectors are always coplanar, it is possible to choose a coplanar reference system

\[
    \vec{R}=
    \begin{pmatrix}
        R_1 \\
        R_2 \\
        0
    \end{pmatrix}
    ,\quad
    \vec{F}=
    \begin{pmatrix}
        F_1 \\
        F_2 \\
        0
    \end{pmatrix}
    ,\quad
    \vec{M}=
    \begin{pmatrix}
        0 \\
        0 \\
        R_1 \cdot F_2 - R_2 \cdot F_1
    \end{pmatrix}
\]

\(\vec{M}\) can also be computed by \(\vec{R}\cdot\vec{F}\cdot\sin(\alpha)\) where \(\alpha\)
is the angle between \(\vec{R}\) and \(\vec{F}\).

Sometimes we interpret the formula \(|\vec{M}|=\vec{R}\cdot\vec{F}_\perp\) as
the twisting effect of the force component perpendicular to \(\vec{R}\), or
\(|\vec{M}|=b\cdot\vec{F}\) where \(b\) is the lever arm.

The sign convention for this measurement is

\[
    \begin{cases}
        +1, \quad \text{if the rotation is counterclockwise} \\
        -1, \quad \text{if the rotation is clockwise}
    \end{cases}
\]

\section{Equilibrium conditions }

For a system to be in equilibrium, the resulting force and twisting moments must be null

\[
    \begin{cases}
        \sum \vec{F}_j=0 \\
        \sum \vec{M}_j=0
    \end{cases}
\]

\end{document}
