\documentclass{article}

\usepackage{amsmath}
\usepackage{amssymb}
\usepackage{parskip}
\usepackage{fullpage}
\usepackage{hyperref}

\hypersetup{
    colorlinks=true,
    linkcolor=black,
    urlcolor=blue,
    pdftitle={Logic},
    pdfpagemode=FullScreen,
}

\title{Logic}
\author{Paolo Bettelini}
\date{}

\begin{document}

\maketitle
\tableofcontents
\pagebreak

\section{Boolean Algebra}

% Associativity
% Commutativity
% Distribuibity

\begin{align*}
    x \lor 0 = x \\
    x \land 0 = 0 \\
    x \lor 1 = 1 \\
    x \land 1 = x \\
    x \lor x = x \\
    x \land x = x \\
    x \land (x \lor y) = x \\
    x \lor (x \land y) = x \\
    x \land \lnot x = 0 \\
    x \lor \lnot x = 1 \\
    \lnot x \land \lnot y = \lnot (x \lor y) \\
    \lnot x \lor \lnot y = \lnot (x \land y)
\end{align*}

\pagebreak

\section{Necessity and sufficiency}

\subsection{Sufficiency}

\subsection{Necessity}

\subsection{Biconditional logical connective}

A biconditional logical connective (written as \textit{iff} or \textit{xnor})
is the relation of equivalence
between two statements \(P\) and \(Q\).
The relation \(P \iff Q\) is both a sufficient condition and
a necessary condition.
\[
    P \iff Q
    \equiv
    (P \implies Q) \land (P \impliedby Q)
\]


\pagebreak

\section{Proof theory}

\subsection{\(k\)-ary Boolean function}

A \(k\)-ary Boolean function is a mapping from \({\{T, F\}}^k \to \{T,F\}\)

\subsection{\(0\)-ary Boolean function}

The \(0\)-ary Boolean function are the \textit{verum} (\(\top\)) and \textit{falsum} (\(\bot\)) connectives.
The represent respectively the True value and the False value.

\subsection{Propositional variable}

A \textit{propositional variable} is an input boolean variable.

\subsection{Propositional formula}

A \textit{propositional formula} is a formula which has a unique truth value given all variables.

\subsection{Truth assignment}

A \textit{truth assignment} is a function which maps a set of propositional
variables \(V=\{p_1, p_2, \ldots, p_n\}\) to a boolean value
\[
    \tau: V \to \{T,F\}
\]

A formula \(A\) involving the variables \(V=\{p_1, p_2, \ldots, p_n\}\)
defines a \(k\)-ary boolean function \(f_A(x_1, x_2, \ldots, x_n)\) where \(x_n = \tau(p_n)\).

\subsection{Language}

A \textit{language} \(L\) is a set of connectives which may be used to describe
an \(L\)-formula.

A language \(L\) is \textit{complete} iff every \(k\)-ary boolean functions can be
defined by an \(L\)-formula.

\subsection{Tautology}

A propositional formula \(A\) is a \textit{tautology} \(\vDash A\) if its \(k\)-ary boolean
function \(f_A\) is always \(T\).

\subsection{Satisfiability}

A propositional formula \(A\) is \textit{satisfiable}
if \(f_A\) is \(T\) for some input.



\end{document}



https://mathweb.ucsd.edu/~sbuss/ResearchWeb/handbookI/ChapterI.pdf
https://www.youtube.com/@LogicwithBo/videos