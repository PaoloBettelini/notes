\documentclass[a4paper]{article}

\usepackage{amsmath}
\usepackage{amssymb}
\usepackage{parskip}
\usepackage{fullpage}
\usepackage{hyperref}
\usepackage{bussproofs}

\usepackage{stackengine}

\hypersetup{
    colorlinks=true,
    linkcolor=black,
    urlcolor=blue,
    pdftitle={Logic},
    pdfpagemode=FullScreen,
}

\newcommand\vdashsub[1]{\mathrel{\stackengine{.4ex}{\vdash}{\;\;\scriptscriptstyle#1}{U}{c}{F}{T}{L}}}

\title{Logic}
\author{Paolo Bettelini}
\date{}

\begin{document}

\maketitle
\tableofcontents
\pagebreak

\section{Boolean Algebra}

% Associativity
% Commutativity
% Distribuibity

\begin{align*}
    x \lor 0 = x \\
    x \land 0 = 0 \\
    x \lor 1 = 1 \\
    x \land 1 = x \\
    x \lor x = x \\
    x \land x = x \\
    x \land (x \lor y) = x \\
    x \lor (x \land y) = x \\
    x \land \lnot x = 0 \\
    x \lor \lnot x = 1 \\
    \lnot x \land \lnot y = \lnot (x \lor y) \\
    \lnot x \lor \lnot y = \lnot (x \land y)
\end{align*}

\pagebreak

\section{Logical inference}

A logical inference is a logical deduction to infer the truth
of a statement given a premise.

There are 4 forms of hypothetical syllogisms.

\subsection{Modus Ponens}

\textit{Modus Ponens} or \textit{affirming the antecedent}
is a \underline{valid} form hypothetical syllogism.

\begin{prooftree}
    \AxiomC{\(P\implies Q\)}
    \AxiomC{\(P\)}
    \BinaryInfC{\(Q\)}
\end{prooftree}
If \(P\) implies \(Q\) and \(P\) is true, then \(Q\) is also true.

\subsection{Modus Tollens}

\textit{Modus Tollens} or \textit{denying the consequent}
is a \underline{valid} form hypothetical syllogism.

\begin{prooftree}
    \AxiomC{\(P\implies Q\)}
    \AxiomC{\(\lnot Q\)}
    \BinaryInfC{\(\lnot P\)}
\end{prooftree}

If \(P\) implies \(Q\) and \(Q\) is false, then \(P\) is also false.

\subsection{Fallacy of affirming the consequent}

\textit{Affirming the consequent}
is an \underline{invalid} form hypothetical syllogism.

\begin{prooftree}
    \AxiomC{\(P\implies Q\)}
    \AxiomC{\(Q\)}
    \BinaryInfC{\(P\)}
\end{prooftree}

If \(P\) implies \(Q\) and \(Q\) is true, then \(P\) is also true.

\subsection{Fallacy of denying the antecedent}

\textit{Denying the antecedent}
is an \underline{invalid} form hypothetical syllogism.

\begin{prooftree}
    \AxiomC{\(P\implies Q\)}
    \AxiomC{\(\lnot P\)}
    \BinaryInfC{\(\lnot Q\)}
\end{prooftree}

If \(P\) implies \(Q\) and \(P\) is false, then \(Q\) is also false.

\pagebreak

\section{Necessity and sufficiency}

\subsection{Sufficiency}

Given two statements \(P\) and \(Q\) where \(P \implies Q\),
\(P\) suffices for \(Q\) to be true.

\subsection{Necessity}

Given two statements \(P\) and \(Q\) where \(P \implies Q\),
\(Q\) is a necessity for \(P\) to be true (\(Q \impliedby P\)), but
\(Q\) does not necessarily imply \(Q\).

\subsection{Biconditional logical connective}

A biconditional logical connective (written as \textit{iff} or \textit{xnor})
is the relation of equivalence
between two statements \(P\) and \(Q\).
The relation \(P \iff Q\) is both a sufficient condition and
a necessary condition.
\[
    P \iff Q
    \equiv
    (P \implies Q) \land (P \impliedby Q)
\]

\pagebreak

\section{Induction}

Induction can be used to prove a statement in the form \(P(n \in \mathbb{N})\)
for all \(n\). In second-order logic
\[
    \forall P \left(
        P(0) \land \forall n \left( P(n) \implies P(n+1) \right)
        \implies \forall n \left( P(n) \right)
    \right)
\]

\pagebreak

\section{Propositional Logic}

\subsection{Propositional variable}

A \textit{propositional variable} is an input boolean variable.
A propositional variable represents the value of a proposition (E:g. \textit{it is snowy today}).

\subsection{Connectives}

Propositional variables can be connected using \textit{connectives}.
They usually are \(\land\), \(\lor\), \(\lnot\) or \(\implies\).
These connectives are not independent and could be defined in terms of the others.
Terms like  \(\land\), \(\lor\) and \(\lnot\) could also be defined as a composition of a single connective.

\subsection{\(k\)-ary Boolean function}

A \(k\)-ary Boolean function is a mapping from \({\{T, F\}}^k \to \{T,F\}\)

\subsection{\(0\)-ary Boolean function}

The \(0\)-ary Boolean function are the \textit{verum} (\(\top\)) and \textit{falsum} (\(\bot\)) connectives.
The represent respectively the True value and the False value.

\subsection{Propositional formula}

A \textit{propositional formula} is a formula which has a unique truth value given all variables.

The set of all formulas is countable.

\subsection{Truth assignment}

A \textit{truth assignment} is a function which maps a set of propositional
variables \(V=\{p_1, p_2, \ldots, p_n\}\) to a boolean value
\[
    \tau: V \to \{T,F\}
\]

A formula \(A\) involving the variables \(V=\{p_1, p_2, \ldots, p_n\}\)
defines a \(k\)-ary boolean function \(f_A(x_1, x_2, \ldots, x_n)\) where \(x_n = \tau(p_n)\).


%%%%%%%%%%% https://www.youtube.com/watch?v=b_FLLGTjreM&list=WL&index=1&t=1052s

%\subsection{Language}
%
%A \textit{language} \(L\) is a set of connectives which may be used to describe
%an \(L\)-formula.
%
%A language \(L\) is \textit{complete} iff every \(k\)-ary boolean functions can be
%defined by an \(L\)-formula.
%
\subsection{Tautology}

A propositional formula \(A\) is a \textit{tautology} \(\vDash A\) if its \(k\)-ary boolean
function \(f_A\) is always \(T\). \\
Otherwise, we say \(\nvDash A\).

\subsection{Satisfiability}

A propositional formula \(A\) is \textit{satisfiable}
if \(f_A\) is \(T\) for some input.

If \(\Gamma\) is a \underline{set} of propositional formulas, \(\Gamma\)
is satisfiable if there are some assignments to satisfy all its members.

\(\Gamma \vDash A\) (tautologically implies \(A\))
if every truth assignment satisfying \(\Gamma\)
also satisfies \(A\).

\subsection{Substitution}

A \textit{substitution} \(\sigma\)
is a mapping from a set of propositional variables
to the set of propositional formulas.
If \(A\) is a propositional formula, \(A\sigma\)
is equal to the formula obtained by simultaneously
replacing each variable appearing in \(A\)
by its image under \(\sigma\).

\subsection{Propositional Proof System}

A \textit{Propositional Proof System}
\(\mathcal{F}\) has every substitution into the
axioms scheme as his axioms and a set of inference rules.
\\
If \(A\) has an \(\mathcal{F}\)-proof,
then \(\vdash A\). %or \(\vdashsub{\mathcal{F}}A\).
If the proof needs extra hypothesis \(\Gamma\)
(which may not be tautologies), then \(\Gamma\vdash A\)
.%or \(\Gamma \vdashsub{\mathcal{F}} A\)

\subsection{Soundness \(\mathcal{F}\)}

\(\mathcal{F}\) is sound iff every \(\mathcal{F}\)-formula is logically valid with respect to the semantics of the system. 

\subsection{Completness \(\mathcal{F}\)}

\(\mathcal{F}\) is complete iff
it can prove any valid formula, meaning that
the semantic noion of validity
and the syntactic notion of provability coincide,
and a formula is valid iff it is has an
\(\mathcal{F}\)-proof.

\end{document}



https://mathweb.ucsd.edu/~sbuss/ResearchWeb/handbookI/ChapterI.pdf
