\documentclass{article}
\usepackage[utf8]{inputenc}
\usepackage{amsmath}
\usepackage{amssymb}
\usepackage{parskip}
\usepackage{fullpage}

\title{Fourier Analysis}
\author{Paolo Bettelini}
\date{}

\newcommand{\integral}[4]{\int\limits_{#1}^{#2} #3\,d#4}

\begin{document}

\maketitle
\tableofcontents
\pagebreak

\section{Introduction}

A function \(f(x)\) is periodic if there is a positive number \(T\) (the period of \(f\)) such that

\[
    f(x + nT) = f(x)
    \quad \forall x \in D_f, n \in \mathbb{Z}
\]

We can lorem ipsum dolor sit amet a bit

\[
    f(x) = h + \sum_{n=1}^{\infty}
        a_n \cos\left(\frac{2\pi n x}{T}\right) +
        b_n \sin\left(\frac{2\pi n x}{T}\right)
\]

\subsection{The h}

We take the integral over the period \([t_0;t_0+T]\) on both sides

\[
    \integral{t_0}{t_0+T}{f(x)}{x} =
    \integral{t_0}{t_0+T}{h}{x} +
    \sum_{n=1}^{\infty}
    \left[
    \integral{t_0}{t_0+T}{a_n \cos\left(\frac{2\pi n x}{T}\right)}{x} +
    \integral{t_0}{t_0+T}{b_n \sin\left(\frac{2\pi n x}{T}\right)}{x}
    \right]
\]

if you think about it, the integral over a period interval of a fuction such as \(\sin(x)\) or \(\cos(x)\) is 0. \\
If we consider \(\sin(w_n x)\) or \(\cos(w_n x)\) the function will make more full cycles in the span of the period \(T\), \\
all of which yield an area of 0.

\begin{align*}
    \integral{t_0}{t_0+T}{f(x)}{x} &=
    \integral{t_0}{t_0+T}{h}{x} \\
    &= h \integral{t_0}{t_0+T}{}{x} \\
    &= h {\left[x\right]}^{t_0+T}_{t_0}
\end{align*}

concluding that

\[
    h = \frac{1}{T} \integral{t_0}{t_0+T}{f(x)}{x}
\]

\subsection{\(a_n\)}

\subsection{\(b_n\)}

\section{Complex form}

\pagebreak

\section{A Simple Example}

Let's look at a simple example. We are going to derive the Fourier series of a function \(f(x)\) defined as such:

\[
    f(x)=
    \begin{cases}
        -1\quad \text{if } -\pi < x < 0 \\
        +1\quad \text{if } 0 < x < \pi \\
    \end{cases}
\]

The period of this function is \(T=2\pi\). We can already simplify the \(\frac{2\pi}{T}\) term, leaving us with

\[
    f(x)=\frac{a_0}{2} + \sum_{n=1}^{\infty} a_n \cos(nx) + b_n \sin(nx)
\]

First, we need to find \(a_n\). Simplifying \(\frac{2\pi}{T}\) and \(\frac{T}{2}\) we get

\[
    a_n=\integral{-\pi}{\pi}{f(x)\cos(nx)}{x}
\]

Looking at the graph we notice that we can split the integral into two parts at \(x=0\).
On the left part, the function is \(-\cos(nx)\), while on the right part the function is \(\cos(nx)\).

\begin{align*}
    a_n &=
    \frac{1}{\pi} \integral{-\pi}{0}{-\cos(nx)}{x} +
    \frac{1}{\pi} \integral{0}{\pi}{\cos(nx)}{x} \\
    &= -\frac{1}{\pi} \integral{-\pi}{0}{\cos(nx)}{x} +
    \frac{1}{\pi} \integral{0}{\pi}{\cos(nx)}{x} \\
    &= -\frac{1}{\pi} {\left[\frac{\sin(xn)}{n}\right]}_{-\pi}^{0} +
    \frac{1}{\pi} {\left[\frac{\sin(xn)}{n}\right]}_{0}^{-\pi} \\
    &= -\frac{1}{\pi} \left[\frac{\sin(\pi n)}{n}\right] +
    \frac{1}{\pi} \left[\frac{\sin(\pi n)}{n}\right] \\
    &= \left(\frac{1}{\pi}-\frac{1}{\pi}\right) \left[\frac{\sin(\pi n)}{n}\right] \\
    &= 0
\end{align*}

\(a_n\) is always going to be 0. (note). We can remove the \(a_n \cos(nx)\) and \(\frac{a_0}{2}\) terms from the series.

Now for \(b_n\)

\[
    b_n=\integral{-\pi}{\pi}{f(x)\sin(nx)}{x}
\]

Again, we split the integral into two parts

\begin{align*}
    b_n &=
    -\frac{1}{\pi} \integral{-\pi}{0}{\sin(nx)}{x} +
    \frac{1}{\pi} \integral{0}{\pi}{\sin(nx)}{x} \\
    &= -\frac{1}{\pi} {\left[\frac{-\cos(xn)}{n}\right]}_{-\pi}^{0} +
    \frac{1}{\pi} {\left[\frac{-\cos(xn)}{n}\right]}_{0}^{-\pi} \\
    &= -\frac{1}{\pi} \left[-\frac{1}{n}+\frac{\cos(\pi n)}{n}\right] +
    \frac{1}{\pi} \left[\frac{-\cos(\pi n)}{n}+\frac{1}{n}\right] \\
    &= -\frac{1}{\pi} \left[\frac{\cos(\pi n)-1}{n}\right] +
    \frac{1}{\pi} \left[\frac{1-\cos(\pi n)}{n}\right] \\
    &= \frac{2}{\pi} \cdot \frac{1-\cos(\pi n)}{n} \\
    &= \frac{2-2\cos(\pi n)}{\pi n}
\end{align*}

Given \(b_n\) our series is now complete!

\[
    f(x)=\sum_{n=1}^{\infty} \frac{2-2\cos(\pi n)}{\pi n} \cdot \sin(nx)
\]

We won't simplify this further, therefore this is our final result. \\
The effort pays off when we graph this function, as more terms are added, the function looks more and more like the original square wave.

% https://planetmath.org/derivationoffouriercoefficients1

\end{document}
