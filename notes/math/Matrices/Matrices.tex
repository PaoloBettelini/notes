\documentclass{article}
\usepackage[utf8]{inputenc}
\usepackage{amsmath}
\usepackage{amssymb}
\usepackage{parskip}
\usepackage{fullpage}

\title{Matrices}
\author{Paolo Bettelini}
\date{}

\begin{document}

\maketitle
\tableofcontents
\pagebreak

\section{Definition}

A matrix is a rectangular array of numbers or variables. \\
Each matrix has a particular size espressed as \(n \times m\). \\
A generic matrix could look like

\[
A = \begin{bmatrix} 
        a_{1,1} & a_{1,2} & \cdots & a_{1,m} \\
        a_{2,1} & a_{2,2} & \cdots & a_{2,m} \\
        \vdots  & \vdots  & \ddots & \vdots  \\
        a_{n,1} & a_{n,2} & \cdots & a_{n,m} 
    \end{bmatrix}
\]

\section{Matrix Multiplication}

Matrix multiplication is not a communitative operation.

\[
    AB\neq BA
\]

However it is an associative operation.

\[
    A(BC)=(AB)C
\]

Matrix multiplication can be computed only if the numer of columns of the first matrix is equal to the number of rows of the other matrix. \\
This menas that we can compute the product of a \(n \times m\) matrix with a \(i \times j\) matrix, only if \(m=i\). \\
The resulting product is a matrix with \(n\) rows and \(j\) columns.

\[
    (n \times m) \cdot (m \times j) = (n \times j)
\]

If we multiply a \(1 \times n\) matrix by a \(n \times 1\) one, the result is a \(1 \times 1\) matrix or a scalar.

\[
    \begin{bmatrix} 
        a_1 & a_2 & \cdots & a_n
    \end{bmatrix}
    \cdot
    \begin{bmatrix} 
        b_1 \\
        b_2 \\
        \vdots \\
        b_n
    \end{bmatrix}
    = \sum_{k=1}^{n} a_k b_k = a_1 b_1 + a_2 b_2 + \cdots + a_n b_n
\]

This operation is pretty much the same as the dot product in vectors.

For bigger matrices, we apply the same `dot product' between every row and every column. \\
Here is an example:

\[
    \begin{bmatrix} 
        a_1 && a_2 && a_3 && a_4 \\
        \mathbf{b_1} && \mathbf{b_2} && \mathbf{b_3} && \mathbf{b_4} \\
        c_1 && c_2 && c_3 && c_4
    \end{bmatrix}
    \cdot
    \begin{bmatrix} 
        d_1 && \mathbf{e_1} \\
        d_2 && \mathbf{e_2} \\
        d_3 && \mathbf{e_3} \\
        d_4 && \mathbf{e_4}
    \end{bmatrix}
    =
    \begin{bmatrix} 
        a \cdot d && a \cdot e \\
        b \cdot d && \mathbf{b \cdot e} \\
        c \cdot d && c \cdot e
    \end{bmatrix}
\]

Where \(a \cdot b\) denotes the `dot product' between a row and a column.

\pagebreak

\section{Linear transformation}

Multiplying a vector by a matrix produces another vector. This is a linear transformation.

For instance, the standard rotation matrix in two dimensions hash the following form:

\[
    R(\theta)=
    \begin{bmatrix} 
        \cos\theta && -\sin\theta \\
        \sin\theta && \cos\theta
    \end{bmatrix}
\]

Multiplying a vector \(\vec{p}\in \mathbb{R}^2\) by \(R(\theta)\) will rotate the point around the origin depending on \(\theta\).

\[
    \vec{p}=
    \begin{pmatrix} 
        x \\
        y
    \end{pmatrix}
\]

We can multiply the rotational matrix by the column vector to obtain the new coordinates

\[
    \begin{bmatrix} 
        x' \\
        y'
    \end{bmatrix}
    =
    \begin{bmatrix} 
        \cos\theta && -\sin\theta \\
        \sin\theta && \cos\theta
    \end{bmatrix}
    \cdot
    \begin{bmatrix} 
        x \\
        y
    \end{bmatrix}
    =
    \begin{bmatrix} 
        x\cos\theta - y\sin\theta \\
        x\sin\theta + y\cos\theta
    \end{bmatrix}
\]

Note: the matrix must be positioned on the right of the vector, otherwise the dimensions would no longer be compatible.

The rotation matrices for a three-dimensional point are

\begin{align*}
    R_x(\theta)&=
    \begin{bmatrix} 
        1 & 0 & 0 \\
        0 & \cos\theta & -\sin\theta \\
        0 & \sin\theta & \cos\theta
    \end{bmatrix}
    \\
    R_y(\theta)&=
    \begin{bmatrix} 
        \cos\theta & 0 & \sin\theta \\
        0 & 1 & 0 \\
        -\sin\theta & 0 & \cos\theta
    \end{bmatrix}
    \\
    R_z(\theta)&=
    \begin{bmatrix} 
        \cos\theta & -\sin\theta & 0 \\
        \sin\theta & \cos\theta & 0 \\
        0 & 0 & 1
    \end{bmatrix}
\end{align*}

\pagebreak

\section{Matrix transpose}

The transpose operator is analogous to that of vectors

\[
    \begin{pmatrix}
        a_1 & a_2 & \cdots & a_n
    \end{pmatrix}^t
    =
    \begin{pmatrix}
        a_1 \\
        a_2 \\
        \vdots \\
        a_n
    \end{pmatrix}
\]

This is actually a special case, where we apply the transpose operator to a matrix column.

The transpose of a generic \(n \times m\) matrix results in a \(m \times n\) one.

\[
    {\left(A_{ij}\right)}^t=A_{ji}
\]

The transpose of a matrix is just a flipped version of the original matrix.
We can transpose a matrix by switching its rows with its columns.
The original rows become the new columns and the original columns become the new rows.

\[
    {\begin{bmatrix} 
        a_{1,1} & a_{1,2} & \cdots & a_{1,m} \\
        a_{2,1} & a_{2,2} & \cdots & a_{2,m} \\
        \vdots  & \vdots  & \ddots & \vdots  \\
        a_{n,1} & a_{n,2} & \cdots & a_{n,m} 
    \end{bmatrix}}^t
    =
    \begin{bmatrix} 
        a_{1,1} & a_{2,1} & \cdots & a_{m,1} \\
        a_{1,2} & a_{2,2} & \cdots & a_{m,2} \\
        \vdots  & \vdots  & \ddots & \vdots  \\
        a_{1,n} & a_{2,n} & \cdots & a_{m,n} 
    \end{bmatrix}
\]

\section{Identify matrix}

The identify matrix is a square matrix with ones on the main diagonal and zeros elsewhere.
This matrix is denoted as \(I_n\) where \(n\) is the length of the side.

\[
    I_n=
    \begin{bmatrix}
        1 & 0 & 0 & \cdots & 0 \\
        0 & 1 & 0 & \cdots & 0 \\
        0 & 0 & 1 & \cdots & 0 \\
        \vdots & \vdots & \vdots & \ddots & \vdots \\
        0 & 0 & 0 & \cdots & 1 \\
    \end{bmatrix}
\]

When \(I_n\) is applied to a matrix or vector, the matrix or vector remains the same.

% definition using the Kronecker delta

\pagebreak

\section{Determinants}

Every square matrix has a scalar called the determinant.

Given a \(2 \times 2\) matrix \(A\)

\[
    A=
    \begin{bmatrix}
        a & b \\
        c & d
    \end{bmatrix}
\]

its determinant is defined as

\[
    \det(A)=
    \begin{vmatrix}
        a & b \\
        c & d
    \end{vmatrix}
    \equiv ad-bc
\]

To compute the determinant of bigger matrices, we must define the minor of a matrix element. \\
The minor of a matrix element is the part of the matrix remaining after excluding the row and the column containing that particular element.

% https://lapastillaroja.net/wp-content/uploads/2016/09/Intro_to_QC_Vol_1_Loceff.pdf

% det(A^t)=det(A)

\[
    \text{minor of }b=
    \begin{bmatrix}
        a & b & c \\
        d & e & f \\
        g & h & i
    \end{bmatrix}
    =
    \begin{vmatrix}
        d & f \\
        g & i
    \end{vmatrix}
\]

\end{document}
