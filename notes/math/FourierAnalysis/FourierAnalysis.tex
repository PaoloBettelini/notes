\documentclass{article}
\usepackage[utf8]{inputenc}
\usepackage{amsmath}
\usepackage{amssymb}
\usepackage{parskip}
\usepackage{fullpage}

\title{Fourier Analysis}
\author{Paolo Bettelini}
\date{}

\newcommand{\integral}[4]{\int\limits_{#1}^{#2} #3\,d#4}

\begin{document}

\maketitle
\tableofcontents
\pagebreak

\section{Introduction}

A function \(f(x)\) is periodic if there is a positive number \(T\) (the period of \(f\)) such that

\[
    f(x + nT) = f(x)
    \quad \forall x \in D_f, n \in \mathbb{Z}
\]

We can lorem ipsum dolor sit amet a bit

\[
    f(x) = h + \sum_{n=1}^{\infty}
        a_n \cos\left(\frac{2\pi n x}{T}\right) +
        b_n \sin\left(\frac{2\pi n x}{T}\right)
\]

\subsection{The h}

We take the integral over the period \([t_0;t_0+T]\) on both sides

\[
    \integral{t_0}{t_0+T}{f(x)}{x} =
    \integral{t_0}{t_0+T}{h}{x} +
    \sum_{n=1}^{\infty}
    \left[
    \integral{t_0}{t_0+T}{a_n \cos\left(\frac{2\pi n x}{T}\right)}{x} +
    \integral{t_0}{t_0+T}{b_n \sin\left(\frac{2\pi n x}{T}\right)}{x}
    \right]
\]

if you think about it, the integral over a period interval of a fuction such as \(\sin(x)\) or \(\cos(x)\) is 0. \\
If we consider \(\sin(w_n x)\) or \(\cos(w_n x)\) the function will make more full cycles in the span of the period \(T\), \\
all of which yield an area of 0.

\begin{align*}
    \integral{t_0}{t_0+T}{f(x)}{x} &=
    \integral{t_0}{t_0+T}{h}{x} \\
    &= h \integral{t_0}{t_0+T}{}{x} \\
    &= h {\left[x\right]}^{t_0+T}_{t_0}
\end{align*}

concluding that

\[
    h = \frac{1}{T} \integral{t_0}{t_0+T}{f(x)}{x}
\]

\subsection{\(a_n\)}

\subsection{\(b_n\)}

\section{Complex form}

\pagebreak

% https://planetmath.org/derivationoffouriercoefficients1

\end{document}
