\documentclass{article}
\usepackage[utf8]{inputenc}
\usepackage{amsmath}
\usepackage{amssymb}
\usepackage{parskip}
\usepackage{fullpage}

\title{Riemann Hypothesis}
\author{Paolo Bettelini}
\date{}

\newcommand{\exceptone}{
    \,\backslash\,\{1\}
}

\begin{document}

\maketitle

\section*{Abstract}
This document contains the main concepts about the
Riemann Hypothesis and some derivations of the formulas and series used.

\pagebreak

\tableofcontents
\pagebreak

\section{Zeta function}

\subsection{Definition}
The zeta function is defined as
\begin{align*}
    \zeta(s)=\sum_{n=1}^{\infty}\frac{1}{n^s},\quad Re(s)>1
\end{align*}

\subsection{Euler product}
The zeta function can be represented as an Euler product.
\\
We will start by using the first prime number: 2, and multiply both sides by \(2^{-s}\).
\begin{align*}
    \zeta(s)\frac{1}{2^s}=\sum_{n=1}^{\infty}\frac{1}{{(2n)}^s}
\end{align*}
We then subtract the second definition from the first one, such that
\begin{align*}
    \zeta(s)-\zeta(s)\frac{1}{2^s}&=
    \sum_{n=1}^{\infty}\left[\frac{1}{n^s}\right]-
    \sum_{n=1}^{\infty}\left[\frac{1}{{(2n)}^s}\right]
    \\
    \zeta(s)\left(1-\frac{1}{2^s}\right)&=
    \sum_{n=1}^{\infty}\frac{1}{n^s},
    \quad n\neq 2k,k\in \mathbb{Z}
\end{align*}
Here we are excluding the multiples of 2 from the series.

If we do the same with the next prime number, which is 3, we get
\begin{align*}
    \zeta(s)\left(1-\frac{1}{2^s}\right)\left(1-\frac{1}{3^s}\right)&=
    \sum_{n=1}^{\infty}\frac{1}{n^s},
    \quad n\neq 2k,n\neq 3k, k\in \mathbb{Z}
\end{align*}

We can repeat this process with every prime number.
\\
Eventually, we will exclude every nth-term to sum as we use every prime number, except for n=1.
\begin{align*}
    \zeta(s)\prod_{p\in P}^{\infty}1-\frac{1}{p^s}=\frac{1}{1^s}=1
\end{align*}
Finally, we get the identity
\begin{align*}
    \zeta(s)=
    \prod_{p\in P}^{\infty}\frac{1}{1-p^{-s}}
\end{align*}

\pagebreak

\section{Analytic continuation}

\subsection{Zeta function for positive Re (s)}

We have seen that the classical zeta function definition only converges for \(Re(s)>1\)
\begin{align*}
    \zeta(s)=\sum_{n=1}^{\infty}\frac{1}{n^s},
    \quad Re(s)>1
\end{align*}

We can use the eta function \(\eta(s)\), which is defined for \(Re(s)>0\exceptone\), to analytically extend the zeta function domain to \(Re(s)>0\exceptone\).
\\
The eta function is a Dirichlet series defined as
\begin{align*}
    \eta(s)=\sum_{n=1}^{\infty}\frac{{(-1)}^{n-1}}{n^s},
    \quad Re(s)>0\exceptone
\end{align*}

We start by splitting the zeta function into two distinct series, one for \(n\) even and the other one for \(n\) odd.
\\
The index for the even series will be \(2n\), while the odd one will use \(2n-1\) as the index.
\begin{align*}
    \zeta(s)=
    \sum_{n=1}^{\infty}\left[\frac{1}{{(2n)}^s}\right]+
    \sum_{n=1}^{\infty}\left[\frac{1}{{(2n-1)}^s}\right]
\end{align*}
We do the same thing with the eta function.
\\
Notice that \({(-1)}^n\) is 1 when \(n\) is even and -1 when \(n\) is odd.
\begin{align*}
    \eta(s)=
    \sum_{n=1}^{\infty}\left[\frac{1}{{(2n)}^s}\right]-
    \sum_{n=1}^{\infty}\left[\frac{1}{{(2n-1)}^s}\right]
\end{align*}
We subtract these two definition from eachother
\begin{align*}
    \zeta(s)-\eta(s)&=
    2\sum_{n=1}^{\infty}\frac{1}{{(2n)}^s}
    \\
    &=2^{1-s}\sum_{n=1}^{\infty}\frac{1}{k^s}
    \\
    &=2^{1-s}\zeta(s)
    \\
    \frac{1}{1-2^{1-s}}\eta(s)&=\zeta(s)
\end{align*}
We finally get
\begin{align*}
    \zeta(s)=\frac{1}{1-2^{1-s}}\sum_{n=1}^{\infty}\frac{{(-1)}^{n-1}}{n^s},
    \quad Re(s)>0\exceptone
\end{align*}

This series can be used to compute value of the zeta function along the critical strip \(0<Re(s)<1\).

\subsection{Zeta function for negative Re (s)}

\subsection{Zeta function for s=0}

\subsection{Zeta function for s=1}

The zeta function is holomorphic everywhere except for a pole at \(s=1\) with residue 1.\\
This is the only value of the complex plane that cannot be evaluated through analytic continuation.

\pagebreak

\section{Zeroes of the zeta function}

\subsection{Trivial zeroes}

Considering the functional equation and analytic continuation of the zeta function
\begin{align*}
    \zeta(s)=2^s\pi^{s-1}\sin\left(\frac{\pi s}{2}\right)\Gamma(1-s)\zeta(1-s)
\end{align*}
We can notice that the term \(\sin\left(\frac{\pi s}{2}\right)\) equals \(0\) when \(s\) is a multiple of \(2\).
\\
However, the gamma function has a pole for every negative integer, this constrains our zeroes to be less or equal to 1.\\
Furthermore, the zeta function has a pole at \(s=1\), excluding the value \(s=0\) from the zeroes, leaving \(\{-2;-4;-6;\cdots\}\)
\begin{align*}
    \zeta(2k)=0,
    \quad k\in \mathbb{Z}^{-}
\end{align*}

These zeroes are called trivial zeroes because they are not relevant to the Riemann hypothesis.

\subsection{Non-trivial zeroes}

The Riemann hypothesis states that every non-trivial zero lies on the critical line \(Re(s)=\frac{1}{2}\).

\pagebreak

\section{Prime-counting function}

\subsection{Properties of the prime-counting function}

The prime-counting function \(\pi(x)\) is defined as the number of primes less or equals than \(x\).

We can consider the difference between \(\pi(x)\) of two consecutive integers
\begin{align*}
    \pi (x)-\pi (x-1)= 
    \begin{cases}
        1,& \text{if } x\in P
        \\
        0,& \text{otherwise}
    \end{cases}
\end{align*}

Given a series over all prime numbers, we can extend it to all integers and multiply each term by this difference.
\\
The terms whose index is not a prime number will be multiplied by 0.
\begin{align*}
    \sum_{p\in P}^{\infty}a_k=\sum_{n=2}^{\infty}\left[\pi(n)-\pi(n-1)\right]a_n
\end{align*}
Here we start at 2 since there are no prime numbers less than 2.

\subsection{Relationship with the zeta function}

We have seen that the zeta function can be written as an Euler Product

\begin{align*}
    \zeta (s)=\prod_{p\in P}^{\infty}\frac{1}{1-p^{-s}}
\end{align*}

However, we need convert this product into a series in order to apply the identity of the last paragraph.
\\
We can take the natural logarithm of both sides and use the multiplication property
\begin{align*}
    \ln\left(\zeta (s)\right)&=\ln\prod_{p\in P}^{\infty}\frac{1}{1-p^{-s}}
    \\
    &=\sum_{p\in P}^{\infty}\ln\left(\frac{1}{1-p^{-s}}\right)
    \\
    &=\sum_{p\in P}^{\infty}-\ln\left(1-p^{-s}\right)
\end{align*}
Now we can apply the identity
\begin{align*}
    \ln\left(\zeta (s)\right)=\sum_{n=2}^{\infty}-\ln\left(1-n^{-s}\right)\left[\pi (n) - \pi (n-1)\right]
\end{align*}

The next goal is to factor out \(\pi (n)\)
\begin{align*}
    \ln\left(\zeta(s)\right)
    &=\sum_{n=2}^{\infty}\left[\pi (n-1)\ln\left(1-n^{-s}\right)\right]
    -\sum_{n=2}^{\infty}\left[\pi (n)\ln\left(1-n^{-s}\right)\right]
    \\&=\sum_{n=2}^{\infty}\left[\pi (n)\ln\left(1-{(n+1)}^{-s}\right)\right]
    -\sum_{n=2}^{\infty}\left[\pi (n)\ln\left(1-n^{-s}\right)\right]
    \\
    &=\sum_{n=2}^{\infty}\pi (n)\left[\ln\left(1-{(n+1)}^{-s}\right)-\ln\left(1-n^{-s}\right)\right]
\end{align*}
To simplify further more, we consider the derivative of the function \(\ln(1-x^{-s})\).
\\
Using the chain rule we get
\begin{align*}
    \frac{d}{dx}\ln\left(1-x^{-s}\right)=
    \frac{s}{x(x^s-1)}
\end{align*}
Therefore,
\begin{align*}
    \ln\left(1-x^{-s}\right)=
    \int \frac{s}{x(x^s-1)}\,dx+C
\end{align*}

Considering \(f(x)=\ln(1-x^{-s})\), our series can be expressed as
\begin{align*}
    \ln\left(\zeta(s)\right)=
    \sum_{n=2}^{\infty}\pi(n)\left[f(n+1)-f(n)\right]
\end{align*}
which can be written as an integral from \(n\) to \(n+1\)
\begin{align*}
    \ln\left(\zeta(s)\right)&=
    \sum_{n=2}^{\infty}\pi(n)
    \int\limits_n^{n+1} f'(x)\,dx
    \\
    &=
    \sum_{n=2}^{\infty}\pi(n)
    \int\limits_n^{n+1}
    \frac{s}{x(x^s-1)}\,dx
    \\
    &=
    \sum_{n=2}^{\infty}
    \int\limits_n^{n+1}
    \frac{s\pi(x)}{x(x^s-1)}\,dx
\end{align*}
Instead of taking the sum of each of these integrals (2 to 3, 3 to 4, \ldots), we can make a single integral
\begin{align*}
    \ln\left(\zeta(s)\right)=
    s\int\limits_2^\infty
    \frac{\pi(x)}{x(x^2-1)}\,dx
\end{align*}

\pagebreak

\subsection{Approximations}

A pretty good approximation to \(\pi(x)\) is
\begin{align*}
    li(x)=\int\limits_0^{x} \frac{dt}{\ln\,t}
\end{align*}
called the logarithmic integral function

\subsection{Exact form}

Riemann proved that the exact form the prime counting function is

\begin{align*}
    \pi(x)=Re(x)-\sum_{p}R(x^p)
\end{align*}
where
\begin{align*}
    R(x)=\sum_{n=1}^{\infty}\frac{\mu(n)}{n}li(\sqrt[n]{x})
\end{align*}
and \(\mu(x)\) is the Möbius function.\\
Also, \(p\) indexes every zero of the zeta function.

\end{document}