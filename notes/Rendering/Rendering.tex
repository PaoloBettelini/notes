\documentclass{article}
\usepackage[utf8]{inputenc}
\usepackage{amsmath}
\usepackage{amssymb}
\usepackage{parskip}
\usepackage{fullpage}
\usepackage{hyperref}

\hypersetup{
    colorlinks=true,
    linkcolor=black,
    urlcolor=blue,
    pdftitle={Physical Rendering},
    pdfpagemode=FullScreen,
}

\title{Physical Rendering}
\author{Paolo Bettelini}
\date{}

\begin{document}

\maketitle
\tableofcontents
\pagebreak

\section{Measurements}

\subsection{Radiant Flux}

The radiant flux (or power) \(\Phi\) is the total amount of energy passing
through a surface per second and is measured in \([W]\) (watts) as \(\frac{J}{s}\).

\subsection{Irradiance}

The irradiance \(E\) is the measurements of the radiant flux per \textit{unit area}
and is measured in \([W]{[M]}^{-2}\) as \(\frac{\Phi}{m^2}\).

\subsection{Radiance}

The radiance \(L\) is the irradiance per unit solid angle (steradian) and is
measured in \([W]{[M]}^{-2}{[M]}^{-2}{[sr]}^{-1}\) as \(\frac{E}{sr}\).

\section{Terminology}

% tikz picture

\begin{itemize}
    \item \(\hat{V}\) direction torwards the camera
    \item \(\hat{N}\) surface normal
    \item \(\hat{L}\) vector pointing torward the light source
    \item \(\hat{R}\) reflected ray direction
    \item \({\theta}_i {\theta}_r\) incident and reflected angles
\end{itemize}

\(\hat{R}=\hat{L}-2\hat{N}(\hat{L}\cdot\hat{N})\)

% #2 08:00

\pagebreak

\section{Rendering equation}

The rendering equation tells us how much light is exiting a \textit{surface point}
in a given direction

% reference two minutes paper

\end{document}
