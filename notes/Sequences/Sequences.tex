\documentclass{article}

\usepackage{amsmath}
\usepackage{amssymb}
\usepackage{amsthm}
\usepackage{parskip}
\usepackage{fullpage}
\usepackage{hyperref}

\hypersetup{
    colorlinks=true,
    linkcolor=black,
    urlcolor=blue,
    pdftitle={Sequences},
    pdfpagemode=FullScreen,
}

\title{Sequences}
\author{Paolo Bettelini}
\date{}

\begin{document}

\maketitle
\tableofcontents
\pagebreak

\section{Definition}

A sequence denotes a series of indexed values.
A sequence may be written as
\[
    \{a_n\} \quad \quad {\{a_n\}}_{n=1}^\infty
\]

If \(\lim_{n \to \infty}a_n\) exists and is finite
we say that the sequence is \textit{convergent}. If 
\(\lim_{n \to \infty}a_n\) doesn't exist or is infinite
we say that the sequence \textit{diverges}.

If \(f(x)\) is a function such that \(f(n)=a_n\)
\[
    \lim_{x\to\infty}f(x)=L \implies
    \lim_{n\to\infty}a_n=L
\]

\subsection{Properties}

if \(\{a_n\}\) and \(\{b_n\}\) are convergent sequences, then

\[
    \lim_{n\to\infty} (a_n \pm b_n) = \lim_{n\to\infty} a_n \pm
    \lim_{n\to\infty} b_n
\]
\[
    \lim_{n\to\infty} ca_n = c \lim_{n\to\infty} a_n
\]
\[
    \lim_{n\to\infty} (a_n b_n) =
    \left(\lim_{n\to\infty} a_n\right)
    \left(\lim_{n\to\infty} b_n\right)
\]
% 2 omitted

\section{Squeeze Theorem}

If \(a_n \leq c_n \leq b_n\) for sufficiently large \(n>N\) for some \(N\)
and \(\lim_{n\to\infty}a_n =\lim_{n\to\infty}b_n=L\)
then \(\lim_{n\to\infty} c_n =L\)

\section{Absolute Value}

Note the following
\[
    -|a_n| \leq a_n \leq |a_n|
\]
Then if we assume
\[
    \lim_{n\to\infty} (-|a_n|) = - \lim_{n\to\infty} |a_n| =0 
\]
by the Squeeze Theorem we get
\[
    \lim_{n\to\infty} a_n =0
\]

We conclude that if \(\lim_{n\to\infty} |a_n|=0\) then
\(\lim_{n\to\infty} a_n=0\).

\section{Exponential sequence} % right name?

The sequence \({\{a^n\}}_{n=0}^\infty\) converges iff \(-1<r\leq 1\)
\[
    \lim_{n\to\infty} a^n = \begin{cases}
        0 & \text{if } -1 < r < 1 \\
        1 & \text{if } r=1
    \end{cases}
\]

\pagebreak

\section{Convergence of even and odd indexes}

\newtheorem*{theorem1}{Theorem}

\begin{theorem1}
    If \(\lim_{n\to\infty}a_{2n}=L\) and \(\lim_{n\to\infty}a_{2n+1}=L\)
    then \(\{a_n\}\) is convergent and \(\lim_{n\to\infty}a_n=L\).
\end{theorem1}
\begin{proof}
    Let \(\epsilon>0\). \\
    Since \(\lim_{n\to\infty}a_{2n}=L\) there exists an \(N_1\) such that
    if \(n>N_1\) then
    \[
        |a_{2n}-L|<\epsilon
    \]
    Also, since \(\lim_{n\to\infty}a_{2n+1}=L\) there exists an \(N_2\) such that
    if \(n>N_2\) then
    \[
        |a_{2n+1}-L|<\epsilon
    \]
    Let \(N=\max(2N_1, 2N_2+1)\) and let \(n>N\).
    Then either \(a_n=a_{2k}\) for some \(k>N_1\) or \(a_n=a_{2k+1}\)
    for some \(k>N_2\). Either way we have
    \[
        |a_n-L|<\epsilon
    \]
    which satisfies the convergence of \(\{a_n\}\).
\end{proof}

\section{Properties of a sequence}

\subsection{Increasing}

A sequence is \textit{increasing} iff \(\forall n:  a_n<a_{n+1}\).

\subsection{Decreasing}

A sequence is \textit{decreasing} iff \(\forall n:  a_n>a_{n+1}\).

\subsection{Monotonic}

If \(\{a_n\}\) is increasing or decrasing it is also called \textit{monotonic}.

\subsection{Bounded below}

If there exists a number \(m\) such that \(\forall n : m \leq a_n\)
the sequence is \textit{bounded below} by a lower bound.

\subsection{Bounded above}

If there exists a number \(m\) such that \(\forall n : m \geq a_n\)
the sequence is \textit{bounded above} by an upper bound.

\subsection{Bounded}

If the sequence is both bounded above and bounded below it is
said \textit{bounded}.

\section{AAAAAAAAA}

If a is bounded and monotonic then it is convergent.

\pagebreak

\end{document}
