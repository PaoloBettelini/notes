\documentclass[a4paper]{article}

\usepackage{amsmath}
\usepackage{amssymb}
\usepackage{parskip}
\usepackage{fullpage}
\usepackage{hyperref}

\hypersetup{
    colorlinks=true,
    linkcolor=black,
    urlcolor=blue,
    pdftitle={GroupTheory},
    pdfpagemode=FullScreen,
}

\title{Group Theory}
\author{Paolo Bettelini}
\date{}

\begin{document}

\maketitle
\tableofcontents
\pagebreak

\section{Groups}

\subsection{Binary operations}

Let \(G\) be a set. A \textit{binary operation} \(\circ\) on \(G\) is a map
\[
    G \times G \to G,
    \quad\quad\quad\quad
    (x,y) \to x \circ y
\]

% closure

\subsection{Cayley tables}

A binary operation \(\circ\) on a finite set \(G\) can be
visualized using a \textit{Cayley table}.

Example: \(G=\{0,1\}\) and \(\circ \equiv \text{multiplication}\).
\begin{tabular}{|c|c|c|}
    \hline
    \(\circ\) & 0 & 1 \\
    \hline
    0 & 0 & 0 \\
    \hline
    1 & 0 & 1 \\
    \hline
\end{tabular}

\subsection{Definition}

A \textit{group} \((G,\circ)\) is a tuple containing a set \(G\) and
a binary operation \(\circ\) where \(\circ\) satisfies.
The operation \(\circ\) between \(a\) and \(b\) may be written as
\(a\circ b\) or just \(ab\).

\begin{enumerate}
    \item \textbf{Associativity}: \(\forall a,b,c\in G a \circ (b \circ c) = (a \circ b) \circ c\)
    \item \textbf{Identity}: \(\exists e \,|\, \forall a \in G, ea=ae=a\) 
    \item \textbf{Inverse}: \(\forall a\in G \exists a^{-1} | a^{-1}a = aa^{-1} = e\)
    \item \textbf{Closure}: \(\forall a,b\in G a \circ b \in G\)
\end{enumerate}

The element \(e\) is unique whereas \(a^{-1}\) depends on \(a\). Every
element has a unique inverse.

\subsection{Proof of uniqueness of the identity element}

Suppose there is more than one identity element, \(e_1\) and \(e_2\).
\begin{align*}
    e_1 &= e_1 \circ e_2 &\text { since \(e_2\) is an identity} \\
    &= e_2 &\text { since \(e_1\) is an identity}
\end{align*}
Thus, \(e_1\) and \(e_2\) must be the same. This reasoning can be extended
to when we may suppose to have \(n\) identity elements.

\subsection{Proof of uniqueness of the inverse element}

Suppose we have \(a\in G\) with inverses \(c\) and \(c\).
\begin{align*}
    b = b \circ e &= b \circ (a \circ c)\\
    (b \circ a) c &= e \circ c \\
    &= c
\end{align*}
Thus, \(b\) and \(c\) must be the same. This reasoning can be extended
to when we may suppose to have \(n\) inverses of \(a\).

% left(and right)- cancellation property
% https://proofwiki.org/wiki/Category:Cancellation_Laws

\subsection{Inverse of Product}

This theorem says that \({(a \circ b)}^{-1} = a^{-1} \circ b^{-1}\).

We start by noticing that by association we have
\begin{align*}
    (a \circ b) \circ (b^{-1} \circ a^{-1}) &= a \circ (b \circ b^-1) \circ a^{-1} \\
    &= a \circ e \circ a^{-1} \\
    &= a \circ a^{-1} \\
    &= e
\end{align*}
This implies that \((a \circ b)\) is the inverse of \((b^{-1} \circ a^{-1})\).
Since \((a\circ b) \circ {(a \circ b)}^{-1} =e\) we have
\begin{align*}
    (a \circ b) \circ (b^{-1} \circ a^{-1}) = e = (a\circ b) \circ {(a \circ b)}^{-1}
\end{align*}
We can clearly see that \((b^{-1} \circ a^{-1}) = {(a \circ b)}^{-1}\).

\section{Subgroups}

\subsection{Definition}

Given a group \(g=(G, \circ)\) and a group \(h=(H, \circ)\), \(h\)
is a subgroup of \(g\) (\(g \leq h\)) if \(H \subseteq G\).

\subsection{One-Step Subgroup Test}

Let \((G, \circ)\) be a group and let \(H \subseteq G\).
Then \((H, \circ)\) is a subgroup of \((G, \circ)\) iff
\begin{itemize}
    \item \(\emptyset \neq H\)
    \item \(\forall a,b \in H a \circ b^{-1} \in H\)
\end{itemize}

% https://youtu.be/BL8qXd-e2QQ?list=PL22w63XsKjqwN7sHsEiy0yqkcjQfXAuVb

% https://cameroncounts.files.wordpress.com/2016/11/groups.pdf

\end{document}
