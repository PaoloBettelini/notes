\documentclass[a4paper]{article}

\usepackage{amsmath}
\usepackage{amssymb}
\usepackage{amsthm}
\usepackage{parskip}
\usepackage{fullpage}
\usepackage{hyperref}

\hypersetup{
    colorlinks=true,
    linkcolor=black,
    urlcolor=blue,
    pdftitle={GroupTheory},
    pdfpagemode=FullScreen,
}

\title{Group Theory}
\author{Paolo Bettelini}
\date{}

\begin{document}

\maketitle
\tableofcontents
\pagebreak

\section{Groups}

\subsection{Cayley tables}

A binary operation \(\circ\) on a finite set \(G\) can be
visualized using a \textit{Cayley table}.

Example: \(G=\{0,1\}\) and \(\circ \equiv \text{multiplication}\).
\begin{tabular}{|c|c|c|}
    \hline
    \(\circ\) & 0 & 1 \\
    \hline
    0 & 0 & 0 \\
    \hline
    1 & 0 & 1 \\
    \hline
\end{tabular}

\subsection{Definition}

A \textit{group} \((G,\circ)\) is a tuple containing a set \(G\) and
a binary operation \(\circ\) on \(G \times G\).
The operation \(\circ\) between \(a\) and \(b\) may be written as
\(a\circ b\) or just \(ab\).

The relation must satisfy the following properties

\begin{enumerate}
    \item \textbf{Associativity}: \(\forall a,b,c\in G, a \circ (b \circ c) = (a \circ b) \circ c\)
    \item \textbf{Identity}: \(\exists e \,|\, \forall a \in G, ea=ae=a\) 
    \item \textbf{Inverse}: \(\forall a \in G, \exists a^{-1} \in G \,|\, a^{-1}a = aa^{-1} = e\)
    \item \textbf{Closure}: \(\forall a,b\in G, a \circ b \in G\)
\end{enumerate}

The element \(e\) is unique whereas \(a^{-1}\) depends on \(a\). Every
element has a unique inverse.

\subsection{Proof of uniqueness of the identity element}

Suppose there is more than one identity element, \(e_1\) and \(e_2\).
\begin{align*}
    e_1 &= e_1 \circ e_2 &\text { since \(e_2\) is an identity} \\
    &= e_2 &\text { since \(e_1\) is an identity}
\end{align*}
Thus, \(e_1\) and \(e_2\) must be the same. This reasoning can be extended
to when we may suppose to have \(n\) identity elements.

\subsection{Proof of uniqueness of the inverse element}

Suppose we have \(a\in G\) with inverses \(b\) and \(c\).
\begin{align*}
    b = b \circ e &= b \circ (a \circ c)\\
    (b \circ a) c &= e \circ c \\
    &= c
\end{align*}
Thus, \(b\) and \(c\) must be the same. This reasoning can be extended
to when we may suppose to have \(n\) inverses of \(a\).

\subsection{Cancellation laws}

Rigth cancellation law
\[
    ba = ca \implies b = c
\]

Left cancellation law
\[
    ab = ac \implies b = c
\]

\subsection{Inverse of Product}

This theorem says that \({(a \circ b)}^{-1} = b^{-1} \circ a^{-1}\).

We start by noticing that by associativity we have
\begin{align*}
    (a \circ b) \circ (b^{-1} \circ a^{-1}) &= a \circ (b \circ b^-1) \circ a^{-1} \\
    &= a \circ e \circ a^{-1} \\
    &= a \circ a^{-1} \\
    &= e
\end{align*}
This implies that \((a \circ b)\) is the inverse of \((b^{-1} \circ a^{-1})\).
Since \((a\circ b) \circ {(a \circ b)}^{-1} =e\) we have
\begin{align*}
    (a \circ b) \circ (b^{-1} \circ a^{-1}) = e = (a\circ b) \circ {(a \circ b)}^{-1}
\end{align*}
We can clearly see that \((b^{-1} \circ a^{-1}) = {(a \circ b)}^{-1}\).

In general, we have
\[
    (a_1 \circ a_2 \circ \dots a_n)^{-1}
    = a_n^{-1} \circ \dots \circ a_2^{-1} \circ a_1^{-1}
\]

\section{Subgroups}

\subsection{Definition}

Given an algebraic structure \(g=(G, \circ)\) and a group \(h=(H, \circ)\), \(h\)
is a subgroup of \(g\) (\(g \leq h\)) if \(H \subseteq G\).

\subsection{One-Step Subgroup Test}

\newtheorem*{theorem1}{Theorem}

\begin{theorem1}
    Let \((G, \circ)\) be a group and let \(H \subseteq G\) where \(\emptyset \neq H\).\\
    Then \((H, \circ)\) is a subgroup of \((G, \circ) \iff
    \forall a,b \in H, a \circ b^{-1} \in H\).
\end{theorem1}
\begin{proof}
    (\(\implies\)): Assume \((H, \circ) \leq (G, \circ)\).
    The properties of a group directly infer \(\forall a,b \in H, a \circ b^{-1} \in H\) \\
    (\(\impliedby\)): Assume \(\forall a,b \in H, a \circ b^{-1} \in H\)
    \begin{itemize}
        \item \textbf{Identity}: let \(a=b\), then \(a\circ a^{-1} H \implies e \in H\).
        \item \textbf{Inverse}: Let \(k\in H\), \(a=e\) and \(b=k\).
        \(a\circ b^{-1} = e \circ k^{-1} \implies k^{-1} \in H\).
        \item \textbf{Closure}: Let \(m, n \in H \implies n^{-1} \in H\) and let \(a=m\) and \(b=n^{-1}\).
        \(a\circ b^{-1} = a \circ (b^{-1})^{-1}=a\circ b\). This implies \(a, b \in H\).
    \end{itemize}
\end{proof}

\subsection{The centralizer subgroup}

Let \(H \leq G\) be groups and define
\[
    \text{C}_G(H) = \{
        g \in G \,|\, \forall h \in H, gh=hg
    \}
\]
as the centralizer of \(H\).
This is the set of all elements of \(G\) such that they commute with every element of \(H\).

\newtheorem*{theorem2}{Theorem}

\begin{theorem2}
    Let \(H \leq G\), then \(\text{C}_G(H) \leq G\).
\end{theorem2}
\begin{proof}
    Suppose \(a,b \in \text{C}_G(H)\).
    We want to show \(ab^{-1} \in \text{C}_G(H)\).\\
    Note that the condition \(gh=hg \iff hg^{-1}=g^{-1}h\).\\
    Consider the expression \((ab^{-1})h = a(b^{-1}h) = ahb^{-1} = h(ab^{-1})\).
    This means that \(ab^{-1} \in \text{C}_G(H)\) and thus in \(H\).
\end{proof}

\subsection{The conjugate subgroup}

Let \(H \leq G\) be groups and define
\[
    g^{-1}Hg = \{
        g^{-1}hg \,|\, h \in H    
    \}
\]
as the conjugate subgroup.

\newtheorem*{theorem3}{Theorem}

\begin{theorem3}
    Let \(H \leq G\), then \(g^{-1}Hg \leq G\).
\end{theorem3}
\begin{proof}
    Suppose \(a,b \in g^{-1}Hg\).
    We want to show \(ab^{-1} \in g^{-1}Hg\).\\
    Note that \(a = g^{-1}h_1g\) and \(b = g^{-1}h_2g\)
    for some \(h_1, h_2 \in H\). \\
    This means that \(ab^{-1}=a{(g^{-1}h_2g)}^{-1} = a(g^{-1}h_2^{-1}g)
    =g^{-1}h_1gg^{-1}h_2^{-1}g = g^{-1} (h_1h_2) g \in g^{-1}Hg \).
\end{proof}

\end{document}
