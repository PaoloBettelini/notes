\documentclass{article}

\usepackage{amsmath}
\usepackage{amssymb}
\usepackage{parskip}
\usepackage{fullpage}
\usepackage{hyperref}
\usepackage{tikz}
\usepackage{makecell}

\usetikzlibrary{ % tikz packages
	cd % tikz-cd communitative diagrams
}

\hypersetup{
    colorlinks=true,
    linkcolor=black,
    urlcolor=blue,
    pdftitle={CategoryTheory},
    pdfpagemode=FullScreen,
}

\title{Category Theory}
\author{Paolo Bettelini}
\date{}

\begin{document}

\maketitle
\tableofcontents
\pagebreak

% after abstraction we notice that everything is the same. Every theory is the same thing.

\section{Category}

A category consists of \textit{objects} and \textit{morphism} or \textit{arrows}.
\\
An arrow has a beginning and an ending, and it goes from one object to another.
\\
Objects serve the purpose of marking the beginning and ending of a morphism.

\[
    \begin{tikzcd}
        a \arrow[r, bend left] \arrow[r, bend left=49, shift left] \arrow[loop, distance=2em, in=215, out=145] & b \arrow[loop, distance=2em, in=35, out=325] \arrow[l, bend left]
    \end{tikzcd}
    \\
    \makecell[l] {
        \text{An example of}
        \\
        \text{objects and morphisms}
    }
\]

\subsection{Composition}

Composition is a property that says that if there is an arrow from
\(a\) to \(b\), and an arrow from \(b\) to \(c\), there must exist an arrow
from \(a\) to \(c\).

\begin{center}
    \begin{tikzcd}
        a \arrow[r, "f"]
        \arrow[rr, "f \circ g", bend left=49] & b
        \arrow[r, "g"] & c
    \end{tikzcd}
\end{center}

\subsection{Identity}

For every object there is an identity arrow.

\begin{center}
    \begin{tikzcd}
        a \arrow["\text{id}_a"', loop, distance=2em, in=35, out=325]
    \end{tikzcd}
\end{center}

The composition of an arrow with an identity is the arrow itself

\begin{center}
    \begin{tikzcd}
        a \arrow[r, "f"] &
        b \arrow["\text{id}_b"', loop, distance=2em, in=35, out=325]
    \end{tikzcd}
\end{center}

\[
    f \circ \text{id}_b = f
\]

and also vice versa

\[
    \text{id}_b \circ f = f
\]

\subsection{Associativity}

Compositions have the associative property

\begin{center}
    \begin{tikzcd}
        a \arrow[r, "f"]
        \arrow[rr, "g \circ f", bend left, shift left=2]
        \arrow[rrr, "h \circ (g \circ f)", bend left=49, shift left=2]
        \arrow[rrr, "(h \circ g) \circ f", bend right=49, shift right=2] & b
        \arrow[r, "g"] \arrow[rr, "h \circ g", bend right, shift right=2] & c
        \arrow[r, "h"] & d
    \end{tikzcd}
\end{center}

\[
    h \circ (g \circ f) = (h \circ g) \circ f
\]

\pagebreak

\section{Homomorphisms}

An Homomorphism is a map between two structures of the same type.

\subsection{Epimorphisms}

An epimorphism is a \textbf{surjective} morphism.

We can define surjectivity only in terms of morphisms.
\\
Consider a morphism \(f: a \rightarrow b\) which maps elements of \(a\) onto \(b\).
Let's also define the morphisms \(g_1\) and \(g_2\) which map elements from \(b\) to \(c\).
The domain of \(g_1\) and \(g_2\) is the codomain of \(f\). These two functions act
as \(f\) for object in the image of \(f\), but may map objects differently
for objects in the codomain of \(f\) but outside the image of \(f\).
If the morphism is surjective, hence if the codomain and the image of \(f\) are the same,
then \(g_1\) and \(g_2\) will always act as \(f\).

\begin{center}
    \begin{tikzcd}
        a \arrow[r, "f"] &
        b \arrow[r, "g_1", shift left]
        \arrow[r, "g_2"', shift right] & c
    \end{tikzcd}
\end{center}

formally,

\[
    \forall c\, \forall g_1, g_2 : b \rightarrow c, g_1 \circ f = g_2 \circ f \Rightarrow g_1 = g_2
\]

An epimorphism is labelled with \(\twoheadrightarrow\).

\subsection{Monomorphisms}

An epimorphism is an \textbf{injective} morphism.

\begin{tikzcd}
    c \arrow[r, "g_1", shift left]
    \arrow[r, "g_2"', shift right] &
    a \arrow[r, "f"] & b
\end{tikzcd}

A morphism \(f: a \rightarrow b\) is a monomorphism if

\[
    \forall c\, \forall g_1, g_2: c \rightarrow a, 
    f \circ g_1 = g \circ g_2 \Rightarrow g_1 = g_2
\]

A monomorphism is labelled with \(\hookrightarrow\).

\subsection{Isomorphisms}

An isomorphism is a \textbf{bijective} morphism (mono and epic, but not every mono and epic
is an isomorphism).

A morphism \(f: a \rightarrow b\) is invertible if there
is a function \(g\) that goes from \(b\) to \(a\)

\[
    b:\quad b \rightarrow a
\]

such that

\begin{align*}
    g \circ f &= \text{id}_b
    \\
    f \circ g &= \text{id}_a
\end{align*}

\begin{center}
    \begin{tikzcd}
        a \arrow[r, "f", bend left] & b \arrow[l, "g", bend left]
    \end{tikzcd}
\end{center}

An isomorphism is labelled with \(\xrightarrow{\sim}\).

\pagebreak

\subsection{Homomorphism sets} % should this go here?

A hom-set is a set of all morphisms between a pair of objects.
It is denoted as
\begin{align*}
    &C(a, b) \\
    &\text{Hom}_C(a, b) \\
    &\text{Hom}(a, b)
\end{align*}

\section{Types of elements}

\subsection{Void}

The void element is equivalent to the logical \textbf{false}.
It is impossible to construct.

\subsection{Singleton}

A singleton is a single empty tuple element and
is equivalent to the logical \textbf{true}. It can be constructed from nothing.

\section{Types of categories}

\subsection{Thin categories}

A thin category is a category in which each pair of objects
has either \(0\) or \(1\) morphism.
Every hom-set has either \(1\) or \(0\) elements.

\subsection{Order categories}

An order category is a thin category where morphisms represent relationships.

For example, here we have an equality relationship

\begin{center}
    \begin{tikzcd}
        a \arrow[r, "\leq"] & b
    \end{tikzcd}
\end{center}

The relationship must be reflexive since there must be an identity morphsim.

\begin{center}
    \begin{tikzcd}
        a \arrow["\leq"', loop, distance=2em, in=215, out=145]
        \end{tikzcd}
\end{center}

\end{document}

% https://youtu.be/aZjhqkD6k6w?list=PLbgaMIhjbmEnaH_LTkxLI7FMa2HsnawM_&t=1697
% https://tikzcd.yichuanshen.de/