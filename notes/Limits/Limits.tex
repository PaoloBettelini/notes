\documentclass{article}
\usepackage[utf8]{inputenc}
\usepackage{amsmath}
\usepackage{amssymb}
\usepackage{parskip}
\usepackage{fullpage}
\usepackage{hyperref}

\hypersetup{
    colorlinks=true,
    linkcolor=black,
    urlcolor=blue,
    pdftitle={Limits},
    pdfpagemode=FullScreen,
}

\title{Limits}
\author{Paolo Bettelini}
\date{}

\begin{document}

\maketitle
\tableofcontents
\pagebreak

\section{Definition}

A limit is usually used to describe the behavior of a function as its argument approaches a given value.

The limit towards a certain value \(c\) within a function can be be approached both from the right and from the left. \\
The limit in a general sense exists if the value approached from both sides is the same and well-defined.

We define the limit of \(x\) approaching \(c\) from the left within the function \(f(x)\) as
\[
    \lim_{x\to c^{-}}f(x)
\]
We define the limit of \(x\) approaching \(c\) from the right within function \(f(x)\) as
\[
    \lim_{x\to c^{+}}f(x)
\]
We define the limit of \(x\) approaching \(c\) within function \(f(x)\) as
\[
    \lim_{x\to c}f(x)
\]

Formally, given a function \(f:D\to \mathbb{R}\) the limit \(L=\lim_{x\to c}f(x)\) exists if given an arbitrary small \(\epsilon >0\) there is another number \(\delta >0\) such that
\[
    |f(x)-L|<\epsilon,\quad
    \forall x\in D \text{ where } 0<|x-c|<\delta
\]

\section{Properties}

If the limit exists
\[
    \lim_{x\to c}f(g(x))=f(\lim_{x\to c}g(x))
\]

\pagebreak

\end{document}
