\documentclass{article}

\usepackage{amsmath}
\usepackage{amssymb}
\usepackage{parskip}
\usepackage{fullpage}
\usepackage{hyperref}
\usepackage{bettelini}

\hypersetup{
    colorlinks=true,
    linkcolor=black,
    urlcolor=blue,
    pdftitle={Integration},
    pdfpagemode=FullScreen,
}

\title{Integration}
\author{Paolo Bettelini}
\date{}

\begin{document}

\maketitle
\tableofcontents
\pagebreak

\section{Indefinite Integrals}

\subsection{Definition}

Given a function \(f(x)\), an \textbf{anti-derivative} or \textbf{primitive}
is any function \(F(x)\) such that
\[
    \frac{dF}{dx} = f(x)
\]
The operator to find a primitive function is called the \textbf{indefinite integral}
\[
    \integral[f(x)][x]=F(x)+C,
    \quad C\in\mathbb{R}
\]
The function to integrate (integrand) is delimited by the integral symbol \(\int\)
and a differential of the variable of integration \(dx\).
\\
A function has infinitely many primitives, hence the \(+ C\) term. This essentially
means that the derivative of a function is the same when the function is shifted
up or down, the rate of change is the same. By reversing the process we don't know
the up or down shift of the original function.
\[
    f(x)=\integral[\frac{df}{dx}][x] + C
\]
for some specific \(C\).

\subsection{Properties}

If \(k\) is a constant
\[
    \integral[kf(x)][x] = k \integral[f(x)][x]
\]

\[
    \integral[f(x) \pm g(x)][x] = \integral[f(x)][x] \pm \integral[g(x)][x]
\]

\subsection{Substitution Rule}

Given an integral in the form
\[
    \integral[f(g(x))g'(x)][x]
\]
Let
\[
    u = g(x)
\]
The differential of u is then
\[
    du=g'(x)dx
\]
meaning that we can rewrite the integral as
\[
    \integral[f(u)][u] = F(u) + C = F(g(x)) + C
\]

\pagebreak

\section{Integration By Parts}

Starting from the product rule
\[
    \frac{d}{dx}\big(f(x)g(x)\big)=f'(x)g(x)+f(x)g'(x)
\]
if we integrate both parts we get
\begin{align*}
    f(x)g(x)+C&=\int f'(x)g(x)\,dx+\int f(x)g'(x)\,dx \\
    \int f(x)g'(x)\,dx &= f(x)g(x)+C - \int f'(x)g(x)\,dx 
\end{align*}
Since the indefinite integral of \(f'(x)g(x)\) is equal to some function plus an arbitrary constant, we can ignore the \(+C\) term.
\[
    \int f(x)g'(x)\,dx = f(x)g(x) - \int f'(x)g(x)\,dx
\]

\pagebreak

%\section{Area Problem}

\end{document}
