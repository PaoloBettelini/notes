\documentclass{article}
\usepackage{amsmath}
\usepackage{parskip}
\usepackage{fullpage}

\title{Diffie–Hellman Key Exchange}
\author{Paolo Bettelini}
\date{}

\begin{document}

\maketitle
\tableofcontents
\pagebreak

\section{Diffie Hellman}

Diffie–Hellman key exchange is a method of securely exchanging cryptographic keys over a public channel.

Scenario: a \textit{client} and a \textit{server} want to establish a shared secret.
\begin{itemize}
	\item The \textit{client} generates a random private key \(k_c\)
	\item The \textit{server} generates a random private key \(k_s\)
	\item The two parts publicly establish a common \(G\) (generator)
\end{itemize}

We define a function
\[
	y=f(G,k)
\]
such that given \(y\) and \(G\) it is very hard to get \(k\).\\
The function must also satisfy the following identity
\[
	f(f(G, k_1), k_2)=f(f(G, k_2), k_1)
\]
For instance the function \(G^k\) would satisfy this identity since \({\left(G^{k_1}\right)}^{k_2}={\left(G^{k_2}\right)}^{k_1}\), but not the first property.

Given the function \(f(G,k)\)
\begin{itemize}
	\item The \textit{client} computes \(y_c=f(G,k_c)\)
	\item The \textit{server} computes \(y_s=f(G,k_s)\)
	\item The two parts publicly exchange \(y_c\) and \(y_s\)
	\item The \textit{client} computes \(y=f(y_s,k_c)\)
	\item The \textit{server} computes \(y=f(y_c,k_s)\)
\end{itemize}
Now the \textit{client} and \textit{server} share the same value of \(y\) since \(f(y_s,k_c)=f(y_c,k_s)\).\\
The value of \(y\) is unknown to anyone who has traced the communication between the \textit{client} and the \text{server}.

\end{document}