\documentclass{article}
\usepackage[utf8]{inputenc}
\usepackage{amsmath}
\usepackage{amssymb}
\usepackage{parskip}
\usepackage{fullpage}

\title{Fundamentals of Quantum Computing}
\author{Paolo Bettelini}
\date{}

\newcommand{\quotes}[1]{``#1''}

\begin{document}

\maketitle
\tableofcontents
\pagebreak

\section{The Probabilistic Nature of Qubits}

A qubit is comprable to a bit in the \quotes{classical} world, but it exists on a sub-atomic level. \\
When a qubit is measured, its state will either be a \quotes{1} or a \quotes{0}. \\
The crucial aspect is that before the measurement, a qubit is in a \textit{superposition} of both states.

For example, a given qubit \(|\Psi\rangle\) can be represented as
\[
    |\Psi\rangle=\alpha |0\rangle+\beta |1\rangle
\]
which means a linear combination of the two states \(|0\rangle\) and \(|1\rangle\).

The coefficients \(\alpha\) and \(\beta\) represent the probability of the qubit collapsing into one of the two states when measured.
The probability of the qubit collapsing into \(|0\rangle\) is \(|\alpha|^2\),
while the probability of collapsing into \(|1\rangle\) is \(|\beta|^2\). \\
Since there is \(100\%\) chance of the qubits collapsing into one of the two states, \(\alpha\) and \(\beta\) must satisfy the following requirement:
\[
    |\alpha|^2+|\beta|^2=1
\]

A uniform superposition of the two states looks like this:
\[
    |\Psi\rangle=\frac{|0\rangle+|1\rangle}{\sqrt{2}}
\]
which means that we have \(50\%\) probability of the state collapsing into a \(|0\rangle\) or \(|1\rangle\)
since \({\left(\frac{1}{\sqrt{2}}\right)}^2=\frac{1}{2}\).

When the qubit is measured, the superposition is destroyed, leaving it in a \quotes{classical} binary state.

You mght have noticed that I used the absolute value in \(|\alpha|^2\) or \(|\beta|^2\), which doesn't usually make sense since squaring the value is already going to give us a positive result.
\\
This is because the coefficients \(\alpha\) and \(\beta\) can also be complex numbers. The absolute value of a complex number is defined as its distance from the origin.

The states \(|1\rangle\) and \(|0\rangle\) can also be written as \(|+\rangle\) and \(|-\rangle\).

\pagebreak

\section{Spin of a particle}

\subsection{Definition}

Spin is intrinsic angular momentum associated with elementary particles.
The particle isn't actually rotating, but spin is a property just like momentum, position, charge and mass.

We can represent a spin \(S\) as we would in the classical world, with a vector representing its components

\[
    S=
    \begin{pmatrix}
        S_x \\
        S_y \\
        S_z
    \end{pmatrix}
\]

This vector has a fixed length depending on the type of the particle. \\
The spin of an electron, which is known as spin 1/2, has a magnitude of \(\frac{\sqrt{3}\hbar}{2}\).

Since the magnitude of the spin is always the same, a common way to express its direction is with the polar and azimuthal angles, \(\theta\) and \(\phi\).

\subsection{Measurement}

A spin \(S\) can be measured. For example, we could measure the projection of the spin on its \(z\)-axis.
This is written as measuring \(S_z\). We could measure \(S_x\), \(S_y\) or we could measure its projection from an arbitrary direction, \(S_{\vec{n}}\).

We are given a set of electrons with randomly-oriented spins. For each of them, we measure \(S_z\).
We would expect a bunch of values between \(\frac{\sqrt{3}\hbar}{2}\) and \(-\frac{\sqrt{3}\hbar}{2}\),
instead, we find that every measurement has an output of either \(\frac{\hbar}{2}\) or \(-\frac{\hbar}{2}\), equally distributed. \\
There are a couple of things things to notice:

\begin{enumerate}
    \item When we measure the spin, its state collapses in either one of the two \textit{eigenvalues}, \(|+\rangle\) or \(|-\rangle\).
    \item The quantity measured is not the entire length of the vector even if it should be either completly up or down.
        This is due to the Heisenberg uncertainity principle, if the spin was any closer to the vertical \(\pm z\)-axis,
        we would have too much simultaneous knowledge about \(S_x\) and \(S_y\).
\end{enumerate}

The state of \(S_z\) is now ``locked', if we measure \(S_z\) of the same particle again, the result will stay the same.

We then measure \(S_x\) of each particle. Since we've already measured \(S_z\) we would except \(S_x\) or \(S_y\)
to satisfy \(\sqrt{{|S_x|}^2 + {|S_y|}^2 + \frac{\hbar^2}{4}} = \frac{\sqrt{3}\hbar}{2}\), instead, we get random eigenvalues as before. \\
Now, if we come back to measuring \(S_z\), the previously locked states are lost, we get new random eigenvalues. \\
The act of measuring \(S_x\) has destroyed the information contained in \(S_z\).

We notices that the eigenvalues \(|+\rangle\) and \(|-\rangle\) have the same probability of being measured. \\
We can make a rotation of \(\theta\) of the direction of measurement from the \(z\)-axis, the state \(\Psi\) is now

\[
    |\Psi\rangle = \cos\left(\frac{\theta}{2}\right) |+\rangle + \sin\left(\frac{\theta}{2}\right)|-\rangle
\]

Before, the rotation was \(\frac{\pi}{2}\) from the \(z\)-axis, so the state was \(|\Psi\rangle=\frac{|+\rangle + |-\rangle}{\sqrt{2}}\),
which means that it is equally likely that the spin will collapse into a \(|+\rangle\) or a \(|-\rangle\).

\pagebreak

\section{Types of Algorithms}

\subsection{Deterministic}

Example: the quantum circuit is run and we measure the output.

\subsection{Probabilistic}

Example: The more we run the circuit, the higher the chance of getting the right or a more precise result.

% pag 154

\end{document}