\documentclass[a4paper]{article}

\usepackage{amsmath}
\usepackage{amssymb}
\usepackage{parskip}
\usepackage{fullpage}
\usepackage{hyperref}
\usepackage{bettelini}
\usepackage{stellar}

\hypersetup{
    colorlinks=true,
    linkcolor=black,
    urlcolor=blue,
    pdftitle={SetTheory},
    pdfpagemode=FullScreen,
}

\title{Set Theory}
\author{Paolo Bettelini}
\date{}

\begin{document}

\maketitle
\tableofcontents
\pagebreak

% Russel's paradox

\section{Definitions}

\subsection{Cardinality}

\sdefinition{Cardinality}{
    The \textit{cardinality} of a set \(A\), denoted \(|A|\),
    is the amount of elements it contians.
}

\subsection{Subset}

\sdefinition{Subset}{
    If \(A\) and \(B\) are sets, then \(A\) is a \textit{subset} of \(B\)
    (\(A\subseteq B\)), if all the elements of \(A\) are also in \(B\).
}

For every set \(A\), \(A \subseteq A\).

\subsection{Proper Subset}

\sdefinition{Proper Subset}{
    Given two sets \(A\) and \(B\), if \(A \subseteq B\) but \(A \neq B\),
    then \(A\) is a \textit{proper} (or \textit{strict}) subset of \(B\)
    \[
        A \subset B
    \]
}

\subsection{Empty Set}

\sdefinition{Proper Subset}{
    The empty set \(\emptyset\) is a subset of all other sets.
    \[
        |\emptyset|=0
    \]
}

For every set \(A\)
\[
    \emptyset \subseteq A
\]

\subsection{Power Set}

\sdefinition{Proper Subset}{
    If \(B\) is a set, then the \textit{power set} \(\mathcal{P}(B)\)
    is defined as the set of all subsets of \(B\)
    \[
        \mathcal{P}(B)=\{A \suchthat A\subseteq B\}
    \]
}
Note that \(B\in\mathcal{P}(B)\).

\stheorem{Cardinality of the power set}{
    The cardinality of \(\mathcal{P}(A)\) is given by
    \[
        |\mathcal{P}(A)| = 2^{|A|}
    \]
}

\subsection{Union}

\sdefinition{Union}{
    If \(A\) and \(B\) are sets, then their \textit{union} is
    \[
        A \cup B = \{x \suchthat x \in A \lor x \in B\}
    \]
}

\subsection{Intersection}

\sdefinition{Intersection}{
    If \(A\) and \(B\) are sets, then their \textit{intersection} is
    \[
        A \cap B = \{x \suchthat x \in A \land x \in B\}
    \]
}

\subsection{Difference}

\sdefinition{Intersection}{
    If \(A\) and \(B\) are sets, then their \textit{difference} is
    \[
        A \backslash B = \{x \suchthat x \in A \land x \notin B\}
    \]
}
Note that
\[
    A \backslash B = B \backslash A
    \iff A = B
\]

\subsection{Subset in terms of relationships}

\scorollary{Subset in terms of relationships}{
    \[
        A \subseteq B
        \iff
        A \cup B = B
        \iff
        A \cap B = A
        \iff
        A \backslash B = \emptyset
    \]
}

\subsection{Disjoint Sets}

\sdefinition{Disjoint Sets}{
    If \(A\) and \(B\) are sets and \(A \cap B = \emptyset \), then \(A\)
    and \(B\) are \textit{disjoint sets}.
}

\subsection{Cartesian Product}

\sdefinition{Cartesian Product}{
    If \(A\) and \(B\) are sets, then their \textit{cartesian product} is
    \[
        A\times B = \{(x,y) \suchthat x \in A \land y \in B\}
    \]
    which is the set of all possible \textit{ordered pairs}.
}

More generally, given \(n\) sets \(A_1, A_2, \ldots, A_2\),
their cartesian product \(A_1 \times A_2 \times \cdots \times A_n\)
is the set of ordered \(n\)-tuples \((a_1, a_2, \ldots, a_n)\) with \(a_i\in A_i\).

\subsection{Cartesian Power}

\sdefinition{Cartesian Power}{
    Given a set \(A\), \(A^n=\underbrace{A\times A\times \cdots \times A}_n\).
}

The \(n\)-dimensional plane of real numbers is a cartesian power \({\mathbb{R}}^n\).

\subsection{Disjoint union}

\sdefinition{Disjoint union}{
    Given sets \(A_{i\in I}\), their disjoint union is
    \[
        \bigsqcup_{i\in I}A_i= \bigcup_{i\in I}\{(x, i) \suchthat x \in A_i\}
    \]
    which consists of prdered pairs where the second element
    is the index of the set.
}

\subsection{Complement}

\sdefinition{Complement}{
    If \(A\) is a set, its \textit{complement} is
    \[
        \bar{A} = \{x \suchthat x \notin A\}
    \]
}

\subsection{Binary Relation}

\sdefinition{Binary Relation}{
    If \(A\) and \(B\) are sets, a function \(f:A\to B\)
    defines a \textit{binary relation} \(R\)
    \[
        R = \{(a,b) \suchthat f(a)=b\}
    \]
}

Note that \(R\subseteq A\times B\)


\subsection{Homogeneous Relation}

\sdefinition{Homogeneous Relation}{
    A \textit{homogeneous relation} on a set \(S\) is a binary relation
    from a \(A\) to \(A\).
}

\subsection{Reflexive relation}

\sdefinition{Reflexive relation}{
    A homogeneous relation \(R\) on a set \(A\) is \textit{reflexive}
    iff
    \[
        \forall a\in A, (a,a) \in R
    \]
}

\subsection{Symmetric relation}

\sdefinition{Symmetric relation}{
    A homogeneous relation \(R\) on a set \(A\) is \textit{symmetric}
    iff
    \[
        \forall (a,b) \in R, (b,a) \in R
    \]
}

\subsection{Transitive relation}

\sdefinition{Transitive relation}{
    A homogeneous relation \(R\) on a set \(A\) is \textit{transitive}
    \[
        \forall a,b,c \in A, (a,b) \in R \land (b,c) \in R \implies (a,c) \in R 
    \]
}

\subsection{Equivalence relation}

\sdefinition{Equivalence relation}{
    An \textit{equivalence relation} is a homogeneous relation \(\sim\) on a set \(A\)
    that is
    \begin{enumerate}
        \item \textit{Reflexive}: \(\forall a \in A, a \sim a\)
        \item \textit{Symmetric}: \(\forall a,b \in A, a \sim b \iff b \sim a\)
        \item \textit{Transitive}: \(\forall a,b,c \in A, a \sim b \land b \sim c \implies a \sim c\)
    \end{enumerate}
}

\subsection{Equivalence class}

\sdefinition{Equivalence relation}{
    Let \(\sim\) be an equivalence relation on a set \(A\).
    Given an element \(a\in A\), the equivalence class of \(a\), is defined as
    \[
        {[a]}_{\sim} = \{x \in A \suchthat a \sim x\}
    \]
}

\stheorem{Shared element in equivalence class}{
    Let \(\sim\) be an equivalence relation on a set \(A\)
    and \(a,b \in A\).
    Then,
    \[
        b \in {[a]}_{\sim} \iff {[a]}_{\sim} = {[b]}_{\sim}
    \]
}

\sproof{Shared element in equivalence class}{
    By the symmetric property we have \(a \in {[a]}_{\sim}\).
    Let \(b \in {[a]}_{\sim}\), meaning \(a \sim b\). \(\forall c \in {[b]}_{\sim}\),
    meaning \(b \sim c\), we have \(a \sim c\) by the transitive property.
    Thus, \(c \in {[a]}_{\sim}\) and \({[b]}_{\sim} \subseteq {[a]}_{\sim}\).
    By the symmetric property we also have \(b \sim a\),
    \(\forall d \in {[a]}_{\sim}\), meaning \(a \sim d\), we have
    \(b \sim d\) by the transitive property. Thus, \(d \in {[b]}_{\sim}\)
    and \({[a]}_{\sim} \subseteq {[b]}_{\sim}\). Hence,
    \[
        b \in {[a]}_{\sim} \iff {[a]}_{\sim} = {[b]}_{\sim}
    \]
}

This means that every element of an equivalence class has the same equivalence class.
Thus, if two classes share an element they are the same.

\subsection{Partition of a set}

\sdefinition{Partition of a set}{
    Given a set \(A\), a \textit{partition of a set} \(P={\{C_i\}}_{i\in I}\) is a collection of
    non-empty subsets of \(A\) such that \(\bigcup_{i\in I} C_i = P\) and
    \(C_i \cap C_j = \emptyset, i \neq j\).
}

In other words the sets \(C_i\)
contain every element of \(A\) exactly once.

Given an equivalence relationship \(\sim\) of a set \(A\),
the set of its equivalence classes form a partition of \(A\).

\subsection{Preorder}

\sdefinition{Preorder order}{
    A \textit{preorder} is a homogeneous relation \(\leq\) on a set \(A\)
    with the following properties:
    \begin{enumerate}
        \item \textit{Reflexive}: \(\forall a \in A, a \leq a\)
        \item \textit{Transitive}: \(\forall a,b,c \in A, a \leq b \land b \leq c \implies a \leq c\)
    \end{enumerate}
}

\subsection{Partial order}

\sdefinition{Partial order}{
    A \textit{partial order} is a homogeneous relation \(\leq\) on a set \(A\)
    with the following properties:
    \begin{enumerate}
        \item \textit{Reflexive}: \(\forall a \in A, a \leq a\)
        \item \textit{Transitive}: \(\forall a,b,c \in A, a \leq b \land b \leq c \implies a \leq c\)
        \item \textit{Antisymmetric}: \(\forall a,b \in A, a \leq b \land b \leq a \implies a=b\)
    \end{enumerate}
}

\subsection{Total order}

\sdefinition{Total order}{
    A \textit{total order} is a homogeneous relation \(\leq\) on a set \(A\)
    with the following properties:
    
    \begin{enumerate}
        \item \textit{Reflexive}: \(\forall a \in A, a \leq a\)
        \item \textit{Transitive}: \(\forall a,b,c \in A, a \leq b \land b \leq c \implies a \leq c\)
        \item \textit{Antisymmetric}: \(\forall a,b \in A, a \leq b \land b \leq a \implies a=b\)
        \item \textit{Strongly connected} (or \textit{total}): \(\forall a,b\in A, a \leq b \lor b\leq a\)
    \end{enumerate}
}

A total order is a partial order where any two elements are comparable.

\subsection{Greatest element}

\sdefinition{Greatest element}{
    Given a partial order on a set \(A\), an element \(g\) is a \textit{greatest element}
    if \(\forall a\in A, a \leq g\).
}

\subsection{Least element}

\sdefinition{Least element}{
    Given a partial order on a set \(A\), an element \(g\) is a \textit{least element}
    if \(\forall a\in A, g \leq a\).
}

\subsection{Maximal element}

\sdefinition{Maximal element}{
    Given a partial order on a set \(A\), an element \(g\in A\) that is
    a greatest element is a \textit{maximal element}.
}

\subsection{Minimal element}

\sdefinition{Minimal element}{
    Given a partial order on a set \(A\), an element \(g\in A\) that is
    a least element is a \textit{minimal element}.
}

\pagebreak

\end{document}
