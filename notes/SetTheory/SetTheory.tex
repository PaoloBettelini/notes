\documentclass{article}

\usepackage{amsmath}
\usepackage{amssymb}
\usepackage{parskip}
\usepackage{fullpage}
\usepackage{hyperref}
\usepackage{bettelini}

\hypersetup{
    colorlinks=true,
    linkcolor=black,
    urlcolor=blue,
    pdftitle={SetTheory},
    pdfpagemode=FullScreen,
}

\title{Set Theory}
\author{Paolo Bettelini}
\date{}

\begin{document}

\maketitle
\tableofcontents
\pagebreak

\section{Definitions}

\subsection{Subset}

If \(A\) and \(B\) are sets, then \(A\) is a \textit{subset} of \(B\)
(\(A\subseteq B\)), iff all the elements of \(A\) are also in \(B\).

\subsection{Empty Set}

The empty set \(\emptyset\) is a subset of all other sets.

\subsection{Power Set}

If \(B\) is a set, then the \textit{power set} \(\mathcal{P}(B)\)
is defined as the set of all subsets of \(B\)
\[
    \mathcal{P}(B)=\{A \suchthat A\subseteq B\}
\]
Note that \(\emptyset\subseteq\mathcal{P}(B)\) and \(B\in\mathcal{P}(B)\)

\subsection{Union}

If \(A\) and \(B\) are sets, then their \textit{union} is
\[
    A \cup B = \{x \suchthat c \in A \lor X \in B\}
\]

\subsection{Intersection}

If \(A\) and \(B\) are sets, then their \textit{intersection} is
\[
    A \cap B = \{x \suchthat x \in A \land x \in B\}
\]

\subsection{Difference}

If \(A\) and \(B\) are sets, then their \textit{difference} is
\[
    A \backslash B = \{x \suchthat x \in A \land x \notin B \lor x \in B \land \notin A\}
\]

\subsection{Cartesian Product}

If \(A\) and \(B\) are sets, then their \textit{cartesian product} is
\[
    A\times B = \{(x,y) \suchthat x \in A \land y \in B\}
\]

\subsection{Complement}

If \(A\) is a set, its \textit{complement} is
\[
    \bar{A} = \{x \suchthat x \notin A\}
\]

\subsection{Binary Relation}

If \(A\) and \(B\) are sets, a function \(f:A\to B\)
is a \textit{binary relation} \(R\)
\[
    R = \{(a,b) \suchthat f(a)=b\}
\]

Note that \(R\subseteq A\times B\)

\subsection{Injection}

A function \(f:A\to B\) is \textit{injective} iff
\[
    \forall a,b \in A \suchthat a\neq b, f(a)\neq f(b)
\]

\subsection{Surjectivity}

A function \(f:A\to B\) is \textit{surjectiv} iff
\[
    \forall b \in B \exists a \suchthat f(a)=b
\]

\subsection{Bijectivity}

A function \(f:A\to B\) is \textit{bijective} iff
it is both surjective and injective.

\subsection{Reflexive relation}

A binary relation \(R\) for \(f:A\to B\) is \textit{reflexive}
iff
\[
    \forall a\in A, (a,a) \in R
\]

\subsection{Symmetric relation}

A binary relation \(R\) for \(f:A\to B\) is \textit{symmetric}
iff
\[
    \forall (a,b)\in R, (b,a) \in R
\]

\subsection{Transitive relation}

A binary relation \(R\) for \(f:A\to B\) is \textit{transitive}
\[
    \forall a,b,c \in A, (b,c) \in R \land (b,c) \in R \implies (a,c) \in R 
\]

\pagebreak

\end{document}
