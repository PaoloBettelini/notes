\documentclass{article}
\usepackage{amsmath}
\usepackage{parskip}
\usepackage{fullpage}

\title{Deep Learning}
\author{Paolo Bettelini}
\date{}

\begin{document}

\maketitle
\tableofcontents
\pagebreak

\section{Simplified brain neurons}

\section{Linear neurons}

% lecture 1.3

A linear neuron is very simple and computationally limited in what it can do.

\[
    y=b+\sum_ix_iw_i
\]

The output \(y\) is given by the bias \(b\) plus the sum of all the input connections \(x_i\) multiplied by their weight \(w_i\).

\section{Binary threshold neurons}

Binary threshold neurons output a \(1\) or a \(0\) depending on its weighted value.

Given a threshold \(\theta=-b\)
\begin{align*}    
    z&=b+\sum_ix_iw_i \\
    y&=\begin{cases}
        1 \text{ if } z\ge 0 \\
        0 \text{ otherwise}
    \end{cases}
\end{align*}

\section{Rectified Linear Neurons or Linear threshold neurons}

They compute a linear weighted sum of their inputs. \\
The output is a non-linear function of the total input.

Given a threshold \(\theta=-b\)
\begin{align*}    
    z&=b+\sum_ix_iw_i \\
    y&=\begin{cases}
        z \text{ if } z > 0 \\
        0 \text{ otherwise}
    \end{cases}
\end{align*}

% *picture*

\section{Sigmoid neurons}

They give a real-valued output that is a smooth and bounded function of their total input.

The logistic function is often used.

Given a threshold \(\theta=-b\)
\begin{align*}    
    z&=b+\sum_ix_iw_i \\
    y&=\frac{1}{1+e^{-z}}
\end{align*}

% *picture*

This function has smooth derivatives that change continuously. \\
This characteristic makes the learning process easier.

\pagebreak

% lecture 1.5



\end{document}