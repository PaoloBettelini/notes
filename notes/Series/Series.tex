\documentclass[a4paper]{article}

\usepackage{amsmath}
\usepackage{amssymb}
\usepackage{amsthm}
\usepackage{parskip}
\usepackage{fullpage}
\usepackage{hyperref}

\hypersetup{
    colorlinks=true,
    linkcolor=black,
    urlcolor=blue,
    pdftitle={Series},
    pdfpagemode=FullScreen,
}

\title{Series}
\author{Paolo Bettelini}
\date{}

\begin{document}

\maketitle
\tableofcontents
\pagebreak

\section{Divergence and convergence}

An infinite series converges if the limit
of its partial sum sequence also converges,
otherwise it diverges.

\section{Properties}

\[
    \left(
        \sum_{n=0}^\infty a_n
    \right)
    \left(
        \sum_{n=0}^\infty b_n
    \right)
    =
    \sum_{n=0}^\infty \sum_{k=0}^n a_k b_{n-k}
\]

\section{Covergence theorem}

\newtheorem*{theorem1}{Theorem}

\begin{theorem1}
    If \(\sum a_n\) converges then \(\lim_{n\to\infty}a_n=0\)
\end{theorem1}
\begin{proof}
    Consider the partial sum
    \[
        s_n = \sum_{k=1}^{n}a_k
    \]
    The sequence \(a_n\) can now be expressed as
    \[
        a_n = s_n - s_{n-1}
    \]
    Since \(\sum a_n\) converges, \(\lim_{n\to\infty}s_n=L\) for \(L\) finite. \\
    The limit \(\lim_{n\to\infty}s_{n-1}=L\) because \(n-1 \to \infty \text{ as } n \to \infty\).
    This implies the following
    \[
        \lim_{n \to \infty} a_n
        = \lim_{n \to \infty} s_n - s_{n-1} = L - L = 0
    \]
\end{proof}

\section{Divergence test}

If \(\lim_{n \to \infty} a_n \neq 0\) then \(\sum a_n\) diverges.

\section{Absolute and conditional convergence}

A series \(\sum a_n\) is said to converge absolutely if
\(\sum |a_n|\) converges.
This is a stronger type of convergence. Every absolutely convergent series is also convergent.

A series that is convergent but not absolutely convergent is called conditionally convergent.

\section{Riemann rearrangement theorem}

If a series is conditionally convergent, then its terms can be rearranged such that
the series converges to any \(r\in \mathbb{R}\) or such that it diverges (to infinity or no finite value).
If the series is absolutely convergent then any rearrangement of its terms will converge to the same value.

\end{document}
