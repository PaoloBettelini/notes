\documentclass{article}

\usepackage{amsmath}
\usepackage{amssymb}
\usepackage{parskip}
\usepackage{fullpage}
\usepackage{hyperref}
\usepackage{tikz}

\usetikzlibrary{ % tikz packages
	cd % tikz-cd communitative diagrams
}

\hypersetup{
    colorlinks=true,
    linkcolor=black,
    urlcolor=blue,
    pdftitle={TheoryOfComputation},
    pdfpagemode=FullScreen,
}

\title{Theory of Computation}
\author{Paolo Bettelini}
\date{}

% Empty string symbol.
\newcommand{\emptyString}{\lambda}

\begin{document}

\maketitle
\tableofcontents
\pagebreak

\section{Fields of Study}

\subsection{Complexity Theory}

Classify problems according to their degree
of "difficulty".

\subsection{Computability Theory}

Classify problems as being solvable or unsolvable.

\subsection{Automata Theory}

Compare different computation models.

\section{Alphabet}

An \textit{alphabet} is a finite set of \textit{symbols}.
For example: \(\{a,b,c,\cdots, z\}\)\\
The set \(\{0,1\}\) is the binary set. The set \(\{0,1\}^*\) is the set of
all binary strings (union of all \(n\)-permutations of \(\{0,1\}\) and an empty string).
In general, if \(\Sigma\) is an alphabet \(\Sigma^*\) is the set
of all strings over \(\Sigma\)
\[
    \Sigma^* = \emptyString \cup \bigcup_{n\in\mathbb{N}} \Sigma^n
\]
where \(\emptyString\) is the empty string.
Note that \(\emptyString \neq \varnothing \neq \{\emptyString\}\).
\\
The length of a string \(w\) is denoted as \(|w|\).

A set of strings is called a \textit{language}.

\section{Finite automaton}

A finite automaton is a machine which process a string
symbol by symbol from left to right. The automaton is in one of his \textit{states}
after processing a symbol. The machine might terminate in an
\textit{accept state} or not.

A finite automaton \(M=(Q, \Sigma, \delta, q, F)\)
\begin{itemize}
    \item \(Q\) is a finite set of \textit{states}
    \item \(\Sigma\) is an alphabet
    \item \(\delta : Q \times \Sigma \to Q\) is the \textit{transition function}
    \item \(q\) is an element of \(Q\) called the \textit{start state}
    \item \(F\) is a subset of \(Q\) which contains the \textit{accept states}
\end{itemize}
The transition function is the logical components, it determines
in which state the machine will be after processing a symbol at any state.

The following automaton processes a binary string.
The start state is \(q_1\) and the only accept state is \(q_3\).
The program moves to the next state only if the symbol is \(1\),
so it will reach \(q_3\) only if the input strings contains at least two \(1\)s.
\begin{center}
    \begin{tikzcd}
        {} \arrow[r] & q_1 \arrow[r, "1"] \arrow["0"', loop, distance=2em, in=125, out=55] & q_2 \arrow[r, "1"] \arrow["0"', loop, distance=2em, in=125, out=55] & q_3 \arrow["{0,1}"', loop, distance=2em, in=125, out=55]
    \end{tikzcd}
\end{center}

\pagebreak

The language of \(M\), denoted \(L(M)\) is the set of all accepted strings
by \(M\).

\end{document}

% https://cglab.ca/~michiel/TheoryOfComputation/TheoryOfComputation.pdf
% todo: proofs
% pag 34